% Options for packages loaded elsewhere
\PassOptionsToPackage{unicode}{hyperref}
\PassOptionsToPackage{hyphens}{url}
\PassOptionsToPackage{dvipsnames,svgnames,x11names}{xcolor}
%
\documentclass[
  single column]{article}

\usepackage{amsmath,amssymb}
\usepackage{iftex}
\ifPDFTeX
  \usepackage[T1]{fontenc}
  \usepackage[utf8]{inputenc}
  \usepackage{textcomp} % provide euro and other symbols
\else % if luatex or xetex
  \usepackage{unicode-math}
  \defaultfontfeatures{Scale=MatchLowercase}
  \defaultfontfeatures[\rmfamily]{Ligatures=TeX,Scale=1}
\fi
\usepackage[]{libertinus}
\ifPDFTeX\else  
    % xetex/luatex font selection
\fi
% Use upquote if available, for straight quotes in verbatim environments
\IfFileExists{upquote.sty}{\usepackage{upquote}}{}
\IfFileExists{microtype.sty}{% use microtype if available
  \usepackage[]{microtype}
  \UseMicrotypeSet[protrusion]{basicmath} % disable protrusion for tt fonts
}{}
\makeatletter
\@ifundefined{KOMAClassName}{% if non-KOMA class
  \IfFileExists{parskip.sty}{%
    \usepackage{parskip}
  }{% else
    \setlength{\parindent}{0pt}
    \setlength{\parskip}{6pt plus 2pt minus 1pt}}
}{% if KOMA class
  \KOMAoptions{parskip=half}}
\makeatother
\usepackage{xcolor}
\usepackage[top=30mm,left=25mm,heightrounded,headsep=22pt,headheight=11pt,footskip=33pt,ignorehead,ignorefoot]{geometry}
\setlength{\emergencystretch}{3em} % prevent overfull lines
\setcounter{secnumdepth}{-\maxdimen} % remove section numbering
% Make \paragraph and \subparagraph free-standing
\makeatletter
\ifx\paragraph\undefined\else
  \let\oldparagraph\paragraph
  \renewcommand{\paragraph}{
    \@ifstar
      \xxxParagraphStar
      \xxxParagraphNoStar
  }
  \newcommand{\xxxParagraphStar}[1]{\oldparagraph*{#1}\mbox{}}
  \newcommand{\xxxParagraphNoStar}[1]{\oldparagraph{#1}\mbox{}}
\fi
\ifx\subparagraph\undefined\else
  \let\oldsubparagraph\subparagraph
  \renewcommand{\subparagraph}{
    \@ifstar
      \xxxSubParagraphStar
      \xxxSubParagraphNoStar
  }
  \newcommand{\xxxSubParagraphStar}[1]{\oldsubparagraph*{#1}\mbox{}}
  \newcommand{\xxxSubParagraphNoStar}[1]{\oldsubparagraph{#1}\mbox{}}
\fi
\makeatother


\providecommand{\tightlist}{%
  \setlength{\itemsep}{0pt}\setlength{\parskip}{0pt}}\usepackage{longtable,booktabs,array}
\usepackage{calc} % for calculating minipage widths
% Correct order of tables after \paragraph or \subparagraph
\usepackage{etoolbox}
\makeatletter
\patchcmd\longtable{\par}{\if@noskipsec\mbox{}\fi\par}{}{}
\makeatother
% Allow footnotes in longtable head/foot
\IfFileExists{footnotehyper.sty}{\usepackage{footnotehyper}}{\usepackage{footnote}}
\makesavenoteenv{longtable}
\usepackage{graphicx}
\makeatletter
\newsavebox\pandoc@box
\newcommand*\pandocbounded[1]{% scales image to fit in text height/width
  \sbox\pandoc@box{#1}%
  \Gscale@div\@tempa{\textheight}{\dimexpr\ht\pandoc@box+\dp\pandoc@box\relax}%
  \Gscale@div\@tempb{\linewidth}{\wd\pandoc@box}%
  \ifdim\@tempb\p@<\@tempa\p@\let\@tempa\@tempb\fi% select the smaller of both
  \ifdim\@tempa\p@<\p@\scalebox{\@tempa}{\usebox\pandoc@box}%
  \else\usebox{\pandoc@box}%
  \fi%
}
% Set default figure placement to htbp
\def\fps@figure{htbp}
\makeatother

\usepackage{booktabs}
\usepackage{longtable}
\usepackage{array}
\usepackage{multirow}
\usepackage{wrapfig}
\usepackage{float}
\usepackage{colortbl}
\usepackage{pdflscape}
\usepackage{tabu}
\usepackage{threeparttable}
\usepackage{threeparttablex}
\usepackage[normalem]{ulem}
\usepackage{makecell}
\usepackage{xcolor}
\usepackage{tabularray}
\usepackage[normalem]{ulem}
\usepackage{graphicx}
\UseTblrLibrary{booktabs}
\UseTblrLibrary{rotating}
\UseTblrLibrary{siunitx}
\NewTableCommand{\tinytableDefineColor}[3]{\definecolor{#1}{#2}{#3}}
\newcommand{\tinytableTabularrayUnderline}[1]{\underline{#1}}
\newcommand{\tinytableTabularrayStrikeout}[1]{\sout{#1}}
\input{/Users/joseph/GIT/latex/latex-for-quarto.tex}
\let\oldtabular\tabular
\renewcommand{\tabular}{\small\oldtabular}
\setlength{\tabcolsep}{4pt}  % Adjust this value as needed
\makeatletter
\@ifpackageloaded{caption}{}{\usepackage{caption}}
\AtBeginDocument{%
\ifdefined\contentsname
  \renewcommand*\contentsname{Table of contents}
\else
  \newcommand\contentsname{Table of contents}
\fi
\ifdefined\listfigurename
  \renewcommand*\listfigurename{List of Figures}
\else
  \newcommand\listfigurename{List of Figures}
\fi
\ifdefined\listtablename
  \renewcommand*\listtablename{List of Tables}
\else
  \newcommand\listtablename{List of Tables}
\fi
\ifdefined\figurename
  \renewcommand*\figurename{Figure}
\else
  \newcommand\figurename{Figure}
\fi
\ifdefined\tablename
  \renewcommand*\tablename{Table}
\else
  \newcommand\tablename{Table}
\fi
}
\@ifpackageloaded{float}{}{\usepackage{float}}
\floatstyle{ruled}
\@ifundefined{c@chapter}{\newfloat{codelisting}{h}{lop}}{\newfloat{codelisting}{h}{lop}[chapter]}
\floatname{codelisting}{Listing}
\newcommand*\listoflistings{\listof{codelisting}{List of Listings}}
\makeatother
\makeatletter
\makeatother
\makeatletter
\@ifpackageloaded{caption}{}{\usepackage{caption}}
\@ifpackageloaded{subcaption}{}{\usepackage{subcaption}}
\makeatother

\usepackage{bookmark}

\IfFileExists{xurl.sty}{\usepackage{xurl}}{} % add URL line breaks if available
\urlstyle{same} % disable monospaced font for URLs
\hypersetup{
  pdftitle={Evidence for Declining Trust in Science From A Large National Panel Study in New Zealand (years 2019-2024)},
  pdfauthor={Authors},
  colorlinks=true,
  linkcolor={blue},
  filecolor={Maroon},
  citecolor={Blue},
  urlcolor={Blue},
  pdfcreator={LaTeX via pandoc}}


\title{Evidence for Declining Trust in Science From A Large National
Panel Study in New Zealand (years 2019-2024)}

\usepackage{academicons}
\usepackage{xcolor}

  \author{Authors}
            \affil{%
             \small{     New Zealand
          ORCID \textcolor[HTML]{A6CE39}{\aiOrcid} ~0000-0003-3169-6576 }
              }
      


\date{2024-12-08}
\begin{document}
\maketitle
\begin{abstract}
The public's perception of science has wide-ranging effects, from the
adoption public health behaviours to climate action. This study uses
nationally representative panel data from New Zealand to assess changes
in trust in science and trust in scientists between late 2019 and late
2024. We draw on a large cohort (N = 42,681, New Zealand Attitudes and
Values Study). Study 1 uses multiple imputation to address biases from
systematic attrition, ensuring more accurate population estimates of
trust. Study 2 models average trust responses over time, showing that
trust in science and scientists rose following New Zealand's Covid-19
response, but later declined. We also find evidence for variation above
and below the mean. Study 3 assesses proportional shifts across low,
medium, and high levels of trust, revealing rising mistrust at both the
lower and higher ends. Considering these shifts at a population level
suggests that nearly 60,000 New Zealanders who once held moderate trust
may now exhibit low trust in science. \textbf{KEYWORDS}:
\emph{Conservativism}; \emph{Institutional Trust}; \emph{Longitudinal};
\emph{Panel}; \emph{Political}; \emph{Science}.
\end{abstract}


\subsection{Introduction}\label{introduction}

\subsubsection{Study 2: Categorical Mistrust in Science and Scientists
over
Time}\label{study-2-categorical-mistrust-in-science-and-scientists-over-time}

\begin{figure}

\centering{

\pandocbounded{\includegraphics[keepaspectratio]{manuscripts_files/figure-pdf/fig-science-categorical-1.pdf}}

}

\caption{\label{fig-science-categorical}(A) Predicted probability of
classification for trust in science without adjusting for missingness
(B) Predicted probability of classification for trust in science
adjusting for missingness. Responses categories are low (1-3), medium
(4-5), high (6-7) on the 1-7 ordinal scale.}

\end{figure}%

\begin{table}

\caption{\label{tbl-categorical-science-observed}Predicted probability
of classification for trust in science without adjusting for missing
responses. Responses categories are low (1-3), medium (4-5), high
(6-7).}

\centering{

\centering\begingroup\fontsize{12}{14}\selectfont

\begin{tabular}[t]{lll}
\toprule
Years & Predicted (95\% CI) & Response\\
\midrule
\cellcolor{gray!10}{0} & \cellcolor{gray!10}{0.09 (0.09, 0.09)} & \cellcolor{gray!10}{low}\\
1 & 0.08 (0.08, 0.09) & low\\
\cellcolor{gray!10}{2} & \cellcolor{gray!10}{0.08 (0.08, 0.08)} & \cellcolor{gray!10}{low}\\
3 & 0.08 (0.08, 0.08) & low\\
\cellcolor{gray!10}{4} & \cellcolor{gray!10}{0.07 (0.07, 0.08)} & \cellcolor{gray!10}{low}\\
\addlinespace
0 & 0.32 (0.31, 0.32) & med\\
\cellcolor{gray!10}{1} & \cellcolor{gray!10}{0.25 (0.24, 0.25)} & \cellcolor{gray!10}{med}\\
2 & 0.23 (0.23, 0.24) & med\\
\cellcolor{gray!10}{3} & \cellcolor{gray!10}{0.24 (0.23, 0.24)} & \cellcolor{gray!10}{med}\\
4 & 0.24 (0.23, 0.24) & med\\
\addlinespace
\cellcolor{gray!10}{0} & \cellcolor{gray!10}{0.59 (0.59, 0.60)} & \cellcolor{gray!10}{high}\\
1 & 0.67 (0.66, 0.67) & high\\
\cellcolor{gray!10}{2} & \cellcolor{gray!10}{0.69 (0.68, 0.69)} & \cellcolor{gray!10}{high}\\
3 & 0.68 (0.68, 0.69) & high\\
\cellcolor{gray!10}{4} & \cellcolor{gray!10}{0.69 (0.68, 0.70)} & \cellcolor{gray!10}{high}\\
\bottomrule
\end{tabular}
\endgroup{}

}

\end{table}%

\begin{table}

\caption{\label{tbl-categorical-science-imputed}Predicted probability of
classification for trust in science, now adjusting for missing
responses. Responses categories are low (1-3), medium (4-5), high
(6-7).}

\centering{

\centering\begingroup\fontsize{12}{14}\selectfont

\begin{tabular}[t]{lll}
\toprule
Years & Predicted (95\% CI) & Response\\
\midrule
\cellcolor{gray!10}{0} & \cellcolor{gray!10}{0.09 (0.09, 0.09)} & \cellcolor{gray!10}{low}\\
1 & 0.09 (0.09, 0.09) & low\\
\cellcolor{gray!10}{2} & \cellcolor{gray!10}{0.09 (0.09, 0.09)} & \cellcolor{gray!10}{low}\\
3 & 0.09 (0.09, 0.10) & low\\
\cellcolor{gray!10}{4} & \cellcolor{gray!10}{0.09 (0.09, 0.09)} & \cellcolor{gray!10}{low}\\
\addlinespace
0 & 0.32 (0.31, 0.32) & med\\
\cellcolor{gray!10}{1} & \cellcolor{gray!10}{0.27 (0.27, 0.28)} & \cellcolor{gray!10}{med}\\
2 & 0.27 (0.27, 0.28) & med\\
\cellcolor{gray!10}{3} & \cellcolor{gray!10}{0.29 (0.28, 0.29)} & \cellcolor{gray!10}{med}\\
4 & 0.30 (0.29, 0.30) & med\\
\addlinespace
\cellcolor{gray!10}{0} & \cellcolor{gray!10}{0.59 (0.58, 0.60)} & \cellcolor{gray!10}{high}\\
1 & 0.63 (0.63, 0.64) & high\\
\cellcolor{gray!10}{2} & \cellcolor{gray!10}{0.63 (0.63, 0.64)} & \cellcolor{gray!10}{high}\\
3 & 0.62 (0.61, 0.62) & high\\
\cellcolor{gray!10}{4} & \cellcolor{gray!10}{0.61 (0.61, 0.62)} & \cellcolor{gray!10}{high}\\
\bottomrule
\end{tabular}
\endgroup{}

}

\end{table}%

Figure~\ref{fig-science-categorical} displays predicted probabilities
for being classified as low (1--3), medium (4--5), or high (6--7) in
trust in science over time. Table~\ref{tbl-categorical-science-observed}
presents these probabilities without adjusting for missing responses.
Under these conditions, the proportion of respondents in the high
category appears to increase from the baseline and remain relatively
elevated. However, after adjusting for missingness (see
Table~\ref{tbl-categorical-science-imputed}), the predicted
probabilities show a less stable pattern, with gains in the high
category tempered and the medium category becoming more prominent. These
differences underscore that ignoring missing responses can create an
artificial appearance of stronger sustained trust.

\begin{figure}

\centering{

\pandocbounded{\includegraphics[keepaspectratio]{manuscripts_files/figure-pdf/fig-scientists-categorical-1.pdf}}

}

\caption{\label{fig-scientists-categorical}(A) Predicted probability of
classification for trust in scientists without adjusting for missingness
(B) Predicted probability of classification for trust in scientists
adjusting for missingness. Responses categories are low (1-3), medium
(4-5), high (6-7) on the 1-7 ordinal scale.}

\end{figure}%

\begin{table}

\caption{\label{tbl-categorical-scientists-observed}Predicted
probability of classification for trust in scientists without adjusting
for missing responses. Responses categories are low (1-3), medium (4-5),
high (6-7).}

\centering{

\centering\begingroup\fontsize{12}{14}\selectfont

\begin{tabular}[t]{lll}
\toprule
Years & Predicted (95\% CI) & Response\\
\midrule
\cellcolor{gray!10}{0} & \cellcolor{gray!10}{0.12 (0.11, 0.12)} & \cellcolor{gray!10}{low}\\
1 & 0.09 (0.09, 0.09) & low\\
\cellcolor{gray!10}{2} & \cellcolor{gray!10}{0.08 (0.08, 0.08)} & \cellcolor{gray!10}{low}\\
3 & 0.08 (0.08, 0.09) & low\\
\cellcolor{gray!10}{4} & \cellcolor{gray!10}{0.09 (0.09, 0.09)} & \cellcolor{gray!10}{low}\\
\addlinespace
0 & 0.34 (0.33, 0.34) & med\\
\cellcolor{gray!10}{1} & \cellcolor{gray!10}{0.29 (0.28, 0.29)} & \cellcolor{gray!10}{med}\\
2 & 0.28 (0.28, 0.28) & med\\
\cellcolor{gray!10}{3} & \cellcolor{gray!10}{0.30 (0.29, 0.30)} & \cellcolor{gray!10}{med}\\
4 & 0.33 (0.32, 0.33) & med\\
\addlinespace
\cellcolor{gray!10}{0} & \cellcolor{gray!10}{0.55 (0.54, 0.55)} & \cellcolor{gray!10}{high}\\
1 & 0.62 (0.62, 0.63) & high\\
\cellcolor{gray!10}{2} & \cellcolor{gray!10}{0.64 (0.63, 0.64)} & \cellcolor{gray!10}{high}\\
3 & 0.62 (0.61, 0.62) & high\\
\cellcolor{gray!10}{4} & \cellcolor{gray!10}{0.58 (0.58, 0.59)} & \cellcolor{gray!10}{high}\\
\bottomrule
\end{tabular}
\endgroup{}

}

\end{table}%

\begin{table}

\caption{\label{tbl-categorical-scientists-imputed}Predicted probability
of classification for trust in scientists, now adjusting for missing
responses. Responses categories are low (1-3), medium (4-5), high
(6-7).}

\centering{

\centering\begingroup\fontsize{12}{14}\selectfont

\begin{tabular}[t]{lll}
\toprule
Years & Predicted (95\% CI) & Response\\
\midrule
\cellcolor{gray!10}{0} & \cellcolor{gray!10}{0.12 (0.12, 0.12)} & \cellcolor{gray!10}{low}\\
1 & 0.11 (0.10, 0.11) & low\\
\cellcolor{gray!10}{2} & \cellcolor{gray!10}{0.11 (0.11, 0.11)} & \cellcolor{gray!10}{low}\\
3 & 0.12 (0.12, 0.12) & low\\
\cellcolor{gray!10}{4} & \cellcolor{gray!10}{0.13 (0.13, 0.13)} & \cellcolor{gray!10}{low}\\
\addlinespace
0 & 0.34 (0.33, 0.35) & med\\
\cellcolor{gray!10}{1} & \cellcolor{gray!10}{0.31 (0.31, 0.32)} & \cellcolor{gray!10}{med}\\
2 & 0.31 (0.31, 0.32) & med\\
\cellcolor{gray!10}{3} & \cellcolor{gray!10}{0.33 (0.33, 0.34)} & \cellcolor{gray!10}{med}\\
4 & 0.35 (0.35, 0.36) & med\\
\addlinespace
\cellcolor{gray!10}{0} & \cellcolor{gray!10}{0.54 (0.53, 0.55)} & \cellcolor{gray!10}{high}\\
1 & 0.58 (0.58, 0.59) & high\\
\cellcolor{gray!10}{2} & \cellcolor{gray!10}{0.58 (0.57, 0.58)} & \cellcolor{gray!10}{high}\\
3 & 0.55 (0.54, 0.56) & high\\
\cellcolor{gray!10}{4} & \cellcolor{gray!10}{0.52 (0.51, 0.52)} & \cellcolor{gray!10}{high}\\
\bottomrule
\end{tabular}
\endgroup{}

}

\end{table}%

A similar pattern emerges for trust in scientists (see
Figure~\ref{fig-scientists-categorical}). Without adjusting for
missingness, the results (shown in
Table~\ref{tbl-categorical-scientists-observed}) suggest that the
proportion classified as high in trust remains consistently large. Once
we incorporate multiple imputation to address missing responses, as
presented in Table~\ref{tbl-categorical-scientists-imputed}, the
predicted probabilities shift. The high category becomes less dominant
over time, and more respondents fall into the medium category than would
have been inferred without proper adjustment. In other words, accounting
for nonresponse reveals a more nuanced pattern of trust that does not
simply trend upward, and this correction prevents inflated estimates of
stable or increasing trust in scientists.

\subsection{Discussion}\label{discussion}

These findings illustrate that failing to adjust for missing responses
can systematically distort estimates of public trust in science and
scientists. Compared to models that ignore missingness, our
imputation-adjusted analyses indicate that the unadjusted models
overestimate the proportion of the population with high trust by several
hundred thousand New Zealand adults and underestimate the proportion
with low trust by tens to hundreds of thousands. For example, when trust
in science is not adjusted for missingness, we appear to be
underestimating the share of the population with low trust by about
82,000 people, while simultaneously overstating the size of the
high-trust group by roughly 330,000. Similarly, for trust in scientists,
the discrepancy in the low-trust population is about 165,000
individuals, and the high-trust category may be inflated by about
248,000. These differences represent substantial distortions in
population-level estimations.

It is important to note that these are minimum bounds of potential
misestimation, as we cannot measure trust among individuals who declined
participation at the study's outset. Moreover, we have not established
any causal relationships, and our objective is not to identify
determinants of changing trust, but rather to describe and characterise
differences in trust distributions over time. Still, given that trust in
science can influence critical actions related to climate change,
vaccine uptake, and following health advice, accurately documenting
trust and its temporal shifts is a pressing concern. By accounting for
missing data, we gain a clearer and more credible picture of
population-level trust in science and scientists, enabling
better-informed public health messaging and policy decisions.

\begin{table}

\caption{\label{tbl-categorical-science-imputed}Predicted probability of
classification for trust in science, now adjusting for missing
responses. Responses categories are low (1-3), medium (4-5), high
(6-7).}

\centering{

\begin{tabular}[t]{rrrrrl}
\toprule
Years & predicted & std.error & conf.low & conf.high & response.level\\
\midrule
0 & 0.09 & 0 & 0.09 & 0.09 & low\\
0 & 0.32 & 0 & 0.31 & 0.32 & med\\
0 & 0.59 & 0 & 0.58 & 0.60 & high\\
1 & 0.09 & 0 & 0.09 & 0.09 & low\\
1 & 0.27 & 0 & 0.27 & 0.28 & med\\
\addlinespace
1 & 0.63 & 0 & 0.63 & 0.64 & high\\
2 & 0.09 & 0 & 0.09 & 0.09 & low\\
2 & 0.27 & 0 & 0.27 & 0.28 & med\\
2 & 0.63 & 0 & 0.63 & 0.64 & high\\
3 & 0.09 & 0 & 0.09 & 0.10 & low\\
\addlinespace
3 & 0.29 & 0 & 0.28 & 0.29 & med\\
3 & 0.62 & 0 & 0.61 & 0.62 & high\\
4 & 0.09 & 0 & 0.09 & 0.09 & low\\
4 & 0.30 & 0 & 0.29 & 0.30 & med\\
4 & 0.61 & 0 & 0.61 & 0.62 & high\\
\bottomrule
\end{tabular}

}

\end{table}%




\end{document}
