% Options for packages loaded elsewhere
\PassOptionsToPackage{unicode}{hyperref}
\PassOptionsToPackage{hyphens}{url}
\PassOptionsToPackage{dvipsnames,svgnames,x11names}{xcolor}
%
\documentclass[
  single column]{article}

\usepackage{amsmath,amssymb}
\usepackage{iftex}
\ifPDFTeX
  \usepackage[T1]{fontenc}
  \usepackage[utf8]{inputenc}
  \usepackage{textcomp} % provide euro and other symbols
\else % if luatex or xetex
  \usepackage{unicode-math}
  \defaultfontfeatures{Scale=MatchLowercase}
  \defaultfontfeatures[\rmfamily]{Ligatures=TeX,Scale=1}
\fi
\usepackage[]{libertinus}
\ifPDFTeX\else  
    % xetex/luatex font selection
\fi
% Use upquote if available, for straight quotes in verbatim environments
\IfFileExists{upquote.sty}{\usepackage{upquote}}{}
\IfFileExists{microtype.sty}{% use microtype if available
  \usepackage[]{microtype}
  \UseMicrotypeSet[protrusion]{basicmath} % disable protrusion for tt fonts
}{}
\makeatletter
\@ifundefined{KOMAClassName}{% if non-KOMA class
  \IfFileExists{parskip.sty}{%
    \usepackage{parskip}
  }{% else
    \setlength{\parindent}{0pt}
    \setlength{\parskip}{6pt plus 2pt minus 1pt}}
}{% if KOMA class
  \KOMAoptions{parskip=half}}
\makeatother
\usepackage{xcolor}
\usepackage[top=30mm,left=25mm,heightrounded,headsep=22pt,headheight=11pt,footskip=33pt,ignorehead,ignorefoot]{geometry}
\setlength{\emergencystretch}{3em} % prevent overfull lines
\setcounter{secnumdepth}{-\maxdimen} % remove section numbering
% Make \paragraph and \subparagraph free-standing
\makeatletter
\ifx\paragraph\undefined\else
  \let\oldparagraph\paragraph
  \renewcommand{\paragraph}{
    \@ifstar
      \xxxParagraphStar
      \xxxParagraphNoStar
  }
  \newcommand{\xxxParagraphStar}[1]{\oldparagraph*{#1}\mbox{}}
  \newcommand{\xxxParagraphNoStar}[1]{\oldparagraph{#1}\mbox{}}
\fi
\ifx\subparagraph\undefined\else
  \let\oldsubparagraph\subparagraph
  \renewcommand{\subparagraph}{
    \@ifstar
      \xxxSubParagraphStar
      \xxxSubParagraphNoStar
  }
  \newcommand{\xxxSubParagraphStar}[1]{\oldsubparagraph*{#1}\mbox{}}
  \newcommand{\xxxSubParagraphNoStar}[1]{\oldsubparagraph{#1}\mbox{}}
\fi
\makeatother


\providecommand{\tightlist}{%
  \setlength{\itemsep}{0pt}\setlength{\parskip}{0pt}}\usepackage{longtable,booktabs,array}
\usepackage{calc} % for calculating minipage widths
% Correct order of tables after \paragraph or \subparagraph
\usepackage{etoolbox}
\makeatletter
\patchcmd\longtable{\par}{\if@noskipsec\mbox{}\fi\par}{}{}
\makeatother
% Allow footnotes in longtable head/foot
\IfFileExists{footnotehyper.sty}{\usepackage{footnotehyper}}{\usepackage{footnote}}
\makesavenoteenv{longtable}
\usepackage{graphicx}
\makeatletter
\newsavebox\pandoc@box
\newcommand*\pandocbounded[1]{% scales image to fit in text height/width
  \sbox\pandoc@box{#1}%
  \Gscale@div\@tempa{\textheight}{\dimexpr\ht\pandoc@box+\dp\pandoc@box\relax}%
  \Gscale@div\@tempb{\linewidth}{\wd\pandoc@box}%
  \ifdim\@tempb\p@<\@tempa\p@\let\@tempa\@tempb\fi% select the smaller of both
  \ifdim\@tempa\p@<\p@\scalebox{\@tempa}{\usebox\pandoc@box}%
  \else\usebox{\pandoc@box}%
  \fi%
}
% Set default figure placement to htbp
\def\fps@figure{htbp}
\makeatother
% definitions for citeproc citations
\NewDocumentCommand\citeproctext{}{}
\NewDocumentCommand\citeproc{mm}{%
  \begingroup\def\citeproctext{#2}\cite{#1}\endgroup}
\makeatletter
 % allow citations to break across lines
 \let\@cite@ofmt\@firstofone
 % avoid brackets around text for \cite:
 \def\@biblabel#1{}
 \def\@cite#1#2{{#1\if@tempswa , #2\fi}}
\makeatother
\newlength{\cslhangindent}
\setlength{\cslhangindent}{1.5em}
\newlength{\csllabelwidth}
\setlength{\csllabelwidth}{3em}
\newenvironment{CSLReferences}[2] % #1 hanging-indent, #2 entry-spacing
 {\begin{list}{}{%
  \setlength{\itemindent}{0pt}
  \setlength{\leftmargin}{0pt}
  \setlength{\parsep}{0pt}
  % turn on hanging indent if param 1 is 1
  \ifodd #1
   \setlength{\leftmargin}{\cslhangindent}
   \setlength{\itemindent}{-1\cslhangindent}
  \fi
  % set entry spacing
  \setlength{\itemsep}{#2\baselineskip}}}
 {\end{list}}
\usepackage{calc}
\newcommand{\CSLBlock}[1]{\hfill\break\parbox[t]{\linewidth}{\strut\ignorespaces#1\strut}}
\newcommand{\CSLLeftMargin}[1]{\parbox[t]{\csllabelwidth}{\strut#1\strut}}
\newcommand{\CSLRightInline}[1]{\parbox[t]{\linewidth - \csllabelwidth}{\strut#1\strut}}
\newcommand{\CSLIndent}[1]{\hspace{\cslhangindent}#1}

\usepackage{booktabs}
\usepackage{longtable}
\usepackage{array}
\usepackage{multirow}
\usepackage{wrapfig}
\usepackage{float}
\usepackage{colortbl}
\usepackage{pdflscape}
\usepackage{tabu}
\usepackage{threeparttable}
\usepackage{threeparttablex}
\usepackage[normalem]{ulem}
\usepackage{makecell}
\usepackage{xcolor}
\usepackage{tabularray}
\usepackage[normalem]{ulem}
\usepackage{graphicx}
\UseTblrLibrary{booktabs}
\UseTblrLibrary{rotating}
\UseTblrLibrary{siunitx}
\NewTableCommand{\tinytableDefineColor}[3]{\definecolor{#1}{#2}{#3}}
\newcommand{\tinytableTabularrayUnderline}[1]{\underline{#1}}
\newcommand{\tinytableTabularrayStrikeout}[1]{\sout{#1}}
\input{/Users/joseph/GIT/latex/latex-for-quarto.tex}
\makeatletter
\@ifpackageloaded{caption}{}{\usepackage{caption}}
\AtBeginDocument{%
\ifdefined\contentsname
  \renewcommand*\contentsname{Table of contents}
\else
  \newcommand\contentsname{Table of contents}
\fi
\ifdefined\listfigurename
  \renewcommand*\listfigurename{List of Figures}
\else
  \newcommand\listfigurename{List of Figures}
\fi
\ifdefined\listtablename
  \renewcommand*\listtablename{List of Tables}
\else
  \newcommand\listtablename{List of Tables}
\fi
\ifdefined\figurename
  \renewcommand*\figurename{Figure}
\else
  \newcommand\figurename{Figure}
\fi
\ifdefined\tablename
  \renewcommand*\tablename{Table}
\else
  \newcommand\tablename{Table}
\fi
}
\@ifpackageloaded{float}{}{\usepackage{float}}
\floatstyle{ruled}
\@ifundefined{c@chapter}{\newfloat{codelisting}{h}{lop}}{\newfloat{codelisting}{h}{lop}[chapter]}
\floatname{codelisting}{Listing}
\newcommand*\listoflistings{\listof{codelisting}{List of Listings}}
\makeatother
\makeatletter
\makeatother
\makeatletter
\@ifpackageloaded{caption}{}{\usepackage{caption}}
\@ifpackageloaded{subcaption}{}{\usepackage{subcaption}}
\makeatother

\usepackage{bookmark}

\IfFileExists{xurl.sty}{\usepackage{xurl}}{} % add URL line breaks if available
\urlstyle{same} % disable monospaced font for URLs
\hypersetup{
  pdftitle={Evidence for Declining Trust in Science From A Large National Panel Study in New Zealand (years 2019-2023)},
  pdfauthor={Authors (and author order TBA)},
  colorlinks=true,
  linkcolor={blue},
  filecolor={Maroon},
  citecolor={Blue},
  urlcolor={Blue},
  pdfcreator={LaTeX via pandoc}}


\title{Evidence for Declining Trust in Science From A Large National
Panel Study in New Zealand (years 2019-2023)}

\usepackage{academicons}
\usepackage{xcolor}

  \author{Authors (and author order TBA)}
            \affil{%
             \small{     New Zealand
          ORCID \textcolor[HTML]{A6CE39}{\aiOrcid} ~0000-0003-3169-6576 }
              }
      


\date{2024-11-11}
\begin{document}
\maketitle
\begin{abstract}
The public perceptions of science has wide-ranging effects, from the
adoption public health behaviours to climate action. Here, drawing on a
nationally diverse panel study in New Zealand, we report attitudes to
trust in the institution of science and trust in scientists from October
2019- October 2023 in a large nationally diverse cohort of (N = 42,681,
New Zealand Attitudes and Values Study) Study 1 focuses on systematic
biases from attrition in the 2019 cohort by comparing trust responses in
the observed sample with those imputed for the full 2019 cohort.
Multiple imputation reveals bias in attrition, particularly at the low
end of the trust in science and trust in scientist scales. Study 2
examines on population level stability and change in average trust
responses from 2019-2023; here we also examine stability and change by
ethnicty, gender, and political conservativism, revealing stability in
the mean but also considerable variation in response. Study 3 examines
proportional change in predicted probabilities across the low, medium,
and high ends of the response scale, finding evidence for increasing
mistrust at both the low end and high ends of the response spectrum.
Summing this change across the New Zealand population reveals
\{XXX\}-fold increase in science skeptism from peak trust immediately
following New Zealand's Covid pandemic response. \textbf{KEYWORDS}:
\emph{Conservativism}; \emph{Institutional Trust}; \emph{Longitudinal};
\emph{Panel}; \emph{Political}; \emph{Science}.
\end{abstract}


\subsection{Introduction}\label{introduction}

Whether people are growing more sceptical of science is question of
considerable interest and concern.

To address this question, we leverage for waves of comprehensive panel
data from 42,681 participants in the New Zealand Attitudes and Values
Study, spanning the years 2019-2023.

Study 2 reports averages response among the 2019 cohort during this
period, and also stratifies responses by ethnicity, gender (binary) and
political orientation.

Study 3 investigates dynamics across the response scale, considering
dynamics at the low, medium, and high end the trust in science and trust
in scientist scales.

The aim of each study is descriptive and exploratory, without testing
specific hypotheses. Our primary objective is to outline and
characterise the patterns of change in institutional trust, focusing on
indicators of trust in science and trust in scientists.

\subsection{Method}\label{method}

\subsubsection{Sample}\label{sample}

\subsubsection{Target Population}\label{target-population}

The target population for this study comprises the cohort New Zealand
residents in 2019 whose mistrust of science would not have prevent them
from participating in New Zealand Attitudes and Values Study that year.
Here our task is to infer population dynamics for this cohort uses
responses from individuals who participated in the New Zealand Attitudes
and Values Study (NZAVS) during 2019 the baseline wave for this study,
weighted by New Zealand Census weights for age, gender, and ethnicity
(refer to Sibley (\citeproc{ref-sibley2021}{2021})). The New Zealand
Attitudes and Values Study is a national probability study designed to
accurately reflect the broader New Zealand population. It uses prize
draws to incentivise participation. Although the New Zealand Attitudes
and Values Study has good demographic representation of the country as a
whole, it tends to under-sample males and individuals of Asian descent
and over-sample females and Māori (the indigenous peoples of New
Zealand). To address enhance the accuracy of our findings for the target
population, we apply 2018 New Zealand Census survey weights to the
sample data. These weights adjust for variations in age, gender, and
ethnicity to better approximate the national demographic composition
(\citeproc{ref-sibley2021}{Sibley 2021}).

\subsubsection{Eligibility Criteria}\label{eligibility-criteria}

To be eligible, participants needed to respond to the New Zealand
Attitudes and Values Study Time 11, years 2019-2020. Missing responses
were permitted in the baseline wave and all follow up waves through
Zealand Attitudes and Values Study Time 14, years 2022-2023. A total of
42,681 individuals met these criteria and were included in the study.
The proportion of missing data for each variable by wave is describe in
\hyperref[appendix-a]{Appendix A} Table~\ref{tbl-baseline} and
Table~\ref{tbl-trustscience}.

\subsection{Measures}\label{measures}

We estimated target population average responses for two indicators of
trust in science, which for simplicity we call ``trust in science'' and
``trust in scientists.''

\paragraph{Trust Science}\label{trust-science}

\emph{Our society places too much emphasis on science (reversed).}

Ordinal response: (1 = Strongly Disagree, 7 = Strongly Agree)
(\citeproc{ref-hartman2017}{Hartman \emph{et al.} 2017}).

\paragraph{Trust Scientists}\label{trust-scientists}

\emph{I have a high degree of confidence in the scientific community.}

Ordinal response: (1 = Strongly Disagree, 7 = Strongly Agree)
(\citeproc{ref-nisbet2015}{Nisbet \emph{et al.} 2015}).

Note that the term ``trust in science'' a shorthand for the value one
places on society's emphasis on science. It is plausible that at least
some who disagree with society's emphasis on science are nevertheless
trusting of scientific institutions. Some might perceive that society
would be better off emphasising culture, religion, the arts, family,
sport, or other domains than science while at the same time trusting
science. We use ``trust in science'' as a convenient short-hand for a
more complex construct

\subsubsection{Study Design}\label{study-design}

In Study 1, we examine potential biases in a attrition by reporting
sample responses over time. Here we report responses in the observed
sample and compare these to multiply imputed responses for the full 2019
cohort. This study clarifies the magnitude of threat for inferences
about population-level trust in science and scientists from non-response
among those who become sceptical of science or scientists.

In Study 2, we report population-level changes in average trust in
science and scientists for the New Zealand population from late 2019 to
late 2022, adjusting for attrition bias. We analyse response patterns
for the entire population and by subgroups defined by ethnicity, binary
gender, and political conservatism. Political conservatism, a continuous
variable, was standardised, and we evaluated predicted responses at
standard deviation units of \({-1, 0, 1, 2}\), representing a spectrum
from liberal (-1 SD from the mean) to conservative (+2 SD from the
mean). These values are within the range of political conservatism
observed in the 2019 cohort data, which spans from -1.7860193 to
2.4778948 standard deviations.

In Study 3, we examined change in trust response in the 2019 cohort at
the low (1-3), medium (4,5), and high (6,7) end of the response scale
both in the general population and in the population as defined by
ethnicity, binary gender, and political conservativism. The purpose of
this study is to investigate how patterns of trust vary across different
levels of response -- not merely at the average response -- and to
describe variation in these patterns across demographic groups defined
by ethnicity, binary gender, and political conservatism. Here we
additionally quantify the approximate sizes of the populations that
changed in low trust.

\subsubsection{Measures}\label{measures-1}

\paragraph{Trust Science}\label{trust-science-1}

\emph{Our society places too much emphasis on science (reversed).}

Ordinal response: (1 = Strongly Disagree, 7 = Strongly Agree)
(\citeproc{ref-hartman2017}{Hartman \emph{et al.} 2017}).

\paragraph{Trust Scientists}\label{trust-scientists-1}

\emph{I have a high degree of confidence in the scientific community.}

Ordinal response: (1 = Strongly Disagree, 7 = Strongly Agree)
(\citeproc{ref-nisbet2015}{Nisbet \emph{et al.} 2015}).

\paragraph{Ethnicity (Categorical)}\label{ethnicity-categorical}

\emph{Which ethnic group(s) do you belong to?}

Responses were coded as follows, using New Zealand Census standards: (1
= New Zealand European, 2 = Māori, 3 = Pacific, 4 = Asian).

\paragraph{Gender (Binary Indicator)}\label{gender-binary-indicator}

\emph{What is your gender?}

Gender was assessed through an open-ended question: ``What is your
gender?.'' Female was coded as 0, Male as 1, and gender diverse as 3 (or
0.5 for responses indicating neither female nor male) following the
coding system described by Fraser et al.~(2020). For the purpose of this
analysis, we coded all participants who identified as Male as 1 and all
others as 0 (\citeproc{ref-fraser_coding_2020}{Fraser \emph{et al.}
2020}).

\paragraph{Political Conservatism}\label{political-conservatism}

\emph{Please rate how politically liberal versus conservative you see
yourself as being.}

Responses were recorded on an ordinal scale from 1 (Extremely Liberal)
to 7 (Extremely Conservative) (\citeproc{ref-jost_end_2006-1}{Jost
2006}).

\subsubsection{Missing Data}\label{missing-data}

Our objective is to estimate the population levels of trust in science
and scientists in the general New Zealand population from late 2019 to
late 2023 using data from the New Zealand Attitudes and Values Study
(NZAVS). We face two primary data challenges when estimating
population-level trends.

First, although the NZAVS is a national probability study that recruits
participants using randomised mailouts from the New Zealand Electoral
Roll, only approximately 10\% of those invited choose to participate. It
is plausible that the proportion of science sceptics is higher among
those who declined participation, as the NZAVS is conducted by
scientists with the aim of fostering scientific understanding.

Second, although the NZAVS maintains an annual retention rate between
70--80\% of its sample, attrition over time is inevitable in any
longitudinal panel. It is credible that participants who develop
scepticism towards science or scientists may be more likely to drop out,
leading to an overestimation of trust levels in the remaining sample.

Although we cannot directly address the first challenge -- we cannot
accurately estimate the density of those who mistrust science among
those who never participated in our scientific study -- the second
challenge, inferring mistrust among those participated in the New
Zealand Attitudes and Values Study in 2019 and subsequently abandoned
the study, can potentially be mitigated. If the probability of missing
responses, both within a wave and over time, can be assumed to be
conditionally independent given observed covariates, we can apply
multiple imputation methods to adjust for bias in attrition. This
approach allows us to systematically incorporate the uncertainty arising
from missing data into our estimates
(\citeproc{ref-blackwell_2017_unified}{Blackwell \emph{et al.} 2017};
\citeproc{ref-bulbulia2023a}{Bulbulia \emph{et al.} 2023}).

Here, we use the Amelia package in R (\citeproc{ref-amelia_2011}{Honaker
\emph{et al.} 2011}) to create ten multiply imputed datasets for the
2019 New Zealand Attitudes and Values Study cohort for waves 11-14
(years 2019-2022). Trust in science and trust in scientists responses
were treated as order and ordinal. The \texttt{Amelia} package is
purpose-built for within-unit imputation in repeated-measures
time-series data. All covariates in Table~\ref{tbl-baseline} were
included in the imputation model, and the time variable (``year'') was
modelled as a cubic spline to account for non-linear trends over the
four-year period.

\subsubsection{Statistical Estimator}\label{statistical-estimator}

In Study 1 we report the retained sample means for trust in science and
trust in scientists from New Zealand Attitudes and Values Study waves
2019-2023 (waves 11-14). We also report retained sample frequencies for
response in low, medium, and high trust over this same time-frame.

To infer target population means and frequencies in the full cohort
population, we apply Rubin's rule to pool the estimates across 10
multiply imputed datasets. This method allows us to combine
within-imputation variance and between-imputation variance to obtain
more accurate estimates and associated uncertainty, thus reflecting the
variability inherent in the imputation process.

In Study 2, we examined mean responses for trust in science and trust in
scientists in the target population over time using generalised
estimating equations (GEE), using the \texttt{geepack} package in R
(\citeproc{ref-geepack_2006}{Halekoh \emph{et al.} 2006}) and applying
Rubin's rule to pool undercertainty over our estimates. Generalised
estimating equations provide robust standard errors for clustering in
the repeated measures, with fewer assumptions than are required for
multi-level models
(\citeproc{ref-mcneish_2017unnecessary}{\textbf{mcneish\_2017unnecessary?}}).
We specified GEE models using participant id as the clustering variable
to adjust for repeated observations across survey waves. To obtain
inferences for the New Zealand population, we incorporated sample
weights derived from the 2018 New Zealand Census data. We modelled
responses separately for the ten imputed datasets and then pooled
uncertainty over these estimates using Rubin's rule (employed using
\texttt{ggeffect:pool\_predictions()}
(\citeproc{ref-ggeffects_2018}{Ludecke 2018})).

In Study 3, we focused the proportional change in predicted
probabilities across low, medium, and high categories of trust
responses. For this purpose, we employed neural networks using the
\texttt{nnet} package in R (\citeproc{ref-nnet_2002}{Venables and Ripley
2002}). The neural network models provided estimates of the probability
of responses falling into each category across different survey waves.
The outputs included predicted probabilities for each response category,
again pooling uncertainty over the models for each imputed dataset using
Rubin's rule. As with Study 2, we used sample weights constructed from
the 2018 New Zealand Census data to ensure that our estimates were
representative of the broader population, and cluster robust prediction
using cluster robust standard errors im \texttt{ggeffects}
(\citeproc{ref-ggeffects_2018}{Ludecke 2018}). Again we used sample
weights to adjust for any sampling biases and ensure that the analyses
produced estimates that generalise to the New Zealand adult population.
We graph the predicted means at confidence intervals using
\texttt{ggeffects}(\citeproc{ref-ggeffects_2018}{Ludecke 2018}). Tables
were created using \texttt{tinytable}
(\citeproc{ref-tinytable_2024}{Arel-Bundock 2024}); \texttt{ggplot2}
(\citeproc{ref-ggplot2_2016}{Wickham 2016}); and the \texttt{margot}
package (\citeproc{ref-margot2024}{Bulbulia 2024}).

\subsection{Results}\label{results}

\subsubsection{Study 1: Bias in Attrition From Mistrust in Science and
Scientists}\label{study-1-bias-in-attrition-from-mistrust-in-science-and-scientists}

\paragraph{Trust Scientists: Sample Means in Retained
Sample}\label{trust-scientists-sample-means-in-retained-sample}

Figure~\ref{fig-hist-outcomes} presents histograms of trust responses
for science and scientists over time in the observed sample. As evident
in the graphs, there is increasing average trust over time, with a large
boost in trust evident in 2020. This boost coincides with the initially
popular New Zealand Covid-19 pandemic response
(\citeproc{ref-sibley2020a}{Sibley \emph{et al.} 2020}). In the retained
sample, there is further growth in for trust in science, and stability
in trust in scientists, following an initial boost.

\newpage{}

\begin{figure}

\centering{

\pandocbounded{\includegraphics[keepaspectratio]{24-kerr-growth-trust-science_files/figure-pdf/fig-hist-outcomes-1.pdf}}

}

\caption{\label{fig-hist-outcomes}}

\end{figure}%

\begin{table}

\caption{\label{tbl-sample-means}Retained sample average response}

\centering{

\centering
\begin{tblr}[         %% tabularray outer open
]                     %% tabularray outer close
{                     %% tabularray inner open
colspec={Q[]Q[]Q[]Q[]},
cell{2}{1}={r=4,}{valign=h,cmd=\bfseries,},
cell{6}{1}={r=4,}{valign=h,cmd=\bfseries,},
column{1}={halign=l,},
column{2}={halign=l,},
column{3}={halign=l,},
column{4}={halign=l,},
}                     %% tabularray inner close
\toprule
Response & Year & mean & missing \\ \midrule %% TinyTableHeader
Trust Science & 2019 & 5.56 &    563 \\
Trust Science & 2020 & 5.8 &  9,632 \\
Trust Science & 2021 & 5.84 & 15,111 \\
Trust Science & 2022 & 5.84 & 18,221 \\
Trust Scientists & 2019 & 5.3 &  1,257 \\
Trust Scientists & 2020 & 5.55 & 10,172 \\
Trust Scientists & 2021 & 5.56 & 15,519 \\
Trust Scientists & 2022 & 5.55 & 18,948 \\
\bottomrule
\end{tblr}

}

\end{table}%

Table~\ref{tbl-sample-means} presents sample average responses for trust
in science over time. Between 2019 and 2022, the average trust levels in
both science and scientists suggests a gradient of increase. For Trust
in Science, the average score rose from 5.56 in 2019 to 5.84 by 2021,
maintaining the same level through 2022. This indicates a steady
improvement in trust over the period in the retained sample.

For Trust in Scientists, the average response increased from 5.30 in
2019 to 5.55 in 2020 and remained stable through 2021 and 2022.

However, number of respondents with unknown trust status increased
substantially over the years for both trust measures. For Trust in
Science, the count rose from 563 in 2019 to 18,221 in 2022. Similarly,
for Trust in Scientists, the unknown category grew from 1,257 in 2019 to
18,948 in 2022.

Hence, the sample data reveals both a positive trend in average trust
levels alongside a notable rise in the number of unknown responses over
the period. Of course, attrition is to be expected in a panel study.
However, we must be cautious when naively interpreting these data
because mistrust in science may be -- and credibly is -- systematically
related to panel attrition and non-response.

\paragraph{Trust Scientists Categorical Responses in Retained
Sample}\label{trust-scientists-categorical-responses-in-retained-sample}

\begin{longtable}[]{@{}
  >{\raggedright\arraybackslash}p{(\linewidth - 10\tabcolsep) * \real{0.2323}}
  >{\raggedright\arraybackslash}p{(\linewidth - 10\tabcolsep) * \real{0.0808}}
  >{\raggedright\arraybackslash}p{(\linewidth - 10\tabcolsep) * \real{0.1616}}
  >{\raggedright\arraybackslash}p{(\linewidth - 10\tabcolsep) * \real{0.1616}}
  >{\raggedright\arraybackslash}p{(\linewidth - 10\tabcolsep) * \real{0.1616}}
  >{\raggedright\arraybackslash}p{(\linewidth - 10\tabcolsep) * \real{0.1616}}@{}}
\caption{Retained sample proportions classified as low, medium, or
high}\label{tbl-sample-cat}\tabularnewline
\toprule\noalign{}
\begin{minipage}[b]{\linewidth}\raggedright
Response
\end{minipage} & \begin{minipage}[b]{\linewidth}\raggedright
Level
\end{minipage} & \begin{minipage}[b]{\linewidth}\raggedright
2019
\end{minipage} & \begin{minipage}[b]{\linewidth}\raggedright
2020
\end{minipage} & \begin{minipage}[b]{\linewidth}\raggedright
2021
\end{minipage} & \begin{minipage}[b]{\linewidth}\raggedright
2022
\end{minipage} \\
\midrule\noalign{}
\endfirsthead
\toprule\noalign{}
\begin{minipage}[b]{\linewidth}\raggedright
Response
\end{minipage} & \begin{minipage}[b]{\linewidth}\raggedright
Level
\end{minipage} & \begin{minipage}[b]{\linewidth}\raggedright
2019
\end{minipage} & \begin{minipage}[b]{\linewidth}\raggedright
2020
\end{minipage} & \begin{minipage}[b]{\linewidth}\raggedright
2021
\end{minipage} & \begin{minipage}[b]{\linewidth}\raggedright
2022
\end{minipage} \\
\midrule\noalign{}
\endhead
\bottomrule\noalign{}
\endlastfoot
\textbf{Trust Science} & low & 3434 (8.2\%) & 2465 (7.5\%) & 2012
(7.3\%) & 1740 (7.1\%) \\
& med & 13210 (31.4\%) & 7855 (23.8\%) & 6223 (22.6\%) & 5720
(23.4\%) \\
& high & 25474 (60.5\%) & 22729 (68.8\%) & 19335 (70.1\%) & 17000
(69.5\%) \\
\textbf{Trust Scientists} & low & 4797 (11.6\%) & 2818 (8.7\%) & 2477
(9.1\%) & 1910 (8.0\%) \\
& med & 14161 (34.2\%) & 9385 (28.9\%) & 7448 (27.4\%) & 7107
(29.9\%) \\
& high & 22466 (54.2\%) & 20306 (62.5\%) & 17237 (63.5\%) & 14716
(62.0\%) \\
\end{longtable}

\begin{figure}

\centering{

\pandocbounded{\includegraphics[keepaspectratio]{24-kerr-growth-trust-science_files/figure-pdf/fig-alluv-st-sample-1.pdf}}

}

\caption{\label{fig-alluv-st-sample}Alluvial plot suggests increasing
flow of trust in scientists for the retained sample.}

\end{figure}%

\begin{figure}

\centering{

\pandocbounded{\includegraphics[keepaspectratio]{24-kerr-growth-trust-science_files/figure-pdf/fig-alluv-scientists-sample-1.pdf}}

}

\caption{\label{fig-alluv-scientists-sample}Alluvial plot suggests
increasing flow of trust in science for the retained sample.}

\end{figure}%

Table~\ref{tbl-sample-cat} summaries responses by low (1-3), medium
(4,5) and high (6,7) levels of trust in science and in scientists as
estimated using multiple imputation for the full 2019 cohort.

For trust in science, the proportion of respondents reporting low trust
declined slightly from 8.2\% (3,434 respondents) in 2019 to 7.1\% (1,740
respondents) in 2022. The medium trust category also saw a decrease,
dropping from 31.4\% (13,210 respondents) in 2019 to 23.4\% (5,720
respondents) in 2022. Conversely, the proportion of respondents
reporting high trust in science increased from 60.5\% (25,474
respondents) in 2019 to 69.5\% (17,000 respondents) in 2022, peaking at
70.1\% (19,335 respondents) in 2021. The most significant rise in high
trust occurred between 2019 and 2020, after which the levels remained
relatively stable. \textbf{?@fig-alluv-science-sample} is an alluvial
graph that illustrates this upward shift in trust in science.

For trust in scientists, a similar pattern emerged. Low trust decreased
from 11.6\% (4,797 respondents) in 2019 to 8.0\% (1,910 respondents) in
2022. Medium trust declined from 34.2\% (14,161 respondents) in 2019 to
29.9\% (7,107 respondents) in 2022. High trust in scientists increased
from 54.2\% (22,466 respondents) in 2019 to 62.0\% (14,716 respondents)
in 2022, with the largest growth occurring between 2019 and 2020.
Figure~\ref{fig-alluv-scientists-sample} is an alluvial graph that
demonstrates this upward trend in trust in scientists.

Overall, these findings suggest that in the observed New Zealand
Attitudes and Values sample---that is, the portion of the 2019 cohort
that remained in the study and responded to trust in science
items---there was a noticeable shift towards higher levels of trust in
both science and scientists over time. This shift is characterised by
declines in low and medium trust and corresponding increases in high
trust, with the most pronounced changes observed between 2019 and 2020.

However, if those with low levels of trust in science or scientists are
more likely to abandon the panel study, then these observed responses
may overstate trust in science for the 2019 cohort. Hence, although the
data suggest a positive trend in average trust levels, an increasing
number of unknown responses implies potential bias in inferences about
the target population. That is, if we suppose, credibly, that those who
become mistrusting of science are more likely to drop out of a
scientific study, the picture of growth in trust may be overly positive.

\subsubsection{Multiply Imputed Full Cohort Means in Full 2019
Cohort}\label{multiply-imputed-full-cohort-means-in-full-2019-cohort}

\begin{table}

\caption{\label{tbl-sample-means-imp}Imputed average responses for the
full 2019 cohort}

\centering{

\centering
\begin{tblr}[         %% tabularray outer open
]                     %% tabularray outer close
{                     %% tabularray inner open
colspec={Q[]Q[]Q[]Q[]Q[]Q[]},
cell{2}{1}={r=4,}{valign=h,cmd=\bfseries,},
cell{6}{1}={r=4,}{valign=h,cmd=\bfseries,},
column{1}={halign=l,},
column{2}={halign=l,},
column{3}={halign=l,},
column{4}={halign=l,},
column{5}={halign=l,},
column{6}={halign=l,},
}                     %% tabularray inner close
\toprule
Response & wave & estimate & se & lower & upper \\ \midrule %% TinyTableHeader
Trust Science & 2019 & 5.56 & 0.00698 & 5.54 & 5.57 \\
Trust Science & 2020 & 5.68 & 0.00706 & 5.67 & 5.7 \\
Trust Science & 2021 & 5.64 & 0.00704 & 5.63 & 5.66 \\
Trust Science & 2022 & 5.61 & 0.00712 & 5.59 & 5.62 \\
Trust Scientists & 2019 & 5.28 & 0.00727 & 5.27 & 5.3 \\
Trust Scientists & 2020 & 5.42 & 0.00709 & 5.41 & 5.44 \\
Trust Scientists & 2021 & 5.37 & 0.00727 & 5.36 & 5.39 \\
Trust Scientists & 2022 & 5.32 & 0.00715 & 5.31 & 5.34 \\
\bottomrule
\end{tblr}

}

\end{table}%

Table~\ref{tbl-sample-means-imp} presents averages pooled over the
mutiply imputed data. For Trust in Science, the imputed average slightly
increases from 5.56 in 2019 to 5.68 in 2020 and then declined gradually
to 5.64 in 2021 and 5.61 in 2022.

For Trust in Scientists, the average starts at 5.28 in 2019 rising to to
5.42 in 2020, but steadily decreases to 5.37 in 2021 and further to 5.32
by 2022. These patterns suggest that initial improvements in average
trust levels following the New Zealand Pandemic response in 2020 were
followed by a slight decrease over the four-year period. Notably the
pattern evident in the imputed dataset differs markedly from the pattern
of increasing growth evident for the observed cohort, which suggests
steady growth in trust, both for science and scientists.

\subsubsection{Multiply Imputed Full Cohort Categorical
Responses}\label{multiply-imputed-full-cohort-categorical-responses}

\begin{figure}

\centering{

\pandocbounded{\includegraphics[keepaspectratio]{24-kerr-growth-trust-science_files/figure-pdf/fig-alluv-science-imp-1.pdf}}

}

\caption{\label{fig-alluv-science-imp}Alluvial plot suggests increasing
flow of trust in science for the retained sample.}

\end{figure}%

\begin{figure}

\centering{

\pandocbounded{\includegraphics[keepaspectratio]{24-kerr-growth-trust-science_files/figure-pdf/fig-alluv-scientists-imp-1.pdf}}

}

\caption{\label{fig-alluv-scientists-imp}Alluvial plot suggests
increasing flow of trust in science for the retained sample.}

\end{figure}%

\begin{longtable}[]{@{}
  >{\raggedright\arraybackslash}p{(\linewidth - 10\tabcolsep) * \real{0.2830}}
  >{\raggedright\arraybackslash}p{(\linewidth - 10\tabcolsep) * \real{0.0755}}
  >{\raggedright\arraybackslash}p{(\linewidth - 10\tabcolsep) * \real{0.1509}}
  >{\raggedright\arraybackslash}p{(\linewidth - 10\tabcolsep) * \real{0.1509}}
  >{\raggedright\arraybackslash}p{(\linewidth - 10\tabcolsep) * \real{0.1509}}
  >{\raggedright\arraybackslash}p{(\linewidth - 10\tabcolsep) * \real{0.1509}}@{}}
\caption{Cohort proportions classified as low, medium, or high
(imputed)}\label{tbl-sample-cat-imp}\tabularnewline
\toprule\noalign{}
\begin{minipage}[b]{\linewidth}\raggedright
Response
\end{minipage} & \begin{minipage}[b]{\linewidth}\raggedright
Level
\end{minipage} & \begin{minipage}[b]{\linewidth}\raggedright
2019
\end{minipage} & \begin{minipage}[b]{\linewidth}\raggedright
2020
\end{minipage} & \begin{minipage}[b]{\linewidth}\raggedright
2021
\end{minipage} & \begin{minipage}[b]{\linewidth}\raggedright
2022
\end{minipage} \\
\midrule\noalign{}
\endfirsthead
\toprule\noalign{}
\begin{minipage}[b]{\linewidth}\raggedright
Response
\end{minipage} & \begin{minipage}[b]{\linewidth}\raggedright
Level
\end{minipage} & \begin{minipage}[b]{\linewidth}\raggedright
2019
\end{minipage} & \begin{minipage}[b]{\linewidth}\raggedright
2020
\end{minipage} & \begin{minipage}[b]{\linewidth}\raggedright
2021
\end{minipage} & \begin{minipage}[b]{\linewidth}\raggedright
2022
\end{minipage} \\
\midrule\noalign{}
\endhead
\bottomrule\noalign{}
\endlastfoot
\textbf{Trust Science Factor} & low & 3523 (8.3\%) & 3782 (8.9\%) & 4003
(9.4\%) & 4163 (9.8\%) \\
& med & 13446 (31.5\%) & 11329 (26.5\%) & 11624 (27.2\%) & 12184
(28.5\%) \\
& high & 25712 (60.2\%) & 27570 (64.6\%) & 27054 (63.4\%) & 26334
(61.7\%) \\
\textbf{Trust Scientists Factor} & low & 5056 (11.8\%) & 4632 (10.9\%) &
5180 (12.1\%) & 5224 (12.2\%) \\
& med & 14660 (34.3\%) & 13279 (31.1\%) & 13345 (31.3\%) & 14498
(34.0\%) \\
& high & 22965 (53.8\%) & 24770 (58.0\%) & 24156 (56.6\%) & 22959
(53.8\%) \\
\end{longtable}

Table~\ref{tbl-sample-cat-imp} displays the distribution of responses
for Trust in Science and Trust in Scientists within pooling over the
tend multiply imputed datasets.

For the Trust Science Factor, the proportion of respondents reporting
low trust increased slightly from 8.3\% (3,523 respondents) in 2019 to
9.8\% (4,163 respondents) in 2022. Medium trust saw a decrease from
31.5\% (13,446 respondents) in 2019 to 28.5\% (12,184 respondents) in
2022. High trust fluctuated over the years, increasing from 60.2\%
(25,712 respondents) in 2019 to 64.6\% (27,570 respondents) in 2020,
before declining to 61.7\% (26,334 respondents) in 2022. This trend
indicates that while there was initial growth in high trust between 2019
and 2020, this was not sustained, leading to a slight decline in
subsequent years. Figure~\ref{fig-alluv-science-imp} presents an
increasingly polarising trend for trust in science in the full 2019
cohort.

For the Trust Scientists Factor, low trust remained relatively stable,
starting at 11.8\% (5,056 respondents) in 2019 and ending at 12.2\%
(5,224 respondents) in 2022. Medium trust showed a minor decline
initially, from 34.3\% (14,660 respondents) in 2019 to 31.1\% (13,279
respondents) in 2020, before returning to 34.0\% (14,498 respondents) in
2022. High trust increased from 53.8\% (22,965 respondents) in 2019 to
58.0\% (24,770 respondents) in 2020 but fell back to 53.8\% (22,959
respondents) in 2022, suggesting an initial rise in trust levels that
was not sustained in the long term. Figure~\ref{fig-alluv-science-imp}
presents an increasingly polarising trend for trust in science among the
full 2019 cohort

These results from the multiply imputed underscore the importance of
adjusting for non-response bias when estimating mistrust in science
within cohorts over time. When missing values are imputed, the pattern
of increasing trust observed in the retained sample reverses. Note that
our missing data model assumes that missing responses may be predicted
from observed variables. Because there are potentially unmeasured
variables that cause both mistrust and attrition, the proportional
declines in trust might be understated Furthermore, we emphasise again
that the patterns we observe will not generalise to the full population
of New Zealand in 2019 under the credible assumption that those who
mistrust in science or scientists are unlikely to have participated in
the baseline wave. Instead, we model shifts in trust among the
population whose disinclinations to trust in science or scientists did
not prevent them from participating in the 2019 panel.
Figure~\ref{fig-alluv-scientists-imp} presents likes presents what
appears to be increasingly polarising trend for trust in scientists
among the full 2019 cohort

To quantitatively evaluate patterns of stability and change for the
population, we next report statistical models that employ all ten
multiply imputed datasets. We emphasise that purposes are descriptive
and exploratory, we do not test specific hypothesis, or address causal
theories.

\subsection{Study 2: Change in Average Trust in Science
2019-2023}\label{study-2-change-in-average-trust-in-science-2019-2023}

\subsubsection{Study 2a: Average Trust in
Science}\label{study-2a-average-trust-in-science}

\begin{longtable}[]{@{}
  >{\raggedright\arraybackslash}p{(\linewidth - 4\tabcolsep) * \real{0.0972}}
  >{\raggedright\arraybackslash}p{(\linewidth - 4\tabcolsep) * \real{0.1667}}
  >{\raggedright\arraybackslash}p{(\linewidth - 4\tabcolsep) * \real{0.1806}}@{}}
\caption{Predicted population average trust in
science.}\label{tbl-marginal-gee-science}\tabularnewline
\toprule\noalign{}
\begin{minipage}[b]{\linewidth}\raggedright
wave
\end{minipage} & \begin{minipage}[b]{\linewidth}\raggedright
Predicted
\end{minipage} & \begin{minipage}[b]{\linewidth}\raggedright
95\% CI
\end{minipage} \\
\midrule\noalign{}
\endfirsthead
\toprule\noalign{}
\begin{minipage}[b]{\linewidth}\raggedright
wave
\end{minipage} & \begin{minipage}[b]{\linewidth}\raggedright
Predicted
\end{minipage} & \begin{minipage}[b]{\linewidth}\raggedright
95\% CI
\end{minipage} \\
\midrule\noalign{}
\endhead
\bottomrule\noalign{}
\endlastfoot
2019 & 5.52 & 5.38, 5.65 \\
2020 & 5.60 & 5.46, 5.74 \\
2021 & 5.58 & 5.44, 5.71 \\
2022 & 5.54 & 5.40, 5.68 \\
\end{longtable}

Table~\ref{tbl-marginal-gee-science} presents the predicted responses
for trust in science from 2019 to 2022. The highest predicted trust
level was observed in 2020 (5.60; 95\% CI: 5.46, 5.74), followed by
slight decreases in subsequent years, although with overlapping
confidence intervals.

\begin{longtable}[]{@{}
  >{\raggedright\arraybackslash}p{(\linewidth - 4\tabcolsep) * \real{0.0972}}
  >{\raggedright\arraybackslash}p{(\linewidth - 4\tabcolsep) * \real{0.1667}}
  >{\raggedright\arraybackslash}p{(\linewidth - 4\tabcolsep) * \real{0.1806}}@{}}
\caption{Predicted population average trust in
science.}\label{tbl-marginal-gee-scietists}\tabularnewline
\toprule\noalign{}
\begin{minipage}[b]{\linewidth}\raggedright
wave
\end{minipage} & \begin{minipage}[b]{\linewidth}\raggedright
Predicted
\end{minipage} & \begin{minipage}[b]{\linewidth}\raggedright
95\% CI
\end{minipage} \\
\midrule\noalign{}
\endfirsthead
\toprule\noalign{}
\begin{minipage}[b]{\linewidth}\raggedright
wave
\end{minipage} & \begin{minipage}[b]{\linewidth}\raggedright
Predicted
\end{minipage} & \begin{minipage}[b]{\linewidth}\raggedright
95\% CI
\end{minipage} \\
\midrule\noalign{}
\endhead
\bottomrule\noalign{}
\endlastfoot
2019 & 5.29 & 5.16, 5.43 \\
2020 & 5.40 & 5.26, 5.53 \\
2021 & 5.35 & 5.21, 5.49 \\
2022 & 5.29 & 5.15, 5.43 \\
\end{longtable}

\begin{figure}

\centering{

\pandocbounded{\includegraphics[keepaspectratio]{24-kerr-growth-trust-science_files/figure-pdf/fig-marginal-gee-1.pdf}}

}

\caption{\label{fig-marginal-gee}}

\end{figure}%

Table~\ref{tbl-marginal-gee-scietists} presents the predicted responses
for trust in scientists from 2019 to 2022. Again highest predicted value
was recorded in 2020 (5.40; 95\% CI: 5.26, 5.53), with minor declines in
2021 and 2022, again with overlapping confidence intervals.

Figure~\ref{fig-marginal-gee} graphs both results: (A) average trust in
science and (B) average trust in scientists, over time. And indicated in
the graph, there is considerable proportions of the population above and
below these average responses.

\subsubsection{Study 2b: Average Trust in Science by
Ethnicity}\label{study-2b-average-trust-in-science-by-ethnicity}

\begin{longtable}[]{@{}
  >{\raggedright\arraybackslash}p{(\linewidth - 4\tabcolsep) * \real{0.0972}}
  >{\raggedright\arraybackslash}p{(\linewidth - 4\tabcolsep) * \real{0.1667}}
  >{\raggedright\arraybackslash}p{(\linewidth - 4\tabcolsep) * \real{0.1806}}@{}}
\caption{Predicted values for trust in science by
ethnicity.}\label{tbl-marginal-gee-science-eth}\tabularnewline
\toprule\noalign{}
\begin{minipage}[b]{\linewidth}\raggedright
wave
\end{minipage} & \begin{minipage}[b]{\linewidth}\raggedright
Predicted
\end{minipage} & \begin{minipage}[b]{\linewidth}\raggedright
95\% CI
\end{minipage} \\
\midrule\noalign{}
\endfirsthead
\toprule\noalign{}
\begin{minipage}[b]{\linewidth}\raggedright
wave
\end{minipage} & \begin{minipage}[b]{\linewidth}\raggedright
Predicted
\end{minipage} & \begin{minipage}[b]{\linewidth}\raggedright
95\% CI
\end{minipage} \\
\midrule\noalign{}
\endhead
\bottomrule\noalign{}
\endlastfoot
\multicolumn{3}{@{}>{\raggedright\arraybackslash}p{(\linewidth - 4\tabcolsep) * \real{0.4444} + 4\tabcolsep}@{}}{%
eth\_cat: euro} \\
2019 & 5.65 & 5.54, 5.75 \\
2020 & 5.75 & 5.64, 5.86 \\
2021 & 5.69 & 5.58, 5.80 \\
2022 & 5.67 & 5.56, 5.78 \\
\multicolumn{3}{@{}>{\raggedright\arraybackslash}p{(\linewidth - 4\tabcolsep) * \real{0.4444} + 4\tabcolsep}@{}}{%
eth\_cat: maori} \\
2019 & 5.34 & 4.86, 5.82 \\
2020 & 5.39 & 4.89, 5.89 \\
2021 & 5.44 & 4.98, 5.91 \\
2022 & 5.31 & 4.82, 5.79 \\
\multicolumn{3}{@{}>{\raggedright\arraybackslash}p{(\linewidth - 4\tabcolsep) * \real{0.4444} + 4\tabcolsep}@{}}{%
eth\_cat: pacific} \\
2019 & 5.30 & 4.43, 6.16 \\
2020 & 5.22 & 4.27, 6.17 \\
2021 & 5.33 & 4.48, 6.19 \\
2022 & 5.28 & 4.34, 6.21 \\
\multicolumn{3}{@{}>{\raggedright\arraybackslash}p{(\linewidth - 4\tabcolsep) * \real{0.4444} + 4\tabcolsep}@{}}{%
eth\_cat: asian} \\
2019 & 5.21 & 4.67, 5.74 \\
2020 & 5.33 & 4.81, 5.85 \\
2021 & 5.30 & 4.77, 5.82 \\
2022 & 5.31 & 4.78, 5.85 \\
\end{longtable}

Table~\ref{tbl-marginal-gee-science-eth} shows the predicted responses
for trust in science over time, broken down by ethnicity. The results
indicate that the population identifying as European consistently
reported higher predicted trust levels compared to other ethnic groups,
with the peak in 2020 (\(5.75\), 95\% CI: \(5.64\), \(5.86\)). Trust
levels among Māori, Pacific, and Asian peoples were generally lower and
showed minor variations across the years. Māori respondents showed
predicted values ranging from \(5.34\) in 2019 to \(5.31\) in 2022. The
Pacific population reported fluctuating but lower trust levels, peaking
at \(5.33\) in 2021. Asians displayed modest increases, peaking in 2020
at \(5.33\) (95\% CI: \(4.81\), \(5.85\)). Overlapping confidence
intervals suggests uncertainty in the location and magnitude of these
changes, again with uncertainty.

\begin{longtable}[]{@{}
  >{\raggedright\arraybackslash}p{(\linewidth - 4\tabcolsep) * \real{0.0972}}
  >{\raggedright\arraybackslash}p{(\linewidth - 4\tabcolsep) * \real{0.1667}}
  >{\raggedright\arraybackslash}p{(\linewidth - 4\tabcolsep) * \real{0.1806}}@{}}
\caption{Predicted values of trust in scientists by
ethnicity.}\label{tbl-marginal-gee-scietists-eth}\tabularnewline
\toprule\noalign{}
\begin{minipage}[b]{\linewidth}\raggedright
wave
\end{minipage} & \begin{minipage}[b]{\linewidth}\raggedright
Predicted
\end{minipage} & \begin{minipage}[b]{\linewidth}\raggedright
95\% CI
\end{minipage} \\
\midrule\noalign{}
\endfirsthead
\toprule\noalign{}
\begin{minipage}[b]{\linewidth}\raggedright
wave
\end{minipage} & \begin{minipage}[b]{\linewidth}\raggedright
Predicted
\end{minipage} & \begin{minipage}[b]{\linewidth}\raggedright
95\% CI
\end{minipage} \\
\midrule\noalign{}
\endhead
\bottomrule\noalign{}
\endlastfoot
\multicolumn{3}{@{}>{\raggedright\arraybackslash}p{(\linewidth - 4\tabcolsep) * \real{0.4444} + 4\tabcolsep}@{}}{%
eth\_cat: euro} \\
2019 & 5.39 & 5.28, 5.50 \\
2020 & 5.50 & 5.39, 5.61 \\
2021 & 5.43 & 5.31, 5.54 \\
2022 & 5.38 & 5.27, 5.49 \\
\multicolumn{3}{@{}>{\raggedright\arraybackslash}p{(\linewidth - 4\tabcolsep) * \real{0.4444} + 4\tabcolsep}@{}}{%
eth\_cat: maori} \\
2019 & 5.06 & 4.56, 5.56 \\
2020 & 5.19 & 4.72, 5.66 \\
2021 & 5.19 & 4.70, 5.68 \\
2022 & 5.08 & 4.59, 5.58 \\
\multicolumn{3}{@{}>{\raggedright\arraybackslash}p{(\linewidth - 4\tabcolsep) * \real{0.4444} + 4\tabcolsep}@{}}{%
eth\_cat: pacific} \\
2019 & 5.07 & 4.12, 6.01 \\
2020 & 5.03 & 4.10, 5.96 \\
2021 & 5.10 & 4.15, 6.05 \\
2022 & 4.95 & 4.00, 5.91 \\
\multicolumn{3}{@{}>{\raggedright\arraybackslash}p{(\linewidth - 4\tabcolsep) * \real{0.4444} + 4\tabcolsep}@{}}{%
eth\_cat: asian} \\
2019 & 5.19 & 4.70, 5.68 \\
2020 & 5.29 & 4.79, 5.79 \\
2021 & 5.28 & 4.78, 5.79 \\
2022 & 5.20 & 4.69, 5.71 \\
\end{longtable}

\begin{figure}

\centering{

\pandocbounded{\includegraphics[keepaspectratio]{24-kerr-growth-trust-science_files/figure-pdf/fig-combo_eth_graph_gee-1.pdf}}

}

\caption{\label{fig-combo_eth_graph_gee}}

\end{figure}%

Table~\ref{tbl-marginal-gee-scietists-eth} presents the predicted values
for trust in scientists by ethnicity. Europeans again had the highest
predicted values across the time frame, peaking in 2020 at \(5.50\)
(95\% CI: \(5.39\), \(5.61\)). The predicted trust levels for Māori
ranged from \(5.06\) in 2019 to \(5.08\) in 2022, with slight increases
observed in 2020 and 2021. Pacific peoples showed more variation, with
the highest trust reported in 2021 at \(5.10\), while Asian peoples
expressed peak trust in 2020 at \(5.29\).

Figure~\ref{fig-combo_eth_graph_gee} graphs the results of trust in
science and trust in scientists by ethnicity. These analyses reveal
ethnic differences, with European respondents showing consistently
higher levels compared to other groups, and subtle yearly fluctuations
across all ethnicities. Again, perhaps the most striking feature is that
there is considerable proportions of the population above and below
these average responses with Ethnicities.

\subsubsection{Study 2c: Average Trust in Science by Binary
Gender}\label{study-2c-average-trust-in-science-by-binary-gender}

\begin{longtable}[]{@{}
  >{\raggedright\arraybackslash}p{(\linewidth - 4\tabcolsep) * \real{0.0972}}
  >{\raggedright\arraybackslash}p{(\linewidth - 4\tabcolsep) * \real{0.1667}}
  >{\raggedright\arraybackslash}p{(\linewidth - 4\tabcolsep) * \real{0.1806}}@{}}
\caption{Predicted values of trust\_science by gender
(binary)}\label{tbl-marginal-gee-science-male}\tabularnewline
\toprule\noalign{}
\begin{minipage}[b]{\linewidth}\raggedright
wave
\end{minipage} & \begin{minipage}[b]{\linewidth}\raggedright
Predicted
\end{minipage} & \begin{minipage}[b]{\linewidth}\raggedright
95\% CI
\end{minipage} \\
\midrule\noalign{}
\endfirsthead
\toprule\noalign{}
\begin{minipage}[b]{\linewidth}\raggedright
wave
\end{minipage} & \begin{minipage}[b]{\linewidth}\raggedright
Predicted
\end{minipage} & \begin{minipage}[b]{\linewidth}\raggedright
95\% CI
\end{minipage} \\
\midrule\noalign{}
\endhead
\bottomrule\noalign{}
\endlastfoot
\multicolumn{3}{@{}>{\raggedright\arraybackslash}p{(\linewidth - 4\tabcolsep) * \real{0.4444} + 4\tabcolsep}@{}}{%
male: not male} \\
2019 & 5.47 & 5.31, 5.64 \\
2020 & 5.60 & 5.42, 5.77 \\
2021 & 5.60 & 5.43, 5.77 \\
2022 & 5.54 & 5.38, 5.71 \\
\multicolumn{3}{@{}>{\raggedright\arraybackslash}p{(\linewidth - 4\tabcolsep) * \real{0.4444} + 4\tabcolsep}@{}}{%
male: male} \\
2019 & 5.57 & 5.34, 5.79 \\
2020 & 5.61 & 5.38, 5.84 \\
2021 & 5.55 & 5.33, 5.77 \\
2022 & 5.54 & 5.30, 5.77 \\
\end{longtable}

Table~\ref{tbl-marginal-gee-science-male} presents predicted trust in
science over time, disaggregated by gender. Predicted trust levels for
those not identifying as male were generally stable, starting at
\(5.47\) (95\% CI: \(5.31\), \(5.64\)) in 2019 and peaking in 2020 and
2021 at \(5.60\). By 2022, the predicted trust level slightly declined
to \(5.54\) (95\% CI: \(5.38\), \(5.71\)).

For the population identifying as male, predicted trust started at
\(5.57\) (95\% CI: \(5.34\), \(5.79\)) in 2019 and showed a peak in 2020
at \(5.61\) (95\% CI: \(5.38\), \(5.84\)). The levels remained
relatively stable through 2021 and 2022, with values of \(5.55\) and
\(5.54\) respectively.

\begin{longtable}[]{@{}
  >{\raggedright\arraybackslash}p{(\linewidth - 4\tabcolsep) * \real{0.0972}}
  >{\raggedright\arraybackslash}p{(\linewidth - 4\tabcolsep) * \real{0.1667}}
  >{\raggedright\arraybackslash}p{(\linewidth - 4\tabcolsep) * \real{0.1806}}@{}}
\caption{Predicted values of trust in scientists by gender
(binary)}\label{tbl-marginal-gee-scietists-male}\tabularnewline
\toprule\noalign{}
\begin{minipage}[b]{\linewidth}\raggedright
wave
\end{minipage} & \begin{minipage}[b]{\linewidth}\raggedright
Predicted
\end{minipage} & \begin{minipage}[b]{\linewidth}\raggedright
95\% CI
\end{minipage} \\
\midrule\noalign{}
\endfirsthead
\toprule\noalign{}
\begin{minipage}[b]{\linewidth}\raggedright
wave
\end{minipage} & \begin{minipage}[b]{\linewidth}\raggedright
Predicted
\end{minipage} & \begin{minipage}[b]{\linewidth}\raggedright
95\% CI
\end{minipage} \\
\midrule\noalign{}
\endhead
\bottomrule\noalign{}
\endlastfoot
\multicolumn{3}{@{}>{\raggedright\arraybackslash}p{(\linewidth - 4\tabcolsep) * \real{0.4444} + 4\tabcolsep}@{}}{%
male: not male} \\
2019 & 5.18 & 5.00, 5.36 \\
2020 & 5.35 & 5.18, 5.52 \\
2021 & 5.35 & 5.18, 5.52 \\
2022 & 5.28 & 5.11, 5.45 \\
\multicolumn{3}{@{}>{\raggedright\arraybackslash}p{(\linewidth - 4\tabcolsep) * \real{0.4444} + 4\tabcolsep}@{}}{%
male: male} \\
2019 & 5.43 & 5.22, 5.64 \\
2020 & 5.45 & 5.23, 5.66 \\
2021 & 5.36 & 5.13, 5.59 \\
2022 & 5.29 & 5.06, 5.52 \\
\end{longtable}

Table~\ref{tbl-marginal-gee-scietists-male} details the predicted trust
in scientists over time for male and not male population. The population
not identifying as male showed a steady increase from 2019, starting at
\(5.18\) (95\% CI: \(5.00\), \(5.36\)), reaching \(5.35\) in both 2020
and 2021, and then declining slightly to \(5.28\) (95\% CI: \(5.11\),
\(5.45\)) in 2022.

For the population identifying as male, predicted trust was higher at
the start, at \(5.43\) (95\% CI: \(5.22\), \(5.64\)) in 2019, and showed
a slight increase in 2020 to \(5.45\) (95\% CI: \(5.23\), \(5.66\)).
This trust level then declined slightly over the next two years,
reaching \(5.29\) (95\% CI: \(5.06\), \(5.52\)) by 2022.

Overall results highlight that predicted average trust levels over time
were mostly similar between genders, with both groups experiencing
slight fluctuations and peaks around 2020 before declining thereafter.

\begin{figure}

\centering{

\pandocbounded{\includegraphics[keepaspectratio]{24-kerr-growth-trust-science_files/figure-pdf/fig-combo_male_graph_gee-1.pdf}}

}

\caption{\label{fig-combo_male_graph_gee}}

\end{figure}%

Figure~\ref{fig-combo_male_graph_gee} graphs these both. Again, we find
considerable proportions in these populations above and below the
average responses.

\subsubsection{Study 2c: Average Trust in Science by Political
Conservativism}\label{study-2c-average-trust-in-science-by-political-conservativism}

\begin{longtable}[]{@{}
  >{\raggedright\arraybackslash}p{(\linewidth - 4\tabcolsep) * \real{0.0972}}
  >{\raggedright\arraybackslash}p{(\linewidth - 4\tabcolsep) * \real{0.1667}}
  >{\raggedright\arraybackslash}p{(\linewidth - 4\tabcolsep) * \real{0.1806}}@{}}
\caption{Predicted values of trust\_science by political conservativism
(standard
deviations)}\label{tbl-marginal-gee-science-pols}\tabularnewline
\toprule\noalign{}
\begin{minipage}[b]{\linewidth}\raggedright
wave
\end{minipage} & \begin{minipage}[b]{\linewidth}\raggedright
Predicted
\end{minipage} & \begin{minipage}[b]{\linewidth}\raggedright
95\% CI
\end{minipage} \\
\midrule\noalign{}
\endfirsthead
\toprule\noalign{}
\begin{minipage}[b]{\linewidth}\raggedright
wave
\end{minipage} & \begin{minipage}[b]{\linewidth}\raggedright
Predicted
\end{minipage} & \begin{minipage}[b]{\linewidth}\raggedright
95\% CI
\end{minipage} \\
\midrule\noalign{}
\endhead
\bottomrule\noalign{}
\endlastfoot
\multicolumn{3}{@{}>{\raggedright\arraybackslash}p{(\linewidth - 4\tabcolsep) * \real{0.4444} + 4\tabcolsep}@{}}{%
political\_conservative\_z: -1} \\
2019 & 5.89 & 5.72, 6.07 \\
2020 & 5.93 & 5.75, 6.11 \\
2021 & 5.88 & 5.70, 6.06 \\
2022 & 5.85 & 5.66, 6.04 \\
\multicolumn{3}{@{}>{\raggedright\arraybackslash}p{(\linewidth - 4\tabcolsep) * \real{0.4444} + 4\tabcolsep}@{}}{%
political\_conservative\_z: 0} \\
2019 & 5.52 & 5.38, 5.65 \\
2020 & 5.58 & 5.44, 5.71 \\
2021 & 5.57 & 5.43, 5.70 \\
2022 & 5.55 & 5.41, 5.69 \\
\multicolumn{3}{@{}>{\raggedright\arraybackslash}p{(\linewidth - 4\tabcolsep) * \real{0.4444} + 4\tabcolsep}@{}}{%
political\_conservative\_z: 1} \\
2019 & 5.14 & 4.93, 5.34 \\
2020 & 5.22 & 5.01, 5.44 \\
2021 & 5.25 & 5.05, 5.45 \\
2022 & 5.25 & 5.05, 5.45 \\
\multicolumn{3}{@{}>{\raggedright\arraybackslash}p{(\linewidth - 4\tabcolsep) * \real{0.4444} + 4\tabcolsep}@{}}{%
political\_conservative\_z: 2} \\
2019 & 4.76 & 4.43, 5.08 \\
2020 & 4.87 & 4.53, 5.20 \\
2021 & 4.94 & 4.62, 5.25 \\
2022 & 4.95 & 4.63, 5.27 \\
\end{longtable}

Table~\ref{tbl-marginal-gee-science-pols} presents the predicted trust
in science across different levels of political orientation, expressed
in standard deviation units (SD). The population with lower political
conservatism scores (\(-1\) SD) consistently reported the highest
predicted trust levels, starting at \(5.89\) (95\% CI: \(5.72\),
\(6.07\)) in 2019, peaking at \(5.93\) in 2020, and slightly declining
to \(5.85\) (95\% CI: \(5.66\), \(6.04\)) by 2022.

For the population with average political conservatism (\(0\) SD),
predicted trust levels ranged from \(5.52\) in 2019 to \(5.55\) in 2022,
showing a small peak in 2020 at \(5.58\) (95\% CI: \(5.44\), \(5.71\)).
The population with higher political conservatism (\(1\) SD) started
with lower predicted trust values, beginning at \(5.14\) (95\% CI:
\(4.93\), \(5.34\)) in 2019 and reaching \(5.25\) (95\% CI: \(5.05\),
\(5.45\)) by 2022.

\begin{longtable}[]{@{}
  >{\raggedright\arraybackslash}p{(\linewidth - 4\tabcolsep) * \real{0.0972}}
  >{\raggedright\arraybackslash}p{(\linewidth - 4\tabcolsep) * \real{0.1667}}
  >{\raggedright\arraybackslash}p{(\linewidth - 4\tabcolsep) * \real{0.1806}}@{}}
\caption{Predicted values of trust in scientists by political
conservativism (standard deviation
units)}\label{tbl-marginal-gee-scietists-pols}\tabularnewline
\toprule\noalign{}
\begin{minipage}[b]{\linewidth}\raggedright
wave
\end{minipage} & \begin{minipage}[b]{\linewidth}\raggedright
Predicted
\end{minipage} & \begin{minipage}[b]{\linewidth}\raggedright
95\% CI
\end{minipage} \\
\midrule\noalign{}
\endfirsthead
\toprule\noalign{}
\begin{minipage}[b]{\linewidth}\raggedright
wave
\end{minipage} & \begin{minipage}[b]{\linewidth}\raggedright
Predicted
\end{minipage} & \begin{minipage}[b]{\linewidth}\raggedright
95\% CI
\end{minipage} \\
\midrule\noalign{}
\endhead
\bottomrule\noalign{}
\endlastfoot
\multicolumn{3}{@{}>{\raggedright\arraybackslash}p{(\linewidth - 4\tabcolsep) * \real{0.4444} + 4\tabcolsep}@{}}{%
political\_conservative\_z: -1} \\
2019 & 5.67 & 5.50, 5.84 \\
2020 & 5.73 & 5.55, 5.91 \\
2021 & 5.66 & 5.46, 5.86 \\
2022 & 5.62 & 5.43, 5.81 \\
\multicolumn{3}{@{}>{\raggedright\arraybackslash}p{(\linewidth - 4\tabcolsep) * \real{0.4444} + 4\tabcolsep}@{}}{%
political\_conservative\_z: 0} \\
2019 & 5.29 & 5.16, 5.43 \\
2020 & 5.37 & 5.23, 5.50 \\
2021 & 5.34 & 5.21, 5.48 \\
2022 & 5.30 & 5.16, 5.43 \\
\multicolumn{3}{@{}>{\raggedright\arraybackslash}p{(\linewidth - 4\tabcolsep) * \real{0.4444} + 4\tabcolsep}@{}}{%
political\_conservative\_z: 1} \\
2019 & 4.91 & 4.72, 5.11 \\
2020 & 5.01 & 4.80, 5.22 \\
2021 & 5.03 & 4.82, 5.23 \\
2022 & 4.98 & 4.78, 5.18 \\
\multicolumn{3}{@{}>{\raggedright\arraybackslash}p{(\linewidth - 4\tabcolsep) * \real{0.4444} + 4\tabcolsep}@{}}{%
political\_conservative\_z: 2} \\
2019 & 4.54 & 4.23, 4.84 \\
2020 & 4.65 & 4.32, 4.98 \\
2021 & 4.71 & 4.38, 5.03 \\
2022 & 4.66 & 4.35, 4.98 \\
\end{longtable}

Table~\ref{tbl-marginal-gee-scietists-pols} presents the predicted trust
in scientists over time across different levels of political orientation
(in standard deviation units). The population with lower conservatism
scores (\(-1\) SD) reported the highest predicted trust levels, starting
at \(5.67\) (95\% CI: \(5.50\), \(5.84\)) in 2019 and peaking at
\(5.73\) (95\% CI: \(5.55\), \(5.91\)) in 2020. By 2022, trust declined
slightly to \(5.62\) (95\% CI: \(5.43\), \(5.81\)).

For those with average political orientation (\(0\) SD), predicted trust
values were lower, starting at \(5.29\) (95\% CI: \(5.16\), \(5.43\)) in
2019 and peaking in 2020 at \(5.37\) (95\% CI: \(5.23\), \(5.50\)). By
2022, trust declined to \(5.30\) (95\% CI: \(5.16\), \(5.43\)).

Those with higher conservatism (\(1\) SD) started with even lower
predicted trust levels, beginning at \(4.91\) (95\% CI: \(4.72\),
\(5.11\)) in 2019 and peaking modestly at \(5.03\) (95\% CI: \(4.82\),
\(5.23\)) in 2021. By 2022, trust levels fell to \(4.98\) (95\% CI:
\(4.78\), \(5.18\)).

The population with the highest conservatism scores (\(2\) SD) had the
lowest trust in scientists, starting at \(4.54\) (95\% CI: \(4.23\),
\(4.84\)) in 2019. Their trust peaked in 2021 at \(4.71\) (95\% CI:
\(4.38\), \(5.03\)) before declining again to \(4.66\) (95\% CI:
\(4.35\), \(4.98\)) in 2022.

These results demonstrate a clear pattern where the population with
lower conservatism show consistently higher predicted trust in
scientists, while those with higher conservatism have lower trust
levels, with considerable stability overtime, albeit with increasing
trust on the high end of the political conservativism spectrum.

Note we caution against any causal interpretation, as those already
trusting of science or of scientists might have become increasingly
conservative, a reminder that even in longitudinal data correlation is
not causation.

\begin{figure}

\centering{

\pandocbounded{\includegraphics[keepaspectratio]{24-kerr-growth-trust-science_files/figure-pdf/fig-combo_pols_graph_gee-1.pdf}}

}

\caption{\label{fig-combo_pols_graph_gee}}

\end{figure}%

Figure~\ref{fig-combo_pols_graph_gee} graphs again presents predicted
trust in science trends for the population by political orientation.
Despite a broad patter of stability in average responses, we again find
considerable proportions in these populations above and below the
average responses.

\subsection{Study 3: Proportional Change in High, Medium, and Low Trust
2019-2023}\label{study-3-proportional-change-in-high-medium-and-low-trust-2019-2023}

\subsection{Discussion}\label{discussion}

\newpage{}

\subsubsection{Ethics}\label{ethics}

The University of Auckland Human Participants Ethics Committee reviews
the NZAVS every three years. Our most recent ethics approval statement
is as follows: The New Zealand Attitudes and Values Study was approved
by the University of Auckland Human Participants Ethics Committee on
26/05/2021 for six years until 26/05/2027, Reference Number UAHPEC22576.

\subsubsection{Data Availability}\label{data-availability}

The data described in the paper are part of the New Zealand Attitudes
and Values Study. Members of the NZAVS management team and research
group hold full copies of the NZAVS data. A de-identified dataset
containing only the variables analysed in this manuscript is available
upon request from the corresponding author or any member of the NZAVS
advisory board for replication or checking of any published study using
NZAVS data. The code for the analysis can be found at:
\url{https://github.com/go-bayes/models/tree/main/scripts/24-POL-CONSERVE-environ}.

\subsubsection{Acknowledgements}\label{acknowledgements}

HERE

The New Zealand Attitudes and Values Study is supported by a grant from
the Templeton Religious Trust (TRT0196; TRT0418). JB received support
from the Max Plank Institute for the Science of Human History. The
funders had no role in preparing the manuscript or deciding to publish
it.

\subsubsection{Author Statement}\label{author-statement}

\textbf{TBA}: so far, JK led the study. CGS led data collection. JB
developed the analytic approach and analysis and did the graphs. All
authors contributed to the manuscript.

\newpage{}

\subsection{Appendix A: Measures}\label{appendix-a}

Table~\ref{tbl-baseline} presents descriptive statistics for baseline
covariates, measure in New Zealand Attitudes and Values Study Time 11,
years 2019-2020.

\subsubsection{Covariate measures at
baseline}\label{covariate-measures-at-baseline}

\begin{table}

\caption{\label{tbl-baseline}Sample descriptive statistics.}

\centering{

\textbar:---------------------------------\textbar:--------------------\textbar:--------------------\textbar:--------------------\textbar:--------------------\textbar{}
\textbar{}\textbf{Age} \textbar NA \textbar NA \textbar NA \textbar NA
\textbar{}
\textbar:---------------------------------\textbar:--------------------\textbar:--------------------\textbar:--------------------\textbar:--------------------\textbar{}
\textbar Mean (SD) \textbar52 (14) \textbar54 (14) \textbar55 (14)
\textbar57 (13) \textbar{} \textbar Min, Max \textbar18, 97 \textbar19,
97 \textbar20, 97 \textbar21, 98 \textbar{} \textbar Q1, Q3 \textbar42,
63 \textbar45, 64 \textbar47, 65 \textbar48, 67 \textbar{}
\textbar Unknown \textbar0 \textbar9,363 \textbar13,615 \textbar17,166
\textbar{} \textbar{}\textbf{Agreeableness} \textbar NA \textbar NA
\textbar NA \textbar NA \textbar{} \textbar Mean (SD) \textbar5.38
(0.98) \textbar5.38 (0.98) \textbar5.36 (1.00) \textbar5.35 (0.99)
\textbar{} \textbar Min, Max \textbar1.00, 7.00 \textbar1.00, 7.00
\textbar1.00, 7.00 \textbar1.00, 7.00 \textbar{} \textbar Q1, Q3
\textbar4.75, 6.00 \textbar4.75, 6.00 \textbar4.75, 6.00 \textbar4.75,
6.00 \textbar{} \textbar Unknown \textbar316 \textbar9,579
\textbar13,831 \textbar17,260 \textbar{} \textbar{}\textbf{Belong}
\textbar NA \textbar NA \textbar NA \textbar NA \textbar{} \textbar Mean
(SD) \textbar5.09 (1.09) \textbar5.02 (1.11) \textbar5.14 (1.12)
\textbar5.14 (1.11) \textbar{} \textbar Min, Max \textbar1.00, 7.00
\textbar1.00, 7.00 \textbar1.00, 7.00 \textbar1.00, 7.00 \textbar{}
\textbar Q1, Q3 \textbar4.33, 6.00 \textbar4.33, 6.00 \textbar4.33, 6.00
\textbar4.33, 6.00 \textbar{} \textbar Unknown \textbar324 \textbar9,587
\textbar13,885 \textbar17,303 \textbar{} \textbar{}\textbf{Born NZ}
\textbar33,316 (78\%) \textbar33,316 (78\%) \textbar33,316 (78\%)
\textbar33,316 (78\%) \textbar{} \textbar Unknown \textbar81 \textbar81
\textbar81 \textbar81 \textbar{} \textbar{}\textbf{Conscientiousness}
\textbar NA \textbar NA \textbar NA \textbar NA \textbar{} \textbar Mean
(SD) \textbar5.06 (1.07) \textbar5.10 (1.05) \textbar5.12 (1.06)
\textbar5.12 (1.05) \textbar{} \textbar Min, Max \textbar1.00, 7.00
\textbar1.00, 7.00 \textbar1.00, 7.00 \textbar1.00, 7.00 \textbar{}
\textbar Q1, Q3 \textbar4.25, 5.75 \textbar4.50, 6.00 \textbar4.50, 6.00
\textbar4.50, 6.00 \textbar{} \textbar Unknown \textbar313 \textbar9,577
\textbar13,823 \textbar17,259 \textbar{} \textbar{}\textbf{Education
Level Coarsen} \textbar NA \textbar NA \textbar NA \textbar NA
\textbar{} \textbar no\_qualification \textbar853 (2.0\%) \textbar754
(1.8\%) \textbar706 (1.7\%) \textbar711 (1.7\%) \textbar{}
\textbar cert\_1\_to\_4 \textbar13,117 (31\%) \textbar12,588 (30\%)
\textbar12,228 (29\%) \textbar11,989 (28\%) \textbar{}
\textbar cert\_5\_to\_6 \textbar5,606 (13\%) \textbar5,714 (14\%)
\textbar5,748 (14\%) \textbar5,807 (14\%) \textbar{} \textbar university
\textbar11,783 (28\%) \textbar11,856 (28\%) \textbar11,823 (28\%)
\textbar11,747 (28\%) \textbar{} \textbar post\_grad \textbar5,363
(13\%) \textbar5,614 (13\%) \textbar5,850 (14\%) \textbar5,992 (14\%)
\textbar{} \textbar masters \textbar4,256 (10\%) \textbar4,440 (10\%)
\textbar4,579 (11\%) \textbar4,685 (11\%) \textbar{} \textbar doctorate
\textbar1,280 (3.0\%) \textbar1,337 (3.2\%) \textbar1,381 (3.3\%)
\textbar1,426 (3.4\%) \textbar{} \textbar Unknown \textbar423
\textbar378 \textbar366 \textbar324 \textbar{}
\textbar{}\textbf{Employed} \textbar31,856 (76\%) \textbar25,061 (76\%)
\textbar21,235 (74\%) \textbar17,725 (71\%) \textbar{} \textbar Unknown
\textbar550 \textbar9,580 \textbar13,942 \textbar17,589 \textbar{}
\textbar{}\textbf{Ethnicity} \textbar NA \textbar NA \textbar NA
\textbar NA \textbar{} \textbar euro \textbar34,990 (83\%)
\textbar34,990 (83\%) \textbar34,990 (83\%) \textbar34,990 (83\%)
\textbar{} \textbar maori \textbar4,639 (11\%) \textbar4,639 (11\%)
\textbar4,639 (11\%) \textbar4,639 (11\%) \textbar{} \textbar pacific
\textbar929 (2.2\%) \textbar929 (2.2\%) \textbar929 (2.2\%) \textbar929
(2.2\%) \textbar{} \textbar asian \textbar1,770 (4.2\%) \textbar1,770
(4.2\%) \textbar1,770 (4.2\%) \textbar1,770 (4.2\%) \textbar{}
\textbar Unknown \textbar353 \textbar353 \textbar353 \textbar353
\textbar{} \textbar{}\textbf{Extraversion} \textbar NA \textbar NA
\textbar NA \textbar NA \textbar{} \textbar Mean (SD) \textbar3.87
(1.19) \textbar3.83 (1.19) \textbar3.77 (1.23) \textbar3.75 (1.23)
\textbar{} \textbar Min, Max \textbar1.00, 7.00 \textbar1.00, 7.00
\textbar1.00, 7.00 \textbar1.00, 7.00 \textbar{} \textbar Q1, Q3
\textbar3.00, 4.75 \textbar3.00, 4.75 \textbar3.00, 4.67 \textbar3.00,
4.50 \textbar{} \textbar Unknown \textbar314 \textbar9,577
\textbar13,832 \textbar17,274 \textbar{} \textbar{}\textbf{Honesty
Humility} \textbar NA \textbar NA \textbar NA \textbar NA \textbar{}
\textbar Mean (SD) \textbar5.56 (1.14) \textbar5.62 (1.11) \textbar5.68
(1.13) \textbar5.72 (1.12) \textbar{} \textbar Min, Max \textbar1.00,
7.00 \textbar1.00, 7.00 \textbar1.00, 7.00 \textbar1.00, 7.00 \textbar{}
\textbar Q1, Q3 \textbar4.75, 6.50 \textbar5.00, 6.50 \textbar5.00, 6.67
\textbar5.00, 6.75 \textbar{} \textbar Unknown \textbar323 \textbar9,587
\textbar13,816 \textbar17,252 \textbar{} \textbar{}\textbf{Hours
Commute} \textbar NA \textbar NA \textbar NA \textbar NA \textbar{}
\textbar Mean (SD) \textbar4.5 (7.0) \textbar4.4 (5.9) \textbar3.7 (5.4)
\textbar4.4 (6.2) \textbar{} \textbar Min, Max \textbar0.0, 168.0
\textbar0.0, 100.0 \textbar0.0, 100.0 \textbar0.0, 100.0 \textbar{}
\textbar Q1, Q3 \textbar1.0, 6.0 \textbar1.0, 6.0 \textbar1.0, 5.0
\textbar1.0, 6.0 \textbar{} \textbar Unknown \textbar829 \textbar10,233
\textbar14,611 \textbar18,092 \textbar{} \textbar{}\textbf{Household
Inc} \textbar NA \textbar NA \textbar NA \textbar NA \textbar{}
\textbar Mean (SD) \textbar117,971 (107,738) \textbar120,324 (106,017)
\textbar124,450 (111,893) \textbar130,963 (146,750) \textbar{}
\textbar Min, Max \textbar1, 4,000,000 \textbar1, 3,000,000 \textbar1,
3,500,000 \textbar1,000, 7,500,000 \textbar{} \textbar Q1, Q3
\textbar55,000, 150,000 \textbar57,000, 150,000 \textbar58,000, 160,000
\textbar59,000, 165,000 \textbar{} \textbar Unknown \textbar2,039
\textbar10,255 \textbar14,527 \textbar17,669 \textbar{}
\textbar{}\textbf{Kessler Latent Anxiety} \textbar NA \textbar NA
\textbar NA \textbar NA \textbar{} \textbar Mean (SD) \textbar1.20
(0.76) \textbar1.17 (0.76) \textbar1.19 (0.77) \textbar1.17 (0.77)
\textbar{} \textbar Min, Max \textbar0.00, 4.00 \textbar0.00, 4.00
\textbar0.00, 4.00 \textbar0.00, 4.00 \textbar{} \textbar Q1, Q3
\textbar0.67, 1.67 \textbar0.67, 1.67 \textbar0.67, 1.67 \textbar0.67,
1.67 \textbar{} \textbar Unknown \textbar344 \textbar9,591
\textbar13,814 \textbar17,257 \textbar{} \textbar{}\textbf{Kessler
Latent Depression} \textbar NA \textbar NA \textbar NA \textbar NA
\textbar{} \textbar Mean (SD) \textbar0.60 (0.75) \textbar0.55 (0.73)
\textbar0.57 (0.74) \textbar0.54 (0.72) \textbar{} \textbar Min, Max
\textbar0.00, 4.00 \textbar0.00, 4.00 \textbar0.00, 4.00 \textbar0.00,
4.00 \textbar{} \textbar Q1, Q3 \textbar0.00, 1.00 \textbar0.00, 1.00
\textbar0.00, 1.00 \textbar0.00, 1.00 \textbar{} \textbar Unknown
\textbar340 \textbar9,588 \textbar13,814 \textbar17,259 \textbar{}
\textbar{}\textbf{Male} \textbar15,228 (36\%) \textbar15,228 (36\%)
\textbar15,228 (36\%) \textbar15,228 (36\%) \textbar{}
\textbar{}\textbf{Neuroticism} \textbar NA \textbar NA \textbar NA
\textbar NA \textbar{} \textbar Mean (SD) \textbar3.50 (1.16)
\textbar3.46 (1.16) \textbar3.41 (1.18) \textbar3.36 (1.17) \textbar{}
\textbar Min, Max \textbar1.00, 7.00 \textbar1.00, 7.00 \textbar1.00,
7.00 \textbar1.00, 7.00 \textbar{} \textbar Q1, Q3 \textbar2.75, 4.25
\textbar2.50, 4.25 \textbar2.50, 4.25 \textbar2.50, 4.25 \textbar{}
\textbar Unknown \textbar315 \textbar9,580 \textbar13,822 \textbar17,261
\textbar{} \textbar{}\textbf{Nz Dep2018} \textbar NA \textbar NA
\textbar NA \textbar NA \textbar{} \textbar Mean (SD) \textbar4.75
(2.72) \textbar4.75 (2.73) \textbar4.75 (2.73) \textbar4.75 (2.73)
\textbar{} \textbar Min, Max \textbar1.00, 10.00 \textbar1.00, 10.00
\textbar1.00, 10.00 \textbar1.00, 10.00 \textbar{} \textbar Q1, Q3
\textbar2.00, 7.00 \textbar2.00, 7.00 \textbar2.00, 7.00 \textbar2.00,
7.00 \textbar{} \textbar Unknown \textbar328 \textbar541 \textbar957
\textbar927 \textbar{} \textbar{}\textbf{Nzsei 13 L} \textbar NA
\textbar NA \textbar NA \textbar NA \textbar{} \textbar Mean (SD)
\textbar56 (16) \textbar56 (16) \textbar56 (16) \textbar57 (15)
\textbar{} \textbar Min, Max \textbar10, 90 \textbar10, 90 \textbar10,
90 \textbar10, 90 \textbar{} \textbar Q1, Q3 \textbar44, 69 \textbar44,
70 \textbar45, 70 \textbar46, 70 \textbar{} \textbar Unknown \textbar338
\textbar3,236 \textbar4,294 \textbar6,107 \textbar{}
\textbar{}\textbf{Openness} \textbar NA \textbar NA \textbar NA
\textbar NA \textbar{} \textbar Mean (SD) \textbar5.01 (1.11)
\textbar5.01 (1.11) \textbar5.02 (1.14) \textbar5.01 (1.15) \textbar{}
\textbar Min, Max \textbar1.00, 7.00 \textbar1.00, 7.00 \textbar1.00,
7.00 \textbar1.00, 7.00 \textbar{} \textbar Q1, Q3 \textbar4.25, 5.75
\textbar4.25, 5.75 \textbar4.25, 6.00 \textbar4.25, 6.00 \textbar{}
\textbar Unknown \textbar316 \textbar9,579 \textbar13,830 \textbar17,265
\textbar{} \textbar{}\textbf{Parent} \textbar31,213 (73\%)
\textbar24,889 (75\%) \textbar21,693 (75\%) \textbar19,597 (77\%)
\textbar{} \textbar Unknown \textbar20 \textbar9,363 \textbar13,752
\textbar17,166 \textbar{} \textbar{}\textbf{Partner} \textbar31,353
(75\%) \textbar24,644 (75\%) \textbar21,377 (75\%) \textbar18,499 (75\%)
\textbar{} \textbar Unknown \textbar786 \textbar9,844 \textbar14,248
\textbar17,868 \textbar{} \textbar{}\textbf{Political Conservative}
\textbar NA \textbar NA \textbar NA \textbar NA \textbar{} \textbar1
\textbar2,703 (6.5\%) \textbar2,289 (7.1\%) \textbar1,654 (6.0\%)
\textbar1,431 (5.9\%) \textbar{} \textbar2 \textbar8,734 (21\%)
\textbar7,118 (22\%) \textbar6,012 (22\%) \textbar5,188 (21\%)
\textbar{} \textbar3 \textbar8,493 (21\%) \textbar6,944 (22\%)
\textbar5,978 (22\%) \textbar4,972 (20\%) \textbar{} \textbar4
\textbar11,838 (29\%) \textbar9,424 (29\%) \textbar8,268 (30\%)
\textbar6,936 (29\%) \textbar{} \textbar5 \textbar5,942 (14\%)
\textbar4,110 (13\%) \textbar3,901 (14\%) \textbar3,577 (15\%)
\textbar{} \textbar6 \textbar2,980 (7.2\%) \textbar1,788 (5.6\%)
\textbar1,639 (5.9\%) \textbar1,853 (7.6\%) \textbar{} \textbar7
\textbar677 (1.6\%) \textbar405 (1.3\%) \textbar335 (1.2\%) \textbar365
(1.5\%) \textbar{} \textbar Unknown \textbar1,314 \textbar10,603
\textbar14,894 \textbar18,359 \textbar{} \textbar{}\textbf{Religion
Identification Level} \textbar NA \textbar NA \textbar NA \textbar NA
\textbar{} \textbar1 \textbar29,148 (69\%) \textbar22,794 (69\%)
\textbar20,119 (70\%) \textbar17,134 (69\%) \textbar{} \textbar2
\textbar1,329 (3.2\%) \textbar919 (2.8\%) \textbar765 (2.7\%)
\textbar812 (3.3\%) \textbar{} \textbar3 \textbar860 (2.0\%) \textbar801
(2.4\%) \textbar670 (2.3\%) \textbar558 (2.3\%) \textbar{} \textbar4
\textbar1,967 (4.7\%) \textbar1,562 (4.7\%) \textbar1,322 (4.6\%)
\textbar1,165 (4.7\%) \textbar{} \textbar5 \textbar2,312 (5.5\%)
\textbar1,882 (5.7\%) \textbar1,656 (5.8\%) \textbar1,270 (5.1\%)
\textbar{} \textbar6 \textbar2,040 (4.8\%) \textbar1,707 (5.2\%)
\textbar1,415 (4.9\%) \textbar1,241 (5.0\%) \textbar{} \textbar7
\textbar4,419 (11\%) \textbar3,258 (9.9\%) \textbar2,764 (9.6\%)
\textbar2,546 (10\%) \textbar{} \textbar Unknown \textbar606
\textbar9,758 \textbar13,970 \textbar17,955 \textbar{}
\textbar{}\textbf{Rural Gch 2018 Levels} \textbar NA \textbar NA
\textbar NA \textbar NA \textbar{} \textbar1 \textbar25,864 (61\%)
\textbar25,578 (61\%) \textbar25,088 (60\%) \textbar24,950 (60\%)
\textbar{} \textbar2 \textbar8,164 (19\%) \textbar8,195 (19\%)
\textbar8,185 (20\%) \textbar8,220 (20\%) \textbar{} \textbar3
\textbar5,362 (13\%) \textbar5,439 (13\%) \textbar5,470 (13\%)
\textbar5,529 (13\%) \textbar{} \textbar4 \textbar2,461 (5.8\%)
\textbar2,475 (5.9\%) \textbar2,495 (6.0\%) \textbar2,551 (6.1\%)
\textbar{} \textbar5 \textbar506 (1.2\%) \textbar507 (1.2\%) \textbar488
(1.2\%) \textbar504 (1.2\%) \textbar{} \textbar Unknown \textbar324
\textbar487 \textbar955 \textbar927 \textbar{} \textbar{}\textbf{Right
Wing Authoritarianism} \textbar NA \textbar NA \textbar NA \textbar NA
\textbar{} \textbar Mean (SD) \textbar3.17 (1.14) \textbar3.25 (1.11)
\textbar3.33 (1.06) \textbar3.30 (1.07) \textbar{} \textbar Min, Max
\textbar1.00, 7.00 \textbar1.00, 7.00 \textbar1.00, 7.00 \textbar1.00,
7.00 \textbar{} \textbar Q1, Q3 \textbar2.33, 4.00 \textbar2.50, 4.00
\textbar2.50, 4.00 \textbar2.50, 4.00 \textbar{} \textbar Unknown
\textbar46 \textbar9,423 \textbar13,803 \textbar17,220 \textbar{}
\textbar{}\textbf{Social Dominance Orientation} \textbar NA \textbar NA
\textbar NA \textbar NA \textbar{} \textbar Mean (SD) \textbar2.22
(0.96) \textbar2.18 (0.95) \textbar2.20 (0.95) \textbar2.23 (0.96)
\textbar{} \textbar Min, Max \textbar1.00, 7.00 \textbar1.00, 7.00
\textbar1.00, 7.00 \textbar1.00, 7.00 \textbar{} \textbar Q1, Q3
\textbar1.50, 2.83 \textbar1.33, 2.83 \textbar1.50, 2.83 \textbar1.50,
2.83 \textbar{} \textbar Unknown \textbar16 \textbar9,383 \textbar13,647
\textbar17,170 \textbar{} \textbar{}\textbf{Social Support (perceived)}
\textbar NA \textbar NA \textbar NA \textbar NA \textbar{} \textbar Mean
(SD) \textbar5.93 (1.15) \textbar5.93 (1.14) \textbar5.93 (1.16)
\textbar5.96 (1.15) \textbar{} \textbar Min, Max \textbar1.00, 7.00
\textbar1.00, 7.00 \textbar1.00, 7.00 \textbar1.00, 7.00 \textbar{}
\textbar Q1, Q3 \textbar5.33, 7.00 \textbar5.33, 7.00 \textbar5.33, 7.00
\textbar5.33, 7.00 \textbar{} \textbar Unknown \textbar28 \textbar9,397
\textbar13,772 \textbar17,268 \textbar{}
\textbar:---------------------------------\textbar:--------------------\textbar:--------------------\textbar:--------------------\textbar:--------------------\textbar{}

}

\end{table}%

\newpage{}

\subsection{Appendix B}\label{appendix-b}

\newpage{}

\begin{table}

\caption{\label{tbl-trustscience}Trust in scientists (sample)}

\centering{

\textbar:----------------\textbar:------------------\textbar:------------------\textbar:------------------\textbar:------------------\textbar{}
\textbar{} Outcomes \textbar{} 2019 (N = 42,681) \textbar{} 2020 (N =
42,681) \textbar{} 2021 (N = 42,681) \textbar{} 2022 (N = 42,681)
\textbar{}
\textbar:----------------\textbar:------------------\textbar:------------------\textbar:------------------\textbar:------------------\textbar{}
\textbar{} \textbf{Trust Science} \textbar{} \textbar{} \textbar{}
\textbar{} \textbar{} \textbar{} 1 \textbar{} 547 (1.3\%) \textbar{} 384
(1.2\%) \textbar{} 301 (1.1\%) \textbar{} 304 (1.2\%) \textbar{}
\textbar{} 2 \textbar{} 909 (2.2\%) \textbar{} 708 (2.1\%) \textbar{}
599 (2.2\%) \textbar{} 471 (1.9\%) \textbar{} \textbar{} 3 \textbar{}
1,978 (4.7\%) \textbar{} 1,373 (4.2\%) \textbar{} 1,112 (4.0\%)
\textbar{} 965 (3.9\%) \textbar{} \textbar{} 4 \textbar{} 7,188 (17\%)
\textbar{} 4,269 (13\%) \textbar{} 3,229 (12\%) \textbar{} 3,039 (12\%)
\textbar{} \textbar{} 5 \textbar{} 6,022 (14\%) \textbar{} 3,586 (11\%)
\textbar{} 2,994 (11\%) \textbar{} 2,681 (11\%) \textbar{} \textbar{} 6
\textbar{} 11,095 (26\%) \textbar{} 8,206 (25\%) \textbar{} 7,138 (26\%)
\textbar{} 5,884 (24\%) \textbar{} \textbar{} 7 \textbar{} 14,379 (34\%)
\textbar{} 14,523 (44\%) \textbar{} 12,197 (44\%) \textbar{} 11,116
(45\%) \textbar{} \textbar{} Missing \textbar{} 563 \textbar{} 9,632
\textbar{} 15,111 \textbar{} 18,221 \textbar{} \textbar{} \textbf{Trust
Scientists} \textbar{} \textbar{} \textbar{} \textbar{} \textbar{}
\textbar{} 1 \textbar{} 1,113 (2.7\%) \textbar{} 661 (2.0\%) \textbar{}
660 (2.4\%) \textbar{} 459 (1.9\%) \textbar{} \textbar{} 2 \textbar{}
1,556 (3.8\%) \textbar{} 889 (2.7\%) \textbar{} 791 (2.9\%) \textbar{}
584 (2.5\%) \textbar{} \textbar{} 3 \textbar{} 2,128 (5.1\%) \textbar{}
1,268 (3.9\%) \textbar{} 1,026 (3.8\%) \textbar{} 867 (3.7\%) \textbar{}
\textbar{} 4 \textbar{} 5,996 (14\%) \textbar{} 3,615 (11\%) \textbar{}
2,823 (10\%) \textbar{} 2,602 (11\%) \textbar{} \textbar{} 5 \textbar{}
8,165 (20\%) \textbar{} 5,770 (18\%) \textbar{} 4,625 (17\%) \textbar{}
4,505 (19\%) \textbar{} \textbar{} 6 \textbar{} 13,210 (32\%) \textbar{}
11,380 (35\%) \textbar{} 9,496 (35\%) \textbar{} 8,407 (35\%) \textbar{}
\textbar{} 7 \textbar{} 9,256 (22\%) \textbar{} 8,926 (27\%) \textbar{}
7,741 (28\%) \textbar{} 6,309 (27\%) \textbar{}
\textbar:----------------\textbar:------------------\textbar:------------------\textbar:------------------\textbar:------------------\textbar{}

}

\end{table}%

Table~\ref{tbl-trustscience} presents sample change in trust in science
over time.

\newpage{}

\subsection*{References}\label{references}
\addcontentsline{toc}{subsection}{References}

\phantomsection\label{refs}
\begin{CSLReferences}{1}{0}
\bibitem[\citeproctext]{ref-tinytable_2024}
Arel-Bundock, V (2024) \emph{Tinytable: Simple and configurable tables
in 'HTML', 'LaTeX', 'markdown', 'word', 'PNG', 'PDF', and 'typst'
formats}. Retrieved from
\url{https://CRAN.R-project.org/package=tinytable}

\bibitem[\citeproctext]{ref-blackwell_2017_unified}
Blackwell, M, Honaker, J, and King, G (2017) A unified approach to
measurement error and missing data: Overview and applications.
\emph{Sociological Methods \& Research}, \textbf{46}(3), 303--341.

\bibitem[\citeproctext]{ref-margot2024}
Bulbulia, JA (2024) \emph{Margot: MARGinal observational
treatment-effects}.
doi:\href{https://doi.org/10.5281/zenodo.10907724}{10.5281/zenodo.10907724}.

\bibitem[\citeproctext]{ref-bulbulia2023a}
Bulbulia, JA, Afzali, MU, Yogeeswaran, K, and Sibley, CG (2023)
Long-term causal effects of far-right terrorism in {N}ew {Z}ealand.
\emph{PNAS Nexus}, \textbf{2}(8), pgad242.

\bibitem[\citeproctext]{ref-fraser_coding_2020}
Fraser, G, Bulbulia, J, Greaves, LM, Wilson, MS, and Sibley, CG (2020)
Coding responses to an open-ended gender measure in a {N}ew {Z}ealand
national sample. \emph{The Journal of Sex Research}, \textbf{57}(8),
979--986.
doi:\href{https://doi.org/10.1080/00224499.2019.1687640}{10.1080/00224499.2019.1687640}.

\bibitem[\citeproctext]{ref-geepack_2006}
Halekoh, U, Højsgaard, S, and Yan, J (2006) The {R} package geepack for
generalized estimating equations. \emph{Journal of Statistical
Software}, \textbf{15/2}, 1--11. Retrieved from
\url{https://www.jstatsoft.org/v15/i02/}

\bibitem[\citeproctext]{ref-hartman2017}
Hartman, RO, Dieckmann, NF, Sprenger, AM, Stastny, BJ, and DeMarree, KG
(2017) Modeling attitudes toward science: Development and validation of
the credibility of science scale. \emph{Basic and Applied Social
Psychology}, \textbf{39}, 358--371.
doi:\href{https://doi.org/10.1080/01973533.2017.1372284}{10.1080/01973533.2017.1372284}.

\bibitem[\citeproctext]{ref-amelia_2011}
Honaker, J, King, G, and Blackwell, M (2011) {Amelia II}: A program for
missing data. \emph{Journal of Statistical Software}, \textbf{45}(7),
1--47.

\bibitem[\citeproctext]{ref-jost_end_2006-1}
Jost, JT (2006) The end of the end of ideology. \emph{American
Psychologist}, \textbf{61}(7), 651--670.
doi:\href{https://doi.org/10.1037/0003-066X.61.7.651}{10.1037/0003-066X.61.7.651}.

\bibitem[\citeproctext]{ref-ggeffects_2018}
Ludecke, D (2018) Ggeffects: Tidy data frames of marginal effects from
regression models. \emph{Journal of Open Source Software},
\textbf{3}(26), 772.
doi:\href{https://doi.org/10.21105/joss.00772}{10.21105/joss.00772}.

\bibitem[\citeproctext]{ref-nisbet2015}
Nisbet, EC, Cooper, KE, and Garrett, RK (2015) The partisan brain: How
dissonant science messages lead conservatives and liberals to (dis)trust
science. \emph{The ANNALS of the American Academy of Political and
Social Science}, \textbf{658}(1), 36--66.
doi:\href{https://doi.org/10.1177/0002716214555474}{10.1177/0002716214555474}.

\bibitem[\citeproctext]{ref-sibley2021}
Sibley, CG (2021)
\emph{\href{https://doi.org/10.31234/osf.io/wgqvy}{Sampling procedure
and sample details for the {N}ew {Z}ealand {A}ttitudes and {V}alues
{S}tudy}}.

\bibitem[\citeproctext]{ref-sibley2020a}
Sibley, CG, Greaves, L, Satherley, N, \ldots{} al, et (2020) What
happened to people in {N}ew {Z}ealand during covid-19 home lockdown?
Institutional trust, attitudes to government, mental health and
subjective wellbeing. Retrieved from
\href{https://osf.io/e765a}{osf.io/e765a}

\bibitem[\citeproctext]{ref-nnet_2002}
Venables, WN, and Ripley, BD (2002) \emph{Modern applied statistics with
s}, Fourth, New York: Springer. Retrieved from
\url{https://www.stats.ox.ac.uk/pub/MASS4/}

\bibitem[\citeproctext]{ref-ggplot2_2016}
Wickham, H (2016) \emph{ggplot2: Elegant graphics for data analysis},
Springer-Verlag New York. Retrieved from
\url{https://ggplot2.tidyverse.org}

\end{CSLReferences}




\end{document}
