% Options for packages loaded elsewhere
\PassOptionsToPackage{unicode}{hyperref}
\PassOptionsToPackage{hyphens}{url}
\PassOptionsToPackage{dvipsnames,svgnames,x11names}{xcolor}
%
\documentclass[
  single column]{article}

\usepackage{amsmath,amssymb}
\usepackage{iftex}
\ifPDFTeX
  \usepackage[T1]{fontenc}
  \usepackage[utf8]{inputenc}
  \usepackage{textcomp} % provide euro and other symbols
\else % if luatex or xetex
  \usepackage{unicode-math}
  \defaultfontfeatures{Scale=MatchLowercase}
  \defaultfontfeatures[\rmfamily]{Ligatures=TeX,Scale=1}
\fi
\usepackage[]{libertinus}
\ifPDFTeX\else  
    % xetex/luatex font selection
\fi
% Use upquote if available, for straight quotes in verbatim environments
\IfFileExists{upquote.sty}{\usepackage{upquote}}{}
\IfFileExists{microtype.sty}{% use microtype if available
  \usepackage[]{microtype}
  \UseMicrotypeSet[protrusion]{basicmath} % disable protrusion for tt fonts
}{}
\makeatletter
\@ifundefined{KOMAClassName}{% if non-KOMA class
  \IfFileExists{parskip.sty}{%
    \usepackage{parskip}
  }{% else
    \setlength{\parindent}{0pt}
    \setlength{\parskip}{6pt plus 2pt minus 1pt}}
}{% if KOMA class
  \KOMAoptions{parskip=half}}
\makeatother
\usepackage{xcolor}
\usepackage[top=30mm,left=25mm,heightrounded,headsep=22pt,headheight=11pt,footskip=33pt,ignorehead,ignorefoot]{geometry}
\setlength{\emergencystretch}{3em} % prevent overfull lines
\setcounter{secnumdepth}{-\maxdimen} % remove section numbering
% Make \paragraph and \subparagraph free-standing
\makeatletter
\ifx\paragraph\undefined\else
  \let\oldparagraph\paragraph
  \renewcommand{\paragraph}{
    \@ifstar
      \xxxParagraphStar
      \xxxParagraphNoStar
  }
  \newcommand{\xxxParagraphStar}[1]{\oldparagraph*{#1}\mbox{}}
  \newcommand{\xxxParagraphNoStar}[1]{\oldparagraph{#1}\mbox{}}
\fi
\ifx\subparagraph\undefined\else
  \let\oldsubparagraph\subparagraph
  \renewcommand{\subparagraph}{
    \@ifstar
      \xxxSubParagraphStar
      \xxxSubParagraphNoStar
  }
  \newcommand{\xxxSubParagraphStar}[1]{\oldsubparagraph*{#1}\mbox{}}
  \newcommand{\xxxSubParagraphNoStar}[1]{\oldsubparagraph{#1}\mbox{}}
\fi
\makeatother


\providecommand{\tightlist}{%
  \setlength{\itemsep}{0pt}\setlength{\parskip}{0pt}}\usepackage{longtable,booktabs,array}
\usepackage{calc} % for calculating minipage widths
% Correct order of tables after \paragraph or \subparagraph
\usepackage{etoolbox}
\makeatletter
\patchcmd\longtable{\par}{\if@noskipsec\mbox{}\fi\par}{}{}
\makeatother
% Allow footnotes in longtable head/foot
\IfFileExists{footnotehyper.sty}{\usepackage{footnotehyper}}{\usepackage{footnote}}
\makesavenoteenv{longtable}
\usepackage{graphicx}
\makeatletter
\newsavebox\pandoc@box
\newcommand*\pandocbounded[1]{% scales image to fit in text height/width
  \sbox\pandoc@box{#1}%
  \Gscale@div\@tempa{\textheight}{\dimexpr\ht\pandoc@box+\dp\pandoc@box\relax}%
  \Gscale@div\@tempb{\linewidth}{\wd\pandoc@box}%
  \ifdim\@tempb\p@<\@tempa\p@\let\@tempa\@tempb\fi% select the smaller of both
  \ifdim\@tempa\p@<\p@\scalebox{\@tempa}{\usebox\pandoc@box}%
  \else\usebox{\pandoc@box}%
  \fi%
}
% Set default figure placement to htbp
\def\fps@figure{htbp}
\makeatother
% definitions for citeproc citations
\NewDocumentCommand\citeproctext{}{}
\NewDocumentCommand\citeproc{mm}{%
  \begingroup\def\citeproctext{#2}\cite{#1}\endgroup}
\makeatletter
 % allow citations to break across lines
 \let\@cite@ofmt\@firstofone
 % avoid brackets around text for \cite:
 \def\@biblabel#1{}
 \def\@cite#1#2{{#1\if@tempswa , #2\fi}}
\makeatother
\newlength{\cslhangindent}
\setlength{\cslhangindent}{1.5em}
\newlength{\csllabelwidth}
\setlength{\csllabelwidth}{3em}
\newenvironment{CSLReferences}[2] % #1 hanging-indent, #2 entry-spacing
 {\begin{list}{}{%
  \setlength{\itemindent}{0pt}
  \setlength{\leftmargin}{0pt}
  \setlength{\parsep}{0pt}
  % turn on hanging indent if param 1 is 1
  \ifodd #1
   \setlength{\leftmargin}{\cslhangindent}
   \setlength{\itemindent}{-1\cslhangindent}
  \fi
  % set entry spacing
  \setlength{\itemsep}{#2\baselineskip}}}
 {\end{list}}
\usepackage{calc}
\newcommand{\CSLBlock}[1]{\hfill\break\parbox[t]{\linewidth}{\strut\ignorespaces#1\strut}}
\newcommand{\CSLLeftMargin}[1]{\parbox[t]{\csllabelwidth}{\strut#1\strut}}
\newcommand{\CSLRightInline}[1]{\parbox[t]{\linewidth - \csllabelwidth}{\strut#1\strut}}
\newcommand{\CSLIndent}[1]{\hspace{\cslhangindent}#1}

\usepackage{booktabs}
\usepackage{longtable}
\usepackage{array}
\usepackage{multirow}
\usepackage{wrapfig}
\usepackage{float}
\usepackage{colortbl}
\usepackage{pdflscape}
\usepackage{tabu}
\usepackage{threeparttable}
\usepackage{threeparttablex}
\usepackage[normalem]{ulem}
\usepackage{makecell}
\usepackage{xcolor}
\input{/Users/joseph/GIT/latex/latex-for-quarto.tex}
\makeatletter
\@ifpackageloaded{caption}{}{\usepackage{caption}}
\AtBeginDocument{%
\ifdefined\contentsname
  \renewcommand*\contentsname{Table of contents}
\else
  \newcommand\contentsname{Table of contents}
\fi
\ifdefined\listfigurename
  \renewcommand*\listfigurename{List of Figures}
\else
  \newcommand\listfigurename{List of Figures}
\fi
\ifdefined\listtablename
  \renewcommand*\listtablename{List of Tables}
\else
  \newcommand\listtablename{List of Tables}
\fi
\ifdefined\figurename
  \renewcommand*\figurename{Figure}
\else
  \newcommand\figurename{Figure}
\fi
\ifdefined\tablename
  \renewcommand*\tablename{Table}
\else
  \newcommand\tablename{Table}
\fi
}
\@ifpackageloaded{float}{}{\usepackage{float}}
\floatstyle{ruled}
\@ifundefined{c@chapter}{\newfloat{codelisting}{h}{lop}}{\newfloat{codelisting}{h}{lop}[chapter]}
\floatname{codelisting}{Listing}
\newcommand*\listoflistings{\listof{codelisting}{List of Listings}}
\makeatother
\makeatletter
\makeatother
\makeatletter
\@ifpackageloaded{caption}{}{\usepackage{caption}}
\@ifpackageloaded{subcaption}{}{\usepackage{subcaption}}
\makeatother

\usepackage{bookmark}

\IfFileExists{xurl.sty}{\usepackage{xurl}}{} % add URL line breaks if available
\urlstyle{same} % disable monospaced font for URLs
\hypersetup{
  pdftitle={Causal effect of Psychopathy on Partner Well-Being: A Six-Year National Longitudinal Study},
  pdfauthor={Aaron Hissey; Hedwig Eisenbarth; Matthew Hammond; Chris G. Sibley; Joseph A. Bulbulia},
  pdfkeywords={Causal
Inference, Relationships, Panel, Psychopathy, Personality, Well-being, Outcome-wide},
  colorlinks=true,
  linkcolor={blue},
  filecolor={Maroon},
  citecolor={Blue},
  urlcolor={Blue},
  pdfcreator={LaTeX via pandoc}}


\title{Causal effect of Psychopathy on Partner Well-Being: A Six-Year
National Longitudinal Study}

\usepackage{academicons}
\usepackage{xcolor}

  \author{Aaron Hissey}
            \affil{%
             \small{     Victoria University of Wellington, New Zealand
          ORCID \textcolor[HTML]{A6CE39}{\aiOrcid} ~ }
              }
      \usepackage{academicons}
\usepackage{xcolor}

  \author{Hedwig Eisenbarth}
            \affil{%
             \small{     Victoria University of Wellington, New Zealand
          ORCID \textcolor[HTML]{A6CE39}{\aiOrcid} ~ }
              }
      \usepackage{academicons}
\usepackage{xcolor}

  \author{Matthew Hammond}
            \affil{%
             \small{     School of Psychology, University of Auckland
          ORCID \textcolor[HTML]{A6CE39}{\aiOrcid} ~ }
              }
      \usepackage{academicons}
\usepackage{xcolor}

  \author{Chris G. Sibley}
            \affil{%
             \small{     School of Psychology, University of Auckland
          ORCID \textcolor[HTML]{A6CE39}{\aiOrcid} ~ }
              }
      \usepackage{academicons}
\usepackage{xcolor}

  \author{Joseph A. Bulbulia}
            \affil{%
             \small{     Victoria University of Wellington, New Zealand
          ORCID \textcolor[HTML]{A6CE39}{\aiOrcid} ~ }
              }
      


\date{2024-12-05}
\begin{document}
\maketitle


\subsection{Introduction}\label{introduction}

\subsubsection{Causal Questions}\label{causal-questions}

A central question in psychopathy research is how it affects
relationships. To address this question, we report a series of
outcome-wide longitudinal studies drawing on a national panel study of
New Zealanders (Studies 1a and 1b: 1012; Studies 2a and 2b: 1832.)

In Study 1, we conduct a three-year investigation, considering the
effect of a one-year intervention on psychopathic personality and its
facets on partner well-being, with control for both self and partner
co-variates in the baseline wave, measuring partner well-being obtained
in the third wave, one-year after the intervention (Planned OSF study).
Included in these baseline controls are baseline indicators of all
exposure variables and all outcomes variables for both individuals and
their partners. (refer to VanderWeele \emph{et al.}
(\citeproc{ref-vanderweele2020}{2020})). \textbf{When assessing
psychopathy, rather than creating a summed score, we compare a setting
in which each of the four psychopathy traits is simultaneously shifted
up or down by one point, contrasted with a setting in which there is no
intervention, that is, in which only the observed exposures and outcomes
are modelled.}

In Study 2, we conduct a six-year investigation, this time shifting
personality up or down for four waves after robust confounding control
for baseline covariates that include baseline measurements of all
exposures and outcomes measured on all partners, and measuring outcomes
on the partners of individuals whose personality is adjusted in the
(sixth) and final wave. \textbf{Again, to assess psychopathy, we compare
a setting in which each of the four psychopathy traits is simultaneously
shifted up or down by one point, contrasted with a setting in which
there is no intervention, that is, in which only the observed exposures
and outcomes are modelled.} These finding examine the effects of
sustained exposures to personality shifts over four years, controlling
for baseline measures in the first year, and assessing well-being one
year after the final intervention. Results generalise to the population
of New Zealand couples sampled in our national longitudional study..

The shift interventions that we consider require projecting the
population up or down from the natural value of the personality
indicators that they report. An advantage of causal effects we estimate
under such interventions is that they are plausible. We can imagine what
it would be like to rate our personality trait one point higher or lower
on a 1-7 Likert Scale. Contrasts are taken from these shifts from
observed levels, thus allowing us to ask how people would respond to the
different intervention as contrasted with how they were observed to
respond without any interventions (\citeproc{ref-diaz2023lmtp}{Díaz
\emph{et al.} 2023}; \citeproc{ref-young2014identification}{Young
\emph{et al.} 2014}). The causal contrasts we consider are also called
`modified treatment policies' (\citeproc{ref-diaz2013assessing}{Dı́az and
Laan 2013}; \citeproc{ref-vanderlain2014discussion}{Laan \emph{et al.}
2014}).

\subsubsection{Motivation}\label{motivation}

The questions we examine hold relevance across several subfields of
psychological science, most apparently forensic psychology, and
relationship research, and also to personality and social psychology
more generally. This study is also broadly significant for advancing
causal questions within observational psychology more generally, where
presently there have been relatively few published applications of
robust workflows for causal inference
(\citeproc{ref-chatton2024causal}{Chatton and Rohrer 2024};
\citeproc{ref-rohrer2022PATH}{Rohrer \emph{et al.} 2022}).

Currently, few studies in relationship research address causality
directly, tending instead to investigate statistical associations
between actors and partners over time. Establishing causality, however,
requires a clearly defined causal contrast of interest, well-defined
treatments, and random assignment of individuals drawn from a population
of interest to treatment conditions. Because we cannot observe an
individual's response to an unassigned treatment, individual causal
effects remain unobservable. However, with randomisation, investigators
can infer population-level responses asymptotically, as individuals in
treatment conditions are exchangeable. These effects are always obtained
for populations or strata of populations, never for individuals. In
relationship research, where the effects of interest are on partners of
individuals, individual partner effects also remain elusive. Even when
repeated measures designs are employed, causal effect estimates in
experimental research are always ``between-treatments'' -- whatever the
reporting habits of experimental researchers.

Notably, for many theoretically relevant questions, randomised
controlled trials are neither feasible nor ethical. In observational
research, identifying causal effects requires assumptions usually
inherent to experiments and a systematic approach in which investigators
specify (1) a target population; (2) a causal question as a contrast of
population averages under distinct treatments; (3) identification
assumptions ensuring assignment to treatment conditions is ``as good as
random''; (4) a statistical estimator for causal quantities; (5) a
statistical model with its assumptions; and (6) sensitivity analyses to
evaluate results (\citeproc{ref-ogburn2021}{Ogburn and Shpitser 2021}).
However in observational research, it is useful to always keep in model
the ideal experiment or ``target trial'' that we investigators would
conduct where such an experiment feasible and ethical
(\citeproc{ref-bulbulia2022}{Bulbulia 2023};
\citeproc{ref-hernan2024WHATIF}{Hernan and Robins 2024};
\citeproc{ref-hernan2008aObservationalStudiesAnalysedLike}{Hernán
\emph{et al.} 2008};
\citeproc{ref-hernan2016_specifying_a_target_trial}{Hernán \emph{et al.}
2016a}).

To begin, when emulating a ``target trial'' observational investigators
investigators must ensure that outcomes they measure do not precede the
treatments being contrasted. In experimental designs, it is clear that
events that occur before manipulation cannot be reported as outcomes,
but this assurance is typically absent in observational research, where
repeated measures over time are usually necessary to ensure the
necessary chronological hygiene. Temporal resolution alone, however, is
insufficient for causal inference. Investigators must define a clear
pair, or set, of treatments (or `exposures') for comparison. In
randomised experiments, treatments and any control conditions define the
contrasted conditions. In observational studies, where exposures are not
administered by the researcher, the exposures and contrasts must be
explicitly specified. Robust causal inference requires that the
exposures under comparison are observed across all levels of covariates
in the statistical model (the positiviety assumption), that the
treatments although not under control of the investigators are
comparable (causal consistency), and that treatment assignment although
not under the control of investigators is ``as good as random''
(conditional exchangeability). Moreover, to obtain incident effects,
investigators must ensure that only initiation into exposures are
counted (\citeproc{ref-hernan2016}{Hernán \emph{et al.} 2016b}). For
instance, examining a harmful exposure without accounting for initial
susceptibility could introduce `healthy-worker bias,' misleadingly
suggesting a positive association between the exposure and outcome when
the true causal relationship is harmful
(\citeproc{ref-robins1986}{Robins 1986}). Despite advanced statistical
modelling and time series data, relationship research has yet to
systematically adopt rigorous causal inference methods
(\citeproc{ref-bulbulia2024swigstime}{Bulbulia 2024c};
\citeproc{ref-rohrer2023withinbetween}{Rohrer and Murayama 2023}). Here,
we seek to address these gaps.

\subsection{Method}\label{method}

\subsubsection{Sample}\label{sample}

Data were collected as part of The New Zealand Attitudes and Values
Study (NZAVS) is an annual longitudinal national probability panel study
of social attitudes, personality, ideology and health outcomes. The
NZAVS began in 2009. It includes questionnaire responses from over
72,000 New Zealand residents. The study includes researchers from many
New Zealand universities, including the University of Auckland, Victoria
University of Wellington, the University of Canterbury, the University
of Otago, and Waikato University. Because the survey asks the same
people to respond each year, it can track subtle changes in attitudes
and values over time and is an important resource for researchers in New
Zealand and worldwide. The NZAVS is university-based, not-for-profit and
independent of political or corporate
funding.https://doi.org/10.17605/OSF.IO/75SNB

\hyperref[appendix-demographics]{Appendix B} and
\hyperref[appendix-demographics-long]{Appendix C} provides information
about demographic covariates used for confounding control (NZAVS time
10, years 2018-2019).

\subsection{Target Population}\label{target-population}

In Studies 1a and 1b, the target population consisted of all New
Zealanders who were in a relationship at NZAVS time 10 (baseline, years
2018-2019) and remained in a relationship at NZAVS time 11 (years
2019-2020), with a per-protocol affect estimated for attrition using
inverse probability of censoring weighting for those lost to follow up
in the final wave (\citeproc{ref-hernan2024WHATIF}{Hernan and Robins
2024}).

In Studies 2a and 2b, the target population consisted of all New
Zealanders who were in a relationship at NZAVS time 10 (baseline, years
2018-2019), and remained in a relationship until NZAVS time 13 (years
2021-2022), with a per-protocol affect estimated for attrition using
inverse probability of censoring weighting for those lost to follow up
in the final wave (\citeproc{ref-hernan2024WHATIF}{Hernan and Robins
2024}). Additionally those who broke up at any time after the baseline
wave were counted as censored.

\subsubsection{Eligibility criteria for Studies 1a and 1b (short-term
interventions)}\label{eligibility-criteria-for-studies-1a-and-1b-short-term-interventions}

The sample consisted of respondents to NZAVS times 10 (baseline, years
2018-2019), time 11 (treatment wave, years 2019-2020), and time 12
(outcome wave) (years 2020-2021).

\begin{itemize}
\tightlist
\item
  Participants who were identified as being in couples in both the
  baseline and exposure waves, where both partners were observed.\\
\item
  Participants who provided full information to the psychopathy measures
  in the 2018 and 2019. Participants may have been lost to follow up in
  the outcome wave 2020.
\item
  Missing data for all variables in the baseline wave baseline were
  allowed. Missing data at baseline were imputed through the
  \texttt{mice} package (\citeproc{ref-vanbuuren2018}{Van Buuren 2018}),
  and an extra column indicator was included to denote whether a
  variable was missing, following protocols described by
  (\citeproc{ref-williams2021}{Williams and Díaz 2021}).
\item
  Inverse probability of censoring weights were calculated as part of
  estimation in \texttt{lmtp} to adjust for missing outcomes at NZAVS
  Time 12 (years 2020-2021, the outcome wave
  (\citeproc{ref-williams2021}{Williams and Díaz 2021}))
\end{itemize}

There were 1012 NZAVS participants who met these criteria in 506
couples.

\subsubsection{Eligibility criteria for Studies 2a and 2b (longer-term
interventions)}\label{eligibility-criteria-for-studies-2a-and-2b-longer-term-interventions}

The sample consisted of respondents to NZAVS times 10 (baseline, years
2018-2019), times 11-13 (treatment waves, years 2019-2022), and time 14
(outcome wave, years 2022-2023).

\begin{itemize}
\tightlist
\item
  Participants who were identified as being in couples in the baseline
  wave, where both partners were observed.
\item
  Participants may have been lost to follow up in the exposure waves and
  outcomes waves.
\item
  Missing data for all variables in the baseline wave were allowed.
  Missing data at baseline were imputed through the \texttt{mice}
  package (\citeproc{ref-vanbuuren2018}{Van Buuren 2018}), and an extra
  column indicator was included to denote whether a variable was
  missing, following protocols described by
  (\citeproc{ref-williams2021}{Williams and Díaz 2021}).
\item
  Inverse probability of censoring weights were calculated as part of
  estimation in \texttt{lmtp} to adjust for missing outcomes at NZAVS
  Time 14 (years 2022-2023, the outcome wave
  (\citeproc{ref-williams2021}{Williams and Díaz 2021})).
\item
  If one or more partner dropped out of the study, both partners were
  censored.
\end{itemize}

There were 1832 NZAVS participants who met these criteria in 916
couples.

\subsubsection{Causal Contrasts in Romantic
Partnerships}\label{causal-contrasts-in-romantic-partnerships}

We specify causal effects non-parametrically by defining contrasts
between potential outcomes under different interventions. This approach
allows us to clearly state what would happen to both individuals and
their partners under specific changes to personality traits.

\paragraph{Notation}\label{notation}

\begin{itemize}
\tightlist
\item
  A romantic partnership is defined as a pair of individuals \((s, p)\),
  where:

  \begin{itemize}
  \tightlist
  \item
    \(s\) = self (the individual)
  \item
    \(p\) = partner
  \end{itemize}
\item
  \(A_s\) = intervention on the self's psychopathic personality or one
  of its facets
\item
  \(A_p\) = partner's psychopathic personality (not intervened upon)
\item
  \(Y_s\) = outcome for self (e.g., well-being)
\item
  \(Y_p\) = outcome for partner
\item
  \(L\) = measured covariates
\end{itemize}

\subsubsection{Potential Outcomes}\label{potential-outcomes}

For any intervention where we set the self's personality facet to level
\(\tilde{a}_s\):

\begin{itemize}
\tightlist
\item
  \(Y_p(\tilde{a}_s)\) denotes the potential outcome for the partner
  when the self is intervened upon.
\item
  \(Y_s(\tilde{a}_s)\) denotes the potential outcome for the self when
  the self is intervened upon.
\end{itemize}

Since we do not intervene on the partner's psychopathic personality, the
partner's exposure remains at its observed value \(\tilde{a}_p = A_p\).

\subsubsection{Assumptions}\label{assumptions}

To identify these potential outcomes, we assume conditional
exchangeability given measured covariates \(L\):

\[
\Big\{ Y_p(\tilde{a}_s), Y_s(\tilde{a}_s) \Big\} \coprod A_s \mid L
\]

This means that, conditional on \(L\), the treatment assignments
\(A_s = \tilde{a}_s\) are `as good as random.' This allows us to
estimate causal effects from observed data.

\subsection{Intervention Types}\label{intervention-types}

We define three specific shift interventions on the psychopathic
personality trait, which is bounded between 1 and 7:

\subsubsection{1. Shift Up (+1)}\label{shift-up-1}

We define the shift-up intervention as:

\[
a_s^+ = f(A_s) = 
\begin{cases}
A_s + 1, & \text{if } A_s + 1 \leq 7 \\
7, & \text{if } A_s + 1 > 7
\end{cases}
\]

This ensures that the shifted value does not exceed the maximum score of
7.

\subsubsection{2. Shift Down (-1)}\label{shift-down--1}

We define the shift-down intervention as:

\[
a_s^- = f(A_s) = 
\begin{cases}
A_s - 1, & \text{if } A_s - 1 \geq 1 \\
1, & \text{if } A_s - 1 < 1
\end{cases}
\]

This ensures that the shifted value does not go below the minimum score
of 1.

\subsubsection{3. Null (No Change)}\label{null-no-change}

The null intervention leaves the exposure at its observed value:

\[
a_s = A_s
\]

\subsubsection{Causal Effects}\label{causal-effects}

We specify causal effects by contrasting potential outcomes under these
interventions. In study 1 this intervention occurred for one year
(baseline + 1 year), and in study 2 it occurred for three years
(baseline + 3 years). Outcomes were measured one year after the end of
the intervention to ensure temporal order in causal inferences
(\citeproc{ref-vanderweele2020}{VanderWeele \emph{et al.} 2020}).

\paragraph{Average Partner Effect
(APE)}\label{average-partner-effect-ape}

The Average Partner Effect (APE) computes how changing the self's trait
affects the partner's outcome.

\subparagraph{For Shift Up:}\label{for-shift-up}

\[
\text{APE}_{\text{up}} = E\left[ Y_p(a_s^+) - Y_p(a_s) \mid L \right]
\]

This Average Partner Effect (APE) computes the average change in the
partner's outcome when we increase the self's trait by 1 point from the
observed treatment level..

\subparagraph{For Shift Down:}\label{for-shift-down}

\[
\text{APE}_{\text{down}} = E\left[ Y_p(a_s^-) - Y_p(a_s) \mid L \right]
\]

This Average Partner Effect (APE) computes the average change in the
partner's outcome when we decrease the self's trait by 1 point from the
observed treatment level.

\paragraph{Average Self Effect (ASE)}\label{average-self-effect-ase}

The Average Self Effect (ASE) measures how changing one's own trait
affects their own outcome.

\subparagraph{For Shift Up:}\label{for-shift-up-1}

\[
\text{ASE}_{\text{up}} = E\left[ Y_s(a_s^+) - Y_s(a_s) \mid L \right]
\]

This Average Self Effect computes the average change in the self's
outcome when we increase their trait by 1 point from the observed
treatment level.

\subparagraph{For Shift Down:}\label{for-shift-down-1}

\[
\text{ASE}_{\text{down}} = E\left[ Y_s(a_s^-) - Y_s(a_s) \mid L \right]
\]

This Average Self Effect computes the average change in the self's
outcome when we decrease their trait by 1 point from the observed
treatment level.

In Studies 1a and 1b, we estimate the Average Partner and Self per
protocol effects, respectively, from one-year shift interventions (both
increases and decreases) on psychopathic personality traits in a
population of New Zealanders who have been in relationships for at least
two years. The interventions occurred in the second year, with outcomes
measured in the third year. We allowed for loss to follow-up and
break-ups in the third year, estimating the per-protocol effect using
inverse probability of censoring weights.

In Studies 2a and 2b, we estimate the Average Partner and Self per
protocol effects from three-year shift interventions on psychopathic
personality traits among New Zealanders at least four year
relationships. The interventions took place from the second to the
fourth year, and outcomes were measured in the fifth year (the end of
study). Loss to follow-up was permitted after baseline. When break-ups
occurred, both partners were censored, and we estimated the per-protocol
effect using inverse probability of censoring weights.

Having stated our causal estimands and target population, we now turn to
our identification assumptions.

\subsubsection{Identification
Assumptions}\label{identification-assumptions}

To consistently estimate causal effects, we rely on three assumptions.
These assumptions are standard in the literature on causal inference
(\citeproc{ref-bulbulia2023}{Bulbulia 2024b};
\citeproc{ref-hernan2024WHATIF}{Hernan and Robins 2024}).

\begin{enumerate}
\def\labelenumi{\arabic{enumi}.}
\item
  \textbf{Causal Consistency:} potential outcomes must align with
  observed outcomes under the treatments in our data. Essentially, we
  assume potential outcomes do not depend on the specific way treatment
  was administered, as long as we consider measured covariates. Note
  that we adjust both for parter and self co-variates at baseline.
\item
  \textbf{Conditional Exchangeability:} given the observed covariates,
  we assume treatment assignment is independent of potential outcomes.
  In simpler terms, this means ``no unmeasured confounding'' -- any
  variables that influence both treatment assignment and outcomes are
  included in our covariate set.
\item
  \textbf{Positivity:} for unbiased estimation, every subject must have
  a non-zero chance of receiving the treatment, regardless of their
  covariate values. We evaluate this assumption in each study by
  examining changes in psychopathy ``treatments'' from baseline (NZAVS
  time 10) to the treatment wave (NZAVS time 11).
  \href{appendix-transition}{Appendix XXX}: Transition matrices shows
  this observed shifts in the treatment variables from baseline (NZAVS
  time 10) to exposure waves (NZAVS time 11).
\end{enumerate}

\paragraph{Estimation}\label{estimation}

In Study 1, we estimate the causal effect of modified treatment policies
over time using Targeted Minimum Loss-based Estimation (TMLE), a
semi-parametric estimator (\citeproc{ref-vanderlaan2011}{Van Der Laan
and Rose 2011}, \citeproc{ref-vanderlaan2018}{2018}). TMLE estimation
was conducted via the lmtp package (\citeproc{ref-duxedaz2021}{Díaz
\emph{et al.} 2021}; \citeproc{ref-hoffman2023}{Hoffman \emph{et al.}
2023}; \citeproc{ref-williams2021}{Williams and Díaz 2021}). TMLE
proceeds in two main steps involving outcome and treatment (exposure)
models. First, machine learning algorithms flexibly model the
relationships among treatments, covariates, and outcomes, allowing TMLE
to capture complex, high-dimensional covariate structures without strict
model assumptions. This step yields an initial set of estimates. In the
second step, TMLE ``targets'' these initial estimates by incorporating
observed data distribution information, improving the accuracy of the
causal effect estimate through an iterative process.

A notable strength of TMLE is its double-robustness property, meaning
that consistent causal effect estimation is achieved as long as either
the outcome or treatment model is correctly specified. TMLE also
incorporates cross-validation to prevent overfitting. These features
collectively create a robust methodology for examining causal effects of
interventions on outcomes. The integration of TMLE with machine learning
reduces reliance on restrictive modelling assumptions. For further
details, see (\citeproc{ref-duxedaz2021}{Díaz \emph{et al.} 2021};
\citeproc{ref-hoffman2022}{Hoffman \emph{et al.} 2022},
\citeproc{ref-hoffman2023}{2023}).

In Study 2, we used a Sequential Doubly-Robust Estimator from the
\texttt{lmtp} package, which is multiply robust for repeated treatments
across multiple waves (\citeproc{ref-diaz2023lmtp}{Díaz \emph{et al.}
2023}; \citeproc{ref-hoffman2023}{Hoffman \emph{et al.} 2023}). This
estimator strengthens robustness to misspecification in either the
outcome or treatment model. The lmtp package relies on the SuperLearner
library, integrating diverse machine learning algorithms
(\citeproc{ref-SuperLearner2023}{Polley \emph{et al.} 2023}). Given our
relatively small sample sizes (fewer than 2,000 participants), we
selected the Ranger estimator, a non-parametric causal forest method
known for its resistance to overfitting {[}Ranger2017{]}. All
statistical models implemented a 10-fold cross-validation.

\paragraph{Statistical Assumptions and Sensitivity
Analysis}\label{statistical-assumptions-and-sensitivity-analysis}

\begin{enumerate}
\def\labelenumi{\arabic{enumi}.}
\tightlist
\item
  We account for statistical dependencies among partners by including an
  indicator for relationship clusters, we assume no further statistical
  dependencies in the data.
\item
  Although semi-parametric, multiply-robust estimators offer improved
  robustness against model misspecification compared to parametric
  models, they are not completely immune to it. We assume correct model
  specification for either the outcome or treatment model.
\item
  We perform sensitivity analyses using the E-value metric
  (\citeproc{ref-linden2020EVALUE}{Linden \emph{et al.} 2020};
  \citeproc{ref-vanderweele2017}{VanderWeele and Ding 2017}). The
  E-value represents the minimum association strength (on the risk ratio
  scale) that an unmeasured confounder would need to have with both the
  exposure and outcome---after adjusting for measured covariates---to
  explain away the observed exposure-outcome association
  (\citeproc{ref-linden2020EVALUE}{Linden \emph{et al.} 2020};
  \citeproc{ref-vanderweele2020}{VanderWeele \emph{et al.} 2020}).
\end{enumerate}

\paragraph{Pre-Registered Analysis
Plan}\label{pre-registered-analysis-plan}

Criteria for analysis in Study 1 was pre-specified and pre-registered in
advance of the analysis at:\url{https://osf.io/ce4t9/}, and followed
standard NZAVS protocols for three-wave causal inference designs
(\citeproc{ref-bulbulia2024PRACTICAL}{Bulbulia 2024a}). The results for
Study 1 are presented in \href{appendix-XXX}{Appendix XXX}.

The Study 1 as presented in the main text we deviated from the current
plan in two ways. First, we incorporated an additional baseline
confounders: reported health disability, relationship status, and length
of relationship. Second, we introduced indicators to denote when
baseline variables had been imputed. These decisions are based on the
assumption that health disability and relationship status might affect
both personality traits and well-being outcomes. Additionally, length in
a relationship might affect both personality facets and well-being
outcomes. We introduced indicators to denote when baseline variables had
been imputed. This refinement aligns our approach with the
recommendations of Hoffman \emph{et al.}
(\citeproc{ref-hoffman2023}{2023}), ensuring robustness in handling
missing data. Essentially machine learning can better adjust for
imputation at baseline by recognising which variables have been imputed
for each individual.

Results for the pre-specified OSF are presented in
\href{appendix-XXX}{Appendix XXX}. \emph{There were no substantive
differences in the augmented design.} However, we report the augmented
design as a model for best practice.

Studies 2 is not pre-registered. It is the result of added capacity to
investigate long-term causal effects that developed in our lab only
after the pre-registration. The motivation for this study is closely
related to that of Study 1. However, instead of investigating the effect
of a single-year exposure to psychopathy we consider the effects of four
year exposures.

This allows us to consider the causal question: 'what would be the
causal effects of increasing (or decreasing) psychopathic personality
traits on partners in the short term (Study 1) and the long term (Study
2) (\citeproc{ref-bulbulia2024wierd}{Bulbulia 2024d}). In both cases,
our results should be considered preliminary and exploratory.

\subsection{Study 1: Three Wave Effect of Psychopathy and Personality
Interventions on Partner
Well-being}\label{study-1-three-wave-effect-of-psychopathy-and-personality-interventions-on-partner-well-being}

\subsubsection{Results Study 1 Antagonism: Partner Shift Down vs
Null}\label{results-study-1-antagonism-partner-shift-down-vs-null}

\begin{longtable}[]{@{}
  >{\raggedright\arraybackslash}p{(\linewidth - 10\tabcolsep) * \real{0.4494}}
  >{\raggedleft\arraybackslash}p{(\linewidth - 10\tabcolsep) * \real{0.1798}}
  >{\raggedleft\arraybackslash}p{(\linewidth - 10\tabcolsep) * \real{0.0674}}
  >{\raggedleft\arraybackslash}p{(\linewidth - 10\tabcolsep) * \real{0.0787}}
  >{\raggedleft\arraybackslash}p{(\linewidth - 10\tabcolsep) * \real{0.0899}}
  >{\raggedleft\arraybackslash}p{(\linewidth - 10\tabcolsep) * \real{0.1348}}@{}}

\caption{\label{tbl-results-antagonism-partner-down}Table for antagonism
effect on partner multi-dimensional well-being: shift down vs null}

\tabularnewline

\toprule\noalign{}
\begin{minipage}[b]{\linewidth}\raggedright
\end{minipage} & \begin{minipage}[b]{\linewidth}\raggedleft
E{[}Y(1){]}-E{[}Y(0){]}
\end{minipage} & \begin{minipage}[b]{\linewidth}\raggedleft
2.5 \%
\end{minipage} & \begin{minipage}[b]{\linewidth}\raggedleft
97.5 \%
\end{minipage} & \begin{minipage}[b]{\linewidth}\raggedleft
E\_Value
\end{minipage} & \begin{minipage}[b]{\linewidth}\raggedleft
E\_Val\_bound
\end{minipage} \\
\midrule\noalign{}
\endhead
\bottomrule\noalign{}
\endlastfoot
Conflict in Relationship: Partner & -0.05 & -0.10 & 0.00 & 1.27 &
1.06 \\
Kessler 6 Anxiety: Partner & -0.02 & -0.06 & 0.02 & 1.16 & 1.00 \\
Kessler 6 Depression: Partner & 0.00 & -0.04 & 0.04 & 1.00 & 1.00 \\
Kessler 6 Distress: Partner & -0.02 & -0.06 & 0.02 & 1.15 & 1.00 \\
Life Satisfaction: Partner & 0.01 & -0.03 & 0.04 & 1.08 & 1.00 \\
Personal Well-Being Index: Partner & 0.01 & -0.03 & 0.04 & 1.09 &
1.00 \\
Satisfaction with Relationship: Partner & 0.01 & -0.03 & 0.06 & 1.13 &
1.00 \\
Self Esteem: Partner & 0.00 & -0.03 & 0.04 & 1.03 & 1.00 \\

\end{longtable}

No reliable causal evidence detected for the reported outcomes.

\begin{figure}

\centering{

\pandocbounded{\includegraphics[keepaspectratio]{4-DEC-manuscript-aaron-psychopathy_files/figure-pdf/fig-results-antagonism-partner-down-1.pdf}}

}

\caption{\label{fig-results-antagonism-partner-down}Results for
antagonism effect on partner multi-dimensional well-being: shift down vs
null}

\end{figure}%

\newpage{}

\subsubsection{Results Study 1 Disinhibition: Partner Shift Up vs
Null}\label{results-study-1-disinhibition-partner-shift-up-vs-null}

\begin{longtable}[]{@{}
  >{\raggedright\arraybackslash}p{(\linewidth - 10\tabcolsep) * \real{0.4494}}
  >{\raggedleft\arraybackslash}p{(\linewidth - 10\tabcolsep) * \real{0.1798}}
  >{\raggedleft\arraybackslash}p{(\linewidth - 10\tabcolsep) * \real{0.0674}}
  >{\raggedleft\arraybackslash}p{(\linewidth - 10\tabcolsep) * \real{0.0787}}
  >{\raggedleft\arraybackslash}p{(\linewidth - 10\tabcolsep) * \real{0.0899}}
  >{\raggedleft\arraybackslash}p{(\linewidth - 10\tabcolsep) * \real{0.1348}}@{}}

\caption{\label{tbl-results-disinhibition-partner-up}Table for
disinhibition effect on partner multi-dimensional well-being: shift up
vs null}

\tabularnewline

\toprule\noalign{}
\begin{minipage}[b]{\linewidth}\raggedright
\end{minipage} & \begin{minipage}[b]{\linewidth}\raggedleft
E{[}Y(1){]}-E{[}Y(0){]}
\end{minipage} & \begin{minipage}[b]{\linewidth}\raggedleft
2.5 \%
\end{minipage} & \begin{minipage}[b]{\linewidth}\raggedleft
97.5 \%
\end{minipage} & \begin{minipage}[b]{\linewidth}\raggedleft
E\_Value
\end{minipage} & \begin{minipage}[b]{\linewidth}\raggedleft
E\_Val\_bound
\end{minipage} \\
\midrule\noalign{}
\endhead
\bottomrule\noalign{}
\endlastfoot
Conflict in Relationship: Partner & 0.03 & -0.02 & 0.07 & 1.18 & 1.00 \\
Kessler 6 Anxiety: Partner & 0.01 & -0.03 & 0.05 & 1.13 & 1.00 \\
Kessler 6 Depression: Partner & 0.03 & 0.00 & 0.07 & 1.21 & 1.03 \\
Kessler 6 Distress: Partner & 0.03 & -0.01 & 0.06 & 1.18 & 1.00 \\
Life Satisfaction: Partner & 0.00 & -0.04 & 0.03 & 1.06 & 1.00 \\
Personal Well-Being Index: Partner & -0.03 & -0.06 & 0.01 & 1.19 &
1.00 \\
Satisfaction with Relationship: Partner & -0.02 & -0.06 & 0.02 & 1.16 &
1.00 \\
Self Esteem: Partner & 0.00 & -0.03 & 0.03 & 1.03 & 1.00 \\

\end{longtable}

No reliable causal evidence detected for the reported outcomes.

\begin{figure}

\centering{

\pandocbounded{\includegraphics[keepaspectratio]{4-DEC-manuscript-aaron-psychopathy_files/figure-pdf/fig-results-disinhibition-partner-up-1.pdf}}

}

\caption{\label{fig-results-disinhibition-partner-up}Results for
disinhibition effect on partner multi-dimensional well-being: shift up
vs null}

\end{figure}%

\subsubsection{Results Study 1 Disinhibition: Partner Shift Down vs
Null}\label{results-study-1-disinhibition-partner-shift-down-vs-null}

No reliable causal evidence detected for the reported outcomes.

\begin{figure}

\centering{

\pandocbounded{\includegraphics[keepaspectratio]{4-DEC-manuscript-aaron-psychopathy_files/figure-pdf/fig-results-disinhibition-partner-down-1.pdf}}

}

\caption{\label{fig-results-disinhibition-partner-down}Results for
disinhibition effect on partner multi-dimensional well-being: shift down
vs null}

\end{figure}%

\newpage{}

\subsubsection{Results Study 1 Emotional Stability: Partner Shift Up vs
Null}\label{results-study-1-emotional-stability-partner-shift-up-vs-null}

\paragraph{Life satisfaction: partner}\label{life-satisfaction-partner}

The effect estimate (rd) is 0.051 (0.009, 0.093). On the original scale,
the estimated effect is 0.056 (0.01, 0.102). E-value lower bound is
1.105, indicating evidence for causality.

All other effect estimates presented either weak or unreliable evidence
for causality.

\begin{figure}

\centering{

\pandocbounded{\includegraphics[keepaspectratio]{4-DEC-manuscript-aaron-psychopathy_files/figure-pdf/fig-results-emotional-stability-partner-up-1.pdf}}

}

\caption{\label{fig-results-emotional-stability-partner-up}Results for
emotional stability effect on partner multi-dimensional well-being:
shift up vs null}

\end{figure}%

\newpage{}

\subsubsection{Results Study 1 Emotional Stability: Partner Shift Down
vs
Null}\label{results-study-1-emotional-stability-partner-shift-down-vs-null}

No reliable causal evidence detected for the reported outcomes.

\begin{figure}

\centering{

\pandocbounded{\includegraphics[keepaspectratio]{4-DEC-manuscript-aaron-psychopathy_files/figure-pdf/fig-results-emotional-stability-partner-down-1.pdf}}

}

\caption{\label{fig-results-emotional-stability-partner-down}Results for
emotional stability effect on partner multi-dimensional well-being:
shift down vs null}

\end{figure}%

\newpage{}

\subsubsection{Results Study 1 Narcissism: Partner Shift Up vs
Null}\label{results-study-1-narcissism-partner-shift-up-vs-null}

No reliable causal evidence detected for the reported outcomes.

\begin{figure}

\centering{

\pandocbounded{\includegraphics[keepaspectratio]{4-DEC-manuscript-aaron-psychopathy_files/figure-pdf/fig-results-narcissism-partner-up-1.pdf}}

}

\caption{\label{fig-results-narcissism-partner-up}Results for narcissism
effect on partner multi-dimensional well-being: shift up vs null}

\end{figure}%

\newpage{}

\subsubsection{Results Study 1 Narcissism: Partner Shift Down vs
Null}\label{results-study-1-narcissism-partner-shift-down-vs-null}

No reliable causal evidence detected for the reported outcomes.

\begin{figure}

\centering{

\pandocbounded{\includegraphics[keepaspectratio]{4-DEC-manuscript-aaron-psychopathy_files/figure-pdf/fig-results-narcissism-partner-down-1.pdf}}

}

\caption{\label{fig-results-narcissism-partner-down}Results for
narcissism effect on partner multi-dimensional well-being: shift down vs
null}

\end{figure}%

\newpage{}

\subsubsection{Results Study 1 Psychopathy: Partner Shift Up vs
Null}\label{results-study-1-psychopathy-partner-shift-up-vs-null}

\paragraph{Conflict in relationship:
partner}\label{conflict-in-relationship-partner}

The effect estimate (rd) is 0.122 (0.067, 0.178). On the original scale,
the estimated effect is 0.161 (0.088, 0.234). E-value lower bound is
1.322, indicating evidence for causality.

\paragraph{Kessler 6 anxiety: partner}\label{kessler-6-anxiety-partner}

The effect estimate (rd) is 0.055 (0.011, 0.098). On the original scale,
the estimated effect is 0.039 (0.008, 0.07). E-value lower bound is
1.116, indicating evidence for causality.

All other effect estimates presented either weak or unreliable evidence
for causality.

\begin{figure}

\centering{

\pandocbounded{\includegraphics[keepaspectratio]{4-DEC-manuscript-aaron-psychopathy_files/figure-pdf/fig-results-psychopathy-partner-up-1.pdf}}

}

\caption{\label{fig-results-psychopathy-partner-up}Results for
psychopathy effect on partner multi-dimensional well-being: shift up vs
null}

\end{figure}%

\newpage{}

\subsubsection{Results Study 1 Psychopathy: Partner Shift Down vs
Null}\label{results-study-1-psychopathy-partner-shift-down-vs-null}

No reliable causal evidence detected for the reported outcomes.

\begin{figure}

\centering{

\pandocbounded{\includegraphics[keepaspectratio]{4-DEC-manuscript-aaron-psychopathy_files/figure-pdf/fig-results-psychopathy-partner-down-1.pdf}}

}

\caption{\label{fig-results-psychopathy-partner-down}Results for
psychopathy effect on partner multi-dimensional well-being: shift down
vs null}

\end{figure}%

\newpage{}

\subsubsection{Results Study 2 Antagonism: Partner Shift Down vs
Null}\label{results-study-2-antagonism-partner-shift-down-vs-null}

\begin{longtable}[]{@{}
  >{\raggedright\arraybackslash}p{(\linewidth - 10\tabcolsep) * \real{0.4494}}
  >{\raggedleft\arraybackslash}p{(\linewidth - 10\tabcolsep) * \real{0.1798}}
  >{\raggedleft\arraybackslash}p{(\linewidth - 10\tabcolsep) * \real{0.0674}}
  >{\raggedleft\arraybackslash}p{(\linewidth - 10\tabcolsep) * \real{0.0787}}
  >{\raggedleft\arraybackslash}p{(\linewidth - 10\tabcolsep) * \real{0.0899}}
  >{\raggedleft\arraybackslash}p{(\linewidth - 10\tabcolsep) * \real{0.1348}}@{}}

\caption{\label{tbl-results-antagonism-partner-down-long}Table for
antagonism effect on partner multi-dimensional well-being (6 waves):
shift down vs null}

\tabularnewline

\toprule\noalign{}
\begin{minipage}[b]{\linewidth}\raggedright
\end{minipage} & \begin{minipage}[b]{\linewidth}\raggedleft
E{[}Y(1){]}-E{[}Y(0){]}
\end{minipage} & \begin{minipage}[b]{\linewidth}\raggedleft
2.5 \%
\end{minipage} & \begin{minipage}[b]{\linewidth}\raggedleft
97.5 \%
\end{minipage} & \begin{minipage}[b]{\linewidth}\raggedleft
E\_Value
\end{minipage} & \begin{minipage}[b]{\linewidth}\raggedleft
E\_Val\_bound
\end{minipage} \\
\midrule\noalign{}
\endhead
\bottomrule\noalign{}
\endlastfoot
Conflict in Relationship: Partner & -0.08 & -0.18 & 0.01 & 1.37 &
1.00 \\
Kessler 6 Anxiety: Partner & -0.16 & -0.39 & 0.06 & 1.58 & 1.00 \\
Kessler 6 Depression: Partner & -0.03 & -0.16 & 0.10 & 1.20 & 1.00 \\
Kessler 6 Distress: Partner & -0.14 & -0.33 & 0.06 & 1.52 & 1.00 \\
Life Satisfaction: Partner & -0.19 & -0.37 & 0.00 & 1.65 & 1.05 \\
Personal Well-Being Index: Partner & -0.02 & -0.15 & 0.10 & 1.16 &
1.00 \\
Satisfaction with Relationship: Partner & -0.09 & -0.19 & 0.01 & 1.38 &
1.00 \\
Self Esteem: Partner & 0.04 & -0.15 & 0.22 & 1.22 & 1.00 \\

\end{longtable}

No reliable causal evidence detected for the reported outcomes.

\begin{figure}

\centering{

\pandocbounded{\includegraphics[keepaspectratio]{4-DEC-manuscript-aaron-psychopathy_files/figure-pdf/fig-results-antagonism-partner-down-long-1.pdf}}

}

\caption{\label{fig-results-antagonism-partner-down-long}Results for
antagonism effect on partner multi-dimensional well-being (6 waves):
shift down vs null}

\end{figure}%

\newpage{}

\subsubsection{Results Study 2 Disinhibition: Partner Shift Up vs
Null}\label{results-study-2-disinhibition-partner-shift-up-vs-null}

\begin{longtable}[]{@{}
  >{\raggedright\arraybackslash}p{(\linewidth - 10\tabcolsep) * \real{0.4494}}
  >{\raggedleft\arraybackslash}p{(\linewidth - 10\tabcolsep) * \real{0.1798}}
  >{\raggedleft\arraybackslash}p{(\linewidth - 10\tabcolsep) * \real{0.0674}}
  >{\raggedleft\arraybackslash}p{(\linewidth - 10\tabcolsep) * \real{0.0787}}
  >{\raggedleft\arraybackslash}p{(\linewidth - 10\tabcolsep) * \real{0.0899}}
  >{\raggedleft\arraybackslash}p{(\linewidth - 10\tabcolsep) * \real{0.1348}}@{}}

\caption{\label{tbl-results-disinhibition-partner-up-long}Table for
disinhibition effect on partner multi-dimensional well-being (6 waves):
shift up vs null}

\tabularnewline

\toprule\noalign{}
\begin{minipage}[b]{\linewidth}\raggedright
\end{minipage} & \begin{minipage}[b]{\linewidth}\raggedleft
E{[}Y(1){]}-E{[}Y(0){]}
\end{minipage} & \begin{minipage}[b]{\linewidth}\raggedleft
2.5 \%
\end{minipage} & \begin{minipage}[b]{\linewidth}\raggedleft
97.5 \%
\end{minipage} & \begin{minipage}[b]{\linewidth}\raggedleft
E\_Value
\end{minipage} & \begin{minipage}[b]{\linewidth}\raggedleft
E\_Val\_bound
\end{minipage} \\
\midrule\noalign{}
\endhead
\bottomrule\noalign{}
\endlastfoot
Conflict in Relationship: Partner & -0.01 & -0.11 & 0.09 & 1.13 &
1.00 \\
Kessler 6 Anxiety: Partner & -0.01 & -0.11 & 0.09 & 1.10 & 1.00 \\
Kessler 6 Depression: Partner & 0.00 & -0.11 & 0.11 & 1.00 & 1.00 \\
Kessler 6 Distress: Partner & 0.04 & -0.05 & 0.13 & 1.25 & 1.00 \\
Life Satisfaction: Partner & -0.12 & -0.22 & -0.03 & 1.48 & 1.19 \\
Personal Well-Being Index: Partner & -0.23 & -0.35 & -0.10 & 1.76 &
1.42 \\
Satisfaction with Relationship: Partner & -0.07 & -0.17 & 0.02 & 1.35 &
1.00 \\
Self Esteem: Partner & -0.09 & -0.16 & -0.01 & 1.38 & 1.10 \\

\end{longtable}

\paragraph{Life satisfaction:
partner}\label{life-satisfaction-partner-1}

The effect estimate (rd) is -0.122 (-0.216, -0.029). On the original
scale, the estimated effect is -0.136 (-0.241, -0.032). E-value lower
bound is 1.189, indicating evidence for causality.

\paragraph{Personal well-being index:
partner}\label{personal-well-being-index-partner}

The effect estimate (rd) is -0.226 (-0.352, -0.1). On the original
scale, the estimated effect is -0.364 (-0.567, -0.161). E-value lower
bound is 1.421, indicating evidence for causality.

\paragraph{Self esteem: partner}\label{self-esteem-partner}

The effect estimate (rd) is -0.086 (-0.163, -0.009). On the original
scale, the estimated effect is -0.107 (-0.203, -0.011). E-value lower
bound is 1.104, indicating evidence for causality.

All other effect estimates presented either weak or unreliable evidence
for causality.

\begin{figure}

\centering{

\pandocbounded{\includegraphics[keepaspectratio]{4-DEC-manuscript-aaron-psychopathy_files/figure-pdf/fig-results-disinhibition-partner-up-long-1.pdf}}

}

\caption{\label{fig-results-disinhibition-partner-up-long}Results for
disinhibition effect on partner multi-dimensional well-being (6 waves):
shift up vs null}

\end{figure}%

\newpage{}

\subsubsection{Results Study 2 Disinhibition: Partner Shift Down vs
Null}\label{results-study-2-disinhibition-partner-shift-down-vs-null}

\begin{longtable}[]{@{}
  >{\raggedright\arraybackslash}p{(\linewidth - 10\tabcolsep) * \real{0.4494}}
  >{\raggedleft\arraybackslash}p{(\linewidth - 10\tabcolsep) * \real{0.1798}}
  >{\raggedleft\arraybackslash}p{(\linewidth - 10\tabcolsep) * \real{0.0674}}
  >{\raggedleft\arraybackslash}p{(\linewidth - 10\tabcolsep) * \real{0.0787}}
  >{\raggedleft\arraybackslash}p{(\linewidth - 10\tabcolsep) * \real{0.0899}}
  >{\raggedleft\arraybackslash}p{(\linewidth - 10\tabcolsep) * \real{0.1348}}@{}}

\caption{\label{tbl-results-disinhibition-partner-down-long}Table for
disinhibition effect on partner multi-dimensional well-being (6 waves):
shift down vs null}

\tabularnewline

\toprule\noalign{}
\begin{minipage}[b]{\linewidth}\raggedright
\end{minipage} & \begin{minipage}[b]{\linewidth}\raggedleft
E{[}Y(1){]}-E{[}Y(0){]}
\end{minipage} & \begin{minipage}[b]{\linewidth}\raggedleft
2.5 \%
\end{minipage} & \begin{minipage}[b]{\linewidth}\raggedleft
97.5 \%
\end{minipage} & \begin{minipage}[b]{\linewidth}\raggedleft
E\_Value
\end{minipage} & \begin{minipage}[b]{\linewidth}\raggedleft
E\_Val\_bound
\end{minipage} \\
\midrule\noalign{}
\endhead
\bottomrule\noalign{}
\endlastfoot
Conflict in Relationship: Partner & -0.16 & -0.30 & -0.01 & 1.57 &
1.13 \\
Kessler 6 Anxiety: Partner & -0.02 & -0.17 & 0.14 & 1.14 & 1.00 \\
Kessler 6 Depression: Partner & 0.20 & -0.05 & 0.46 & 1.70 & 1.00 \\
Kessler 6 Distress: Partner & 0.07 & -0.11 & 0.24 & 1.32 & 1.00 \\
Life Satisfaction: Partner & -0.18 & -0.44 & 0.08 & 1.64 & 1.00 \\
Personal Well-Being Index: Partner & 0.06 & 0.00 & 0.13 & 1.31 & 1.00 \\
Satisfaction with Relationship: Partner & 0.03 & -0.06 & 0.11 & 1.19 &
1.00 \\
Self Esteem: Partner & -0.22 & -0.52 & 0.07 & 1.75 & 1.00 \\

\end{longtable}

\paragraph{Conflict in relationship:
partner}\label{conflict-in-relationship-partner-1}

The effect estimate (rd) is -0.155 (-0.296, -0.015). On the original
scale, the estimated effect is -0.203 (-0.387, -0.019). E-value lower
bound is 1.128, indicating evidence for causality.

All other effect estimates presented either weak or unreliable evidence
for causality.

\begin{figure}

\centering{

\pandocbounded{\includegraphics[keepaspectratio]{4-DEC-manuscript-aaron-psychopathy_files/figure-pdf/fig-results-disinhibition-partner-down-long-1.pdf}}

}

\caption{\label{fig-results-disinhibition-partner-down-long}Results for
disinhibition effect on partner multi-dimensional well-being (6 waves):
shift down vs null}

\end{figure}%

\newpage{}

\subsubsection{Results Study 2 Emotional Stability: Partner Shift Up vs
Null}\label{results-study-2-emotional-stability-partner-shift-up-vs-null}

\begin{longtable}[]{@{}
  >{\raggedright\arraybackslash}p{(\linewidth - 10\tabcolsep) * \real{0.4494}}
  >{\raggedleft\arraybackslash}p{(\linewidth - 10\tabcolsep) * \real{0.1798}}
  >{\raggedleft\arraybackslash}p{(\linewidth - 10\tabcolsep) * \real{0.0674}}
  >{\raggedleft\arraybackslash}p{(\linewidth - 10\tabcolsep) * \real{0.0787}}
  >{\raggedleft\arraybackslash}p{(\linewidth - 10\tabcolsep) * \real{0.0899}}
  >{\raggedleft\arraybackslash}p{(\linewidth - 10\tabcolsep) * \real{0.1348}}@{}}

\caption{\label{tbl-results-emotional-stability-partner-up-long}Table
for emotional stability effect on partner multi-dimensional well-being
(6 waves): shift up vs null}

\tabularnewline

\toprule\noalign{}
\begin{minipage}[b]{\linewidth}\raggedright
\end{minipage} & \begin{minipage}[b]{\linewidth}\raggedleft
E{[}Y(1){]}-E{[}Y(0){]}
\end{minipage} & \begin{minipage}[b]{\linewidth}\raggedleft
2.5 \%
\end{minipage} & \begin{minipage}[b]{\linewidth}\raggedleft
97.5 \%
\end{minipage} & \begin{minipage}[b]{\linewidth}\raggedleft
E\_Value
\end{minipage} & \begin{minipage}[b]{\linewidth}\raggedleft
E\_Val\_bound
\end{minipage} \\
\midrule\noalign{}
\endhead
\bottomrule\noalign{}
\endlastfoot
Conflict in Relationship: Partner & 0.17 & 0.02 & 0.32 & 1.60 & 1.16 \\
Kessler 6 Anxiety: Partner & -0.15 & -0.27 & -0.04 & 1.57 & 1.24 \\
Kessler 6 Depression: Partner & -0.15 & -0.25 & -0.05 & 1.56 & 1.27 \\
Kessler 6 Distress: Partner & -0.19 & -0.33 & -0.05 & 1.67 & 1.28 \\
Life Satisfaction: Partner & 0.13 & 0.05 & 0.22 & 1.51 & 1.25 \\
Personal Well-Being Index: Partner & 0.00 & -0.10 & 0.09 & 1.07 &
1.00 \\
Satisfaction with Relationship: Partner & -0.09 & -0.19 & 0.01 & 1.39 &
1.00 \\
Self Esteem: Partner & 0.01 & -0.13 & 0.16 & 1.13 & 1.00 \\

\end{longtable}

\paragraph{Conflict in relationship:
partner}\label{conflict-in-relationship-partner-2}

The effect estimate (rd) is 0.168 (0.022, 0.315). On the original scale,
the estimated effect is 0.22 (0.028, 0.412). E-value lower bound is
1.161, indicating evidence for causality.

\paragraph{Kessler 6 anxiety:
partner}\label{kessler-6-anxiety-partner-1}

The effect estimate (rd) is -0.154 (-0.267, -0.041). On the original
scale, the estimated effect is -0.109 (-0.189, -0.029). E-value lower
bound is 1.242, indicating evidence for causality.

\paragraph{Kessler 6 depression:
partner}\label{kessler-6-depression-partner}

The effect estimate (rd) is -0.152 (-0.252, -0.052). On the original
scale, the estimated effect is -0.095 (-0.158, -0.033). E-value lower
bound is 1.275, indicating evidence for causality.

\paragraph{Kessler 6 distress:
partner}\label{kessler-6-distress-partner}

The effect estimate (rd) is -0.192 (-0.332, -0.053). On the original
scale, the estimated effect is -0.689 (-1.19, -0.189). E-value lower
bound is 1.278, indicating evidence for causality.

\paragraph{Life satisfaction:
partner}\label{life-satisfaction-partner-2}

The effect estimate (rd) is 0.134 (0.046, 0.223). On the original scale,
the estimated effect is 0.15 (0.051, 0.249). E-value lower bound is
1.254, indicating evidence for causality.

All other effect estimates presented either weak or unreliable evidence
for causality.

\begin{figure}

\centering{

\pandocbounded{\includegraphics[keepaspectratio]{4-DEC-manuscript-aaron-psychopathy_files/figure-pdf/fig-results-emotional-stability-partner-up-long-1.pdf}}

}

\caption{\label{fig-results-emotional-stability-partner-up-long}Results
for emotional stability effect on partner multi-dimensional well-being
(6 waves): shift up vs null}

\end{figure}%

\newpage{}

\subsubsection{Results Study 2 Emotional Stability: Partner Shift Down
vs
Null}\label{results-study-2-emotional-stability-partner-shift-down-vs-null}

\begin{longtable}[]{@{}
  >{\raggedright\arraybackslash}p{(\linewidth - 10\tabcolsep) * \real{0.4494}}
  >{\raggedleft\arraybackslash}p{(\linewidth - 10\tabcolsep) * \real{0.1798}}
  >{\raggedleft\arraybackslash}p{(\linewidth - 10\tabcolsep) * \real{0.0674}}
  >{\raggedleft\arraybackslash}p{(\linewidth - 10\tabcolsep) * \real{0.0787}}
  >{\raggedleft\arraybackslash}p{(\linewidth - 10\tabcolsep) * \real{0.0899}}
  >{\raggedleft\arraybackslash}p{(\linewidth - 10\tabcolsep) * \real{0.1348}}@{}}

\caption{\label{tbl-results-emotional-stability-partner-down-long}Table
for emotional stability effect on partner multi-dimensional well-being
(6 waves): shift down vs null}

\tabularnewline

\toprule\noalign{}
\begin{minipage}[b]{\linewidth}\raggedright
\end{minipage} & \begin{minipage}[b]{\linewidth}\raggedleft
E{[}Y(1){]}-E{[}Y(0){]}
\end{minipage} & \begin{minipage}[b]{\linewidth}\raggedleft
2.5 \%
\end{minipage} & \begin{minipage}[b]{\linewidth}\raggedleft
97.5 \%
\end{minipage} & \begin{minipage}[b]{\linewidth}\raggedleft
E\_Value
\end{minipage} & \begin{minipage}[b]{\linewidth}\raggedleft
E\_Val\_bound
\end{minipage} \\
\midrule\noalign{}
\endhead
\bottomrule\noalign{}
\endlastfoot
Conflict in Relationship: Partner & -0.10 & -0.18 & -0.01 & 1.40 &
1.12 \\
Kessler 6 Anxiety: Partner & 0.00 & -0.09 & 0.08 & 1.06 & 1.00 \\
Kessler 6 Depression: Partner & 0.00 & -0.08 & 0.08 & 1.05 & 1.00 \\
Kessler 6 Distress: Partner & 0.00 & -0.08 & 0.07 & 1.06 & 1.00 \\
Life Satisfaction: Partner & -0.04 & -0.13 & 0.06 & 1.23 & 1.00 \\
Personal Well-Being Index: Partner & -0.02 & -0.10 & 0.07 & 1.14 &
1.00 \\
Satisfaction with Relationship: Partner & 0.03 & -0.06 & 0.12 & 1.19 &
1.00 \\
Self Esteem: Partner & -0.09 & -0.18 & 0.00 & 1.39 & 1.00 \\

\end{longtable}

\paragraph{Conflict in relationship:
partner}\label{conflict-in-relationship-partner-3}

The effect estimate (rd) is -0.095 (-0.178, -0.013). On the original
scale, the estimated effect is -0.124 (-0.233, -0.016). E-value lower
bound is 1.121, indicating evidence for causality.

All other effect estimates presented either weak or unreliable evidence
for causality.

\begin{figure}

\centering{

\pandocbounded{\includegraphics[keepaspectratio]{4-DEC-manuscript-aaron-psychopathy_files/figure-pdf/fig-results-emotional-stability-partner-down-long-1.pdf}}

}

\caption{\label{fig-results-emotional-stability-partner-down-long}Results
for emotional stability effect on partner multi-dimensional well-being
(6 waves): shift down vs null}

\end{figure}%

\newpage{}

\subsubsection{Results Study 2a Narcissism: Partner Shift Up vs
Null}\label{results-study-2a-narcissism-partner-shift-up-vs-null}

\begin{longtable}[]{@{}
  >{\raggedright\arraybackslash}p{(\linewidth - 10\tabcolsep) * \real{0.4494}}
  >{\raggedleft\arraybackslash}p{(\linewidth - 10\tabcolsep) * \real{0.1798}}
  >{\raggedleft\arraybackslash}p{(\linewidth - 10\tabcolsep) * \real{0.0674}}
  >{\raggedleft\arraybackslash}p{(\linewidth - 10\tabcolsep) * \real{0.0787}}
  >{\raggedleft\arraybackslash}p{(\linewidth - 10\tabcolsep) * \real{0.0899}}
  >{\raggedleft\arraybackslash}p{(\linewidth - 10\tabcolsep) * \real{0.1348}}@{}}

\caption{\label{tbl-results-narcissism-partner-up-long}Table for
narcissism effect on partner multi-dimensional well-being (6 waves):
shift up vs null}

\tabularnewline

\toprule\noalign{}
\begin{minipage}[b]{\linewidth}\raggedright
\end{minipage} & \begin{minipage}[b]{\linewidth}\raggedleft
E{[}Y(1){]}-E{[}Y(0){]}
\end{minipage} & \begin{minipage}[b]{\linewidth}\raggedleft
2.5 \%
\end{minipage} & \begin{minipage}[b]{\linewidth}\raggedleft
97.5 \%
\end{minipage} & \begin{minipage}[b]{\linewidth}\raggedleft
E\_Value
\end{minipage} & \begin{minipage}[b]{\linewidth}\raggedleft
E\_Val\_bound
\end{minipage} \\
\midrule\noalign{}
\endhead
\bottomrule\noalign{}
\endlastfoot
Conflict in Relationship: Partner & -0.04 & -0.14 & 0.06 & 1.23 &
1.00 \\
Kessler 6 Anxiety: Partner & -0.01 & -0.10 & 0.09 & 1.09 & 1.00 \\
Kessler 6 Depression: Partner & 0.06 & -0.03 & 0.16 & 1.31 & 1.00 \\
Kessler 6 Distress: Partner & 0.03 & -0.07 & 0.13 & 1.20 & 1.00 \\
Life Satisfaction: Partner & 0.02 & -0.07 & 0.11 & 1.16 & 1.00 \\
Personal Well-Being Index: Partner & -0.16 & -0.28 & -0.04 & 1.58 &
1.22 \\
Satisfaction with Relationship: Partner & 0.16 & 0.06 & 0.26 & 1.58 &
1.30 \\
Self Esteem: Partner & -0.05 & -0.13 & 0.03 & 1.27 & 1.00 \\

\end{longtable}

\paragraph{Personal well-being index:
partner}\label{personal-well-being-index-partner-1}

The effect estimate (rd) is -0.16 (-0.284, -0.036). On the original
scale, the estimated effect is -0.258 (-0.457, -0.058). E-value lower
bound is 1.222, indicating evidence for causality.

\paragraph{Satisfaction with relationship:
partner}\label{satisfaction-with-relationship-partner}

The effect estimate (rd) is 0.159 (0.058, 0.259). On the original scale,
the estimated effect is 0.184 (0.068, 0.3). E-value lower bound is
1.297, indicating evidence for causality.

All other effect estimates presented either weak or unreliable evidence
for causality.

\begin{figure}

\centering{

\pandocbounded{\includegraphics[keepaspectratio]{4-DEC-manuscript-aaron-psychopathy_files/figure-pdf/fig-results-narcissism-partner-up-long-1.pdf}}

}

\caption{\label{fig-results-narcissism-partner-up-long}Results for
narcissism effect on partner multi-dimensional well-being (6 waves):
shift up vs null}

\end{figure}%

\newpage{}

\subsubsection{Results Study 2 Narcissism: Partner Shift Down vs
Null}\label{results-study-2-narcissism-partner-shift-down-vs-null}

\begin{longtable}[]{@{}
  >{\raggedright\arraybackslash}p{(\linewidth - 10\tabcolsep) * \real{0.4494}}
  >{\raggedleft\arraybackslash}p{(\linewidth - 10\tabcolsep) * \real{0.1798}}
  >{\raggedleft\arraybackslash}p{(\linewidth - 10\tabcolsep) * \real{0.0674}}
  >{\raggedleft\arraybackslash}p{(\linewidth - 10\tabcolsep) * \real{0.0787}}
  >{\raggedleft\arraybackslash}p{(\linewidth - 10\tabcolsep) * \real{0.0899}}
  >{\raggedleft\arraybackslash}p{(\linewidth - 10\tabcolsep) * \real{0.1348}}@{}}

\caption{\label{tbl-results-narcissism-partner-down-long}Table for
narcissism effect on partner multi-dimensional well-being (6 waves):
shift down vs null}

\tabularnewline

\toprule\noalign{}
\begin{minipage}[b]{\linewidth}\raggedright
\end{minipage} & \begin{minipage}[b]{\linewidth}\raggedleft
E{[}Y(1){]}-E{[}Y(0){]}
\end{minipage} & \begin{minipage}[b]{\linewidth}\raggedleft
2.5 \%
\end{minipage} & \begin{minipage}[b]{\linewidth}\raggedleft
97.5 \%
\end{minipage} & \begin{minipage}[b]{\linewidth}\raggedleft
E\_Value
\end{minipage} & \begin{minipage}[b]{\linewidth}\raggedleft
E\_Val\_bound
\end{minipage} \\
\midrule\noalign{}
\endhead
\bottomrule\noalign{}
\endlastfoot
Conflict in Relationship: Partner & -0.01 & -0.09 & 0.06 & 1.13 &
1.00 \\
Kessler 6 Anxiety: Partner & 0.09 & 0.02 & 0.15 & 1.38 & 1.16 \\
Kessler 6 Depression: Partner & 0.05 & -0.02 & 0.12 & 1.27 & 1.00 \\
Kessler 6 Distress: Partner & 0.05 & -0.01 & 0.12 & 1.28 & 1.00 \\
Life Satisfaction: Partner & -0.06 & -0.13 & 0.01 & 1.29 & 1.00 \\
Personal Well-Being Index: Partner & -0.11 & -0.17 & -0.05 & 1.45 &
1.26 \\
Satisfaction with Relationship: Partner & -0.08 & -0.16 & 0.01 & 1.35 &
1.00 \\
Self Esteem: Partner & -0.04 & -0.10 & 0.02 & 1.24 & 1.00 \\

\end{longtable}

\paragraph{Kessler 6 anxiety:
partner}\label{kessler-6-anxiety-partner-2}

The effect estimate (rd) is 0.085 (0.019, 0.15). On the original scale,
the estimated effect is 0.06 (0.014, 0.106). E-value lower bound is
1.157, indicating evidence for causality.

\paragraph{Personal well-being index:
partner}\label{personal-well-being-index-partner-2}

The effect estimate (rd) is -0.11 (-0.172, -0.048). On the original
scale, the estimated effect is -0.177 (-0.277, -0.077). E-value lower
bound is 1.265, indicating evidence for causality.

All other effect estimates presented either weak or unreliable evidence
for causality.

\begin{figure}

\centering{

\pandocbounded{\includegraphics[keepaspectratio]{4-DEC-manuscript-aaron-psychopathy_files/figure-pdf/fig-results-narcissism-partner-down-long-1.pdf}}

}

\caption{\label{fig-results-narcissism-partner-down-long}Results for
narcissism effect on partner multi-dimensional well-being (6 waves):
shift down vs null}

\end{figure}%

\newpage{}

\subsubsection{Results Study 2 Psychopathy: Partner Shift Up vs
Null}\label{results-study-2-psychopathy-partner-shift-up-vs-null}

\begin{longtable}[]{@{}
  >{\raggedright\arraybackslash}p{(\linewidth - 10\tabcolsep) * \real{0.4494}}
  >{\raggedleft\arraybackslash}p{(\linewidth - 10\tabcolsep) * \real{0.1798}}
  >{\raggedleft\arraybackslash}p{(\linewidth - 10\tabcolsep) * \real{0.0674}}
  >{\raggedleft\arraybackslash}p{(\linewidth - 10\tabcolsep) * \real{0.0787}}
  >{\raggedleft\arraybackslash}p{(\linewidth - 10\tabcolsep) * \real{0.0899}}
  >{\raggedleft\arraybackslash}p{(\linewidth - 10\tabcolsep) * \real{0.1348}}@{}}

\caption{\label{tbl-results-psychopathy-partner-up-long}Table for
psychopathy effect on partner multi-dimensional well-being (6 waves):
shift up vs null}

\tabularnewline

\toprule\noalign{}
\begin{minipage}[b]{\linewidth}\raggedright
\end{minipage} & \begin{minipage}[b]{\linewidth}\raggedleft
E{[}Y(1){]}-E{[}Y(0){]}
\end{minipage} & \begin{minipage}[b]{\linewidth}\raggedleft
2.5 \%
\end{minipage} & \begin{minipage}[b]{\linewidth}\raggedleft
97.5 \%
\end{minipage} & \begin{minipage}[b]{\linewidth}\raggedleft
E\_Value
\end{minipage} & \begin{minipage}[b]{\linewidth}\raggedleft
E\_Val\_bound
\end{minipage} \\
\midrule\noalign{}
\endhead
\bottomrule\noalign{}
\endlastfoot
Conflict in Relationship: Partner & 0.12 & 0.05 & 0.19 & 1.48 & 1.28 \\
Kessler 6 Anxiety: Partner & 0.01 & -0.05 & 0.08 & 1.13 & 1.00 \\
Kessler 6 Depression: Partner & 0.08 & 0.01 & 0.14 & 1.35 & 1.11 \\
Kessler 6 Distress: Partner & 0.03 & -0.03 & 0.10 & 1.21 & 1.00 \\
Life Satisfaction: Partner & -0.04 & -0.11 & 0.02 & 1.25 & 1.00 \\
Personal Well-Being Index: Partner & -0.15 & -0.21 & -0.09 & 1.55 &
1.38 \\
Satisfaction with Relationship: Partner & -0.09 & -0.16 & -0.01 & 1.38 &
1.11 \\
Self Esteem: Partner & -0.06 & -0.12 & 0.00 & 1.31 & 1.06 \\

\end{longtable}

\paragraph{Conflict in relationship:
partner}\label{conflict-in-relationship-partner-4}

The effect estimate (rd) is 0.123 (0.053, 0.194). On the original scale,
the estimated effect is 0.161 (0.069, 0.253). E-value lower bound is
1.276, indicating evidence for causality.

\paragraph{Kessler 6 depression:
partner}\label{kessler-6-depression-partner-1}

The effect estimate (rd) is 0.077 (0.01, 0.144). On the original scale,
the estimated effect is 0.048 (0.006, 0.09). E-value lower bound is
1.108, indicating evidence for causality.

\paragraph{Personal well-being index:
partner}\label{personal-well-being-index-partner-3}

The effect estimate (rd) is -0.148 (-0.211, -0.085). On the original
scale, the estimated effect is -0.238 (-0.34, -0.137). E-value lower
bound is 1.376, indicating evidence for causality.

\paragraph{Satisfaction with relationship:
partner}\label{satisfaction-with-relationship-partner-1}

The effect estimate (rd) is -0.088 (-0.164, -0.012). On the original
scale, the estimated effect is -0.102 (-0.189, -0.014). E-value lower
bound is 1.115, indicating evidence for causality.

All other effect estimates presented either weak or unreliable evidence
for causality.

\begin{figure}

\centering{

\pandocbounded{\includegraphics[keepaspectratio]{4-DEC-manuscript-aaron-psychopathy_files/figure-pdf/fig-results-psychopathy-partner-up-long-1.pdf}}

}

\caption{\label{fig-results-psychopathy-partner-up-long}Results for
psychopathy effect on partner multi-dimensional well-being (6 waves):
shift up vs null}

\end{figure}%

\newpage{}

\subsubsection{Results Study 2 Psychopathy: Partner Shift Down vs
Null}\label{results-study-2-psychopathy-partner-shift-down-vs-null}

\begin{longtable}[]{@{}
  >{\raggedright\arraybackslash}p{(\linewidth - 10\tabcolsep) * \real{0.4494}}
  >{\raggedleft\arraybackslash}p{(\linewidth - 10\tabcolsep) * \real{0.1798}}
  >{\raggedleft\arraybackslash}p{(\linewidth - 10\tabcolsep) * \real{0.0674}}
  >{\raggedleft\arraybackslash}p{(\linewidth - 10\tabcolsep) * \real{0.0787}}
  >{\raggedleft\arraybackslash}p{(\linewidth - 10\tabcolsep) * \real{0.0899}}
  >{\raggedleft\arraybackslash}p{(\linewidth - 10\tabcolsep) * \real{0.1348}}@{}}

\caption{\label{tbl-results-psychopathy-partner-down-long}Table for
psychopathy effect on partner multi-dimensional well-being (6 waves):
shift down vs null}

\tabularnewline

\toprule\noalign{}
\begin{minipage}[b]{\linewidth}\raggedright
\end{minipage} & \begin{minipage}[b]{\linewidth}\raggedleft
E{[}Y(1){]}-E{[}Y(0){]}
\end{minipage} & \begin{minipage}[b]{\linewidth}\raggedleft
2.5 \%
\end{minipage} & \begin{minipage}[b]{\linewidth}\raggedleft
97.5 \%
\end{minipage} & \begin{minipage}[b]{\linewidth}\raggedleft
E\_Value
\end{minipage} & \begin{minipage}[b]{\linewidth}\raggedleft
E\_Val\_bound
\end{minipage} \\
\midrule\noalign{}
\endhead
\bottomrule\noalign{}
\endlastfoot
Conflict in Relationship: Partner & 0.00 & -0.08 & 0.07 & 1.07 & 1 \\
Kessler 6 Anxiety: Partner & -0.03 & -0.10 & 0.03 & 1.21 & 1 \\
Kessler 6 Depression: Partner & 0.04 & -0.02 & 0.11 & 1.24 & 1 \\
Kessler 6 Distress: Partner & -0.01 & -0.08 & 0.05 & 1.12 & 1 \\
Life Satisfaction: Partner & -0.05 & -0.11 & 0.01 & 1.26 & 1 \\
Personal Well-Being Index: Partner & 0.00 & -0.06 & 0.06 & 1.03 & 1 \\
Satisfaction with Relationship: Partner & -0.05 & -0.13 & 0.03 & 1.27 &
1 \\
Self Esteem: Partner & -0.04 & -0.10 & 0.02 & 1.23 & 1 \\

\end{longtable}

No reliable causal evidence detected for the reported outcomes.

\begin{figure}

\centering{

\pandocbounded{\includegraphics[keepaspectratio]{4-DEC-manuscript-aaron-psychopathy_files/figure-pdf/fig-results-psychopathy-partner-down-long-1.pdf}}

}

\caption{\label{fig-results-psychopathy-partner-down-long}Results for
psychopathy effect on partner multi-dimensional well-being (6 waves):
shift down vs null}

\end{figure}%

\newpage{}

\subsubsection{Importance of Research}\label{importance-of-research}

These findings reveal that focussing on specific traits of psychopathic
personality affects people differently than does psychopathy as a
whole\ldots{}

When psychopathy is treated as a single, combined dimension, partners
tend to suffer more, and these negative effects are strong over longer
exposure periods.

The results we present here are not merely associational. They are
causal. Under the assumptions of positivity, consistency, and
exchangeability (no unmeasured confounding), we infer that psychopathy
harms romantic partnerships. Previously reported statistical
associations are not the artefacts of, for example, the attractiveness
of psychopathic personality to partners with lower well-being and
self-esteem.

\newpage{}

\subsubsection{Ethics}\label{ethics}

The NZAVS is reviewed every three years by the University of Auckland
Human Participants Ethics Committee. Our most recent ethics approval
statement is as follows: The New Zealand Attitudes and Values Study was
approved by the University of Auckland Human Participants Ethics
Committee on 26/05/2021 for six years until 26/05/2027, Reference Number
UAHPEC22576.

\subsubsection{Acknowledgements}\label{acknowledgements}

The New Zealand Attitudes and Values Study is supported by a grant from
the TempletoReligion Trust (TRT0196; TRT0418). JB received support from
the Max Planck Institute for the Science of Human History. The funders
had no role in preparing the manuscript or the decision to publish.

\subsubsection{Author Statement}\label{author-statement}

TBA\ldots{} e.g.~

\begin{itemize}
\tightlist
\item
  AH conceived of the study, led the validation study, and developed the
  theory
\item
  HE \ldots{} MH \ldots contributed to the theory.
\item
  CS led data collection.
\item
  JB developed the inferential approach and did the analysis.\\
\item
  All authors contributed to writing the final version of the
  manuscript, which was substantially AH's work.
\end{itemize}

\newpage{}

\subsection{References}\label{references}

\phantomsection\label{refs}
\begin{CSLReferences}{1}{0}
\bibitem[\citeproctext]{ref-atkinson2019}
Atkinson, J, Salmond, C, and Crampton, P (2019) \emph{NZDep2018 index of
deprivation, user{'}s manual.}, Wellington.

\bibitem[\citeproctext]{ref-bulbulia2022}
Bulbulia, JA (2023) A workflow for causal inference in cross-cultural
psychology. \emph{Religion, Brain \& Behavior}, \textbf{13}(3),
291--306.
doi:\href{https://doi.org/10.1080/2153599X.2022.2070245}{10.1080/2153599X.2022.2070245}.

\bibitem[\citeproctext]{ref-bulbulia2024PRACTICAL}
Bulbulia, JA (2024a) A practical guide to causal inference in three-wave
panel studies. \emph{PsyArXiv Preprints}.
doi:\href{https://doi.org/10.31234/osf.io/uyg3d}{10.31234/osf.io/uyg3d}.

\bibitem[\citeproctext]{ref-bulbulia2023}
Bulbulia, JA (2024b) Methods in causal inference part 1: Causal diagrams
and confounding. \emph{Evolutionary Human Sciences}, \textbf{6}, e40.
doi:\href{https://doi.org/10.1017/ehs.2024.35}{10.1017/ehs.2024.35}.

\bibitem[\citeproctext]{ref-bulbulia2024swigstime}
Bulbulia, JA (2024c) Methods in causal inference part 2: Interaction,
mediation, and time-varying treatments. \emph{Evolutionary Human
Sciences}, \textbf{6}, e41.
doi:\href{https://doi.org/10.1017/ehs.2024.32}{10.1017/ehs.2024.32}.

\bibitem[\citeproctext]{ref-bulbulia2024wierd}
Bulbulia, JA (2024d) Methods in causal inference part 3: Measurement
error and external validity threats. \emph{Evolutionary Human Sciences},
\textbf{6}, e42.
doi:\href{https://doi.org/10.1017/ehs.2024.33}{10.1017/ehs.2024.33}.

\bibitem[\citeproctext]{ref-chatton2024causal}
Chatton, A, and Rohrer, JM (2024) The causal cookbook: Recipes for
propensity scores, g-computation, and doubly robust standardization.
\emph{Advances in Methods and Practices in Psychological Science},
\textbf{7}(1), 25152459241236149.

\bibitem[\citeproctext]{ref-duxedaz2021}
Díaz, I, Williams, N, Hoffman, KL, and Schenck, EJ (2021) Non-parametric
causal effects based on longitudinal modified treatment policies.
\emph{Journal of the American Statistical Association}.
doi:\href{https://doi.org/10.1080/01621459.2021.1955691}{10.1080/01621459.2021.1955691}.

\bibitem[\citeproctext]{ref-diaz2023lmtp}
Díaz, I, Williams, N, Hoffman, KL, and Schenck, EJ (2023) Nonparametric
causal effects based on longitudinal modified treatment policies.
\emph{Journal of the American Statistical Association},
\textbf{118}(542), 846--857.
doi:\href{https://doi.org/10.1080/01621459.2021.1955691}{10.1080/01621459.2021.1955691}.

\bibitem[\citeproctext]{ref-diaz2013assessing}
Dı́az, I, and Laan, MJ van der (2013) Assessing the causal effect of
policies: An example using stochastic interventions. \emph{The
International Journal of Biostatistics}, \textbf{9}(2), 161--174.

\bibitem[\citeproctext]{ref-fahy2017}
Fahy, KM, Lee, A, and Milne, BJ (2017) \emph{{N}ew {Z}ealand
socio-economic index 2013}, Wellington, New Zealand: Statistics New
Zealand-Tatauranga Aotearoa.

\bibitem[\citeproctext]{ref-fraser_coding_2020}
Fraser, G, Bulbulia, J, Greaves, LM, Wilson, MS, and Sibley, CG (2020)
Coding responses to an open-ended gender measure in a {N}ew {Z}ealand
national sample. \emph{The Journal of Sex Research}, \textbf{57}(8),
979--986.
doi:\href{https://doi.org/10.1080/00224499.2019.1687640}{10.1080/00224499.2019.1687640}.

\bibitem[\citeproctext]{ref-hernan2024WHATIF}
Hernan, MA, and Robins, JM (2024) \emph{Causal inference: What if?},
Taylor \& Francis. Retrieved from
\url{https://www.hsph.harvard.edu/miguel-hernan/causal-inference-book/}

\bibitem[\citeproctext]{ref-hernan2008aObservationalStudiesAnalysedLike}
Hernán, MA, Alonso, A, Logan, R, \ldots{} Robins, JM (2008)
Observational studies analyzed like randomized experiments: An
application to postmenopausal hormone therapy and coronary heart
disease. \emph{Epidemiology}, \textbf{19}(6), 766.
doi:\href{https://doi.org/10.1097/EDE.0b013e3181875e61}{10.1097/EDE.0b013e3181875e61}.

\bibitem[\citeproctext]{ref-hernan2016}
Hernán, MA, Sauer, BC, Hernández-Díaz, S, Platt, R, and Shrier, I
(2016b) Specifying a target trial prevents immortal time bias and other
self-inflicted injuries in observational analyses. \emph{Journal of
Clinical Epidemiology}, \textbf{79}, 70--75.

\bibitem[\citeproctext]{ref-hernan2016_specifying_a_target_trial}
Hernán, MA, Sauer, BC, Hernández-Díaz, S, Platt, R, and Shrier, I
(2016a) Specifying a target trial prevents immortal time bias and other
self-inflicted injuries in observational analyses. \emph{Journal of
Clinical Epidemiology}, \textbf{79}, 70--75.

\bibitem[\citeproctext]{ref-hoffman2023}
Hoffman, KL, Salazar-Barreto, D, Rudolph, KE, and Díaz, I (2023)
Introducing longitudinal modified treatment policies: A unified
framework for studying complex exposures.
doi:\href{https://doi.org/10.48550/arXiv.2304.09460}{10.48550/arXiv.2304.09460}.

\bibitem[\citeproctext]{ref-hoffman2022}
Hoffman, KL, Schenck, EJ, Satlin, MJ, \ldots{} Díaz, I (2022) Comparison
of a target trial emulation framework vs cox regression to estimate the
association of corticosteroids with COVID-19 mortality. \emph{JAMA
Network Open}, \textbf{5}(10), e2234425.
doi:\href{https://doi.org/10.1001/jamanetworkopen.2022.34425}{10.1001/jamanetworkopen.2022.34425}.

\bibitem[\citeproctext]{ref-jost_end_2006-1}
Jost, JT (2006) The end of the end of ideology. \emph{American
Psychologist}, \textbf{61}(7), 651--670.
doi:\href{https://doi.org/10.1037/0003-066X.61.7.651}{10.1037/0003-066X.61.7.651}.

\bibitem[\citeproctext]{ref-vanderlain2014discussion}
Laan, MJ van der, Luedtke, AR, and Dı́az, I (2014) Discussion of
identification, estimation and approximation of risk under interventions
that depend on the natural value of treatment using observational data,
by jessica young, miguel hern{á}n, and james robins. \emph{Epidemiologic
Methods}, \textbf{3}(1), 21--31.

\bibitem[\citeproctext]{ref-linden2020EVALUE}
Linden, A, Mathur, MB, and VanderWeele, TJ (2020) Conducting sensitivity
analysis for unmeasured confounding in observational studies using
e-values: The evalue package. \emph{The Stata Journal}, \textbf{20}(1),
162--175.

\bibitem[\citeproctext]{ref-ogburn2021}
Ogburn, EL, and Shpitser, I (2021) Causal modelling: The two cultures.
\emph{Observational Studies}, \textbf{7}(1), 179--183.
doi:\href{https://doi.org/10.1353/obs.2021.0006}{10.1353/obs.2021.0006}.

\bibitem[\citeproctext]{ref-SuperLearner2023}
Polley, E, LeDell, E, Kennedy, C, and van der Laan, M (2023)
\emph{SuperLearner: Super learner prediction}. Retrieved from
\url{https://github.com/ecpolley/SuperLearner}

\bibitem[\citeproctext]{ref-robins1986}
Robins, J (1986) A new approach to causal inference in mortality studies
with a sustained exposure period---application to control of the healthy
worker survivor effect. \emph{Mathematical Modelling}, \textbf{7}(9-12),
1393--1512.

\bibitem[\citeproctext]{ref-rohrer2022PATH}
Rohrer, JM, Hünermund, P, Arslan, RC, and Elson, M (2022) That's a lot
to process! Pitfalls of popular path models. \emph{Advances in Methods
and Practices in Psychological Science}, \textbf{5}(2).
doi:\href{https://doi.org/10.1177/25152459221095827}{10.1177/25152459221095827}.

\bibitem[\citeproctext]{ref-rohrer2023withinbetween}
Rohrer, JM, and Murayama, K (2023) These are not the effects you are
looking for: Causality and the within-/between-persons distinction in
longitudinal data analysis. \emph{Advances in Methods and Practices in
Psychological Science}, \textbf{6}(1), 25152459221140842.

\bibitem[\citeproctext]{ref-vanbuuren2018}
Van Buuren, S (2018) \emph{Flexible imputation of missing data}, CRC
press.

\bibitem[\citeproctext]{ref-vanderlaan2011}
Van Der Laan, MJ, and Rose, S (2011) \emph{Targeted Learning: Causal
Inference for Observational and Experimental Data}, New York, NY:
Springer. Retrieved from
\url{https://link.springer.com/10.1007/978-1-4419-9782-1}

\bibitem[\citeproctext]{ref-vanderlaan2018}
Van Der Laan, MJ, and Rose, S (2018) \emph{Targeted Learning in Data
Science: Causal Inference for Complex Longitudinal Studies}, Cham:
Springer International Publishing. Retrieved from
\url{http://link.springer.com/10.1007/978-3-319-65304-4}

\bibitem[\citeproctext]{ref-vanderweele2017}
VanderWeele, TJ, and Ding, P (2017) Sensitivity analysis in
observational research: Introducing the {E}-value. \emph{Annals of
Internal Medicine}, \textbf{167}(4), 268--274.
doi:\href{https://doi.org/10.7326/M16-2607}{10.7326/M16-2607}.

\bibitem[\citeproctext]{ref-vanderweele2020}
VanderWeele, TJ, Mathur, MB, and Chen, Y (2020) Outcome-wide
longitudinal designs for causal inference: A new template for empirical
studies. \emph{Statistical Science}, \textbf{35}(3), 437--466.

\bibitem[\citeproctext]{ref-verbrugge1997}
Verbrugge, LM (1997) A global disability indicator. \emph{Journal of
Aging Studies}, \textbf{11}(4), 337--362.
doi:\href{https://doi.org/10.1016/S0890-4065(97)90026-8}{10.1016/S0890-4065(97)90026-8}.

\bibitem[\citeproctext]{ref-williams2021}
Williams, NT, and Díaz, I (2021) \emph{{l}mtp: Non-parametric causal
effects of feasible interventions based on modified treatment policies}.
doi:\href{https://doi.org/10.5281/zenodo.3874931}{10.5281/zenodo.3874931}.

\bibitem[\citeproctext]{ref-young2014identification}
Young, JG, Hernán, MA, and Robins, JM (2014) Identification, estimation
and approximation of risk under interventions that depend on the natural
value of treatment using observational data. \emph{Epidemiologic
Methods}, \textbf{3}(1), 1--19.

\end{CSLReferences}

\newpage{}

\subsection{Appendix A: Measurues}\label{appendix-measures}

\paragraph{Age (waves: 1-15)}\label{age-waves-1-15}

We asked participants' age in an open-ended question (``What is your
age?'' or ``What is your date of birth'').

\paragraph{Disability}\label{disability}

We assessed disability with a one-item indicator adapted from Verbrugge
(\citeproc{ref-verbrugge1997}{1997}). It asks, ``Do you have a health
condition or disability that limits you and that has lasted for 6+
months?'' (1 = Yes, 0 = No).

\paragraph{Education Attainment (waves: 1,
4-15)}\label{education-attainment-waves-1-4-15}

Participants were asked ``What is your highest level of
qualification?''. We coded participans highest finished degree according
to the New Zealand Qualifications Authority. Ordinal-Rank 0-10 NZREG
codes (with overseas school quals coded as Level 3, and all other
ancillary categories coded as missing)
See:https://www.nzqa.govt.nz/assets/Studying-in-NZ/New-Zealand-Qualification-Framework/requirements-nzqf.pdf

\paragraph{Ethnicity (waves: 3)}\label{ethnicity-waves-3}

Based on the New Zealand Census, we asked participants ``Which ethnic
group(s) do you belong to?''. The responses were: (1) New Zealand
European; (2) Māori; (3) Samoan; (4) Cook Island Māori; (5) Tongan; (6)
Niuean; (7) Chinese; (8) Indian; (9) Other such as DUTCH, JAPANESE,
TOKELAUAN. Please state:. We coded their answers into four groups:
Maori, Pacific, Asian, and Euro (except for Time 3, which used an
open-ended measure).

\paragraph{Gender (waves: 1-15)}\label{gender-waves-1-15}

We asked participants' gender in an open-ended question: ``what is your
gender?'' or ``Are you male or female?'' (waves: 1-5). Female was coded
as 0, Male was coded as 1, and gender diverse coded as 3
(\citeproc{ref-fraser_coding_2020}{Fraser \emph{et al.} 2020}). (or 0.5
= neither female nor male)

Here, we coded all those who responded as Male as 1, and those who did
not as 0.

\paragraph{Income (waves: 1-3, 4-15)}\label{income-waves-1-3-4-15}

Participants were asked ``Please estimate your total household income
(before tax) for the year XXXX''. To stablise this indicator, we first
took the natural log of the response + 1, and then centred and
standardised the log-transformed indicator.

\paragraph{Parent (waves: 5-15)}\label{parent-waves-5-15}

Participants were asked ``If you are a parent, what is the birth date of
your eldest child?'' or ``If you are a parent, in which year was your
eldest child born?'' (waves: 10-15). Parents were coded as 1, while the
others were coded as 0.

\paragraph{Political Orientation}\label{political-orientation}

We measured participants' political orientation using a single item
adapted from Jost (\citeproc{ref-jost_end_2006-1}{2006}).

``Please rate how politically liberal versus conservative you see
yourself as being.''

(1 = Extremely Liberal to 7 = Extremely Conservative)

\paragraph{NZSEI-13 (waves: 8-15)}\label{nzsei-13-waves-8-15}

We assessed occupational prestige and status using the New Zealand
Socio-economic Index 13 (NZSEI-13) (\citeproc{ref-fahy2017}{Fahy
\emph{et al.} 2017}). This index uses the income, age, and education of
a reference group, in this case, the 2013 New Zealand census, to
calculate a score for each occupational group. Scores range from 10
(Lowest) to 90 (Highest). This list of index scores for occupational
groups was used to assign each participant a NZSEI-13 score based on
their occupation.

Participants were asked ``If you are a parent, what is the birth date of
your eldest child?''.

\paragraph{Living with Partner}\label{living-with-partner}

Participants were asked ``Do you live with your partner?'' (1 = Yes, 0 =
No).

\paragraph{Living in an Urban Area (waves:
1-15)}\label{living-in-an-urban-area-waves-1-15}

We coded whether they are living in an urban or rural area (1 = Urban, 0
= Rural) based on the addresses provided.

We coded whether they are living in an urban or rural area (1 = Urban, 0
= Rural) based on the addresses provided.

\paragraph{NZ Deprivation Index (waves:
1-15)}\label{nz-deprivation-index-waves-1-15}

We used the NZ Deprivation Index to assign each participant a score
based on where they live (\citeproc{ref-atkinson2019}{Atkinson \emph{et
al.} 2019}). This score combines data such as income, home ownership,
employment, qualifications, family structure, housing, and access to
transport and communication for an area into one deprivation score.

\subsection{Appendix B. Baseline Demographic Statistics: Three Wave
Study}\label{appendix-demographics}

\begin{longtable}[]{@{}ll@{}}

\caption{\label{tbl-table-demography-code}Baseline demographic
statistics: three wave study}

\tabularnewline

\toprule\noalign{}
Variable & Summary Statistics (N = 1,012) \\
\midrule\noalign{}
\endhead
\bottomrule\noalign{}
\endlastfoot
\textbf{Age} & NA \\
Mean (SD) & 53 (11) \\
Min, Max & 22, 90 \\
Q1, Q3 & 46, 61 \\
\textbf{Education Level Coarsen} & NA \\
1 & 12 (1.2\%) \\
2 & 288 (29\%) \\
3 & 120 (12\%) \\
4 & 288 (29\%) \\
5 & 133 (13\%) \\
6 & 122 (12\%) \\
7 & 44 (4.4\%) \\
Unknown & 5 \\
\textbf{Eth Cat} & NA \\
1 & 874 (87\%) \\
2 & 73 (7.2\%) \\
3 & 24 (2.4\%) \\
4 & 37 (3.7\%) \\
Unknown & 4 \\
\textbf{Hlth Disability} & NA \\
0 & 797 (80\%) \\
1 & 197 (20\%) \\
Unknown & 18 \\
\textbf{Hours Housework log} & NA \\
Mean (SD) & 2.11 (0.77) \\
Min, Max & 0.00, 4.04 \\
Q1, Q3 & 1.61, 2.71 \\
Unknown & 26 \\
\textbf{Hours Work log} & NA \\
Mean (SD) & 2.73 (1.55) \\
Min, Max & 0.00, 4.39 \\
Q1, Q3 & 1.95, 3.78 \\
Unknown & 26 \\
\textbf{Kessler Latent Anxiety} & NA \\
Mean (SD) & 1.08 (0.72) \\
Min, Max & 0.00, 3.67 \\
Q1, Q3 & 0.67, 1.67 \\
Unknown & 4 \\
\textbf{Kessler Latent Depression} & NA \\
Mean (SD) & 0.45 (0.66) \\
Min, Max & 0.00, 3.33 \\
Q1, Q3 & 0.00, 0.67 \\
Unknown & 4 \\
\textbf{Kessler6 Sum} & NA \\
Mean (SD) & 5 (4) \\
Min, Max & 0, 19 \\
Q1, Q3 & 2, 6 \\
Unknown & 4 \\
\textbf{Male} & 492 (49\%) \\
\textbf{Nz Dep2018} & NA \\
Mean (SD) & 4.33 (2.55) \\
Min, Max & 1.00, 10.00 \\
Q1, Q3 & 2.00, 6.00 \\
Unknown & 2 \\
\textbf{Parent} & NA \\
0 & 224 (22\%) \\
1 & 788 (78\%) \\
\textbf{Political Conservative} & NA \\
1 & 56 (5.8\%) \\
2 & 233 (24\%) \\
3 & 221 (23\%) \\
4 & 224 (23\%) \\
5 & 144 (15\%) \\
6 & 72 (7.4\%) \\
7 & 19 (2.0\%) \\
Unknown & 43 \\
\textbf{Rel Length Year} & NA \\
Mean (SD) & 23 (13) \\
Min, Max & 0, 61 \\
Q1, Q3 & 13, 34 \\
Unknown & 18 \\
\textbf{Rel Status l3} & NA \\
Dating/Girl- or Boy- Friend & 2 (0.2\%) \\
Separated but living together & 1 (\textless0.1\%) \\
Partner/Committed/Relationship & 26 (2.6\%) \\
De-Facto & 121 (12\%) \\
Civil Partnership & 10 (1.0\%) \\
Married & 850 (84\%) \\
Unknown & 2 \\
\textbf{Religion Church Coarsen} & NA \\
zero & 814 (82\%) \\
one & 26 (2.6\%) \\
less\_four & 56 (5.6\%) \\
four\_up & 97 (9.8\%) \\
Unknown & 19 \\
\textbf{Urban} & 820 (81\%) \\
Unknown & 2 \\

\end{longtable}

\newpage{}

\subsection{Appendix C. Baseline Demographic Statistics: Six Wave
Study}\label{appendix-demographics-long}

\begin{longtable}[]{@{}ll@{}}

\caption{\label{tbl-table-demography-five-code-long}Baseline demographic
statistics: six wave study}

\tabularnewline

\toprule\noalign{}
Variable & Summary Statistics (N = 1,080) \\
\midrule\noalign{}
\endhead
\bottomrule\noalign{}
\endlastfoot
\textbf{Age} & NA \\
Mean (SD) & 50 (12) \\
Min, Max & 19, 90 \\
Q1, Q3 & 42, 60 \\
\textbf{Education Level Coarsen} & NA \\
1 & 24 (1.3\%) \\
2 & 561 (31\%) \\
3 & 220 (12\%) \\
4 & 506 (28\%) \\
5 & 231 (13\%) \\
6 & 186 (10\%) \\
7 & 62 (3.5\%) \\
Unknown & 42 \\
\textbf{Eth Cat} & NA \\
1 & 1,519 (84\%) \\
2 & 165 (9.1\%) \\
3 & 36 (2.0\%) \\
4 & 88 (4.9\%) \\
Unknown & 24 \\
\textbf{Hlth Disability} & 343 (19\%) \\
Unknown & 37 \\
\textbf{Hours Housework log} & NA \\
Mean (SD) & 2.12 (0.77) \\
Min, Max & 0.00, 4.72 \\
Q1, Q3 & 1.61, 2.71 \\
Unknown & 50 \\
\textbf{Hours Work log} & NA \\
Mean (SD) & 2.81 (1.50) \\
Min, Max & 0.00, 4.39 \\
Q1, Q3 & 2.40, 3.83 \\
Unknown & 50 \\
\textbf{Kessler Latent Anxiety} & NA \\
Mean (SD) & 1.14 (0.74) \\
Min, Max & 0.00, 4.00 \\
Q1, Q3 & 0.67, 1.67 \\
Unknown & 11 \\
\textbf{Kessler Latent Depression} & NA \\
Mean (SD) & 0.49 (0.68) \\
Min, Max & 0.00, 4.00 \\
Q1, Q3 & 0.00, 0.67 \\
Unknown & 11 \\
\textbf{Kessler6 Sum} & NA \\
Mean (SD) & 5 (4) \\
Min, Max & 0, 24 \\
Q1, Q3 & 2, 7 \\
Unknown & 11 \\
\textbf{Male} & 888 (49\%) \\
Unknown & 4 \\
\textbf{Nz Dep2018} & NA \\
Mean (SD) & 4.45 (2.59) \\
Min, Max & 1.00, 10.00 \\
Q1, Q3 & 2.00, 6.00 \\
Unknown & 2 \\
\textbf{Parent} & 1,424 (78\%) \\
Unknown & 12 \\
\textbf{Political Conservative} & NA \\
1 & 98 (5.6\%) \\
2 & 378 (22\%) \\
3 & 376 (22\%) \\
4 & 458 (26\%) \\
5 & 270 (16\%) \\
6 & 125 (7.2\%) \\
7 & 32 (1.8\%) \\
Unknown & 95 \\
\textbf{Rel Length Year} & NA \\
Mean (SD) & 22 (13) \\
Min, Max & 0, 61 \\
Q1, Q3 & 11, 32 \\
Unknown & 64 \\
\textbf{Rel Status l3} & NA \\
Dating/Girl- or Boy- Friend & 4 (0.2\%) \\
Divorced/Separated & 1 (\textless0.1\%) \\
Separated but living together & 1 (\textless0.1\%) \\
Single & 25 (1.4\%) \\
Partner/Committed/Relationship & 52 (2.8\%) \\
De-Facto & 272 (15\%) \\
Civil Partnership & 10 (0.5\%) \\
Engaged & 24 (1.3\%) \\
Married & 1,435 (79\%) \\
Polyamorous & 1 (\textless0.1\%) \\
Unknown & 7 \\
\textbf{Religion Church Coarsen} & NA \\
zero & 1,480 (83\%) \\
one & 55 (3.1\%) \\
less\_four & 80 (4.5\%) \\
four\_up & 166 (9.3\%) \\
Unknown & 51 \\
\textbf{Urban} & 1,490 (81\%) \\
Unknown & 2 \\

\end{longtable}

\newpage{}

\subsection{Appendix D: Baseline and Treatment Wave Exposure Statistics
Three Wave Study}\label{appendix-exposures}

\begin{table}

\caption{\label{tbl-table-exposures-code-antagonism}Exposures at
baseline and treatment wave: Antagonism (three wave study)}

\centering{

\centering\begingroup\fontsize{9}{11}\selectfont

\resizebox{\ifdim\width>\linewidth\linewidth\else\width\fi}{!}{
\begin{tabular}[t]{lll}
\toprule
**Exposure Variable: Antagonism by Wave** & **2018**  
N = 1,012 & **2019**  
N = 1,012\\
\midrule
\_\_Antagonism\_\_ & 2.17 (1.67, 2.83) & 2.17 (1.67, 2.67)\\
Unknown & 0 & 0\\
\bottomrule
\end{tabular}}
\endgroup{}

}

\end{table}%

\begin{table}

\caption{\label{tbl-table-exposures-code-disinhibition}Exposures at
baseline and treatment wave: Disinhibition (three wave study)}

\centering{

\centering\begingroup\fontsize{9}{11}\selectfont

\resizebox{\ifdim\width>\linewidth\linewidth\else\width\fi}{!}{
\begin{tabular}[t]{lll}
\toprule
**Exposure Variable: Disinhibition by Wave** & **2018**  
N = 1,012 & **2019**  
N = 1,012\\
\midrule
\_\_Disinhibition\_\_ & 2.67 (2.17, 3.17) & 2.67 (2.17, 3.17)\\
Unknown & 0 & 0\\
\bottomrule
\end{tabular}}
\endgroup{}

}

\end{table}%

\begin{table}

\caption{\label{tbl-table-exposures-code-emotional-stability}Exposures
at baseline and treatment wave: Emotional Stability (three wave study)}

\centering{

\centering\begingroup\fontsize{9}{11}\selectfont

\resizebox{\ifdim\width>\linewidth\linewidth\else\width\fi}{!}{
\begin{tabular}[t]{lll}
\toprule
**Exposure Variable: Emotional Stability by Wave** & **2018**  
N = 1,012 & **2019**  
N = 1,012\\
\midrule
\_\_Emotional Stability\_\_ & 4.17 (3.67, 4.50) & 4.00 (3.50, 4.50)\\
Unknown & 0 & 0\\
\bottomrule
\end{tabular}}
\endgroup{}

}

\end{table}%

\begin{table}

\caption{\label{tbl-table-exposures-code-narcissism}Exposures at
baseline and treatment wave: Narcissism (three wave study)}

\centering{

\centering\begingroup\fontsize{9}{11}\selectfont

\resizebox{\ifdim\width>\linewidth\linewidth\else\width\fi}{!}{
\begin{tabular}[t]{lll}
\toprule
**Exposure Variable: Narcissism by Wave** & **2018**  
N = 1,012 & **2019**  
N = 1,012\\
\midrule
\_\_Narcissism\_\_ & 3.00 (2.67, 3.50) & 3.17 (2.67, 3.50)\\
Unknown & 0 & 0\\
\bottomrule
\end{tabular}}
\endgroup{}

}

\end{table}%

\begin{table}

\caption{\label{tbl-table-exposures-code-psychopathy_combined}Exposures
at baseline and treatment wave: Psychopathy (three wave study)}

\centering{

\centering\begingroup\fontsize{9}{11}\selectfont

\resizebox{\ifdim\width>\linewidth\linewidth\else\width\fi}{!}{
\begin{tabular}[t]{lll}
\toprule
**Exposure Variable: Psychopathy Combined by Wave** & **2018**  
N = 1,012 & **2019**  
N = 1,012\\
\midrule
\_\_Psychopathy Combined\_\_ & 3.00 (2.75, 3.29) & 3.00 (2.75, 3.29)\\
Unknown & 0 & 0\\
\bottomrule
\end{tabular}}
\endgroup{}

}

\end{table}%

\newpage{}

\subsection{Appendix E: Baseline and Treatment Wave Exposure Statistics
Six Wave Study}\label{appendix-exposures-five}

\begin{table}

\caption{\label{tbl-table-exposures-code-antagonism-long}Exposures at
baseline and treatment wave: Antagonism (six wave study)}

\centering{

\centering\begingroup\fontsize{9}{11}\selectfont

\resizebox{\ifdim\width>\linewidth\linewidth\else\width\fi}{!}{
\begin{tabular}[t]{llllll}
\toprule
**Exposure Variable: Antagonism by Wave** & **2018**  
N = 1,832 & **2019**  
N = 1,832 & **2020**  
N = 1,832 & **2021**  
N = 1,832 & **2022**  
N = 1,832\\
\midrule
\_\_Antagonism\_\_ & 2.17 (1.67, 2.83) & 2.17 (1.67, 2.67) & 2.17 (1.67, 2.67) & 2.17 (1.67, 2.67) & 2.17 (1.67, 2.67)\\
Unknown & 0 & 501 & 430 & 759 & 869\\
\bottomrule
\end{tabular}}
\endgroup{}

}

\end{table}%

\begin{table}

\caption{\label{tbl-table-exposures-code-disinhibition-long}Exposures at
baseline and treatment wave: Disinhibition (six wave study)}

\centering{

\centering\begingroup\fontsize{9}{11}\selectfont

\resizebox{\ifdim\width>\linewidth\linewidth\else\width\fi}{!}{
\begin{tabular}[t]{llllll}
\toprule
**Exposure Variable: Disinhibition by Wave** & **2018**  
N = 1,832 & **2019**  
N = 1,832 & **2020**  
N = 1,832 & **2021**  
N = 1,832 & **2022**  
N = 1,832\\
\midrule
\_\_Disinhibition\_\_ & 2.67 (2.17, 3.33) & 2.67 (2.17, 3.17) & 2.67 (2.17, 3.17) & 2.67 (2.17, 3.33) & 2.67 (2.17, 3.17)\\
Unknown & 1 & 502 & 430 & 760 & 869\\
\bottomrule
\end{tabular}}
\endgroup{}

}

\end{table}%

\begin{table}

\caption{\label{tbl-table-exposures-code-emotional-stability-long}Exposures
at baseline and treatment wave: Emotional Stability (six wave study)}

\centering{

\centering\begingroup\fontsize{9}{11}\selectfont

\resizebox{\ifdim\width>\linewidth\linewidth\else\width\fi}{!}{
\begin{tabular}[t]{llllll}
\toprule
**Exposure Variable: Emotional Stability by Wave** & **2018**  
N = 1,832 & **2019**  
N = 1,832 & **2020**  
N = 1,832 & **2021**  
N = 1,832 & **2022**  
N = 1,832\\
\midrule
\_\_Emotional Stability\_\_ & 4.17 (3.67, 4.50) & 4.00 (3.50, 4.50) & 3.83 (3.50, 4.33) & 4.00 (3.50, 4.50) & 4.00 (3.50, 4.50)\\
Unknown & 0 & 501 & 430 & 759 & 869\\
\bottomrule
\end{tabular}}
\endgroup{}

}

\end{table}%

\begin{table}

\caption{\label{tbl-table-exposures-code-narcissism-long}Exposures at
baseline and treatment wave: Narcissism (six wave study)}

\centering{

\centering\begingroup\fontsize{9}{11}\selectfont

\resizebox{\ifdim\width>\linewidth\linewidth\else\width\fi}{!}{
\begin{tabular}[t]{llllll}
\toprule
**Exposure Variable: Narcissism by Wave** & **2018**  
N = 1,832 & **2019**  
N = 1,832 & **2020**  
N = 1,832 & **2021**  
N = 1,832 & **2022**  
N = 1,832\\
\midrule
\_\_Narcissism\_\_ & 3.17 (2.67, 3.67) & 3.17 (2.67, 3.50) & 3.17 (2.67, 3.67) & 3.00 (2.50, 3.50) & 3.00 (2.00, 3.67)\\
Unknown & 1 & 502 & 430 & 759 & 873\\
\bottomrule
\end{tabular}}
\endgroup{}

}

\end{table}%

\begin{table}

\caption{\label{tbl-table-exposures-code-psychopathy_combined-long}Exposures
at baseline and treatment wave: Psychopathy (six wave study)}

\centering{

\centering\begingroup\fontsize{9}{11}\selectfont

\resizebox{\ifdim\width>\linewidth\linewidth\else\width\fi}{!}{
\begin{tabular}[t]{llllll}
\toprule
**Exposure Variable: Psychopathy Combined by Wave** & **2018**  
N = 1,832 & **2019**  
N = 1,832 & **2020**  
N = 1,832 & **2021**  
N = 1,832 & **2022**  
N = 1,832\\
\midrule
\_\_Psychopathy Combined\_\_ & 3.04 (2.79, 3.34) & 3.00 (2.75, 3.29) & 2.96 (2.75, 3.25) & 3.00 (2.71, 3.25) & 2.96 (2.67, 3.25)\\
Unknown & 1 & 502 & 430 & 760 & 873\\
\bottomrule
\end{tabular}}
\endgroup{}

}

\end{table}%

\newpage{}

\subsection{Appendix F: Baseline and End of Study Outcome Statistics
Three Wave Study}\label{appendix-outcomes}

\begin{table}

\caption{\label{tbl-table-demographic_vars-code}Outcomes at baseline and
end-of-study: (three wave study)}

\centering{

[!h]
\centering\begingroup\fontsize{8}{10}\selectfont

\begin{tabular}[t]{lll}
\toprule
**Outcome Variables by Wave** & **2018**  
N = 1,012 & **2020**  
N = 1,012\\
\midrule
\_\_Conflict in Relationship\_\_ & NA & NA\\
1 & 105 (11\%) & 86 (9.7\%)\\
2 & 474 (49\%) & 410 (46\%)\\
3 & 158 (16\%) & 144 (16\%)\\
4 & 109 (11\%) & 114 (13\%)\\
\addlinespace
5 & 74 (7.7\%) & 103 (12\%)\\
6 & 37 (3.8\%) & 25 (2.8\%)\\
7 & 7 (0.7\%) & 4 (0.5\%)\\
Unknown & 48 & 126\\
\_\_Kessler Latent Anxiety\_\_ & 1.00 (0.67, 1.67) & 1.00 (0.67, 1.67)\\
\addlinespace
Unknown & 4 & 105\\
\_\_Kessler Latent Depression\_\_ & 0.17 (0.00, 0.67) & 0.33 (0.00, 0.67)\\
Unknown & 4 & 105\\
\_\_Kessler6 Sum\_\_ & 4 (2, 6) & 4 (2, 6)\\
Unknown & 4 & 105\\
\addlinespace
\_\_Lifesat\_\_ & 6.00 (5.00, 6.50) & 5.50 (5.00, 6.00)\\
Unknown & 4 & 112\\
\_\_Pwi\_\_ & 7.75 (6.75, 8.50) & 7.75 (6.75, 8.50)\\
Unknown & 0 & 105\\
\_\_Sat Relationship\_\_ & NA & NA\\
\addlinespace
1 & 2 (0.2\%) & 3 (0.3\%)\\
2 & 8 (0.8\%) & 9 (1.0\%)\\
3 & 18 (1.9\%) & 18 (2.0\%)\\
4 & 37 (3.8\%) & 30 (3.4\%)\\
5 & 83 (8.6\%) & 103 (12\%)\\
\addlinespace
6 & 349 (36\%) & 324 (37\%)\\
7 & 473 (49\%) & 398 (45\%)\\
Unknown & 42 & 127\\
\_\_Self Esteem\_\_ & 5.67 (4.67, 6.33) & 5.33 (4.33, 6.00)\\
Unknown & 4 & 105\\
\bottomrule
\end{tabular}
\endgroup{}

}

\end{table}%

\newpage{}

\subsection{Appendix E: Baseline and End of Study Outcome Statistics Six
Wave Study}\label{appendix-outcomes-long}

\begin{table}

\caption{\label{tbl-table-demographic_vars-long}Outcomes at baseline and
end-of-study: six wave study}

\centering{

[!h]
\centering\begingroup\fontsize{8}{10}\selectfont

\begin{tabular}[t]{lll}
\toprule
**Outcome Variables by Wave** & **2018**  
N = 1,832 & **2023**  
N = 1,832\\
\midrule
\_\_Conflict in Relationship\_\_ & NA & NA\\
1 & 169 (9.9\%) & 74 (9.4\%)\\
2 & 795 (46\%) & 372 (47\%)\\
3 & 269 (16\%) & 125 (16\%)\\
4 & 232 (14\%) & 110 (14\%)\\
\addlinespace
5 & 156 (9.1\%) & 79 (10.0\%)\\
6 & 74 (4.3\%) & 29 (3.7\%)\\
7 & 20 (1.2\%) & 2 (0.3\%)\\
Unknown & 117 & 1,041\\
\_\_Kessler Latent Anxiety\_\_ & 1.00 (0.67, 1.67) & 1.00 (0.67, 1.67)\\
\addlinespace
Unknown & 11 & 980\\
\_\_Kessler Latent Depression\_\_ & 0.33 (0.00, 0.67) & 0.00 (0.00, 0.67)\\
Unknown & 11 & 980\\
\_\_Kessler6 Sum\_\_ & 4 (2, 7) & 4 (2, 6)\\
Unknown & 11 & 980\\
\addlinespace
\_\_Lifesat\_\_ & 5.50 (5.00, 6.00) & 5.50 (5.00, 6.00)\\
Unknown & 9 & 987\\
\_\_Pwi\_\_ & 7.75 (6.75, 8.50) & 7.50 (6.50, 8.50)\\
Unknown & 0 & 981\\
\_\_Sat Relationship\_\_ & NA & NA\\
\addlinespace
1 & 6 (0.3\%) & 8 (1.0\%)\\
2 & 16 (0.9\%) & 7 (0.9\%)\\
3 & 33 (1.9\%) & 20 (2.5\%)\\
4 & 66 (3.8\%) & 32 (4.1\%)\\
5 & 173 (10\%) & 107 (14\%)\\
\addlinespace
6 & 595 (35\%) & 290 (37\%)\\
7 & 831 (48\%) & 324 (41\%)\\
Unknown & 112 & 1,044\\
\_\_Self Esteem\_\_ & 5.67 (4.67, 6.33) & 5.67 (4.67, 6.33)\\
Unknown & 9 & 979\\
\bottomrule
\end{tabular}
\endgroup{}

}

\end{table}%

\newpage{}

\subsection{Appendix G: Transition Matrices: Three wave
study}\label{appendix-g-transition-matrices-three-wave-study}

\subsubsection{Transition Table: Antagonism - Three wave
study}\label{transition-table-antagonism---three-wave-study}

\begin{longtable}[]{@{}ccccccc@{}}
\caption{Transition matrix for change in Antagonism (three-wave
study)}\label{tbl-table-transition-antagonism}\tabularnewline
\toprule\noalign{}
From & State 1 & State 2 & State 3 & State 4 & State 5 & State 6 \\
\midrule\noalign{}
\endfirsthead
\toprule\noalign{}
From & State 1 & State 2 & State 3 & State 4 & State 5 & State 6 \\
\midrule\noalign{}
\endhead
\bottomrule\noalign{}
\endlastfoot
State 1 & \textbf{81} & 59 & 3 & 0 & 0 & 0 \\
State 2 & 51 & \textbf{395} & 95 & 4 & 0 & 0 \\
State 3 & 6 & 90 & \textbf{110} & 30 & 1 & 0 \\
State 4 & 1 & 14 & 25 & \textbf{38} & 4 & 2 \\
State 5 & 0 & 0 & 1 & 2 & \textbf{0} & 0 \\
State 6 & 0 & 0 & 0 & 0 & 0 & \textbf{0} \\
\end{longtable}

Table~\ref{tbl-table-transition-antagonism}: This transition matrix
captures shifts in states across Antagonism from the baseline to
exposure wave in the three-wave panel study. Each cell in the matrix
represents the count of individuals transitioning from one state to
another. The rows correspond to the treatment at baseline (From), and
the columns correspond to the state at the following wave (To).
\textbf{Diagonal entries} (in \textbf{bold}) correspond to the number of
individuals who remained in their initial state across both waves.
\textbf{Off-diagonal entries} correspond to the transitions of
individuals from their baseline state to a different state in the
treatment wave. A higher number on the diagonal relative to the
off-diagonal entries in the same row indicates greater stability in a
state. Conversely, higher off-diagonal numbers suggest more frequent
shifts from the baseline state to other states.

\subsubsection{Transition Table: Disinhibition - Three wave
study}\label{transition-table-disinhibition---three-wave-study}

\begin{longtable}[]{@{}ccccccc@{}}
\caption{Transition matrix for change in Disinhibition (three-wave
study)}\label{tbl-table-transition-disinhibition}\tabularnewline
\toprule\noalign{}
From & State 1 & State 2 & State 3 & State 4 & State 5 & State 6 \\
\midrule\noalign{}
\endfirsthead
\toprule\noalign{}
From & State 1 & State 2 & State 3 & State 4 & State 5 & State 6 \\
\midrule\noalign{}
\endhead
\bottomrule\noalign{}
\endlastfoot
State 1 & \textbf{13} & 20 & 2 & 0 & 0 & 0 \\
State 2 & 20 & \textbf{310} & 99 & 6 & 0 & 0 \\
State 3 & 3 & 94 & \textbf{186} & 60 & 3 & 0 \\
State 4 & 1 & 8 & 68 & \textbf{96} & 6 & 0 \\
State 5 & 0 & 0 & 0 & 11 & \textbf{4} & 1 \\
State 6 & 0 & 1 & 0 & 0 & 0 & \textbf{0} \\
\end{longtable}

Table~\ref{tbl-table-transition-disinhibition}: This transition matrix
captures shifts in states across Disinhibition from the baseline to
exposure wave in the three-wave panel study. Each cell in the matrix
represents the count of individuals transitioning from one state to
another. The rows correspond to the treatment at baseline (From), and
the columns correspond to the state at the following wave (To).
\textbf{Diagonal entries} (in \textbf{bold}) correspond to the number of
individuals who remained in their initial state across both waves.
\textbf{Off-diagonal entries} correspond to the transitions of
individuals from their baseline state to a different state in the
treatment wave. A higher number on the diagonal relative to the
off-diagonal entries in the same row indicates greater stability in a
state. Conversely, higher off-diagonal numbers suggest more frequent
shifts from the baseline state to other states.

\subsubsection{Transition Table: Emotional Stability - Three wave
study}\label{transition-table-emotional-stability---three-wave-study}

\begin{longtable}[]{@{}
  >{\centering\arraybackslash}p{(\linewidth - 14\tabcolsep) * \real{0.1250}}
  >{\centering\arraybackslash}p{(\linewidth - 14\tabcolsep) * \real{0.1250}}
  >{\centering\arraybackslash}p{(\linewidth - 14\tabcolsep) * \real{0.1250}}
  >{\centering\arraybackslash}p{(\linewidth - 14\tabcolsep) * \real{0.1250}}
  >{\centering\arraybackslash}p{(\linewidth - 14\tabcolsep) * \real{0.1250}}
  >{\centering\arraybackslash}p{(\linewidth - 14\tabcolsep) * \real{0.1250}}
  >{\centering\arraybackslash}p{(\linewidth - 14\tabcolsep) * \real{0.1250}}
  >{\centering\arraybackslash}p{(\linewidth - 14\tabcolsep) * \real{0.1250}}@{}}
\caption{Transition matrix for change in Emotional Stability (three-wave
study)}\label{tbl-table-transition-emotional-stability}\tabularnewline
\toprule\noalign{}
\begin{minipage}[b]{\linewidth}\centering
From
\end{minipage} & \begin{minipage}[b]{\linewidth}\centering
State 1
\end{minipage} & \begin{minipage}[b]{\linewidth}\centering
State 2
\end{minipage} & \begin{minipage}[b]{\linewidth}\centering
State 3
\end{minipage} & \begin{minipage}[b]{\linewidth}\centering
State 4
\end{minipage} & \begin{minipage}[b]{\linewidth}\centering
State 5
\end{minipage} & \begin{minipage}[b]{\linewidth}\centering
State 6
\end{minipage} & \begin{minipage}[b]{\linewidth}\centering
State 7
\end{minipage} \\
\midrule\noalign{}
\endfirsthead
\toprule\noalign{}
\begin{minipage}[b]{\linewidth}\centering
From
\end{minipage} & \begin{minipage}[b]{\linewidth}\centering
State 1
\end{minipage} & \begin{minipage}[b]{\linewidth}\centering
State 2
\end{minipage} & \begin{minipage}[b]{\linewidth}\centering
State 3
\end{minipage} & \begin{minipage}[b]{\linewidth}\centering
State 4
\end{minipage} & \begin{minipage}[b]{\linewidth}\centering
State 5
\end{minipage} & \begin{minipage}[b]{\linewidth}\centering
State 6
\end{minipage} & \begin{minipage}[b]{\linewidth}\centering
State 7
\end{minipage} \\
\midrule\noalign{}
\endhead
\bottomrule\noalign{}
\endlastfoot
State 1 & \textbf{0} & 0 & 0 & 1 & 0 & 0 & 0 \\
State 2 & 0 & \textbf{9} & 9 & 3 & 0 & 0 & 0 \\
State 3 & 1 & 12 & \textbf{53} & 64 & 3 & 0 & 0 \\
State 4 & 0 & 3 & 91 & \textbf{428} & 99 & 3 & 1 \\
State 5 & 0 & 0 & 10 & 117 & \textbf{75} & 14 & 0 \\
State 6 & 0 & 0 & 0 & 4 & 9 & \textbf{3} & 0 \\
State 7 & 0 & 0 & 0 & 0 & 0 & 0 & \textbf{0} \\
\end{longtable}

Table~\ref{tbl-table-transition-emotional-stability}: This transition
matrix captures shifts in states across Emotional Stability from the
baseline to exposure wave in the three-wave panel study. Each cell in
the matrix represents the count of individuals transitioning from one
state to another. The rows correspond to the treatment at baseline
(From), and the columns correspond to the state at the following wave
(To). \textbf{Diagonal entries} (in \textbf{bold}) correspond to the
number of individuals who remained in their initial state across both
waves. \textbf{Off-diagonal entries} correspond to the transitions of
individuals from their baseline state to a different state in the
treatment wave. A higher number on the diagonal relative to the
off-diagonal entries in the same row indicates greater stability in a
state. Conversely, higher off-diagonal numbers suggest more frequent
shifts from the baseline state to other states.

\subsubsection{Transition Table: Narcissism - Three wave
study}\label{transition-table-narcissism---three-wave-study}

\begin{longtable}[]{@{}ccccccc@{}}
\caption{Transition matrix for change in Narcissism (three-wave
study)}\label{tbl-table-transition-narcissism}\tabularnewline
\toprule\noalign{}
From & State 1 & State 2 & State 3 & State 4 & State 5 & State 6 \\
\midrule\noalign{}
\endfirsthead
\toprule\noalign{}
From & State 1 & State 2 & State 3 & State 4 & State 5 & State 6 \\
\midrule\noalign{}
\endhead
\bottomrule\noalign{}
\endlastfoot
State 1 & \textbf{1} & 5 & 0 & 1 & 0 & 0 \\
State 2 & 1 & \textbf{126} & 81 & 16 & 1 & 0 \\
State 3 & 0 & 89 & \textbf{273} & 85 & 0 & 0 \\
State 4 & 0 & 9 & 114 & \textbf{176} & 8 & 1 \\
State 5 & 0 & 0 & 3 & 14 & \textbf{6} & 0 \\
State 6 & 0 & 0 & 0 & 1 & 0 & \textbf{1} \\
\end{longtable}

Table~\ref{tbl-table-transition-narcissism}: This transition matrix
captures shifts in states across Narcissism from the baseline to
exposure wave in the three-wave panel study. Each cell in the matrix
represents the count of individuals transitioning from one state to
another. The rows correspond to the treatment at baseline (From), and
the columns correspond to the state at the following wave (To).
\textbf{Diagonal entries} (in \textbf{bold}) correspond to the number of
individuals who remained in their initial state across both waves.
\textbf{Off-diagonal entries} correspond to the transitions of
individuals from their baseline state to a different state in the
treatment wave. A higher number on the diagonal relative to the
off-diagonal entries in the same row indicates greater stability in a
state. Conversely, higher off-diagonal numbers suggest more frequent
shifts from the baseline state to other states.

\subsubsection{Transition Table: Psychopathy Combined - Three wave
study}\label{transition-table-psychopathy-combined---three-wave-study}

\begin{longtable}[]{@{}ccccc@{}}
\caption{Transition matrix for change in Psychopathy Combined (six-wave
study)}\label{tbl-table-transition-psychopathy-combined}\tabularnewline
\toprule\noalign{}
From & State 2 & State 3 & State 4 & State 5 \\
\midrule\noalign{}
\endfirsthead
\toprule\noalign{}
From & State 2 & State 3 & State 4 & State 5 \\
\midrule\noalign{}
\endhead
\bottomrule\noalign{}
\endlastfoot
State 2 & \textbf{42} & 49 & 0 & 0 \\
State 3 & 67 & \textbf{654} & 56 & 0 \\
State 4 & 0 & 68 & \textbf{73} & 1 \\
State 5 & 0 & 0 & 1 & \textbf{1} \\
\end{longtable}

Table~\ref{tbl-table-transition-psychopathy-combined}: This transition
matrix captures shifts in states across Emotional Stability from the
baseline to exposure wave in the three-wave panel study. Each cell in
the matrix represents the count of individuals transitioning from one
state to another. The rows correspond to the treatment at baseline
(From), and the columns correspond to the state at the following wave
(To). \textbf{Diagonal entries} (in \textbf{bold}) correspond to the
number of individuals who remained in their initial state across both
waves. \textbf{Off-diagonal entries} correspond to the transitions of
individuals from their baseline state to a different state in the
treatment wave. A higher number on the diagonal relative to the
off-diagonal entries in the same row indicates greater stability in a
state. Conversely, higher off-diagonal numbers suggest more frequent
shifts from the baseline state to other states.

\subsection{Appendix H: Transition Matrices: Six wave
study}\label{appendix-h-transition-matrices-six-wave-study}

\subsubsection{Transition Table: Antagonism - Six wave
study}\label{transition-table-antagonism---six-wave-study}

\begin{longtable}[]{@{}ccccccc@{}}
\caption{Transition matrix for change in Antagonism (six-wave
study)}\label{tbl-table-transition-antagonism-long}\tabularnewline
\toprule\noalign{}
From & State 1 & State 2 & State 3 & State 4 & State 5 & State 6 \\
\midrule\noalign{}
\endfirsthead
\toprule\noalign{}
From & State 1 & State 2 & State 3 & State 4 & State 5 & State 6 \\
\midrule\noalign{}
\endhead
\bottomrule\noalign{}
\endlastfoot
State 1 & \textbf{321} & 275 & 6 & 3 & 0 & 0 \\
State 2 & 249 & \textbf{1734} & 381 & 27 & 2 & 1 \\
State 3 & 20 & 382 & \textbf{428} & 130 & 3 & 0 \\
State 4 & 2 & 38 & 127 & \textbf{143} & 13 & 2 \\
State 5 & 1 & 1 & 3 & 13 & \textbf{4} & 1 \\
State 6 & 0 & 0 & 0 & 0 & 1 & \textbf{1} \\
\end{longtable}

Table~\ref{tbl-table-transition-antagonism-long}: This transition matrix
captures shifts in states across Antagonism exposures in the six-wave
panel study. Each cell in the matrix represents the count of individuals
transitioning from one state to another. The rows correspond to the
treatment at baseline (From), and the columns correspond to the state at
the following wave (To). \textbf{Diagonal entries} (in \textbf{bold})
correspond to the number of individuals who remained in their initial
state across both waves. \textbf{Off-diagonal entries} correspond to the
transitions of individuals from their baseline state to a different
state in the treatment wave. A higher number on the diagonal relative to
the off-diagonal entries in the same row indicates greater stability in
a state. Conversely, higher off-diagonal numbers suggest more frequent
shifts from the baseline state to other states.

\subsubsection{Transition Table: Disinhibition - Six wave
study}\label{transition-table-disinhibition---six-wave-study}

\begin{longtable}[]{@{}ccccccc@{}}
\caption{Transition matrix for change in Disinhibition (six-wave
study)}\label{tbl-table-transition-disinhibition-long}\tabularnewline
\toprule\noalign{}
From & State 1 & State 2 & State 3 & State 4 & State 5 & State 6 \\
\midrule\noalign{}
\endfirsthead
\toprule\noalign{}
From & State 1 & State 2 & State 3 & State 4 & State 5 & State 6 \\
\midrule\noalign{}
\endhead
\bottomrule\noalign{}
\endlastfoot
State 1 & \textbf{70} & 79 & 6 & 2 & 0 & 0 \\
State 2 & 88 & \textbf{1283} & 430 & 48 & 2 & 0 \\
State 3 & 10 & 415 & \textbf{813} & 256 & 6 & 0 \\
State 4 & 3 & 44 & 257 & \textbf{407} & 30 & 1 \\
State 5 & 0 & 2 & 7 & 31 & \textbf{14} & 2 \\
State 6 & 0 & 1 & 0 & 0 & 2 & \textbf{1} \\
\end{longtable}

Table~\ref{tbl-table-transition-disinhibition-long}: This transition
matrix captures shifts in states across Disinhibition exposures in the
six-wave panel study. Each cell in the matrix represents the count of
individuals transitioning from one state to another. The rows correspond
to the treatment at baseline (From), and the columns correspond to the
state at the following wave (To). \textbf{Diagonal entries} (in
\textbf{bold}) correspond to the number of individuals who remained in
their initial state across both waves. \textbf{Off-diagonal entries}
correspond to the transitions of individuals from their baseline state
to a different state in the treatment wave. A higher number on the
diagonal relative to the off-diagonal entries in the same row indicates
greater stability in a state. Conversely, higher off-diagonal numbers
suggest more frequent shifts from the baseline state to other states.

\subsubsection{Transition Table: Emotional Stability - Six wave
study}\label{transition-table-emotional-stability---six-wave-study}

\begin{longtable}[]{@{}
  >{\centering\arraybackslash}p{(\linewidth - 14\tabcolsep) * \real{0.1233}}
  >{\centering\arraybackslash}p{(\linewidth - 14\tabcolsep) * \real{0.1233}}
  >{\centering\arraybackslash}p{(\linewidth - 14\tabcolsep) * \real{0.1233}}
  >{\centering\arraybackslash}p{(\linewidth - 14\tabcolsep) * \real{0.1233}}
  >{\centering\arraybackslash}p{(\linewidth - 14\tabcolsep) * \real{0.1370}}
  >{\centering\arraybackslash}p{(\linewidth - 14\tabcolsep) * \real{0.1233}}
  >{\centering\arraybackslash}p{(\linewidth - 14\tabcolsep) * \real{0.1233}}
  >{\centering\arraybackslash}p{(\linewidth - 14\tabcolsep) * \real{0.1233}}@{}}
\caption{Transition matrix for change in Emotional Stability (six-wave
study)}\label{tbl-table-transition-emotional_stability-long}\tabularnewline
\toprule\noalign{}
\begin{minipage}[b]{\linewidth}\centering
From
\end{minipage} & \begin{minipage}[b]{\linewidth}\centering
State 1
\end{minipage} & \begin{minipage}[b]{\linewidth}\centering
State 2
\end{minipage} & \begin{minipage}[b]{\linewidth}\centering
State 3
\end{minipage} & \begin{minipage}[b]{\linewidth}\centering
State 4
\end{minipage} & \begin{minipage}[b]{\linewidth}\centering
State 5
\end{minipage} & \begin{minipage}[b]{\linewidth}\centering
State 6
\end{minipage} & \begin{minipage}[b]{\linewidth}\centering
State 7
\end{minipage} \\
\midrule\noalign{}
\endfirsthead
\toprule\noalign{}
\begin{minipage}[b]{\linewidth}\centering
From
\end{minipage} & \begin{minipage}[b]{\linewidth}\centering
State 1
\end{minipage} & \begin{minipage}[b]{\linewidth}\centering
State 2
\end{minipage} & \begin{minipage}[b]{\linewidth}\centering
State 3
\end{minipage} & \begin{minipage}[b]{\linewidth}\centering
State 4
\end{minipage} & \begin{minipage}[b]{\linewidth}\centering
State 5
\end{minipage} & \begin{minipage}[b]{\linewidth}\centering
State 6
\end{minipage} & \begin{minipage}[b]{\linewidth}\centering
State 7
\end{minipage} \\
\midrule\noalign{}
\endhead
\bottomrule\noalign{}
\endlastfoot
State 1 & \textbf{0} & 0 & 1 & 1 & 0 & 0 & 0 \\
State 2 & 0 & \textbf{33} & 39 & 25 & 1 & 0 & 0 \\
State 3 & 1 & 49 & \textbf{282} & 341 & 21 & 1 & 0 \\
State 4 & 0 & 26 & 376 & \textbf{1835} & 398 & 23 & 1 \\
State 5 & 0 & 1 & 36 & 443 & \textbf{288} & 32 & 0 \\
State 6 & 0 & 0 & 0 & 14 & 34 & \textbf{9} & 0 \\
State 7 & 0 & 0 & 0 & 1 & 0 & 0 & \textbf{0} \\
\end{longtable}

Table~\ref{tbl-table-transition-emotional_stability-long}: This
transition matrix captures shifts in states across Emotional Stability
exposures in the six-wave panel study. Each cell in the matrix
represents the count of individuals transitioning from one state to
another. The rows correspond to the treatment at baseline (From), and
the columns correspond to the state at the following wave (To).
\textbf{Diagonal entries} (in \textbf{bold}) correspond to the number of
individuals who remained in their initial state across both waves.
\textbf{Off-diagonal entries} correspond to the transitions of
individuals from their baseline state to a different state in the
treatment wave. A higher number on the diagonal relative to the
off-diagonal entries in the same row indicates greater stability in a
state. Conversely, higher off-diagonal numbers suggest more frequent
shifts from the baseline state to other states.

\subsubsection{Transition Table: Narcissism - Six wave
study}\label{transition-table-narcissism---six-wave-study}

\begin{longtable}[]{@{}
  >{\centering\arraybackslash}p{(\linewidth - 14\tabcolsep) * \real{0.1233}}
  >{\centering\arraybackslash}p{(\linewidth - 14\tabcolsep) * \real{0.1233}}
  >{\centering\arraybackslash}p{(\linewidth - 14\tabcolsep) * \real{0.1233}}
  >{\centering\arraybackslash}p{(\linewidth - 14\tabcolsep) * \real{0.1370}}
  >{\centering\arraybackslash}p{(\linewidth - 14\tabcolsep) * \real{0.1233}}
  >{\centering\arraybackslash}p{(\linewidth - 14\tabcolsep) * \real{0.1233}}
  >{\centering\arraybackslash}p{(\linewidth - 14\tabcolsep) * \real{0.1233}}
  >{\centering\arraybackslash}p{(\linewidth - 14\tabcolsep) * \real{0.1233}}@{}}
\caption{Transition matrix for change in Narcissism (six-wave
study)}\label{tbl-table-transition-narcissism-long}\tabularnewline
\toprule\noalign{}
\begin{minipage}[b]{\linewidth}\centering
From
\end{minipage} & \begin{minipage}[b]{\linewidth}\centering
State 1
\end{minipage} & \begin{minipage}[b]{\linewidth}\centering
State 2
\end{minipage} & \begin{minipage}[b]{\linewidth}\centering
State 3
\end{minipage} & \begin{minipage}[b]{\linewidth}\centering
State 4
\end{minipage} & \begin{minipage}[b]{\linewidth}\centering
State 5
\end{minipage} & \begin{minipage}[b]{\linewidth}\centering
State 6
\end{minipage} & \begin{minipage}[b]{\linewidth}\centering
State 7
\end{minipage} \\
\midrule\noalign{}
\endfirsthead
\toprule\noalign{}
\begin{minipage}[b]{\linewidth}\centering
From
\end{minipage} & \begin{minipage}[b]{\linewidth}\centering
State 1
\end{minipage} & \begin{minipage}[b]{\linewidth}\centering
State 2
\end{minipage} & \begin{minipage}[b]{\linewidth}\centering
State 3
\end{minipage} & \begin{minipage}[b]{\linewidth}\centering
State 4
\end{minipage} & \begin{minipage}[b]{\linewidth}\centering
State 5
\end{minipage} & \begin{minipage}[b]{\linewidth}\centering
State 6
\end{minipage} & \begin{minipage}[b]{\linewidth}\centering
State 7
\end{minipage} \\
\midrule\noalign{}
\endhead
\bottomrule\noalign{}
\endlastfoot
State 1 & \textbf{2} & 13 & 0 & 1 & 0 & 0 & 0 \\
State 2 & 69 & \textbf{508} & 322 & 45 & 4 & 0 & 0 \\
State 3 & 23 & 415 & \textbf{1125} & 376 & 13 & 1 & 0 \\
State 4 & 2 & 58 & 414 & \textbf{726} & 81 & 7 & 1 \\
State 5 & 0 & 0 & 9 & 57 & \textbf{29} & 2 & 0 \\
State 6 & 0 & 0 & 0 & 2 & 3 & \textbf{1} & 0 \\
State 7 & 0 & 0 & 0 & 0 & 0 & 0 & \textbf{0} \\
\end{longtable}

Table~\ref{tbl-table-transition-narcissism-long}: This transition matrix
captures shifts in states across Narcissism from the baseline to
exposure wave. Each cell in the matrix represents the count of
individuals transitioning from one state to another. The rows correspond
to the treatment at baseline (From), and the columns correspond to the
state at the following wave (To). \textbf{Diagonal entries} (in
\textbf{bold}) correspond to the number of individuals who remained in
their initial state across both waves. \textbf{Off-diagonal entries}
correspond to the transitions of individuals from their baseline state
to a different state in the treatment wave. A higher number on the
diagonal relative to the off-diagonal entries in the same row indicates
greater stability in a state. Conversely, higher off-diagonal numbers
suggest more frequent shifts from the baseline state to other states.

\subsubsection{Transition Table: Psychopathy Combined - Six wave
study}\label{transition-table-psychopathy-combined---six-wave-study}

\begin{longtable}[]{@{}ccccc@{}}
\caption{Transition matrix for change in Psychopathy Combined (six-wave
study)}\label{tbl-table-transition-psychopathy-combined-long}\tabularnewline
\toprule\noalign{}
From & State 2 & State 3 & State 4 & State 5 \\
\midrule\noalign{}
\endfirsthead
\toprule\noalign{}
From & State 2 & State 3 & State 4 & State 5 \\
\midrule\noalign{}
\endhead
\bottomrule\noalign{}
\endlastfoot
State 2 & \textbf{217} & 207 & 3 & 0 \\
State 3 & 268 & \textbf{2817} & 244 & 0 \\
State 4 & 2 & 273 & \textbf{269} & 2 \\
State 5 & 0 & 0 & 4 & \textbf{2} \\
\end{longtable}

Table~\ref{tbl-table-transition-psychopathy-combined-long}: This
transition matrix captures shifts in states across Emotional Stability
exposures in the six-wave panel study. Each cell in the matrix
represents the count of individuals transitioning from one state to
another. The rows correspond to the treatment at baseline (From), and
the columns correspond to the state at the following wave (To).
\textbf{Diagonal entries} (in \textbf{bold}) correspond to the number of
individuals who remained in their initial state across both waves.
\textbf{Off-diagonal entries} correspond to the transitions of
individuals from their baseline state to a different state in the
treatment wave. A higher number on the diagonal relative to the
off-diagonal entries in the same row indicates greater stability in a
state. Conversely, higher off-diagonal numbers suggest more frequent
shifts from the baseline state to other states.

\newpage{}

\subsection{Appendix XX: OSF Planned Study Results}\label{appendix-xx}

\subsubsection{Results Study 1 Antagonism: Partner Shift Up vs Null (OSF
Comparison)}\label{results-study-1-antagonism-partner-shift-up-vs-null-osf-comparison}

\begin{longtable}[]{@{}
  >{\raggedright\arraybackslash}p{(\linewidth - 10\tabcolsep) * \real{0.4494}}
  >{\raggedleft\arraybackslash}p{(\linewidth - 10\tabcolsep) * \real{0.1798}}
  >{\raggedleft\arraybackslash}p{(\linewidth - 10\tabcolsep) * \real{0.0674}}
  >{\raggedleft\arraybackslash}p{(\linewidth - 10\tabcolsep) * \real{0.0787}}
  >{\raggedleft\arraybackslash}p{(\linewidth - 10\tabcolsep) * \real{0.0899}}
  >{\raggedleft\arraybackslash}p{(\linewidth - 10\tabcolsep) * \real{0.1348}}@{}}

\caption{\label{tbl-results-antagonism-partner-up-osf}Table for
antagonism effect on partner multi-dimensional well-being: shift up vs
null (OSF)}

\tabularnewline

\toprule\noalign{}
\begin{minipage}[b]{\linewidth}\raggedright
\end{minipage} & \begin{minipage}[b]{\linewidth}\raggedleft
E{[}Y(1){]}-E{[}Y(0){]}
\end{minipage} & \begin{minipage}[b]{\linewidth}\raggedleft
2.5 \%
\end{minipage} & \begin{minipage}[b]{\linewidth}\raggedleft
97.5 \%
\end{minipage} & \begin{minipage}[b]{\linewidth}\raggedleft
E\_Value
\end{minipage} & \begin{minipage}[b]{\linewidth}\raggedleft
E\_Val\_bound
\end{minipage} \\
\midrule\noalign{}
\endhead
\bottomrule\noalign{}
\endlastfoot
Conflict in Relationship: Partner & 0.03 & -0.03 & 0.08 & 1.18 & 1 \\
Kessler 6 Anxiety: Partner & 0.02 & -0.03 & 0.06 & 1.15 & 1 \\
Kessler 6 Depression: Partner & 0.01 & -0.03 & 0.05 & 1.10 & 1 \\
Kessler 6 Distress: Partner & 0.02 & -0.02 & 0.06 & 1.14 & 1 \\
Life Satisfaction: Partner & 0.01 & -0.03 & 0.05 & 1.13 & 1 \\
Personal Well-Being Index: Partner & 0.00 & -0.04 & 0.04 & 1.04 & 1 \\
Satisfaction with Relationship: Partner & 0.02 & -0.03 & 0.06 & 1.14 &
1 \\
Self Esteem: Partner & 0.01 & -0.03 & 0.05 & 1.08 & 1 \\

\end{longtable}

No reliable causal evidence detected for the reported outcomes.

\begin{figure}

\centering{

\pandocbounded{\includegraphics[keepaspectratio]{4-DEC-manuscript-aaron-psychopathy_files/figure-pdf/fig-results-antagonism-partner-up-comparison-1.pdf}}

}

\caption{\label{fig-results-antagonism-partner-up-comparison}Comparison
of OSF and original results for antagonism effect on partner
multi-dimensional well-being: shift up vs null}

\end{figure}%

\newpage{}

\subsubsection{Results Study 1 Antagonism: Partner Shift Down vs Null
(OSF
Comparison)}\label{results-study-1-antagonism-partner-shift-down-vs-null-osf-comparison}

\begin{longtable}[]{@{}
  >{\raggedright\arraybackslash}p{(\linewidth - 10\tabcolsep) * \real{0.4494}}
  >{\raggedleft\arraybackslash}p{(\linewidth - 10\tabcolsep) * \real{0.1798}}
  >{\raggedleft\arraybackslash}p{(\linewidth - 10\tabcolsep) * \real{0.0674}}
  >{\raggedleft\arraybackslash}p{(\linewidth - 10\tabcolsep) * \real{0.0787}}
  >{\raggedleft\arraybackslash}p{(\linewidth - 10\tabcolsep) * \real{0.0899}}
  >{\raggedleft\arraybackslash}p{(\linewidth - 10\tabcolsep) * \real{0.1348}}@{}}

\caption{\label{tbl-results-antagonism-partner-down-osf}Table for
antagonism effect on partner multi-dimensional well-being: shift down vs
null (OSF)}

\tabularnewline

\toprule\noalign{}
\begin{minipage}[b]{\linewidth}\raggedright
\end{minipage} & \begin{minipage}[b]{\linewidth}\raggedleft
E{[}Y(1){]}-E{[}Y(0){]}
\end{minipage} & \begin{minipage}[b]{\linewidth}\raggedleft
2.5 \%
\end{minipage} & \begin{minipage}[b]{\linewidth}\raggedleft
97.5 \%
\end{minipage} & \begin{minipage}[b]{\linewidth}\raggedleft
E\_Value
\end{minipage} & \begin{minipage}[b]{\linewidth}\raggedleft
E\_Val\_bound
\end{minipage} \\
\midrule\noalign{}
\endhead
\bottomrule\noalign{}
\endlastfoot
Conflict in Relationship: Partner & -0.06 & -0.11 & 0.00 & 1.29 &
1.06 \\
Kessler 6 Anxiety: Partner & -0.02 & -0.07 & 0.02 & 1.17 & 1.00 \\
Kessler 6 Depression: Partner & -0.01 & -0.05 & 0.04 & 1.08 & 1.00 \\
Kessler 6 Distress: Partner & -0.01 & -0.06 & 0.03 & 1.13 & 1.00 \\
Life Satisfaction: Partner & 0.00 & -0.04 & 0.05 & 1.05 & 1.00 \\
Personal Well-Being Index: Partner & 0.01 & -0.03 & 0.05 & 1.09 &
1.00 \\
Satisfaction with Relationship: Partner & 0.02 & -0.03 & 0.06 & 1.14 &
1.00 \\
Self Esteem: Partner & 0.00 & -0.04 & 0.04 & 1.05 & 1.00 \\

\end{longtable}

No reliable causal evidence detected for the reported outcomes.

\begin{figure}

\centering{

\pandocbounded{\includegraphics[keepaspectratio]{4-DEC-manuscript-aaron-psychopathy_files/figure-pdf/fig-results-antagonism-partner-down-comparison-1.pdf}}

}

\caption{\label{fig-results-antagonism-partner-down-comparison}Comparison
of OSF and original results for antagonism effect on partner
multi-dimensional well-being: shift down vs null}

\end{figure}%

\newpage{}

\subsubsection{Results Study 1 Disinhibition: Partner Shift Up vs Null
(OSF
Comparison)}\label{results-study-1-disinhibition-partner-shift-up-vs-null-osf-comparison}

\begin{longtable}[]{@{}
  >{\raggedright\arraybackslash}p{(\linewidth - 10\tabcolsep) * \real{0.4494}}
  >{\raggedleft\arraybackslash}p{(\linewidth - 10\tabcolsep) * \real{0.1798}}
  >{\raggedleft\arraybackslash}p{(\linewidth - 10\tabcolsep) * \real{0.0674}}
  >{\raggedleft\arraybackslash}p{(\linewidth - 10\tabcolsep) * \real{0.0787}}
  >{\raggedleft\arraybackslash}p{(\linewidth - 10\tabcolsep) * \real{0.0899}}
  >{\raggedleft\arraybackslash}p{(\linewidth - 10\tabcolsep) * \real{0.1348}}@{}}

\caption{\label{tbl-results-disinhibition-partner-up-osf}Table for
disinhibition effect on partner multi-dimensional well-being: shift up
vs null (OSF)}

\tabularnewline

\toprule\noalign{}
\begin{minipage}[b]{\linewidth}\raggedright
\end{minipage} & \begin{minipage}[b]{\linewidth}\raggedleft
E{[}Y(1){]}-E{[}Y(0){]}
\end{minipage} & \begin{minipage}[b]{\linewidth}\raggedleft
2.5 \%
\end{minipage} & \begin{minipage}[b]{\linewidth}\raggedleft
97.5 \%
\end{minipage} & \begin{minipage}[b]{\linewidth}\raggedleft
E\_Value
\end{minipage} & \begin{minipage}[b]{\linewidth}\raggedleft
E\_Val\_bound
\end{minipage} \\
\midrule\noalign{}
\endhead
\bottomrule\noalign{}
\endlastfoot
Conflict in Relationship: Partner & 0.02 & -0.03 & 0.07 & 1.16 & 1 \\
Kessler 6 Anxiety: Partner & 0.02 & -0.02 & 0.06 & 1.16 & 1 \\
Kessler 6 Depression: Partner & 0.03 & -0.01 & 0.07 & 1.20 & 1 \\
Kessler 6 Distress: Partner & 0.03 & 0.00 & 0.07 & 1.20 & 1 \\
Life Satisfaction: Partner & 0.00 & -0.04 & 0.04 & 1.03 & 1 \\
Personal Well-Being Index: Partner & -0.03 & -0.07 & 0.01 & 1.18 & 1 \\
Satisfaction with Relationship: Partner & -0.03 & -0.07 & 0.01 & 1.20 &
1 \\
Self Esteem: Partner & 0.00 & -0.03 & 0.04 & 1.05 & 1 \\

\end{longtable}

No reliable causal evidence detected for the reported outcomes.

\begin{figure}

\centering{

\pandocbounded{\includegraphics[keepaspectratio]{4-DEC-manuscript-aaron-psychopathy_files/figure-pdf/fig-results-disinhibition-partner-up-comparison-1.pdf}}

}

\caption{\label{fig-results-disinhibition-partner-up-comparison}Comparison
of OSF and original results for disinhibition effect on partner
multi-dimensional well-being: shift up vs null}

\end{figure}%

\newpage{}

\subsubsection{Results Study 1 Disinhibition: Partner Shift Down vs Null
(OSF
Comparison)}\label{results-study-1-disinhibition-partner-shift-down-vs-null-osf-comparison}

\begin{longtable}[]{@{}
  >{\raggedright\arraybackslash}p{(\linewidth - 10\tabcolsep) * \real{0.4494}}
  >{\raggedleft\arraybackslash}p{(\linewidth - 10\tabcolsep) * \real{0.1798}}
  >{\raggedleft\arraybackslash}p{(\linewidth - 10\tabcolsep) * \real{0.0674}}
  >{\raggedleft\arraybackslash}p{(\linewidth - 10\tabcolsep) * \real{0.0787}}
  >{\raggedleft\arraybackslash}p{(\linewidth - 10\tabcolsep) * \real{0.0899}}
  >{\raggedleft\arraybackslash}p{(\linewidth - 10\tabcolsep) * \real{0.1348}}@{}}

\caption{\label{tbl-results-disinhibition-partner-down-osf}Table for
disinhibition effect on partner multi-dimensional well-being: shift down
vs null (OSF)}

\tabularnewline

\toprule\noalign{}
\begin{minipage}[b]{\linewidth}\raggedright
\end{minipage} & \begin{minipage}[b]{\linewidth}\raggedleft
E{[}Y(1){]}-E{[}Y(0){]}
\end{minipage} & \begin{minipage}[b]{\linewidth}\raggedleft
2.5 \%
\end{minipage} & \begin{minipage}[b]{\linewidth}\raggedleft
97.5 \%
\end{minipage} & \begin{minipage}[b]{\linewidth}\raggedleft
E\_Value
\end{minipage} & \begin{minipage}[b]{\linewidth}\raggedleft
E\_Val\_bound
\end{minipage} \\
\midrule\noalign{}
\endhead
\bottomrule\noalign{}
\endlastfoot
Conflict in Relationship: Partner & -0.02 & -0.07 & 0.03 & 1.15 & 1 \\
Kessler 6 Anxiety: Partner & -0.04 & -0.09 & 0.00 & 1.25 & 1 \\
Kessler 6 Depression: Partner & 0.00 & -0.04 & 0.04 & 1.03 & 1 \\
Kessler 6 Distress: Partner & -0.03 & -0.06 & 0.01 & 1.18 & 1 \\
Life Satisfaction: Partner & 0.01 & -0.03 & 0.06 & 1.12 & 1 \\
Personal Well-Being Index: Partner & 0.02 & -0.02 & 0.06 & 1.17 & 1 \\
Satisfaction with Relationship: Partner & 0.03 & -0.01 & 0.07 & 1.20 &
1 \\
Self Esteem: Partner & 0.00 & -0.04 & 0.04 & 1.06 & 1 \\

\end{longtable}

No reliable causal evidence detected for the reported outcomes.

\begin{figure}

\centering{

\pandocbounded{\includegraphics[keepaspectratio]{4-DEC-manuscript-aaron-psychopathy_files/figure-pdf/fig-results-disinhibition-partner-down-comparison-1.pdf}}

}

\caption{\label{fig-results-disinhibition-partner-down-comparison}Comparison
of OSF and original results for disinhibition effect on partner
multi-dimensional well-being: shift down vs null}

\end{figure}%

\newpage{}

\subsubsection{Results Study 1 Emotional Stability: Partner Shift Up vs
Null (OSF
Comparison)}\label{results-study-1-emotional-stability-partner-shift-up-vs-null-osf-comparison}

\begin{longtable}[]{@{}
  >{\raggedright\arraybackslash}p{(\linewidth - 10\tabcolsep) * \real{0.4494}}
  >{\raggedleft\arraybackslash}p{(\linewidth - 10\tabcolsep) * \real{0.1798}}
  >{\raggedleft\arraybackslash}p{(\linewidth - 10\tabcolsep) * \real{0.0674}}
  >{\raggedleft\arraybackslash}p{(\linewidth - 10\tabcolsep) * \real{0.0787}}
  >{\raggedleft\arraybackslash}p{(\linewidth - 10\tabcolsep) * \real{0.0899}}
  >{\raggedleft\arraybackslash}p{(\linewidth - 10\tabcolsep) * \real{0.1348}}@{}}
\caption{Table for emotional stability effect on partner
multi-dimensional well-being: shift up vs null
(OSF)}\label{tbl-results-emotional-stability-partner-up-osf}\tabularnewline
\toprule\noalign{}
\begin{minipage}[b]{\linewidth}\raggedright
\end{minipage} & \begin{minipage}[b]{\linewidth}\raggedleft
E{[}Y(1){]}-E{[}Y(0){]}
\end{minipage} & \begin{minipage}[b]{\linewidth}\raggedleft
2.5 \%
\end{minipage} & \begin{minipage}[b]{\linewidth}\raggedleft
97.5 \%
\end{minipage} & \begin{minipage}[b]{\linewidth}\raggedleft
E\_Value
\end{minipage} & \begin{minipage}[b]{\linewidth}\raggedleft
E\_Val\_bound
\end{minipage} \\
\midrule\noalign{}
\endfirsthead
\toprule\noalign{}
\begin{minipage}[b]{\linewidth}\raggedright
\end{minipage} & \begin{minipage}[b]{\linewidth}\raggedleft
E{[}Y(1){]}-E{[}Y(0){]}
\end{minipage} & \begin{minipage}[b]{\linewidth}\raggedleft
2.5 \%
\end{minipage} & \begin{minipage}[b]{\linewidth}\raggedleft
97.5 \%
\end{minipage} & \begin{minipage}[b]{\linewidth}\raggedleft
E\_Value
\end{minipage} & \begin{minipage}[b]{\linewidth}\raggedleft
E\_Val\_bound
\end{minipage} \\
\midrule\noalign{}
\endhead
\bottomrule\noalign{}
\endlastfoot
Conflict in Relationship: Partner & 0.00 & -0.06 & 0.06 & 1.06 & 1.00 \\
Kessler 6 Anxiety: Partner & -0.03 & -0.09 & 0.03 & 1.20 & 1.00 \\
Kessler 6 Depression: Partner & -0.02 & -0.07 & 0.03 & 1.16 & 1.00 \\
Kessler 6 Distress: Partner & -0.03 & -0.08 & 0.02 & 1.20 & 1.00 \\
Life Satisfaction: Partner & 0.06 & 0.01 & 0.10 & 1.28 & 1.09 \\
Personal Well-Being Index: Partner & 0.05 & 0.00 & 0.09 & 1.25 & 1.05 \\
Satisfaction with Relationship: Partner & 0.02 & -0.04 & 0.07 & 1.16 &
1.00 \\
Self Esteem: Partner & 0.02 & -0.03 & 0.07 & 1.16 & 1.00 \\
\end{longtable}

No reliable causal evidence detected for the reported outcomes.

\begin{figure}

\centering{

\pandocbounded{\includegraphics[keepaspectratio]{4-DEC-manuscript-aaron-psychopathy_files/figure-pdf/fig-results-emotional-stability-partner-up-comparison-1.pdf}}

}

\caption{\label{fig-results-emotional-stability-partner-up-comparison}Comparison
of OSF and original results for emotional stability effect on partner
multi-dimensional well-being: shift up vs null}

\end{figure}%

\newpage{}

\subsubsection{Results Study 1 Emotional Stability: Partner Shift Down
vs Null (OSF
Comparison)}\label{results-study-1-emotional-stability-partner-shift-down-vs-null-osf-comparison}

\begin{longtable}[]{@{}
  >{\raggedright\arraybackslash}p{(\linewidth - 10\tabcolsep) * \real{0.4494}}
  >{\raggedleft\arraybackslash}p{(\linewidth - 10\tabcolsep) * \real{0.1798}}
  >{\raggedleft\arraybackslash}p{(\linewidth - 10\tabcolsep) * \real{0.0674}}
  >{\raggedleft\arraybackslash}p{(\linewidth - 10\tabcolsep) * \real{0.0787}}
  >{\raggedleft\arraybackslash}p{(\linewidth - 10\tabcolsep) * \real{0.0899}}
  >{\raggedleft\arraybackslash}p{(\linewidth - 10\tabcolsep) * \real{0.1348}}@{}}

\caption{\label{tbl-results-emotional-stability-partner-down-osf}Table
for emotional stability effect on partner multi-dimensional well-being:
shift down vs null (OSF)}

\tabularnewline

\toprule\noalign{}
\begin{minipage}[b]{\linewidth}\raggedright
\end{minipage} & \begin{minipage}[b]{\linewidth}\raggedleft
E{[}Y(1){]}-E{[}Y(0){]}
\end{minipage} & \begin{minipage}[b]{\linewidth}\raggedleft
2.5 \%
\end{minipage} & \begin{minipage}[b]{\linewidth}\raggedleft
97.5 \%
\end{minipage} & \begin{minipage}[b]{\linewidth}\raggedleft
E\_Value
\end{minipage} & \begin{minipage}[b]{\linewidth}\raggedleft
E\_Val\_bound
\end{minipage} \\
\midrule\noalign{}
\endhead
\bottomrule\noalign{}
\endlastfoot
Conflict in Relationship: Partner & -0.04 & -0.10 & 0.01 & 1.25 &
1.00 \\
Kessler 6 Anxiety: Partner & 0.00 & -0.05 & 0.05 & 1.03 & 1.00 \\
Kessler 6 Depression: Partner & -0.01 & -0.06 & 0.03 & 1.13 & 1.00 \\
Kessler 6 Distress: Partner & -0.01 & -0.05 & 0.04 & 1.09 & 1.00 \\
Life Satisfaction: Partner & -0.05 & -0.09 & -0.01 & 1.26 & 1.08 \\
Personal Well-Being Index: Partner & -0.02 & -0.07 & 0.02 & 1.17 &
1.00 \\
Satisfaction with Relationship: Partner & -0.03 & -0.08 & 0.03 & 1.18 &
1.00 \\
Self Esteem: Partner & -0.04 & -0.08 & 0.01 & 1.23 & 1.00 \\

\end{longtable}

No reliable causal evidence detected for the reported outcomes.

\begin{figure}

\centering{

\pandocbounded{\includegraphics[keepaspectratio]{4-DEC-manuscript-aaron-psychopathy_files/figure-pdf/fig-results-emotional-stability-partner-down-comparison-1.pdf}}

}

\caption{\label{fig-results-emotional-stability-partner-down-comparison}Comparison
of OSF and original results for emotional stability effect on partner
multi-dimensional well-being: shift down vs null}

\end{figure}%

\newpage{}

\subsubsection{Results Study 1 Narcissism: Partner Shift Up vs Null (OSF
Comparison)}\label{results-study-1-narcissism-partner-shift-up-vs-null-osf-comparison}

\begin{longtable}[]{@{}
  >{\raggedright\arraybackslash}p{(\linewidth - 10\tabcolsep) * \real{0.4494}}
  >{\raggedleft\arraybackslash}p{(\linewidth - 10\tabcolsep) * \real{0.1798}}
  >{\raggedleft\arraybackslash}p{(\linewidth - 10\tabcolsep) * \real{0.0674}}
  >{\raggedleft\arraybackslash}p{(\linewidth - 10\tabcolsep) * \real{0.0787}}
  >{\raggedleft\arraybackslash}p{(\linewidth - 10\tabcolsep) * \real{0.0899}}
  >{\raggedleft\arraybackslash}p{(\linewidth - 10\tabcolsep) * \real{0.1348}}@{}}

\caption{\label{tbl-results-narcissism-partner-up-osf}Table for
narcissism effect on partner multi-dimensional well-being: shift up vs
null (OSF)}

\tabularnewline

\toprule\noalign{}
\begin{minipage}[b]{\linewidth}\raggedright
\end{minipage} & \begin{minipage}[b]{\linewidth}\raggedleft
E{[}Y(1){]}-E{[}Y(0){]}
\end{minipage} & \begin{minipage}[b]{\linewidth}\raggedleft
2.5 \%
\end{minipage} & \begin{minipage}[b]{\linewidth}\raggedleft
97.5 \%
\end{minipage} & \begin{minipage}[b]{\linewidth}\raggedleft
E\_Value
\end{minipage} & \begin{minipage}[b]{\linewidth}\raggedleft
E\_Val\_bound
\end{minipage} \\
\midrule\noalign{}
\endhead
\bottomrule\noalign{}
\endlastfoot
Conflict in Relationship: Partner & 0.03 & -0.02 & 0.09 & 1.21 & 1 \\
Kessler 6 Anxiety: Partner & -0.01 & -0.05 & 0.03 & 1.12 & 1 \\
Kessler 6 Depression: Partner & -0.02 & -0.06 & 0.03 & 1.14 & 1 \\
Kessler 6 Distress: Partner & -0.01 & -0.05 & 0.03 & 1.09 & 1 \\
Life Satisfaction: Partner & 0.02 & -0.02 & 0.06 & 1.15 & 1 \\
Personal Well-Being Index: Partner & 0.02 & -0.03 & 0.06 & 1.14 & 1 \\
Satisfaction with Relationship: Partner & 0.00 & -0.05 & 0.05 & 1.03 &
1 \\
Self Esteem: Partner & 0.04 & -0.01 & 0.08 & 1.22 & 1 \\

\end{longtable}

No reliable causal evidence detected for the reported outcomes.

\begin{figure}

\centering{

\pandocbounded{\includegraphics[keepaspectratio]{4-DEC-manuscript-aaron-psychopathy_files/figure-pdf/fig-results-narcissism-partner-up-comparison-1.pdf}}

}

\caption{\label{fig-results-narcissism-partner-up-comparison}Comparison
of OSF and original results for narcissism effect on partner
multi-dimensional well-being: shift up vs null}

\end{figure}%

\newpage{}

\subsubsection{Results Study 1 Narcissism: Partner Shift Down vs Null
(OSF
Comparison)}\label{results-study-1-narcissism-partner-shift-down-vs-null-osf-comparison}

\begin{longtable}[]{@{}
  >{\raggedright\arraybackslash}p{(\linewidth - 10\tabcolsep) * \real{0.4494}}
  >{\raggedleft\arraybackslash}p{(\linewidth - 10\tabcolsep) * \real{0.1798}}
  >{\raggedleft\arraybackslash}p{(\linewidth - 10\tabcolsep) * \real{0.0674}}
  >{\raggedleft\arraybackslash}p{(\linewidth - 10\tabcolsep) * \real{0.0787}}
  >{\raggedleft\arraybackslash}p{(\linewidth - 10\tabcolsep) * \real{0.0899}}
  >{\raggedleft\arraybackslash}p{(\linewidth - 10\tabcolsep) * \real{0.1348}}@{}}

\caption{\label{tbl-results-narcissism-partner-down-osf}Table for
narcissism effect on partner multi-dimensional well-being: shift down vs
null (OSF)}

\tabularnewline

\toprule\noalign{}
\begin{minipage}[b]{\linewidth}\raggedright
\end{minipage} & \begin{minipage}[b]{\linewidth}\raggedleft
E{[}Y(1){]}-E{[}Y(0){]}
\end{minipage} & \begin{minipage}[b]{\linewidth}\raggedleft
2.5 \%
\end{minipage} & \begin{minipage}[b]{\linewidth}\raggedleft
97.5 \%
\end{minipage} & \begin{minipage}[b]{\linewidth}\raggedleft
E\_Value
\end{minipage} & \begin{minipage}[b]{\linewidth}\raggedleft
E\_Val\_bound
\end{minipage} \\
\midrule\noalign{}
\endhead
\bottomrule\noalign{}
\endlastfoot
Conflict in Relationship: Partner & -0.04 & -0.09 & 0.01 & 1.22 & 1 \\
Kessler 6 Anxiety: Partner & 0.01 & -0.04 & 0.06 & 1.09 & 1 \\
Kessler 6 Depression: Partner & 0.00 & -0.04 & 0.05 & 1.00 & 1 \\
Kessler 6 Distress: Partner & 0.00 & -0.04 & 0.05 & 1.03 & 1 \\
Life Satisfaction: Partner & 0.03 & -0.01 & 0.07 & 1.19 & 1 \\
Personal Well-Being Index: Partner & -0.01 & -0.05 & 0.03 & 1.12 & 1 \\
Satisfaction with Relationship: Partner & 0.00 & -0.04 & 0.04 & 1.05 &
1 \\
Self Esteem: Partner & 0.01 & -0.03 & 0.05 & 1.11 & 1 \\

\end{longtable}

No reliable causal evidence detected for the reported outcomes.

\begin{figure}

\centering{

\pandocbounded{\includegraphics[keepaspectratio]{4-DEC-manuscript-aaron-psychopathy_files/figure-pdf/fig-results-narcissism-partner-down-comparison-1.pdf}}

}

\caption{\label{fig-results-narcissism-partner-down-comparison}Comparison
of OSF and original results for narcissism effect on partner
multi-dimensional well-being: shift down vs null}

\end{figure}%

\newpage{}

\subsubsection{Results Study 1 Psychopathy: Partner Shift Up vs Null
(OSF
Comparison)}\label{results-study-1-psychopathy-partner-shift-up-vs-null-osf-comparison}

\begin{longtable}[]{@{}
  >{\raggedright\arraybackslash}p{(\linewidth - 10\tabcolsep) * \real{0.4494}}
  >{\raggedleft\arraybackslash}p{(\linewidth - 10\tabcolsep) * \real{0.1798}}
  >{\raggedleft\arraybackslash}p{(\linewidth - 10\tabcolsep) * \real{0.0674}}
  >{\raggedleft\arraybackslash}p{(\linewidth - 10\tabcolsep) * \real{0.0787}}
  >{\raggedleft\arraybackslash}p{(\linewidth - 10\tabcolsep) * \real{0.0899}}
  >{\raggedleft\arraybackslash}p{(\linewidth - 10\tabcolsep) * \real{0.1348}}@{}}

\caption{\label{tbl-results-psychopathy-partner-up-osf}Table for
psychopathy effect on partner multi-dimensional well-being: shift up vs
null (OSF)}

\tabularnewline

\toprule\noalign{}
\begin{minipage}[b]{\linewidth}\raggedright
\end{minipage} & \begin{minipage}[b]{\linewidth}\raggedleft
E{[}Y(1){]}-E{[}Y(0){]}
\end{minipage} & \begin{minipage}[b]{\linewidth}\raggedleft
2.5 \%
\end{minipage} & \begin{minipage}[b]{\linewidth}\raggedleft
97.5 \%
\end{minipage} & \begin{minipage}[b]{\linewidth}\raggedleft
E\_Value
\end{minipage} & \begin{minipage}[b]{\linewidth}\raggedleft
E\_Val\_bound
\end{minipage} \\
\midrule\noalign{}
\endhead
\bottomrule\noalign{}
\endlastfoot
Conflict in Relationship: Partner & 0.14 & 0.08 & 0.20 & 1.53 & 1.36 \\
Kessler 6 Anxiety: Partner & 0.06 & 0.01 & 0.11 & 1.29 & 1.09 \\
Kessler 6 Depression: Partner & 0.03 & -0.02 & 0.07 & 1.18 & 1.00 \\
Kessler 6 Distress: Partner & 0.06 & 0.01 & 0.10 & 1.28 & 1.10 \\
Life Satisfaction: Partner & 0.04 & 0.00 & 0.09 & 1.25 & 1.00 \\
Personal Well-Being Index: Partner & -0.03 & -0.08 & 0.02 & 1.20 &
1.00 \\
Satisfaction with Relationship: Partner & 0.01 & -0.04 & 0.07 & 1.13 &
1.00 \\
Self Esteem: Partner & 0.05 & 0.01 & 0.10 & 1.28 & 1.11 \\

\end{longtable}

\paragraph{Conflict in relationship:
partner}\label{conflict-in-relationship-partner-5}

The effect estimate (rd) is 0.142 (0.08, 0.204). On the original scale,
the estimated effect is 0.187 (0.105, 0.269). E-value lower bound is
1.359, indicating evidence for causality.

\paragraph{Kessler 6 distress:
partner}\label{kessler-6-distress-partner-1}

The effect estimate (rd) is 0.055 (0.009, 0.1). On the original scale,
the estimated effect is 0.199 (0.034, 0.364). E-value lower bound is
1.105, indicating evidence for causality.

\paragraph{Self esteem: partner}\label{self-esteem-partner-1}

The effect estimate (rd) is 0.054 (0.011, 0.098). On the original scale,
the estimated effect is 0.065 (0.013, 0.118). E-value lower bound is
1.111, indicating evidence for causality.

All other effect estimates presented either weak or unreliable evidence
for causality.

\begin{figure}

\centering{

\pandocbounded{\includegraphics[keepaspectratio]{4-DEC-manuscript-aaron-psychopathy_files/figure-pdf/fig-results-psychopathy-partner-up-comparison-1.pdf}}

}

\caption{\label{fig-results-psychopathy-partner-up-comparison}Comparison
of OSF and original results for psychopathy effect on partner
multi-dimensional well-being: shift up vs null}

\end{figure}%

\newpage{}

\subsubsection{Results Study 1 Psychopathy: Partner Shift Down vs Null
(OSF
Comparison)}\label{results-study-1-psychopathy-partner-shift-down-vs-null-osf-comparison}

\begin{longtable}[]{@{}
  >{\raggedright\arraybackslash}p{(\linewidth - 10\tabcolsep) * \real{0.4494}}
  >{\raggedleft\arraybackslash}p{(\linewidth - 10\tabcolsep) * \real{0.1798}}
  >{\raggedleft\arraybackslash}p{(\linewidth - 10\tabcolsep) * \real{0.0674}}
  >{\raggedleft\arraybackslash}p{(\linewidth - 10\tabcolsep) * \real{0.0787}}
  >{\raggedleft\arraybackslash}p{(\linewidth - 10\tabcolsep) * \real{0.0899}}
  >{\raggedleft\arraybackslash}p{(\linewidth - 10\tabcolsep) * \real{0.1348}}@{}}

\caption{\label{tbl-results-psychopathy-partner-down-osf}Table for
psychopathy effect on partner multi-dimensional well-being: shift down
vs null (OSF)}

\tabularnewline

\toprule\noalign{}
\begin{minipage}[b]{\linewidth}\raggedright
\end{minipage} & \begin{minipage}[b]{\linewidth}\raggedleft
E{[}Y(1){]}-E{[}Y(0){]}
\end{minipage} & \begin{minipage}[b]{\linewidth}\raggedleft
2.5 \%
\end{minipage} & \begin{minipage}[b]{\linewidth}\raggedleft
97.5 \%
\end{minipage} & \begin{minipage}[b]{\linewidth}\raggedleft
E\_Value
\end{minipage} & \begin{minipage}[b]{\linewidth}\raggedleft
E\_Val\_bound
\end{minipage} \\
\midrule\noalign{}
\endhead
\bottomrule\noalign{}
\endlastfoot
Conflict in Relationship: Partner & -0.01 & -0.08 & 0.06 & 1.10 & 1 \\
Kessler 6 Anxiety: Partner & -0.04 & -0.10 & 0.02 & 1.22 & 1 \\
Kessler 6 Depression: Partner & -0.02 & -0.07 & 0.03 & 1.15 & 1 \\
Kessler 6 Distress: Partner & -0.02 & -0.07 & 0.03 & 1.16 & 1 \\
Life Satisfaction: Partner & 0.00 & -0.06 & 0.05 & 1.06 & 1 \\
Personal Well-Being Index: Partner & -0.02 & -0.06 & 0.03 & 1.15 & 1 \\
Satisfaction with Relationship: Partner & 0.01 & -0.05 & 0.06 & 1.09 &
1 \\
Self Esteem: Partner & -0.01 & -0.06 & 0.03 & 1.10 & 1 \\

\end{longtable}

No reliable causal evidence detected for the reported outcomes.

\begin{figure}

\centering{

\pandocbounded{\includegraphics[keepaspectratio]{4-DEC-manuscript-aaron-psychopathy_files/figure-pdf/fig-results-psychopathy-partner-down-comparison-1.pdf}}

}

\caption{\label{fig-results-psychopathy-partner-down-comparison}Comparison
of OSF and original results for psychopathy effect on partner
multi-dimensional well-being: shift down vs null}

\end{figure}%

\newpage{}




\end{document}
