% Options for packages loaded elsewhere
\PassOptionsToPackage{unicode}{hyperref}
\PassOptionsToPackage{hyphens}{url}
\PassOptionsToPackage{dvipsnames,svgnames,x11names}{xcolor}
%
\documentclass[
  single column]{article}

\usepackage{amsmath,amssymb}
\usepackage{iftex}
\ifPDFTeX
  \usepackage[T1]{fontenc}
  \usepackage[utf8]{inputenc}
  \usepackage{textcomp} % provide euro and other symbols
\else % if luatex or xetex
  \usepackage{unicode-math}
  \defaultfontfeatures{Scale=MatchLowercase}
  \defaultfontfeatures[\rmfamily]{Ligatures=TeX,Scale=1}
\fi
\usepackage[]{libertinus}
\ifPDFTeX\else  
    % xetex/luatex font selection
\fi
% Use upquote if available, for straight quotes in verbatim environments
\IfFileExists{upquote.sty}{\usepackage{upquote}}{}
\IfFileExists{microtype.sty}{% use microtype if available
  \usepackage[]{microtype}
  \UseMicrotypeSet[protrusion]{basicmath} % disable protrusion for tt fonts
}{}
\makeatletter
\@ifundefined{KOMAClassName}{% if non-KOMA class
  \IfFileExists{parskip.sty}{%
    \usepackage{parskip}
  }{% else
    \setlength{\parindent}{0pt}
    \setlength{\parskip}{6pt plus 2pt minus 1pt}}
}{% if KOMA class
  \KOMAoptions{parskip=half}}
\makeatother
\usepackage{xcolor}
\usepackage[top=30mm,left=25mm,heightrounded,headsep=22pt,headheight=11pt,footskip=33pt,ignorehead,ignorefoot]{geometry}
\setlength{\emergencystretch}{3em} % prevent overfull lines
\setcounter{secnumdepth}{-\maxdimen} % remove section numbering
% Make \paragraph and \subparagraph free-standing
\makeatletter
\ifx\paragraph\undefined\else
  \let\oldparagraph\paragraph
  \renewcommand{\paragraph}{
    \@ifstar
      \xxxParagraphStar
      \xxxParagraphNoStar
  }
  \newcommand{\xxxParagraphStar}[1]{\oldparagraph*{#1}\mbox{}}
  \newcommand{\xxxParagraphNoStar}[1]{\oldparagraph{#1}\mbox{}}
\fi
\ifx\subparagraph\undefined\else
  \let\oldsubparagraph\subparagraph
  \renewcommand{\subparagraph}{
    \@ifstar
      \xxxSubParagraphStar
      \xxxSubParagraphNoStar
  }
  \newcommand{\xxxSubParagraphStar}[1]{\oldsubparagraph*{#1}\mbox{}}
  \newcommand{\xxxSubParagraphNoStar}[1]{\oldsubparagraph{#1}\mbox{}}
\fi
\makeatother


\providecommand{\tightlist}{%
  \setlength{\itemsep}{0pt}\setlength{\parskip}{0pt}}\usepackage{longtable,booktabs,array}
\usepackage{calc} % for calculating minipage widths
% Correct order of tables after \paragraph or \subparagraph
\usepackage{etoolbox}
\makeatletter
\patchcmd\longtable{\par}{\if@noskipsec\mbox{}\fi\par}{}{}
\makeatother
% Allow footnotes in longtable head/foot
\IfFileExists{footnotehyper.sty}{\usepackage{footnotehyper}}{\usepackage{footnote}}
\makesavenoteenv{longtable}
\usepackage{graphicx}
\makeatletter
\def\maxwidth{\ifdim\Gin@nat@width>\linewidth\linewidth\else\Gin@nat@width\fi}
\def\maxheight{\ifdim\Gin@nat@height>\textheight\textheight\else\Gin@nat@height\fi}
\makeatother
% Scale images if necessary, so that they will not overflow the page
% margins by default, and it is still possible to overwrite the defaults
% using explicit options in \includegraphics[width, height, ...]{}
\setkeys{Gin}{width=\maxwidth,height=\maxheight,keepaspectratio}
% Set default figure placement to htbp
\makeatletter
\def\fps@figure{htbp}
\makeatother
% definitions for citeproc citations
\NewDocumentCommand\citeproctext{}{}
\NewDocumentCommand\citeproc{mm}{%
  \begingroup\def\citeproctext{#2}\cite{#1}\endgroup}
\makeatletter
 % allow citations to break across lines
 \let\@cite@ofmt\@firstofone
 % avoid brackets around text for \cite:
 \def\@biblabel#1{}
 \def\@cite#1#2{{#1\if@tempswa , #2\fi}}
\makeatother
\newlength{\cslhangindent}
\setlength{\cslhangindent}{1.5em}
\newlength{\csllabelwidth}
\setlength{\csllabelwidth}{3em}
\newenvironment{CSLReferences}[2] % #1 hanging-indent, #2 entry-spacing
 {\begin{list}{}{%
  \setlength{\itemindent}{0pt}
  \setlength{\leftmargin}{0pt}
  \setlength{\parsep}{0pt}
  % turn on hanging indent if param 1 is 1
  \ifodd #1
   \setlength{\leftmargin}{\cslhangindent}
   \setlength{\itemindent}{-1\cslhangindent}
  \fi
  % set entry spacing
  \setlength{\itemsep}{#2\baselineskip}}}
 {\end{list}}
\usepackage{calc}
\newcommand{\CSLBlock}[1]{\hfill\break\parbox[t]{\linewidth}{\strut\ignorespaces#1\strut}}
\newcommand{\CSLLeftMargin}[1]{\parbox[t]{\csllabelwidth}{\strut#1\strut}}
\newcommand{\CSLRightInline}[1]{\parbox[t]{\linewidth - \csllabelwidth}{\strut#1\strut}}
\newcommand{\CSLIndent}[1]{\hspace{\cslhangindent}#1}

\usepackage{booktabs}
\usepackage{longtable}
\usepackage{array}
\usepackage{multirow}
\usepackage{wrapfig}
\usepackage{float}
\usepackage{colortbl}
\usepackage{pdflscape}
\usepackage{tabu}
\usepackage{threeparttable}
\usepackage{threeparttablex}
\usepackage[normalem]{ulem}
\usepackage{makecell}
\usepackage{xcolor}
\input{/Users/joseph/GIT/latex/latex-for-quarto.tex}
\makeatletter
\@ifpackageloaded{caption}{}{\usepackage{caption}}
\AtBeginDocument{%
\ifdefined\contentsname
  \renewcommand*\contentsname{Table of contents}
\else
  \newcommand\contentsname{Table of contents}
\fi
\ifdefined\listfigurename
  \renewcommand*\listfigurename{List of Figures}
\else
  \newcommand\listfigurename{List of Figures}
\fi
\ifdefined\listtablename
  \renewcommand*\listtablename{List of Tables}
\else
  \newcommand\listtablename{List of Tables}
\fi
\ifdefined\figurename
  \renewcommand*\figurename{Figure}
\else
  \newcommand\figurename{Figure}
\fi
\ifdefined\tablename
  \renewcommand*\tablename{Table}
\else
  \newcommand\tablename{Table}
\fi
}
\@ifpackageloaded{float}{}{\usepackage{float}}
\floatstyle{ruled}
\@ifundefined{c@chapter}{\newfloat{codelisting}{h}{lop}}{\newfloat{codelisting}{h}{lop}[chapter]}
\floatname{codelisting}{Listing}
\newcommand*\listoflistings{\listof{codelisting}{List of Listings}}
\makeatother
\makeatletter
\makeatother
\makeatletter
\@ifpackageloaded{caption}{}{\usepackage{caption}}
\@ifpackageloaded{subcaption}{}{\usepackage{subcaption}}
\makeatother

\ifLuaTeX
  \usepackage{selnolig}  % disable illegal ligatures
\fi
\usepackage{bookmark}

\IfFileExists{xurl.sty}{\usepackage{xurl}}{} % add URL line breaks if available
\urlstyle{same} % disable monospaced font for URLs
\hypersetup{
  pdftitle={Methods in Causal Inference Part 2: Interaction, Mediation, and Time-Varying Treatments},
  pdfauthor={Joseph A. Bulbulia},
  colorlinks=true,
  linkcolor={blue},
  filecolor={Maroon},
  citecolor={Blue},
  urlcolor={Blue},
  pdfcreator={LaTeX via pandoc}}


\title{Methods in Causal Inference Part 2: Interaction, Mediation, and
Time-Varying Treatments}

\usepackage{academicons}
\usepackage{xcolor}

  \author{Joseph A. Bulbulia}
            \affil{%
             \small{     Victoria University of Wellington, NEW ZEALAND
          ORCID \textcolor[HTML]{A6CE39}{\aiOrcid} ~0000-0002-5861-2056 }
              }
      


\date{2024-10-01}
\begin{document}
\maketitle
\begin{abstract}
The analysis of `moderation', `interaction', `mediation', and
`longitudinal growth' is widespread in the human sciences, yet subject
to confusion. To clarify these concepts, it is essential to state causal
estimands, which requires specifying counterfactual contrasts for a
target population on an appropriate scale. Once causal estimands are
defined, we must consider their identification. I employ causal directed
acyclic graphs (causal DAGs) and Single World Intervention Graphs to
elucidate identification workflows. I show that when multiple treatments
exist, common methods for statistical inference, such as multi-level
regressions and statistical structural equation models, cannot typically
recover the causal quantities we seek. By properly framing and
addressing causal questions of interaction, mediation, and time-varying
treatments, we can expose the limitations of popular methods and guide
researchers to a clearer understanding of the causal questions that
animate our interests.

\textbf{KEYWORDS}: \emph{DAGs}; \emph{Mediation}; \emph{Moderation};
\emph{SWIGs}; \emph{Time-varying Treatments}
\end{abstract}


\subsection{Introduction}\label{id-introduction}

The young Charles Darwin was a keen fossil hunter and amateur geologist.
In August 1831, he accompanied the geologist Adam Sedgwick to the
Glyderau mountain range in northern Wales.

\begin{quote}
We spent many hours in Cwm Idwal, \ldots{} but neither of us saw a trace
of the wonderful glacial phenomena all around us; we did not notice the
plainly scored rocks, the perched boulders, the lateral and terminal
moraines. Yet these phenomena are so conspicuous that \ldots{} a house
burnt down by fire did not tell its story more plainly than did this
valley. If it had still been filled by a glacier, the phenomena would
have been less distinct than they now are.
(\citeproc{ref-darwin1887life}{Darwin, 1887}: p.25)
\end{quote}

This `striking instance of how easy it is to overlook phenomena, however
conspicuous' (\citeproc{ref-darwin1887life}{Darwin, 1887}: p.25) is
cited in cultural evolution to emphasise the importance of theory for
organising observations (\citeproc{ref-wilson2008evolution}{Wilson,
2008}). However, the importance of theory to scientific discovery
carries broader relevance: it applies to the statistical methods
scientists routinely apply to the data they collect. Without a clear
framework that relates statistical models to observations, the
understanding we seek from our data remains elusive.

Across many human sciences, we apply statistical models to data and
report `moderation', `interaction', `mediation', and `longitudinal
growth'. How are we to interpret the results of these models? It is
often unclear. The confidence with which investigators report findings
does not make interpretation any clearer.

Answering a causal question requires a careful workflow that begins by
defining the causal quantities and the target population of interest. We
specify a causal quantity, or estimand, as the contrast between
counterfactual outcomes under two (or more) clearly defined
interventions. Mathematical proofs establish that we can consistently
estimate the size of this counterfactual difference using data, provided
certain assumptions are met (discussed below). The subsequent steps in a
causal inferential workflow involve assessing the credibility of these
necessary assumptions, constructing appropriate estimators, and
obtaining relevant data. Only after these steps should statistical
analysis be attempted. Without adhering to a causal inferential
workflow, the associations derived from statistical models will reflect
true causal relationships only by chance, regardless of the
sophistication of our statistical methods
(\citeproc{ref-westreich2013}{Westreich \& Greenland, 2013}).

There is good news. Progress in the health sciences, computer science,
and economics has led to a common vocabulary and robust workflows that
enable investigators to formulate causal questions addressable with
data. These advancements allow for the evaluation of assumptions
necessary for obtaining consistent estimates, the construction of valid
statistical estimators, and the application of statistical models at the
end of the workflow. This conceptual framework, grounded in mathematical
proofs, enables investigators to clarify, communicate, and evaluate
causal questions using observational data. The consensus that has
emerged in causal inference over the past several decades is, in my
view, as transformative for the human sciences as the theory of
glaciation was for geology, or as Darwin's theory of evolution was for
biology. By refreaming questions of interaction, mediation, and
time-varying treatments as causal questions, I aim to clarify the
framework's interest, relevance, and power.

Several excellent resources are available that clarify workflows for
causal inference, from stating causal questions to communicating results
(\citeproc{ref-hernan2024WHATIF}{Hernan \& Robins, 2024};
\citeproc{ref-tlverse_handbook}{Laan et al., 2023};
\citeproc{ref-montgomery2018}{Montgomery et al., 2018};
\citeproc{ref-morgan2014}{Morgan \& Winship, 2014};
\citeproc{ref-neal2020introduction}{Neal, 2020};
\citeproc{ref-pearl2009a}{Pearl, 2009};
\citeproc{ref-grf2024}{Tibshirani et al., 2024};
\citeproc{ref-vanderweele2015}{VanderWeele, 2015}).

Here, my ambition is focussed.

\hyperref[id-sec-1]{Part 1} considers how to ask causal questions when
our interest is in comparing effect magnitudes between groups (effect
modification).

\hyperref[id-sec-2]{Part 2} considers how to ask causal questions when
our interest is in evaluating the joint effects of two independent
interventions (interaction).

\hyperref[id-sec-3]{Part 3} considers how to ask causal questions when
our interest is in evaluating the joint effects of two dependent
interventions (mediation analysis).

\hyperref[id-sec-4]{Part 4} considers how to ask causal questions when
our interest is in evaluating two or more sequential treatments of the
same kind (time-varying treatments).

I begin with a brief introduction to key concepts and terminology.

\subsubsection{Fundamental Assumptions for Causal
Inference}\label{fundamental-assumptions-for-causal-inference}

Consider indicators \(A\) and \(Y\) measuring states of the world. Let
\(A\) denote the `treatment' or `exposure' and \(Y\) the `outcome'. For
unit \(i\), we say that \(A_i\) causes \(Y_i\) if changing \(A_i\) from
one level, say \(A_i = a^*\), to another level, \(A_i = a\), leads to a
different outcome for \(Y_i\). We assume \(A_i\) occurs before \(Y_i\).
To compare these outcomes, we use the notation \(Y_i(\tilde{a})\), which
represents the outcome for unit \(i\) under the treatment level
\(A_i = \tilde{a}\). To determine whether \(Y_i(\tilde{a})\)
quantitatively differs under two treatment levels on the difference
scale, we would compute the contrast \(Y_i(a^*) - Y_i(a)\). If
\(Y_i(a^*) - Y_i(a) \neq 0\), we would say there is a causal effect of
\(A\) on \(Y\) for individual \(i\).

Note that, for any given application of \(A\) for unit \(i\), we can
only observe one level of treatment. Therefore, we refer to
\(Y_i(a^*) - Y_i(a) \neq 0\) as a counterfactual contrast, or
equivalently, as a contrast of potential outcomes. Because an individual
may only receive one of two treatments at any given time, individual
causal effects cannot generally be observed. However, when certain
assumptions are satisfied, we may compute average treatment effects by
aggregating individual observations under different treatment
conditions. For a binary treatment, the difference in the average of the
potential outcomes under two different treatment levels for the
population from which a sample is drawn may be expressed as the
difference in mean outcomes: \(\mathbb{E}[Y(1)] - \mathbb{E}[Y(0)]\) or
equivalently as the average of the differences of the potential
outcomes: \(\mathbb{E}[Y(1) - Y(0)]\).

This counterfactual contrast represents the quantity obtained from an
ideally conducted randomised controlled trial, where any common cause of
the treatment and the outcome would occur only by chance. There are
three fundamental assumptions for computing average treatment effects
that, although generally satisfied in an ideally randomised experiment,
may not be satisfied in observational or `real world' data:

\begin{enumerate}
\def\labelenumi{\arabic{enumi}.}
\tightlist
\item
  \textbf{Causal Consistency}: Treatment levels remain consistent within
  the treatment arms to be compared. There must be at least two arms.
\item
  \textbf{(Conditional) Exchangeability}: Covariates that might affect
  outcomes under treatment are balanced across all arms (implied by
  randomisation).
\item
  \textbf{Positivity}: Each covariate that might affect treatment in the
  target population has a non-zero probability of being observed within
  each treatment condition (refer to Westreich \& Cole
  (\citeproc{ref-westreich2010}{2010}); Bulbulia et al.
  (\citeproc{ref-bulbulia2023a}{2023})).
\end{enumerate}

Note that we speak of an `ideal' experiment because real-world
experiments may fail these assumptions
(\citeproc{ref-bulbulia_2024_experiments}{Bulbulia, 2024c};
\citeproc{ref-hernan2017per}{Hernán \& Robins, 2017}). Our interest here
is restricted to observational or `real world' data.

\subsubsection{Schematic Workflow for Inferring Causal Effects from
Real-World Data Before Stating Statistical Estimators and Performing
Statistical
Analysis}\label{schematic-workflow-for-inferring-causal-effects-from-real-world-data-before-stating-statistical-estimators-and-performing-statistical-analysis}

In causal inference, we do not apply statistical models to data until
after we have stated a causal question and considered whether and how
the question may be identified from the data. We take the following
steps \emph{before} considering a statistical estimator or estimation:

\begin{enumerate}
\def\labelenumi{\arabic{enumi}.}
\item
  \textbf{State a well-defined treatment.} Clearly define the treatment
  (or equivalently exposure) that states the hypothetical intervention
  to which all population members will be exposed. For example, the
  treatment `weight loss' is a vaguely stated intervention because there
  are many ways one can lose weight -- exercise, diet, depression,
  cancer, amputation, and others (\citeproc{ref-hernan2024WHATIF}{Hernan
  \& Robins, 2024}). The intervention `weight loss by at least 30
  minutes of vigorous exercise each data' is more clearly defined
  (\citeproc{ref-hernan2008aObservationalStudiesAnalysedLike}{Hernán et
  al., 2008}).
\item
  \textbf{State a well-defined outcome.} Specify the outcome measure so
  that a causal contrast is interpretable. For example, `well-being' is
  arguably a vaguely stated outcome. However, the outcome,
  `psychological distress measured one year after the intervention using
  the Kessler-6 distress scale (\citeproc{ref-kessler2002}{Kessler et
  al., 2002})' is more clearly defined.
\item
  \textbf{Clarify the target population.} Define the population to whom
  the results will generalise. The eligibility criteria for a study will
  define the source population from which units in the study are
  sampled. However, sampling from the source population may yield a
  study population that differs from the source population in variables
  that modify the effects of treatment
  (\citeproc{ref-bulbulia2024wierd}{Bulbulia, 2024b};
  \citeproc{ref-dahabreh2019generalizing}{Dahabreh et al., 2019};
  \citeproc{ref-dahabreh2019}{Dahabreh \& Hernán, 2019};
  \citeproc{ref-stuart2018generalizability}{Stuart et al., 2018}).
  Investigators may also seek to generalise beyond the source
  population, which requires additional assumptions and may require
  additional knowledge (\citeproc{ref-bareinboim2013general}{Bareinboim
  \& Pearl, 2013}; \citeproc{ref-dahabreh2019}{Dahabreh \& Hernán,
  2019}; \citeproc{ref-deffner2022}{Deffner et al., 2022};
  \citeproc{ref-pearl2022external}{Pearl \& Bareinboim, 2022};
  \citeproc{ref-westreich2017transportability}{Westreich et al., 2017}).
\item
  \textbf{Evaluate whether treatment groups, conditional on measured
  covariates, are exchangeable.} The potential outcomes must be
  independent of treatments conditional on measured covariates
  (\citeproc{ref-angrist2009mostly}{Angrist \& Pischke, 2009};
  \citeproc{ref-hernan2024WHATIF}{Hernan \& Robins, 2024};
  \citeproc{ref-morgan2014}{Morgan \& Winship, 2014};
  \citeproc{ref-neal2020introduction}{Neal, 2020}).
\item
  \textbf{Ensure treatments to be compared satisfy causal consistency.}
  The versions of treatments over which a causal effect is estimated
  must be independent of the potential outcomes to be compared
  conditional on measured covariates
  (\citeproc{ref-hernan2024WHATIF}{Hernan \& Robins, 2024};
  \citeproc{ref-vanderweele2013}{VanderWeele \& Hernan, 2013},
  \citeproc{ref-vanderweele2013}{2013}).
\item
  \textbf{Check if the positivity assumption is satisfied.} There must
  be a non-zero probability of receiving each treatment level at every
  level of covariate required to satisfy the conditional exchangeability
  assumption, and if there are many versions of treatment, to satisfy
  the causal consistency assumption
  (\citeproc{ref-westreich2010}{Westreich \& Cole, 2010}).
\item
  \textbf{Ensure that the measures relate to the scientific questions at
  hand.} In particular, evaluate structural features of measurement
  error bias (\citeproc{ref-bulbulia2024wierd}{Bulbulia, 2024b};
  \citeproc{ref-hernan2024WHATIF}{Hernan \& Robins, 2024};
  \citeproc{ref-vanderweele2012MEASUREMENT}{VanderWeele \& Hernán,
  2012}).
\item
  \textbf{Consider strategies to ensure the study group measured at the
  end of the study represents the target population.} If the study
  population differs in the distribution of variables that modify the
  effect of a treatment on the outcome at both the beginning and end of
  treatment, the study will be biased when there is a treatment effect;
  as such, investigators must develop strategies to address attrition,
  non-response, and structural sources of bias from measurement error
  (\citeproc{ref-bulbulia2024wierd}{Bulbulia, 2024b};
  \citeproc{ref-hernan2004STRUCTURAL}{Hernán et al., 2004};
  \citeproc{ref-hernan2017SELECTIONWITHOUTCOLLIDER}{Hernán, 2017};
  \citeproc{ref-hernan2017per}{Hernán \& Robins, 2017}).
\item
  \textbf{Clearly communicate the reasoning, evidence, and
  decision-making that inform steps 1-8.} Provide transparent and
  thorough documentation of the decisions in steps 1-8. This includes
  detailing investigators' causal assumptions and any disagreements
  about these assumptions (\citeproc{ref-ogburn2021}{Ogburn \& Shpitser,
  2021}).
\end{enumerate}

\subsubsection{Conventions Used in This
Article}\label{conventions-used-in-this-article}

Table~\ref{tbl-terminologylocalconventions} reports our variables.
Table~\ref{tbl-terminologygeneral} describes our graphical conventions.
Here, we use two types of graphical tools to clarify causal questions:
causal directed acyclic graphs and Single World Intervention Graphs.
(Refer to supplementary materials \textbf{S1} for a glossary for
commonly used causal inference terms).

\begin{table}

\caption{\label{tbl-terminologylocalconventions}Terminology}

\centering{

\terminologylocalconventions

}

\end{table}%

\begin{table}

\caption{\label{tbl-terminologygeneral}Elements of Causal Graphs}

\centering{

\terminologygeneral

}

\end{table}%

Throughout, for clarity, we repeat our definitions and graphical
conventions as they are used. To begin, we define the following:

\begin{itemize}
\item
  \textbf{Node}: or equivalently a `variable,' denotes properties or
  characteristics of units within a population. In causal directed
  acyclic graphs, we draw nodes with respect to features in a
  \emph{target population}, which is the population for whom we seek
  causal inferences (\citeproc{ref-suzuki2020}{Suzuki et al., 2020}). A
  time-indexed node, \(X_t\), allows us to index measurements within
  time intervals \(t \in 1\dots T\), denoting relative chronology. If
  relative timing is not known, we may use \(X_{\phi t}\). The
  directions of arrows on a causal directed acyclic graph imply
  causation, and causation implies temporal order.
\item
  \textbf{Arrow} (\(\rightarrowNEW\)): Denotes a causal relationship
  from the node at the base of the arrow (a `parent') to the node at the
  tip of the arrow (a `child'). In causal directed acyclic graphs, we
  refrain from drawing an arrow from treatment to outcome to avoid
  asserting a causal path from \(A\) to \(Y\). Our purpose is to
  ascertain whether causality can be identified for this path. All other
  nodes and paths, including the absence of nodes and paths, are
  typically assumed.
\item
  \textbf{Boxed Variable} \(\boxed{X}\): Denotes conditioning or
  adjustment for \(X\).
\end{itemize}

Judea Pearl demonstrated that causal dependencies in a directed acyclic
graph could be evaluated using observable probability distributions
according to rules known as `d-separation'
(\citeproc{ref-pearl1995}{Pearl, 1995},
\citeproc{ref-pearl2009a}{2009}).

The rules, presented in Table~\ref{tbl-terminologydirectedgraph}, are as
follows:

\begin{enumerate}[a)]
     \item  {\bf Fork rule} ($B \leftarrowNEW \boxed{A} \rightarrowNEW C$): $B$ and $C$ are independent when conditioning on $A$: ($B \coprod C \mid A$).
     \item  {\bf Chain rule} ($A \rightarrowNEW \boxed{B} \rightarrowNEW C$): Conditioning on $B$ blocks the path between $A$ and $C$: ($A \coprod C \mid B$).
     \item  {\bf Collider rule} ($A \rightarrowNEW \boxed{C} \leftarrowNEW B$): $A$ and $B$ are independent until conditioning on $C$, which introduces dependence: ($A \cancel{\coprod} B \mid C$). 
 \end{enumerate}

From d-separation, Pearl derived a `backdoor adjustment theorem', which
provides a general identification algorithm given structural assumptions
encoded in a causal directed acyclic graph Pearl
(\citeproc{ref-pearl2009a}{2009}): In a causal directed acyclic graph
(causal DAG), a set of variables \(L\) satisfies the backdoor adjustment
theorem relative to the treatment \(A\) and the outcome \(Y\) if \(L\)
blocks every path between \(A\) and \(Y\) that contains an arrow
pointing into \(A\) (a backdoor path). Formally, \(L\) must:

\begin{enumerate}
\def\labelenumi{\arabic{enumi}.}
\tightlist
\item
  not be a descendant of \(A\);
\item
  block all backdoor paths from \(A\) to \(Y\).
\end{enumerate}

If \(L\) satisfies these conditions, the causal effect of \(A\) on \(Y\)
is identified by conditioning on
\(\boxed{L}\)(\citeproc{ref-pearl2009a}{Pearl, 2009}). In what follows,
I will assume readers are familiar with causal directed acyclic graphs.
Accessible introductions to causal directed acyclic graphs can be found
in Pearl (\citeproc{ref-pearl2009a}{2009}); Barrett
(\citeproc{ref-barrett2021}{2021}); McElreath
(\citeproc{ref-mcelreath2020}{2020}); Neal
(\citeproc{ref-neal2020introduction}{2020}); Hernan \& Robins
(\citeproc{ref-hernan2024WHATIF}{2024}); Bulbulia
(\citeproc{ref-bulbulia2023}{2024a}). (For an introduction to Single
World Intervention Graphs, used below, refer to Richardson \& Robins
(\citeproc{ref-richardson2013}{2013a}); Richardson \& Robins
(\citeproc{ref-richardson2013swigsprimer}{2013b}).)

\begin{table}

\caption{\label{tbl-terminologydirectedgraph}Five elementary causal
structures in a causal directed acyclic graph.}

\centering{

\terminologydirectedgraph

}

\end{table}%

\newpage{}

\subsection{Part 1: Interaction as `Effect
Modification'}\label{id-sec-1}

We have said that in causal inference, we must explicitly define our
causal question before applying statistical models to data. What
question might the analysis of interaction answer? In causal inference,
we think of interaction in two ways:

\begin{enumerate}
\def\labelenumi{\arabic{enumi}.}
\item
  \textbf{Interaction as Effect Modification from a Single
  Intervention}: We want to understand how an intervention varies in its
  effect across the strata of the target population in which we are
  interested. For example, we might ask: does the one-year effect of
  attending weekly religious service differ among people born in
  Australia compared with people born in Egypt? Note that here, we do
  not imagine intervening on birthplace.
\item
  \textbf{Interaction as Joint Intervention}: We want to understand
  whether the combined effects of two treatments administered together
  differ from the separate effect of each treatment acting alone. For
  example, we might ask: does the one-year effect of attending weekly
  religious service and the one-year effect of being at least one
  standard deviation above population average wealth differ from not
  attending any religious service and being at the population average in
  wealth? Here there are two interventions that might act individually,
  separately, or in concert.
\end{enumerate}

\hyperref[id-sec-1]{Part 1} considers interaction as effect
modification. Readers who do not wish to use the `effect modification'
may prefer the term `moderation.'

\subsubsection{Effect Modification}\label{effect-modification}

First, we define the `sharp-null hypothesis' as the hypothesis that
there is no effect of the exposure on the outcome for any unit in the
target population. Unless the sharp-null hypothesis is false, there may
be effect modification (Bulbulia
(\citeproc{ref-bulbulia2024wierd}{2024b})). Clearly, the variability of
causal effect modification cannot be assessed from descriptive measures
of individuals in one's sample. Such heterogeneity must be modelled
(refer to Tibshirani et al. (\citeproc{ref-grf2024}{2024}); Vansteelandt
\& Dukes (\citeproc{ref-vansteelandt2022a}{2022})). Alternatively, we
might seek to compare whether the effects of treatments vary by strata
within the target population. For example we may ask whether effects
vary by culture group membership, gender or another grouping variable.

\begin{table}

\caption{\label{tbl-terminologyeffectmodification}Graphical conventions
we use for representing effect modification.}

\centering{

\terminologyeffectmodification

}

\end{table}%

Table~\ref{tbl-terminologyeffectmodification} describes conventions to
clarify how to ask a causal question of effect modification. We assume
no confounding of the treatment on the outcome and that \(A\) has been
randomised (i.e.~\(\mathcal{R} \rightarrowNEW A\)). As such, we will not
use causal directed acyclic graphs to evaluate a treatment effect. We
will assume \(\mathcal{R}  \to A \to Y\).

To sharpen focus on our interest in effect modification, we will not
draw a causal arrow from the direct effect modifier \(F\) to the outcome
\(Y\). This convention is specific to this article. (Refer to Hernan \&
Robins (\citeproc{ref-hernan2024WHATIF}{2024}), pp.~126-127, for a
discussion of `noncausal' arrows.)

\begin{table}

\caption{\label{tbl-terminologyeffectmodificationtypes}Effect
Modification}

\centering{

\terminologyeffectmodificationtypes

}

\end{table}%

In Table~\ref{tbl-terminologyeffectmodificationtypes} \(\mathcal{G}_1\),
we represent that \(F\) is a direct effect modifier for the effect of
\(A\) on \(Y\). The open arrow indicates that we are not attributing
causality to \(F\). Because our estimand does not involve intervening on
\(Z\), there is no need to close its backdoor paths. Note that if \(F\)
were to affect \(A\), we could still estimate the effect modification of
\(A\) on \(Y\) because \(F\) has no causal interpretation. However, if
\(A\) were to cause \(F\), and \(F\) were to cause \(Y\), then by the
chain rule (recall Table~\ref{tbl-terminologygeneral}
\(\mathcal{G}_4\)), conditioning on \(F\) would bias the effect estimate
of \(A\) on \(Y\).

In Table~\ref{tbl-terminologyeffectmodificationtypes} \(\mathcal{G}_2\),
we represent that \(F\) is an unobserved direct effect modifier of \(A\)
to \(Y\). When the distribution of direct effect modifiers \(F\) differs
between two populations and effect modification is non-linear, marginal
treatment effects between populations will generally differ and will not
easily transport from one population to another. The concept of an
average treatment effect has no meaning without a population over which
the effect marginalises. This point, although obvious, has profound
implications when investigators seek to assess whether their research
generalises; refer to Hernan \& Robins
(\citeproc{ref-hernan2024WHATIF}{2024}), Bulbulia
(\citeproc{ref-bulbulia2024wierd}{2024b}). For example, if the study
population differs in the distribution of features that modify a
treatment effect, and no correction is applied, effect estimates will be
biased for the target population in at least one measure of effect
(\citeproc{ref-bulbulia2024wierd}{Bulbulia, 2024b};
\citeproc{ref-greenland2009commentary}{Greenland, 2009};
\citeproc{ref-lash2020}{Lash et al., 2020})

We present two candidate effect modifiers in
Table~\ref{tbl-terminologyeffectmodificationtypes} \(\mathcal{G}_3\).
Notice that whether a variable is an effect modifier also depends on
which other variables are included in the model. Here, \(F\) is a direct
effect modifier and \(G\), a descendant of \(F\), is an indirect effect
modifier. Suppose we were interested in whether treatment effects vary
(on the difference scale) within levels of \(F\). For example, let \(F\)
denote childhood deprivation, \(G\) denote educational achievement,
\(A\) denote a government educational initiative, and \(Y\) denote
recycling. If we were to condition on \(F\), we would not observe effect
modification by education \(G\) for the effect of the government
initiative \(A\) on recycling behaviour \(Y\): \(\boxed{F}\) blocks the
path \(G \association \boxed{F} \association Y\).

We present the same causal structure in
Table~\ref{tbl-terminologyeffectmodificationtypes} \(\mathcal{G}_4\).
However, we do not condition on the direct effect modifier \(F\), but
rather condition only on \(G\), the indirect effect modifier. In this
scenario, we would find that the effectiveness of the government
initiative \(A\) on recycling behaviour \(Y\) varies by educational
achievement \(G\). Thus, we would observe \(G\) as an effect modifier
because this path is open: \(G \association F \association Y\).

In Table~\ref{tbl-terminologyeffectmodificationtypes} \(\mathcal{G}_5\),
suppose we add another variable to our model, depression, denoted by
\(B\). We imagine \(B\) to be a stable trait or that investigators
measured childhood depression (that is, \(B\) precedes \(G\)). Suppose
we do not condition on the direct effect modifier \(F\) (childhood
deprivation), but we condition on educational attainment (\(G\)) and
depression (\(B\)). In this graph, \(G\) is a collider of \(F\) and
\(B\). Thus, conditioning on \(G\) (but not \(F\)) opens a path from
\(B \association G \association Z \association Y\). The investigators
would find evidence for effect modification by depression on the
effectiveness of the government intervention \(A\) on recycling (\(Y\)).
However, they should not interpret this result to mean that if levels of
depression were to change within the population the treatment effect
would change. \(B\) is not causally related to \(Y\) in this scenario.
Here, association is not causation.

In Table~\ref{tbl-terminologyeffectmodificationtypes} \(\mathcal{G}_6\),
we will not find evidence for effect modification for \(B\) and \(G\)
because conditioning on \(F\) blocks the flow of information that was
open in \(\mathcal{G}_4\) and \(\mathcal{G}_5\). This again underscores
the relativity of effect modification to (1) the structure of causality
in the world and (2) an investigator's statistical modelling strategy.

These examples reveal the power---and simplicity---of causal diagrams to
`transform the obvious'. Using causal directed acyclic graphs and
Pearl's rules of d-separation, it is clear that the analysis of effect
modification cannot be conducted without reference to an assumed causal
order and an explicit statement about which variables within that order
investigators have included in their statistical models
(\citeproc{ref-vanderweele2012}{VanderWeele, 2012}). Investigators and
policymakers may make incorrect policy decisions if they do not
understand the relevance of effect modification to such parameters. It
is important to remember that when evaluating evidence for effect
modification, we are not assessing the effects of intervening on
variables other than the treatment. Instead, we qualitatively evaluate
whether treatment effects vary across subgroups. For more on effect
modification, refer to VanderWeele
(\citeproc{ref-vanderweele2012}{2012}); VanderWeele \& Robins
(\citeproc{ref-vanderweele2007}{2007}); Suzuki et al.
(\citeproc{ref-suzuki2013counterfactual}{2013}).

\subsubsection{Example Showing Scale Dependence of Effect
Modification}\label{example-showing-scale-dependence-of-effect-modification}

Suppose investigators are interested in whether treatment varies across
levels of another variable, an effect modifier. We next illustrate how
causal inferences about the presence or absence of effect modification
depend on the scale that is used to measure the contrast. We show that
an effect modifier on the ratio scale may not be an effect modifier on
the difference scale, and vice versa.

Recall individual treatment effects are not observed. Assume a binary
treatment is randomised, and we have \(A = a \in \{0,1\}\).
Investigators are interested in comparing the magnitude of this
treatment effect across two levels of \(F = f \in \{0,1\}\).

We define the average treatment effects for each group under each
intervention as follows: \[
\mathbb{E}[Y \mid A = 0, F = 1] = \mu_{01}, \quad \mathbb{E}[Y \mid  A = 1, F = 1] = \mu_{11}
\] \[
\mathbb{E}[Y \mid  A = 0, F = 0] = \mu_{00}, \quad \mathbb{E}[Y \mid A = 1, F = 0] = \mu_{10}
\]

The treatment effect for each group on the difference scale (absolute
scale) is given:

\[
\text{ATE}_{F = 0} = \mu_{10} - \mu_{00}
\]

\[
\text{ATE}_{F = 1} = \mu_{11} - \mu_{01}
\]

The treatment effect on the ratio scale (relative scale) for each group
is:

\[
\text{RR}_{F = 0} = \frac{\mu_{10}}{\mu_{00}}
\] \[
\text{RR}_{F = 1} = \frac{\mu_{11}}{\mu_{01}}
\]

We say there is effect modification on the difference scale if: \[
\text{ATE}{F = 1} \neq \text{ATE}{F = 0} \implies \mu_{11} - \mu_{01} \neq \mu_{10} - \mu_{00}
\]

We say there is effect modification on the ratio scale if: \[
\text{RR}{F = 1} \neq \text{RR}{F = 0} \implies \frac{\mu_{11}}{\mu_{01}} \neq \frac{\mu_{10}}{\mu_{00}}
\]

We have stated each causal question in relation to well-defined causal
contrast and population, here defined by membership in \(F\).

Imagine we obtain the following estimates from our study:

Outcomes under A = 0:

\begin{itemize}
\tightlist
\item
  \(\mu_{00} = 5\)
\item
  \(\mu_{01} = 15\)
\end{itemize}

Outcomes under A = 1:

\begin{itemize}
\tightlist
\item
  \(\mu_{10} = 10\)
\item
  \(\mu_{11} = 20\)
\end{itemize}

Next, we calculate the treatment effects on the difference and ratio
scales for each group:

\textbf{Difference Scale:}

\[
\text{ATE}_{F = 0} = \mu_{10} - \mu_{00} = 10 - 5 = 5
\] \[
\text{ATE}_{F = 1} = \mu_{11} - \mu_{01} = 20 - 15 = 5
\]

Both groups have the same treatment effect on the difference scale,
\(\text{ATE}_{F = 0} = \text{ATE}_{F = 1} = 5\). investigators conclude
there is no evidence for effect modification on the difference scale.

\textbf{Ratio Scale:} \[
\text{RR}_{F = 0} = \frac{\mu_{10}}{\mu_{00}} = \frac{10}{5} = 2.00
\] \[
\text{RR}_{F = 1} = \frac{\mu_{11}}{\mu_{01}} = \frac{20}{15} \approx 1.33
\]

The treatment effect on the ratio scale is different between the two
groups, \(\text{RR}_{F = 0} = 2 \neq \text{RR}_{F = 1} \approx 1.33\).
Hence, investigators find evidence for effect modification on the ratio
scale.

The discrepancy in evidence for effect modification depending on the
scale we choose arises because the two scales measure different aspects
of the treatment effect: the absolute difference in outcomes versus the
relative change in outcomes. Parallel considerations apply to the
analysis of interaction, where we imagine a joint intervention. For this
reason, it is important to state the causal effect scale of interest in
advance of estimation (\citeproc{ref-bulbulia2023}{Bulbulia, 2024a}). We
next consider interaction as a joint intervention.

\subsection{Part 2: Interaction}\label{id-sec-2}

\subsubsection{Introducing Single World Intervention
Graphs}\label{introducing-single-world-intervention-graphs}

When evaluating evidence for interaction, we must assess whether the
combined effects of two treatments differ from the unique effects of
each treatment relative to a baseline where neither treatment is
administered. Understanding multiple interventions can be facilitated by
using Single World Intervention Graphs (SWIGs)
(\citeproc{ref-richardson2013}{Richardson \& Robins, 2013a}).

SWIGs employ Pearl's rules of d-separation but offer additional benefits
by graphically representing the complex factorisations required for
identification, presenting distinct interventions in separate graphs.
The first advantage is \textbf{greater precision and clarity}: SWIGs
allow us to consider identification conditions for each counterfactual
outcome individually. Such precision is useful because identification
conditions may differ for one, but not another, of the treatments to be
compared. Node-splitting also makes it easier to determine
identification conditions that are obscured in causal directed acylic
graphs, for an example refer to supplementary materials \textbf{S2}. The
second advantage \textbf{Single World Intervention Graphs unify the
potential outcomes framework with Pearl's structural causal model
framework}: any causal relationship that can be represented in a causal
directed acyclic graph can also be represented in a Single World
Intervention Graph (\citeproc{ref-richardson2013swigsprimer}{Richardson
\& Robins, 2013b}).

\begin{table}

\caption{\label{tbl-swigtable}Single World Interventions Graphs
(\(\mathcal{G}_{3-4}\)) present separate causal diagrams for each
treatment to be contrasted. A Single World Intervention Template
(\(\mathcal{G}_{2}\)) is a `graph value function' that produces the
individual counterfactual graphs
(\citeproc{ref-richardson2013}{Richardson \& Robins, 2013a}). On the
other hand, causal directed acyclic graphs, such as \(\mathcal{G}_1\),
require positing interventional distributions. The formalism
underpinning these interventional distributions is mathematically
equivalent to formalism underpinning the potential outcomes framework,
assuming the errors of the underlying structural causal models that
define the nodes on which interventions occur are independent
(\citeproc{ref-richardson2013}{Richardson \& Robins, 2013a}). Single
World Intervention Graphs (SWIGs), however, permit the comparison of
distinct interventions in our causal diagram without requiring that the
non-parametric structural equations that correspond to nodes on a causal
graph have independent error structures. This is useful when attempting
to identify the causal effects of sequential treatments, refer to
supplementary materials \textbf{S2}.}

\centering{

\swigtable

}

\end{table}%

\paragraph{Single World Intervention Graphs Work by
Node-Splitting}\label{single-world-intervention-graphs-work-by-node-splitting}

We create a Single World Intervention Graph by `node-splitting' at each
intervention such that the random variable that is intervened upon is
presented on one side and the level at which the random variable is
fixed is presented on the other.

Consider a template graph Table~\ref{tbl-swigtable} \(\mathcal{G}\).
Applying node-splitting to \(A\) involves creating separate graphs for
each value of \(A\) to be contrasted.

\begin{enumerate}
\def\labelenumi{\arabic{enumi}.}
\tightlist
\item
  \textbf{SWIG for \(A\) = 0}: Denoted as \(\mathcal{G}_{A=0}\), this
  graph shows the hypothetical scenario where \(A\) is set to 0.
\item
  \textbf{SWIG for \(A\) = 1}: Denoted as \(\mathcal{G}_{A=1}\), this
  graph shows the hypothetical scenario where \(A\) is set to 1.
\end{enumerate}

In these graphs, the node corresponding to the treatment \(A\) is split,
relabelled with the random and fixed component, and then each node that
follows is labelled with the fixed component until the next
intervention. Here, \(Y(\tilde{a})\) is the only variable to follow
\(A\) and it is relabelled either \(Y(0)\) or \(Y(1)\) corresponding to
whether \(A=1\) or \(A=0\); hence \(Y(\tilde{a})\) is relabelled as
either \(Y(0)\) or \(Y(1)\). Note that we do not place both \(Y(0)\) and
\(Y(1)\) on the same Single World Intervention Graph because the
variables are not jointly observed. Hence, we evaluate identification
for \(Y(0)\coprod A = 0| L\) and \(Y(1)\coprod A = 1 | L\) separately.

\subsubsection{Interaction as a Joint
Intervention}\label{interaction-as-a-joint-intervention}

We now use Single World Intervention Graphs (SWIGs) to clarify the
concept of causal interaction as a joint intervention.

Consider two treatments, denoted as \(A\) and \(B\), and a single
outcome, \(Y\). A joint intervention causal interaction examines whether
the combined effect of \(A\) and \(B\) on \(Y\) (denoted as \(Y(a,b)\))
is greater than, less than, or equal to the effect of each individual
treatment. What does this mean?

First, we obtain the expected outcomes when the entire target population
is treated at each level of the treatments to be compared. These
potential outcomes are illustrated in Table~\ref{tbl-interactionpuzzle}:

\begin{itemize}
\tightlist
\item
  Table~\ref{tbl-interactionpuzzle} \(\mathcal{G}_1\): Neither treatment
  \(A\) nor treatment \(B\) is given.
\item
  Table~\ref{tbl-interactionpuzzle} \(\mathcal{G}_2\): Both treatment
  \(A\) and treatment \(B\) are given.
\item
  Table~\ref{tbl-interactionpuzzle} \(\mathcal{G}_3\): Treatment \(A\)
  is given, and treatment \(B\) is not given.
\item
  Table~\ref{tbl-interactionpuzzle} \(\mathcal{G}_4\): Treatment \(A\)
  is not given, and treatment \(B\) is given.
\end{itemize}

By comparing these expected outcomes, we can determine the presence and
nature of causal interaction between treatments \(A\) and \(B\) with
respect to the outcome \(Y\).

\begin{table}

\caption{\label{tbl-interactionpuzzle}}

\centering{

\interactionpuzzle

}

\end{table}%

\subsubsection{Example}\label{example}

Consider the effect of beliefs in big gods (exposure \(A\)) and a
culture's monumental architecture (exposure \(B\)) on social complexity
(outcome \(Y\)). Both interventions have equal status; we are not
investigating effect modification of one by the other. The interventions
must be well-defined, with clear measures for `big gods', `monumental
architecture', and `social complexity' at specified time intervals after
the interventions are first observed.

We need to define the population of interest and choose the appropriate
scale to assess the individual and combined effects of \(A\) and \(B\).
Suppose our population consists of societies of primary urban genesis
(\citeproc{ref-wheatley1971pivot}{Wheatley, 1971}). We look for evidence
of causal interaction on the additive (difference) scale. Evidence for
interaction is present if the following inequality holds:

\begin{itemize}
\tightlist
\item
  \(\mathbb{E}[Y(1,1)]:\) Mean outcome for those jointly exposed to both
  treatments.
\item
  \(\mathbb{E}[Y(1,0)]:\) Mean outcome for those exposed only to big
  gods.
\item
  \(\mathbb{E}[Y(0,1)]:\) Mean outcome for those exposed only to
  monumental architecture.
\item
  \(\mathbb{E}[Y(0,0)]:\) Mean outcome for those exposed to neither
  treatment.
\end{itemize}

Evidence for interaction on the additive scale exists if:

\[
\left( \mathbb{E}[Y(1,1)] - \mathbb{E}[Y(0,0)] \right) - \left[ \left( \mathbb{E}[Y(1,0)] - \mathbb{E}[Y(0,0)] \right) + \left( \mathbb{E}[Y(0,1)] - \mathbb{E}[Y(0,0)] \right) \right] \neq 0
\]

Simplifying:

\[
\mathbb{E}[Y(1,1)] - \mathbb{E}[Y(1,0)] - \mathbb{E}[Y(0,1)] + \mathbb{E}[Y(0,0)] \neq 0
\]

A positive value indicates a synergistic (super-additive) interaction,
while a negative value indicates a sub-additive interaction. A value
close to zero suggests no interaction on the additive scale.

To identify these causal effects, we need to adjust for all confounders
of the relationships between \(A\), \(B\), and \(Y\). This includes any
variables that influence:

\begin{itemize}
\tightlist
\item
  Both \(A\) and \(Y\),
\item
  Both \(B\) and \(Y\),
\item
  Both \(A\) and \(B\),
\item
  Or all three variables simultaneously.
\end{itemize}

As with effect modification, evidence for causal interaction may differ
depending on the measurement scale used
(\citeproc{ref-vanderweele2012}{VanderWeele, 2012};
\citeproc{ref-vanderweele2014}{VanderWeele \& Knol, 2014}). For most
policy settings, the additive scale is recommended because it directly
relates to differences in outcome levels, which are often more
actionable.

Note that if \(A\) and \(B\) potentially influence each other over time,
we would need to collect longitudinal data and estimate causal effects
using mediation analysis. Indeed, if there has been a co-evolution of
religious culture, monumental architecture, and social complexity---as
archaeologists have long reported (\citeproc{ref-decoulanges1903}{De
Coulanges, 1903}; \citeproc{ref-wheatley1971pivot}{Wheatley,
1971})---mediation analysis may be more appropriate. However, the
requirements for causal mediation analysis are more stringent than those
for causal interaction analysis, which we will consider next.

\subsection{Part 3: Causal Mediation Analysis}\label{id-sec-3}

In 1992, Robins and Greenland clarified the objectives of interpretable
causal mediation analysis: to decompose the total effect into natural
direct and indirect effects within a set of hypothetical interventions,
contrasting their counterfactual outcomes
(\citeproc{ref-robins1992}{Robins \& Greenland, 1992}). This landmark
paper has been to mediation analysis of what `On the Origin of Species'
has been to evolutionary biology
(\citeproc{ref-darwin1859origin}{Darwin, 1859}). However, mediation
analysis in the human sciences remains rife with confusion. The primary
source of this confusion is the application of statistical models to
data without first defining the causal quantities of interest.
Associations derived from mediation analysis do not necessarily imply
causation and are typically uninterpretable. This section considers how
to formulate causal questions in mediation analysis.

\subsubsection{Defining a Mediation Analysis
Estimand}\label{defining-a-mediation-analysis-estimand}

To understand causal mediation, we deconstruct the total effect into
natural direct and indirect effects.

Again, the total effect of treatment \(A\) on outcome \(Y\) is defined
as the difference between potential outcomes when the treatment is
applied versus when it is not. The estimand for the total (or average,
or `marginal') treatment effect is given:

\[
\text{Total Treatment Effect} = \mathbb{E}[Y(1)] - \mathbb{E}[Y(0)]
\]

The total effect can be further decomposed into direct and indirect
effects, addressing questions of mediation. The potential outcome
\(Y(1)\), considering the mediator, expands to:

\[ 
\mathbb{E}[Y(1)] = \mathbb{E}[Y(1, M(1))]
\]

This considers the effect of the exposure \(A = 1\) and the mediator at
its natural value when \(A = 1\). Similarly, the potential outcome
\(\mathbb{E}[Y(0)]\), considering the mediator, expands to:

\[ 
\mathbb{E}[Y(0)] = \mathbb{E}[Y(0, M(0))]
\]

This quantity denotes the effect of exposure \(A = 0\) and the mediator
at its natural value when \(A = 0\).

Next, we clarify our estimand by decomposing the Total Effect (TE) into
the Natural Direct Effect (NDE) and the Natural Indirect Effect (NIE).

\textbf{Natural Direct Effect (NDE)} is the effect of the treatment on
the outcome while maintaining the mediator at the level it would have
been if the treatment had not been applied:

\[
\text{Natural Direct Effect} = \textcolor{blue}{\mathbb{E}[Y(1, M(0))]} - \mathbb{E}[Y(0, M(0))]
\]

Here, the counterfactual quantities not directly realised in the data
are highlighted in blue: \(\textcolor{blue}{\mathbb{E}[Y(1, M(0))]}\).
Notice we add this term to the potential outcomes when \(A = 0\),
recalling \(\mathbb{E}[Y(0, M(0))] = Y(0)\).

\textbf{Natural Indirect Effect (NIE)} is the effect of the exposure on
the outcome that is mediated. To obtain these quantities, we compare the
potential outcome \(Y\) under treatment, where the mediator assumes its
natural level under treatment, with the potential outcome when the
mediator assumes its natural value under no treatment:

\[
\text{Natural Indirect Effect} = \mathbb{E}[Y(1, M(1))] - \textcolor{blue}{\mathbb{E}[Y(1, M(0))]}
\]

Here, the counterfactual quantities not directly realised in the data
are again highlighted in blue:
\(\textcolor{blue}{\mathbb{E}[Y(1, M(0))]}\). Notice we subtract this
term from the potential outcomes when \(A = 1\), recalling
\(\mathbb{E}[Y(1, M(1))] = \mathbb{E}[Y(1)]\).

By rearranging this decomposition, we find that the total effect (TE) is
the sum of the NDE and NIE. This is shown by adding and subtracting the
term \(\textcolor{blue}{\mathbb{E}[Y(1, M(0))]}\) to our equation:

\[
\text{Total Effect (TE)} = \underbrace{\bigg\{\mathbb{E}[Y(1, M(1))] - \textcolor{blue}{\mathbb{E}[Y(1, M(0))]}\bigg\}}_{\text{Natural Indirect Effect (NIE)}} + \underbrace{\bigg\{\textcolor{blue}{\mathbb{E}[Y(1, M(0))]} - \mathbb{E}[Y(0, M(0))]\bigg\}}_{\text{Natural Direct Effect (NDE)}}
\]

\begin{table}

\caption{\label{tbl-mediationpuzzle}In causal mediation, the quantities
that we require to obtain natural direct and indirect effects, namely
\(\mathbb{E}[Y\big(1,M(0)\big)]\), cannot be experimentally observed
because we cannot treat someone and observe the level of their mediator
if they were not treated.}

\centering{

\mediationpuzzle

}

\end{table}%

Table~\ref{tbl-mediationpuzzle} presents a conceptual challenge for
causal mediation analysis. Suppose we randomise a binary treatment
\(A \in \{0,1\}\). Although randomising \(A\) does not ensure there is
no confounding of the mediator/outcome path, we assume no unmeasured
confounding for either the treatment or the mediator. (We will relax
this assumption in the next section.)

Table~\ref{tbl-mediationpuzzle} \(\mathcal{G}_1\) is a Single World
Intervention Template (SWIT), which generates Single World Intervention
Graphs (SWIGs) for each condition.

Table~\ref{tbl-mediationpuzzle} \(\mathcal{G}_2\) presents
counterfactual outcomes for condition \(A = 0\); here, the natural value
of \(M\) is \(M(a = 0)\), and the counterfactual outcome is given by
\(Y\big(\textcolor{cyan}{0}, M(\textcolor{cyan}{0})\big)\).

Table~\ref{tbl-mediationpuzzle} \(\mathcal{G}_3\) presents
counterfactual outcomes for condition \(A = 1\); here, the natural value
of \(M\) is \(M(a = 1)\), and the counterfactual outcome is given by
\(Y\big(\textcolor{cyan}{1}, M(\textcolor{cyan}{1})\big)\).

These Single World Intervention Graphs clarify that we cannot identify
natural direct and indirect effects from observations on individual
units under treatment because \(\mathbb{E}[Y(1, M(0))]\) is not
observable. (\citeproc{ref-shi2021}{Shi et al., 2021};
\citeproc{ref-steen2017}{Steen et al., 2017};
\citeproc{ref-valeri2014}{Valeri et al., 2014};
\citeproc{ref-vanderweele2015}{VanderWeele, 2015};
\citeproc{ref-vanderweele2014a}{T. VanderWeele \& Vansteelandt, 2014};
\citeproc{ref-vansteelandt2012}{Vansteelandt et al., 2012}). Expressing
these quantities requires a counterfactual framework. Here, we see that
a counterfactual formulation of mediation analysis has made the familiar
strange. However, under assumptions, we can sometimes recover natural
direct and indirect effects from data
(\citeproc{ref-vanderweele2015}{VanderWeele, 2015}), given that our
interest is in contrasts obtained for the target population, not for
individuals, where we assume no causal effects are directly observed.

\subsubsection{Assumptions of Causal Mediation
Analysis}\label{assumptions-of-causal-mediation-analysis}

Table~\ref{tbl-medationassumptions} \(\mathcal{G}_1\) presents a Single
World Intervention Template (SWIT) that specifies the assumptions
required for inferring natural direct and indirect effects. This
template highlights that, when estimating natural mediated effects, we
only intervene on the treatment. Therefore, we must infer the mediated
effect of the treatment under the condition that the mediator is set to
the level it would naturally take under the control condition.

Additionally, Table~\ref{tbl-medationassumptions} \(\mathcal{G}_1\) also
clarifies the assumptions needed for inferring controlled direct
effects, where the mediator is fixed to a level specified by the
investigators. In this scenario, we obtain causal contrasts by fixing
variables to specific states.

Consider the hypothesis that cultural beliefs in `Big Gods' influence
social complexity, with political authority mediating. Assuming we have
well-defined interventions and outcomes, what requirements are necessary
to decompose this causal effect into natural direct and indirect
effects?

\begin{table}

\caption{\label{tbl-medationassumptions}\textbf{Assumptions of Causal
Mediation Analysis}}

\centering{

\mediationassumptionsswig

}

\end{table}%

\begin{enumerate}
\def\labelenumi{\arabic{enumi}.}
\item
  \textbf{No Unmeasured Exposure-Outcome Confounder}

  This requirement is expressed as: \[
  Y(a, m) \coprod A \mid L
  \] After accounting for the covariates in set \(L\), there must be no
  unmeasured confounders influencing cultural beliefs in Big Gods
  (\(A\)) and social complexity (\(Y\)). For example, if our study
  examines the causal effect of cultural beliefs in Big Gods on social
  complexity, and the covariates in \(L\) include factors such as
  geographic location and historical context, we need to ensure that
  these covariates effectively block any confounding paths between \(A\)
  and \(Y\). The relevant path in Table~\ref{tbl-medationassumptions}
  \(\mathcal{G}_1\) is the confounder of the path \(A \rightarrow Y\).
\item
  \textbf{No Unmeasured Mediator-Outcome Confounder}

  This requirement is expressed as: \[
  Y(a, m) \coprod M \mid Z
  \] After controlling for the covariate set \(Z\), we must ensure that
  no other unmeasured confounders affect political authority (\(M\)) and
  social complexity (\(Y\)). For instance, if trade networks affect both
  political authority and social complexity, we must account for trade
  networks to block the path linking our mediator and outcome. The
  relevant path in Table~\ref{tbl-medationassumptions} \(\mathcal{G}_1\)
  is the confounder of the path \(M \rightarrow Y\).
\item
  \textbf{No Unmeasured Exposure-Mediator Confounder}

  This requirement is expressed as: \[
  M(a) \coprod A \mid Q
  \] After controlling for the covariate set \(Q\), we must ensure that
  no additional unmeasured confounders affect cultural beliefs in Big
  Gods (\(A\)) and political authority (\(M\)). For example, the
  capability to construct large ritual theatres may influence both the
  belief in Big Gods and the level of political authority. If we have
  indicators for this technology measured before the emergence of Big
  Gods (these indicators being \(Q\)), we must assume that accounting
  for \(Q\) closes the backdoor path between the exposure and the
  mediator. The relevant path in Table~\ref{tbl-medationassumptions}
  \(\mathcal{G}_1\) is the confounder of the path \(A \rightarrow M\).
\item
  \textbf{No Mediator-Outcome Confounder Affected by the Exposure}

  This assumption requires that there are no unmeasured confounders of
  the mediator-outcome relationship that are themselves affected by the
  exposure \(A\). Such confounders cannot be adjusted for without
  introducing bias.

  \textbf{Clarification:}

  \begin{itemize}
  \tightlist
  \item
    There must be no variables \(Z\) that both affect \(M\) and \(Y\),
    are affected by \(A\), and are not appropriately controlled for.
  \item
    This assumption cannot be easily expressed using independence
    notation but is crucial for the identification of natural direct and
    indirect effects.
  \end{itemize}

  The relevant path in Table~\ref{tbl-medationassumptions}
  \(\mathcal{G}_1\) involves confounders of the \(M \rightarrow Y\) path
  that are influenced by \(A\).
\end{enumerate}

Satisfying Assumption 4 imposes considerable demands on causal mediation
analysis. When the exposure influences a confounder of the mediator and
the outcome, we face a dilemma. Without adjusting for this confounder, a
backdoor path between the mediator and the outcome remains open,
introducing bias. However, adjusting for it can block part of the effect
of the exposure on the mediator, also leading to bias. In this setting,
we cannot recover the natural direct and indirect effects from
observational data. We may need to settle for investigating controlled
direct effects, estimate jointly mediated effects of \(Z\) and \(M\)
together, or consider alternative methods as suggested by VanderWeele
and others (\citeproc{ref-Diaz2023}{Dı́az et al., 2023};
\citeproc{ref-robins2010alternative}{Robins \& Richardson, 2010};
\citeproc{ref-vanderweele2014effect}{VanderWeele et al., 2014};
\citeproc{ref-vanderweele2015}{VanderWeele, 2015};
\citeproc{ref-vanderweele2017mediation}{VanderWeele \& Tchetgen
Tchetgen, 2017}; \citeproc{ref-vo2024recanting}{Vo et al., 2024}).

Notice that even when Assumptions 1--4 are satisfied, natural direct
effect estimates and natural indirect effect estimates require
conceptualizing a counterfactual that is never directly observed on any
individual, namely: \[
Y\big(1, M(0)\big)
\] Such effects are only identified in distribution (refer to
VanderWeele (\citeproc{ref-vanderweele2015}{2015})).

\subsubsection{Controlled Direct
Effects}\label{controlled-direct-effects}

Consider another identification challenge, as described in template
Table~\ref{tbl-medationassumptions} \(\mathcal{G}_1\). Suppose we aim to
understand the effect of a stringent pandemic lockdown (\(A\)) on
psychological distress (\(Y\)), focusing on trust in government (\(M\))
as a mediator. Further, suppose that pandemic lockdowns may plausibly
influence attitudes towards the government through pathways that also
affect psychological distress. For instance, people might trust the
government more when it provides income relief payments (\(Z\)), which
may also reduce psychological distress.

Under the rules of d-separation, conditioning on income relief payments
(\(Z\)) could block necessary causal paths:

\begin{itemize}
\item
  \textbf{If we adjust for \(Z\):}

  \begin{itemize}
  \tightlist
  \item
    We might block part of the effect of \(A\) on \(M\) and \(Y\) that
    operates through \(Z\), potentially biasing the estimation of the
    natural indirect effect.
  \item
    The paths \(A \rightarrow Z \rightarrow M\) and
    \(A \rightarrow Z \rightarrow Y\) are blocked.
  \end{itemize}
\item
  \textbf{If we do not adjust for \(Z\):}

  \begin{itemize}
  \tightlist
  \item
    \(Z\) acts as an unmeasured confounder of the \(M \rightarrow Y\)
    relationship since \(Z\) influences both \(M\) and \(Y\).
  \item
    The path \(M \leftarrowred Z \rightarrowred Y\) remains open,
    introducing bias.
  \end{itemize}
\end{itemize}

In such a scenario, it is not feasible to consistently decompose the
total effect of the exposure (pandemic lockdowns) on the outcome
(psychological distress) into natural indirect and direct effects.
However, if all other assumptions hold, we might obtain an unbiased
estimate for the controlled direct effect of pandemic lockdowns on
psychological distress at a fixed level of government trust.

Table~\ref{tbl-medationassumptions} \(\mathcal{G}_2\) presents the
weaker assumptions required to identify a controlled direct effect. We
might examine the effect of the pandemic lockdown if we could intervene
and set everyone's trust in government to, say, one standard deviation
above the baseline, compared with fixing trust in government to the
average level at baseline. We might use modified treatment policies that
specify interventions as functions of the data. For instance, we might
investigate interventions that `shift' only those whose mistrust of
government was below the mean level of trust at baseline and compare
these potential outcomes with those observed. Asking and answering
precisely formulated causal questions such as this might lead to clearer
policy advice, especially in situations where policymakers can influence
public attitudes towards the government; see
(\citeproc{ref-duxedaz2021}{Díaz et al., 2021};
\citeproc{ref-hoffman2022}{Hoffman et al., 2022},
\citeproc{ref-hoffman2023}{2023}; \citeproc{ref-williams2021}{Williams
\& Díaz, 2021}).

In any case, I hope this discussion of causal mediation analysis
clarifies that it would be unwise to simply examine the coefficients
obtained from structural equation models and interpret them as
meaningful in statistical mediation analysis. To answer any causal
question, we must first state it with respect to clearly defined
counterfactual contrasts and a target population. Once we state our
causal question, we find we have no guarantees that the coefficients
from statistical models are straightforwardly interpretable
(\citeproc{ref-vanderweele2015}{VanderWeele, 2015}).

For those interested in statistical estimators for causal mediation
analysis, I recommend visiting the CMAverse website
\url{https://bs1125.github.io/CMAverse/articles/overview.html}, accessed
12 October 2023. This excellent resource provides comprehensive
documentation, software, and practical examples, including sensitivity
analyses. Next, we will consider more complex scenarios that involve
feedback between treatments and confounders across multiple time
points---settings in which traditional statistical methods also fail to
provide valid causal inferences.

\subsection{Part 4: Time-fixed and Time-Varying Sequential Treatments
(Treatment Strategies, Modified Treatment Policies)}\label{id-sec-4}

Our discussion of causal mediation analysis focused on how effects from
two sequential treatments -- the initial treatment and that of a
mediator affected by the treatment -- may combine to affect an outcome.

This concept can be expanded to investigate the causal effects of
multiple sequential exposures, referred to as `treatment regimes',
`treatment strategies', or `modified treatment policies'. Researchers
often use longitudinal growth and multi-level models in many human
sciences, where longitudinal data are collected. How may we identify
such effects? What do they mean?

As before, to answer a causal question, we must first clearly state it.
This involves specifying the counterfactual contrast of interest,
including the treatments to be compared, the scale on which the contrast
will be computed, and the target population for whom inferences are
valid. Without such clarity, our statistical models are often
uninterpretable.

\subsubsection{Worked Example: Does Marriage Affect
Happiness?}\label{worked-example-does-marriage-affect-happiness}

Richard McElreath considers whether marriage affects happiness and
provides a simulation to clarify how age structure complicates causal
inferences (\citeproc{ref-mcelreath2020}{McElreath, 2020, pp.
123--144}). We expand on this example by first clearly stating a causal
question and then considering how time-varying confounding invalidates
the use of standard estimation methods such as multi-level regression.

Let \(A_t = 1\) denote the state of being married at time \(t\) and
\(A_t = 0\) denote the state of not being married, where
\(t \in \{0, 1, \tau\}\) and \(\tau\) is the end of the study.
\(Y_\tau\) denotes happiness at the end of study. For simplicity, assume
this concept is well-defined and measured without error.

The table below reveals four treatment strategies and six causal
contrasts that we may estimate for each treatment strategy combination.

\textbf{Treatment Strategies:}

\begin{longtable}[]{@{}ll@{}}
\caption{Table outlines four fixed treatment regimens and six causal
contrasts in time-series data where exposure
varies}\label{tbl-regimens-marriage}\tabularnewline
\toprule\noalign{}
Regime & Counterfactual Outcome \\
\midrule\noalign{}
\endfirsthead
\toprule\noalign{}
Regime & Counterfactual Outcome \\
\midrule\noalign{}
\endhead
\bottomrule\noalign{}
\endlastfoot
Always married & \(Y(1,1)\) \\
Never married & \(Y(0,0)\) \\
Divorced & \(Y(1,0)\) \\
Gets married & \(Y(0,1)\) \\
\end{longtable}

\textbf{Causal Contrasts:}

\begin{longtable}[]{@{}
  >{\raggedright\arraybackslash}p{(\columnwidth - 2\tabcolsep) * \real{0.5000}}
  >{\raggedright\arraybackslash}p{(\columnwidth - 2\tabcolsep) * \real{0.5000}}@{}}
\caption{Table outlines four fixed treatment regimens and six causal
contrasts in time-series data where exposure
varies.}\label{tbl-regimens-marriage-contrasts}\tabularnewline
\toprule\noalign{}
\begin{minipage}[b]{\linewidth}\raggedright
Comparison
\end{minipage} & \begin{minipage}[b]{\linewidth}\raggedright
Counterfactual Outcome
\end{minipage} \\
\midrule\noalign{}
\endfirsthead
\toprule\noalign{}
\begin{minipage}[b]{\linewidth}\raggedright
Comparison
\end{minipage} & \begin{minipage}[b]{\linewidth}\raggedright
Counterfactual Outcome
\end{minipage} \\
\midrule\noalign{}
\endhead
\bottomrule\noalign{}
\endlastfoot
Always married vs.~Never married & \(\mathbb{E}[Y(1,1) - Y(0,0)]\) \\
Always married vs.~Divorced & \(\mathbb{E}[Y(1,1) - Y(1,0)]\) \\
Always married vs.~Gets married & \(\mathbb{E}[Y(1,1) - Y(0,1)]\) \\
Never married vs.~Divorced & \(\mathbb{E}[Y(0,0) - Y(1,0)]\) \\
Never married vs.~Gets married & \(\mathbb{E}[Y(0,0) - Y(0,1)]\) \\
Divorced vs.~Gets married & \(\mathbb{E}[Y(1,0) - Y(0,1)]\) \\
\end{longtable}

To answer our causal question, we need to:

\begin{enumerate}
\def\labelenumi{\arabic{enumi}.}
\tightlist
\item
  \textbf{Specify Treatments}: Define the treatment strategies being
  compared (e.g., always married vs.~never married).
\item
  \textbf{Define the Contrast}: State the counterfactual contrast of
  interest (e.g., \(\mathbb{E}[Y(1,1) - Y(0,0)]\)).
\item
  \textbf{Identify the Population}: Specify the population for which the
  inferences are valid (e.g., adults aged 20-40).
\end{enumerate}

\begin{table}

\caption{\label{tbl-swigtabledeveloped}Single World Intervention Graph
for Sequential Treatments.}

\centering{

\swigtabledeveloped

}

\end{table}%

\subsubsection{Time-Varying Confounding with Treatment-confounder
Feedback}\label{time-varying-confounding-with-treatment-confounder-feedback}

Table~\ref{tbl-swigtabledeveloped} \(\mathcal{G}_1\) and
Table~\ref{tbl-swigtabledeveloped} \(\mathcal{G}_2\) represent two
subsets of possible confounding structures for a treatment regime
conducted over two intervals. Covariates in \(L_t\) denote measured
confounders, and \(U\) denotes unmeasured confounders. \(A_t\) denotes
the treatment, `Marriage Status,' at time \(t\). \(Y\) denotes
`Happiness' measured at the end of the study.

Consider the structure of confounding presented in
Table~\ref{tbl-swigtabledeveloped} \(\mathcal{G}_1\). To close the
backdoor path from \(A_1\) to \(Y\), we must condition on \(L_0\). To
close the backdoor path from \(A_2\) to \(Y\), we must condition on
\(L_2\). However, \(L_2\) is a collider of treatment \(A_1\) and
unmeasured confounders, such that conditioning on \(L_2\) opens a
backdoor path between \(A_1\) and \(Y\). This path is highlighted in
red:
\(A_1 \associationred L_2({\mathbf{\tilde{a_1}}}) \associationred U \associationred ({\mathbf{\tilde{a_1}, \tilde{a_2}}})\).

If Table~\ref{tbl-swigtabledeveloped} \(\mathcal{G}_1\) faithfully
represents causality, it might seem that we cannot obtain valid
inferences for any of the six causal contrasts we have defined. Indeed,
using standard methods, we could not obtain valid causal inferences.
However, J. Robins (\citeproc{ref-robins1986}{1986}) first described a
consistent estimation function that can be constructed where there is
time-varying confounding (\citeproc{ref-hernan2004STRUCTURAL}{Hernán et
al., 2004}; \citeproc{ref-robins2008estimation}{J. Robins \& Hernan,
2008}).

Table~\ref{tbl-swigtabledeveloped} \(\mathcal{G}_3\) presents a Single
World Intervention Template that clarifies how identification may be
obtained in fixed treatment regimes where there is time-varying
confounding as observed in Table~\ref{tbl-swigtabledeveloped}
\(\mathcal{G}_1\). When constructing a Single World Intervention Graph
(or Template), we obtain factorisations for counterfactual outcomes
under a specific treatment regime by employing `node-splitting,' such
that all nodes following an intervention are relabelled as
counterfactual states under the preceding intervention. After
node-splitting, a fixed intervention is no longer a random variable.
Thus, under fixed treatment regimes, the counterfactual states that
follow an intervention are independent of the states that occur before
node-splitting if there are no backdoor paths into the random partition
of the node that has been split.

If all backdoor paths are closed into the random partitions of the nodes
on which interventions occur, we can graphically verify that the
treatment is independent of the counterfactual outcome for that
intervention node. Where there are multiple interventions, we ensure
sequential exchangeability at the following node---which we likewise
split and relabel---by closing all backdoor paths between the random
portion of the following treatment node. We have sequential independence
if, for each intervention node, all backdoor paths are closed (refer to
Robins \& Richardson (\citeproc{ref-robins2010alternative}{2010});
Richardson \& Robins (\citeproc{ref-richardson2013swigsprimer}{2013b});
Richardson \& Robins (\citeproc{ref-richardson2023potential}{2023})).

The Single World Intervention Template
Table~\ref{tbl-swigtabledeveloped} \(\mathcal{G}_3\) makes it clear that
sequential identification may be obtained. \(A_1\) is d-separated from
\(Y\) by conditioning on \(L_0\); \(A_2\) is d-separated from \(Y\) by
conditioning on \(L_2\).

Notice that we cannot estimate the combined effect of a treatment
strategy over \(A_1\) and \(A_2\) by employing regression, multi-level
regression, statistical structural equation models, or propensity score
matching. However, special estimators may be constructed (refer to J.
Robins (\citeproc{ref-robins1986}{1986}); J. Robins \& Hernan
(\citeproc{ref-robins2008estimation}{2008}); Van Der Laan \& Rose
(\citeproc{ref-vanderlaan2011}{2011}); Dı́az et al.
(\citeproc{ref-diaz2021nonparametric}{2021})). For recent reviews of
special estimators refer to Hernan \& Robins
(\citeproc{ref-hernan2024WHATIF}{2024}); Chatton et al.
(\citeproc{ref-chatton2020}{2020}); Van Der Laan \& Rose
(\citeproc{ref-vanderlaan2018}{2018}); Chatton \& Rohrer
(\citeproc{ref-chatton2024causal}{2024})).

\subsubsection{\texorpdfstring{Time-Varying Confounding \emph{without}
Treatment-Confounder
Feedback}{Time-Varying Confounding without Treatment-Confounder Feedback}}\label{time-varying-confounding-without-treatment-confounder-feedback}

Consider how we may have time-varying confounding in the absence of
treatment-confounder feedback. Suppose we are interested in computing a
causal effect estimate for a two-treatment `marriage' intervention on
`happiness'. We assume that all variables are well-defined, that
`marriage' can be intervened upon, that we have specified a target
population, and that our questions are scientifically interesting. Here,
we focus on the challenges in addressing certain causal questions with
time-varying confounding without treatment confounder feedback.
Table~\ref{tbl-swigtabledeveloped} \(\mathcal{G}_1\) presents such a
structure of time-varying confounding.

Let \(U_{AL}\) denote an over-confident personality, an unmeasured
variable that is causally associated with decisions to marry early and
with income. We do not suppose that \(U_{AL}\) affects happiness. Taken
in isolation, \(U_{AL}\) is not a confounder.

Let \(L_t\) denote income. Suppose that income affects whether one stays
married. For example, suppose it takes wealth to divorce. We can sharpen
focus on risks of time-varying confounding by assuming that income
itself does not affect happiness. Even with this weaker assumption, we
shall see a problem arises.

Let \(U_{AY}\) denote a common cause of income, \(L_2\), and of
happiness at the end of the study, \(Y\). An example of such a
confounder might be educational opportunity, which by supposition
affects both income and happiness. Table~\ref{tbl-swigtabledeveloped}
\(\mathcal{G}_2\) presents the confounding structure for this problem in
its simplest form. To declutter, we remove the baseline node, \(L_0\),
which we assume to be measured.

Notice that in this example there is no treatment-confounder feedback.
We have not imagined that marriage affects wealth, or even that wealth
affects happiness. Nevertheless, there is confounding. To obtain valid
causal inference for the effect of \(A_2\) on \(Y\), we must adjust for
\(L_2\) {[}otherwise:
\(A_t\associationred L_t \associationred U_{AL}\associationred Y(a)\){]}.
However, \(L_2\) is a collider of \(U_{AL}\) and \(U_{AY}\). In this
setting, adjusting for \(L_2\) opens the path:

\[
A_1 \associationred U_{AL} \associationred L_2({\mathbf{\tilde{a_1}}})  \associationred U_{AY} \associationred Y({\mathbf{\tilde{a_1}, \tilde{a_2}}})
\]

We have confounding without treatment-confounder feedback (refer to
Hernan \& Robins (\citeproc{ref-hernan2024WHATIF}{2024}))

Table~\ref{tbl-swigtabledeveloped} \(\mathcal{G}_4\) clarifies that
sequential exchangeability can be obtained in the fixed treatment
regime. To estimate the effect of \(A_2\) on \(Y\), we must condition on
\(L_2\). When estimating the effect of \(A_1\) on \(Y\), all backdoor
paths are closed because \(L_2({\mathbf{\tilde{a_1}}})\) is a collider,
and \(A_0 \coprod Y({\mathbf{\tilde{a_1}, \tilde{a_2}}})\). Note that
because a Singal World Intervention Template does not represent the
joint distributions of more than one treatment, treatment sequence, or
time-varying treatment, to evaluate the conditional independences we
must specify the interventions of interest for \(A_1\) and \(A_2\). That
is, we would need to evaluate at least two Single World Intervention
Graphs that correspond to the treatment level we wish to contrast. Note
further that because there is time-varying confounding, we cannot use
standard estimators such as multi-level regressions or structural
equation models. Estimation must occur stepwise (refer, for example to
Dı́az et al. (\citeproc{ref-diaz2021nonparametric}{2021}), Williams \&
Díaz (\citeproc{ref-williams2021}{2021}).{]}

\subsubsection{Confounding Under Dynamic Treatment Strategies (Modified
Treatment
Policies)}\label{confounding-under-dynamic-treatment-strategies-modified-treatment-policies}

Suppose we were interested in the population average effect of divorce
on happiness if divorce was only permitted for those with incomes.

This question is easy to ask but deceptively difficult to answer. For
example, we cannot fit an interaction of time \(\times\) social status
\(\times\) marriage status because marital status might affect personal
wealth. Yet even if marriage did not affect personal wealth, as in
Table~\ref{tbl-swigtabledeveloped} \(\mathcal{G}_4\), regression would
not produce valid estimates for the counterfactual question we asked.

Suppose further that we want to consider the following fixed treatment
strategy. Define the treatment policy as follows:

\(g^\phi(\cdot)\): remain married for at least two additional years
beyone baseline:

\[
A_t^{+}(\mathbf{g}^\phi) = \mathbf{g}^\phi (A_{t}) = \begin{cases} 
   a_{1} = 1 & \\ 
   a_{2} = 1 &   
\end{cases}
\]

This regime is identified. The setting is identical to
Table~\ref{tbl-swigtabledeveloped} \(\mathcal{G}_3\) with no unmeasured
variables and no arrow from \(A_1\) to \(L_2\).

However, every causal contrast requires a comparison of at least two
interventions.

Next, define the treatment policy \(g^{\lambda}(\cdot)\) as divorce only
if one's personal wealth is at least 50\% greater than average and one
would have divorced in the absence of intervention; otherwise, enforce
marriage:

\[
A_t^{+}(\mathbf{g}^{\lambda}) = \mathbf{g}^{\lambda}(A_{t}) = \begin{cases} 
   a_{1} = 0 & \text{if income} > 1.5 \times  \mu_{\text{income}} \& A_1 = 0 \\ 
   a_{2} = 0 & \text{if income} > 1.5 \times  \mu_{\text{income}} \& A_2 = 0 \\ 
   a_{t} = 1 & \text{otherwise} 
\end{cases}
\]

Notice that for the treatment policy \(\mathbf{g}^\lambda(\cdot)\),
treatment is computed as a function of income at both the natural value
of \(A_t\) and of wealth \(L_t\). Again, to declutter our graphs, we
leave \(L\) at baseline off the graph, noting that adjustment at
baseline does not change the confounding structure.

Template Table~\ref{tbl-swigtabledeveloped} \(\mathcal{G}_5\) presents
this confounding structure. To convey the dependence of the fixed node
on covariate history under treatment, we use Richardson \& Robins
(\citeproc{ref-richardson2013}{2013a})`s conventions and draw a dashed
line (\(\rightarrowdottedgreen\)) to indicate paths from the variables
on which the time-varying treatment regime depends on the deterministic
portion of the intervention node. This strategy clarifies that setting
the treatment level requires information about prior variables,
including the 'natural value of the treatment' in the absence of any
intervention (\citeproc{ref-young2014identification}{Young et al.,
2014}).

The reason for noting these dependencies on a Single World Intervention
Graph (SWIG) is that such dependencies impose stronger identification
assumptions. At every time \(t\), \(A_t\) must be conditionally
independent not only of the potential outcome at the end of the study
but also of any future variable that might lead to a non-causal
association between future treatments and the potential outcome at the
end of the study. We clarify this additional requirement for
\emph{strong sequential exchangeability} in the next section.

\paragraph{Identification of Dynamic Time-Varying Treatment Strategies
using an Extension of Robin's Dynamic
G-formula}\label{identification-of-dynamic-time-varying-treatment-strategies-using-an-extension-of-robins-dynamic-g-formula}

Supplementary materials \textbf{S3} describes Richardson and Robins
extension of J. Robins (\citeproc{ref-robins1986}{1986}) dynamic
g-formula. Essentially, the algorithm can be stated as follows.

\textbf{Step 1}: where \(\mathbf{g}(\cdot)\) is a modified treatment
policy, identify all the variables that influence the outcome
\(Y(\mathbf{g})\), excluding those that are current or past treatment
variables or covariates.

\textbf{Step 2}: For each treatment at time \(t\), check if the
treatment is independent of the variables identified in \textbf{Step 1},
after accounting for past covariates and treatments, in each Single
World Intervention Graph where the treatment values are fixed. This step
amounts to removing the dotted green arrows from the dynamic Single
World Intervention Graph in Table~\ref{tbl-swigtabledeveloped}
\(\mathcal{G}_5\), and doing so gives us
Table~\ref{tbl-swigtabledeveloped} \(\mathcal{G}_4\). For each time
point, we recover a set of future counterfactual variables that includes
the potential outcome for the treatment regime under consideration,
\(Y(\mathbf{g})\), and other variables that the treatment might affect,
including the natural value of future treatments. All backdoor paths
must be closed to each member of this set of counterfactual variables.
We call the more stringent assumptions required for identification in
time-varying treatments (or equivalently longitudinal modified treatment
policies (\citeproc{ref-diaz2023lmtp}{Díaz et al., 2023})):
\textbf{strong sequential exchangeability}.

Where:

\begin{enumerate}
\def\labelenumi{\arabic{enumi}.}
\tightlist
\item
  \textbf{\(\mathbb{Z}_t(\mathbf{a}^*)\)}: denotes the subset of
  vertices in \(\mathcal{G}(\mathbf{a}^*)\) corresponding to
  \(\mathbb{Z}_t(\mathbf{g})\).
\item
  \textbf{\(A_t(\mathbf{a}^*) = a^*_t\)}: denotes the specific value of
  the treatment variable at time \(t\) under the intervention
  \(\mathbf{a}^*\).
\item
  \textbf{\(\bar{\mathbb{L}}_t(\mathbf{a}^*)\)}: denotes the set of
  covariates up to time \(t\) under the intervention \(\mathbf{a}^*\).
\item
  \textbf{\(\bar{\mathbb{A}}_{t-1}(\mathbf{a}^*)\)}: denotes the set of
  past treatment variables up to time \(t-1\) under the intervention
  \(\mathbf{a}^*\). :
\end{enumerate}

Applying Richardson \& Robins (\citeproc{ref-richardson2013}{2013a})'s
dynamic extended g-formula, we obtain the following sets of future
variables for which each current treatment must be independent:

\[
\begin{aligned}
\mathbb{Z}(\mathbf{g}) &= \{A_1, L_1(\mathbf{g}), A_1(\mathbf{g}), A_2(\mathbf{g}), Y(\mathbf{g})\} \\
\mathbb{Z}_1(\mathbf{g}) &= \{A_1(\mathbf{g}), L_1(\mathbf{g}), Y(\mathbf{g})\} \\
\mathbb{Z}_2(\mathbf{g}) &= \{Y(\mathbf{g})\}
\end{aligned}
\]

Having determined which variables must remain conditionally independent
of each treatment in a sequence of dynamic treatments to be compared, we
then consider whether strong sequential exchangeability holds. We do
this by inspecting template Table~\ref{tbl-swigtabledeveloped}
\(\mathcal{G}_4\) (recall this is Table~\ref{tbl-swigtabledeveloped}
\(\mathcal{G}_5\) without the dashed green arrows). On inspection of
\(\mathcal{G}_4\) (the dynamic SWIG without dashed green arrows), we
discover that this dynamic treatment strategy is not identified because
we have the following open backdoor path:

\[
A_1 \associationred U_{AL} \associationred L_2(\mathbf{g})
\]

We also have:

\[
A_1 \associationred U_{AL} \associationred L_2(\mathbf{g}) \associationred A_2(\mathbf{g})
\]

Strong sequential exchangeability fails for \(A_1\). We might consider
lowering our sights and estimating a fixed or time-varying treatment
strategy that can be identified.

Note that certain time-varying treatment strategies impose weaker
assumptions than time-fixed strategies. For example, with a continuous
intervention, we might consider intervening only if the observed
treatment does not reach a specific threshold, such as:

\[
\mathbf{g}^{\phi} (A_i) = \begin{cases}  
\mu_A & \text{if } A_i < \mu_A \\ 
A_i & \text{otherwise} 
\end{cases}
\]

This is a weaker intervention than setting everyone whose natural value
of treatment is above this threshold to precisely the threshold's value:

\[
\mathbf{g}^{\lambda} (A_i) = \begin{cases}   
\mu_A & \text{if } A_i \neq \mu_A \\ 
A_i & \text{otherwise} 
\end{cases}
\]

Whereas \(\mathbf{g}^{\lambda}\) sets everyone in the population to the
same treatment level, \(\mathbf{g}^{\phi}\) sets only those below a
certain threshold to a fixed level but does not estimate treatment
effects for those above (\citeproc{ref-hoffman2023}{Hoffman et al.,
2023}). We can also write stochastic treatment functions
(\citeproc{ref-diaz2021nonparametric}{Dı́az et al., 2021};
\citeproc{ref-diaz2012population}{Muñoz \& Van Der Laan, 2012};
\citeproc{ref-vanderweele2014a}{T. VanderWeele \& Vansteelandt, 2014};
\citeproc{ref-young2014identification}{Young et al., 2014}), see
supplementary materials \textbf{S4}.

Of course, the details of every problem must be developed in relation to
the scientific context and the practical questions that address gaps in
present science. However, causal inference teaches us that the questions
we ask---seemingly coherent and tractable questions such as whether
marriage makes people happy---demand considerable scrutiny to become
interpretable. When such questions are made interpretable, causal
inference often reveals that answers may elude us, regardless of the
quality and abundance of our data, or even if we randomise
interventions. Modest treatment functions, however, might be more
credible and useful for many scientific and practical questions. Such
functions often cannot be estimated using the models routinely taught in
the human sciences, such as multi-level modelling and statistical
structural equation modelling.

\subsection{Conclusions}\label{id-sec-5}

Philosophical interests in causality are ancient. Democritus once
declared, ``I would rather discover one cause than gain the kingdom of
Persia'' (\citeproc{ref-freeman1948ancilla}{Freeman, 1948}). Hume
provided a general account of causality by referencing counterfactuals:
``\ldots{} where, if the first object had not been, the second never
would have existed'' (\citeproc{ref-hume1902}{Hume, 1902}). However, it
was not until Jerzy Neyman's master's thesis that a quantitative
analysis of causality was formalised
(\citeproc{ref-neyman1923}{Splawa-Neyman, 1990 (orig. 1923)}).
Remarkably, Neyman's work went largely unnoticed until the 1970s, when
Harvard statistician Donald Rubin formalised what became known as the
`Rubin Causal Model' (also the Rubin-Neyman Causal Model)
(\citeproc{ref-holland1986}{Holland, 1986};
\citeproc{ref-rubin1976}{Rubin, 1976}).

In 1986, Harvard statistician James Robins extended the potential
outcomes framework to time-varying treatments, laying the foundation for
powerful new longitudinal data science methods
(\citeproc{ref-robins1986}{J. Robins, 1986}). Judea Pearl introduced
directed acyclic graphs, making identification problems transparent and
accessible to non-specialists (\citeproc{ref-pearl1995}{Pearl, 1995}).
Robins and Richardson extended Pearl's graphical models to evaluate
counterfactual causal contrasts on graphs, building on Robins' earlier
work. Concurrently, the causal revolution in economics opened new,
fertile frontiers in causal data sciences. By the early 2000s, targeted
learning frameworks were being developed
(\citeproc{ref-vanderlaan2011}{Van Der Laan \& Rose, 2011}), along with
causal mediation analysis methods (\citeproc{ref-Diaz2023}{Dı́az et al.,
2023}; \citeproc{ref-pearl2009a}{Pearl, 2009};
\citeproc{ref-robins1992}{Robins \& Greenland, 1992};
\citeproc{ref-rudolph2024mediation}{Rudolph et al., 2024};
\citeproc{ref-vanderweele2015}{VanderWeele, 2015};
\citeproc{ref-vanderweele2014a}{T. VanderWeele \& Vansteelandt, 2014};
\citeproc{ref-vansteelandt2012}{Vansteelandt et al., 2012}), and
techniques for analysing time-varying treatments
(\citeproc{ref-diaz2012population}{Muñoz \& Van Der Laan, 2012};
\citeproc{ref-richardson2013}{Richardson \& Robins, 2013a};
\citeproc{ref-richardson2023potential}{Richardson \& Robins, 2023};
\citeproc{ref-robins1986}{J. Robins, 1986};
\citeproc{ref-robins1999}{Robins et al., 1999};
\citeproc{ref-robins2008estimation}{J. Robins \& Hernan, 2008};
\citeproc{ref-shpitser2022multivariate}{Shpitser et al., 2022};
\citeproc{ref-young2014identification}{Young et al., 2014}).

Readers should note that the causal inference literature contains
vigorous debates at the horizons of discovery. However, there is a
shared consensus about the foundations of causal inference and a common
conceptual and mathematical vocabulary within which to express
disagreements and accumulate progress---a hallmark of a productive
science. Old debates resolve and new debates arise, the hallmark of a
vibrant science.

Despite the progress and momentum of the causal revolution in certain
human sciences, many areas have yet to participate and benefit. The
demands for researchers to acquire new skills, coupled with the
intensive requirement for data collection, have significant implications
for research design, funding, and the accepted pace of scientific
publishing. To foster essential changes in causal inference education
and practice, the human sciences need to shift from a predominantly
output-focused, correlation-reporting culture to a slow, careful,
creative culture that promotes retraining and funds time-series data
collection. Such investments are worthwhile. Much as Darwin's theory
transformed the biological sciences from speculative taxonomy, causal
inference is slowly but steadily transforming the human sciences from
butterfly collections of correlations to causal inferential sciences
capable of addressing the causal questions that animate our curiosities.

\newpage{}

\subsection{Acknowledgements}\label{acknowledgements}

I am grateful to Dr.~Inkuk Kim for checking previous versions of this
manuscript and offering feedback, to two anonymous reviewers and the
editors, Charles Efferson and Ruth Mace, for their constructive
feedback.

Any remaining errors are my own.

\subsection{Conflict of Interest}\label{conflict-of-interest}

The author declares no conflicts of interest

\subsection{Financial Support}\label{financial-support}

This work is supported by a grant from the Templeton Religion Trust
(TRT0418) and RSNZ Marsden 3721245, 20-UOA-123; RSNZ 19-UOO-090. I also
received support from the Max Planck Institute for the Science of Human
History. The Funders had no role in preparing the manuscript or deciding
to publish it.

\subsection{Research Transparency and
Reproducibility}\label{research-transparency-and-reproducibility}

No data were used in this manuscript.

\newpage{}

\subsection{References}\label{references}

\phantomsection\label{refs}
\begin{CSLReferences}{1}{0}
\bibitem[\citeproctext]{ref-angrist2009mostly}
Angrist, J. D., \& Pischke, J.-S. (2009). \emph{Mostly harmless
econometrics: An empiricist's companion}. Princeton University Press.

\bibitem[\citeproctext]{ref-bareinboim2013general}
Bareinboim, E., \& Pearl, J. (2013). A general algorithm for deciding
transportability of experimental results. \emph{Journal of Causal
Inference}, \emph{1}(1), 107--134.

\bibitem[\citeproctext]{ref-barrett2021}
Barrett, M. (2021). \emph{Ggdag: Analyze and create elegant directed
acyclic graphs}. \url{https://CRAN.R-project.org/package=ggdag}

\bibitem[\citeproctext]{ref-bulbulia2023}
Bulbulia, J. A. (2024a). Methods in causal inference part 1: Causal
diagrams and confounding. \emph{Evolutionary Human Sciences}, \emph{6}.
\url{https://osf.io/b23k7}

\bibitem[\citeproctext]{ref-bulbulia2024wierd}
Bulbulia, J. A. (2024b). Methods in causal inference part 3: Measurement
error and external validity threats. \emph{Evolutionary Human Sciences},
\emph{6}. \url{https://osf.io/preprints/psyarxiv/kj7rv}

\bibitem[\citeproctext]{ref-bulbulia_2024_experiments}
Bulbulia, J. A. (2024c). Methods in causal inference part 4: Confounding
in experiments. \emph{Evolutionary Human Sciences}, \emph{6}.
\url{https://osf.io/preprints/psyarxiv/6rnj5}

\bibitem[\citeproctext]{ref-bulbulia2023a}
Bulbulia, J. A., Afzali, M. U., Yogeeswaran, K., \& Sibley, C. G.
(2023). Long-term causal effects of far-right terrorism in {N}ew
{Z}ealand. \emph{PNAS Nexus}, \emph{2}(8), pgad242.

\bibitem[\citeproctext]{ref-chatton2020}
Chatton, A., Le Borgne, F., Leyrat, C., Gillaizeau, F., Rousseau, C.,
Barbin, L., Laplaud, D., Léger, M., Giraudeau, B., \& Foucher, Y.
(2020). G-computation, propensity score-based methods, and targeted
maximum likelihood estimator for causal inference with different
covariates sets: a comparative simulation study. \emph{Scientific
Reports}, \emph{10}(1), 9219.
\url{https://doi.org/10.1038/s41598-020-65917-x}

\bibitem[\citeproctext]{ref-chatton2024causal}
Chatton, A., \& Rohrer, J. M. (2024). The causal cookbook: Recipes for
propensity scores, g-computation, and doubly robust standardization.
\emph{Advances in Methods and Practices in Psychological Science},
\emph{7}(1), 25152459241236149.

\bibitem[\citeproctext]{ref-dahabreh2019}
Dahabreh, I. J., \& Hernán, M. A. (2019). Extending inferences from a
randomized trial to a target population. \emph{European Journal of
Epidemiology}, \emph{34}(8), 719--722.
\url{https://doi.org/10.1007/s10654-019-00533-2}

\bibitem[\citeproctext]{ref-dahabreh2019generalizing}
Dahabreh, I. J., Robins, J. M., Haneuse, S. J., \& Hernán, M. A. (2019).
Generalizing causal inferences from randomized trials: Counterfactual
and graphical identification. \emph{arXiv Preprint arXiv:1906.10792}.

\bibitem[\citeproctext]{ref-darwin1859origin}
Darwin, C. (1859). \emph{On the origin of species: Facsimile of the
first edition}.

\bibitem[\citeproctext]{ref-darwin1887life}
Darwin, C. (1887). \emph{The life and letters of charles darwin, volume
i} (F. Darwin, Ed.). D. Appleton; Company.
\url{https://charles-darwin.classic-literature.co.uk/the-life-and-letters-of-charles-darwin-volume-i/}

\bibitem[\citeproctext]{ref-decoulanges1903}
De Coulanges, F. (1903). \emph{La cité antique: Étude sur le culte, le
droit, les institutions de la grèce et de rome}. Hachette.

\bibitem[\citeproctext]{ref-deffner2022}
Deffner, D., Rohrer, J. M., \& McElreath, R. (2022). A Causal Framework
for Cross-Cultural Generalizability. \emph{Advances in Methods and
Practices in Psychological Science}, \emph{5}(3), 25152459221106366.
\url{https://doi.org/10.1177/25152459221106366}

\bibitem[\citeproctext]{ref-duxedaz2021}
Díaz, I., Williams, N., Hoffman, K. L., \& Schenck, E. J. (2021).
Non-parametric causal effects based on longitudinal modified treatment
policies. \emph{Journal of the American Statistical Association}.
\url{https://doi.org/10.1080/01621459.2021.1955691}

\bibitem[\citeproctext]{ref-diaz2023lmtp}
Díaz, I., Williams, N., Hoffman, K. L., \& Schenck, E. J. (2023).
Nonparametric causal effects based on longitudinal modified treatment
policies. \emph{Journal of the American Statistical Association},
\emph{118}(542), 846--857.
\url{https://doi.org/10.1080/01621459.2021.1955691}

\bibitem[\citeproctext]{ref-diaz2021nonparametric}
Dı́az, I., Hejazi, N. S., Rudolph, K. E., \& Der Laan, M. J. van. (2021).
Nonparametric efficient causal mediation with intermediate confounders.
\emph{Biometrika}, \emph{108}(3), 627--641.

\bibitem[\citeproctext]{ref-Diaz2023}
Dı́az, I., Williams, N., \& Rudolph, K. E. (2023). \emph{Journal of
Causal Inference}, \emph{11}(1), 20220077.
\url{https://doi.org/doi:10.1515/jci-2022-0077}

\bibitem[\citeproctext]{ref-freeman1948ancilla}
Freeman, K. (1948). \emph{Ancilla to the pre-socratic philosophers}
(Reprint edition). Harvard University Press.

\bibitem[\citeproctext]{ref-greenland2009commentary}
Greenland, S. (2009). Commentary: Interactions in epidemiology:
Relevance, identification, and estimation. \emph{Epidemiology},
\emph{20}(1), 14--17.

\bibitem[\citeproctext]{ref-hernan2024WHATIF}
Hernan, M. A., \& Robins, J. M. (2024). \emph{Causal inference: What
if?} Taylor \& Francis.
\url{https://www.hsph.harvard.edu/miguel-hernan/causal-inference-book/}

\bibitem[\citeproctext]{ref-hernan2017SELECTIONWITHOUTCOLLIDER}
Hernán, M. A. (2017). Invited commentary: Selection bias without
colliders \textbar{} american journal of epidemiology \textbar{} oxford
academic. \emph{American Journal of Epidemiology}, \emph{185}(11),
1048--1050. \url{https://doi.org/10.1093/aje/kwx077}

\bibitem[\citeproctext]{ref-hernan2008aObservationalStudiesAnalysedLike}
Hernán, M. A., Alonso, A., Logan, R., Grodstein, F., Michels, K. B.,
Willett, W. C., Manson, J. E., \& Robins, J. M. (2008). Observational
studies analyzed like randomized experiments: An application to
postmenopausal hormone therapy and coronary heart disease.
\emph{Epidemiology}, \emph{19}(6), 766.
\url{https://doi.org/10.1097/EDE.0b013e3181875e61}

\bibitem[\citeproctext]{ref-hernan2004STRUCTURAL}
Hernán, M. A., Hernández-Díaz, S., \& Robins, J. M. (2004). A structural
approach to selection bias. \emph{Epidemiology}, \emph{15}(5), 615--625.
\url{https://www.jstor.org/stable/20485961}

\bibitem[\citeproctext]{ref-hernan2017per}
Hernán, M. A., \& Robins, J. M. (2017). Per-protocol analyses of
pragmatic trials. \emph{N Engl J Med}, \emph{377}(14), 1391--1398.

\bibitem[\citeproctext]{ref-hoffman2023}
Hoffman, K. L., Salazar-Barreto, D., Rudolph, K. E., \& Díaz, I. (2023).
\emph{Introducing longitudinal modified treatment policies: A unified
framework for studying complex exposures}.
\url{https://doi.org/10.48550/arXiv.2304.09460}

\bibitem[\citeproctext]{ref-hoffman2022}
Hoffman, K. L., Schenck, E. J., Satlin, M. J., Whalen, W., Pan, D.,
Williams, N., \& Díaz, I. (2022). Comparison of a target trial emulation
framework vs cox regression to estimate the association of
corticosteroids with COVID-19 mortality. \emph{JAMA Network Open},
\emph{5}(10), e2234425.
\url{https://doi.org/10.1001/jamanetworkopen.2022.34425}

\bibitem[\citeproctext]{ref-holland1986}
Holland, P. W. (1986). Statistics and causal inference. \emph{Journal of
the American Statistical Association}, \emph{81}(396), 945--960.

\bibitem[\citeproctext]{ref-hume1902}
Hume, D. (1902). \emph{Enquiries Concerning the Human Understanding: And
Concerning the Principles of Morals}. Clarendon Press.

\bibitem[\citeproctext]{ref-kessler2002}
Kessler, R. ~C., Andrews, G., Colpe, L. ~J., Hiripi, E., Mroczek, D.
~K., Normand, S.-L. ~T., Walters, E. ~E., \& Zaslavsky, A. ~M. (2002).
Short screening scales to monitor population prevalences and trends in
non-specific psychological distress. \emph{Psychological Medicine},
\emph{32}(6), 959--976. \url{https://doi.org/10.1017/S0033291702006074}

\bibitem[\citeproctext]{ref-tlverse_handbook}
Laan, M. van der, Coyle, J., Hejazi, N., Malenica, I., Phillips, R., \&
Hubbard, A. (2023). \emph{Targeted learning in r: Causal data science
with the tlverse software ecosystem}. Collins Foundation Press.
\url{https://tlverse.org/tlverse-handbook/index.html}

\bibitem[\citeproctext]{ref-lash2020}
Lash, T. L., Rothman, K. J., VanderWeele, T. J., \& Haneuse, S. (2020).
\emph{Modern epidemiology}. Wolters Kluwer.
\url{https://books.google.co.nz/books?id=SiTSnQEACAAJ}

\bibitem[\citeproctext]{ref-mcelreath2020}
McElreath, R. (2020). \emph{Statistical rethinking: A {B}ayesian course
with examples in {R} and {S}tan}. CRC press.

\bibitem[\citeproctext]{ref-montgomery2018}
Montgomery, J. M., Nyhan, B., \& Torres, M. (2018). How conditioning on
posttreatment variables can ruin your experiment and what to do about
It. \emph{American Journal of Political Science}, \emph{62}(3),
760--775. \url{https://doi.org/10.1111/ajps.12357}

\bibitem[\citeproctext]{ref-morgan2014}
Morgan, S. L., \& Winship, C. (2014). \emph{Counterfactuals and causal
inference: Methods and principles for social research} (2nd ed.).
Cambridge University Press.
\url{https://doi.org/10.1017/CBO9781107587991}

\bibitem[\citeproctext]{ref-diaz2012population}
Muñoz, I. D., \& Van Der Laan, M. (2012). Population intervention causal
effects based on stochastic interventions. \emph{Biometrics},
\emph{68}(2), 541--549.

\bibitem[\citeproctext]{ref-neal2020introduction}
Neal, B. (2020). Introduction to causal inference from a machine
learning perspective. \emph{Course Lecture Notes (Draft)}.
\url{https://www.bradyneal.com/Introduction_to_Causal_Inference-Dec17_2020-Neal.pdf}

\bibitem[\citeproctext]{ref-ogburn2021}
Ogburn, E. L., \& Shpitser, I. (2021). Causal modelling: The two
cultures. \emph{Observational Studies}, \emph{7}(1), 179--183.
\url{https://doi.org/10.1353/obs.2021.0006}

\bibitem[\citeproctext]{ref-pearl1995}
Pearl, J. (1995). Causal diagrams for empirical research.
\emph{Biometrika}, \emph{82}(4), 669--688.

\bibitem[\citeproctext]{ref-pearl2009a}
Pearl, J. (2009). \emph{Causality}. Cambridge University Press.

\bibitem[\citeproctext]{ref-pearl2022external}
Pearl, J., \& Bareinboim, E. (2022). External validity: From do-calculus
to transportability across populations. In \emph{Probabilistic and
causal inference: The works of judea pearl} (pp. 451--482).

\bibitem[\citeproctext]{ref-richardson2013}
Richardson, T. S., \& Robins, J. M. (2013a). \emph{Single world
intervention graphs: A primer}.
\url{https://core.ac.uk/display/102673558}

\bibitem[\citeproctext]{ref-richardson2013swigsprimer}
Richardson, T. S., \& Robins, J. M. (2013b). Single world intervention
graphs: A primer. \emph{Second UAI Workshop on Causal Structure
Learning, {B}ellevue, {W}ashington}.
\url{https://citeseerx.ist.psu.edu/document?repid=rep1&type=pdf&doi=07bbcb458109d2663acc0d098e8913892389a2a7}

\bibitem[\citeproctext]{ref-richardson2023potential}
Richardson, T. S., \& Robins, J. M. (2023). Potential outcome and
decision theoretic foundations for statistical causality. \emph{Journal
of Causal Inference}, \emph{11}(1), 20220012.

\bibitem[\citeproctext]{ref-robins1986}
Robins, J. (1986). A new approach to causal inference in mortality
studies with a sustained exposure period---application to control of the
healthy worker survivor effect. \emph{Mathematical Modelling},
\emph{7}(9-12), 1393--1512.

\bibitem[\citeproctext]{ref-robins1992}
Robins, J. M., \& Greenland, S. (1992). Identifiability and
exchangeability for direct and indirect effects. \emph{Epidemiology},
\emph{3}(2), 143--155.

\bibitem[\citeproctext]{ref-robins1999}
Robins, J. M., Greenland, S., \& Hu, F.-C. (1999). Estimation of the
causal effect of a time-varying exposure on the marginal mean of a
repeated binary outcome. \emph{Journal of the American Statistical
Association}, \emph{94}(447), 687--700.
\url{https://doi.org/10.1080/01621459.1999.10474168}

\bibitem[\citeproctext]{ref-robins2010alternative}
Robins, J. M., \& Richardson, T. S. (2010). Alternative graphical causal
models and the identification of direct effects. \emph{Causality and
Psychopathology: Finding the Determinants of Disorders and Their Cures},
\emph{84}, 103--158.

\bibitem[\citeproctext]{ref-robins2008estimation}
Robins, J., \& Hernan, M. (2008). Estimation of the causal effects of
time-varying exposures. \emph{Chapman \& Hall/CRC Handbooks of Modern
Statistical Methods}, 553--599.

\bibitem[\citeproctext]{ref-rubin1976}
Rubin, D. B. (1976). Inference and missing data. \emph{Biometrika},
\emph{63}(3), 581--592. \url{https://doi.org/10.1093/biomet/63.3.581}

\bibitem[\citeproctext]{ref-rudolph2024mediation}
Rudolph, K. E., Williams, N. T., \& Diaz, I. (2024). {Practical causal
mediation analysis: extending nonparametric estimators to accommodate
multiple mediators and multiple intermediate confounders}.
\emph{Biostatistics}, kxae012.
\url{https://doi.org/10.1093/biostatistics/kxae012}

\bibitem[\citeproctext]{ref-shi2021}
Shi, B., Choirat, C., Coull, B. A., VanderWeele, T. J., \& Valeri, L.
(2021). CMAverse: A suite of functions for reproducible causal mediation
analyses. \emph{Epidemiology}, \emph{32}(5), e20--e22.

\bibitem[\citeproctext]{ref-shpitser2022multivariate}
Shpitser, I., Richardson, T. S., \& Robins, J. M. (2022). Multivariate
counterfactual systems and causal graphical models. In
\emph{Probabilistic and causal inference: The works of {J}udea {P}earl}
(pp. 813--852). Association for Computing Machinery (ACM).

\bibitem[\citeproctext]{ref-neyman1923}
Splawa-Neyman, J. (1990 (orig. 1923)). On the application of probability
theory to agricultural experiments. Essay on principles. Section 9.
(1923). \emph{Statistical Science}, \emph{5}(4), 465--472.

\bibitem[\citeproctext]{ref-steen2017}
Steen, J., Loeys, T., Moerkerke, B., \& Vansteelandt, S. (2017).
Medflex: An {R} package for flexible mediation analysis using natural
effect models. \emph{Journal of Statistical Software}, \emph{76}, 1--46.

\bibitem[\citeproctext]{ref-stuart2018generalizability}
Stuart, E. A., Ackerman, B., \& Westreich, D. (2018). Generalizability
of randomized trial results to target populations: Design and analysis
possibilities. \emph{Research on Social Work Practice}, \emph{28}(5),
532--537.

\bibitem[\citeproctext]{ref-suzuki2013counterfactual}
Suzuki, E., Mitsuhashi, T., Tsuda, T., \& Yamamoto, E. (2013). A
counterfactual approach to bias and effect modification in terms of
response types. \emph{BMC Medical Research Methodology}, \emph{13}(1),
1--17.

\bibitem[\citeproctext]{ref-suzuki2020}
Suzuki, E., Shinozaki, T., \& Yamamoto, E. (2020). Causal Diagrams:
Pitfalls and Tips. \emph{Journal of Epidemiology}, \emph{30}(4),
153--162. \url{https://doi.org/10.2188/jea.JE20190192}

\bibitem[\citeproctext]{ref-grf2024}
Tibshirani, J., Athey, S., Sverdrup, E., \& Wager, S. (2024). \emph{Grf:
Generalized random forests}. \url{https://github.com/grf-labs/grf}

\bibitem[\citeproctext]{ref-valeri2014}
Valeri, L., Lin, X., \& VanderWeele, T. J. (2014). Mediation analysis
when a continuous mediator is measured with error and the outcome
follows a generalized linear model. \emph{Statistics in Medicine},
\emph{33}(28), 4875--4890.

\bibitem[\citeproctext]{ref-vanderlaan2011}
Van Der Laan, M. J., \& Rose, S. (2011). \emph{Targeted Learning: Causal
Inference for Observational and Experimental Data}. Springer.
\url{https://link.springer.com/10.1007/978-1-4419-9782-1}

\bibitem[\citeproctext]{ref-vanderlaan2018}
Van Der Laan, M. J., \& Rose, S. (2018). \emph{Targeted Learning in Data
Science: Causal Inference for Complex Longitudinal Studies}. Springer
International Publishing.
\url{http://link.springer.com/10.1007/978-3-319-65304-4}

\bibitem[\citeproctext]{ref-vanderweele2012}
VanderWeele, T. J. (2012). Confounding and Effect Modification:
Distribution and Measure. \emph{Epidemiologic Methods}, \emph{1}(1),
55--82. \url{https://doi.org/10.1515/2161-962X.1004}

\bibitem[\citeproctext]{ref-vanderweele2015}
VanderWeele, T. J. (2015). \emph{Explanation in causal inference:
Methods for mediation and interaction}. Oxford University Press.

\bibitem[\citeproctext]{ref-vanderweele2013}
VanderWeele, T. J., \& Hernan, M. A. (2013). Causal inference under
multiple versions of treatment. \emph{Journal of Causal Inference},
\emph{1}(1), 1--20.

\bibitem[\citeproctext]{ref-vanderweele2012MEASUREMENT}
VanderWeele, T. J., \& Hernán, M. A. (2012). Results on differential and
dependent measurement error of the exposure and the outcome using signed
directed acyclic graphs. \emph{American Journal of Epidemiology},
\emph{175}(12), 1303--1310. \url{https://doi.org/10.1093/aje/kwr458}

\bibitem[\citeproctext]{ref-vanderweele2014}
VanderWeele, T. J., \& Knol, M. J. (2014). A tutorial on interaction.
\emph{Epidemiologic Methods}, \emph{3}(1), 33--72.

\bibitem[\citeproctext]{ref-vanderweele2007}
VanderWeele, T. J., \& Robins, J. M. (2007). Four types of effect
modification: a classification based on directed acyclic graphs.
\emph{Epidemiology (Cambridge, Mass.)}, \emph{18}(5), 561--568.
\url{https://doi.org/10.1097/EDE.0b013e318127181b}

\bibitem[\citeproctext]{ref-vanderweele2017mediation}
VanderWeele, T. J., \& Tchetgen Tchetgen, E. J. (2017). Mediation
analysis with time varying exposures and mediators. \emph{Journal of the
Royal Statistical Society Series B: Statistical Methodology},
\emph{79}(3), 917--938.

\bibitem[\citeproctext]{ref-vanderweele2014effect}
VanderWeele, T. J., Vansteelandt, S., \& Robins, J. M. (2014). Effect
decomposition in the presence of an exposure-induced mediator-outcome
confounder. \emph{Epidemiology}, \emph{25}(2), 300--306.

\bibitem[\citeproctext]{ref-vanderweele2014a}
VanderWeele, T., \& Vansteelandt, S. (2014). Mediation analysis with
multiple mediators. \emph{Epidemiologic Methods}, \emph{2}(1), 95--115.

\bibitem[\citeproctext]{ref-vansteelandt2012}
Vansteelandt, S., Bekaert, M., \& Lange, T. (2012). Imputation
strategies for the estimation of natural direct and indirect effects.
\emph{Epidemiologic Methods}, \emph{1}(1), 131--158.

\bibitem[\citeproctext]{ref-vansteelandt2022a}
Vansteelandt, S., \& Dukes, O. (2022). Assumption-lean inference for
generalised linear model parameters. \emph{Journal of the Royal
Statistical Society Series B: Statistical Methodology}, \emph{84}(3),
657--685.

\bibitem[\citeproctext]{ref-vo2024recanting}
Vo, T.-T., Williams, N., Liu, R., Rudolph, K. E., \& Dıaz, I. (2024).
Recanting twins: Addressing intermediate confounding in mediation
analysis. \emph{arXiv Preprint arXiv:2401.04450}.

\bibitem[\citeproctext]{ref-westreich2010}
Westreich, D., \& Cole, S. R. (2010). Invited commentary: positivity in
practice. \emph{American Journal of Epidemiology}, \emph{171}(6).
\url{https://doi.org/10.1093/aje/kwp436}

\bibitem[\citeproctext]{ref-westreich2017transportability}
Westreich, D., Edwards, J. K., Lesko, C. R., Stuart, E., \& Cole, S. R.
(2017). Transportability of trial results using inverse odds of sampling
weights. \emph{American Journal of Epidemiology}, \emph{186}(8),
1010--1014.

\bibitem[\citeproctext]{ref-westreich2013}
Westreich, D., \& Greenland, S. (2013). The table 2 fallacy: Presenting
and interpreting confounder and modifier coefficients. \emph{American
Journal of Epidemiology}, \emph{177}(4), 292--298.

\bibitem[\citeproctext]{ref-wheatley1971pivot}
Wheatley, P. (1971). \emph{The pivot of the four quarters: A preliminary
enquiry into the origins and character of the ancient chinese city} (p.
602). Aldine Publishing Company.

\bibitem[\citeproctext]{ref-williams2021}
Williams, N. T., \& Díaz, I. (2021). \emph{{l}mtp: Non-parametric causal
effects of feasible interventions based on modified treatment policies}.
\url{https://doi.org/10.5281/zenodo.3874931}

\bibitem[\citeproctext]{ref-wilson2008evolution}
Wilson, D. S. (2008). Evolution and religion: The transformation of the
obvious. In J. Bulbulia, R. Sosis, E. Harris, R. Genet, C. Genet, \& K.
Wyman (Eds.), \emph{The evolution of religion: Studies, theories, \&
critiques} (pp. 23--29). Collins Foundation Press.

\bibitem[\citeproctext]{ref-young2014identification}
Young, J. G., Hernán, M. A., \& Robins, J. M. (2014). Identification,
estimation and approximation of risk under interventions that depend on
the natural value of treatment using observational data.
\emph{Epidemiologic Methods}, \emph{3}(1), 1--19.

\end{CSLReferences}




\end{document}
