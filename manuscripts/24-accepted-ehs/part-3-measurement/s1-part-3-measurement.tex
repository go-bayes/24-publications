% Options for packages loaded elsewhere
\PassOptionsToPackage{unicode}{hyperref}
\PassOptionsToPackage{hyphens}{url}
\PassOptionsToPackage{dvipsnames,svgnames,x11names}{xcolor}
%
\documentclass[
  single column]{article}

\usepackage{amsmath,amssymb}
\usepackage{iftex}
\ifPDFTeX
  \usepackage[T1]{fontenc}
  \usepackage[utf8]{inputenc}
  \usepackage{textcomp} % provide euro and other symbols
\else % if luatex or xetex
  \usepackage{unicode-math}
  \defaultfontfeatures{Scale=MatchLowercase}
  \defaultfontfeatures[\rmfamily]{Ligatures=TeX,Scale=1}
\fi
\usepackage[]{libertinus}
\ifPDFTeX\else  
    % xetex/luatex font selection
\fi
% Use upquote if available, for straight quotes in verbatim environments
\IfFileExists{upquote.sty}{\usepackage{upquote}}{}
\IfFileExists{microtype.sty}{% use microtype if available
  \usepackage[]{microtype}
  \UseMicrotypeSet[protrusion]{basicmath} % disable protrusion for tt fonts
}{}
\makeatletter
\@ifundefined{KOMAClassName}{% if non-KOMA class
  \IfFileExists{parskip.sty}{%
    \usepackage{parskip}
  }{% else
    \setlength{\parindent}{0pt}
    \setlength{\parskip}{6pt plus 2pt minus 1pt}}
}{% if KOMA class
  \KOMAoptions{parskip=half}}
\makeatother
\usepackage{xcolor}
\usepackage[top=30mm,left=25mm,heightrounded,headsep=22pt,headheight=11pt,footskip=33pt,ignorehead,ignorefoot]{geometry}
\setlength{\emergencystretch}{3em} % prevent overfull lines
\setcounter{secnumdepth}{-\maxdimen} % remove section numbering
% Make \paragraph and \subparagraph free-standing
\makeatletter
\ifx\paragraph\undefined\else
  \let\oldparagraph\paragraph
  \renewcommand{\paragraph}{
    \@ifstar
      \xxxParagraphStar
      \xxxParagraphNoStar
  }
  \newcommand{\xxxParagraphStar}[1]{\oldparagraph*{#1}\mbox{}}
  \newcommand{\xxxParagraphNoStar}[1]{\oldparagraph{#1}\mbox{}}
\fi
\ifx\subparagraph\undefined\else
  \let\oldsubparagraph\subparagraph
  \renewcommand{\subparagraph}{
    \@ifstar
      \xxxSubParagraphStar
      \xxxSubParagraphNoStar
  }
  \newcommand{\xxxSubParagraphStar}[1]{\oldsubparagraph*{#1}\mbox{}}
  \newcommand{\xxxSubParagraphNoStar}[1]{\oldsubparagraph{#1}\mbox{}}
\fi
\makeatother

\usepackage{color}
\usepackage{fancyvrb}
\newcommand{\VerbBar}{|}
\newcommand{\VERB}{\Verb[commandchars=\\\{\}]}
\DefineVerbatimEnvironment{Highlighting}{Verbatim}{commandchars=\\\{\}}
% Add ',fontsize=\small' for more characters per line
\usepackage{framed}
\definecolor{shadecolor}{RGB}{241,243,245}
\newenvironment{Shaded}{\begin{snugshade}}{\end{snugshade}}
\newcommand{\AlertTok}[1]{\textcolor[rgb]{0.68,0.00,0.00}{#1}}
\newcommand{\AnnotationTok}[1]{\textcolor[rgb]{0.37,0.37,0.37}{#1}}
\newcommand{\AttributeTok}[1]{\textcolor[rgb]{0.40,0.45,0.13}{#1}}
\newcommand{\BaseNTok}[1]{\textcolor[rgb]{0.68,0.00,0.00}{#1}}
\newcommand{\BuiltInTok}[1]{\textcolor[rgb]{0.00,0.23,0.31}{#1}}
\newcommand{\CharTok}[1]{\textcolor[rgb]{0.13,0.47,0.30}{#1}}
\newcommand{\CommentTok}[1]{\textcolor[rgb]{0.37,0.37,0.37}{#1}}
\newcommand{\CommentVarTok}[1]{\textcolor[rgb]{0.37,0.37,0.37}{\textit{#1}}}
\newcommand{\ConstantTok}[1]{\textcolor[rgb]{0.56,0.35,0.01}{#1}}
\newcommand{\ControlFlowTok}[1]{\textcolor[rgb]{0.00,0.23,0.31}{\textbf{#1}}}
\newcommand{\DataTypeTok}[1]{\textcolor[rgb]{0.68,0.00,0.00}{#1}}
\newcommand{\DecValTok}[1]{\textcolor[rgb]{0.68,0.00,0.00}{#1}}
\newcommand{\DocumentationTok}[1]{\textcolor[rgb]{0.37,0.37,0.37}{\textit{#1}}}
\newcommand{\ErrorTok}[1]{\textcolor[rgb]{0.68,0.00,0.00}{#1}}
\newcommand{\ExtensionTok}[1]{\textcolor[rgb]{0.00,0.23,0.31}{#1}}
\newcommand{\FloatTok}[1]{\textcolor[rgb]{0.68,0.00,0.00}{#1}}
\newcommand{\FunctionTok}[1]{\textcolor[rgb]{0.28,0.35,0.67}{#1}}
\newcommand{\ImportTok}[1]{\textcolor[rgb]{0.00,0.46,0.62}{#1}}
\newcommand{\InformationTok}[1]{\textcolor[rgb]{0.37,0.37,0.37}{#1}}
\newcommand{\KeywordTok}[1]{\textcolor[rgb]{0.00,0.23,0.31}{\textbf{#1}}}
\newcommand{\NormalTok}[1]{\textcolor[rgb]{0.00,0.23,0.31}{#1}}
\newcommand{\OperatorTok}[1]{\textcolor[rgb]{0.37,0.37,0.37}{#1}}
\newcommand{\OtherTok}[1]{\textcolor[rgb]{0.00,0.23,0.31}{#1}}
\newcommand{\PreprocessorTok}[1]{\textcolor[rgb]{0.68,0.00,0.00}{#1}}
\newcommand{\RegionMarkerTok}[1]{\textcolor[rgb]{0.00,0.23,0.31}{#1}}
\newcommand{\SpecialCharTok}[1]{\textcolor[rgb]{0.37,0.37,0.37}{#1}}
\newcommand{\SpecialStringTok}[1]{\textcolor[rgb]{0.13,0.47,0.30}{#1}}
\newcommand{\StringTok}[1]{\textcolor[rgb]{0.13,0.47,0.30}{#1}}
\newcommand{\VariableTok}[1]{\textcolor[rgb]{0.07,0.07,0.07}{#1}}
\newcommand{\VerbatimStringTok}[1]{\textcolor[rgb]{0.13,0.47,0.30}{#1}}
\newcommand{\WarningTok}[1]{\textcolor[rgb]{0.37,0.37,0.37}{\textit{#1}}}

\providecommand{\tightlist}{%
  \setlength{\itemsep}{0pt}\setlength{\parskip}{0pt}}\usepackage{longtable,booktabs,array}
\usepackage{calc} % for calculating minipage widths
% Correct order of tables after \paragraph or \subparagraph
\usepackage{etoolbox}
\makeatletter
\patchcmd\longtable{\par}{\if@noskipsec\mbox{}\fi\par}{}{}
\makeatother
% Allow footnotes in longtable head/foot
\IfFileExists{footnotehyper.sty}{\usepackage{footnotehyper}}{\usepackage{footnote}}
\makesavenoteenv{longtable}
\usepackage{graphicx}
\makeatletter
\def\maxwidth{\ifdim\Gin@nat@width>\linewidth\linewidth\else\Gin@nat@width\fi}
\def\maxheight{\ifdim\Gin@nat@height>\textheight\textheight\else\Gin@nat@height\fi}
\makeatother
% Scale images if necessary, so that they will not overflow the page
% margins by default, and it is still possible to overwrite the defaults
% using explicit options in \includegraphics[width, height, ...]{}
\setkeys{Gin}{width=\maxwidth,height=\maxheight,keepaspectratio}
% Set default figure placement to htbp
\makeatletter
\def\fps@figure{htbp}
\makeatother
% definitions for citeproc citations
\NewDocumentCommand\citeproctext{}{}
\NewDocumentCommand\citeproc{mm}{%
  \begingroup\def\citeproctext{#2}\cite{#1}\endgroup}
\makeatletter
 % allow citations to break across lines
 \let\@cite@ofmt\@firstofone
 % avoid brackets around text for \cite:
 \def\@biblabel#1{}
 \def\@cite#1#2{{#1\if@tempswa , #2\fi}}
\makeatother
\newlength{\cslhangindent}
\setlength{\cslhangindent}{1.5em}
\newlength{\csllabelwidth}
\setlength{\csllabelwidth}{3em}
\newenvironment{CSLReferences}[2] % #1 hanging-indent, #2 entry-spacing
 {\begin{list}{}{%
  \setlength{\itemindent}{0pt}
  \setlength{\leftmargin}{0pt}
  \setlength{\parsep}{0pt}
  % turn on hanging indent if param 1 is 1
  \ifodd #1
   \setlength{\leftmargin}{\cslhangindent}
   \setlength{\itemindent}{-1\cslhangindent}
  \fi
  % set entry spacing
  \setlength{\itemsep}{#2\baselineskip}}}
 {\end{list}}
\usepackage{calc}
\newcommand{\CSLBlock}[1]{\hfill\break\parbox[t]{\linewidth}{\strut\ignorespaces#1\strut}}
\newcommand{\CSLLeftMargin}[1]{\parbox[t]{\csllabelwidth}{\strut#1\strut}}
\newcommand{\CSLRightInline}[1]{\parbox[t]{\linewidth - \csllabelwidth}{\strut#1\strut}}
\newcommand{\CSLIndent}[1]{\hspace{\cslhangindent}#1}

\input{/Users/joseph/GIT/latex/latex-for-quarto.tex}
\makeatletter
\@ifpackageloaded{caption}{}{\usepackage{caption}}
\AtBeginDocument{%
\ifdefined\contentsname
  \renewcommand*\contentsname{Table of contents}
\else
  \newcommand\contentsname{Table of contents}
\fi
\ifdefined\listfigurename
  \renewcommand*\listfigurename{List of Figures}
\else
  \newcommand\listfigurename{List of Figures}
\fi
\ifdefined\listtablename
  \renewcommand*\listtablename{List of Tables}
\else
  \newcommand\listtablename{List of Tables}
\fi
\ifdefined\figurename
  \renewcommand*\figurename{Figure}
\else
  \newcommand\figurename{Figure}
\fi
\ifdefined\tablename
  \renewcommand*\tablename{Table}
\else
  \newcommand\tablename{Table}
\fi
}
\@ifpackageloaded{float}{}{\usepackage{float}}
\floatstyle{ruled}
\@ifundefined{c@chapter}{\newfloat{codelisting}{h}{lop}}{\newfloat{codelisting}{h}{lop}[chapter]}
\floatname{codelisting}{Listing}
\newcommand*\listoflistings{\listof{codelisting}{List of Listings}}
\makeatother
\makeatletter
\makeatother
\makeatletter
\@ifpackageloaded{caption}{}{\usepackage{caption}}
\@ifpackageloaded{subcaption}{}{\usepackage{subcaption}}
\makeatother
\ifLuaTeX
  \usepackage{selnolig}  % disable illegal ligatures
\fi
\usepackage{bookmark}

\IfFileExists{xurl.sty}{\usepackage{xurl}}{} % add URL line breaks if available
\urlstyle{same} % disable monospaced font for URLs
\hypersetup{
  pdftitle={Supplementary files for ``Methods in Causal Inference Part 2: Interaction, Mediation, and Time-Varying Treatments''},
  pdfauthor={Joseph A. Bulbulia},
  colorlinks=true,
  linkcolor={blue},
  filecolor={Maroon},
  citecolor={Blue},
  urlcolor={Blue},
  pdfcreator={LaTeX via pandoc}}

\title{Supplementary files for ``Methods in Causal Inference Part 2:
Interaction, Mediation, and Time-Varying Treatments''}

\usepackage{academicons}
\usepackage{xcolor}

  \author{Joseph A. Bulbulia}
            \affil{%
             \small{     Victoria University of Wellington, New Zealand
          ORCID \textcolor[HTML]{A6CE39}{\aiOrcid} ~0000-0002-5861-2056 }
              }
      


\date{2024-06-20}
\begin{document}
\maketitle

\renewcommand*\contentsname{Table of contents}
{
\hypersetup{linkcolor=}
\setcounter{tocdepth}{2}
\tableofcontents
}
\listoftables
\newpage{}

\subsection{S1. Glossary}\label{id-app-a}

\begin{table}

\caption{\label{tbl-experiments}Glossary}

\centering{

\glossaryTerms

}

\end{table}%

\newpage{}

\subsection{S2. Generalisability and Transportability}\label{id-app-b}

Generalisability: When a study sample is drawn randomly from the target
population, we may generalise from the sample to the target population
as follows.

Suppose we sample randomly from the target population, where:

\begin{itemize}
\tightlist
\item
  \(n_S\) denotes the size of the study's analytic sample \(S\).
\item
  \(N_T\) denotes the total size of the target population \(T\).
\item
  \(\widehat{ATE}_{n_S}\) denotes the estimated average treatment effect
  in the analytic sample \(S\).
\item
  \(ATE_{T}\) denotes the true average treatment effect in the target
  population \(T\).
\item
  \(\epsilon\) denotes an arbitrarily small positive value.
\end{itemize}

Assuming the rest of the causal inference workflow goes to plan
(randomisation succeeds, there is no measurement error, no model
misspecification, etc.), as the random sample size \(n_S\) increases,
the estimated treatment effect in the analytic sample \(S\) converges in
probability to the true treatment effect in the target population \(T\):

\[
\lim_{n_S \to N_T} P(|\widehat{ATE}_{n_S} - ATE_{T}| < \epsilon) = 1
\]

for any small positive value of \(\epsilon\).

Transportability: When the analytic sample is not drawn from the target
population, we cannot directly generalise the findings. However, we can
transport the estimated causal effect from the source population to the
target population under certain assumptions. This involves adjusting for
differences in the distributions of effect modifiers between the two
populations. The closer the source population is to the target
population, the more plausible the transportability assumptions are, and
the less we need to rely on complex adjustment methods. Suppose we have
an analytic sample \(n_S\) drawn from a source population \(S\), and we
want to estimate the average treatment effect in a target population
\(T\). Define:

\(\widehat{ATE}_{S}\) as the estimated average treatment effect in the
analytic sample drawn from the source population \(S\).
\(\widehat{ATE}_{T}\) as the estimated average treatment effect in the
target population \(T\). \(f(n_S, R)\) as the mapping function that
adjusts the estimated effect in the analytic sample using a set of
measured covariates \(R\), allowing for valid projection from the source
population to the target population.

The transportability assumption is that there exists a function \(f\)
such that: \[ \widehat{ATE}_{T} = f(n_S, R) \]

Finding a suitable function \(f\) is the central challenge in adjusting
for sampling bias and achieving transportability
(\citeproc{ref-bareinboim2013general}{Bareinboim \& Pearl, 2013};
\citeproc{ref-dahabreh2019generalizing}{Dahabreh et al., 2019};
\citeproc{ref-deffner2022}{Deffner et al., 2022};
\citeproc{ref-westreich2017transportability}{Westreich et al., 2017}).

\newpage{}

\subsection{S3. A Mathematical Explanation for the Difference in
Marginal Effects between Censored and Uncensored
Populations}\label{id-app-c}

This appendix provides an explanation for why marginal effects may
differ between the censored and uncensored sample population in the
absence of unmeasured confounding.

\paragraph{Definitions:}\label{definitions}

\begin{itemize}
\tightlist
\item
  \textbf{\(A\)}: Exposure variable, where \(a\) represents the
  reference level and \(a^*\) represents the comparison level
\item
  \textbf{\(Y\)}: Outcome variable
\item
  \textbf{\(F\)}: Effect modifier
\item
  \textbf{\(C\)}: Indicator for the uncensored population (\(C = 0\)) or
  the censored population (\(C = 1\))
\end{itemize}

\paragraph{Average Treatment Effects:}\label{average-treatment-effects}

The average treatment effects for the uncensored and censored
populations are defined as:

\[ \Delta_{\text{uncensored}} = \mathbb{E}[Y(a^*) - Y(a) \mid C = 0] \]

\[ \Delta_{\text{censored}} = \mathbb{E}[Y(a^*) - Y(a) \mid C = 1] \]

\paragraph{Potential Outcomes:}\label{potential-outcomes}

By causal consistency, potential outcomes can be expressed in terms of
observed outcomes:

\[ \Delta_{\text{uncensored}} = \mathbb{E}[Y \mid A=a^*, C=0] - \mathbb{E}[Y \mid A=a, C=0] \]

\[ \Delta_{\text{censored}} = \mathbb{E}[Y \mid A=a^*, C=1] - \mathbb{E}[Y \mid A=a, C=1] \]

\paragraph{Law of Total Probability:}\label{law-of-total-probability}

Applying the Law of Total Probability, we can weight the average
treatment effects by the conditional probability of the effect modifier
\(F\):

\[ \Delta_{\text{uncensored}} = \sum_{f} \left\{\mathbb{E}[Y \mid A=a^*, F=f, C=0] - \mathbb{E}[Y \mid A=a, F=f, C=0]\right\} \times \Pr(F=f \mid C=0) \]

\[ 
\Delta_{\text{censored}} = \sum_{f} \left\{\mathbb{E}[Y \mid A=a^*, F=f, C=1] - \mathbb{E}[Y \mid A=a, F=f, C=1]\right\} \times \Pr(F=f \mid C=1) 
\]

\paragraph{Assumption of Informative
Censoring:}\label{assumption-of-informative-censoring}

We assume that the distribution of the effect modifier \(F\) differs
between the censored and uncensored populations:

\[ \Pr(F=f \mid C=0) \neq \Pr(F=f \mid C=1) \]

Under this assumption, the probability weights used to calculate the
marginal effects for the uncensored and censored populations differ.

\paragraph{Effect Estimates for Censored and Uncensored
Populations:}\label{effect-estimates-for-censored-and-uncensored-populations}

Given that \(\Pr(F=f \mid C=0) \neq \Pr(F=f \mid C=1)\), we cannot
guarantee that:

\[ 
\Delta_{\text{uncensored}} = \Delta_{\text{censored}} 
\]

The equality of marginal effects between the two populations will only
hold if there is a universal null effect (i.e., no effect of the
exposure on the outcome for any individual) across all units, by chance,
or under specific conditions discussed by VanderWeele \& Robins
(\citeproc{ref-vanderweele2007}{2007}) and further elucidated by Suzuki
et al. (\citeproc{ref-suzuki2013counterfactual}{2013}). Otherwise:

\[ \Delta_{\text{uncensored}} \ne \Delta_{\text{censored}} \]

Furthermore, VanderWeele (\citeproc{ref-vanderweele2012}{2012}) proved
that if there is effect modification of \(A\) by \(F\), there will be a
difference in at least one scale of causal contrast, such that:

\[ \Delta^{\text{risk ratio}}_{\text{uncensored}} \ne \Delta^{\text{risk ratio}}_{\text{censored}} \]

or

\[ \Delta^{\text{difference}}_{\text{uncensored}} \ne \Delta^{\text{difference}}_{\text{censored}} \]

For comprehensive discussions on sampling and inference, refer to
Dahabreh \& Hernán (\citeproc{ref-dahabreh2019}{2019}) and Dahabreh et
al. (\citeproc{ref-dahabreh2021study}{2021}).

\newpage{}

\subsection{S4. R Simulation to Clarify Why The Distribution of Effect
Modifiers Matters For Estimating Treatment Effects For A Target
Population}\label{id-app-d}

First, we load the \texttt{stdReg} library, which obtains marginal
effect estimates by simulating counterfactuals under different levels of
treatment (\citeproc{ref-sjuxf6lander2016}{Sjölander, 2016}). If a
treatment is continuous, the levels can be specified.

We also load the \texttt{parameters} library, which creates nice tables
(\citeproc{ref-parameters2020}{Lüdecke et al., 2020}).

\begin{Shaded}
\begin{Highlighting}[]
\CommentTok{\#|label: loadlibs}

\CommentTok{\# to obtain marginal effects}
\ControlFlowTok{if}\NormalTok{ (}\SpecialCharTok{!}\FunctionTok{requireNamespace}\NormalTok{(}\StringTok{\textquotesingle{}stdReg\textquotesingle{}}\NormalTok{, }\AttributeTok{quietly =} \ConstantTok{TRUE}\NormalTok{)) }\FunctionTok{install.packages}\NormalTok{(}\StringTok{\textquotesingle{}stdReg\textquotesingle{}}\NormalTok{)}
\FunctionTok{library}\NormalTok{(stdReg)}

\CommentTok{\#  to view data}
\ControlFlowTok{if}\NormalTok{ (}\SpecialCharTok{!}\FunctionTok{requireNamespace}\NormalTok{(}\StringTok{\textquotesingle{}skimr\textquotesingle{}}\NormalTok{, }\AttributeTok{quietly =} \ConstantTok{TRUE}\NormalTok{)) }\FunctionTok{install.packages}\NormalTok{(}\StringTok{\textquotesingle{}skimr\textquotesingle{}}\NormalTok{)}
\FunctionTok{library}\NormalTok{(skimr)}

\CommentTok{\# to create nice tables}
\ControlFlowTok{if}\NormalTok{ (}\SpecialCharTok{!}\FunctionTok{requireNamespace}\NormalTok{(}\StringTok{\textquotesingle{}parameters\textquotesingle{}}\NormalTok{, }\AttributeTok{quietly =} \ConstantTok{TRUE}\NormalTok{)) }\FunctionTok{install.packages}\NormalTok{(}\StringTok{\textquotesingle{}parameters\textquotesingle{}}\NormalTok{)}
\FunctionTok{library}\NormalTok{(parameters)}
\end{Highlighting}
\end{Shaded}

Next, we write a function to simulate data for the sample and target
populations.

We assume the treatment effect is the same in the sample and target
populations, that the coefficient for the effect modifier and the
coefficient for interaction are the same, that there is no unmeasured
confounding throughout the study, and that there is only selective
attrition of one effect modifier such that the baseline population
differs from the analytic sample population at the end of the study.

That is: \textbf{the distribution of effect modifiers is the only
respect in which the sample will differ from the target population.}

This function will generate data under a range of scenarios. Refer to
documentation in the \texttt{margot} package: Bulbulia
(\citeproc{ref-margot2024}{2024})

\begin{Shaded}
\begin{Highlighting}[]
\CommentTok{\# function to generate data for the sample and population, }
\CommentTok{\# Along with precise sample weights for the population, there are differences }
\CommentTok{\# in the distribution of the true effect modifier but no differences in the treatment effect }
\CommentTok{\# or the effect modification. all that differs between the sample and the population is }
\CommentTok{\# the distribution of effect modifiers.}

\CommentTok{\# seed}
\FunctionTok{set.seed}\NormalTok{(}\DecValTok{123}\NormalTok{)}

\CommentTok{\# simulate data {-}{-} you can use different parameters}
\NormalTok{data }\OtherTok{\textless{}{-}}\NormalTok{ margot}\SpecialCharTok{::}\FunctionTok{simulate\_ate\_data\_with\_weights}\NormalTok{(}
  \AttributeTok{n\_sample =} \DecValTok{10000}\NormalTok{,}
  \AttributeTok{n\_population =} \DecValTok{100000}\NormalTok{,}
  \AttributeTok{p\_z\_sample =} \FloatTok{0.1}\NormalTok{,}
  \AttributeTok{p\_z\_population =} \FloatTok{0.5}\NormalTok{,}
  \AttributeTok{beta\_a =} \DecValTok{1}\NormalTok{,}
  \AttributeTok{beta\_z =} \FloatTok{2.5}\NormalTok{,}
  \AttributeTok{noise\_sd =} \FloatTok{0.5}
\NormalTok{)}

\CommentTok{\# inspect}
\CommentTok{\# skimr::skim(data)}
\end{Highlighting}
\end{Shaded}

We have generated both sample and population data.

Next, we verify that the distributions of effect modifiers differ in the
sample and in the target population:

\begin{Shaded}
\begin{Highlighting}[]
\CommentTok{\# obtain the generated data}
\NormalTok{sample\_data }\OtherTok{\textless{}{-}}\NormalTok{ data}\SpecialCharTok{$}\NormalTok{sample\_data}
\NormalTok{population\_data }\OtherTok{\textless{}{-}}\NormalTok{ data}\SpecialCharTok{$}\NormalTok{population\_data}

\CommentTok{\# check imbalance}
\FunctionTok{table}\NormalTok{(sample\_data}\SpecialCharTok{$}\NormalTok{z\_sample) }\CommentTok{\# type 1 is rare}
\end{Highlighting}
\end{Shaded}

\begin{verbatim}

   0    1 
9055  945 
\end{verbatim}

\begin{Shaded}
\begin{Highlighting}[]
\FunctionTok{table}\NormalTok{(population\_data}\SpecialCharTok{$}\NormalTok{z\_population) }\CommentTok{\# type 1 is common}
\end{Highlighting}
\end{Shaded}

\begin{verbatim}

    0     1 
49916 50084 
\end{verbatim}

The sample and population distributions differ.

Next, consider the question: `What are the differences in the
coefficients that we obtain from the study population at the end of the
study, compared with those we would obtain for the target population?'

First, we obtain the regression coefficients for the sample. They are as
follows:

\begin{Shaded}
\begin{Highlighting}[]
\CommentTok{\# model coefficients sample}
\NormalTok{model\_sample  }\OtherTok{\textless{}{-}} \FunctionTok{glm}\NormalTok{(y\_sample }\SpecialCharTok{\textasciitilde{}}\NormalTok{ a\_sample }\SpecialCharTok{*}\NormalTok{ z\_sample, }
  \AttributeTok{data =}\NormalTok{ sample\_data)}

\CommentTok{\# summary}
\NormalTok{parameters}\SpecialCharTok{::}\FunctionTok{model\_parameters}\NormalTok{(model\_sample, }\AttributeTok{ci\_method =} \StringTok{\textquotesingle{}wald\textquotesingle{}}\NormalTok{)}
\end{Highlighting}
\end{Shaded}

\begin{verbatim}
Parameter           | Coefficient |       SE |        95% CI | t(9996) |      p
-------------------------------------------------------------------------------
(Intercept)         |   -6.89e-03 | 7.38e-03 | [-0.02, 0.01] |   -0.93 | 0.350 
a sample            |        1.01 |     0.01 | [ 0.99, 1.03] |   95.84 | < .001
z sample            |        2.47 |     0.02 | [ 2.43, 2.52] |  104.09 | < .001
a sample × z sample |        0.51 |     0.03 | [ 0.44, 0.57] |   14.82 | < .001
\end{verbatim}

Next, we obtain the regression coefficients for the weighted regression
of the sample. Notice that the coefficients are virtually the same:

\begin{Shaded}
\begin{Highlighting}[]
\CommentTok{\# model the sample weighted to the population, again note that these coefficients are similar }
\NormalTok{model\_weighted\_sample }\OtherTok{\textless{}{-}} \FunctionTok{glm}\NormalTok{(y\_sample }\SpecialCharTok{\textasciitilde{}}\NormalTok{ a\_sample }\SpecialCharTok{*}\NormalTok{ z\_sample, }
  \AttributeTok{data =}\NormalTok{ sample\_data, }\AttributeTok{weights =}\NormalTok{ weights)}

\CommentTok{\# summary}
\FunctionTok{summary}\NormalTok{(parameters}\SpecialCharTok{::}\FunctionTok{model\_parameters}\NormalTok{(model\_weighted\_sample, }
  \AttributeTok{ci\_method =} \StringTok{\textquotesingle{}wald\textquotesingle{}}\NormalTok{))}
\end{Highlighting}
\end{Shaded}

\begin{verbatim}
Parameter           | Coefficient |        95% CI |      p
----------------------------------------------------------
(Intercept)         |   -6.89e-03 | [-0.03, 0.01] | 0.480 
a sample            |        1.01 | [ 0.98, 1.04] | < .001
z sample            |        2.47 | [ 2.45, 2.50] | < .001
a sample × z sample |        0.51 | [ 0.47, 0.55] | < .001

Model: y_sample ~ a_sample * z_sample (10000 Observations)
Residual standard deviation: 0.494 (df = 9996)
\end{verbatim}

We might be tempted to infer that weighting wasn't relevant to the
analysis. However, we'll see that such an interpretation would be a
mistake.

Next, we obtain model coefficients for the population. Note again there
is no difference -- only narrower errors owing to the large sample size.

\begin{Shaded}
\begin{Highlighting}[]
\CommentTok{\# model coefficients population {-}{-} note that these coefficients are very similar. }
\NormalTok{model\_population }\OtherTok{\textless{}{-}} \FunctionTok{glm}\NormalTok{(y\_population }\SpecialCharTok{\textasciitilde{}}\NormalTok{ a\_population }\SpecialCharTok{*}\NormalTok{ z\_population, }
  \AttributeTok{data =}\NormalTok{ population\_data)}

\NormalTok{parameters}\SpecialCharTok{::}\FunctionTok{model\_parameters}\NormalTok{(model\_population, }\AttributeTok{ci\_method =} \StringTok{\textquotesingle{}wald\textquotesingle{}}\NormalTok{)}
\end{Highlighting}
\end{Shaded}

\begin{verbatim}
Parameter                   | Coefficient |       SE |        95% CI | t(99996) |      p
----------------------------------------------------------------------------------------
(Intercept)                 |    2.49e-03 | 3.18e-03 | [ 0.00, 0.01] |     0.78 | 0.434 
a population                |        1.00 | 4.49e-03 | [ 0.99, 1.01] |   222.35 | < .001
z population                |        2.50 | 4.49e-03 | [ 2.49, 2.51] |   556.80 | < .001
a population × z population |        0.50 | 6.35e-03 | [ 0.49, 0.51] |    78.80 | < .001
\end{verbatim}

Again, there is no difference. That is, we find that all model
coefficients are practically equivalent. The different distribution of
effect modifiers does not result in different coefficient values for the
treatment effect, the effect-modifier `effect,' or the interaction of
the effect modifier and treatment.

Consider why this is the case: in a large sample where the causal
effects are invariant -- as we have simulated them to be -- we will have
good replication in the effect modifiers within the sample, so our
statistical model can recover the \emph{coefficients} for the population
without challenge.

However, in causal inference, we are interested in the marginal effect
of the treatment within a population of interest or within strata of
this population. That is, we seek an estimate for the counterfactual
contrast in which everyone in a pre-specified population or stratum of a
population was subject to one level of treatment compared with a
counterfactual condition in which everyone in a population was subject
to another level of the same treatment.

\textbf{The marginal effect estimates will differ in at least one
measure of effect when the analytic sample population has a different
distribution of effect modifiers compared to the target population.}

To see this, we use the \texttt{stdReg} package to recover marginal
effect estimates, comparing (1) the sample ATE, (2) the true oracle ATE
for the population, and (3) the weighted sample ATE. We will use the
outputs of the same models above. The only difference is that we will
calculate marginal effects from these outputs. We will contrast a
difference from an intervention in which everyone receives treatment = 0
with one in which everyone receives treatment = 1; however, this choice
is arbitrary, and the general lessons apply irrespective of the
estimand.

First, consider this Average Treatment Effect for the analytic
population:

\begin{Shaded}
\begin{Highlighting}[]
\CommentTok{\# What inference do we draw?  }
\CommentTok{\# we cannot say the models are unbiased for the marginal effect estimates. }
\CommentTok{\# regression standardisation }
\FunctionTok{library}\NormalTok{(stdReg) }\CommentTok{\# to obtain marginal effects }

\CommentTok{\# obtain sample ate}
\NormalTok{fit\_std\_sample }\OtherTok{\textless{}{-}}\NormalTok{ stdReg}\SpecialCharTok{::}\FunctionTok{stdGlm}\NormalTok{(model\_sample, }
  \AttributeTok{data =}\NormalTok{ sample\_data, }\AttributeTok{X =} \StringTok{\textquotesingle{}a\_sample\textquotesingle{}}\NormalTok{)}

\CommentTok{\# summary}
\FunctionTok{summary}\NormalTok{(fit\_std\_sample, }\AttributeTok{contrast =} \StringTok{\textquotesingle{}difference\textquotesingle{}}\NormalTok{, }\AttributeTok{reference =} \DecValTok{0}\NormalTok{)}
\end{Highlighting}
\end{Shaded}

\begin{verbatim}

Formula: y_sample ~ a_sample * z_sample
Family: gaussian 
Link function: identity 
Exposure:  a_sample 
Reference level:  a_sample = 0 
Contrast:  difference 

  Estimate Std. Error lower 0.95 upper 0.95
0     0.00     0.0000       0.00       0.00
1     1.06     0.0101       1.04       1.08
\end{verbatim}

The treatment effect is given as a 1.06 unit change in the outcome
across the analytic population, with a confidence interval from 1.04 to
1.08.

Next, we obtain the true (oracle) treatment effect for the target
population under the same intervention:

\begin{Shaded}
\begin{Highlighting}[]
\DocumentationTok{\#\# note the population effect is different}

\CommentTok{\# obtain true ate}
\NormalTok{fit\_std\_population }\OtherTok{\textless{}{-}}\NormalTok{ stdReg}\SpecialCharTok{::}\FunctionTok{stdGlm}\NormalTok{(model\_population, }
  \AttributeTok{data =}\NormalTok{ population\_data, }\AttributeTok{X =} \StringTok{\textquotesingle{}a\_population\textquotesingle{}}\NormalTok{)}

\CommentTok{\# summary}
\FunctionTok{summary}\NormalTok{(fit\_std\_population, }\AttributeTok{contrast =} \StringTok{\textquotesingle{}difference\textquotesingle{}}\NormalTok{, }\AttributeTok{reference =} \DecValTok{0}\NormalTok{)}
\end{Highlighting}
\end{Shaded}

\begin{verbatim}

Formula: y_population ~ a_population * z_population
Family: gaussian 
Link function: identity 
Exposure:  a_population 
Reference level:  a_population = 0 
Contrast:  difference 

  Estimate Std. Error lower 0.95 upper 0.95
0     0.00    0.00000       0.00       0.00
1     1.25    0.00327       1.24       1.26
\end{verbatim}

Note that the true treatment effect is a 1.25-unit change in the
population, with a confidence bound between 1.24 and 1.26. This is well
outside the ATE that we obtain from the analytic population!

Next, consider the ATE in the weighted regression, where the analytic
sample was weighted to the target population's true distribution of
effect modifiers:

\begin{Shaded}
\begin{Highlighting}[]
\DocumentationTok{\#\# next try weights adjusted ate where we correctly assign population weights to the sample}
\NormalTok{fit\_std\_weighted\_sample\_weights }\OtherTok{\textless{}{-}}\NormalTok{ stdReg}\SpecialCharTok{::}\FunctionTok{stdGlm}\NormalTok{(model\_weighted\_sample, }
  \AttributeTok{data =}\NormalTok{ sample\_data, }\AttributeTok{X =} \StringTok{\textquotesingle{}a\_sample\textquotesingle{}}\NormalTok{)}

\CommentTok{\# this gives us the right answer}
\FunctionTok{summary}\NormalTok{(fit\_std\_weighted\_sample\_weights, }\AttributeTok{contrast =} \StringTok{\textquotesingle{}difference\textquotesingle{}}\NormalTok{, }\AttributeTok{reference =} \DecValTok{0}\NormalTok{)}
\end{Highlighting}
\end{Shaded}

\begin{verbatim}

Formula: y_sample ~ a_sample * z_sample
Family: gaussian 
Link function: identity 
Exposure:  a_sample 
Reference level:  a_sample = 0 
Contrast:  difference 

  Estimate Std. Error lower 0.95 upper 0.95
0     0.00     0.0000       0.00       0.00
1     1.25     0.0172       1.22       1.29
\end{verbatim}

We find that we obtain the population-level causal effect estimate with
accurate coverage by weighting the sample to the target population. So
with appropriate weights, our results generalise from the sample to the
target population.

\subsection{Lessons}\label{lessons}

\begin{itemize}
\tightlist
\item
  \textbf{Regression coefficients do not clarify the problem of
  sample/target population mismatch} --- or selection bias as discussed
  in this manuscript.
\item
  \textbf{Investigators should not rely on regression coefficients
  alone} when evaluating the biases that arise from sample attrition.
  This advice applies to both methods that authors use to investigate
  threats of bias. To implement this advice, authors must first take it
  themselves.
\item
  \textbf{Observed data are generally insufficient for assessing
  threats}. Observed data do not clarify structural sources of bias, nor
  do they clarify effect-modification in the full counterfactual data
  condition where all receive the treatment and all do not receive the
  treatment (at the same level).
\item
  \textbf{To properly assess bias, one needs access to the
  counterfactual outcome} --- what would have happened to the missing
  participants had they not been lost to follow-up or had they
  responded? The joint distributions over `full data' are inherently
  unobservable (\citeproc{ref-vanderlaan2011}{Van Der Laan \& Rose,
  2011}).
\item
  \textbf{In simple settings, like the one we just simulated, we can
  address the gap between the sample and target population using methods
  such as modelling the censoring (e.g., censoring weighting).} However,
  we never know what setting we are in or whether it is simple---such
  modelling must be handled carefully. There is a large and growing
  epidemiology literature on this topic (see, for example, Li et al.
  (\citeproc{ref-li2023non}{2023})).
\end{itemize}

\newpage{}

\subsection{S5. Bias Correction as Interventions on
Reporters}\label{id-app-F}

\begin{table}

\caption{\label{tbl-tblswigme}Single World Intervention Graph reveals
strategies for redressing measurement error.}

\centering{

\tblswigme

}

\end{table}%

Single World Intervention Graphs (SWIGs) help us understand why bias
correction works. We can think of bias correction without relying on
mathematically restrictive models by considering reporters of the true
but unobserved states of the world as elements of a causal reality that
we represent in SWIGs.

Table~\ref{tbl-tblswigme}\(\mathcal{G}_{1.1}\) shows how to represent
the true counterfactual outcome as a function
\(Y(\mathbf{h}(E^A, B(\tilde{a})))\). If this function were known, we
could intervene to correct the bias in reporter \(B\) when
\(A = \tilde{a}\) to obtain \(Y(\tilde{a})\). The dotted green arrows
indicate the counterfactual variables whose functional relationship to
the observed values \(B(\tilde{a})\) are relevant for correcting this
bias. Like an optometrist fitting spectacles to correct vision, knowing
how \(B(\tilde{a})\) relates to \(E^A\) would allow us to recover
\(A = \tilde{a}\) from \(B(\tilde{a})\) and thus obtain
\(\mathbb{E}[Y(\tilde{a})]\) from \(\mathbb{E}[Y(B(\tilde{a}))]\).

Similarly, Table~\ref{tbl-tblswigme}\(\mathcal{G}_{1.2}\) shows how to
represent the true counterfactual outcome as a function
\(V(\mathbf{h}(E^Y, V(\tilde{a})))\). If this function were known, we
could intervene to correct the bias of outcome reporter \(V(\tilde{a})\)
when \(A = \tilde{a}\) to recover the true state \(Y(\tilde{a})\) from
its distorted representation in \(V(\tilde{a})\). The dotted green
arrows indicate the counterfactual variables relevant for correcting
this bias. Knowing how \(V(\tilde{a})\) relates to \(E^Y\) would allow
us to recover \(Y(\tilde{a})\) from \(V(\tilde{a})\).

\begin{table}

\caption{\label{tbl-tblswigmex}Single World Intervention Graph reveals
strategies for redressing measurement error when errors are directed or
correlated.}

\centering{

\tblswigmex

}

\end{table}%

Table~\ref{tbl-tblswigmex}\(\mathcal{G}_{1.1-1.2}\) reveals that
obtaining corrections for biased reporters requires additional
information when there is a directed measurement error. In this setting,
bias correction requires knowledge of a function in which the treatment
and unmeasured sources of error interact to distort reported potential
outcomes under treatment. The SWIG shows that directed measurement error
bias can occur if the treatment affects the outcome reporter, even
without a direct effect of the treatment on the error terms of the
outcome reporter.

Table~\ref{tbl-tblswigmex}\(\mathcal{G}_{2.1-2.2}\) clarifies that
correlated biases in the errors of the treatment and outcome reporters
create additional demands for measurement error correction. The
behaviour of the correlated error must be evaluated for both
\(B(\tilde{a})\) and \(V(\tilde{a})\). To obtain \(V(\tilde{a})\), we
must first obtain \(\tilde{a}\) from a function
\(f_{B}(B(\tilde{a}), E^{AY})\), which cannot be derived from the data
because \(E^{AY}\) is unobserved. Similarly, a function that recovers
\(Y(\tilde{a})\) from \(V(\tilde{a})\) cannot be obtained from the data
because of the unobserved \(E^{AY}\). Further complications arise when
considering bias in settings with both directed and correlated
measurement errors.

Recall from the main article \textbf{Part 3} that we considered how the
distribution of effect modifiers across populations complicates
inference. These problems are compounded when we include treatment and
outcome reporters in our SWIGs. Even if treatment effects were constant
across populations, there might be effect modification in the
mismeasurement of treatments across populations. Statistical tests alone
cannot distinguish between effect modification from treatment effect
heterogeneity and effect modification from heterogeneous reporting of
treatments or outcomes.

\subsubsection{Summary}\label{summary}

Our interest in SWIGs has been to understand the causal underpinnings of
certain population restriction biases and measurement error biases that
arise absent confounding biases. Even assuming strong sequential
exchangeability, we can use SWIGs to clarify the mechanisms by which
non-confounding biases operate, methods for correcting such biases, and
the challenges of comparative research where the distribution of effect
modifiers of bias in reporters must be considered to obtain valid causal
contrasts for potential outcomes under treatment.

Considerations when using Single World Intervention Graphs for
clarifying structural sources of measurement error bias (and other
biases):

\begin{enumerate}
\def\labelenumi{\arabic{enumi}.}
\tightlist
\item
  There must be a directed edge from a latent variable to its reporter.
\item
  If the reporter of the treatment has an arrow entering it from another
  variable, and causal contrasts are obtained from outcomes
  under-reported treatments, there will generally be measurement error
  bias on at least one causal contrast scale (ignoring accidental
  cancellations of errors), see main article \textbf{Part 4}.
\item
  Likewise, if the reporter of an outcome has an arrow entering it from
  another variable, and causal contrasts are obtained from reported
  outcomes, there will generally be measurement error bias on at least
  one causal contrast scale (ignoring accidental cancellations of
  errors); see the main article, Part 4.
\item
  We cannot often control for measurement error biases by
  \emph{conditioning} on variables in the model because these biases are
  not confounding biases.
\item
  However, if the functions that lead to differences between unobserved
  variables of interest and their reporters are known, investigators can
  correct for such differences by reweighting the data or applying
  direct corrections (\citeproc{ref-carroll2006measurement}{Carroll et
  al., 2006}; \citeproc{ref-lash2009applying}{Lash et al., 2009}).
\item
  Certain population restriction biases can be viewed as varieties of
  measurement error bias, as discussed in the main articles \textbf{Part
  2} and \textbf{Part 3}. SWIGs clarify that certain measurement error
  biases arise from effect modification, where the error term interacts
  with the underlying variable of interest, as discussed in the main
  article \textbf{Part 4}.
\item
  Using SWIGs to approach measurement errors as effect modification is
  useful because errors might not operate at all intervention levels.
  Causal DAGs do not readily allow investigators to appreciate these
  prospects.
\item
  Despite the formal equivalence of certain forms of measurement error
  bias and certain forms of population restriction bias, we may use
  Single World Intervention Graphs to show that both biases may operate
  together and in conjunction with confounding biases. We would add
  effect modifier nodes to the SWIGs in Table~\ref{tbl-tblswigme} and
  Table~\ref{tbl-tblswigmex}.
\item
  Despite the utility of Single World Intervention Graphs (and causal
  DAGs) for clarifying structural features of bias, whether confounding
  or otherwise, investigators should not be distracted from the goal
  when using these tools: to understand whether and how valid causal
  effects may be obtained from observational data for the populations of
  interest. Every inclination to use causal diagrams should be resisted
  if their use complicates this objective.
\end{enumerate}

\subsection{References}\label{references}

\phantomsection\label{refs}
\begin{CSLReferences}{1}{0}
\bibitem[\citeproctext]{ref-bareinboim2013general}
Bareinboim, E., \& Pearl, J. (2013). A general algorithm for deciding
transportability of experimental results. \emph{Journal of Causal
Inference}, \emph{1}(1), 107--134.

\bibitem[\citeproctext]{ref-margot2024}
Bulbulia, J. A. (2024). \emph{Margot: MARGinal observational
treatment-effects}. \url{https://doi.org/10.5281/zenodo.10907724}

\bibitem[\citeproctext]{ref-carroll2006measurement}
Carroll, R. J., Ruppert, D., Stefanski, L. A., \& Crainiceanu, C. M.
(2006). \emph{Measurement error in nonlinear models: A modern
perspective}. Chapman; Hall/CRC.

\bibitem[\citeproctext]{ref-dahabreh2021study}
Dahabreh, I. J., Haneuse, S. J. A., Robins, J. M., Robertson, S. E.,
Buchanan, A. L., Stuart, E. A., \& Hernán, M. A. (2021). Study designs
for extending causal inferences from a randomized trial to a target
population. \emph{American Journal of Epidemiology}, \emph{190}(8),
1632--1642.

\bibitem[\citeproctext]{ref-dahabreh2019}
Dahabreh, I. J., \& Hernán, M. A. (2019). Extending inferences from a
randomized trial to a target population. \emph{European Journal of
Epidemiology}, \emph{34}(8), 719--722.
\url{https://doi.org/10.1007/s10654-019-00533-2}

\bibitem[\citeproctext]{ref-dahabreh2019generalizing}
Dahabreh, I. J., Robins, J. M., Haneuse, S. J., \& Hernán, M. A. (2019).
Generalizing causal inferences from randomized trials: Counterfactual
and graphical identification. \emph{arXiv Preprint arXiv:1906.10792}.

\bibitem[\citeproctext]{ref-deffner2022}
Deffner, D., Rohrer, J. M., \& McElreath, R. (2022). A Causal Framework
for Cross-Cultural Generalizability. \emph{Advances in Methods and
Practices in Psychological Science}, \emph{5}(3), 25152459221106366.
\url{https://doi.org/10.1177/25152459221106366}

\bibitem[\citeproctext]{ref-lash2009applying}
Lash, T. L., Fox, M. P., \& Fink, A. K. (2009). \emph{Applying
quantitative bias analysis to epidemiologic data}. Springer.

\bibitem[\citeproctext]{ref-li2023non}
Li, W., Miao, W., \& Tchetgen Tchetgen, E. (2023). Non-parametric
inference about mean functionals of non-ignorable non-response data
without identifying the joint distribution. \emph{Journal of the Royal
Statistical Society Series B: Statistical Methodology}, \emph{85}(3),
913--935.

\bibitem[\citeproctext]{ref-parameters2020}
Lüdecke, D., Ben-Shachar, M. S., Patil, I., \& Makowski, D. (2020).
Extracting, computing and exploring the parameters of statistical models
using {R}. \emph{Journal of Open Source Software}, \emph{5}(53), 2445.
\url{https://doi.org/10.21105/joss.02445}

\bibitem[\citeproctext]{ref-sjuxf6lander2016}
Sjölander, A. (2016). Regression standardization with the R package
stdReg. \emph{European Journal of Epidemiology}, \emph{31}(6), 563--574.
\url{https://doi.org/10.1007/s10654-016-0157-3}

\bibitem[\citeproctext]{ref-suzuki2013counterfactual}
Suzuki, E., Mitsuhashi, T., Tsuda, T., \& Yamamoto, E. (2013). A
counterfactual approach to bias and effect modification in terms of
response types. \emph{BMC Medical Research Methodology}, \emph{13}(1),
1--17.

\bibitem[\citeproctext]{ref-vanderlaan2011}
Van Der Laan, M. J., \& Rose, S. (2011). \emph{Targeted Learning: Causal
Inference for Observational and Experimental Data}. Springer.
\url{https://link.springer.com/10.1007/978-1-4419-9782-1}

\bibitem[\citeproctext]{ref-vanderweele2012}
VanderWeele, T. J. (2012). Confounding and Effect Modification:
Distribution and Measure. \emph{Epidemiologic Methods}, \emph{1}(1),
55--82. \url{https://doi.org/10.1515/2161-962X.1004}

\bibitem[\citeproctext]{ref-vanderweele2007}
VanderWeele, T. J., \& Robins, J. M. (2007). Four types of effect
modification: a classification based on directed acyclic graphs.
\emph{Epidemiology (Cambridge, Mass.)}, \emph{18}(5), 561--568.
\url{https://doi.org/10.1097/EDE.0b013e318127181b}

\bibitem[\citeproctext]{ref-westreich2017transportability}
Westreich, D., Edwards, J. K., Lesko, C. R., Stuart, E., \& Cole, S. R.
(2017). Transportability of trial results using inverse odds of sampling
weights. \emph{American Journal of Epidemiology}, \emph{186}(8),
1010--1014.

\end{CSLReferences}



\end{document}
