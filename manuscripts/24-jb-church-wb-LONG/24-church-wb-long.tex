% Options for packages loaded elsewhere
\PassOptionsToPackage{unicode}{hyperref}
\PassOptionsToPackage{hyphens}{url}
\PassOptionsToPackage{dvipsnames,svgnames,x11names}{xcolor}
%
\documentclass[
  single column]{article}

\usepackage{amsmath,amssymb}
\usepackage{iftex}
\ifPDFTeX
  \usepackage[T1]{fontenc}
  \usepackage[utf8]{inputenc}
  \usepackage{textcomp} % provide euro and other symbols
\else % if luatex or xetex
  \usepackage{unicode-math}
  \defaultfontfeatures{Scale=MatchLowercase}
  \defaultfontfeatures[\rmfamily]{Ligatures=TeX,Scale=1}
\fi
\usepackage[]{libertinus}
\ifPDFTeX\else  
    % xetex/luatex font selection
\fi
% Use upquote if available, for straight quotes in verbatim environments
\IfFileExists{upquote.sty}{\usepackage{upquote}}{}
\IfFileExists{microtype.sty}{% use microtype if available
  \usepackage[]{microtype}
  \UseMicrotypeSet[protrusion]{basicmath} % disable protrusion for tt fonts
}{}
\makeatletter
\@ifundefined{KOMAClassName}{% if non-KOMA class
  \IfFileExists{parskip.sty}{%
    \usepackage{parskip}
  }{% else
    \setlength{\parindent}{0pt}
    \setlength{\parskip}{6pt plus 2pt minus 1pt}}
}{% if KOMA class
  \KOMAoptions{parskip=half}}
\makeatother
\usepackage{xcolor}
\usepackage[top=30mm,left=25mm,heightrounded,headsep=22pt,headheight=11pt,footskip=33pt,ignorehead,ignorefoot]{geometry}
\setlength{\emergencystretch}{3em} % prevent overfull lines
\setcounter{secnumdepth}{-\maxdimen} % remove section numbering
% Make \paragraph and \subparagraph free-standing
\makeatletter
\ifx\paragraph\undefined\else
  \let\oldparagraph\paragraph
  \renewcommand{\paragraph}{
    \@ifstar
      \xxxParagraphStar
      \xxxParagraphNoStar
  }
  \newcommand{\xxxParagraphStar}[1]{\oldparagraph*{#1}\mbox{}}
  \newcommand{\xxxParagraphNoStar}[1]{\oldparagraph{#1}\mbox{}}
\fi
\ifx\subparagraph\undefined\else
  \let\oldsubparagraph\subparagraph
  \renewcommand{\subparagraph}{
    \@ifstar
      \xxxSubParagraphStar
      \xxxSubParagraphNoStar
  }
  \newcommand{\xxxSubParagraphStar}[1]{\oldsubparagraph*{#1}\mbox{}}
  \newcommand{\xxxSubParagraphNoStar}[1]{\oldsubparagraph{#1}\mbox{}}
\fi
\makeatother


\providecommand{\tightlist}{%
  \setlength{\itemsep}{0pt}\setlength{\parskip}{0pt}}\usepackage{longtable,booktabs,array}
\usepackage{calc} % for calculating minipage widths
% Correct order of tables after \paragraph or \subparagraph
\usepackage{etoolbox}
\makeatletter
\patchcmd\longtable{\par}{\if@noskipsec\mbox{}\fi\par}{}{}
\makeatother
% Allow footnotes in longtable head/foot
\IfFileExists{footnotehyper.sty}{\usepackage{footnotehyper}}{\usepackage{footnote}}
\makesavenoteenv{longtable}
\usepackage{graphicx}
\makeatletter
\newsavebox\pandoc@box
\newcommand*\pandocbounded[1]{% scales image to fit in text height/width
  \sbox\pandoc@box{#1}%
  \Gscale@div\@tempa{\textheight}{\dimexpr\ht\pandoc@box+\dp\pandoc@box\relax}%
  \Gscale@div\@tempb{\linewidth}{\wd\pandoc@box}%
  \ifdim\@tempb\p@<\@tempa\p@\let\@tempa\@tempb\fi% select the smaller of both
  \ifdim\@tempa\p@<\p@\scalebox{\@tempa}{\usebox\pandoc@box}%
  \else\usebox{\pandoc@box}%
  \fi%
}
% Set default figure placement to htbp
\def\fps@figure{htbp}
\makeatother
% definitions for citeproc citations
\NewDocumentCommand\citeproctext{}{}
\NewDocumentCommand\citeproc{mm}{%
  \begingroup\def\citeproctext{#2}\cite{#1}\endgroup}
\makeatletter
 % allow citations to break across lines
 \let\@cite@ofmt\@firstofone
 % avoid brackets around text for \cite:
 \def\@biblabel#1{}
 \def\@cite#1#2{{#1\if@tempswa , #2\fi}}
\makeatother
\newlength{\cslhangindent}
\setlength{\cslhangindent}{1.5em}
\newlength{\csllabelwidth}
\setlength{\csllabelwidth}{3em}
\newenvironment{CSLReferences}[2] % #1 hanging-indent, #2 entry-spacing
 {\begin{list}{}{%
  \setlength{\itemindent}{0pt}
  \setlength{\leftmargin}{0pt}
  \setlength{\parsep}{0pt}
  % turn on hanging indent if param 1 is 1
  \ifodd #1
   \setlength{\leftmargin}{\cslhangindent}
   \setlength{\itemindent}{-1\cslhangindent}
  \fi
  % set entry spacing
  \setlength{\itemsep}{#2\baselineskip}}}
 {\end{list}}
\usepackage{calc}
\newcommand{\CSLBlock}[1]{\hfill\break\parbox[t]{\linewidth}{\strut\ignorespaces#1\strut}}
\newcommand{\CSLLeftMargin}[1]{\parbox[t]{\csllabelwidth}{\strut#1\strut}}
\newcommand{\CSLRightInline}[1]{\parbox[t]{\linewidth - \csllabelwidth}{\strut#1\strut}}
\newcommand{\CSLIndent}[1]{\hspace{\cslhangindent}#1}

\usepackage{booktabs}
\usepackage{longtable}
\usepackage{array}
\usepackage{multirow}
\usepackage{wrapfig}
\usepackage{float}
\usepackage{colortbl}
\usepackage{pdflscape}
\usepackage{tabu}
\usepackage{threeparttable}
\usepackage{threeparttablex}
\usepackage[normalem]{ulem}
\usepackage{makecell}
\usepackage{xcolor}
\input{/Users/joseph/GIT/latex/latex-for-quarto.tex}
\let\oldtabular\tabular
\renewcommand{\tabular}{\small\oldtabular}
\setlength{\tabcolsep}{4pt}  % Adjust this value as needed
\makeatletter
\@ifpackageloaded{caption}{}{\usepackage{caption}}
\AtBeginDocument{%
\ifdefined\contentsname
  \renewcommand*\contentsname{Table of contents}
\else
  \newcommand\contentsname{Table of contents}
\fi
\ifdefined\listfigurename
  \renewcommand*\listfigurename{List of Figures}
\else
  \newcommand\listfigurename{List of Figures}
\fi
\ifdefined\listtablename
  \renewcommand*\listtablename{List of Tables}
\else
  \newcommand\listtablename{List of Tables}
\fi
\ifdefined\figurename
  \renewcommand*\figurename{Figure}
\else
  \newcommand\figurename{Figure}
\fi
\ifdefined\tablename
  \renewcommand*\tablename{Table}
\else
  \newcommand\tablename{Table}
\fi
}
\@ifpackageloaded{float}{}{\usepackage{float}}
\floatstyle{ruled}
\@ifundefined{c@chapter}{\newfloat{codelisting}{h}{lop}}{\newfloat{codelisting}{h}{lop}[chapter]}
\floatname{codelisting}{Listing}
\newcommand*\listoflistings{\listof{codelisting}{List of Listings}}
\makeatother
\makeatletter
\makeatother
\makeatletter
\@ifpackageloaded{caption}{}{\usepackage{caption}}
\@ifpackageloaded{subcaption}{}{\usepackage{subcaption}}
\makeatother

\usepackage{bookmark}

\IfFileExists{xurl.sty}{\usepackage{xurl}}{} % add URL line breaks if available
\urlstyle{same} % disable monospaced font for URLs
\hypersetup{
  pdftitle={Religious Service Attendance Increases Meaning, Gratitude, and Sexual Satisfaction (But Not Much Else) in Secular New Zealand},
  pdfauthor={Joseph A. Bulbulia; Don E Davis; Cyrstal Park; Kenneth G. Rice; Geoffrey Troughton; Daryl R. Van Tongeren; Chris G. Sibley},
  pdfkeywords={Use, use},
  colorlinks=true,
  linkcolor={blue},
  filecolor={Maroon},
  citecolor={Blue},
  urlcolor={Blue},
  pdfcreator={LaTeX via pandoc}}


\title{Religious Service Attendance Increases Meaning, Gratitude, and
Sexual Satisfaction (But Not Much Else) in Secular New Zealand}

\usepackage{academicons}
\usepackage{xcolor}

  \author{Joseph A. Bulbulia}
            \affil{%
             \small{     Victoria University of Wellington, New Zealand
          ORCID \textcolor[HTML]{A6CE39}{\aiOrcid} ~0000-0002-5861-2056 }
              }
      \usepackage{academicons}
\usepackage{xcolor}

  \author{Don E Davis}
            \affil{%
             \small{     Georgia State University, Matheny Center for
the Study of Stress, Trauma, and Resilience
          ORCID \textcolor[HTML]{A6CE39}{\aiOrcid} ~0000-0003-3169-6576 }
              }
      \usepackage{academicons}
\usepackage{xcolor}

  \author{Cyrstal Park}
            \affil{%
             \small{     University of Connecticut, Department of
Psychological Sciences
          ORCID \textcolor[HTML]{A6CE39}{\aiOrcid} ~0000-0001-6572-7321 }
              }
      \usepackage{academicons}
\usepackage{xcolor}

  \author{Kenneth G. Rice}
            \affil{%
             \small{     Georgia State University, Matheny Center for
the Study of Stress, Trauma, and Resilience
          ORCID \textcolor[HTML]{A6CE39}{\aiOrcid} ~0000-0002-0558-2818 }
              }
      \usepackage{academicons}
\usepackage{xcolor}

  \author{Geoffrey Troughton}
            \affil{%
             \small{     School of Social and Cultural Studies, Victoria
University of Wellington
          ORCID \textcolor[HTML]{A6CE39}{\aiOrcid} ~0000-0001-7423-0640 }
              }
      \usepackage{academicons}
\usepackage{xcolor}

  \author{Daryl R. Van Tongeren}
            \affil{%
             \small{     Hope College
          ORCID \textcolor[HTML]{A6CE39}{\aiOrcid} ~0000-0002-1810-9448 }
              }
      \usepackage{academicons}
\usepackage{xcolor}

  \author{Chris G. Sibley}
            \affil{%
             \small{     School of Psychology, University of Auckland
          ORCID \textcolor[HTML]{A6CE39}{\aiOrcid} ~0000-0002-4064-8800 }
              }
      


\date{2024-12-24}
\begin{document}
\maketitle
\begin{abstract}
We use nationally representative longitudinal data from 46,377 New
Zealanders (2018--2023) to estimate the causal effects of religious
service attendance on multiple dimensions of well-being. We compare two
hypothetical four-year scenarios: (a) everyone attends services at least
four times per month, and (b) no one attends. We adjust for baseline
covariates measured before these interventions, including baseline
service attendance and baseline measurements of multi-dimensional
well-being. We also adjust for time-varying counfounders such as
employment, disability, relationships, and parenting. We assess outcomes
in the year following these sequential treatment regimes. Using
cross-validation and doubly robust machine-learning estimators, we find
that regular attendance increases a sense of meaning and purpose,
enhances gratitude, and, somewhat unexpectedly, improves sexual
satisfaction. Effects on mental health are modest, and effects on
biological health and social well-being are negligible or small. In a
mostly secular population, attending religious services modestly boosts
meaning, gratitude, and sexual satisfaction, and more strongly increases
a sense of purpose. Beyond these gains, religious attendance yields few
other measurable benefits to health and well-being.

\textbf{KEYWORDS}: \emph{Causal Inference}; \emph{Church};
\emph{Cross-validation}; \emph{Distress}; \emph{Health};
\emph{Longitudinal}; \emph{Machine Learning}; \emph{Religion};
\emph{Semi-parametric}; \emph{Targeted Learning}.
\end{abstract}


\subsection{Introduction}\label{introduction}

Collective religious activity is a cornerstone of many cultures, and
questions about whether attending religious services affects well-being
are long-standing (\citeproc{ref-koenig2012handbook}{Koenig \emph{et
al.} 2012}). Within religious traditions, the belief that religious
practices benefit adherents permeates theology, moral teaching, and
community life, from Buddhism's emphasis on meditation as a path to
enlightenment and reduced suffering to Christianity's view of prayer and
worship as a means of spiritual growth and divine connection, to
Indigenous spiritual practices that are seen as maintaining harmony
between people, nature, and the sacred. More recently, empirical science
has linked religious service attendance with multiple dimensions of
well-being, including social integration, life satisfaction, mental
health, and meaning-making (\citeproc{ref-chen2020religious}{Chen
\emph{et al.} 2020}; \citeproc{ref-dunbar2021religiosity}{Dunbar 2021};
\citeproc{ref-highland2022national}{Highland \emph{et al.} 2022};
\citeproc{ref-vanderweele2022invited_church}{VanderWeele \emph{et al.}
2022}). These studies have demonstrated that, on average, individuals
who attend services report more favourable social and psychological
outcomes. Is this association causal? For whom?

Although recent studies have applied causal methods
(\citeproc{ref-vanderweelechurchmortality}{Li \emph{et al.} 2016};
\citeproc{ref-pawlikowski2019religious}{Pawlikowski \emph{et al.} 2019};
\citeproc{ref-vanderweele2021effectsReligiousServiceMetanalysis}{VanderWeele
2021}; \citeproc{ref-vanderweele2020vchenrespond}{VanderWeele and Chen
2020}), debates over their implications persist
(\citeproc{ref-highland2019attitudes}{Highland \emph{et al.} 2019}).
Some investigations do not fully adjust for baseline health or baseline
religious service attendance, nor other critical confounders affecting
both religious participation and well-being
(\citeproc{ref-brown2023_church}{Brown \emph{et al.} 2023};
\citeproc{ref-jokela2024_church}{Jokela and Laakasuo 2024}). For
instance, in their analysis of four U.S. cohort studies (N=6,592),
Jokela and Laakasuo found that poorer health trajectories preceded
religious disengagement, suggesting that declining practice might
reflect underlying health problems rather than cause them. Similarly,
Iyer and Rosso's global data (\(N=65,564\)) show that frequent attenders
report about five per cent higher life satisfaction than infrequent
attenders, though unmeasured confounding or reverse causation could
explain this association.

Existing work often relies on non-representative samples or
predominantly religious settings, rarely incorporates time-varying
confounders, and seldom examines populations outside the north atlantic
(\citeproc{ref-price2024science}{Price and Johnson 2024}), or in
societies with relatively low religious engagement. Brown \emph{et al.}
(\citeproc{ref-brown2023_church}{2023}) propose that community-building
itself, rather than specifically religious community-building, may
explain well-being benefits. Supporting evidence from New Zealand shows
that following the 2010--2011 Christchurch earthquakes, people who lost
faith exhibited declining subjective health, whereas those who
maintained faith or lacked faith altogether did not
(\citeproc{ref-sibley2012}{Sibley and Bulbulia 2012}). These findings
hint that broader communal factors -- not necessarily religious practice
-- could drive well-being, especially in secular contexts where
religious service might not confer distinctive advantages. Moreover,
Chen and colleagues observe that well-being encompasses emotional,
physical, social, and psychological domains and that some---such as
meaning, purpose, and social connection---may be more sensitive to
religious engagement than others
(\citeproc{ref-chen2022_church_longitudinal}{Chen \emph{et al.} 2022}).
However, most studies on religious attendance and well-being focus on
just one or a few of these domains.

In sum, there is a need for studies that move beyond cross-sectional
correlations to address causation, address causation robustly, and
systematically evaluate multiple dimensions of well-being, particularly
in secular societies. Without such evidence, policy, clinical, and
theoretical questions about the causal role of religious service
attendance in supporting human flourishing remain unanswered.

Here, we address these gaps by estimating the causal effects of regular
religious service attendance on multidimensional well-being using data
from a large, nationally diverse sample of New Zealanders (baseline
\(N=46,377\)) enrolled in the New Zealand Attitudes and Values Study
(NZAVS). New Zealand is characterised by high religious diversity and
secularisation. Over half the population identifies as having no
religion, while multiple religious traditions -- Christianity, Hinduism,
Islam, Buddhism, and others -- coexist
(\citeproc{ref-shaver2016religion}{Shaver \emph{et al.} 2016}). We draw
on an informationally rich national panel study and recent innovations
in causal inference (\citeproc{ref-bulbulia2023}{Bulbulia 2024b};
\citeproc{ref-hernan2024WHATIF}{Hernan and Robins 2024};
\citeproc{ref-williams2021}{Williams and Díaz 2021}) and systematically
examine whether a settings in which the adult population attended
services regularly differs in its effects across multi-dimensional
well-being from a setting in which no-one attends. We estimate these
effects using a workflow in which non-parametric machine learning
reduces statistical modelling assumptions. We contribute to
understanding by estimating causal effects in a distinctly secular, and
culturally diverse context and obtain generalisable inferences about
whether religious service attendance causally affects a well-being
phenotype across dimensions of biological health, psychological health,
person-focussed well-being, life-focussed well-being and social
well-being.

\subsection{Method}\label{method}

\subsubsection{Sample}\label{sample}

Data were collected as part of the New Zealand Attitudes and Values
Study (NZAVS), an annual longitudinal national probability panel
assessing New Zealand residents' social attitudes, personality,
ideology, and health outcomes. The panel began in 2009 and has since
expanded to include over fifty researchers, with responses from 76,409
participants to date. The study operates independently of political or
corporate funding and is based at a university. It employs prize draws
to incentivise participation. The NZAVS tends to slightly under-sample
males and individuals of Asian descent and to over-sample females and
Māori (the Indigenous people of New Zealand). To enhance the
representativeness of our sample population estimates for the target
population of New Zealand, we apply census-based survey weights that
adjust for age, gender, and ethnicity (New Zealand European, Asian,
Māori, Pacific) (\citeproc{ref-sibley2021}{Sibley 2021}). For more
information about the NZAVS, visit:
\href{https://doi.org/10.17605/OSF.IO/75SNB}{OSF.IO/75SNB}. Refer to
\hyperref[appendix-timeline]{Appendix A} for a histogram of daily
responses for this cohort.

\subsubsection{Target Population}\label{target-population}

The target population for this study comprises the cohort of New Zealand
residents in New Zealand Attitudes and Values Study wave 10 (years
2018-2019) (\citeproc{ref-sibley2021}{Sibley 2021}).

\subsubsection{Treatment Indicator}\label{treatment-indicator}

The New Zealand Attitudes and Values Study assesses religious service
attendance using the following question:

\begin{itemize}
\tightlist
\item
  \emph{Do you identify with a religion and/or spiritual group? If yes,
  how many times did you attend a church or place of worship during the
  last month?}
\end{itemize}

We rounded response to the nearest whole number. Because few
participants reported attending more than eight times, we capped
responses above eight at eight (refer to
\hyperref[appendix-baseline]{Appendix B}, variable
\texttt{religion\_church\_round}).

\subsubsection{Baseline Covariates}\label{baseline-covariates}

We adjusted for a rich set of demographic, personality, and behavioural
indicators measured at the baseline wave, NZAVS time 10 (Wave 2018,
years 2018-2019) (see \hyperref[appendix-baseline]{Appendix B} for full
measures). These variables included age, gender, ethnicity, education
level, personality traits (Agreeableness, Conscientiousness,
Extraversion, Honesty-Humility, Neuroticism, and Openness), household
income, employment status, parenting status, relationship status,
religious belonging, and health-related behaviours (e.g.~smoking,
alcohol use, hours spent exercising). We selected only those outcome
variables measured in the baseline wave and controlled for these
variables. Moreover we controlled for religious service attendance at
baseline (refer to \hyperref[appendix-baseline]{Appendix B} and
\hyperref[appendix-confounding]{Appendix C}) This strategy of
confounding control is powerful because for any confounder to affect
subsequent treatments and the outcome, it would need to do so
independently of the baseline outcome variables, the baseline exposure,
and the rich set of demographic indicators measured at baseline
(\citeproc{ref-vanderweele2020}{VanderWeele \emph{et al.} 2020}).

\subsubsection{Outcomes}\label{outcomes}

We investigate well-being using an VanderWeele and colleague's
``outcome-wide'' approach. We categorised outcomes into five
domains---health, psychological well-being, present-reflective outcomes,
life-reflective outcomes, and social outcomes---based on validated
scales and measures. Outcomes were based on those modelled in an earlier
outcome-wide paper Rosa \emph{et al.}
(\citeproc{ref-pedro_2024effects}{2024}). Table~\ref{tbl-outcomes}
summarises each domain and its associated measures. For instance, health
outcomes included BMI and hours of sleep, whereas psychological
well-being included anxiety and depression. Outcomes were converted to
z-scores (standardised), and the causal effect estimates may be
therefore be interpreted as effect sizes.

\begin{longtable}[]{@{}
  >{\raggedright\arraybackslash}p{(\linewidth - 2\tabcolsep) * \real{0.2787}}
  >{\raggedright\arraybackslash}p{(\linewidth - 2\tabcolsep) * \real{0.7213}}@{}}
\caption{Outcome domains and example dimensions. Data summaries for all
measures used in this study are provided in
\hyperref[appendix-baseline]{Appendix
B}.}\label{tbl-outcomes}\tabularnewline
\toprule\noalign{}
\begin{minipage}[b]{\linewidth}\raggedright
Domain
\end{minipage} & \begin{minipage}[b]{\linewidth}\raggedright
Example Measures
\end{minipage} \\
\midrule\noalign{}
\endfirsthead
\toprule\noalign{}
\begin{minipage}[b]{\linewidth}\raggedright
Domain
\end{minipage} & \begin{minipage}[b]{\linewidth}\raggedright
Example Measures
\end{minipage} \\
\midrule\noalign{}
\endhead
\bottomrule\noalign{}
\endlastfoot
Health & BMI, Hours of Sleep, Hours of Exercise, SF-Health \\
Psychological Well-Being & Anxiety, Depression, Fatigue, Rumination \\
Present-Reflective & Body Satisfaction, Forgiveness, Self-Esteem, Sexual
Satisfaction \\
Life-Reflective & Gratitude, Life Satisfaction, Meaning (Sense \&
Purpose) \\
Social & Belonging, Support, Community \\
\end{longtable}

\subsubsection{Causal Interventions}\label{causal-interventions}

When psychologists analyse time-series data, they often use growth
models to describe how variables evolve over time. However, many
questions are causal: we want to know what would happen if we could
intervene on certain variables (such as religious service attendance).
To investigate these questions with observational time-series data, we
must clearly define our causal question and design our analysis to
emulate a hypothetical randomised controlled trial---often called a
\textbf{target trial} (\citeproc{ref-hernan2016}{Hernán \emph{et al.}
2016}). A target trial asks, setting aside practicalities and ethics,
\emph{what experiment are we attempting to emulate with our data?}
Without explicitly stating this hypothetical experiment, it can be
unclear which causal effect we are actually estimating.

Here, we ask:

\begin{quote}
``What if, at each wave, we intervened to set religious service
attendance to a certain level, and then measured everyone's well-being
at the final wave?''
\end{quote}

To answer this, we compare two hypothetical interventions. Each
intervention shifts religious service attendance across four exposure
waves, with outcomes measured after the year following the fourth
exposure wave. Again, we control for a rich set of baseline covariates
in the wave before the first exposure wave as well as for time-varying
confounders at each exposure wave (refer to @Table~\ref{tbl-plan}). We
using the six most recent NZAVS waves (Times 10--15) because Wave 10
containes the largest cohort, enabling us to maximise power.

Following the modified treatment policies approach, we define
\textbf{shift functions} describing each intervention:

\begin{enumerate}
\def\labelenumi{\arabic{enumi}.}
\item
  \textbf{Regular Religious Service Attendance}\\
  At each wave, if attendance is below four times per month, we shift it
  to four; otherwise, we leave it unchanged.
\item
  \textbf{No Religious Service Attendance}\\
  At each wave, if attendance is above zero, we shift it to zero;
  otherwise, we leave it unchanged.
\end{enumerate}

\begin{table}

\caption{\label{tbl-plan}We contrast well-being outcomes from two
treatment regimes: (1) regular attendance for four years and (2) no
service attendance for four years. \(a^{+}\) denotes regular religious
service attendance; \(a^{-}\) denotes no attendance. Our statistical
models control for baseline-wave confounders, and subsequent
time-varying confounders for all exposure waves. We include baseline
measurments of religious service attendance and baseline measurements of
all outcomes as confounders. We assume that conditional on these
confounders, treatment assignment is ``as good as random.'' Outcomes,
here denoted \(Y_\tau\), are measured in the wave following the final
treatment.}

\centering{

\vizfive

}

\end{table}%

We then organise our data to resemble a randomised sequential experiment
that assigns each person to one of two longitudinal treatment strategies
``regular religious service for four waves'' and ``no religious service
for four waves.'' We define a ``confounder'' is a variable that, once
included in the model, along with other included variables, removes any
non-causal association between the treatment and outcome. Here, as
mentioned, we adjust for a rich set of demographic and personality
variables, as well as baseline religious service attendance and baseline
measures of all outcomes. We also adjust for time-varying confounders at
each wave (physical disability, employment status, partner status, and
parenting status). We assume these time-varying confounders can
influence religious service attendance and later well-being, potentially
biasing our estimates. We ensure there is no reverse causation by
measuring the outcomes at the end of the study, one year after the final
treatment wave.

\subsubsection{Causal Contrasts}\label{causal-contrasts}

We compared the average expected outcome under the ``regular
attendance'' and ``no attendance'' regime, estimating contrasts on the
difference scale (i.e., subtracting the expected outcome under ``no
attendance'' from that under ``regular attendance''). We obtained
confidence intervals using the cross-fitted influence-function approach
in the \texttt{lmtp} package (\citeproc{ref-williams2021}{Williams and
Díaz 2021}). This approach employs sequentially doubly robust (SDR)
estimator, as developed by Díaz \emph{et al.}
(\citeproc{ref-diaz2021_non_parametric_lmtp}{2021a}), which remains
valid if either the outcome model or the propensity model is correctly
specified, thereby requiring weaker assumptions than standard
approaches. By setting up our data as if it came from a hypothetical
experiment, we gain clarity about which causal effects we are estimating
and as well as confidence about our causal effect estimates (refer to
Hernan and Robins (\citeproc{ref-hernan2024WHATIF}{2024}), Bulbulia
(\citeproc{ref-bulbulia2022}{2023a}); Bulbulia
(\citeproc{ref-bulbulia2023}{2024b})).

We estimate this average treatment effect on the difference scale for
each outcome across five well-being domains under the assumptions of
no-unmeasured confounding, causal consistency, and positivity (refer to
\hyperref[appendix-assumptions]{Appendix D} and
\hyperref[appendix-transition]{Appendix E}). Again, the target
population is all adults in New Zealand from the years 2018-2024.

\subsubsection{Comment on Methods for Causal Inference Compared with
Growth Curve
Analysis}\label{comment-on-methods-for-causal-inference-compared-with-growth-curve-analysis}

Although both growth models and causal inference methods pertain to
questions about changes over time, each method focusses on different
questions. Growth models quantify how individuals vary in their natural
trajectories. In contrast, causal models explicitly ask: ``What happens
if we intervene?'' -- which of course requires that we define the
interventions we seek to compare. By clearly defining the series of
interventions to be compared, adjusting for confounders, and ensuring
that outcomes are measured after the interventions, investigators can
rigorously address ``what if?'' questions, bridging the gap between
observational data and actionable insights. Causal inference opens new
avenues for psychological scientists to design and evaluate
interventions that might guide policy and practice with evidence. Such
advice cannot be obtained from statistical models of change over time,
no matter how sophisticated (\citeproc{ref-hernan2024WHATIF}{Hernan and
Robins 2024}).

\subsubsection{Statistical Estimator}\label{statistical-estimator}

We estimate causal effects of time-varying treatment policies using a
Sequential Doubly-Robust (SDR) estimator in the \texttt{lmtp} package
(\citeproc{ref-williams2021}{Williams and Díaz 2021}). SDR proceeds in
two steps. First, we use machine learning to flexibly model
relationships among treatments, covariates, and outcomes. This approach
captures complex, high-dimensional structures without strict assumptions
(\citeproc{ref-duxedaz2021}{Díaz \emph{et al.} 2021b}). Second, SDR
``targets'' these initial estimates by incorporating information from
the observed data distribution. This step iteratively refines the
accuracy of our causal estimates.

The SDR estimator is multiply robust when treatments repeat over
multiple waves (\citeproc{ref-diaz2023lmtp}{Díaz \emph{et al.} 2023};
\citeproc{ref-hoffman2023}{Hoffman \emph{et al.} 2023}). This design
maintains consistency if either the outcome model or treatment model is
correctly specified. The \texttt{lmtp} package relies on the
\texttt{SuperLearner} library in R
(\citeproc{ref-SuperLearner2023}{Polley \emph{et al.} 2023b}). We used
\texttt{SL.ranger}, \texttt{SL.glmnet}, and \texttt{SL.xgboost}
(\citeproc{ref-xgboost2023}{Chen \emph{et al.} 2023};
\citeproc{ref-polley2023}{Polley \emph{et al.} 2023a};
\citeproc{ref-Ranger2017}{Wright and Ziegler 2017}) as our base
learners. \textbf{\texttt{SL.ranger}}: implements a random forest
algorithm, capturing non-linear relationships and complex interactions.
\textbf{\texttt{SL.glmnet}}: provides regularised linear models for
high-dimensional data. \textbf{\texttt{SL.xgboost}}: uses gradient
boosting to capture intricate patterns without over-fitting.

\texttt{SuperLearner} combines these learners adaptively to optimise
predictive performance. We created graphs, tables, and output reports
with the \texttt{margot} package (\citeproc{ref-margot2024}{Bulbulia
2024a}). For more details on targeted learning with \texttt{lmtp}, see
(\citeproc{ref-duxedaz2021}{Díaz \emph{et al.} 2021b};
\citeproc{ref-hoffman2022}{Hoffman \emph{et al.} 2022},
\citeproc{ref-hoffman2023}{2023}).

\subsubsection{Handling of Missing Data}\label{handling-of-missing-data}

\paragraph{Baseline Missingness}\label{baseline-missingness}

We used predictive mean matching from the \texttt{mice} package
(\citeproc{ref-vanbuuren2018}{Van Buuren 2018}) to impute missing
baseline values (comprising 1.2942581 of the baseline data). Following
(\citeproc{ref-zhang2023shouldMultipleImputation}{Zhang \emph{et al.}
2023}), we performed single imputation using only baseline data. For
each column with missing values, we created a binary indicator of
missingness so that the machine learning algorithms we employed could
condition on missingness information during estimation (see
\texttt{lmtp} documentation (\citeproc{ref-williams2021}{Williams and
Díaz 2021})).

\paragraph{Missingness in Time-Varying
Variables}\label{missingness-in-time-varying-variables}

When a time-varying value was missing in any wave but a future value was
observed, we carried forward the previous response and included a
missingness indicator. Again, this approach let the patterns of
missingness inform nonparametric machine learnering. If no future value
was observed, we considered the participant censored and used inverse
probability of treatment weights to address attrition.

\paragraph{Outcome Missingness}\label{outcome-missingness}

Finally, to handle confounding and selection bias arising from missing
outcomes and panel attrition, we applied inverse probability of
censoring weights, estimated via nonparametric machine learning
ensembles in the \texttt{lmtp} package
(\citeproc{ref-williams2021}{Williams and Díaz 2021}).

\subsubsection{Sensitivity Analysis}\label{sensitivity-analysis}

We perform sensitivity analyses using the E-value metric
(\citeproc{ref-linden2020EVALUE}{Linden \emph{et al.} 2020};
\citeproc{ref-vanderweele2017}{VanderWeele and Ding 2017}). The E-value
represents the minimum association strength (on the risk ratio scale)
that an unmeasured confounder would need to have with both the exposure
and outcome---after adjusting for measured covariates---to explain away
the observed exposure-outcome association
(\citeproc{ref-linden2020EVALUE}{Linden \emph{et al.} 2020};
\citeproc{ref-vanderweele2020}{VanderWeele \emph{et al.} 2020}).

\subsection{Results}\label{results}

\subsubsection{Health}\label{health}

\begin{figure}

\centering{

\pandocbounded{\includegraphics[keepaspectratio]{24-church-wb-long_files/figure-pdf/fig-health-1.pdf}}

}

\caption{\label{fig-health}Health effects}

\end{figure}%

\begin{longtable}[]{@{}
  >{\raggedright\arraybackslash}p{(\linewidth - 10\tabcolsep) * \real{0.3288}}
  >{\raggedleft\arraybackslash}p{(\linewidth - 10\tabcolsep) * \real{0.2192}}
  >{\raggedleft\arraybackslash}p{(\linewidth - 10\tabcolsep) * \real{0.0822}}
  >{\raggedleft\arraybackslash}p{(\linewidth - 10\tabcolsep) * \real{0.0959}}
  >{\raggedleft\arraybackslash}p{(\linewidth - 10\tabcolsep) * \real{0.1096}}
  >{\raggedleft\arraybackslash}p{(\linewidth - 10\tabcolsep) * \real{0.1644}}@{}}

\caption{\label{tbl-health}Health effects}

\tabularnewline

\toprule\noalign{}
\begin{minipage}[b]{\linewidth}\raggedright
\end{minipage} & \begin{minipage}[b]{\linewidth}\raggedleft
E{[}Y(1){]}-E{[}Y(0){]}
\end{minipage} & \begin{minipage}[b]{\linewidth}\raggedleft
2.5 \%
\end{minipage} & \begin{minipage}[b]{\linewidth}\raggedleft
97.5 \%
\end{minipage} & \begin{minipage}[b]{\linewidth}\raggedleft
E\_Value
\end{minipage} & \begin{minipage}[b]{\linewidth}\raggedleft
E\_Val\_bound
\end{minipage} \\
\midrule\noalign{}
\endhead
\bottomrule\noalign{}
\endlastfoot
BMI & -0.03 & -0.05 & -0.02 & 1.21 & 1.16 \\
Sleep & 0.01 & -0.01 & 0.04 & 1.12 & 1.00 \\
Hours of Exercise (log) & -0.06 & -0.09 & -0.04 & 1.31 & 1.22 \\
Short Form Health & -0.02 & -0.04 & 0.00 & 1.15 & 1.00 \\

\end{longtable}

\paragraph{Bmi}\label{bmi}

The effect estimate (rd) is -0.034 (-0.049, -0.019). On the original
scale, the estimated effect is -0.206 (-0.297, -0.115). E-value lower
bound is 1.156, indicating evidence for causality.

\paragraph{Hours of exercise (log)}\label{hours-of-exercise-log}

The effect estimate (rd) is -0.063 (-0.09, -0.036). On the original
scale, the estimated effect is -16.294 minutes (-23.021 to -9.415
minutes). E-value lower bound is 1.217, indicating evidence for
causality.

All other effect estimates presented either weak or unreliable evidence
for causality.

\newpage{}

\subsubsection{Psychological Well-Being}\label{psychological-well-being}

\begin{figure}

\centering{

\pandocbounded{\includegraphics[keepaspectratio]{24-church-wb-long_files/figure-pdf/fig-psych-1.pdf}}

}

\caption{\label{fig-psych}Effects on Psychological Well-Being}

\end{figure}%

\begin{longtable}[]{@{}lrrrrr@{}}

\caption{\label{tbl-psych}Effects on Psychological Well-Being}

\tabularnewline

\toprule\noalign{}
& E{[}Y(1){]}-E{[}Y(0){]} & 2.5 \% & 97.5 \% & E\_Value &
E\_Val\_bound \\
\midrule\noalign{}
\endhead
\bottomrule\noalign{}
\endlastfoot
Fatigue & 0.00 & -0.02 & 0.02 & 1.03 & 1.00 \\
Anxiety & -0.02 & -0.04 & 0.00 & 1.16 & 1.00 \\
Depression & -0.04 & -0.07 & -0.02 & 1.24 & 1.14 \\
Rumination & -0.02 & -0.04 & 0.01 & 1.14 & 1.00 \\

\end{longtable}

\paragraph{Depression}\label{depression}

The effect estimate (rd) is -0.042 (-0.067, -0.017). On the original
scale, the estimated effect is -0.03 (-0.048, -0.012). E-value lower
bound is 1.139, indicating evidence for causality.

All other effect estimates presented either weak or unreliable evidence
for causality.

\newpage{}

\subsubsection{Present-Focussed
Well-Being}\label{present-focussed-well-being}

\begin{figure}

\centering{

\pandocbounded{\includegraphics[keepaspectratio]{24-church-wb-long_files/figure-pdf/fig-present-1.pdf}}

}

\caption{\label{fig-present}Effects on Person-Focussed Well-Being}

\end{figure}%

\begin{longtable}[]{@{}
  >{\raggedright\arraybackslash}p{(\linewidth - 10\tabcolsep) * \real{0.3951}}
  >{\raggedleft\arraybackslash}p{(\linewidth - 10\tabcolsep) * \real{0.1975}}
  >{\raggedleft\arraybackslash}p{(\linewidth - 10\tabcolsep) * \real{0.0741}}
  >{\raggedleft\arraybackslash}p{(\linewidth - 10\tabcolsep) * \real{0.0864}}
  >{\raggedleft\arraybackslash}p{(\linewidth - 10\tabcolsep) * \real{0.0988}}
  >{\raggedleft\arraybackslash}p{(\linewidth - 10\tabcolsep) * \real{0.1481}}@{}}

\caption{\label{tbl-present}Effects on Person-Focussed Well-Being}

\tabularnewline

\toprule\noalign{}
\begin{minipage}[b]{\linewidth}\raggedright
\end{minipage} & \begin{minipage}[b]{\linewidth}\raggedleft
E{[}Y(1){]}-E{[}Y(0){]}
\end{minipage} & \begin{minipage}[b]{\linewidth}\raggedleft
2.5 \%
\end{minipage} & \begin{minipage}[b]{\linewidth}\raggedleft
97.5 \%
\end{minipage} & \begin{minipage}[b]{\linewidth}\raggedleft
E\_Value
\end{minipage} & \begin{minipage}[b]{\linewidth}\raggedleft
E\_Val\_bound
\end{minipage} \\
\midrule\noalign{}
\endhead
\bottomrule\noalign{}
\endlastfoot
Body Satisfaction & 0.00 & -0.03 & 0.02 & 1.05 & 1.00 \\
Forgiveness & 0.01 & -0.01 & 0.04 & 1.13 & 1.00 \\
Perfectionism & -0.01 & -0.03 & 0.01 & 1.09 & 1.00 \\
PWB Standard Living & 0.05 & 0.02 & 0.07 & 1.26 & 1.18 \\
PWB Your Future Security & 0.00 & -0.02 & 0.03 & 1.04 & 1.00 \\
PWB Your Health & 0.00 & -0.03 & 0.02 & 1.06 & 1.00 \\
PWB Your Relationships & 0.01 & -0.02 & 0.03 & 1.09 & 1.00 \\
Self Control Have Lots & -0.01 & -0.04 & 0.01 & 1.13 & 1.00 \\
Self Control Wish More Reversed & -0.01 & -0.04 & 0.01 & 1.13 & 1.00 \\
Self Esteem & 0.00 & -0.02 & 0.02 & 1.00 & 1.00 \\
Sexual Satisfaction & 0.09 & 0.06 & 0.11 & 1.38 & 1.30 \\

\end{longtable}

\paragraph{Pwb standard living}\label{pwb-standard-living}

The effect estimate (rd) is 0.049 (0.024, 0.073). On the original scale,
the estimated effect is 0.102 (0.051, 0.154). E-value lower bound is
1.179, indicating evidence for causality.

\paragraph{Sexual satisfaction}\label{sexual-satisfaction}

The effect estimate (rd) is 0.085 (0.061, 0.108). On the original scale,
the estimated effect is 0.149 (0.108, 0.191). E-value lower bound is
1.304, indicating evidence for causality.

All other effect estimates presented either weak or unreliable evidence
for causality.

\newpage{}

\subsubsection{Life-Focussed Well-Being}\label{life-focussed-well-being}

\begin{figure}

\centering{

\pandocbounded{\includegraphics[keepaspectratio]{24-church-wb-long_files/figure-pdf/fig-life-1.pdf}}

}

\caption{\label{fig-life}Effects on Life-Focussed Well-Being}

\end{figure}%

\begin{longtable}[]{@{}
  >{\raggedright\arraybackslash}p{(\linewidth - 10\tabcolsep) * \real{0.2687}}
  >{\raggedleft\arraybackslash}p{(\linewidth - 10\tabcolsep) * \real{0.2388}}
  >{\raggedleft\arraybackslash}p{(\linewidth - 10\tabcolsep) * \real{0.0896}}
  >{\raggedleft\arraybackslash}p{(\linewidth - 10\tabcolsep) * \real{0.1045}}
  >{\raggedleft\arraybackslash}p{(\linewidth - 10\tabcolsep) * \real{0.1194}}
  >{\raggedleft\arraybackslash}p{(\linewidth - 10\tabcolsep) * \real{0.1791}}@{}}

\caption{\label{tbl-life}Effects on Life-Focussed Well-Being}

\tabularnewline

\toprule\noalign{}
\begin{minipage}[b]{\linewidth}\raggedright
\end{minipage} & \begin{minipage}[b]{\linewidth}\raggedleft
E{[}Y(1){]}-E{[}Y(0){]}
\end{minipage} & \begin{minipage}[b]{\linewidth}\raggedleft
2.5 \%
\end{minipage} & \begin{minipage}[b]{\linewidth}\raggedleft
97.5 \%
\end{minipage} & \begin{minipage}[b]{\linewidth}\raggedleft
E\_Value
\end{minipage} & \begin{minipage}[b]{\linewidth}\raggedleft
E\_Val\_bound
\end{minipage} \\
\midrule\noalign{}
\endhead
\bottomrule\noalign{}
\endlastfoot
Gratitude & 0.09 & 0.06 & 0.11 & 1.38 & 1.32 \\
Life Satisfaction & -0.01 & -0.03 & 0.01 & 1.12 & 1.00 \\
Meaning: Purpose & 0.16 & 0.14 & 0.18 & 1.59 & 1.53 \\
Meaning: Sense & 0.08 & 0.06 & 0.10 & 1.36 & 1.29 \\

\end{longtable}

\paragraph{Gratitude}\label{gratitude}

The effect estimate (rd) is 0.086 (0.065, 0.106). On the original scale,
the estimated effect is 0.078 (0.06, 0.097). E-value lower bound is
1.32, indicating evidence for causality.

\paragraph{Meaning: purpose}\label{meaning-purpose}

The effect estimate (rd) is 0.163 (0.142, 0.185). On the original scale,
the estimated effect is 0.231 (0.2, 0.261). E-value lower bound is
1.533, indicating evidence for causality.

\paragraph{Meaning: sense}\label{meaning-sense}

The effect estimate (rd) is 0.081 (0.058, 0.105). On the original scale,
the estimated effect is 0.097 (0.069, 0.125). E-value lower bound is
1.292, indicating evidence for causality.

All other effect estimates presented either weak or unreliable evidence
for causality.

\newpage{}

\subsubsection{Social-Focussed
Well-Being}\label{social-focussed-well-being}

\begin{figure}

\centering{

\pandocbounded{\includegraphics[keepaspectratio]{24-church-wb-long_files/figure-pdf/fig-social-1.pdf}}

}

\caption{\label{fig-social}Effects on Life-Focussed Well-Being}

\end{figure}%

\begin{longtable}[]{@{}lrrrrr@{}}

\caption{\label{tbl-social}Effects on Social Well-Being}

\tabularnewline

\toprule\noalign{}
& E{[}Y(1){]}-E{[}Y(0){]} & 2.5 \% & 97.5 \% & E\_Value &
E\_Val\_bound \\
\midrule\noalign{}
\endhead
\bottomrule\noalign{}
\endlastfoot
Belonging & 0.04 & 0.01 & 0.06 & 1.22 & 1.13 \\
Community & 0.05 & 0.03 & 0.07 & 1.27 & 1.19 \\
Support & -0.01 & -0.03 & 0.01 & 1.08 & 1.00 \\

\end{longtable}

\paragraph{Belonging}\label{belonging}

The effect estimate (rd) is 0.037 (0.015, 0.059). On the original scale,
the estimated effect is 0.041 (0.016, 0.065). E-value lower bound is
1.134, indicating evidence for causality.

\paragraph{Community}\label{community}

The effect estimate (rd) is 0.052 (0.029, 0.075). On the original scale,
the estimated effect is 0.081 (0.045, 0.117). E-value lower bound is
1.191, indicating evidence for causality.

All other effect estimates presented either weak or unreliable evidence
for causality.

\newpage{}

\subsection{Discussion}\label{discussion}

Our findings indicate that regular religious service attendance modestly
enhances meaning, gratitude, and sexual satisfaction while more strongly
boosting a sense of purpose, even in a predominantly secular setting
such as New Zealand. Although these causal effect estimates withstand
our robust handling of the time-series data, we suggest the following
caution when interpreting the results.

First, the absence of consistent or robust evidence for many
hypothesised effects does not imply that they do not exist.
Uncertainties may arise from unmeasured confounding or non-directed
measurement error, the later of which can bias estimates toward the null
(\citeproc{ref-bulbulia2024wierd}{Bulbulia 2024d};
\citeproc{ref-hernan2009MEASUREMENT}{Hernán and Cole 2009};
\citeproc{ref-vanderweele2012a}{VanderWeele and Hernán 2012}). In other
words, what appears to be a lack of effect might reflect limitations in
our measures rather than a genuine absence of causation.

Second, our causal effect estimates average over the entire target
population. We do not consider how causal effects of religious
attendance vary by faith tradition, intensity of belief, or aspects of
religious (or secular) upbringing. To investigate such heterogeneity
requires theoretically guiding interests, and the availability of
relevant observations for the populations of interest -- matters for
future interdisciplinary research.

Third, we contrast (1) regular religious service attendance for four
years with (2) no attendance for four years, however, numerous other
causal contrasts exist. For example, investigators might consider
``conversion'' or ``de-conversion'' scenarios. That is, one might
examine how well-being would differ if everyone in a target population
regularly attended for two waves and then ceased for two waves or if
everyone did not attend for two waves but then took up the attendance
for two waves (\citeproc{ref-vantongeren2020}{Van Tongeren \emph{et al.}
2020}). Contrasts could be made between these regimes and those we
considered here, or perhaps others. One might focus on longer or shorter
periods of exposure, as well. One might also limit analyses to
sub-populations for whom shifting attendance is more realistic (e.g.,
those already identifying as religious). Because there are innumerable
combinations of potential interventions and target population, it is
important to clarify specific research questions rather than vaguely
referring to the ``causal effects of religious service.'' Although we
cannot estimate all possible hypothetical experiments in a single study,
framing questions as target trials helps identify possible
interventions, another matter for future investigations.

Finally, it is important to emphasise that, at an individual level,
responses to religious activities are likely to differ. Our findings
represent an average treatment effect for the adult population of New
Zealand, but they should not be construed as life advice for any
particular individual.

Setting aside these limitations, one of the most robust results is that
religious engagement fosters a sense of meaning and purpose (and
especially of purpose). This finding supports long-standing traditions
of theory and empirical research on religion and well-being
(\citeproc{ref-eliade1984quest}{Eliade 1984};
\citeproc{ref-park2005religion}{Park 2005};
\citeproc{ref-park2005religion_meaning}{Park \emph{et al.} 2005}).
However, understanding why religious service attendance enhances a sense
of meaning, particularly a sense of purpose, remains an open question.
Evolutionary investigators might ask why a sense of meaning and purpose,
if it confers biological advantages, is not a universal ``default''
cognitive state (\citeproc{ref-sterelny2018religion}{Sterelny 2018})?
Some evolutionary investigators suggest that religious activity might
function partly as a hard-to-fake signal of commitment rather than as an
end in itself (\citeproc{ref-sosis2003cooperation}{Sosis and Bressler
2003}). Another compatible explanation is that religious service confers
purpose by assigning tasks and objective -- both secular and collective.
Charitable work, shared worship, and other communal practices can
promote reciprocity, mutual support, and a shared responsibility that
elevates abstract values into concrete goals. Notably, collective
purposes have been shown to enhance the cooperative effects of
collective activities, which may favour religious service in processes
of cultural selection (\citeproc{ref-reddish2013let}{Reddish \emph{et
al.} 2013}). However, why religious attendance might amplify meaning and
(especially) purpose awaits future inquiries.

We also observe a positive causal association between regular attendance
and gratitude. In classical Christian thought (for example, Aquinas),
religion is described as a virtue of justice called ``piety,''
acknowledging the sources of one's existence as a debt that cannot be
repaid (\citeproc{ref-bulbulia2023understanding}{Bulbulia 2023b}). In
contemporary language, we might call this a conception of religion as a
virtue of gratitude. Across religious traditions, gratitude to the gods
and ancestors, or a single God, is commonly rehearsed
(\citeproc{ref-watts2015pulotu}{Watts \emph{et al.} 2015}). Some
hypothesise that religions actively cultivate gratitude to foster
prosocial behaviours (\citeproc{ref-bono2004gratitude}{Bono \emph{et
al.} 2004}; \citeproc{ref-mccullough2001gratitude}{McCullough \emph{et
al.} 2001}). In any case, the conjecture that religious participation
expresses gratitude is consistent with our findings.

We find that religious attendance is causally associated with modest
improvements in sexual satisfaction. Earlier cross-sectional studies
have observed correlations between religiosity and marital sexual
satisfaction (\citeproc{ref-dew2020joint}{Dew \emph{et al.} 2020}).
Although these studies do reliably estimate causal magnitudes, our
results suggest a genuine but modest causal effect. Again, our purposes
here are descriptive, and we do not evaluate specific theories.
Speculating, religious service might align values, enhance
communication, and foster stable social support networks within couples,
thereby strengthening intimacy and trust and reduce conflict. From a
bio-functional perspective, such intimacy and trust would be especially
valuable for parenting -- the ``ultimate cooperation problem'' that
links biological fates (\citeproc{ref-Bulbulia_2015}{Bulbulia \emph{et
al.} 2015}). On the other hand, a darker side to such cooperation might
be sexual inequality (\citeproc{ref-deak2021individuals}{Deak \emph{et
al.} 2021}). Again, our findings should not be considered as offering
life advice. However, our findings motivate future investigation within
the emerging field of religion, biological sex, and parenting (refer to
Shaver \emph{et al.} (\citeproc{ref-shaver2020church}{2020}); Shaver
\emph{et al.} (\citeproc{ref-shaver2024religious}{2024}); Shaver and
White (\citeproc{ref-shaver2022integrating}{2022})). We contribute to
these horizons by observing that in contemporary New Zealand, the
relationship between religious attendance and sexual satisfaction is
likely causal.

Finally, we observe small but detectable effects on certain health
behaviours. Specifically, regular attendance slightly reduced hours of
exercise. This finding is consistent with a lost-opportunity
explanation, whereby time spent at religious services replaces other
physical activities. We observed a modest negative effect on BMI,
suggesting that religious engagement might offset this reduced exercise
through dietary practices or social norms around body weight, although
this is speculative and mechanisms await further investigation. These
smaller estimates illustrate how religious involvement can reallocate
time and attention in ways that shape diverse health outcomes, even if
the overall patterns are more modest than those observed for meaning,
gratitude, sexual satisfaction, and a sense of purpose

In summary, our findings show that in New Zeland, religious service
attendance can enrich certain facets of well-being, especially meaning,
gratitude, sexual satisfaction, and a sense of purpose. We find only
limited evidence for broader effects on other domains. Future research
might consider different populations and subgroups within New Zealand,
and a wider range of populations
(\citeproc{ref-johnson2022global}{Johnson and VanderWeele 2022};
\citeproc{ref-price2024science}{Price and Johnson 2024}). Such research
should also examine proximate and evolutionary mechanisms by which
religious service affects well-being, broadly conceived to include
biological well-being (\citeproc{ref-shaver2024religious}{Shaver
\emph{et al.} 2024}; \citeproc{ref-spake2024practical}{Spake \emph{et
al.} 2024}) and where possible, include objective measures both of
attendance and well-being (\citeproc{ref-shaver2021comparison}{Shaver
\emph{et al.} 2021}). As with all causal investigations, we should also
remain mindful that a practice that benefits certain groups, on average,
may not benefit every individual similarly. We recommend care when
imparting individual advice.

\subsubsection{Ethics}\label{ethics}

The University of Auckland Human Participants Ethics Committee reviews
the NZAVS every three years. Our most recent ethics approval statement
is as follows: The New Zealand Attitudes and Values Study was approved
by the University of Auckland Human Participants Ethics Committee on
26/05/2021 for six years until 26/05/2027, Reference Number UAHPEC22576.

\subsubsection{Data Availability}\label{data-availability}

The data described in the paper are part of the New Zealand Attitudes
and Values Study. Members of the NZAVS management team and research
group hold full copies of the NZAVS data. A de-identified dataset
containing only the variables analysed in this manuscript is available
upon request from the corresponding author or any member of the NZAVS
advisory board for replication or checking of any published study using
NZAVS data. The code for the analysis can be found at
\href{https://osf.io/ab7cx/}{OSF link}.

\subsubsection{Acknowledgements}\label{acknowledgements}

The New Zealand Attitudes and Values Study is supported by a grant from
the Templeton Religious Trust (TRT0196; TRT0418). JB received support
from the Max Plank Institute for the Science of Human History. The
funders had no role in preparing the manuscript or deciding to publish
it.

\subsubsection{Author Statement}\label{author-statement}

All authors had input into the manuscript. JB did the analysis and
developed the approach. CGS led data collection. JB, DD, CGS, KR, at GT
obtained funding.

\newpage{}

\subsection{Appendix A: Daily Data Collection}\label{appendix-timeline}

\newpage{}

Figure~\ref{fig-timeline} presents the New Zealand Attitudes and Values
Study Data Collection (2018 retained cohort) from years 2018-2014 (NZAVS
time 10--time 15).

\begin{figure}

\centering{

\pandocbounded{\includegraphics[keepaspectratio]{24-church-wb-long_files/figure-pdf/fig-timeline-1.pdf}}

}

\caption{\label{fig-timeline}Historgram of New Zealand Attitudes and
Values Study Daily Data Collection for Time 10 cohort: years 2018-2024.}

\end{figure}%

\newpage{}

\subsection{Appendix B: Measures and Demographic
Statistics}\label{appendix-baseline}

\subsubsection{Measures}\label{measures}

\subsection{Baseline Variables}\label{baseline-variables}

\paragraph{Age}\label{age}

\emph{What is your date of birth?}

We asked participants' ages in an open-ended question (``What is your
age?'' or ``What is your date of birth''). Developed for the NZAVS.

\paragraph{Agreeableness}\label{agreeableness}

\begin{itemize}
\tightlist
\item
  I sympathize with others' feelings.
\item
  I am not interested in other people's problems.
\item
  I feel others' emotions.
\item
  I am not really interested in others (reversed).
\end{itemize}

Mini-IPIP6 Agreeableness dimension: (i) I sympathize with others'
feelings. (ii) I am not interested in other people's problems. (r) (iii)
I feel others' emotions. (iv) I am not really interested in others. (r)
(\citeproc{ref-sibley2011}{Sibley \emph{et al.} 2011})

\paragraph{Alcohol Frequency}\label{alcohol-frequency}

\emph{``How often do you have a drink containing alcohol?''}

Participants could chose between the following responses: `(1 = Never -
I don't drink, 2 = Monthly or less, 3 = Up to 4 times a month, 4 = Up to
3 times a week, 5 = 4 or more times a week, 6 = Don't know)'
(\citeproc{ref-Ministry_of_Health_2013}{Health 2013})

\paragraph{Alcohol Intensity}\label{alcohol-intensity}

\emph{``How many drinks containing alcohol do you have on a typical day
when drinking alcohol? (number of drinks on a typical day when
drinking)''}

Participants responded using an open-ended box.
(\citeproc{ref-Ministry_of_Health_2013}{Health 2013})

\paragraph{Belong}\label{belong}

\begin{itemize}
\tightlist
\item
  Know that people in my life accept and value me.
\item
  Feel like an outsider (reversed).
\item
  Know that people around me share my attitudes and beliefs.
\end{itemize}

We assessed felt belongingness with three items adapted from the Sense
of Belonging Instrument (Hagerty \& Patusky, 1995): (1) ``Know that
people in my life accept and value me''; (2) ``Feel like an outsider'';
(3) ``Know that people around me share my attitudes and beliefs''.
Participants responded on a scale from 1 (Very Inaccurate) to 7 (Very
Accurate). The second item was reversely coded.
(\citeproc{ref-hagerty1995}{Hagerty and Patusky 1995})

\paragraph{Born in NZ}\label{born-in-nz}

\emph{Where were you born? (please be specific, e.g., which town/city?)}

Coded binary (1 = New Zealand; 0 = elsewhere.) Developed for the NZAVS.

\paragraph{Conscientiousness}\label{conscientiousness}

\begin{itemize}
\tightlist
\item
  I get chores done right away.
\item
  I like order.
\item
  I make a mess of things.
\item
  I often forget to put things back in their proper place.
\end{itemize}

Mini-IPIP6 Conscientiousness dimension: (i) I get chores done right
away. (ii) I like order. (iii) I make a mess of things. (r) (iv) I often
forget to put things back in their proper place. (r)
(\citeproc{ref-sibley2011}{Sibley \emph{et al.} 2011})

\paragraph{Education Level}\label{education-level}

\emph{What is your highest level of qualification?}

We asked participants, ``What is your highest level of qualification?''.
We coded participans highest finished degree according to the New
Zealand Qualifications Authority. Ordinal-Rank 0-10 NZREG codes (with
overseas school qualifications coded as Level 3, and all other ancillary
categories coded as missing) Developed for the NZAVS.

\paragraph{Employed (binary)}\label{employed-binary}

\emph{Are you currently employed (This includes self-employed of casual
work)?}

Binary response: (0 = No, 1 = Yes) Stats NZ Census Question

\paragraph{Ethnicity}\label{ethnicity}

\emph{Which ethnic group(s) do you belong to?}

Coded string: (1 = New Zealand European; 2 = Māori; 3 = Pacific; 4 =
Asian) NZ Census coding.

\paragraph{Extraversion}\label{extraversion}

\begin{itemize}
\tightlist
\item
  I am the life of the party.
\item
  I don't talk a lot (reversed).
\item
  I keep in the background (reversed).
\item
  I talk to a lot of different people at parties.
\end{itemize}

Mini-IPIP6 Extraversion dimension: (i) I am the life of the party. (ii)
I don't talk a lot. (r) (iii) I keep in the background. (r) (iv) I talk
to a lot of different people at parties.
(\citeproc{ref-sibley2011}{Sibley \emph{et al.} 2011})

\paragraph{Hlth Disability Binary}\label{hlth-disability-binary}

\emph{Do you have a health condition or disability that limits you and
that has lasted for 6+ months?}

We assessed disability with a one-item indicator adapted from Verbrugge
(1997). It asks, ``Do you have a health condition or disability that
limits you and that has lasted for 6+ months?'' (1 = Yes, 0 = No).
(\citeproc{ref-verbrugge1997}{Verbrugge 1997})

\paragraph{Honesty Humility}\label{honesty-humility}

\begin{itemize}
\tightlist
\item
  I feel entitled to more of everything (reversed).
\item
  I deserve more things in life (reversed).
\item
  I deserve more things in life (reversed).
\item
  I would get a lot of pleasure from owning expensive luxury goods
  (reversed).
\end{itemize}

Mini-IPIP6 Honesty-Humility dimension: (i) I feel entitled to more of
everything. (r) (ii) I deserve more things in life. (r) (iii) I would
like to be seen driving around in a very expensive car. (r) (iv) I would
get a lot of pleasure from owning expensive luxury goods. (r)
(\citeproc{ref-sibley2011}{Sibley \emph{et al.} 2011})

\paragraph{Log Hours Children}\label{log-hours-children}

\emph{Hours spent\ldots looking after children.}

We took the natural log of the response + 1.
(\citeproc{ref-sibley2011}{Sibley \emph{et al.} 2011})

\paragraph{Log Hours Commute}\label{log-hours-commute}

\emph{Hours spent\ldots travelling/commuting.}

We took the natural log of the response + 1. Developed for the NZAVS.

\paragraph{Log Hours Exercise}\label{log-hours-exercise}

\emph{Hours spent\ldots exercising/physical activity.}

We took the natural log of the response + 1.
(\citeproc{ref-sibley2011}{Sibley \emph{et al.} 2011})

\paragraph{Log Hours Housework}\label{log-hours-housework}

\emph{Hours spent\ldots housework/cooking.}

We took the natural log of the response + 1.
(\citeproc{ref-sibley2011}{Sibley \emph{et al.} 2011})

\paragraph{Log Household Income}\label{log-household-income}

\emph{Please estimate your total household income (before tax) for the
year XXXX.}

We took the natural log of the response + 1. Developed for the NZAVS.

\paragraph{Male (binary)}\label{male-binary}

\emph{We asked participants' gender in an open-ended question: ``what is
your gender?''}

Here, we coded all those who responded as Male as 1, and those who did
not as 0. (\citeproc{ref-fraser_coding_2020}{Fraser \emph{et al.} 2020})

\paragraph{Neuroticism}\label{neuroticism}

\begin{itemize}
\tightlist
\item
  I have frequent mood swings.
\item
  I am relaxed most of the time (reversed).
\item
  I get upset easily.
\item
  I seldom feel blue (reversed).
\end{itemize}

Mini-IPIP6 Neuroticism dimension: (i) I have frequent mood swings. (ii)
I am relaxed most of the time. (r) (iii) I get upset easily. (iv) I
seldom feel blue. (r) (\citeproc{ref-sibley2011}{Sibley \emph{et al.}
2011})

\paragraph{Not Heterosexual Binary}\label{not-heterosexual-binary}

\emph{How would you describe your sexual orientation? (e.g.,
heterosexual, homosexual, straight, gay, lesbian, bisexual, etc.)}

Open-ended question, coded as binary (not heterosexual = 1).
(\citeproc{ref-greaves2017diversity}{Greaves \emph{et al.} 2017})

\paragraph{NZ Deprevation Index 2018}\label{nz-deprevation-index-2018}

\emph{New Zealand Deprivation - Decile Index - Using 2018 Census Data}

Numerical: (1-10) (\citeproc{ref-atkinson2019}{Atkinson \emph{et al.}
2019})

\paragraph{NZSEI (Occupational Prestige
Index)}\label{nzsei-occupational-prestige-index}

\emph{We assessed occupational prestige and status using the New Zealand
Socio-economic Index 13 (NZSEI-13).}

This index uses the income, age, and education of a reference group, in
this case, the 2013 New Zealand census, to calculate a score for each
occupational group. Scores range from 10 (Lowest) to 90 (Highest). This
list of index scores for occupational groups was used to assign each
participant a NZSEI-13 score based on their occupation.
(\citeproc{ref-fahy2017}{Fahy \emph{et al.} 2017})

\paragraph{Openness}\label{openness}

\begin{itemize}
\tightlist
\item
  I have a vivid imagination.
\item
  I have difficulty understanding abstract ideas (reversed).
\item
  I do not have a good imagination (reversed).
\item
  I am not interested in abstract ideas (reversed).
\end{itemize}

Mini-IPIP6 Openness to Experience dimension: (i) I have a vivid
imagination. (ii) I have difficulty understanding abstract ideas. (r)
(iii) I do not have a good imagination. (r) (iv) I am not interested in
abstract ideas. (r) (\citeproc{ref-sibley2011}{Sibley \emph{et al.}
2011})

\paragraph{Parent (binary)}\label{parent-binary}

\emph{If you are a parent, in which year was your eldest child born?}

Parents were coded as 1, while the others were coded as 0. Developed for
the NZAVS.

\paragraph{Partner (binary)}\label{partner-binary}

\emph{What is your relationship status? (e.g., single, married,
de-facto, civil union, widowed, living together, etc.)}

Coded as binary (has partner = 1). Developed for the NZAVS.

\paragraph{Political Conservative}\label{political-conservative}

\emph{Please rate how politically liberal versus conservative you see
yourself as being.}

Ordinal response: (1 = Extremely Liberal, 7 = Extremely Conservative)
(\citeproc{ref-jost_end_2006-1}{Jost 2006})

\paragraph{Power No Control Composite}\label{power-no-control-composite}

\begin{itemize}
\tightlist
\item
  I do not have enough power or control over important parts of my life.
\item
  Other people have too much power or control over important parts of my
  life.
\end{itemize}

Ordinal response: (1 = Strongly Disagree, 7 = Strongly Agree)
(\citeproc{ref-overall2016power}{Overall \emph{et al.} 2016})

\paragraph{Rural Gch 2018 Levels}\label{rural-gch-2018-levels}

\emph{High Urban Accessibility = 1, Medium Urban Accessibility = 2, Low
Urban Accessibility = 3, Remote = 4, Very Remote = 5.}

``Participants residence locations were coded according to a five-level
ordinal categorisation ranging from Urban to Rural.''
(\citeproc{ref-whitehead2023unmasking}{Whitehead \emph{et al.} 2023})

\paragraph{Right Wing
Authoritarianism}\label{right-wing-authoritarianism}

\begin{itemize}
\tightlist
\item
  It is always better to trust the judgment of the proper authorities in
  government and religion than to listen to the noisy rabble-rousers in
  our society who are trying to create doubt in people's minds.
\item
  It would be best for everyone if the proper authorities censored
  magazines so that people could not get their hands on trashy and
  disgusting material.
\item
  Our country will be destroyed some day if we do not smash the
  perversions eating away at our moral fibre and traditional beliefs.
\item
  People should pay less attention to The Bible and other old
  traditional forms of religious guidance, and instead develop their own
  personal standards of what is moral and immoral.
\item
  Atheists and others who have rebelled against established religions
  are no doubt every bit as good and virtuous as those who attend church
  regularly.
\item
  Some of the best people in our country are those who are challenging
  our government, criticizing religion, and ignoring the ``normal way''
  things are supposed to be done (reversed).
\end{itemize}

(\citeproc{ref-altemeyer1996authoritarian}{Altemeyer 1996})

\paragraph{Sample Frame Opt-In
(binary)}\label{sample-frame-opt-in-binary}

\emph{Participant was not randomly sampled from the New Zealand
Electoral Roll.}

Code string (Binary): (0 = No, 1 = Yes) Developed for the NZAVS.

\paragraph{Social Dominance
Orientation}\label{social-dominance-orientation}

\begin{itemize}
\tightlist
\item
  It is OK if some groups have more of a chance in life than others.
\item
  Inferior groups should stay in their place.
\item
  To get ahead in life, it is sometimes okay to step on other groups.
\item
  We should have increased social equality (reversed).
\item
  It would be good if groups could be equal (reversed).
\item
  We should do what we can to equalise conditions for different groups
  (reversed).
\end{itemize}

(\citeproc{ref-sidanius1999social}{Sidanius and Pratto 1999})

\paragraph{Smoker (binary)}\label{smoker-binary}

\emph{Do you currently smoke tobacco cigarettes?}

Binary smoking indicator (0 = No, 1 = Yes). Developed for NZAVS.

\subsection{Exposure Variable}\label{exposure-variable}

\paragraph{Monthly Religious Service}\label{monthly-religious-service}

\emph{Do you identify with a religion and/or spiritual group?
-\textgreater{} How many times did you attend a church or place of
worship in the last month?}

Numerical: open-ended response (\citeproc{ref-sibley2012}{Sibley and
Bulbulia 2012})

\subsection{Outcome Variables}\label{outcome-variables}

\subsubsection{Health}\label{health-1}

\paragraph{BMI}\label{bmi-1}

\emph{What is your height? (metres)'' and ``What is your weight? (kg).}

Based on participants indication of their height and weight we
calculated the BMI by dividing the weight in kilograms by the square of
the height in meters. Developed for the NZAVS.

\paragraph{Sleep}\label{sleep}

\emph{During the past month, on average, how many hours of actual sleep
did you get per night?}

Open ended response. (\citeproc{ref-buysse1989pittsburgh}{Buysse
\emph{et al.} 1989})

\paragraph{Hours of Exercise (log)}\label{hours-of-exercise-log-1}

\emph{Hours spent\ldots exercising/physical activity.}

We took the natural log of the response + 1.
(\citeproc{ref-sibley2011}{Sibley \emph{et al.} 2011})

\paragraph{Short Form Health}\label{short-form-health}

\emph{In general, would you say your health is\ldots{}}

Ordinal response: (1 = Poor, 7 = Excellent)
(\citeproc{ref-instrument1992mos}{Instrument Ware Jr and Sherbourne
1992})

\subsubsection{Psychological}\label{psychological}

\paragraph{Fatigue}\label{fatigue}

\emph{During the last 30 days, how often did \ldots{} you feel
exhausted?}

Ordinal response: (0 = None Of The Time; 1 = A Little Of The Time; 2=
Some Of The Time; 3 = Most Of The Time; 4 = All Of The Time)
(\citeproc{ref-sibley2020}{Sibley \emph{et al.} 2020})

\paragraph{Anxiety}\label{anxiety}

\begin{itemize}
\tightlist
\item
  During the past 30 days, how often did\ldots you feel restless or
  fidgety?
\item
  During the past 30 days, how often did\ldots you feel that everything
  was an effort?
\item
  During the past 30 days, how often did\ldots you feel nervous?
\end{itemize}

Ordinal response: (0 = None Of The Time; 1 = A Little Of The Time; 2=
Some Of The Time; 3 = Most Of The Time; 4 = All Of The Time)
(\citeproc{ref-kessler2002}{Kessler \emph{et al.} 2002})

\paragraph{Depression}\label{depression-1}

\begin{itemize}
\tightlist
\item
  During the past 30 days, how often did\ldots you feel hopeless?
\item
  During the past 30 days, how often did\ldots you feel so depressed
  that nothing could cheer you up?
\item
  During the past 30 days, how often did\ldots you feel you feel
  restless or fidgety?
\end{itemize}

Ordinal response: (0 = None Of The Time; 1 = A Little Of The Time; 2=
Some Of The Time; 3 = Most Of The Time; 4 = All Of The Time)
(\citeproc{ref-kessler2002}{Kessler \emph{et al.} 2002})

\paragraph{Rumination}\label{rumination}

\emph{During the last 30 days, how often did\ldots you have negative
thoughts that repeated over and over?}

Ordinal responses: 0 = None of The Time, 1 = A little of The Time, 2 =
Some of The Time, 3 = Most of The Time, 4 = All of The Time.
(\citeproc{ref-nolen-hoeksema_effects_1993}{Nolen-hoeksema and Morrow
1993})

\subsubsection{Present Focussed}\label{present-focussed}

\paragraph{Body Satisfaction}\label{body-satisfaction}

\emph{I am satisfied with the appearance, size and shape of my body.}

Ordinal response scale (1 = Very inaccurate to 7 = Very accurate).
(\citeproc{ref-stronge2015facebook}{Stronge \emph{et al.} 2015})

\paragraph{Forgiveness}\label{forgiveness}

\begin{itemize}
\tightlist
\item
  Sometimes I can't sleep because of thinking about past wrongs I have
  suffered.
\item
  I can usually forgive and forget when someone does me wrong.
\item
  I find myself regularly thinking about past times that I have been
  wronged.
\end{itemize}

We assessed participants' forgiveness using reversed scores of a the
NZAVS ``vengeful rumination scale.'' This scale contains three items,
adapted from Caprara (\citeproc{ref-caprara_indicators_1986}{1986}) and
Berry \emph{et al.} (\citeproc{ref-berry_forgivingness_2005}{2005}), and
developed for NZAVS, ordinal response scale 1-7 (1 = Strongly Disagree
to 7 = Strongly Agree). Combines Berry \emph{et al.}
(\citeproc{ref-berry_forgivingness_2005}{2005}) and Caprara
(\citeproc{ref-caprara_indicators_1986}{1986}).

\paragraph{Perfectionism}\label{perfectionism}

\begin{itemize}
\tightlist
\item
  Doing my best never seems to be enough.
\item
  My performance rarely measures up to my standards.
\item
  I am hardly ever satisfied with my performance.
\end{itemize}

Ordinal response (1 = Strongly Disagree to 7 = Strongly Agree).
(\citeproc{ref-rice_short_2014}{Rice \emph{et al.} 2014})

\paragraph{PWB Standard Living}\label{pwb-standard-living-1}

\emph{Please rate your level of satisfaction with the following aspects
of your life\ldots Your standard of living.}

Ordinal response (0 = completely dissatisfied to 10 = completely
satisfied). (\citeproc{ref-cummins_developing_2003}{Cummins \emph{et
al.} 2003})

\paragraph{Pwb your Future Security}\label{pwb-your-future-security}

\emph{Please rate your level of satisfaction with the following aspects
of your life\ldots Your future security.}

Ordinal response (0 = completely dissatisfied to 10 = completely
satisfied). (\citeproc{ref-cummins_developing_2003}{Cummins \emph{et
al.} 2003})

\paragraph{PWB Your Health}\label{pwb-your-health}

\emph{Please rate your level of satisfaction with the following aspects
of your life\ldots Your health.}

Ordinal response (0 = completely dissatisfied to 10 = completely
satisfied). (\citeproc{ref-cummins_developing_2003}{Cummins \emph{et
al.} 2003})

\paragraph{PWB Your Relationships}\label{pwb-your-relationships}

\emph{Please rate your level of satisfaction with the following aspects
of your life\ldots Your personal relationships.}

Ordinal response (0 = completely dissatisfied to 10 = completely
satisfied). (\citeproc{ref-cummins_developing_2003}{Cummins \emph{et
al.} 2003})

\paragraph{Self Control Have Lots}\label{self-control-have-lots}

\emph{In general, I have a lot of self-control.}

Ordinal response (1 = Strongly Disagree to 7 = Strongly Agree).
(\citeproc{ref-tangney_high_2004}{Tangney \emph{et al.} 2004})

\paragraph{Self Control Wish More
Reversed}\label{self-control-wish-more-reversed}

\emph{I wish I had more self-discipline.}

Ordinal response (1 = Strongly Disagree to 7 = Strongly Agree).
(\citeproc{ref-tangney_high_2004}{Tangney \emph{et al.} 2004})

\paragraph{Self-Esteem}\label{self-esteem}

\begin{itemize}
\tightlist
\item
  On the whole am satisfied with myself.
\item
  Take a positive attitude toward myself.
\item
  Am inclined to feel that I am a failure (reversed).
\end{itemize}

Ordinal response (1 = Very inaccurate to 7 = Very accurate).
(\citeproc{ref-Rosenberg1965}{Rosenberg 1965})

\paragraph{Sexual Satisfaction}\label{sexual-satisfaction-1}

\emph{How satisfied are you with your sex life?}

Participants were asked to report their sexual orientation; ordinal
response: 1 = Not satisfied to 7 = Very satisfied. Developed for the
NZAVS

\subsubsection{Life Focussed}\label{life-focussed}

\paragraph{Gratitude}\label{gratitude-1}

\begin{itemize}
\tightlist
\item
  I have much in my life to be thankful for.
\item
  When I look at the world, I don't see much to be grateful for
  (reversed).
\item
  I am grateful to a wide variety of people.
\end{itemize}

Ordinal response scale 1 = Strongly Disagree to 7 = Strongly Agree.
(\citeproc{ref-mccullough_grateful_2002}{McCullough \emph{et al.} 2002})

\paragraph{Life Satisfaction}\label{life-satisfaction}

\begin{itemize}
\tightlist
\item
  I am satisfied with my life.
\item
  In most ways my life is close to ideal.
\end{itemize}

Ordinal response (1 = Strongly Disagree to 7 = Strongly Agree).
(\citeproc{ref-diener1985a}{Diener \emph{et al.} 1985})

\paragraph{Meaning Purpose}\label{meaning-purpose-1}

\emph{My life has a clear sense of purpose}

Ordinal response (1 = Strongly Disagree to 7 = Strongly Agree).
(\citeproc{ref-steger_meaning_2006}{Steger \emph{et al.} 2006})

\paragraph{Meaning Sense}\label{meaning-sense-1}

\emph{I have a good sense of what makes my life meaningful.}

Ordinal response (1 = Strongly Disagree to 7 = Strongly Agree).
(\citeproc{ref-steger_meaning_2006}{Steger \emph{et al.} 2006})

\subsubsection{Social}\label{social}

\paragraph{Belonging}\label{belonging-1}

\begin{itemize}
\tightlist
\item
  Know that people in my life accept and value me.
\item
  Feel like an outsider (reversed).
\item
  Know that people around me share my attitudes and beliefs.
\end{itemize}

We assessed felt belongingness with three items adapted from the Sense
of Belonging Instrument (Hagerty \& Patusky, 1995): (1) ``Know that
people in my life accept and value me''; (2) ``Feel like an outsider'';
(3) ``Know that people around me share my attitudes and beliefs''.
Participants responded on a scale from 1 (Very Inaccurate) to 7 (Very
Accurate). The second item was reversely coded.
(\citeproc{ref-hagerty1995}{Hagerty and Patusky 1995})

\paragraph{Community Belonging}\label{community-belonging}

\emph{I feel a sense of community with others in my local
neighbourhood.}

Ordinal response (1 = Strongly Disagree to 7 = Strongly Agree).
(\citeproc{ref-sengupta2013}{Sengupta \emph{et al.} 2013})

\paragraph{Support}\label{support}

\begin{itemize}
\tightlist
\item
  There are people I can depend on to help me if I really need it.
\item
  There is no one I can turn to for guidance in times of stress
  (reversed).
\item
  I know there are people I can turn to when I need help.
\end{itemize}

Ordinal response: (1 = Strongly Disagree, 7 = Strongly Agree)
(\citeproc{ref-cutrona1987}{Cutrona and Russell 1987})

\subsubsection{Sample Demographic
Statistics}\label{sample-demographic-statistics}

Table~\ref{tbl-baseline} presents sample demographic statistics.

\begingroup\fontsize{6}{8}\selectfont
\begingroup\fontsize{6}{8}\selectfont

\begin{longtable}[t]{lllllll}

\caption{\label{tbl-baseline}Demographic statistics for New Zealand
Attitudes and Values Cohort waves 2018.}

\tabularnewline

\toprule
  & 2018 & 2019 & 2020 & 2021 & 2022 & 2023\\
\midrule
\endfirsthead
\multicolumn{7}{@{}l}{\textit{(continued)}}\\
\toprule
  & 2018 & 2019 & 2020 & 2021 & 2022 & 2023\\
\midrule
\endhead

\endfoot
\bottomrule
\endlastfoot
\cellcolor{gray!10}{} & \cellcolor{gray!10}{(N=46377)} & \cellcolor{gray!10}{(N=46377)} & \cellcolor{gray!10}{(N=46377)} & \cellcolor{gray!10}{(N=46377)} & \cellcolor{gray!10}{(N=46377)} & \cellcolor{gray!10}{(N=46377)}\\
\addlinespace[0.3em]
\multicolumn{7}{l}{\textbf{Age}}\\
\hspace{1em}Mean (SD) & 48.6 (13.9) & 51.2 (13.5) & 52.7 (13.4) & 54.0 (13.3) & 55.4 (13.2) & 56.4 (13.1)\\
\cellcolor{gray!10}{\hspace{1em}Median [Min, Max]} & \cellcolor{gray!10}{51.0 [18.0, 99.0]} & \cellcolor{gray!10}{54.0 [19.0, 96.0]} & \cellcolor{gray!10}{55.0 [20.0, 96.0]} & \cellcolor{gray!10}{57.0 [21.0, 97.0]} & \cellcolor{gray!10}{58.0 [22.0, 98.0]} & \cellcolor{gray!10}{59.0 [23.0, 99.0]}\\
\hspace{1em}Missing & 0 (0\%) & 12422 (26.8\%) & 15053 (32.5\%) & 19456 (42.0\%) & 22677 (48.9\%) & 25006 (53.9\%)\\
\addlinespace[0.3em]
\multicolumn{7}{l}{\textbf{Agreeableness}}\\
\cellcolor{gray!10}{\hspace{1em}Mean (SD)} & \cellcolor{gray!10}{5.35 (0.988)} & \cellcolor{gray!10}{5.37 (0.972)} & \cellcolor{gray!10}{5.37 (0.977)} & \cellcolor{gray!10}{5.35 (1.00)} & \cellcolor{gray!10}{5.33 (0.993)} & \cellcolor{gray!10}{5.33 (0.998)}\\
\hspace{1em}Median [Min, Max] & 5.50 [1.00, 7.00] & 5.50 [1.00, 7.00] & 5.50 [1.00, 7.00] & 5.50 [1.00, 7.00] & 5.50 [1.00, 7.00] & 5.50 [1.00, 7.00]\\
\cellcolor{gray!10}{\hspace{1em}Missing} & \cellcolor{gray!10}{400 (0.9\%)} & \cellcolor{gray!10}{12705 (27.4\%)} & \cellcolor{gray!10}{15266 (32.9\%)} & \cellcolor{gray!10}{19657 (42.4\%)} & \cellcolor{gray!10}{22777 (49.1\%)} & \cellcolor{gray!10}{25054 (54.0\%)}\\
\addlinespace[0.3em]
\multicolumn{7}{l}{\textbf{Alcohol Frequency}}\\
\hspace{1em}Mean (SD) & 2.16 (1.34) & 2.17 (1.35) & 2.18 (1.35) & 2.13 (1.38) & 2.10 (1.37) & 2.04 (1.36)\\
\cellcolor{gray!10}{\hspace{1em}Median [Min, Max]} & \cellcolor{gray!10}{2.00 [0, 5.00]} & \cellcolor{gray!10}{2.00 [0, 5.00]} & \cellcolor{gray!10}{2.00 [0, 5.00]} & \cellcolor{gray!10}{2.00 [0, 5.00]} & \cellcolor{gray!10}{2.00 [0, 5.00]} & \cellcolor{gray!10}{2.00 [0, 5.00]}\\
\hspace{1em}Missing & 1598 (3.4\%) & 13069 (28.2\%) & 15644 (33.7\%) & 19841 (42.8\%) & 22832 (49.2\%) & 25538 (55.1\%)\\
\addlinespace[0.3em]
\multicolumn{7}{l}{\textbf{Alcohol Intensity}}\\
\cellcolor{gray!10}{\hspace{1em}Mean (SD)} & \cellcolor{gray!10}{2.17 (2.17)} & \cellcolor{gray!10}{2.00 (1.99)} & \cellcolor{gray!10}{1.97 (1.98)} & \cellcolor{gray!10}{1.88 (1.85)} & \cellcolor{gray!10}{1.86 (1.86)} & \cellcolor{gray!10}{2.00 (1.89)}\\
\hspace{1em}Median [Min, Max] & 2.00 [0, 30.0] & 2.00 [0, 36.0] & 2.00 [0, 36.0] & 2.00 [0, 32.0] & 1.50 [0, 40.0] & 2.00 [0, 48.0]\\
\cellcolor{gray!10}{\hspace{1em}Missing} & \cellcolor{gray!10}{2751 (5.9\%)} & \cellcolor{gray!10}{13816 (29.8\%)} & \cellcolor{gray!10}{16302 (35.2\%)} & \cellcolor{gray!10}{20437 (44.1\%)} & \cellcolor{gray!10}{23607 (50.9\%)} & \cellcolor{gray!10}{27939 (60.2\%)}\\
\addlinespace[0.3em]
\multicolumn{7}{l}{\textbf{Belong}}\\
\hspace{1em}Mean (SD) & 5.14 (1.08) & 5.13 (1.07) & 5.05 (1.09) & 5.16 (1.10) & 5.16 (1.10) & 5.23 (1.10)\\
\cellcolor{gray!10}{\hspace{1em}Median [Min, Max]} & \cellcolor{gray!10}{5.33 [1.00, 7.00]} & \cellcolor{gray!10}{5.33 [1.00, 7.00]} & \cellcolor{gray!10}{5.00 [1.00, 7.00]} & \cellcolor{gray!10}{5.33 [1.00, 7.00]} & \cellcolor{gray!10}{5.33 [1.00, 7.00]} & \cellcolor{gray!10}{5.33 [1.00, 7.00]}\\
\hspace{1em}Missing & 397 (0.9\%) & 12704 (27.4\%) & 15274 (32.9\%) & 19702 (42.5\%) & 22810 (49.2\%) & 25100 (54.1\%)\\
\addlinespace[0.3em]
\multicolumn{7}{l}{\textbf{Born in NZ}}\\
\cellcolor{gray!10}{\hspace{1em}Mean (SD)} & \cellcolor{gray!10}{0.783 (0.412)} & \cellcolor{gray!10}{0.783 (0.412)} & \cellcolor{gray!10}{0.783 (0.412)} & \cellcolor{gray!10}{0.783 (0.412)} & \cellcolor{gray!10}{0.783 (0.412)} & \cellcolor{gray!10}{0.783 (0.412)}\\
\hspace{1em}Median [Min, Max] & 1.00 [0, 1.00] & 1.00 [0, 1.00] & 1.00 [0, 1.00] & 1.00 [0, 1.00] & 1.00 [0, 1.00] & 1.00 [0, \vphantom{3} 1.00]\\
\cellcolor{gray!10}{\hspace{1em}Missing} & \cellcolor{gray!10}{152 (0.3\%)} & \cellcolor{gray!10}{152 (0.3\%)} & \cellcolor{gray!10}{152 (0.3\%)} & \cellcolor{gray!10}{152 (0.3\%)} & \cellcolor{gray!10}{152 (0.3\%)} & \cellcolor{gray!10}{152 (0.3\%)}\\
\addlinespace[0.3em]
\multicolumn{7}{l}{\textbf{Conscientiousness}}\\
\hspace{1em}Mean (SD) & 5.11 (1.06) & 5.13 (1.04) & 5.14 (1.03) & 5.15 (1.05) & 5.16 (1.04) & 5.16 (1.04)\\
\cellcolor{gray!10}{\hspace{1em}Median [Min, Max]} & \cellcolor{gray!10}{5.25 [1.00, 7.00]} & \cellcolor{gray!10}{5.25 [1.00, 7.00]} & \cellcolor{gray!10}{5.25 [1.00, 7.00]} & \cellcolor{gray!10}{5.25 [1.00, 7.00]} & \cellcolor{gray!10}{5.25 [1.00, 7.00]} & \cellcolor{gray!10}{5.25 [1.00, 7.00]}\\
\hspace{1em}Missing & 392 (0.8\%) & 12702 (27.4\%) & 15265 (32.9\%) & 19657 (42.4\%) & 22774 (49.1\%) & 25079 (54.1\%)\\
\addlinespace[0.3em]
\multicolumn{7}{l}{\textbf{Education Level}}\\
\cellcolor{gray!10}{\hspace{1em}no\_qualification} & \cellcolor{gray!10}{1177 (2.5\%)} & \cellcolor{gray!10}{1044 (2.3\%)} & \cellcolor{gray!10}{956 (2.1\%)} & \cellcolor{gray!10}{916 (2.0\%)} & \cellcolor{gray!10}{900 (1.9\%)} & \cellcolor{gray!10}{881 (1.9\%)}\\
\hspace{1em}cert\_1\_to\_4 & 16277 (35.1\%) & 15353 (33.1\%) & 14813 (31.9\%) & 14465 (31.2\%) & 14214 (30.6\%) & 14037 (30.3\%)\\
\cellcolor{gray!10}{\hspace{1em}cert\_5\_to\_6} & \cellcolor{gray!10}{5821 (12.6\%)} & \cellcolor{gray!10}{6016 (13.0\%)} & \cellcolor{gray!10}{6116 (13.2\%)} & \cellcolor{gray!10}{6151 (13.3\%)} & \cellcolor{gray!10}{6206 (13.4\%)} & \cellcolor{gray!10}{6209 (13.4\%)}\\
\hspace{1em}university & 12311 (26.5\%) & 12529 (27.0\%) & 12585 (27.1\%) & 12572 (27.1\%) & 12495 (26.9\%) & 12448 (26.8\%)\\
\cellcolor{gray!10}{\hspace{1em}post\_grad} & \cellcolor{gray!10}{5034 (10.9\%)} & \cellcolor{gray!10}{5376 (11.6\%)} & \cellcolor{gray!10}{5655 (12.2\%)} & \cellcolor{gray!10}{5881 (12.7\%)} & \cellcolor{gray!10}{6042 (13.0\%)} & \cellcolor{gray!10}{6098 (13.1\%)}\\
\hspace{1em}masters & 3857 (8.3\%) & 4088 (8.8\%) & 4262 (9.2\%) & 4386 (9.5\%) & 4504 (9.7\%) & 4601 (9.9\%)\\
\cellcolor{gray!10}{\hspace{1em}doctorate} & \cellcolor{gray!10}{1111 (2.4\%)} & \cellcolor{gray!10}{1182 (2.5\%)} & \cellcolor{gray!10}{1227 (2.6\%)} & \cellcolor{gray!10}{1268 (2.7\%)} & \cellcolor{gray!10}{1314 (2.8\%)} & \cellcolor{gray!10}{1414 (3.0\%)}\\
\hspace{1em}Missing & 789 (1.7\%) & 789 (1.7\%) & 763 (1.6\%) & 738 (1.6\%) & 702 (1.5\%) & 689 (1.5\%)\\
\addlinespace[0.3em]
\multicolumn{7}{l}{\textbf{Employed (binary)}}\\
\cellcolor{gray!10}{\hspace{1em}Mean (SD)} & \cellcolor{gray!10}{0.795 (0.404)} & \cellcolor{gray!10}{0.774 (0.418)} & \cellcolor{gray!10}{0.779 (0.415)} & \cellcolor{gray!10}{0.758 (0.428)} & \cellcolor{gray!10}{0.722 (0.448)} & \cellcolor{gray!10}{0.709 (0.454)}\\
\hspace{1em}Median [Min, Max] & 1.00 [0, 1.00] & 1.00 [0, 1.00] & 1.00 [0, 1.00] & 1.00 [0, 1.00] & 1.00 [0, 1.00] & 1.00 [0, \vphantom{2} 1.00]\\
\cellcolor{gray!10}{\hspace{1em}Missing} & \cellcolor{gray!10}{11 (0.0\%)} & \cellcolor{gray!10}{12693 (27.4\%)} & \cellcolor{gray!10}{15206 (32.8\%)} & \cellcolor{gray!10}{19586 (42.2\%)} & \cellcolor{gray!10}{22816 (49.2\%)} & \cellcolor{gray!10}{25540 (55.1\%)}\\
\addlinespace[0.3em]
\multicolumn{7}{l}{\textbf{Ethnicity}}\\
\hspace{1em}euro & 36915 (79.6\%) & 36915 (79.6\%) & 36915 (79.6\%) & 36915 (79.6\%) & 36915 (79.6\%) & 36915 (79.6\%)\\
\cellcolor{gray!10}{\hspace{1em}maori} & \cellcolor{gray!10}{5311 (11.5\%)} & \cellcolor{gray!10}{5311 (11.5\%)} & \cellcolor{gray!10}{5311 (11.5\%)} & \cellcolor{gray!10}{5311 (11.5\%)} & \cellcolor{gray!10}{5311 (11.5\%)} & \cellcolor{gray!10}{5311 (11.5\%)}\\
\hspace{1em}pacific & 1109 (2.4\%) & 1109 (2.4\%) & 1109 (2.4\%) & 1109 (2.4\%) & 1109 (2.4\%) & 1109 (2.4\%)\\
\cellcolor{gray!10}{\hspace{1em}asian} & \cellcolor{gray!10}{2453 (5.3\%)} & \cellcolor{gray!10}{2453 (5.3\%)} & \cellcolor{gray!10}{2453 (5.3\%)} & \cellcolor{gray!10}{2453 (5.3\%)} & \cellcolor{gray!10}{2453 (5.3\%)} & \cellcolor{gray!10}{2453 (5.3\%)}\\
\hspace{1em}Missing & 589 (1.3\%) & 589 (1.3\%) & 589 (1.3\%) & 589 (1.3\%) & 589 (1.3\%) & 589 (1.3\%)\\
\addlinespace[0.3em]
\multicolumn{7}{l}{\textbf{Extraversion}}\\
\cellcolor{gray!10}{\hspace{1em}Mean (SD)} & \cellcolor{gray!10}{3.91 (1.20)} & \cellcolor{gray!10}{3.85 (1.19)} & \cellcolor{gray!10}{3.82 (1.19)} & \cellcolor{gray!10}{3.77 (1.23)} & \cellcolor{gray!10}{3.75 (1.23)} & \cellcolor{gray!10}{3.75 (1.23)}\\
\hspace{1em}Median [Min, Max] & 4.00 [1.00, 7.00] & 3.75 [1.00, 7.00] & 3.75 [1.00, 7.00] & 3.75 [1.00, 7.00] & 3.75 [1.00, 7.00] & 3.75 [1.00, 7.00]\\
\cellcolor{gray!10}{\hspace{1em}Missing} & \cellcolor{gray!10}{392 (0.8\%)} & \cellcolor{gray!10}{12703 (27.4\%)} & \cellcolor{gray!10}{15263 (32.9\%)} & \cellcolor{gray!10}{19659 (42.4\%)} & \cellcolor{gray!10}{22783 (49.1\%)} & \cellcolor{gray!10}{25067 (54.1\%)}\\
\addlinespace[0.3em]
\multicolumn{7}{l}{\textbf{Hlth Disability Binary}}\\
\hspace{1em}Mean (SD) & 0.225 (0.418) & 0.234 (0.423) & 0.263 (0.440) & 0.276 (0.447) & 0.311 (0.463) & 0.321 (0.467)\\
\cellcolor{gray!10}{\hspace{1em}Median [Min, Max]} & \cellcolor{gray!10}{0 [0, 1.00]} & \cellcolor{gray!10}{0 [0, 1.00]} & \cellcolor{gray!10}{0 [0, 1.00]} & \cellcolor{gray!10}{0 [0, 1.00]} & \cellcolor{gray!10}{0 [0, 1.00]} & \cellcolor{gray!10}{0 [0, \vphantom{4} 1.00]}\\
\hspace{1em}Missing & 893 (1.9\%) & 12912 (27.8\%) & 15526 (33.5\%) & 19848 (42.8\%) & 23135 (49.9\%) & 25268 (54.5\%)\\
\addlinespace[0.3em]
\multicolumn{7}{l}{\textbf{Honesty Humility}}\\
\cellcolor{gray!10}{\hspace{1em}Mean (SD)} & \cellcolor{gray!10}{5.41 (1.18)} & \cellcolor{gray!10}{5.55 (1.14)} & \cellcolor{gray!10}{5.59 (1.13)} & \cellcolor{gray!10}{5.65 (1.13)} & \cellcolor{gray!10}{5.69 (1.13)} & \cellcolor{gray!10}{5.71 (1.12)}\\
\hspace{1em}Median [Min, Max] & 5.50 [1.00, 7.00] & 5.75 [1.00, 7.00] & 5.75 [1.00, 7.00] & 5.75 [1.00, 7.00] & 6.00 [1.00, 7.00] & 6.00 [1.00, 7.00]\\
\cellcolor{gray!10}{\hspace{1em}Missing} & \cellcolor{gray!10}{396 (0.9\%)} & \cellcolor{gray!10}{12705 (27.4\%)} & \cellcolor{gray!10}{15274 (32.9\%)} & \cellcolor{gray!10}{19640 (42.3\%)} & \cellcolor{gray!10}{22767 (49.1\%)} & \cellcolor{gray!10}{25049 (54.0\%)}\\
\addlinespace[0.3em]
\multicolumn{7}{l}{\textbf{Hours Children}}\\
\hspace{1em}Mean (SD) & 14.0 (32.3) & 12.9 (31.1) & 11.9 (30.0) & 10.4 (27.4) & 10.1 (27.0) & 9.97 (26.8)\\
\cellcolor{gray!10}{\hspace{1em}Median [Min, Max]} & \cellcolor{gray!10}{0 [0, 168]} & \cellcolor{gray!10}{0 [0, 168]} & \cellcolor{gray!10}{0 [0, 168]} & \cellcolor{gray!10}{0 [0, 168]} & \cellcolor{gray!10}{0 [0, 168]} & \cellcolor{gray!10}{0 [0, 168]}\\
\hspace{1em}Missing & 1442 (3.1\%) & 13105 (28.3\%) & 15860 (34.2\%) & 20348 (43.9\%) & 23539 (50.8\%) & 25903 \vphantom{1} (55.9\%)\\
\addlinespace[0.3em]
\multicolumn{7}{l}{\textbf{Hours Commute}}\\
\cellcolor{gray!10}{\hspace{1em}Mean (SD)} & \cellcolor{gray!10}{5.29 (6.40)} & \cellcolor{gray!10}{4.88 (7.10)} & \cellcolor{gray!10}{4.48 (5.91)} & \cellcolor{gray!10}{3.80 (5.49)} & \cellcolor{gray!10}{4.52 (6.17)} & \cellcolor{gray!10}{4.62 (6.36)}\\
\hspace{1em}Median [Min, Max] & 4.00 [0, 80.0] & 3.00 [0, 168] & 3.00 [0, 100] & 2.00 [0, 100] & 3.00 [0, 100] & 3.00 [0, 100]\\
\cellcolor{gray!10}{\hspace{1em}Missing} & \cellcolor{gray!10}{1442 (3.1\%)} & \cellcolor{gray!10}{13104 (28.3\%)} & \cellcolor{gray!10}{15860 (34.2\%)} & \cellcolor{gray!10}{20338 (43.9\%)} & \cellcolor{gray!10}{23533 (50.7\%)} & \cellcolor{gray!10}{25897 \vphantom{1} (55.8\%)}\\
\addlinespace[0.3em]
\multicolumn{7}{l}{\textbf{Hours Exercise}}\\
\hspace{1em}Mean (SD) & 5.78 (7.70) & 6.21 (8.21) & 6.17 (7.44) & 6.20 (7.04) & 6.19 (7.12) & 6.42 (7.29)\\
\cellcolor{gray!10}{\hspace{1em}Median [Min, Max]} & \cellcolor{gray!10}{4.00 [0, 80.0]} & \cellcolor{gray!10}{4.00 [0, 168]} & \cellcolor{gray!10}{5.00 [0, 80.0]} & \cellcolor{gray!10}{5.00 [0, 85.0]} & \cellcolor{gray!10}{5.00 [0, 80.0]} & \cellcolor{gray!10}{5.00 [0, 80.0]}\\
\hspace{1em}Missing & 1442 (3.1\%) & 13106 (28.3\%) & 15860 (34.2\%) & 20338 (43.9\%) & 23536 (50.7\%) & 25897 \vphantom{1} (55.8\%)\\
\addlinespace[0.3em]
\multicolumn{7}{l}{\textbf{Hours Housework}}\\
\cellcolor{gray!10}{\hspace{1em}Mean (SD)} & \cellcolor{gray!10}{10.3 (10.1)} & \cellcolor{gray!10}{10.3 (9.24)} & \cellcolor{gray!10}{10.8 (9.95)} & \cellcolor{gray!10}{10.8 (9.31)} & \cellcolor{gray!10}{10.9 (9.33)} & \cellcolor{gray!10}{11.0 (8.96)}\\
\hspace{1em}Median [Min, Max] & 8.00 [0, 168] & 8.00 [0, 168] & 10.0 [0, 168] & 10.0 [0, 168] & 10.0 [0, 168] & 10.0 [0, 168]\\
\cellcolor{gray!10}{\hspace{1em}Missing} & \cellcolor{gray!10}{1442 (3.1\%)} & \cellcolor{gray!10}{13104 (28.3\%)} & \cellcolor{gray!10}{15860 (34.2\%)} & \cellcolor{gray!10}{20338 (43.9\%)} & \cellcolor{gray!10}{23532 (50.7\%)} & \cellcolor{gray!10}{25896 \vphantom{1} (55.8\%)}\\
\addlinespace[0.3em]
\multicolumn{7}{l}{\textbf{Household Income}}\\
\hspace{1em}Mean (SD) & 115000 (92100) & 120000 (110000) & 123000 (107000) & 127000 (113000) & 134000 (148000) & 136000 (125000)\\
\cellcolor{gray!10}{\hspace{1em}Median [Min, Max]} & \cellcolor{gray!10}{100000 [1.00, 3010000]} & \cellcolor{gray!10}{100000 [1.00, 4000000]} & \cellcolor{gray!10}{100000 [1.00, 3000000]} & \cellcolor{gray!10}{100000 [1.00, 3500000]} & \cellcolor{gray!10}{100000 [1000, 7500000]} & \cellcolor{gray!10}{110000 [0, 5000000]}\\
\hspace{1em}Missing & 2940 (6.3\%) & 13806 (29.8\%) & 15874 (34.2\%) & 20174 (43.5\%) & 23113 (49.8\%) & 25824 \vphantom{1} (55.7\%)\\
\addlinespace[0.3em]
\multicolumn{7}{l}{\textbf{Log Hours Children}}\\
\cellcolor{gray!10}{\hspace{1em}Mean (SD)} & \cellcolor{gray!10}{1.16 (1.61)} & \cellcolor{gray!10}{1.08 (1.58)} & \cellcolor{gray!10}{1.02 (1.54)} & \cellcolor{gray!10}{0.952 (1.48)} & \cellcolor{gray!10}{0.934 (1.47)} & \cellcolor{gray!10}{0.936 (1.46)}\\
\hspace{1em}Median [Min, Max] & 0 [0, 5.13] & 0 [0, 5.13] & 0 [0, 5.13] & 0 [0, 5.13] & 0 [0, 5.13] & 0 [0, 5.13]\\
\cellcolor{gray!10}{\hspace{1em}Missing} & \cellcolor{gray!10}{1442 (3.1\%)} & \cellcolor{gray!10}{13105 (28.3\%)} & \cellcolor{gray!10}{15860 (34.2\%)} & \cellcolor{gray!10}{20348 (43.9\%)} & \cellcolor{gray!10}{23539 (50.8\%)} & \cellcolor{gray!10}{25903 (55.9\%)}\\
\addlinespace[0.3em]
\multicolumn{7}{l}{\textbf{Log Hours Commute}}\\
\hspace{1em}Mean (SD) & 1.50 (0.834) & 1.40 (0.860) & 1.34 (0.863) & 1.19 (0.853) & 1.34 (0.858) & 1.36 (0.851)\\
\cellcolor{gray!10}{\hspace{1em}Median [Min, Max]} & \cellcolor{gray!10}{1.61 [0, 4.39]} & \cellcolor{gray!10}{1.39 [0, 5.13]} & \cellcolor{gray!10}{1.39 [0, 4.62]} & \cellcolor{gray!10}{1.10 [0, 4.62]} & \cellcolor{gray!10}{1.39 [0, 4.62]} & \cellcolor{gray!10}{1.39 [0, 4.62]}\\
\hspace{1em}Missing & 1442 (3.1\%) & 13104 (28.3\%) & 15860 (34.2\%) & 20338 (43.9\%) & 23533 (50.7\%) & 25897 (55.8\%)\\
\addlinespace[0.3em]
\multicolumn{7}{l}{\textbf{Log Hours Exercise}}\\
\cellcolor{gray!10}{\hspace{1em}Mean (SD)} & \cellcolor{gray!10}{1.54 (0.848)} & \cellcolor{gray!10}{1.63 (0.826)} & \cellcolor{gray!10}{1.62 (0.839)} & \cellcolor{gray!10}{1.64 (0.833)} & \cellcolor{gray!10}{1.63 (0.838)} & \cellcolor{gray!10}{1.67 (0.830)}\\
\hspace{1em}Median [Min, Max] & 1.61 [0, 4.39] & 1.61 [0, 5.13] & 1.79 [0, 4.39] & 1.79 [0, 4.45] & 1.79 [0, 4.39] & 1.79 [0, 4.39]\\
\cellcolor{gray!10}{\hspace{1em}Missing} & \cellcolor{gray!10}{1442 (3.1\%)} & \cellcolor{gray!10}{13106 (28.3\%)} & \cellcolor{gray!10}{15860 (34.2\%)} & \cellcolor{gray!10}{20338 (43.9\%)} & \cellcolor{gray!10}{23536 (50.7\%)} & \cellcolor{gray!10}{25897 (55.8\%)}\\
\addlinespace[0.3em]
\multicolumn{7}{l}{\textbf{Log Hours Housework}}\\
\hspace{1em}Mean (SD) & 2.14 (0.781) & 2.16 (0.756) & 2.21 (0.754) & 2.22 (0.735) & 2.23 (0.751) & 2.24 (0.739)\\
\cellcolor{gray!10}{\hspace{1em}Median [Min, Max]} & \cellcolor{gray!10}{2.20 [0, 5.13]} & \cellcolor{gray!10}{2.20 [0, 5.13]} & \cellcolor{gray!10}{2.40 [0, 5.13]} & \cellcolor{gray!10}{2.40 [0, 5.13]} & \cellcolor{gray!10}{2.40 [0, 5.13]} & \cellcolor{gray!10}{2.40 [0, 5.13]}\\
\hspace{1em}Missing & 1442 (3.1\%) & 13104 (28.3\%) & 15860 (34.2\%) & 20338 (43.9\%) & 23532 (50.7\%) & 25896 (55.8\%)\\
\addlinespace[0.3em]
\multicolumn{7}{l}{\textbf{Log Household Income}}\\
\cellcolor{gray!10}{\hspace{1em}Mean (SD)} & \cellcolor{gray!10}{11.4 (0.768)} & \cellcolor{gray!10}{11.4 (0.851)} & \cellcolor{gray!10}{11.4 (0.807)} & \cellcolor{gray!10}{11.5 (0.826)} & \cellcolor{gray!10}{11.5 (0.789)} & \cellcolor{gray!10}{11.5 (0.975)}\\
\hspace{1em}Median [Min, Max] & 11.5 [0.693, 14.9] & 11.5 [0.693, 15.2] & 11.5 [0.693, 14.9] & 11.5 [0.693, 15.1] & 11.5 [6.91, 15.8] & 11.6 [0, 15.4]\\
\cellcolor{gray!10}{\hspace{1em}Missing} & \cellcolor{gray!10}{2940 (6.3\%)} & \cellcolor{gray!10}{13806 (29.8\%)} & \cellcolor{gray!10}{15874 (34.2\%)} & \cellcolor{gray!10}{20174 (43.5\%)} & \cellcolor{gray!10}{23113 (49.8\%)} & \cellcolor{gray!10}{25824 (55.7\%)}\\
\addlinespace[0.3em]
\multicolumn{7}{l}{\textbf{Male (binary)}}\\
\hspace{1em}Mean (SD) & 0.371 (0.483) & 0.363 (0.481) & 0.363 (0.481) & 0.359 (0.480) & 0.365 (0.482) & 0.364 (0.481)\\
\cellcolor{gray!10}{\hspace{1em}Median [Min, Max]} & \cellcolor{gray!10}{0 [0, 1.00]} & \cellcolor{gray!10}{0 [0, 1.00]} & \cellcolor{gray!10}{0 [0, 1.00]} & \cellcolor{gray!10}{0 [0, 1.00]} & \cellcolor{gray!10}{0 [0, 1.00]} & \cellcolor{gray!10}{0 [0, \vphantom{3} 1.00]}\\
\hspace{1em}Missing & 108 (0.2\%) & 12532 (27.0\%) & 15178 (32.7\%) & 19585 (42.2\%) & 22778 (49.1\%) & 25122 (54.2\%)\\
\addlinespace[0.3em]
\multicolumn{7}{l}{\textbf{Neuroticism}}\\
\cellcolor{gray!10}{\hspace{1em}Mean (SD)} & \cellcolor{gray!10}{3.49 (1.15)} & \cellcolor{gray!10}{3.48 (1.16)} & \cellcolor{gray!10}{3.45 (1.15)} & \cellcolor{gray!10}{3.41 (1.18)} & \cellcolor{gray!10}{3.36 (1.16)} & \cellcolor{gray!10}{3.33 (1.17)}\\
\hspace{1em}Median [Min, Max] & 3.50 [1.00, 7.00] & 3.50 [1.00, 7.00] & 3.50 [1.00, 7.00] & 3.25 [1.00, 7.00] & 3.25 [1.00, 7.00] & 3.25 [1.00, 7.00]\\
\cellcolor{gray!10}{\hspace{1em}Missing} & \cellcolor{gray!10}{402 (0.9\%)} & \cellcolor{gray!10}{12704 (27.4\%)} & \cellcolor{gray!10}{15267 (32.9\%)} & \cellcolor{gray!10}{19651 (42.4\%)} & \cellcolor{gray!10}{22778 (49.1\%)} & \cellcolor{gray!10}{25057 (54.0\%)}\\
\addlinespace[0.3em]
\multicolumn{7}{l}{\textbf{Not Heterosexual Binary}}\\
\hspace{1em}Mean (SD) & 0.0672 (0.250) & 0.0717 (0.258) & 0.0765 (0.266) & 0.0789 (0.270) & 0.0775 (0.267) & 0.0801 (0.271)\\
\cellcolor{gray!10}{\hspace{1em}Median [Min, Max]} & \cellcolor{gray!10}{0 [0, 1.00]} & \cellcolor{gray!10}{0 [0, 1.00]} & \cellcolor{gray!10}{0 [0, 1.00]} & \cellcolor{gray!10}{0 [0, 1.00]} & \cellcolor{gray!10}{0 [0, 1.00]} & \cellcolor{gray!10}{0 [0, \vphantom{2} 1.00]}\\
\hspace{1em}Missing & 1175 (2.5\%) & 12747 (27.5\%) & 15417 (33.2\%) & 19585 (42.2\%) & 22903 (49.4\%) & 25172 (54.3\%)\\
\addlinespace[0.3em]
\multicolumn{7}{l}{\textbf{NZ Deprevation Index 2018}}\\
\cellcolor{gray!10}{\hspace{1em}Mean (SD)} & \cellcolor{gray!10}{4.77 (2.73)} & \cellcolor{gray!10}{4.77 (2.74)} & \cellcolor{gray!10}{4.77 (2.75)} & \cellcolor{gray!10}{4.77 (2.75)} & \cellcolor{gray!10}{4.77 (2.75)} & \cellcolor{gray!10}{4.76 (2.76)}\\
\hspace{1em}Median [Min, Max] & 4.00 [1.00, 10.0] & 4.00 [1.00, 10.0] & 4.00 [1.00, 10.0] & 4.00 [1.00, 10.0] & 4.00 [1.00, 10.0] & 4.00 [1.00, 10.0]\\
\cellcolor{gray!10}{\hspace{1em}Missing} & \cellcolor{gray!10}{308 (0.7\%)} & \cellcolor{gray!10}{469 (1.0\%)} & \cellcolor{gray!10}{595 (1.3\%)} & \cellcolor{gray!10}{940 (2.0\%)} & \cellcolor{gray!10}{881 (1.9\%)} & \cellcolor{gray!10}{966 (2.1\%)}\\
\addlinespace[0.3em]
\multicolumn{7}{l}{\textbf{NZSEI (Occupational Prestige Index)}}\\
\hspace{1em}Mean (SD) & 54.1 (16.5) & 55.1 (16.4) & 55.2 (16.7) & 55.7 (16.6) & 56.1 (15.9) & 56.0 (16.2)\\
\cellcolor{gray!10}{\hspace{1em}Median [Min, Max]} & \cellcolor{gray!10}{54.0 [10.0, 90.0]} & \cellcolor{gray!10}{57.0 [10.0, 90.0]} & \cellcolor{gray!10}{57.0 [10.0, 90.0]} & \cellcolor{gray!10}{60.0 [10.0, 90.0]} & \cellcolor{gray!10}{60.0 [10.0, 90.0]} & \cellcolor{gray!10}{60.0 [10.0, 90.0]}\\
\hspace{1em}Missing & 346 (0.7\%) & 4220 (9.1\%) & 5184 (11.2\%) & 6236 (13.4\%) & 7752 (16.7\%) & 9063 (19.5\%)\\
\addlinespace[0.3em]
\multicolumn{7}{l}{\textbf{Openness}}\\
\cellcolor{gray!10}{\hspace{1em}Mean (SD)} & \cellcolor{gray!10}{4.96 (1.12)} & \cellcolor{gray!10}{4.96 (1.12)} & \cellcolor{gray!10}{4.96 (1.11)} & \cellcolor{gray!10}{4.96 (1.14)} & \cellcolor{gray!10}{4.95 (1.14)} & \cellcolor{gray!10}{4.96 (1.16)}\\
\hspace{1em}Median [Min, Max] & 5.00 [1.00, 7.00] & 5.00 [1.00, 7.00] & 5.00 [1.00, 7.00] & 5.00 [1.00, 7.00] & 5.00 [1.00, 7.00] & 5.00 [1.00, 7.00]\\
\cellcolor{gray!10}{\hspace{1em}Missing} & \cellcolor{gray!10}{394 (0.8\%)} & \cellcolor{gray!10}{12704 (27.4\%)} & \cellcolor{gray!10}{15266 (32.9\%)} & \cellcolor{gray!10}{19651 (42.4\%)} & \cellcolor{gray!10}{22774 (49.1\%)} & \cellcolor{gray!10}{25072 (54.1\%)}\\
\addlinespace[0.3em]
\multicolumn{7}{l}{\textbf{Parent (binary)}}\\
\hspace{1em}Mean (SD) & 0.708 (0.455) & 0.736 (0.441) & 0.748 (0.434) & 0.747 (0.434) & 0.767 (0.423) & 0.767 (0.423)\\
\cellcolor{gray!10}{\hspace{1em}Median [Min, Max]} & \cellcolor{gray!10}{1.00 [0, 1.00]} & \cellcolor{gray!10}{1.00 [0, 1.00]} & \cellcolor{gray!10}{1.00 [0, 1.00]} & \cellcolor{gray!10}{1.00 [0, 1.00]} & \cellcolor{gray!10}{1.00 [0, 1.00]} & \cellcolor{gray!10}{1.00 [0, \vphantom{1} 1.00]}\\
\hspace{1em}Missing & 0 (0\%) & 12422 (26.8\%) & 15053 (32.5\%) & 19494 (42.0\%) & 22677 (48.9\%) & 25006 (53.9\%)\\
\addlinespace[0.3em]
\multicolumn{7}{l}{\textbf{Partner Binary}}\\
\cellcolor{gray!10}{\hspace{1em}Mean (SD)} & \cellcolor{gray!10}{0.750 (0.433)} & \cellcolor{gray!10}{0.761 (0.426)} & \cellcolor{gray!10}{0.761 (0.426)} & \cellcolor{gray!10}{0.761 (0.427)} & \cellcolor{gray!10}{0.755 (0.430)} & \cellcolor{gray!10}{0.750 (0.433)}\\
\hspace{1em}Median [Min, Max] & 1.00 [0, 1.00] & 1.00 [0, 1.00] & 1.00 [0, 1.00] & 1.00 [0, 1.00] & 1.00 [0, 1.00] & 1.00 [0, 1.00]\\
\cellcolor{gray!10}{\hspace{1em}Missing} & \cellcolor{gray!10}{535 (1.2\%)} & \cellcolor{gray!10}{13004 (28.0\%)} & \cellcolor{gray!10}{15471 (33.4\%)} & \cellcolor{gray!10}{20013 (43.2\%)} & \cellcolor{gray!10}{23312 (50.3\%)} & \cellcolor{gray!10}{25651 (55.3\%)}\\
\addlinespace[0.3em]
\multicolumn{7}{l}{\textbf{Political Conservative}}\\
\hspace{1em}Mean (SD) & 3.59 (1.38) & 3.58 (1.39) & 3.47 (1.35) & 3.53 (1.34) & 3.61 (1.38) & 3.57 (1.40)\\
\cellcolor{gray!10}{\hspace{1em}Median [Min, Max]} & \cellcolor{gray!10}{4.00 [1.00, 7.00]} & \cellcolor{gray!10}{4.00 [1.00, 7.00]} & \cellcolor{gray!10}{4.00 [1.00, 7.00]} & \cellcolor{gray!10}{4.00 [1.00, 7.00]} & \cellcolor{gray!10}{4.00 [1.00, 7.00]} & \cellcolor{gray!10}{4.00 [1.00, 7.00]}\\
\hspace{1em}Missing & 1902 (4.1\%) & 13491 (29.1\%) & 16285 (35.1\%) & 20637 (44.5\%) & 23810 (51.3\%) & 26087 (56.2\%)\\
\addlinespace[0.3em]
\multicolumn{7}{l}{\textbf{Power No Control Composite}}\\
\cellcolor{gray!10}{\hspace{1em}Mean (SD)} & \cellcolor{gray!10}{2.97 (1.41)} & \cellcolor{gray!10}{2.96 (1.43)} & \cellcolor{gray!10}{2.78 (1.40)} & \cellcolor{gray!10}{2.85 (1.43)} & \cellcolor{gray!10}{NA (NA)} & \cellcolor{gray!10}{NA (NA)}\\
\hspace{1em}Median [Min, Max] & 3.00 [1.00, 7.00] & 3.00 [1.00, 7.00] & 2.50 [1.00, 7.00] & 2.50 [1.00, 7.00] & NA [NA, NA] & NA [NA, NA]\\
\cellcolor{gray!10}{\hspace{1em}Missing} & \cellcolor{gray!10}{220 (0.5\%)} & \cellcolor{gray!10}{12632 (27.2\%)} & \cellcolor{gray!10}{15131 (32.6\%)} & \cellcolor{gray!10}{19999 (43.1\%)} & \cellcolor{gray!10}{46377 (100\%)} & \cellcolor{gray!10}{46377 (100\%)}\\
\addlinespace[0.3em]
\multicolumn{7}{l}{\textbf{Rural Gch 2018 Levels}}\\
\hspace{1em}High Urban Accessibility & 28607 (61.7\%) & 28312 (61.0\%) & 28008 (60.4\%) & 27584 (59.5\%) & 27501 (59.3\%) & 27306 (58.9\%)\\
\cellcolor{gray!10}{\hspace{1em}Medium Urban Accessibility} & \cellcolor{gray!10}{8695 (18.7\%)} & \cellcolor{gray!10}{8731 (18.8\%)} & \cellcolor{gray!10}{8788 (18.9\%)} & \cellcolor{gray!10}{8776 (18.9\%)} & \cellcolor{gray!10}{8812 (19.0\%)} & \cellcolor{gray!10}{8858 (19.1\%)}\\
\hspace{1em}Low Urban Accessibility & 5639 (12.2\%) & 5711 (12.3\%) & 5811 (12.5\%) & 5861 (12.6\%) & 5896 (12.7\%) & 5915 (12.8\%)\\
\cellcolor{gray!10}{\hspace{1em}Remote} & \cellcolor{gray!10}{2581 (5.6\%)} & \cellcolor{gray!10}{2608 (5.6\%)} & \cellcolor{gray!10}{2658 (5.7\%)} & \cellcolor{gray!10}{2680 (5.8\%)} & \cellcolor{gray!10}{2738 (5.9\%)} & \cellcolor{gray!10}{2777 (6.0\%)}\\
\hspace{1em}Very Remote & 549 (1.2\%) & 548 (1.2\%) & 559 (1.2\%) & 540 (1.2\%) & 552 (1.2\%) & 558 (1.2\%)\\
\cellcolor{gray!10}{\hspace{1em}Missing} & \cellcolor{gray!10}{306 (0.7\%)} & \cellcolor{gray!10}{467 (1.0\%)} & \cellcolor{gray!10}{553 (1.2\%)} & \cellcolor{gray!10}{936 (2.0\%)} & \cellcolor{gray!10}{878 (1.9\%)} & \cellcolor{gray!10}{963 (2.1\%)}\\
\addlinespace[0.3em]
\multicolumn{7}{l}{\textbf{Right Wing Authoritarianism}}\\
\hspace{1em}Mean (SD) & 3.28 (1.15) & 3.20 (1.14) & 3.30 (1.10) & 3.36 (1.05) & 3.35 (1.07) & 3.27 (1.10)\\
\cellcolor{gray!10}{\hspace{1em}Median [Min, Max]} & \cellcolor{gray!10}{3.20 [1.00, 7.00]} & \cellcolor{gray!10}{3.17 [1.00, 7.00]} & \cellcolor{gray!10}{3.17 [1.00, 7.00]} & \cellcolor{gray!10}{3.33 [1.00, 7.00]} & \cellcolor{gray!10}{3.33 [1.00, 7.00]} & \cellcolor{gray!10}{3.20 [1.00, 7.00]}\\
\hspace{1em}Missing & 7 (0.0\%) & 12459 (26.9\%) & 15106 (32.6\%) & 19609 (42.3\%) & 22730 (49.0\%) & 25084 (54.1\%)\\
\addlinespace[0.3em]
\multicolumn{7}{l}{\textbf{Sample Frame Opt-In (binary)}}\\
\cellcolor{gray!10}{\hspace{1em}Mean (SD)} & \cellcolor{gray!10}{0.0297 (0.170)} & \cellcolor{gray!10}{0.0297 (0.170)} & \cellcolor{gray!10}{0.0297 (0.170)} & \cellcolor{gray!10}{0.0297 (0.170)} & \cellcolor{gray!10}{0.0297 (0.170)} & \cellcolor{gray!10}{0.0297 (0.170)}\\
\hspace{1em}Median [Min, Max] & 0 [0, 1.00] & 0 [0, 1.00] & 0 [0, 1.00] & 0 [0, 1.00] & 0 [0, 1.00] & 0 [0, \vphantom{1} 1.00]\\
\addlinespace[0.3em]
\multicolumn{7}{l}{\textbf{Social Dominance Orientation}}\\
\cellcolor{gray!10}{\hspace{1em}Mean (SD)} & \cellcolor{gray!10}{2.32 (0.961)} & \cellcolor{gray!10}{2.25 (0.950)} & \cellcolor{gray!10}{2.21 (0.945)} & \cellcolor{gray!10}{2.22 (0.944)} & \cellcolor{gray!10}{2.25 (0.957)} & \cellcolor{gray!10}{2.25 (0.961)}\\
\hspace{1em}Median [Min, Max] & 2.17 [1.00, 7.00] & 2.17 [1.00, 7.00] & 2.00 [1.00, 7.00] & 2.17 [1.00, 7.00] & 2.17 [1.00, 7.00] & 2.17 [1.00, 7.00]\\
\cellcolor{gray!10}{\hspace{1em}Missing} & \cellcolor{gray!10}{1 (0.0\%)} & \cellcolor{gray!10}{12435 (26.8\%)} & \cellcolor{gray!10}{15073 (32.5\%)} & \cellcolor{gray!10}{19483 (42.0\%)} & \cellcolor{gray!10}{22682 (48.9\%)} & \cellcolor{gray!10}{25025 (54.0\%)}\\
\addlinespace[0.3em]
\multicolumn{7}{l}{\textbf{Smoker (binary)}}\\
\hspace{1em}Mean (SD) & 0.0730 (0.260) & 0.0572 (0.232) & 0.0490 (0.216) & 0.0417 (0.200) & 0.0370 (0.189) & 0.0339 (0.181)\\
\cellcolor{gray!10}{\hspace{1em}Median [Min, Max]} & \cellcolor{gray!10}{0 [0, 1.00]} & \cellcolor{gray!10}{0 [0, 1.00]} & \cellcolor{gray!10}{0 [0, 1.00]} & \cellcolor{gray!10}{0 [0, 1.00]} & \cellcolor{gray!10}{0 [0, 1.00]} & \cellcolor{gray!10}{0 [0, 1.00]}\\
\hspace{1em}Missing & 1185 (2.6\%) & 12736 (27.5\%) & 15526 (33.5\%) & 19637 (42.3\%) & 22678 (48.9\%) & 25552 (55.1\%)\\*

\end{longtable}

\endgroup{}
\endgroup{}

\subsubsection{Exposure Variable: Religious Service
Attendance}\label{appendix-exposure}

Table~\ref{tbl-sample-exposures} presents sample statistics for the
exposure variable, religious service attendance, during the baseline and
exposure waves. This variable was not measured in part of NZAVS time 12
(years 2020-2021) and part of NZAVS time 13 (years 2021-2022). To
address missingness, if a value was observed after NZAVS time 14, we
carried the previous observation forward and created and NA indicator.
If there was no future observation, the participant was treated as
censored, and inverse probability of censoring weights were applied,
following our standard method for handling missing observations (see
mansucript \textbf{Method}/\textbf{Handling of Missing Data}). Here, our
carry-forward imputation approach may result in conservative causal
effect estimation because it introduces measurement error. However, this
approach would not generally bias causal effect estimation away from the
null because the measurement error is unsystematic and random and
unrelated to the outcomes.

\begingroup\fontsize{12}{14}\selectfont
\begingroup\fontsize{8}{10}\selectfont

\begin{longtable}[t]{llllll}

\caption{\label{tbl-sample-exposures}Demographic statistics for New
Zealand Attitudes and Values Cohort waves 2018.}

\tabularnewline

\toprule
  & 2018 & 2019 & 2020 & 2021 & 2022\\
\midrule
\endfirsthead
\multicolumn{6}{@{}l}{\textit{(continued)}}\\
\toprule
  & 2018 & 2019 & 2020 & 2021 & 2022\\
\midrule
\endhead

\endfoot
\bottomrule
\endlastfoot
\cellcolor{gray!10}{} & \cellcolor{gray!10}{(N=46377)} & \cellcolor{gray!10}{(N=46377)} & \cellcolor{gray!10}{(N=46377)} & \cellcolor{gray!10}{(N=46377)} & \cellcolor{gray!10}{(N=46377)}\\
\addlinespace[0.3em]
\multicolumn{6}{l}{\textbf{Monthly Religious Service}}\\
\hspace{1em}0 & 38479 (83.0\%) & 28536 (61.5\%) & 0 (0\%) & 18913 (40.8\%) & 19781 (42.7\%)\\
\cellcolor{gray!10}{\hspace{1em}1} & \cellcolor{gray!10}{1517 (3.3\%)} & \cellcolor{gray!10}{896 (1.9\%)} & \cellcolor{gray!10}{0 (0\%)} & \cellcolor{gray!10}{87 (0.2\%)} & \cellcolor{gray!10}{628 (1.4\%)}\\
\hspace{1em}2 & 1125 (2.4\%) & 737 (1.6\%) & 0 (0\%) & 72 (0.2\%) & 523 (1.1\%)\\
\cellcolor{gray!10}{\hspace{1em}3} & \cellcolor{gray!10}{907 (2.0\%)} & \cellcolor{gray!10}{639 (1.4\%)} & \cellcolor{gray!10}{0 (0\%)} & \cellcolor{gray!10}{58 (0.1\%)} & \cellcolor{gray!10}{455 (1.0\%)}\\
\hspace{1em}4 & 2478 (5.3\%) & 1672 (3.6\%) & 0 (0\%) & 150 (0.3\%) & 1079 (2.3\%)\\
\cellcolor{gray!10}{\hspace{1em}5} & \cellcolor{gray!10}{475 (1.0\%)} & \cellcolor{gray!10}{308 (0.7\%)} & \cellcolor{gray!10}{0 (0\%)} & \cellcolor{gray!10}{24 (0.1\%)} & \cellcolor{gray!10}{171 (0.4\%)}\\
\hspace{1em}6 & 326 (0.7\%) & 205 (0.4\%) & 0 (0\%) & 23 (0.0\%) & 116 (0.3\%)\\
\cellcolor{gray!10}{\hspace{1em}7} & \cellcolor{gray!10}{106 (0.2\%)} & \cellcolor{gray!10}{68 (0.1\%)} & \cellcolor{gray!10}{0 (0\%)} & \cellcolor{gray!10}{6 (0.0\%)} & \cellcolor{gray!10}{31 (0.1\%)}\\
\hspace{1em}8 & 964 (2.1\%) & 645 (1.4\%) & 0 (0\%) & 68 (0.1\%) & 353 (0.8\%)\\
\cellcolor{gray!10}{\hspace{1em}Missing} & \cellcolor{gray!10}{0 (0\%)} & \cellcolor{gray!10}{12671 (27.3\%)} & \cellcolor{gray!10}{46377 (100\%)} & \cellcolor{gray!10}{26976 (58.2\%)} & \cellcolor{gray!10}{23240 (50.1\%)}\\*

\end{longtable}

\endgroup{}
\endgroup{}

\newpage{}

\subsubsection{Outcome Variables}\label{appendix-outcomes}

\subsubsection{Health Outcome Variables}\label{health-outcome-variables}

\begingroup\fontsize{12}{14}\selectfont
\begingroup\fontsize{8}{10}\selectfont

\begin{longtable}[t]{lll}

\caption{\label{tbl-sample-outcomes-health}Health variables measured at
baseline (NZAVS time 10, years 2018-2019, and time 15, years
2023-2024).}

\tabularnewline

\toprule
  & 2018 & 2023\\
\midrule
\endfirsthead
\multicolumn{3}{@{}l}{\textit{(continued)}}\\
\toprule
  & 2018 & 2023\\
\midrule
\endhead

\endfoot
\bottomrule
\endlastfoot
\cellcolor{gray!10}{} & \cellcolor{gray!10}{(N=46377)} & \cellcolor{gray!10}{(N=46377)}\\
\addlinespace[0.3em]
\multicolumn{3}{l}{\textbf{Body Mass Index}}\\
\hspace{1em}Mean (SD) & 27.2 (5.87) & 27.8 (6.07)\\
\cellcolor{gray!10}{\hspace{1em}Median [Min, Max]} & \cellcolor{gray!10}{26.2 [12.3, 73.6]} & \cellcolor{gray!10}{26.7 [13.2, 87.4]}\\
\hspace{1em}Missing & 1171 (2.5\%) & 25123 (54.2\%)\\
\addlinespace[0.3em]
\multicolumn{3}{l}{\textbf{Weekly Hours Sleep}}\\
\cellcolor{gray!10}{\hspace{1em}Mean (SD)} & \cellcolor{gray!10}{6.94 (1.13)} & \cellcolor{gray!10}{6.93 (1.12)}\\
\hspace{1em}Median [Min, Max] & 7.00 [2.50, 16.0] & 7.00 [2.00, 16.0]\\
\cellcolor{gray!10}{\hspace{1em}Missing} & \cellcolor{gray!10}{2365 (5.1\%)} & \cellcolor{gray!10}{26025 (56.1\%)}\\
\addlinespace[0.3em]
\multicolumn{3}{l}{\textbf{Weekly Hours Excercise (log)}}\\
\hspace{1em}Mean (SD) & 1.54 (0.848) & 1.67 (0.830)\\
\cellcolor{gray!10}{\hspace{1em}Median [Min, Max]} & \cellcolor{gray!10}{1.61 [0, 4.39]} & \cellcolor{gray!10}{1.79 [0, 4.39]}\\
\hspace{1em}Missing & 1442 (3.1\%) & 25897 (55.8\%)\\
\addlinespace[0.3em]
\multicolumn{3}{l}{\textbf{Short Form Health}}\\
\cellcolor{gray!10}{\hspace{1em}Mean (SD)} & \cellcolor{gray!10}{5.04 (1.17)} & \cellcolor{gray!10}{4.84 (1.17)}\\
\hspace{1em}Median [Min, Max] & 5.00 [1.00, 7.00] & 5.00 [1.00, 7.00]\\
\cellcolor{gray!10}{\hspace{1em}Missing} & \cellcolor{gray!10}{9 (0.0\%)} & \cellcolor{gray!10}{25080 (54.1\%)}\\*

\end{longtable}

\endgroup{}
\endgroup{}

\subsubsection{Psychological Well-Being Outcome
Variables}\label{psychological-well-being-outcome-variables}

\begingroup\fontsize{12}{14}\selectfont
\begingroup\fontsize{8}{10}\selectfont

\begin{longtable}[t]{lll}

\caption{\label{tbl-sample-outcomes-psych}Psychological well-being
variables measured at baseline (NZAVS time 10, years 2018-2019, and time
15, years 2023-2024).}

\tabularnewline

\toprule
  & 2018 & 2023\\
\midrule
\endfirsthead
\multicolumn{3}{@{}l}{\textit{(continued)}}\\
\toprule
  & 2018 & 2023\\
\midrule
\endhead

\endfoot
\bottomrule
\endlastfoot
\cellcolor{gray!10}{} & \cellcolor{gray!10}{(N=46377)} & \cellcolor{gray!10}{(N=46377)}\\
\addlinespace[0.3em]
\multicolumn{3}{l}{\textbf{Fatigue}}\\
\hspace{1em}Mean (SD) & 1.64 (1.09) & 1.62 (1.07)\\
\cellcolor{gray!10}{\hspace{1em}Median [Min, Max]} & \cellcolor{gray!10}{2.00 [0, 4.00]} & \cellcolor{gray!10}{2.00 [0, 4.00]}\\
\hspace{1em}Missing & 500 (1.1\%) & 25117 (54.2\%)\\
\addlinespace[0.3em]
\multicolumn{3}{l}{\textbf{Kessler 6 Anxiety}}\\
\cellcolor{gray!10}{\hspace{1em}Mean (SD)} & \cellcolor{gray!10}{1.21 (0.773)} & \cellcolor{gray!10}{1.16 (0.764)}\\
\hspace{1em}Median [Min, Max] & 1.00 [0, 4.00] & 1.00 [0, 4.00]\\
\cellcolor{gray!10}{\hspace{1em}Missing} & \cellcolor{gray!10}{437 (0.9\%)} & \cellcolor{gray!10}{25048 (54.0\%)}\\
\addlinespace[0.3em]
\multicolumn{3}{l}{\textbf{Kessler 6 Depression}}\\
\hspace{1em}Mean (SD) & 0.585 (0.753) & 0.526 (0.722)\\
\cellcolor{gray!10}{\hspace{1em}Median [Min, Max]} & \cellcolor{gray!10}{0.333 [0, 4.00]} & \cellcolor{gray!10}{0.333 [0, 4.00]}\\
\hspace{1em}Missing & 440 (0.9\%) & 25048 (54.0\%)\\
\addlinespace[0.3em]
\multicolumn{3}{l}{\textbf{Rumination}}\\
\cellcolor{gray!10}{\hspace{1em}Mean (SD)} & \cellcolor{gray!10}{0.847 (1.01)} & \cellcolor{gray!10}{0.759 (0.956)}\\
\hspace{1em}Median [Min, Max] & 1.00 [0, 4.00] & 0 [0, 4.00]\\
\cellcolor{gray!10}{\hspace{1em}Missing} & \cellcolor{gray!10}{538 (1.2\%)} & \cellcolor{gray!10}{25108 (54.1\%)}\\*

\end{longtable}

\endgroup{}
\endgroup{}

\subsubsection{Present-Focussed Well-Being
Indicators}\label{present-focussed-well-being-indicators}

\begingroup\fontsize{12}{14}\selectfont
\begingroup\fontsize{8}{10}\selectfont

\begin{longtable}[t]{lll}

\caption{\label{tbl-sample-outcomes-present}Present-focussed well-being
variables measured at baseline (NZAVS time 10, years 2018-2019, and time
15, years 2023-2024).}

\tabularnewline

\toprule
  & 2018 & 2023\\
\midrule
\endfirsthead
\multicolumn{3}{@{}l}{\textit{(continued)}}\\
\toprule
  & 2018 & 2023\\
\midrule
\endhead

\endfoot
\bottomrule
\endlastfoot
\cellcolor{gray!10}{} & \cellcolor{gray!10}{(N=46377)} & \cellcolor{gray!10}{(N=46377)}\\
\addlinespace[0.3em]
\multicolumn{3}{l}{\textbf{Body Satisifaction}}\\
\hspace{1em}Mean (SD) & 4.24 (1.70) & 4.23 (1.69)\\
\cellcolor{gray!10}{\hspace{1em}Median [Min, Max]} & \cellcolor{gray!10}{4.00 [1.00, 7.00]} & \cellcolor{gray!10}{4.00 [1.00, 7.00]}\\
\hspace{1em}Missing & 537 (1.2\%) & 25931 (55.9\%)\\
\addlinespace[0.3em]
\multicolumn{3}{l}{\textbf{Forgiveness}}\\
\cellcolor{gray!10}{\hspace{1em}Mean (SD)} & \cellcolor{gray!10}{5.02 (1.27)} & \cellcolor{gray!10}{5.20 (1.26)}\\
\hspace{1em}Median [Min, Max] & 5.33 [1.00, 7.00] & 5.33 [1.00, 7.00]\\
\cellcolor{gray!10}{\hspace{1em}Missing} & \cellcolor{gray!10}{11 (0.0\%)} & \cellcolor{gray!10}{25125 (54.2\%)}\\
\addlinespace[0.3em]
\multicolumn{3}{l}{\textbf{Perfectionism}}\\
\hspace{1em}Mean (SD) & 3.18 (1.34) & 2.98 (1.40)\\
\cellcolor{gray!10}{\hspace{1em}Median [Min, Max]} & \cellcolor{gray!10}{3.00 [1.00, 7.00]} & \cellcolor{gray!10}{2.67 [1.00, 7.00]}\\
\hspace{1em}Missing & 6 (0.0\%) & 25174 (54.3\%)\\
\addlinespace[0.3em]
\multicolumn{3}{l}{\textbf{Pwb Standard Living}}\\
\cellcolor{gray!10}{\hspace{1em}Mean (SD)} & \cellcolor{gray!10}{7.58 (2.04)} & \cellcolor{gray!10}{7.54 (2.09)}\\
\hspace{1em}Median [Min, Max] & 8.00 [0, 10.0] & 8.00 [0, \vphantom{1} 10.0]\\
\cellcolor{gray!10}{\hspace{1em}Missing} & \cellcolor{gray!10}{174 (0.4\%)} & \cellcolor{gray!10}{25125 (54.2\%)}\\
\addlinespace[0.3em]
\multicolumn{3}{l}{\textbf{Pwb Your Future Security}}\\
\hspace{1em}Mean (SD) & 6.26 (2.35) & 6.13 (2.49)\\
\cellcolor{gray!10}{\hspace{1em}Median [Min, Max]} & \cellcolor{gray!10}{7.00 [0, 10.0]} & \cellcolor{gray!10}{7.00 [0, \vphantom{1} 10.0]}\\
\hspace{1em}Missing & 168 (0.4\%) & 25151 (54.2\%)\\
\addlinespace[0.3em]
\multicolumn{3}{l}{\textbf{Pwb Your Health}}\\
\cellcolor{gray!10}{\hspace{1em}Mean (SD)} & \cellcolor{gray!10}{6.76 (2.31)} & \cellcolor{gray!10}{6.54 (2.42)}\\
\hspace{1em}Median [Min, Max] & 7.00 [0, 10.0] & 7.00 [0, 10.0]\\
\cellcolor{gray!10}{\hspace{1em}Missing} & \cellcolor{gray!10}{188 (0.4\%)} & \cellcolor{gray!10}{25221 (54.4\%)}\\
\addlinespace[0.3em]
\multicolumn{3}{l}{\textbf{Pwb Your Relationships}}\\
\hspace{1em}Mean (SD) & 7.74 (2.24) & 7.66 (2.22)\\
\cellcolor{gray!10}{\hspace{1em}Median [Min, Max]} & \cellcolor{gray!10}{8.00 [0, 10.0]} & \cellcolor{gray!10}{8.00 [0, 10.0]}\\
\hspace{1em}Missing & 193 (0.4\%) & 25164 (54.3\%)\\
\addlinespace[0.3em]
\multicolumn{3}{l}{\textbf{Self Control Have Lots}}\\
\cellcolor{gray!10}{\hspace{1em}Mean (SD)} & \cellcolor{gray!10}{5.12 (1.38)} & \cellcolor{gray!10}{5.10 (1.41)}\\
\hspace{1em}Median [Min, Max] & 5.00 [1.00, 7.00] & 5.00 [1.00, 7.00]\\
\cellcolor{gray!10}{\hspace{1em}Missing} & \cellcolor{gray!10}{1261 (2.7\%)} & \cellcolor{gray!10}{26032 (56.1\%)}\\
\addlinespace[0.3em]
\multicolumn{3}{l}{\textbf{Self Control: Wish More (r)}}\\
\hspace{1em}Mean (SD) & 3.71 (1.83) & 3.87 (1.81)\\
\cellcolor{gray!10}{\hspace{1em}Median [Min, Max]} & \cellcolor{gray!10}{3.00 [1.00, 7.00]} & \cellcolor{gray!10}{4.00 [1.00, 7.00]}\\
\hspace{1em}Missing & 197 (0.4\%) & 25974 (56.0\%)\\
\addlinespace[0.3em]
\multicolumn{3}{l}{\textbf{Self Esteem}}\\
\cellcolor{gray!10}{\hspace{1em}Mean (SD)} & \cellcolor{gray!10}{5.14 (1.28)} & \cellcolor{gray!10}{5.27 (1.30)}\\
\hspace{1em}Median [Min, Max] & 5.33 [1.00, 7.00] & 5.67 [1.00, 7.00]\\
\cellcolor{gray!10}{\hspace{1em}Missing} & \cellcolor{gray!10}{400 (0.9\%)} & \cellcolor{gray!10}{25104 (54.1\%)}\\
\addlinespace[0.3em]
\multicolumn{3}{l}{\textbf{Sexual Satisfaction}}\\
\hspace{1em}Mean (SD) & 4.58 (1.77) & 4.38 (1.76)\\
\cellcolor{gray!10}{\hspace{1em}Median [Min, Max]} & \cellcolor{gray!10}{5.00 [1.00, 7.00]} & \cellcolor{gray!10}{4.00 [1.00, 7.00]}\\
\hspace{1em}Missing & 2958 (6.4\%) & 26494 (57.1\%)\\*

\end{longtable}

\endgroup{}
\endgroup{}

\subsubsection{Life-Focussed Well-Being
Indicators}\label{life-focussed-well-being-indicators}

\begingroup\fontsize{12}{14}\selectfont
\begingroup\fontsize{8}{10}\selectfont

\begin{longtable}[t]{lll}

\caption{\label{tbl-sample-outcomes-life}Life-reflective well-being
variables measured at baseline (NZAVS time 10, years 2018-2019, and time
15, years 2023-2024).}

\tabularnewline

\toprule
  & 2018 & 2023\\
\midrule
\endfirsthead
\multicolumn{3}{@{}l}{\textit{(continued)}}\\
\toprule
  & 2018 & 2023\\
\midrule
\endhead

\endfoot
\bottomrule
\endlastfoot
\cellcolor{gray!10}{} & \cellcolor{gray!10}{(N=46377)} & \cellcolor{gray!10}{(N=46377)}\\
\addlinespace[0.3em]
\multicolumn{3}{l}{\textbf{Gratitude}}\\
\hspace{1em}Mean (SD) & 5.90 (0.886) & 5.90 (0.910)\\
\cellcolor{gray!10}{\hspace{1em}Median [Min, Max]} & \cellcolor{gray!10}{6.00 [1.00, 7.00]} & \cellcolor{gray!10}{6.00 [1.00, \vphantom{1} 7.00]}\\
\hspace{1em}Missing & 9 (0.0\%) & 25102 (54.1\%)\\
\addlinespace[0.3em]
\multicolumn{3}{l}{\textbf{Lifesat}}\\
\cellcolor{gray!10}{\hspace{1em}Mean (SD)} & \cellcolor{gray!10}{5.29 (1.20)} & \cellcolor{gray!10}{5.22 (1.24)}\\
\hspace{1em}Median [Min, Max] & 5.50 [1.00, 7.00] & 5.50 [1.00, 7.00]\\
\cellcolor{gray!10}{\hspace{1em}Missing} & \cellcolor{gray!10}{223 (0.5\%)} & \cellcolor{gray!10}{25348 (54.7\%)}\\
\addlinespace[0.3em]
\multicolumn{3}{l}{\textbf{Meaning Purpose}}\\
\hspace{1em}Mean (SD) & 5.20 (1.42) & 5.16 (1.41)\\
\cellcolor{gray!10}{\hspace{1em}Median [Min, Max]} & \cellcolor{gray!10}{5.00 [1.00, 7.00]} & \cellcolor{gray!10}{5.00 [1.00, 7.00]}\\
\hspace{1em}Missing & 1224 (2.6\%) & 26377 (56.9\%)\\
\addlinespace[0.3em]
\multicolumn{3}{l}{\textbf{Meaning Sense}}\\
\cellcolor{gray!10}{\hspace{1em}Mean (SD)} & \cellcolor{gray!10}{5.71 (1.22)} & \cellcolor{gray!10}{5.78 (1.20)}\\
\hspace{1em}Median [Min, Max] & 6.00 [1.00, 7.00] & 6.00 [1.00, 7.00]\\
\cellcolor{gray!10}{\hspace{1em}Missing} & \cellcolor{gray!10}{160 (0.3\%)} & \cellcolor{gray!10}{26039 (56.1\%)}\\*

\end{longtable}

\endgroup{}
\endgroup{}

\subsubsection{Social Well-Being
Indicators}\label{social-well-being-indicators}

\begingroup\fontsize{12}{14}\selectfont
\begingroup\fontsize{8}{10}\selectfont

\begin{longtable}[t]{lll}

\caption{\label{tbl-sample-outcomes-social}Life-reflective well-being
variables measured at baseline (NZAVS time 10, years 2018-2019, and time
15, years 2023-2024).}

\tabularnewline

\toprule
  & 2018 & 2023\\
\midrule
\endfirsthead
\multicolumn{3}{@{}l}{\textit{(continued)}}\\
\toprule
  & 2018 & 2023\\
\midrule
\endhead

\endfoot
\bottomrule
\endlastfoot
\cellcolor{gray!10}{} & \cellcolor{gray!10}{(N=46377)} & \cellcolor{gray!10}{(N=46377)}\\
\addlinespace[0.3em]
\multicolumn{3}{l}{\textbf{Belonging}}\\
\hspace{1em}Mean (SD) & 5.14 (1.08) & 5.23 (1.10)\\
\cellcolor{gray!10}{\hspace{1em}Median [Min, Max]} & \cellcolor{gray!10}{5.33 [1.00, 7.00]} & \cellcolor{gray!10}{5.33 [1.00, 7.00]}\\
\hspace{1em}Missing & 397 (0.9\%) & 25100 (54.1\%)\\
\addlinespace[0.3em]
\multicolumn{3}{l}{\textbf{Sense of Neighbourhood Community}}\\
\cellcolor{gray!10}{\hspace{1em}Mean (SD)} & \cellcolor{gray!10}{4.19 (1.67)} & \cellcolor{gray!10}{4.63 (1.56)}\\
\hspace{1em}Median [Min, Max] & 4.00 [1.00, 7.00] & 5.00 [1.00, 7.00]\\
\cellcolor{gray!10}{\hspace{1em}Missing} & \cellcolor{gray!10}{252 (0.5\%)} & \cellcolor{gray!10}{25906 (55.9\%)}\\
\addlinespace[0.3em]
\multicolumn{3}{l}{\textbf{Social Support}}\\
\hspace{1em}Mean (SD) & 5.95 (1.12) & 6.00 (1.13)\\
\cellcolor{gray!10}{\hspace{1em}Median [Min, Max]} & \cellcolor{gray!10}{6.33 [1.00, 7.00]} & \cellcolor{gray!10}{6.33 [1.00, 7.00]}\\
\hspace{1em}Missing & 37 (0.1\%) & 25133 (54.2\%)\\*

\end{longtable}

\endgroup{}
\endgroup{}

\newpage{}

\subsection{Appendix C: Confouding Control}\label{appendix-confounding}

\begin{table}

\caption{\label{tbl-C}Table~\ref{tbl-C} presents single-world
intervention graphs showing time-fixed and time-varying sources of bias
in our six waves (baseline, four exposure waves, followed by the outcome
wave.) Time-fixed confounders are included in the baseline wave.
Time-varying confounders are included in each of the four treatment
waves (abbreviated here by `\(\dots\)' to declutter the graph). When
there is more than one exposure wave, identifying causal effects
requires adjustment for time-varying confounders
(\citeproc{ref-bulbulia2024swigstime}{Bulbulia 2024c};
\citeproc{ref-richardson2013}{Richardson and Robins 2013};
\citeproc{ref-robins2008estimation}{Robins and Hernan 2008}).}

\centering{

\tvtable

}

\end{table}%

For confounding control, we employ a modified disjunctive cause
criterion (\citeproc{ref-vanderweele2019}{VanderWeele 2019}), which
involves:

\begin{enumerate}
\def\labelenumi{\arabic{enumi}.}
\tightlist
\item
  Identifying all common causes of both the treatment and outcomes.
\item
  Excluding instrumental variables that affect the exposure but not the
  outcome.
\item
  Including proxies for unmeasured confounders affecting both exposure
  and outcome.
\item
  Controlling for baseline exposure and baseline outcome, serving as
  proxies for unmeasured common causes
  (\citeproc{ref-vanderweele2020}{VanderWeele \emph{et al.} 2020}).
\end{enumerate}

Additionally, we control for time-varying confounders at each exposure
wave (\citeproc{ref-bulbulia2024swigstime}{Bulbulia 2024c};
\citeproc{ref-richardson2013}{Richardson and Robins 2013};
\citeproc{ref-robins2008estimation}{Robins and Hernan 2008}).

The covariates included for confounding control are described in Rosa
\emph{et al.} (\citeproc{ref-pedro_2024effects}{2024}).

Where there are multiple exposures, causal inference may be threatened
by time-varying confounding
(\citeproc{ref-bulbulia2024swigstime}{Bulbulia 2024c}).

\newpage{}

\subsection{Appendix D: Causal Contrasts and Causal
Assumptions}\label{appendix-assumptions}

\subsubsection{Notation}\label{notation}

\begin{itemize}
\tightlist
\item
  \(A_k\): Observed religious service attendance at Wave \(k\), for
  \(k = 1, \dots, 4\).\\
\item
  \(Y_\tau\): Outcome measured at the end of the study (Wave 5).\\
\item
  \(W_0\): Confounders measured at baseline (Wave 0).\\
\item
  \(L_k\): Time-varying confounders measured at Wave \(k\) (for
  \(k = 1, \dots, 4\)).
\end{itemize}

\subsubsection{Shift Functions}\label{shift-functions}

Let \(\boldsymbol{\text{d}}(a_k)^+\) represent the \textbf{regular
attendance} treatment sequence and \(\boldsymbol{\text{d}}(a_k)^-\) the
\textbf{no attendance} treatment sequence, where the interventions
occure at each wave \(k = 1\dots 4; k\in \{0\dots 5\}\). Formally:

\paragraph{\texorpdfstring{Regular Attendance Regime
\(\bigl(\boldsymbol{\text{d}}(a_k^+)\bigr)\)}{Regular Attendance Regime \textbackslash bigl(\textbackslash boldsymbol\{\textbackslash text\{d\}\}(a\_k\^{}+)\textbackslash bigr)}}\label{regular-attendance-regime-biglboldsymboltextda_kbigr}

\[
\boldsymbol{\text{d}} (a_k^+) 
\;=\; 
\begin{cases}
4, & \text{if } A_k < 4,\\[6pt]
A_k, & \text{otherwise.}
\end{cases}
\]

\paragraph{\texorpdfstring{No Attendance Regime
\(\bigl(\boldsymbol{\text{d}}(a_k^-)\bigr)\)}{No Attendance Regime \textbackslash bigl(\textbackslash boldsymbol\{\textbackslash text\{d\}\}(a\_k\^{}-)\textbackslash bigr)}}\label{no-attendance-regime-biglboldsymboltextda_k-bigr}

\[
\boldsymbol{\text{d}}(a_k^-) 
\;=\; 
\begin{cases}
0, & \text{if } A_k > 0,\\[6pt]
A_k, & \text{otherwise.}
\end{cases}
\]

Here, \(A_k\) is the observed attendance at Wave \(k\). The shift
function \(\boldsymbol{\text{d}}\) ``nudges'' \(A_k\) to a target level
(four times per month or zero) only if the current value is below (for
regular attendance) or above (for no attendance) that target. Across the
four waves, these shifts form a sequence
\(\boldsymbol{\bar{\boldsymbol{\text{d}}}}\), which defines a complete
intervention regime.

\subsubsection{Causal Contrast}\label{causal-contrast}

We compare the average well-being under the \textbf{regular attendance}
regime, \(\boldsymbol{\bar{\boldsymbol{\text{d}}}}(a^+)\), to the
average well-being under the \textbf{no attendance} regime,
\(\boldsymbol{\bar{\boldsymbol{\text{d}}}}(a^-)\). Specifically, the
average treatment effect (ATE) is given:

\[
\text{ATE}^{\text{wellbeing}} 
\;=\; 
\mathbb{E}
\Bigl[
  Y_\tau\!\bigl(\boldsymbol{\text{d}}(a^+)\bigr) 
  \;-\; 
  Y_\tau\!\bigl(\boldsymbol{\text{d}}(a^-)\bigr)
\Bigr].
\]

\subsubsection{Assumptions}\label{assumptions}

To estimate this effect from observational data, we assume:

\begin{enumerate}
\def\labelenumi{\arabic{enumi}.}
\tightlist
\item
  \textbf{Conditional Exchangeability:} Once we condition on \(W_0\) and
  each \(L_k\), the interventions
  \(\boldsymbol{\bar{\boldsymbol{\text{d}}}}(a^+)\) or
  \(\boldsymbol{\bar{\boldsymbol{\text{d}}}}(a^-)\) are effectively
  random with respect to potential outcomes.
\item
  \textbf{Consistency:} The potential outcome under a given regime
  matches the observed outcome when that regime is followed.
\item
  \textbf{Positivity:} Everyone has a non-zero probability of receiving
  each level of attendance (i.e., a chance to be ``shifted'' up or down)
  given their covariates. The positivity assumption is the only causal
  assumption that can be evaluated with data. We evaluate this
  assumption in \hyperref[appendix-transition]{Appendix E}).
\end{enumerate}

Mathematically, for conditional exchangeability, we write:

\[
\Bigl\{
  Y\bigl(\boldsymbol{\text{d}}(a^+)\bigr), 
  \; 
  Y\bigl(\boldsymbol{\text{d}}(a^-)\bigr)
\Bigr\}
\coprod
A_k |
W_0,
L_k
\] That is, we assume the potential outcomes under each treatment regime
are independent of each treatment at every time point, conditional on
baseline confounders and time-varying confounders.

Under these assumptions, our statistical models permit us to estimate
\(\text{ATE}^{\text{wellbeing}}\) from observational data. That contrast
is (1) \emph{regularly attend religious service for four wave} versus
(2) \emph{never attend religious service at least four waves}. We define
the target population as the New Zealand Population from 2019-2024, the
years in which measurements were taken.

\newpage{}

\subsection{Appendix E: Transition Matrix to Check The Positivity
Assumption}\label{appendix-transition}

\begin{longtable}[]{@{}cccccc@{}}
\caption{Transition Matrix Showing
Change}\label{tbl-transition}\tabularnewline
\toprule\noalign{}
From & State 0 & State 1 & State 2 & State 3 & State 4 \\
\midrule\noalign{}
\endfirsthead
\toprule\noalign{}
From & State 0 & State 1 & State 2 & State 3 & State 4 \\
\midrule\noalign{}
\endhead
\bottomrule\noalign{}
\endlastfoot
State 0 & \textbf{40232} & 482 & 216 & 82 & 279 \\
State 1 & 672 & \textbf{246} & 92 & 45 & 73 \\
State 2 & 243 & 105 & \textbf{197} & 110 & 149 \\
State 3 & 118 & 54 & 111 & \textbf{171} & 225 \\
State 4 & 283 & 102 & 185 & 266 & \textbf{2338} \\
\end{longtable}

This transition matrix captures shifts in states across across the
treatment intervals. Each cell in the matrix represents the count of
individuals transitioning from one state to another. The rows correspond
to the treatment at baseline (From), and the columns correspond to the
state at the following wave (To). \textbf{Diagonal entries} (in
\textbf{bold}) correspond to the number of individuals who remained in
their initial state across both waves. \textbf{Off-diagonal entries}
correspond to the transitions of individuals from their baseline state
to a different state in the treatment wave. A higher number on the
diagonal relative to the off-diagonal entries in the same row indicates
greater stability in a state. Conversely, higher off-diagonal numbers
suggest more frequent shifts from the baseline state to other states.

\newpage{}

\subsection*{References}\label{references}
\addcontentsline{toc}{subsection}{References}

\phantomsection\label{refs}
\begin{CSLReferences}{1}{0}
\bibitem[\citeproctext]{ref-altemeyer1996authoritarian}
Altemeyer, B (1996) \emph{The authoritarian spectre}, London: Harvard
University Press.

\bibitem[\citeproctext]{ref-atkinson2019}
Atkinson, J, Salmond, C, and Crampton, P (2019) \emph{NZDep2018 index of
deprivation, user{'}s manual.}, Wellington.

\bibitem[\citeproctext]{ref-berry_forgivingness_2005}
Berry, JW, Worthington Jr., EL, O'Connor, LE, Parrott III, L, and Wade,
NG (2005) Forgivingness, vengeful rumination, and affective traits.
\emph{Journal of Personality}, \textbf{73}(1), 183--226.
doi:\href{https://doi.org/10.1111/j.1467-6494.2004.00308.x}{10.1111/j.1467-6494.2004.00308.x}.

\bibitem[\citeproctext]{ref-bono2004gratitude}
Bono, G, Emmons, RA, and McCullough, ME (2004) Gratitude in practice and
the practice of gratitude. \emph{Positive Psychology in Practice},
464--481.

\bibitem[\citeproctext]{ref-brown2023_church}
Brown, JE, Van Mulukom, V, Charles, SJ, and Farias, M (2023) Do you need
religion to enjoy the benefits of church services? Social bonding,
morality and quality of life among religious and secular congregations.
\emph{Psychology of Religion and Spirituality}, \textbf{15}(2), 308.

\bibitem[\citeproctext]{ref-bulbulia2022}
Bulbulia, JA (2023a) A workflow for causal inference in cross-cultural
psychology. \emph{Religion, Brain \& Behavior}, \textbf{13}(3),
291--306.
doi:\href{https://doi.org/10.1080/2153599X.2022.2070245}{10.1080/2153599X.2022.2070245}.

\bibitem[\citeproctext]{ref-bulbulia2023understanding}
Bulbulia, JA (2023b) Understanding the relationship between science and
religion using bayes' theorem. \emph{Studies in Christian Ethics},
\textbf{36}(4), 866--878.

\bibitem[\citeproctext]{ref-margot2024}
Bulbulia, JA (2024a) \emph{Margot: MARGinal observational
treatment-effects}.
doi:\href{https://doi.org/10.5281/zenodo.10907724}{10.5281/zenodo.10907724}.

\bibitem[\citeproctext]{ref-bulbulia2023}
Bulbulia, JA (2024b) Methods in causal inference part 1: Causal diagrams
and confounding. \emph{Evolutionary Human Sciences}, \textbf{6}, e40.
doi:\href{https://doi.org/10.1017/ehs.2024.35}{10.1017/ehs.2024.35}.

\bibitem[\citeproctext]{ref-bulbulia2024swigstime}
Bulbulia, JA (2024c) Methods in causal inference part 2: Interaction,
mediation, and time-varying treatments. \emph{Evolutionary Human
Sciences}, \textbf{6}, e41.
doi:\href{https://doi.org/10.1017/ehs.2024.32}{10.1017/ehs.2024.32}.

\bibitem[\citeproctext]{ref-bulbulia2024wierd}
Bulbulia, JA (2024d) Methods in causal inference part 3: Measurement
error and external validity threats. \emph{Evolutionary Human Sciences},
\textbf{6}, e42.
doi:\href{https://doi.org/10.1017/ehs.2024.33}{10.1017/ehs.2024.33}.

\bibitem[\citeproctext]{ref-Bulbulia_2015}
Bulbulia, JA, Shaver, JH, Greaves, L, Sosis, R, and Sibley, CG (2015)
Religion and parental cooperation: An empirical test of slone's sexual
signaling model. In S. J. D. amd Van Slyke J., ed., \emph{The attraction
of religion: A sexual selectionist account}, Bloomsbury Press, 29--62.

\bibitem[\citeproctext]{ref-buysse1989pittsburgh}
Buysse, DJ, Reynolds III, CF, Monk, TH, Berman, SR, and Kupfer, DJ
(1989) The pittsburgh sleep quality index: A new instrument for
psychiatric practice and research. \emph{Psychiatry Research},
\textbf{28}(2), 193--213.

\bibitem[\citeproctext]{ref-caprara_indicators_1986}
Caprara, GV (1986) Indicators of aggression: The dissipation-rumination
scale. \emph{Personality and Individual Differences}, \textbf{7}(6),
763--769.
doi:\href{https://doi.org/10.1016/0191-8869(86)90074-7}{10.1016/0191-8869(86)90074-7}.

\bibitem[\citeproctext]{ref-xgboost2023}
Chen, T, He, T, Benesty, M, \ldots{} Yuan, J (2023) \emph{Xgboost:
Extreme gradient boosting}. Retrieved from
\url{https://CRAN.R-project.org/package=xgboost}

\bibitem[\citeproctext]{ref-chen2020religious}
Chen, Y, Kim, ES, and VanderWeele, TJ (2020) Religious-service
attendance and subsequent health and well-being throughout adulthood:
Evidence from three prospective cohorts. \emph{International Journal of
Epidemiology}, \textbf{49}(6), 2030--2040.

\bibitem[\citeproctext]{ref-chen2022_church_longitudinal}
Chen, Y, Weziak-Bialowolska, D, Lee, MT, Bialowolski, P, McNeely, E, and
VanderWeele, TJ (2022) Longitudinal associations between domains of
flourishing. \emph{Scientific Reports}, \textbf{12}(1), 2740.

\bibitem[\citeproctext]{ref-cummins_developing_2003}
Cummins, RA, Eckersley, R, Pallant, J, Vugt, J van, and Misajon, R
(2003) Developing a national index of subjective wellbeing: The
australian unity wellbeing index. \emph{Social Indicators Research},
\textbf{64}(2), 159--190.
doi:\href{https://doi.org/10.1023/A:1024704320683}{10.1023/A:1024704320683}.

\bibitem[\citeproctext]{ref-cutrona1987}
Cutrona, CE, and Russell, DW (1987) The provisions of social
relationships and adaptation to stress. \emph{Advances in Personal
Relationships}, \textbf{1}, 37--67.

\bibitem[\citeproctext]{ref-deak2021individuals}
Deak, CK, Hammond, MD, Sibley, CG, and Bulbulia, J (2021) Individuals'
number of children is associated with benevolent sexism. \emph{PloS
One}, \textbf{16}(5), e0252194.

\bibitem[\citeproctext]{ref-dew2020joint}
Dew, JP, Uecker, JE, and Willoughby, BJ (2020) Joint religiosity and
married couples' sexual satisfaction. \emph{Psychology of Religion and
Spirituality}, \textbf{12}(2), 201.

\bibitem[\citeproctext]{ref-duxedaz2021}
Díaz, I, Williams, N, Hoffman, KL, and Schenck, EJ (2021b)
Non-parametric causal effects based on longitudinal modified treatment
policies. \emph{Journal of the American Statistical Association}.
doi:\href{https://doi.org/10.1080/01621459.2021.1955691}{10.1080/01621459.2021.1955691}.

\bibitem[\citeproctext]{ref-diaz2021_non_parametric_lmtp}
Díaz, I, Williams, N, Hoffman, KL, and Schenck, EJ (2021a)
Non-parametric causal effects based on longitudinal modified treatment
policies. \emph{Journal of the American Statistical Association}.
doi:\href{https://doi.org/10.1080/01621459.2021.1955691}{10.1080/01621459.2021.1955691}.

\bibitem[\citeproctext]{ref-diaz2023lmtp}
Díaz, I, Williams, N, Hoffman, KL, and Schenck, EJ (2023) Nonparametric
causal effects based on longitudinal modified treatment policies.
\emph{Journal of the American Statistical Association},
\textbf{118}(542), 846--857.
doi:\href{https://doi.org/10.1080/01621459.2021.1955691}{10.1080/01621459.2021.1955691}.

\bibitem[\citeproctext]{ref-diener1985a}
Diener, E, Emmons, RA, Larsen, RJ, and Griffin, S (1985) The
Satisfaction With Life Scale. \emph{Journal of Personality Assessment},
\textbf{49}(1), 71--75.

\bibitem[\citeproctext]{ref-dunbar2021religiosity}
Dunbar, RI (2021) Religiosity and religious attendance as factors in
wellbeing and social engagement. \emph{Religion, Brain \& Behavior},
\textbf{11}(1), 17--26.

\bibitem[\citeproctext]{ref-eliade1984quest}
Eliade, M (1984) \emph{The quest: History and meaning in religion},
University of Chicago Press.

\bibitem[\citeproctext]{ref-fahy2017}
Fahy, KM, Lee, A, and Milne, BJ (2017) \emph{{N}ew {Z}ealand
socio-economic index 2013}, Wellington, New Zealand: Statistics New
Zealand-Tatauranga Aotearoa.

\bibitem[\citeproctext]{ref-fraser_coding_2020}
Fraser, G, Bulbulia, J, Greaves, LM, Wilson, MS, and Sibley, CG (2020)
Coding responses to an open-ended gender measure in a {N}ew {Z}ealand
national sample. \emph{The Journal of Sex Research}, \textbf{57}(8),
979--986.
doi:\href{https://doi.org/10.1080/00224499.2019.1687640}{10.1080/00224499.2019.1687640}.

\bibitem[\citeproctext]{ref-greaves2017diversity}
Greaves, LM, Barlow, FK, Lee, CH, et al.others (2017) The diversity and
prevalence of sexual orientation self-labels in a {N}ew {Z}ealand
national sample. \emph{Archives of Sexual Behavior}, \textbf{46},
1325--1336.

\bibitem[\citeproctext]{ref-hagerty1995}
Hagerty, BMK, and Patusky, K (1995) Developing a Measure Of Sense of
Belonging: \emph{Nursing Research}, \textbf{44}(1), 9--13.
doi:\href{https://doi.org/10.1097/00006199-199501000-00003}{10.1097/00006199-199501000-00003}.

\bibitem[\citeproctext]{ref-Ministry_of_Health_2013}
Health, Ministry of (2013) \emph{The {N}ew {Z}ealand {H}ealth {S}urvey:
Content guide 2012-2013}, Princeton University Press.

\bibitem[\citeproctext]{ref-hernan2024WHATIF}
Hernan, MA, and Robins, JM (2024) \emph{Causal inference: What if?},
Taylor \& Francis. Retrieved from
\url{https://www.hsph.harvard.edu/miguel-hernan/causal-inference-book/}

\bibitem[\citeproctext]{ref-hernan2009MEASUREMENT}
Hernán, MA, and Cole, SR (2009) Invited commentary: Causal diagrams and
measurement bias. \emph{American Journal of Epidemiology},
\textbf{170}(8), 959--962.
doi:\href{https://doi.org/10.1093/aje/kwp293}{10.1093/aje/kwp293}.

\bibitem[\citeproctext]{ref-hernan2016}
Hernán, MA, Sauer, BC, Hernández-Díaz, S, Platt, R, and Shrier, I (2016)
Specifying a target trial prevents immortal time bias and other
self-inflicted injuries in observational analyses. \emph{Journal of
Clinical Epidemiology}, \textbf{79}, 70--75.

\bibitem[\citeproctext]{ref-highland2019attitudes}
Highland, BR, Troughton, G, Shaver, J, Barrett, JL, Sibley, CG, and
Bulbulia, J (2019) Attitudes to religion predict warmth for muslims in
new zealand. \emph{New Zealand Journal of Psychology (Online)},
\textbf{48}(1), 122--132.

\bibitem[\citeproctext]{ref-highland2022national}
Highland, B, Worthington, EL, Davis, DE, Sibley, CG, and Bulbulia, JA
(2022) National longitudinal evidence for growth in subjective
well-being from spiritual beliefs. \emph{Journal of Health Psychology},
\textbf{27}(7), 1738--1752.

\bibitem[\citeproctext]{ref-hoffman2023}
Hoffman, KL, Salazar-Barreto, D, Rudolph, KE, and Díaz, I (2023)
Introducing longitudinal modified treatment policies: A unified
framework for studying complex exposures.
doi:\href{https://doi.org/10.48550/arXiv.2304.09460}{10.48550/arXiv.2304.09460}.

\bibitem[\citeproctext]{ref-hoffman2022}
Hoffman, KL, Schenck, EJ, Satlin, MJ, \ldots{} Díaz, I (2022) Comparison
of a target trial emulation framework vs cox regression to estimate the
association of corticosteroids with COVID-19 mortality. \emph{JAMA
Network Open}, \textbf{5}(10), e2234425.
doi:\href{https://doi.org/10.1001/jamanetworkopen.2022.34425}{10.1001/jamanetworkopen.2022.34425}.

\bibitem[\citeproctext]{ref-instrument1992mos}
Instrument Ware Jr, J, and Sherbourne, C (1992) The MOS 36-item
short-form health survey (SF-36): I. Conceptual framework and item
selection. \emph{Medical Care}, \textbf{30}(6), 473--483.

\bibitem[\citeproctext]{ref-johnson2022global}
Johnson, BR, and VanderWeele, TJ (2022) The global flourishing study: A
new era for the study of well-being. \emph{International Bulletin of
Mission Research}, \textbf{46}(2), 272--275.

\bibitem[\citeproctext]{ref-jokela2024_church}
Jokela, M, and Laakasuo, M (2024) Health trajectories of individuals who
quit active religious attendance: Analysis of four prospective cohort
studies in the united states. \emph{Social Psychiatry and Psychiatric
Epidemiology}, \textbf{59}(5), 871--878.

\bibitem[\citeproctext]{ref-jost_end_2006-1}
Jost, JT (2006) The end of the end of ideology. \emph{American
Psychologist}, \textbf{61}(7), 651--670.
doi:\href{https://doi.org/10.1037/0003-066X.61.7.651}{10.1037/0003-066X.61.7.651}.

\bibitem[\citeproctext]{ref-kessler2002}
Kessler, R~C, Andrews, G, Colpe, L~J, \ldots{} Zaslavsky, A~M (2002)
Short screening scales to monitor population prevalences and trends in
non-specific psychological distress. \emph{Psychological Medicine},
\textbf{32}(6), 959--976.
doi:\href{https://doi.org/10.1017/S0033291702006074}{10.1017/S0033291702006074}.

\bibitem[\citeproctext]{ref-koenig2012handbook}
Koenig, HG, King, D, and Carson, VB (2012) \emph{Handbook of religion
and health}, Oxford University Press.

\bibitem[\citeproctext]{ref-vanderweelechurchmortality}
Li, S, Stampfer, MJ, Williams, DR, and VanderWeele, TJ (2016)
{Association of Religious Service Attendance With Mortality Among
Women}. \emph{JAMA Internal Medicine}, \textbf{176}(6), 777--785.
doi:\href{https://doi.org/10.1001/jamainternmed.2016.1615}{10.1001/jamainternmed.2016.1615}.

\bibitem[\citeproctext]{ref-linden2020EVALUE}
Linden, A, Mathur, MB, and VanderWeele, TJ (2020) Conducting sensitivity
analysis for unmeasured confounding in observational studies using
e-values: The evalue package. \emph{The Stata Journal}, \textbf{20}(1),
162--175.

\bibitem[\citeproctext]{ref-mccullough_grateful_2002}
McCullough, ME, Emmons, RA, and Tsang, J-A (2002) The grateful
disposition: A conceptual and empirical topography. \emph{Journal of
Personality and Social Psychology}, \textbf{82}(1), 112--127.
doi:\href{https://doi.org/10.1037/0022-3514.82.1.112}{10.1037/0022-3514.82.1.112}.

\bibitem[\citeproctext]{ref-mccullough2001gratitude}
McCullough, ME, Kilpatrick, SD, Emmons, RA, and Larson, DB (2001) Is
gratitude a moral affect? \emph{Psychological Bulletin},
\textbf{127}(2), 249.

\bibitem[\citeproctext]{ref-nolen-hoeksema_effects_1993}
Nolen-hoeksema, S, and Morrow, J (1993) Effects of rumination and
distraction on naturally occurring depressed mood. \emph{Cognition and
Emotion}, \textbf{7}(6), 561--570.
doi:\href{https://doi.org/10.1080/02699939308409206}{10.1080/02699939308409206}.

\bibitem[\citeproctext]{ref-overall2016power}
Overall, NC, Hammond, MD, McNulty, JK, and Finkel, EJ (2016) When power
shapes interpersonal behavior: Low relationship power predicts men's
aggressive responses to low situational power. \emph{Journal of
Personality and Social Psychology}, \textbf{111}(2), 195.

\bibitem[\citeproctext]{ref-park2005religion_meaning}
Park, CL et al. (2005) Religion and meaning. \emph{Handbook of the
Psychology of Religion and Spirituality}, \textbf{2}, 357--379.

\bibitem[\citeproctext]{ref-park2005religion}
Park, CL (2005) Religion as a meaning-making framework in coping with
life stress. \emph{Journal of Social Issues}, \textbf{61}(4), 707--729.

\bibitem[\citeproctext]{ref-pawlikowski2019religious}
Pawlikowski, J, Białowolski, P, Węziak-Białowolska, D, and VanderWeele,
TJ (2019) Religious service attendance, health behaviors and
well-being---an outcome-wide longitudinal analysis. \emph{European
Journal of Public Health}, \textbf{29}(6), 1177--1183.

\bibitem[\citeproctext]{ref-polley2023}
Polley, E, LeDell, E, Kennedy, C, and Laan, M van der (2023a)
\emph{SuperLearner: Super learner prediction}. Retrieved from
\url{https://CRAN.R-project.org/package=SuperLearner}

\bibitem[\citeproctext]{ref-SuperLearner2023}
Polley, E, LeDell, E, Kennedy, C, and van der Laan, M (2023b)
\emph{SuperLearner: Super learner prediction}. Retrieved from
\url{https://github.com/ecpolley/SuperLearner}

\bibitem[\citeproctext]{ref-price2024science}
Price, ME, and Johnson, DD (2024) Science and religion around the world:
Compatibility between belief systems predicts increased well-being.
\emph{Religion, Brain \& Behavior}, 1--20.

\bibitem[\citeproctext]{ref-reddish2013let}
Reddish, P, Fischer, R, and Bulbulia, J (2013) Let's dance together:
Synchrony, shared intentionality and cooperation. \emph{PloS One},
\textbf{8}(8), e71182.

\bibitem[\citeproctext]{ref-rice_short_2014}
Rice, KG, Richardson, CME, and Tueller, S (2014) The short form of the
revised almost perfect scale. \emph{Journal of Personality Assessment},
\textbf{96}(3), 368--379.
doi:\href{https://doi.org/10.1080/00223891.2013.838172}{10.1080/00223891.2013.838172}.

\bibitem[\citeproctext]{ref-richardson2013}
Richardson, TS, and Robins, JM (2013) Single world intervention graphs:
A primer. In, Citeseer. Retrieved from
\url{https://core.ac.uk/display/102673558}

\bibitem[\citeproctext]{ref-robins2008estimation}
Robins, J, and Hernan, M (2008) Estimation of the causal effects of
time-varying exposures. \emph{Chapman \& Hall/CRC Handbooks of Modern
Statistical Methods}, 553--599.

\bibitem[\citeproctext]{ref-pedro_2024effects}
Rosa, PA de la, Cowden, RG, Bulbulia, JA, Sibley, CG, and VanderWeele,
TJ (2024) Effects of screen-based leisure time on 24 subsequent health
and wellbeing outcomes: A longitudinal outcome-wide analysis.
\emph{International Journal of Behavioral Medicine}, 1--20.
doi:\url{https://doi.org/10.1007/s12529-024-10307-0}.

\bibitem[\citeproctext]{ref-Rosenberg1965}
Rosenberg, M (1965) \emph{Society and the adolescent self-image},
Princeton, NJ: Princeton University Press.

\bibitem[\citeproctext]{ref-sengupta2013}
Sengupta, NK, Luyten, N, Greaves, LM, \ldots{} Sibley, CG (2013) Sense
of Community in {N}ew {Z}ealand Neighbourhoods: A Multi-Level Model
Predicting Social Capital. \emph{New Zealand Journal of Psychology},
\textbf{42}(1), 36--45.

\bibitem[\citeproctext]{ref-shaver2024religious}
Shaver, JH, Chvaja, R, Spake, L, \ldots{} Sosis, R (2024) Religious
involvement is associated with higher fertility and lower maternal
investment, but more alloparental support among gambian mothers.
\emph{American Journal of Human Biology}, e24144.

\bibitem[\citeproctext]{ref-shaver2020church}
Shaver, JH, Power, EA, Purzycki, BG, \ldots{} Bulbulia, JA (2020) Church
attendance and alloparenting: An analysis of fertility, social support
and child development among {E}nglish mothers. \emph{Philosophical
Transactions of the Royal Society B}, \textbf{375}(1805), 20190428.

\bibitem[\citeproctext]{ref-shaver2016religion}
Shaver, JH, Troughton, G, Sibley, CG, and Bulbulia, JA (2016) Religion
and the unmaking of prejudice toward muslims: Evidence from a large
national sample. \emph{PloS One}, \textbf{11}(3), e0150209.

\bibitem[\citeproctext]{ref-shaver2021comparison}
Shaver, JH, White, TA, Vakaoti, P, and Lang, M (2021) A comparison of
self-report, systematic observation and third-party judgments of church
attendance in a rural fijian village. \emph{Plos One}, \textbf{16}(10),
e0257160.

\bibitem[\citeproctext]{ref-shaver2022integrating}
Shaver, J, and White, T (2022) Integrating pacific research
methodologies with western social science research methods: Quantifying
pentecostalism's effects on fijian relationality.

\bibitem[\citeproctext]{ref-sibley2012}
Sibley, C. G., and Bulbulia, JA (2012) Healing those who need healing:
How religious practice affects social belonging. \emph{Journal for the
Cognitive Science of Religion}, \textbf{1}, 29--45.

\bibitem[\citeproctext]{ref-sibley2021}
Sibley, CG (2021)
\emph{\href{https://doi.org/10.31234/osf.io/wgqvy}{Sampling procedure
and sample details for the {N}ew {Z}ealand {A}ttitudes and {V}alues
{S}tudy}}.

\bibitem[\citeproctext]{ref-sibley2020}
Sibley, CG, Afzali, MU, Satherley, N, \ldots{} others (2020) Prejudice
toward muslims in {N}ew {Z}ealand: Insights from the {N}ew {Z}ealand
{A}ttitudes and {V}alues {S}tudy. \emph{New Zealand Journal of
Psychology}, \textbf{49}(1).

\bibitem[\citeproctext]{ref-sibley2011}
Sibley, CG, Luyten, N, Purnomo, M, \ldots{} Robertson, A (2011) The
Mini-IPIP6: Validation and extension of a short measure of the Big-Six
factors of personality in {N}ew {Z}ealand. \emph{New Zealand Journal of
Psychology}, \textbf{40}(3), 142--159.

\bibitem[\citeproctext]{ref-sidanius1999social}
Sidanius, J, and Pratto, F (1999) \emph{Social dominance: An intergroup
theory of social hierarchy and oppression}, Cambridge: Cambridge
University Press.

\bibitem[\citeproctext]{ref-sosis2003cooperation}
Sosis, R, and Bressler, ER (2003) Cooperation and commune longevity: A
test of the costly signaling theory of religion. \emph{Cross-Cultural
Research}, \textbf{37}(2), 211--239.

\bibitem[\citeproctext]{ref-spake2024practical}
Spake, L, Hassan, A, Schaffnit, SB, et al.others (2024) A practical
guide to cross-cultural and multi-sited data collection in the
biological and behavioural sciences. \emph{Proceedings of the Royal
Society B}, \textbf{291}(2021), 20231422.

\bibitem[\citeproctext]{ref-steger_meaning_2006}
Steger, MF, Frazier, P, Oishi, S, and Kaler, M (2006) The meaning in
life questionnaire: Assessing the presence of and search for meaning in
life. \emph{Journal of Counseling Psychology}, \textbf{53}(1), 80--93.
doi:\href{https://doi.org/10.1037/0022-0167.53.1.80}{10.1037/0022-0167.53.1.80}.

\bibitem[\citeproctext]{ref-sterelny2018religion}
Sterelny, K (2018) Religion re-explained. \emph{Religion, Brain \&
Behavior}, \textbf{8}(4), 406--425.

\bibitem[\citeproctext]{ref-stronge2015facebook}
Stronge, S, Greaves, LM, Milojev, P, West-Newman, T, Barlow, FK, and
Sibley, CG (2015) Facebook is linked to body dissatisfaction: Comparing
users and non-users. \emph{Sex Roles}, \textbf{73}, 200--213.

\bibitem[\citeproctext]{ref-tangney_high_2004}
Tangney, JP, Baumeister, RF, and Boone, AL (2004) High self-control
predicts good adjustment, less pathology, better grades, and
interpersonal success. \emph{Journal of Personality}, \textbf{72}(2),
271--324.
doi:\href{https://doi.org/10.1111/j.0022-3506.2004.00263.x}{10.1111/j.0022-3506.2004.00263.x}.

\bibitem[\citeproctext]{ref-vanbuuren2018}
Van Buuren, S (2018) \emph{Flexible imputation of missing data}, CRC
press.

\bibitem[\citeproctext]{ref-vantongeren2020}
Van Tongeren, DR, DeWall, CN, Chen, Z, Sibley, CG, and Bulbulia, J
(2020) Religious residue: Cross-cultural evidence that religious
psychology and behavior persist following deidentification.
\emph{Journal of Personality and Social Psychology}.

\bibitem[\citeproctext]{ref-vanderweele2019}
VanderWeele, TJ (2019) Principles of confounder selection.
\emph{European Journal of Epidemiology}, \textbf{34}(3), 211--219.

\bibitem[\citeproctext]{ref-vanderweele2021effectsReligiousServiceMetanalysis}
VanderWeele, TJ (2021) Effects of religious service attendance and
religious importance on depression: Examining the meta-analytic
evidence. \emph{The International Journal for the Psychology of
Religion}, \textbf{31}(1), 21--26.

\bibitem[\citeproctext]{ref-vanderweele2022invited_church}
VanderWeele, TJ, Balboni, TA, and Koh, HK (2022) Invited commentary:
Religious service attendance and implications for clinical care,
community participation, and public health. \emph{American Journal of
Epidemiology}, \textbf{191}(1), 31--35.

\bibitem[\citeproctext]{ref-vanderweele2020vchenrespond}
VanderWeele, TJ, and Chen, Y (2020) VanderWeele and chen respond to
{``religion as a social determinant of health.''} \emph{American Journal
of Epidemiology}, \textbf{189}(12), 1464--1466.

\bibitem[\citeproctext]{ref-vanderweele2017}
VanderWeele, TJ, and Ding, P (2017) Sensitivity analysis in
observational research: Introducing the {E}-value. \emph{Annals of
Internal Medicine}, \textbf{167}(4), 268--274.
doi:\href{https://doi.org/10.7326/M16-2607}{10.7326/M16-2607}.

\bibitem[\citeproctext]{ref-vanderweele2012a}
VanderWeele, TJ, and Hernán, MA (2012) Results on differential and
dependent measurement error of the exposure and the outcome using signed
directed acyclic graphs. \emph{American Journal of Epidemiology},
\textbf{175}(12), 1303--1310.
doi:\href{https://doi.org/10.1093/aje/kwr458}{10.1093/aje/kwr458}.

\bibitem[\citeproctext]{ref-vanderweele2020}
VanderWeele, TJ, Mathur, MB, and Chen, Y (2020) Outcome-wide
longitudinal designs for causal inference: A new template for empirical
studies. \emph{Statistical Science}, \textbf{35}(3), 437--466.

\bibitem[\citeproctext]{ref-verbrugge1997}
Verbrugge, LM (1997) A global disability indicator. \emph{Journal of
Aging Studies}, \textbf{11}(4), 337--362.
doi:\href{https://doi.org/10.1016/S0890-4065(97)90026-8}{10.1016/S0890-4065(97)90026-8}.

\bibitem[\citeproctext]{ref-watts2015pulotu}
Watts, J, Sheehan, O, Greenhill, SJ, \ldots{} Gray, RD (2015) Pulotu:
Database of austronesian supernatural beliefs and practices. \emph{PloS
One}, \textbf{10}(9), e0136783.

\bibitem[\citeproctext]{ref-whitehead2023unmasking}
Whitehead, J, Davie, G, Graaf, B de, \ldots{} Nixon, G (2023) Unmasking
hidden disparities: A comparative observational study examining the
impact of different rurality classifications for health research in
aotearoa new zealand. \emph{BMJ Open}, \textbf{13}(4), e067927.

\bibitem[\citeproctext]{ref-williams2021}
Williams, NT, and Díaz, I (2021) \emph{{l}mtp: Non-parametric causal
effects of feasible interventions based on modified treatment policies}.
doi:\href{https://doi.org/10.5281/zenodo.3874931}{10.5281/zenodo.3874931}.

\bibitem[\citeproctext]{ref-Ranger2017}
Wright, MN, and Ziegler, A (2017) {ranger}: A fast implementation of
random forests for high dimensional data in {C++} and {R}. \emph{Journal
of Statistical Software}, \textbf{77}(1), 1--17.
doi:\href{https://doi.org/10.18637/jss.v077.i01}{10.18637/jss.v077.i01}.

\bibitem[\citeproctext]{ref-zhang2023shouldMultipleImputation}
Zhang, J, Dashti, SG, Carlin, JB, Lee, KJ, and Moreno-Betancur, M (2023)
Should multiple imputation be stratified by exposure group when
estimating causal effects via outcome regression in observational
studies? \emph{BMC Medical Research Methodology}, \textbf{23}(1), 42.

\end{CSLReferences}




\end{document}
