% Options for packages loaded elsewhere
\PassOptionsToPackage{unicode}{hyperref}
\PassOptionsToPackage{hyphens}{url}
\PassOptionsToPackage{dvipsnames,svgnames,x11names}{xcolor}
%
\documentclass[
  single column]{article}

\usepackage{amsmath,amssymb}
\usepackage{iftex}
\ifPDFTeX
  \usepackage[T1]{fontenc}
  \usepackage[utf8]{inputenc}
  \usepackage{textcomp} % provide euro and other symbols
\else % if luatex or xetex
  \usepackage{unicode-math}
  \defaultfontfeatures{Scale=MatchLowercase}
  \defaultfontfeatures[\rmfamily]{Ligatures=TeX,Scale=1}
\fi
\usepackage[]{libertinus}
\ifPDFTeX\else  
    % xetex/luatex font selection
\fi
% Use upquote if available, for straight quotes in verbatim environments
\IfFileExists{upquote.sty}{\usepackage{upquote}}{}
\IfFileExists{microtype.sty}{% use microtype if available
  \usepackage[]{microtype}
  \UseMicrotypeSet[protrusion]{basicmath} % disable protrusion for tt fonts
}{}
\makeatletter
\@ifundefined{KOMAClassName}{% if non-KOMA class
  \IfFileExists{parskip.sty}{%
    \usepackage{parskip}
  }{% else
    \setlength{\parindent}{0pt}
    \setlength{\parskip}{6pt plus 2pt minus 1pt}}
}{% if KOMA class
  \KOMAoptions{parskip=half}}
\makeatother
\usepackage{xcolor}
\usepackage[top=30mm,left=25mm,heightrounded,headsep=22pt,headheight=11pt,footskip=33pt,ignorehead,ignorefoot]{geometry}
\setlength{\emergencystretch}{3em} % prevent overfull lines
\setcounter{secnumdepth}{-\maxdimen} % remove section numbering
% Make \paragraph and \subparagraph free-standing
\makeatletter
\ifx\paragraph\undefined\else
  \let\oldparagraph\paragraph
  \renewcommand{\paragraph}{
    \@ifstar
      \xxxParagraphStar
      \xxxParagraphNoStar
  }
  \newcommand{\xxxParagraphStar}[1]{\oldparagraph*{#1}\mbox{}}
  \newcommand{\xxxParagraphNoStar}[1]{\oldparagraph{#1}\mbox{}}
\fi
\ifx\subparagraph\undefined\else
  \let\oldsubparagraph\subparagraph
  \renewcommand{\subparagraph}{
    \@ifstar
      \xxxSubParagraphStar
      \xxxSubParagraphNoStar
  }
  \newcommand{\xxxSubParagraphStar}[1]{\oldsubparagraph*{#1}\mbox{}}
  \newcommand{\xxxSubParagraphNoStar}[1]{\oldsubparagraph{#1}\mbox{}}
\fi
\makeatother

\usepackage{color}
\usepackage{fancyvrb}
\newcommand{\VerbBar}{|}
\newcommand{\VERB}{\Verb[commandchars=\\\{\}]}
\DefineVerbatimEnvironment{Highlighting}{Verbatim}{commandchars=\\\{\}}
% Add ',fontsize=\small' for more characters per line
\usepackage{framed}
\definecolor{shadecolor}{RGB}{241,243,245}
\newenvironment{Shaded}{\begin{snugshade}}{\end{snugshade}}
\newcommand{\AlertTok}[1]{\textcolor[rgb]{0.68,0.00,0.00}{#1}}
\newcommand{\AnnotationTok}[1]{\textcolor[rgb]{0.37,0.37,0.37}{#1}}
\newcommand{\AttributeTok}[1]{\textcolor[rgb]{0.40,0.45,0.13}{#1}}
\newcommand{\BaseNTok}[1]{\textcolor[rgb]{0.68,0.00,0.00}{#1}}
\newcommand{\BuiltInTok}[1]{\textcolor[rgb]{0.00,0.23,0.31}{#1}}
\newcommand{\CharTok}[1]{\textcolor[rgb]{0.13,0.47,0.30}{#1}}
\newcommand{\CommentTok}[1]{\textcolor[rgb]{0.37,0.37,0.37}{#1}}
\newcommand{\CommentVarTok}[1]{\textcolor[rgb]{0.37,0.37,0.37}{\textit{#1}}}
\newcommand{\ConstantTok}[1]{\textcolor[rgb]{0.56,0.35,0.01}{#1}}
\newcommand{\ControlFlowTok}[1]{\textcolor[rgb]{0.00,0.23,0.31}{\textbf{#1}}}
\newcommand{\DataTypeTok}[1]{\textcolor[rgb]{0.68,0.00,0.00}{#1}}
\newcommand{\DecValTok}[1]{\textcolor[rgb]{0.68,0.00,0.00}{#1}}
\newcommand{\DocumentationTok}[1]{\textcolor[rgb]{0.37,0.37,0.37}{\textit{#1}}}
\newcommand{\ErrorTok}[1]{\textcolor[rgb]{0.68,0.00,0.00}{#1}}
\newcommand{\ExtensionTok}[1]{\textcolor[rgb]{0.00,0.23,0.31}{#1}}
\newcommand{\FloatTok}[1]{\textcolor[rgb]{0.68,0.00,0.00}{#1}}
\newcommand{\FunctionTok}[1]{\textcolor[rgb]{0.28,0.35,0.67}{#1}}
\newcommand{\ImportTok}[1]{\textcolor[rgb]{0.00,0.46,0.62}{#1}}
\newcommand{\InformationTok}[1]{\textcolor[rgb]{0.37,0.37,0.37}{#1}}
\newcommand{\KeywordTok}[1]{\textcolor[rgb]{0.00,0.23,0.31}{\textbf{#1}}}
\newcommand{\NormalTok}[1]{\textcolor[rgb]{0.00,0.23,0.31}{#1}}
\newcommand{\OperatorTok}[1]{\textcolor[rgb]{0.37,0.37,0.37}{#1}}
\newcommand{\OtherTok}[1]{\textcolor[rgb]{0.00,0.23,0.31}{#1}}
\newcommand{\PreprocessorTok}[1]{\textcolor[rgb]{0.68,0.00,0.00}{#1}}
\newcommand{\RegionMarkerTok}[1]{\textcolor[rgb]{0.00,0.23,0.31}{#1}}
\newcommand{\SpecialCharTok}[1]{\textcolor[rgb]{0.37,0.37,0.37}{#1}}
\newcommand{\SpecialStringTok}[1]{\textcolor[rgb]{0.13,0.47,0.30}{#1}}
\newcommand{\StringTok}[1]{\textcolor[rgb]{0.13,0.47,0.30}{#1}}
\newcommand{\VariableTok}[1]{\textcolor[rgb]{0.07,0.07,0.07}{#1}}
\newcommand{\VerbatimStringTok}[1]{\textcolor[rgb]{0.13,0.47,0.30}{#1}}
\newcommand{\WarningTok}[1]{\textcolor[rgb]{0.37,0.37,0.37}{\textit{#1}}}

\providecommand{\tightlist}{%
  \setlength{\itemsep}{0pt}\setlength{\parskip}{0pt}}\usepackage{longtable,booktabs,array}
\usepackage{calc} % for calculating minipage widths
% Correct order of tables after \paragraph or \subparagraph
\usepackage{etoolbox}
\makeatletter
\patchcmd\longtable{\par}{\if@noskipsec\mbox{}\fi\par}{}{}
\makeatother
% Allow footnotes in longtable head/foot
\IfFileExists{footnotehyper.sty}{\usepackage{footnotehyper}}{\usepackage{footnote}}
\makesavenoteenv{longtable}
\usepackage{graphicx}
\makeatletter
\def\maxwidth{\ifdim\Gin@nat@width>\linewidth\linewidth\else\Gin@nat@width\fi}
\def\maxheight{\ifdim\Gin@nat@height>\textheight\textheight\else\Gin@nat@height\fi}
\makeatother
% Scale images if necessary, so that they will not overflow the page
% margins by default, and it is still possible to overwrite the defaults
% using explicit options in \includegraphics[width, height, ...]{}
\setkeys{Gin}{width=\maxwidth,height=\maxheight,keepaspectratio}
% Set default figure placement to htbp
\makeatletter
\def\fps@figure{htbp}
\makeatother
% definitions for citeproc citations
\NewDocumentCommand\citeproctext{}{}
\NewDocumentCommand\citeproc{mm}{%
  \begingroup\def\citeproctext{#2}\cite{#1}\endgroup}
\makeatletter
 % allow citations to break across lines
 \let\@cite@ofmt\@firstofone
 % avoid brackets around text for \cite:
 \def\@biblabel#1{}
 \def\@cite#1#2{{#1\if@tempswa , #2\fi}}
\makeatother
\newlength{\cslhangindent}
\setlength{\cslhangindent}{1.5em}
\newlength{\csllabelwidth}
\setlength{\csllabelwidth}{3em}
\newenvironment{CSLReferences}[2] % #1 hanging-indent, #2 entry-spacing
 {\begin{list}{}{%
  \setlength{\itemindent}{0pt}
  \setlength{\leftmargin}{0pt}
  \setlength{\parsep}{0pt}
  % turn on hanging indent if param 1 is 1
  \ifodd #1
   \setlength{\leftmargin}{\cslhangindent}
   \setlength{\itemindent}{-1\cslhangindent}
  \fi
  % set entry spacing
  \setlength{\itemsep}{#2\baselineskip}}}
 {\end{list}}
\usepackage{calc}
\newcommand{\CSLBlock}[1]{\hfill\break\parbox[t]{\linewidth}{\strut\ignorespaces#1\strut}}
\newcommand{\CSLLeftMargin}[1]{\parbox[t]{\csllabelwidth}{\strut#1\strut}}
\newcommand{\CSLRightInline}[1]{\parbox[t]{\linewidth - \csllabelwidth}{\strut#1\strut}}
\newcommand{\CSLIndent}[1]{\hspace{\cslhangindent}#1}

\usepackage{booktabs}
\usepackage{longtable}
\usepackage{array}
\usepackage{multirow}
\usepackage{wrapfig}
\usepackage{float}
\usepackage{colortbl}
\usepackage{pdflscape}
\usepackage{tabu}
\usepackage{threeparttable}
\usepackage{threeparttablex}
\usepackage[normalem]{ulem}
\usepackage{makecell}
\usepackage{xcolor}
\input{/Users/joseph/GIT/latex/latex-for-quarto.tex}
\makeatletter
\@ifpackageloaded{caption}{}{\usepackage{caption}}
\AtBeginDocument{%
\ifdefined\contentsname
  \renewcommand*\contentsname{Table of contents}
\else
  \newcommand\contentsname{Table of contents}
\fi
\ifdefined\listfigurename
  \renewcommand*\listfigurename{List of Figures}
\else
  \newcommand\listfigurename{List of Figures}
\fi
\ifdefined\listtablename
  \renewcommand*\listtablename{List of Tables}
\else
  \newcommand\listtablename{List of Tables}
\fi
\ifdefined\figurename
  \renewcommand*\figurename{Figure}
\else
  \newcommand\figurename{Figure}
\fi
\ifdefined\tablename
  \renewcommand*\tablename{Table}
\else
  \newcommand\tablename{Table}
\fi
}
\@ifpackageloaded{float}{}{\usepackage{float}}
\floatstyle{ruled}
\@ifundefined{c@chapter}{\newfloat{codelisting}{h}{lop}}{\newfloat{codelisting}{h}{lop}[chapter]}
\floatname{codelisting}{Listing}
\newcommand*\listoflistings{\listof{codelisting}{List of Listings}}
\makeatother
\makeatletter
\makeatother
\makeatletter
\@ifpackageloaded{caption}{}{\usepackage{caption}}
\@ifpackageloaded{subcaption}{}{\usepackage{subcaption}}
\makeatother
\ifLuaTeX
  \usepackage{selnolig}  % disable illegal ligatures
\fi
\usepackage{bookmark}

\IfFileExists{xurl.sty}{\usepackage{xurl}}{} % add URL line breaks if available
\urlstyle{same} % disable monospaced font for URLs
\hypersetup{
  pdftitle={The Weirdest Causal Inferences: Why Comparative Cultural Research Requires a Causal Understanding of Measurement Error Bias},
  pdfauthor={Joseph A. Bulbulia},
  colorlinks=true,
  linkcolor={blue},
  filecolor={Maroon},
  citecolor={Blue},
  urlcolor={Blue},
  pdfcreator={LaTeX via pandoc}}

\title{The Weirdest Causal Inferences: Why Comparative Cultural Research
Requires a Causal Understanding of Measurement Error Bias}

\usepackage{academicons}
\usepackage{xcolor}

  \author{Joseph A. Bulbulia}
            \affil{%
             \small{     Victoria University of Wellington, New Zealand
          ORCID \textcolor[HTML]{A6CE39}{\aiOrcid} ~0000-0002-5861-2056 }
              }
      


\date{2024-06-09}
\begin{document}
\maketitle
\begin{abstract}
The human sciences, like all sciences, should seek generalisations where
generalisations may be found. For ethical and scientific reasons, it is
desirable to sample more broadly than `Western, Educated,
Industrialised, Rich, and Democratic' (WEIRD) societies. However,
restricting the target population is sometimes necessary; for example,
young children should not be recruited into studies on elderly care.
Under which conditions is unrestricted sampling desirable or
undesirable? Here, we use causal diagrams to clarify the structural
features of measurement error bias and target population restriction
bias in comparative cultural research. We define any study that exhibits
such biases, or other biases, as \textbf{weird} (\textbf{w}rongly
\textbf{e}stimated inferences due to \textbf{i}nappropriate
\textbf{r}estriction and \textbf{d}istortion). We explain why the first
step in comparative study design is to mitigate `weirdness.' We discuss
the fundamental challenges and explain how workflows for causal
inference provide the preflight checklists needed for ambitious,
effective, and safe comparative cultural research.

\textbf{KEYWORDS}: \emph{Causal Inference}; \emph{Comparative};
\emph{Cross-Cultural}; \emph{DAGs}; \emph{Experiments};
\emph{Longitudinal}; \emph{Measurement Error Bias\textbf{; }Selection
Bias}; \emph{Single World Intervention Graphs}, \emph{SWIGs},
\emph{Target Validity}; \emph{WEIRD}
\end{abstract}

\subsection{Introduction}\label{id-sec-intro}

Human scientists ask and answer questions. To anchor answers in facts,
we collect data.

Most publishing human scientists work in what Joseph Henrich, Steven
Heine, and Ara Norenzayan have termed `WEIRD' societies: `Western,
Educated, Industrialised, Rich, and Democratic Societies'
(\citeproc{ref-henrich2010weirdest}{Henrich \emph{et al.} 2010}).
Unsurprisingly, WEIRD samples are over-represented in human science
datasets (\citeproc{ref-arnett2008neglected}{Arnett 2008};
\citeproc{ref-sears1986college}{Sears 1986}). Henrich et al.~illustrate
how WEIRD samples differ from non-WEIRD samples in areas such as spatial
cognition and perceptions of fairness, while showing continuities in
basic emotion recognition, positive self-views, and motivation to punish
anti-social behaviour. Because science seeks generalisation wherever it
can, Henrich et al.~urge that sampling from non-WEIRD populations is
desirable.

Recently, a host of institutional diversity and inclusion initiatives
have been developed that commend researchers to obtain data from global
samples. In my view, the motivation for these mission statements is
ethically laudable. The injunction for a broader science of humanity
also accords with institutional missions. For example, the scientific
mission of the American Psychological Association (APA) is `to promote
the advancement, communication, and application of psychological science
and knowledge to benefit society and improve lives.' The APA does not
state that it wants to understand and benefit only North Atlantic
Societies (https://www.apa.org/pubs/authors/equity-diversity-inclusion,
accessed March 2024). It is therefore tempting to use such a mission
statement as an ideal by which to evaluate the samples used in human
scientific research.

https://www.apa.org/pubs/authors/equity-diversity-inclusion

Suppose we agree that promoting a globally diverse science makes ethical
sense. Does the sampling of globally diverse populations always advance
this ideal? It is easy to find examples in which restricting our source
population makes better scientific sense. Suppose we are interested in
the psychological effects of restorative justice among victims of
violent crime. Here, it would make little scientific sense to sample
from a population that has not experienced violent crime. Nor would it
make ethical sense. The scientific question, which may have important
ethical implications, is not served by casting a wider net. Suppose we
want to investigate the health effects of calorie restriction. It might
be unethical to include children or the elderly. It makes little sense
to investigate the psychological impact of vasectomy in biological
females or hysterectomy in biological males.

In the cases we just considered, the scientific questions pertained to a
sub-sample of the human population and so could be sensibly restricted
(refer also to Gaechter (\citeproc{ref-gachter2010}{2010}), Machery
(\citeproc{ref-machery2010}{2010})). However, even for questions that
relate to all of humanity, sampling from all of humanity might be
undesirable. For example, if we were interested in the effects of a
vaccine on disease, sampling from one population might be as good as
sampling from all. Sampling from one population might spare time and
expense, which come with opportunity costs. We might conclude that
sampling universally, where unnecessary, is wasteful and unethical.

We might agree with our mission statements in judging that ethical
aspirations must guide research at every phase. More fundamentally, we
cannot assess the bandwidth of human diversity from the armchair,
without empirical study, and this is a motivation to investigate. Yet,
mistaking our aspirations for sampling directives risks wasteful
science. Because waste carries opportunity costs, wasteful science is
unethical science.

I present these examples to remind ourselves of the importance of
addressing questions of sampling in relation the scientific question at
hand.

During the past twenty years, causal data science, also known as `causal
inference' or `CI', has enabled tremendous clarity for questions of
research design and analysis
(\citeproc{ref-richardson2014causal}{Richardson and Rotnitzky 2014}).
Here, we examine how the workflows developed for causal inference
clarify threats and opportunities for causal inference in comparative
human research. These workflows require that we state our causal
question in terms of well-defined counterfactual quantities, state the
population of interest, and evaluate assumptions under which it is
possible to obtain valid quantitative estimates of the counterfactual
quantities we seek from data. Application of these these workflows to
comparative questions enable us to clarify when comparative research is
possible, and also whether it is desirable. Not all questions are
causal, of course. However, because manifest associations in a dataset
may not be evidence of \emph{association} in the world, even those who
seek comparative descriptive understanding may benefit from causal
inference workflows (\citeproc{ref-vansteelandt2022a}{Vansteelandt and
Dukes 2022a}).

In the remainder of the introduction, I review causal directed acyclic
graphs (causal DAGs). Readers familiar with causal directed acyclic
graphs may skip this section. I encourage readers unfamiliar with causal
directed acyclic graphs to develop familiarity before proceeding:
(\citeproc{ref-barrett2021}{Barrett 2021};
\citeproc{ref-bulbulia2023}{Bulbulia 2023};
\citeproc{ref-hernan2024WHATIF}{Hernan and Robins 2024 Chapter 6};
\citeproc{ref-mcelreath2020}{McElreath 2020 Chapters 5, 6};
\citeproc{ref-neal2020introduction}{Neal 2020};
\citeproc{ref-pearl2009a}{Pearl 2009}). Because directed acyclic graphs
encode causal assumptions, we will use the terms `structural' and
`causal' synonymously.

\hyperref[id-sec-1]{Part 1} uses causal diagrams to clarify five
structural features of measurement-error bias. Understanding measurement
error bias is essential in all research, especially in comparative human
science, where it casts a long shadow.

\hyperref[id-sec-2]{Part 2} examines structural sources of bias arising
from attrition and non-response, also known as `right-censoring' or
simply `censoring'. Censoring may lead to restriction of the target
population at baseline. If the sample population at baseline is meant to
be the target population, censoring at baseline may lead to bias.

\hyperref[id-sec-3]{Part 3} addresses biases that arise at the start of
a study when there is a mismatch between the sample population and the
target population. When the target population is restricted in the
sample population at baseline, results may be biased. I focus on
structural threats to inference when the sample population is (1) too
restrictive (e.g., too WEIRD - Western, Educated, Industrialised, Rich,
and Democratic) and (2) insufficiently restrictive (leading to bias from
WEIRD sampling). We find that population-restriction biases are formally
equivalent to certain measurement error biases. This structural parallel
is crucial because it shows that many biases in comparative research can
be treated as measurement error biases. As these biases are
structural---causal in nature---they cannot be assessed using the
statistical estimation methods typically employed by comparative
researchers.

\hyperref[id-sec-4]{Part 4} uses Single World Intervention Graphs
(SWIGs) to enhance understanding of measurement-error bias, which is not
easily conveyed through causal directed acyclic graphs (DAGs). Causal
DAGs are designed to evaluate assumptions of `no unmeasured
confounding'. Consequently, they do not fully elucidate
population-restriction and measurement-error biases that do not stem
from confounding. Although SWIGs are also built to evaluate `no
unmeasured confounding', they represent counterfactual dependencies
directly on a graph. By placing the measurements---or `reporters'---of
latent realities we aim to quantify, along with the variables that
perturb these reporters so that the reported quantities differ from the
latent realities, we can advance the structural understanding of
measurement problems. This approach better diagnoses threats to
comparative human science and elucidates their remedies.

The importance of causal inference for comparative research has been
highlighted in several recent studies
(\citeproc{ref-bulbulia2022}{Bulbulia 2022};
\citeproc{ref-deffner2022}{Deffner \emph{et al.} 2022}). Here, I focus
on challenges arising from structural features of (1) measurement error
bias, (2) target population restriction bias from censoring, and (3)
target population restriction bias at a study's baseline. I clarify that
the basis of these biases is causal, not statistical, by demonstrating
their roots in measurement error bias. This understanding is essential
because comparative researchers often rely on statistical methods, such
as configurable scalar and metric invariance, to address measurement
issues. However, if the problems are causal, such methods are
inadequate. They fail to clarify the dependencies between reality, its
measurements, and the contextual and cultural features that modify the
effects of reality on its measurements
(\citeproc{ref-vanderweele2022}{VanderWeele 2022};
\citeproc{ref-vanderweele2022a}{VanderWeele and Vansteelandt 2022}).

I begin with a brief overview of causal inference, causal directed
acyclic graphs (causal DAGs), and our terminology.

\subsubsection{What is Causality?}\label{what-is-causality}

To quantify a causal effect, we must contrast the world as it is -- in
principle, observable -- with the world as it might have been -- in
principle, not observable.

Consider a binary treatment variable \(A \in \{0,1\}\) representing the
randomised administration of a vaccine to individuals \(i\) in the set
\(\{1, 2, \ldots, n\}\). \(A_i = 1\) denotes vaccine administration, and
\(A_i = 0\) denotes no vaccine. The potential outcomes for each
individual are \(Y_i(0)\) and \(Y_i(1)\), representing outcomes yet to
be realised before administration. Thus, they are called `potential' or
`counterfactual' outcomes. For an individual \(i\), we define a causal
effect as the contrast between the outcome observed under one
intervention level and the outcome observed under another. This
contrast, for the \(i^{th}\) individual, can be expressed on the
difference scale as:

\[
\text{Individual Treatment Effect} = Y_i(1) - Y_i(0)
\]

where the `Individual Treatment Effect' is the difference in the
outcomes for an individual under two treatment conditions, where
\(Y_i(1) - Y_i(0) \neq 0\) denotes a causal effect of \(A\) on \(Y\) for
individual \(i\) on the difference scale. Similarly,
\(\frac{Y_i(1)}{Y_i(0)} \neq 1\) denotes a causal effect of treatment
\(A\) for individual \(i\) on the risk ratio scale. These quantities
cannot be computed from observational data for any individual \(i\). The
inability to observe individual-level causal effects is the
\emph{Fundamental Problem of Causal Inference}
(\citeproc{ref-holland1986}{Holland 1986};
\citeproc{ref-rubin1976}{Rubin 1976}). This problem has long puzzled
philosophers (\citeproc{ref-hume1902}{Hume 1902};
\citeproc{ref-lewis1973}{Lewis 1973}). However, although individual
causal effects are generally unobservable, we can sometimes recover
average causal effects by treatment group.

\subsubsection{How We Obtain Average Causal Effect Estimates from
Ideally Conducted Randomised
Experiments}\label{how-we-obtain-average-causal-effect-estimates-from-ideally-conducted-randomised-experiments}

The Average Treatment Effect (ATE) measures the difference in outcomes
between treated and control groups as follows:

\[
\text{Average Treatment Effect} = \mathbb{E}[Y(1)] - \mathbb{E}[Y(0)]
\]

Here, \(\mathbb{E}[Y(1)]\) and \(\mathbb{E}[Y(0)]\) represent the
average outcome for the target population if \emph{everyone} in the
population were subjected to the treatment and control conditions,
respectively.

In a randomised experiment, we estimate these averages assuming that the
sample population matches the target population. We do this by
considering the average observed and unobserved outcomes under the
treatment conditions:

\[
\text{ATE} = \left(\mathbb{E}[Y(1) | A = 1] + \underbrace{\textcolor{red}{\mathbb{E}[Y(1) | A = 0]}}_{\text{unobserved}}\right) - \left(\mathbb{E}[Y(0) | A = 0] + \underbrace{\textcolor{red}{\mathbb{E}[Y(0) | A = 1]}}_{\text{unobserved}}\right)
\]

Effective randomisation ensures that potential outcomes are similarly
distributed across both groups. Thus, any differences in the averages of
the treatment groups can be attributed to the treatment. Therefore, in
an ideally conducted randomised experiment, the average outcomes are
expected to be equal across different treatment conditions for the
population from which the sample is drawn:

\[
\underbrace{\left[\widehat{\mathbb{E}}[Y(0) | A = 1]\right]_{\text{unobserved}} = \left[\widehat{\mathbb{E}}[Y(0) | A = 0]\right]_{\text{observed}}}_{\text{Under } A = 0}, \quad \underbrace{\left[\widehat{\mathbb{E}}[Y(1) | A = 1]\right]_{\text{observed}} = \left[\widehat{\mathbb{E}}[Y(1) | A = 0]\right]_{\text{unobserved}}}_{\text{Under } A = 1}
\]

Because treatment groups are exchangeable, an ideally randomised
controlled experiment provides an unbiased estimate of the Average
Treatment Effect:

\[
\widehat{\text{ATE}} = \widehat{\mathbb{E}}[Y | A = 1] - \widehat{\mathbb{E}}[Y | A = 0]
\]

Note that in the context of our imagined experiment,
\(\widehat{\text{ATE}}\) applies to the population from which the
experimental participants were drawn and is calculated on the difference
scale. A more explicit notation would define this effect estimate by
referencing its scale and population:
\(\widehat{\text{ATE}}^{a'-a}_{\text{S}}\), where \(a'-a\) denotes the
difference scale, and \(S\) denotes the source population. I will return
to this point in \hyperref[id-sec-2]{Part 2} and
\hyperref[id-sec-3]{Part 3}, but it is important to build intuition
early that in causal inference we must specify: (1) the causal effect of
interest; (2) a scale of contrast; and (3) a target population for whom
a causal effect estimate is meant to generalise.

\subsubsection{Three Fundamental Assumptions For Causal
Inference}\label{three-fundamental-assumptions-for-causal-inference}

An observational study aims to estimate the average treatment effects
without researchers controlling treatments or randomising treatment
assignments. We can consistently estimate counterfactual contrasts only
under strict assumptions. Three fundamental assumptions are required to
obtain the counterfactual quantities required to compute causal
contrasts from observational data.

\paragraph{Assumption 1. Causal
Consistency}\label{assumption-1.-causal-consistency}

Causal consistency states that the observed outcome for each individual
under the treatment they actually received is equal to their potential
outcome under that treatment. This means if an individual \(i\) received
treatment \(A_i = 1\), their observed outcome \(Y_i\) is the same as
their potential outcome under treatment, denoted as \(Y_i(1)\).
Similarly, if they did not receive the treatment (\(A_i = 0\)), their
observed outcome is the same as their potential outcome without
treatment, denoted as \(Y_i(0)\), such that:

\[
Y_i = A_i \cdot Y_i(1) + (1 - A_i) \cdot Y_i(0)
\]

where:

\begin{itemize}
\tightlist
\item
  \(Y_i\) is the observed outcome for individual \(i\);
\item
  \(A_i\) is the treatment status for individual \(i\), with \(A_i = 1\)
  indicating treatment received and \(A_i = 0\) indicating no treatment;
\item
  \(Y_i(1)\) and \(Y_i(0)\) are the potential outcomes for individual
  \(i\) under treatment and no treatment, respectively (refer to Morgan
  and Winship (\citeproc{ref-morgan2014}{2014}); VanderWeele
  (\citeproc{ref-vanderweele2009}{2009})).
\end{itemize}

The causal consistency assumption is necessary to link the theoretical
concept of potential outcomes --- the target quantities of interest ---
with observable data (see Bulbulia \emph{et al.}
(\citeproc{ref-bulbulia2023a}{2023})).

\paragraph{Assumption 2. Conditional Exchangeability (or
Ignorability)}\label{assumption-2.-conditional-exchangeability-or-ignorability}

Conditional exchangeability states that given a set of measured
covariates \(L\), the potential outcomes are independent of the
treatment assignment. Once we control for \(L\), the treatment
assignment \(A\) is as good as random with respect to the potential
outcomes:

\[
Y(a) \coprod A | L
\]

where:

\begin{itemize}
\tightlist
\item
  \(Y(a)\) represents the potential outcomes for a particular treatment
  level \(a\).
\item
  \(\coprod\) denotes conditional independence.
\item
  \(A\) represents the treatment levels to be contrasted.
\item
  \(L\) represents the measured covariates.
\end{itemize}

Under the conditional exchangeability assumption, any differences in
outcomes between treatment groups can be attributed to the treatment.
This assumption requires that all confounding variables affecting both
the treatment assignment \(A\) and the potential outcomes \(Y(a)\) are
measured and included in \(L\) (For further clarification, see
\hyperref[id-app-a]{Appendix A}).

\paragraph{Assumption 3. Positivity}\label{assumption-3.-positivity}

The positivity assumption requires that every individual in the
population has a non-zero probability of receiving each treatment level,
given their covariates. Formally,

\[
0 < Pr(A = a | L = l) < 1, \quad \forall a \in A, \, \forall l \in L \, \text{ such that } \, Pr(L = l) > 0
\]

where:

\begin{itemize}
\tightlist
\item
  \(A\) is the treatment or exposure variable.
\item
  \(L\) is a vector of covariates.
\item
  \(l\) represent specific values of treatment.
\end{itemize}

The positivity assumption, essential for valid treatment effect
estimates, requires that every individual has a non-zero chance of
receiving each treatment across all covariates in \(L\)
(\citeproc{ref-bulbulia2023a}{Bulbulia \emph{et al.} 2023};
\citeproc{ref-chatton2020}{Chatton \emph{et al.} 2020};
\citeproc{ref-hernan2024WHATIF}{Hernan and Robins 2024};
\citeproc{ref-westreich2010}{Westreich and Cole 2010}). In practice,
verifying this assumption faces two main challenges. The first challenge
is data sparsity, where certain covariate combinations are rare or
unobserved in the data, making it difficult to empirically confirm
positivity for these groups. The second challenge is model dependence,
as researchers rely on statistical models to estimate treatment
probabilities for all covariate patterns due to data sparsity. However,
these assessments are only as reliable as the models used, which may be
subject to misspecification or inaccuracies. For a discussion of causal
assumptions in relation to target validity, refer to Imai \emph{et al.}
(\citeproc{ref-imai2008misunderstandings}{2008}).

\subsubsection{Terminology}\label{terminology}

To avoid confusion, we define the meanings of our terms:

\begin{itemize}
\item
  \textbf{Unit/individual}: An entity, such as an object, person, or
  culture. We will use the term `individual' instead of the more general
  term `unit'. Think `row' in a dataset.
\item
  \textbf{Variable}: A feature of an individual, transient or permanent.
  For example, `Albert was sleepy but is no longer' or `Alice was born
  30 November'.
\item
  \textbf{Treatment}: Equivalent to `exposure', an event that might
  change a variable. For instance, `Albert was sleepy; we intervened
  with coffee; he is now wide awake' or `Alice was born in November;
  nothing can change that'. The `cause'.
\item
  \textbf{Outcome}: The response variable or `effect'. In causal
  inference, we contrast `potential' or `counterfactual outcomes'. In
  observational or `real-world' studies where treatments are not
  randomised, the assumptions for obtaining contrasts of counterfactual
  outcomes are typically much stronger than in randomised controlled
  experiments.
\item
  \textbf{Confounding}: A state where the treatment and outcome share a
  common cause and no adjustment is made to remove the non-causal
  association, or where the treatment and outcome share a common effect,
  and adjustment is made for this common effect, or when the effect of
  the treatment on the outcome is mediated by a variable which is
  conditioned upon. In each case, the observed association will not
  reflect a causal association. Causal directed acyclic graphs clarify
  strategies for confounding control.
\item
  \textbf{Measurement}: A recorded trace of a variable, such as a column
  in a dataset. When placing measurements within causal settings, we
  call measurements \textbf{Reporters}.
\item
  \textbf{Measurement error}: A misalignment between the true state of a
  variable and its recorded state. For example, `Alice was born on 30
  November; records were lost, and her birthday was recorded as 1
  December'.
\item
  \textbf{Population}: An abstraction from statistics, denoting the set
  of all individuals defined by certain features. John belongs to the
  set of all individuals who ignore instructions.
\item
  \textbf{Super-population}: An abstraction, the population of all
  possible individuals of a given kind. John and Alice belong to a
  super-population of hominins.
\item
  \textbf{Restricted population}: Population \(p\) is restricted
  relative to another population \(P\) if the individuals \(p \in P\)
  share some but not all features of \(P\). `The living' is a
  restriction of hominins.
\item
  \textbf{Target population}: A restriction of the super-population
  whose features interest investigators. An investigator who defines
  their interests is a member of the population of `good investigators'.
\item
  \textbf{Source population}: The population from which the study's
  sample is drawn. Investigators wanted to recruit from a general
  population but recruited from the pool of first-year university
  psychology students.
\item
  \textbf{Sample population at baseline}: or equivalently the `analytic
  population.' The abstract set of individuals from which the units in a
  study at treatment assignment belong, e.g., `the set of all first-year
  university psychology students who might end up in this study'. Unless
  stated otherwise, we will consider the baseline sample population to
  be representative of the \emph{source population}; we will consider
  the \emph{source population} at baseline to be representative of the
  \emph{target population}.
\item
  \textbf{Selection into the sample}: Selection occurs and is under
  investigator control when a target population is defined from a
  super-population or when investigators apply eligibility criteria for
  inclusion in the analytic sample. Selection into the sample is often
  out of the investigator's control. Investigators might aspire to
  answer questions about all of humanity but find themselves limited to
  undergraduate samples. Investigators might sample from a source
  population but recover an analytic sample that differs from it in ways
  they cannot measure, such as mistrust of scientists. There is
  typically attrition of an analytic sample over time, and this is not
  typically fully within investigator control. Because the term
  `selection' has different meanings in different areas of human
  science, we will speak of `target population restriction at the start
  of study'. Note that to evaluate this bias, it is important for
  investigators to state a causal effect of interest with respect to
  \emph{the full data} that includes the counterfactual quantities for
  the treatments to be compared in a clearly defined target population
  where all members of the target population are exposed to each level
  of treatment to be contrasted (\citeproc{ref-westreich2017}{Westreich
  \emph{et al.} 2017a}).
\item
  \textbf{Censored sample population at the end of study}: The
  population from which the censored units are drawn. Censoring is
  uninformative if there is no treatment effect for everyone in the
  baseline population (the sharp causal null hypothesis). Censoring is
  informative if there is an effect of the treatment, and this effect
  varies in at least one stratum of the baseline population. If no
  correction is applied, unbiased effect estimates for the analytic
  sample will bias causal effect estimates for the target population
  (\citeproc{ref-greenland2009commentary}{Greenland 2009};
  \citeproc{ref-lash2020}{Lash \emph{et al.} 2020};
  \citeproc{ref-vanderweele2012}{VanderWeele 2012}). We call such bias
  from right censoring `target population restriction at the end of
  study'. Note again that to evaluate this bias, the causal effect of
  interest must be stated with respect to \emph{the full data} that
  includes the counterfactual quantities for the treatments to be
  compared in a clearly defined target population where all members of
  the target population are exposed to each level of treatment to be
  contrasted (\citeproc{ref-westreich2017}{Westreich \emph{et al.}
  2017a}).
\item
  \textbf{Target population restriction bias}: Bias occurs when the
  distribution of effect modifiers in the sample population differs from
  that in the target population, either at the start, at the end, or
  throughout the study. Here we consider: \emph{target population
  restriction bias at the start of study} and \emph{target population
  restriction bias at the end of study}. If this bias occurs at the
  start of the study, it will generally occur at the end of the study
  (and at intervals between), except by accident. We require validity to
  be non-accidental.
\item
  \textbf{Generalisability}: A study's findings generalise to a target
  population if the effects observed in the study group at the end of
  study are also valid for the target population for structurally valid
  reasons (i.e., non-accidentally), see \hyperref[id-app-b]{Appendix B}.
\item
  \textbf{Transportability}: When the study sample is not drawn from the
  target population, we cannot directly generalise the findings.
  However, we can transport the estimated causal effect from the source
  population to the target population under certain assumptions. This
  involves adjusting for differences in the distributions of effect
  modifiers between the two populations. The closer the source
  population is to the target population, the more plausible the
  transportability assumptions and the less we need to rely on complex
  adjustment methods see \hyperref[id-app-b]{Appendix B}.
\item
  \textbf{Marginal effect}: Typically a synonym for the average
  treatment effect --- always relative to some population specified by
  investigators.
\item
  \textbf{Intention-to-treat effect}: The marginal effect of random
  treatment assignment.
\item
  \textbf{Per-protocol effect}: The effect of adherence to a randomly
  assigned treatment if adherence were perfect
  (\citeproc{ref-hernan2017per}{Hernán \emph{et al.} 2017}). We have no
  guarantee that the intention-to-treat effect will be the same as the
  per-protocol effect. A safe assumption is that: \[
  \widehat{ATE}_{\text{target}}^{\text{Per-Protocol}} \ne \widehat{ATE}_{\text{target}}^{\text{Intention-to-Treat}}
  \]
\end{itemize}

When evaluating evidence for causality, in addition to specifying their
causal contrast, effect measure, and target population, investigators
should specify whether they are estimating an intention-to-treat or
per-protocol effect (\citeproc{ref-hernuxe1n2004}{Hernán 2004};
\citeproc{ref-tripepi2007}{Tripepi \emph{et al.} 2007}).

\begin{itemize}
\item
  \textbf{WEIRD}: A sample of `Western, Educated, Industrialised, Rich,
  and Democratic Societies' (\citeproc{ref-henrich2010weirdest}{Henrich
  \emph{et al.} 2010}).
\item
  \textbf{weird}: (\textbf{w}rongly \textbf{e}stimated inferences due to
  \textbf{i}nappropriate \textbf{r}estriction and \textbf{d}istortion) A
  causal effect estimate that is not valid for the target population,
  either from confounding bias, measurement error bias, target
  population restriction at the start of study, or target population
  restriction at the end of study.
\end{itemize}

Note that our terminology differs in causal inference for the concepts
we have defined here (refer to Dahabreh \emph{et al.}
(\citeproc{ref-dahabreh2021study}{2021}); Imai \emph{et al.}
(\citeproc{ref-imai2008misunderstandings}{2008}); Cole and Stuart
(\citeproc{ref-cole2010generalizing}{2010}); Westreich \emph{et al.}
(\citeproc{ref-westreich2017transportability}{2017b})). A clear
decomposition of key concepts needed to assess generalisability --- or
what we call `target validity' --- is given in Imai \emph{et al.}
(\citeproc{ref-imai2008misunderstandings}{2008}). For a less technical,
pragmatically useful discussion, refer to Stuart \emph{et al.}
(\citeproc{ref-stuart2018generalizability}{2018}).

\subsubsection{Graphical Conventions}\label{graphical-conventions}

\begin{itemize}
\item
  \textbf{\(A\)}: Denotes the `treatment' or `exposure' --- a random
  variable, `the cause'.
\item
  \textbf{\(Y\)}: Denotes the outcome or response, measured at the end
  of the study. \(Y\) is the `effect'.
\item
  \textbf{\(L\)}: Denotes a measured confounder or set of confounders.
\item
  \textbf{\(U\)}: Denotes an unmeasured confounder or confounders.
\item
  \textbf{\(\mathbf{\mathcal{R}}\)}: Denotes randomisation to treatment
  condition \(\big(\mathcal{R} \rightarrow A\big)\).
\item
  \textbf{Node}: Represents characteristics or features of units within
  a population on a causal diagram --- that is, a `variable'. In causal
  directed acyclic graphs, nodes are drawn with respect to the
  \emph{target population}, which is the population for whom
  investigators seek causal inferences (\citeproc{ref-suzuki2020}{Suzuki
  \emph{et al.} 2020}). Time-indexed nodes: \(X_t\) denotes relative
  chronology.
\item
  \textbf{Edge without an Arrow} (\(\association\)): Path of
  association, causality not asserted.
\item
  \textbf{Red Edge without an Arrow} (\(\associationred\)): Confounding
  path, ignoring arrows to clarify statistical dependencies.
\item
  \textbf{Arrow} (\(\rightarrow\)): Denotes a causal relationship from
  the node at the base of the arrow (a `parent') to the node at the tip
  of the arrow (a `child'). In causal directed acyclic graphs, it is
  conventional to refrain from drawing an arrow from treatment to
  outcome to avoid asserting a causal path from \(A\) to \(Y\) because
  we aim to ascertain whether causality can be identified for this path.
  All other nodes and paths --- including the absence of nodes and paths
  --- are typically assumed.
\item
  \textbf{Red Arrow} (\(\rightarrowred\)): Denotes a path of non-causal
  association between the treatment and outcome. Despite the arrows,
  this path is associational and may flow against time.
\item
  \textbf{Open Blue Arrow} (\(\rightarrowblue\)): Denotes effect
  modification, which occurs when the effect of treatment varies within
  levels of a covariate. We do not assess the causal effect of the
  effect modifier on the outcome, recognising that it may be incoherent
  to consider intervening on the effect modifier. However, if the
  distribution of effect modifiers in the sample population differs from
  that in the target population, then at least one measure of causal
  effect will differ between the two populations.
\item
  \textbf{Boxed Variable} \(\big(\boxed{X}\big)\): Denotes conditioning
  or adjustment for \(X\).
\item
  \textbf{Red-Boxed Variable} \(\big(\boxedred{X}\big)\): Highlights the
  source of confounding bias from adjustment.
\item
  \textbf{Dashed Circle} \(\big(\circledotted{X}\big)\): Denotes no
  adjustment is made for a variable (implied for unmeasured
  confounders).
\item
  \textbf{\(\mathcal{G}\)}: Names a causal diagram.
\item
  \textbf{Split Node (SWIGs)} \(\nodesplit\): Convention used in Single
  World Intervention Graphs (SWIGs) that allows for factorisation of
  counterfactuals by splitting a node at intervention with
  post-intervention nodes relabeled to match the treatment. We introduce
  Single World Intervention Graphs in \hyperref[id-sec-4]{Part 4}.
\item
  \textbf{Unobserved Node (SWIGs)} \(\swiglatent\): Our convention when
  using Single World Intervention Graphs to denote an unobserved node
  (SWIGs): \(X\) unmeasured.
\end{itemize}

\subsubsection{Causal Directed Acyclic Graphs
(DAGs)}\label{causal-directed-acyclic-graphs-dags}

Judea Pearl proved that, based on assumptions about causal structure,
researchers can identify causal effects from joint distributions of
observed data (\citeproc{ref-pearl1995}{Pearl 1995},
\citeproc{ref-pearl2009a}{2009}). The rules of d-separation are given in
Table~\ref{tbl-terminologygeneral}.

\begin{table}

\caption{\label{tbl-terminologygeneral}Elements of Causal Graphs}

\centering{

\terminologydirectedgraph

}

\end{table}%

Pearl's rules of d-separation are as follows:

\begin{itemize}
\tightlist
\item
  \textbf{Fork rule} (\(B \leftarrowNEW \boxed{A} \rightarrowNEW C\)):
  \(B\) and \(C\) are independent when conditioned on \(A\)
  (\(B \coprod C \mid A\)).
\item
  \textbf{Chain rule} (\(A \rightarrowNEW \boxed{B} \rightarrowNEW C\)):
  Conditioning on \(B\) blocks the path between \(A\) and \(C\)
  (\(A \coprod C \mid B\)).
\item
  \textbf{Collider rule}
  (\(A \rightarrowNEW \boxed{C} \leftarrowNEW B\)): \(A\) and \(B\) are
  independent until conditioned on \(C\), which introduces dependence
  (\(A \cancel{\coprod} B \mid C\)).
\end{itemize}

Table~\ref{tbl-terminologygeneral} shows causal directed acyclic graphs
corresponding to these rules. Because all causal relationships can be
assembled from combinations of the five structures presented in
Table~\ref{tbl-terminologygeneral}, we can use causal graphs to evaluate
whether and how causal effects may be identified from data
(\citeproc{ref-bulbulia2023}{Bulbulia 2023}). Pearl's general
identification algorithm is known as the `back door adjustment theorem'
(\citeproc{ref-pearl2009a}{Pearl 2009}).

\subsubsection{Effect-Modification on Causal Directed Acyclic
Graphs}\label{effect-modification-on-causal-directed-acyclic-graphs}

The primary function of a causal directed acyclic graph is to apply the
Pearl's backdoor adjustment theorem to establish relations of
conditional independence Table~\ref{tbl-terminologygeneral} for the
purposes of causal identification. We have noted that modifying a causal
effect within one or more strata of the target population opens the
possibility for biased average treatment effect estimates when the
distribution of these effect modifiers differs in the sample population.

We do not generally represent non-linearities in causal directed acyclic
graphs, which are tools for obtaining relationships of conditional and
unconditional independence from assumed structural relationships encoded
in a causal diagram that may lead to a non-causal treatment/outcome
association.

Table~\ref{tbl-terminologygeneral} presents our convention for
highlighting a relationship of effect modification in settings where (1)
we assume no confounding of treatment and outcome and (2) there is
effect modification such that the effect of \(A\) on \(Y\) differs in at
least one stratum of the target population.

\begin{table}

\caption{\label{tbl-terminologygeneral}Elements of Causal Graphs}

\centering{

\terminologyeffectmodification

}

\end{table}%

To focus on effect modification, we do not draw a causal arrow from the
direct effect modifier \(F\) to the outcome \(Y\). This convention is
specific to this article (refer to Hernan and Robins
(\citeproc{ref-hernan2024WHATIF}{2024}), pp.~126-127, for a discussion
of `non-causal' arrows).

\subsection{\texorpdfstring{Part 1: How Measurement Error Bias Makes
Your Causal Inferences \textbf{weird} (\textbf{w}rongly
\textbf{e}stimated inferences due to \textbf{i}nappropriate
\textbf{r}estriction and
\textbf{d}istortion)}{Part 1: How Measurement Error Bias Makes Your Causal Inferences weird (wrongly estimated inferences due to inappropriate restriction and distortion)}}\label{id-sec-1}

Measurements record reality, but they are not always accurate. Whenever
variables are measured with error, our results can be misleading. Every
study must therefore consider how its measurements might mislead. Causal
directed acyclic graphs can deepen understanding because -- as implied
by the concept of `record' ---there are structural or causal properties
of measurement error. Understanding these properties can greatly assist
with study design, data collection, data analysis, and inference.

Measurement error can take various forms, each with distinct
implications for causal inference. Causal diagrams clarify these types
of measurement error bias:

\begin{itemize}
\tightlist
\item
  \textbf{Independent (undirected) /uncorrelated}: Errors in different
  variables do not influence each other.
\item
  \textbf{Independent (undirected) and correlated}: Errors in different
  variables are related through a shared cause.
\item
  \textbf{Dependent (directed) and uncorrelated}: Errors in one variable
  influence the measurement of another, but these influences are not
  related through a shared cause.
\item
  \textbf{Dependent (directed) and correlated}: Errors in one variable
  influence the measurement of another, and these influences are related
  through a shared cause (\citeproc{ref-hernuxe1n2009}{Hernán and Cole
  2009}; \citeproc{ref-vanderweele2012a}{VanderWeele and Hernán 2012}).
\end{itemize}

The six causal diagrams presented in
Table~\ref{tbl-terminologymeasurementerror} illustrate structural
features of measurement error bias and clarify how these structural
features compromise causal inferences.

\begin{table}

\caption{\label{tbl-terminologymeasurementerror}Example of measurement
error bias}

\centering{

\terminologymeasurementerror

}

\end{table}%

Understanding structural features of measurement error bias will help
use to understand why measurement error bias cannot typically be
evaluated with statistical models, and will prepare us to understand how
target-population restriction biases are linked to measurement error.

\subsubsection{Example 1: Uncorrelated Non-Differential Errors under
Sharp Null: No Treatment
Effect}\label{example-1-uncorrelated-non-differential-errors-under-sharp-null-no-treatment-effect}

Table~\ref{tbl-terminologymeasurementerror} \(\mathcal{G}_1\)
illustrates uncorrelated non-differential measurement error under the
`sharp-null,' which arises when the error terms in the exposure and
outcome are independent. In this setting, the measurement error
structure is not expected to produce bias.

For example, consider a study investigating the causal effect of beliefs
in big Gods on social complexity in ancient societies. Imagine that
societies either randomly omitted or inaccurately recorded details about
their beliefs in big Gods and their social complexities. This might
occur because of varying preservation in the records of cultures which
is unrelated to the actual beliefs or social complexity. In this
scenario, we imagine the errors in historical records for beliefs in big
Gods and for social complexity are independent. When the treatment is
randomised, uncorrelated and undirected errors will generally not
introduce bias \emph{under the sharp null of no treatment effect for any
unit}. In this extreme condition, we will not generally find evidence
for effects in the absence of true effects. However, if we knew whether
the sharp null held, we would not require science. Therefore, we set
this structure aside as of theoretical but not practical interest.

\subsubsection{Example 2: Uncorrelated Non-Differential Errors `Off The
Null' (True Treatment Effect) Biases True Effects toward the
Null}\label{example-2-uncorrelated-non-differential-errors-off-the-null-true-treatment-effect-biases-true-effects-toward-the-null}

Table~\ref{tbl-terminologymeasurementerror} \(\mathcal{G}_2\)
illustrates uncorrelated non-differential measurement error, which
arises when the error terms in the exposure and outcome are independent
This bias is also called information bias
(\citeproc{ref-lash2009applying}{Lash \emph{et al.} 2009}). In this
setting, bias will often attenuate a true treatment effect; however,
there are no gaurantees that bias will be toward the null
(\citeproc{ref-jurek2005proper}{Jurek \emph{et al.} 2005},
\citeproc{ref-jurek2008brief}{2008};
\citeproc{ref-jurek2006exposure}{Jurek \emph{et al.} 2006};
\citeproc{ref-lash2009applying}{Lash \emph{et al.} 2009 p. 93}).
Moreover, Robins \emph{et al.} (\citeproc{ref-robins2004effects}{2004})
(p.2216) discusses how in non-experimental settings, mismeasured
confounders can introduce bias even when the measurement errors of the
treatment and outcome are uncorrelated and undirected and there is no
treatment effect. This is because mismeasured confounders will not
control for confounding bias. We present an illustration of this bias in
Table~\ref{tbl-terminologyselectionrestrictionbaseline}
\(\mathcal{G}_6\), where we discuss challenges to comparative research
in which the accuracy of confounder measurements varies across the sites
to be compared.

Consider again the example of a study investigating a causal effect of
beliefs in big Gods on social complexity in ancient societies, where
there are uncorrelated errors in the treatment and outcome. In this
case, measurement error will often make it seem that the true causal
effects of beliefs in big Gods are smaller than they are, or perhaps
even that such an effect is absent. However, to repeat, attenuation is
not guaranteed.

Uncorrelated undirected measurement error in the presence of a true
effect leads to distortion of true causal effects, inviting
\textbf{weird} results (\textbf{w}rongly \textbf{e}stimated inferences
due to \textbf{i}nappropriate \textbf{r}estriction and
\textbf{d}istortion).

\subsubsection{Example 3: Correlated Errors Non-Differential
(Undirected) Measurement
Errors}\label{example-3-correlated-errors-non-differential-undirected-measurement-errors}

Table~\ref{tbl-terminologymeasurementerror} \(\mathcal{G}_3\)
illustrates the structure of correlated non-differential (undirected)
measurement error bias, which arises when the error terms of the
treatment and outcome share a common cause.

Consider an example: imagine that societies with more sophisticated
record-keeping systems tend to offer more precise and comprehensive
records of both beliefs in big Gods and of social complexity. In this
setting, it is the record-keeping systems that give the illusion of a
relationship between big Gods and social complexity. This might occur
without any causal effect of big-God beliefs on measuring social
complexity or vice versa. Nevertheless, the correlated sources of error
for both the exposure and outcome might suggest causation in its
absence.

Correlated non-differential measurement error invites \textbf{weird}
results (\textbf{w}rongly \textbf{e}stimated inferences due to
\textbf{i}nappropriate \textbf{r}estriction and \textbf{d}istortion).

\subsubsection{Example 4: Uncorrelated Differential Measurement Error:
Exposure Affects Error of
Outcome}\label{example-4-uncorrelated-differential-measurement-error-exposure-affects-error-of-outcome}

Table~\ref{tbl-terminologymeasurementerror} \(\mathcal{G}_4\)
illustrates the structure of uncorrelated differential (or directed)
measurement error, where a non-causal path is opened linking the
treatment, the outcome, or a common cause of the treatment and outcome.

Continuing with our previous example, imagine that beliefs in big Gods
lead to inflated records of social complexity in a culture's
record-keeping. This might happen because the record keepers in
societies that believe in big Gods prefer societies to reflect the
grandeur of their big Gods. Suppose further that cultures lacking
beliefs in big Gods prefer Bacchanalian-style feasting to
record-keeping. In this scenario, societies with record keepers who
believe in big Gods would appear to have more social complexity than
equally complex societies without such record keepers.

Uncorrelated directed measurement error bias also invites \textbf{weird}
results (\textbf{w}rongly \textbf{e}stimated inferences due to
\textbf{i}nappropriate \textbf{r}estriction and \textbf{d}istortion).

\subsubsection{Example 5: Uncorrelated Differential Measurement Error:
Outcome Affects Error of
Exposure}\label{example-5-uncorrelated-differential-measurement-error-outcome-affects-error-of-exposure}

Table~\ref{tbl-terminologymeasurementerror} \(\mathcal{G}_5\)
illustrates the structure of uncorrelated differential (or directed)
measurement error, this time when the outcome affects the recording of
the treatment that preceded the outcome.

Suppose that `history is written by the victors.' Can we give a
structural account of measurement error bias arising from such a
selective retention of the past. Suppose that social complexity causes
beliefs in big Gods. Perhaps kings create big Gods after the image of
kings. If the kings prefer a history in which big Gods were historically
present, this might bias the historical record, opening a path of
association that reverses the order of causation. Such results would be
\textbf{weird}: (\textbf{w}rongly \textbf{e}stimated inferences due to
\textbf{i}nappropriate \textbf{r}estriction and \textbf{d}istortion).

\subsubsection{Example 6: Uncorrelated Differential Error: Outcome
Affects Error of
Exposure}\label{example-6-uncorrelated-differential-error-outcome-affects-error-of-exposure}

Table~\ref{tbl-terminologymeasurementerror} \(\mathcal{G}_6\)
illustrates the structure of correlated differential (directed)
measurement error, which occurs when the exposure affects levels of
already correlated error terms.

Suppose social complexity produces a flattering class of religious
elites who produce vainglorious depictions of kings and their dominions,
and also of the extent and scope of their society's beliefs in big Gods.
For example, such elites might downplay widespread cultural practices of
worshipping lesser gods, inflate population estimates, and overstate the
range of the kings political economy. In this scenario, the errors of
the exposure and of the outcome are both correlated and differential.

Results based on such measures might be \textbf{weird}:
(\textbf{w}rongly \textbf{e}stimated inferences due to
\textbf{i}nappropriate \textbf{r}estriction and \textbf{d}istortion).

\subsubsection{Summary}\label{summary}

In Part 1, we examined four types of measurement error bias:
independent, correlated, dependent, and correlated dependent. The
structural features of measurement error bias clarify how measurement
errors threaten causal inferences. Considerably more could be said about
this topic. For example, VanderWeele and Hernán
(\citeproc{ref-vanderweele2012a}{2012}) demonstrate that, under specific
conditions, we can infer the direction of a causal effect from observed
associations. Specifically, if:

\begin{enumerate}
\def\labelenumi{\arabic{enumi}.}
\tightlist
\item
  The association between the measured variables \(A^{\prime}_{1}\) and
  \(Y^{\prime}_{2}\) is positive,
\item
  The measurement errors for these variables are not correlated, and
\item
  We assume distributional monotonicity for the effect of \(A\) on \(Y\)
  (applicable when both are binary),
\end{enumerate}

then a positive observed association implies a positive causal effect
from \(A\) to \(Y\). Conversely, a negative observed association
provides stronger evidence for a negative causal effect if the error
terms are positively correlated than if they are independent. This
conclusion relies on the assumption of distributional monotonicity for
the effect of \(A\) on \(Y\). For now, the four elementary structures of
measurement error bias will enable us to clarify the connections between
the structures of measurement error bias, target population restriction
bias at the end of a study, and target-restriction bias at the start of
a study.

We will return to measurement error again in \hyperref[id-sec-4]{Part
4}. Next, we focus on structural features of bias when there is an
inappropriate restriction of the target population in the analytic
sample at baseline.

\subsection{\texorpdfstring{Part 2: How Target Population Restriction
Bias At The End of Study Makes Your Causal Inferences weird
(\textbf{w}rongly \textbf{e}stimated inferences due to
\textbf{i}nappropriate \textbf{r}estriction and
\textbf{d}istortion)}{Part 2: How Target Population Restriction Bias At The End of Study Makes Your Causal Inferences weird (wrongly estimated inferences due to inappropriate restriction and distortion)}}\label{id-sec-2}

Suppose the sample population at the start of a study matches the source
population from which it is drawn and that this source population aligns
with the target population. Right-censoring, typically abbreviated to
`censoring' and also known as `attrition and non-response', may bias
causal effect estimates in two ways: by opening pathways of confounding
bias (distortion) or by inappropriately restricting the sample
population at the end of a study, so it is no longer aligned with the
target population as it was at the start of the study. We next consider
how censoring can make a study \textbf{weird}: (\textbf{w}rongly
\textbf{e}stimated inferences due to \textbf{i}nappropriate
\textbf{r}estriction and \textbf{d}istortion).

\begin{table}

\caption{\label{tbl-terminologycensoring}Five examples of censoring
bias.}

\centering{

\terminologycensoring

}

\end{table}%

\subsubsection{Example 1: Confounding by common cause of treatment and
attrition}\label{example-1-confounding-by-common-cause-of-treatment-and-attrition}

Table~\ref{tbl-terminologycensoring} \(\mathcal{G}_1\) illustrates
confounding by common cause of treatment and outcome in the censored
such that the potential outcomes of the population at baseline \(Y(a)\)
may differ from those of the censored population at the end of study
\(Y'(a)\) such that \(Y'(a) \neq Y(a)\).

Suppose investigators are interested in whether religious service
attendance affects volunteering. Suppose that an unmeasured variable,
loyalty, affects religious service attendance, attrition, and
volunteering. The structure of this bias reveals an open backdoor path
from the treatment to the outcome.

We have encountered this bias before. The structure we observe here is
one of correlated measurement errors
(Table~\ref{tbl-terminologymeasurementerror} \(\mathcal{G}_3\)). In this
example, attrition may exacerbate measurement error bias by opening a
path from
\(A \associationred U \associationred U_{\Delta{A}}  \associationred Y'\)

The results obtained from such a study would be distorted -- that is,
\textbf{weird}: (\textbf{w}rongly \textbf{e}stimated inferences due to
\textbf{i}nappropriate \textbf{r}estriction and \textbf{d}istortion).
Here, distortion operates through the restriction of the target
population in the sample population at the end of the study.

\subsubsection{Example 2: Treatment affects
censoring}\label{example-2-treatment-affects-censoring}

Table~\ref{tbl-terminologycensoring} \(\mathcal{G}_2\) illustrates
confounding bias in which the treatment affects the censoring process.

Consider a study investigating the effects of mediation on well-being.
Suppose there is no treatment effect but that Buddha-like detachment
increases attrition. If \(\mathcal{G}_4\) faithfully represents reality,
there will be no bias in the treatment effect estimate. That is, there
will be no risk that attrition will induce the appearance of a causal
effect in its absence. The biasing path runs:
\(A \associationred U_{\Delta{A\to Y}}  \associationred Y'\).

We have encountered this structural bias before. The structure we
observe here is one of directed uncorrelated measurement error
(Table~\ref{tbl-terminologymeasurementerror} \(\mathcal{G}_4\)).
Randomisation ensures no backdoor paths. However, if the intervention
affects both attrition and bias in the outcome we may expect measurement
error bias.

The results obtained from this study risk distortation and as such
invite \textbf{weirdness}: (\textbf{w}rongly \textbf{e}stimated
inferences due to \textbf{i}nappropriate \textbf{r}estriction and
\textbf{d}istortion). Here, distortion operates through the restriction
of the target population at the the end of the study, assuming the
analytic sample at the start of study represented that target population
(or was weighted to represent it.)

\subsubsection{Example 3: No treatment effect when outcome causing
censoring}\label{example-3-no-treatment-effect-when-outcome-causing-censoring}

Table~\ref{tbl-terminologycensoring} \(\mathcal{G}_3\) illustrates the
structure of bias when there is no treatment effect yet the outcome
affects censoring.

If \(\mathcal{G}_3\) faithfully represents reality, under the sharp null
we would generally not expect bias in the average treatment effect
estimate from attrition. The structure we observe here is again
familiar: it is one of undirected uncorrelated measurement
error(Table~\ref{tbl-terminologymeasurementerror} \(\mathcal{G}_1\)).
However, as before, it is generally never clear whether the sharp null
holds. In theory, however, although the sample population would be
restricted, such restriction of the target population should not bias
the null result. Again, this example is theoretical; no statistical test
could validate what amounts to a structural assumption of the sharp
null.

\subsubsection{Example 4: Treatment effect when outcome causes censoring
and there is a true treatment
effect}\label{example-4-treatment-effect-when-outcome-causes-censoring-and-there-is-a-true-treatment-effect}

Table~\ref{tbl-terminologycensoring} \(\mathcal{G}_4\) illustrates the
structure of bias when the outcome affects censoring in the presence of
a treatment effect. If the true outcome is an effect modifier of the
measured outcome, we can expect bias in at least one measure of effect
(e.g., the risk ratio or the causal difference scale). We return to this
form of bias with a worked example in \hyperref[id-sec-4]{Part 4}, where
we clarify how such bias may arise even in the absence of confounding.
We shall see that the bias described in
Table~\ref{tbl-terminologycensoring} \(\mathcal{G}_4\) is equivalent to
measurement error bias.

Nevertheless, where the treatment and the outcome share the same sign
and if we may assume distributional monotonicity for the effect of \(A\)
on \(Y\), we would infer that the bias of censoring for causal inference
in the target population in this setting would typically be reduced.
Strictly speaking, however, results would be \textbf{weird}:
(\textbf{w}rongly \textbf{e}stimated inferences due to
\textbf{i}nappropriate \textbf{r}estriction and \textbf{d}istortion).

\subsubsection{Example 5: Treatment effect and effect-modifiers differ
in censored (restriction bias without
confounding)}\label{example-5-treatment-effect-and-effect-modifiers-differ-in-censored-restriction-bias-without-confounding}

Table~\ref{tbl-terminologycensoring} \(\mathcal{G}_5\) represents a
setting in which there is a true treatment effect, but the distribution
of effect-modifiers -- variables that interact with the treatment --
differ among the sample at baseline and the sample at the end of study.
Knowing nothing else, we might expect this setting to be standard. Where
measured variables are sufficient to predict attrition, that is, where
missingness is at random, we can obtain valid estimates for a treatment
effect by inverse probability of treatment weighting
(\citeproc{ref-cole2008}{Cole and Hernán 2008};
\citeproc{ref-leyrat2021}{Leyrat \emph{et al.} 2021}) or by multiple
imputation -- on the assumption that our models are correctly specified
(\citeproc{ref-shiba2021}{Shiba and Kawahara 2021}). However, if
missingness is not completely at random, or if our models are otherwise
misspecified, then unbiased causal estimation is compromised
(\citeproc{ref-malinsky2022semiparametric}{Malinsky \emph{et al.} 2022};
\citeproc{ref-tchetgen2017general}{Tchetgen Tchetgen and Wirth 2017}).

Note that Table~\ref{tbl-terminologycensoring} \(\mathcal{G}_5\) closely
resembles a measurement structure we have considered before, in
\textbf{Part 1:} Table~\ref{tbl-terminologymeasurementerror}
\(\mathcal{G}_2\) Replacing the unmeasured effect modifiers
\(\circledotted{F}\) and \(U_{\Delta F}\) in
Table~\ref{tbl-terminologycensoring} \(\mathcal{G}_5\) for
\(\circledotted{U_Y}\) in Table~\ref{tbl-terminologymeasurementerror}
\(\mathcal{G}_2\) reveals that the unmeasured effect modification in the
present setting can be viewed as an example of uncorrelated independent
measurment error when there is a treatment effect (i.e.~off the null.)

Here there is no distortion for the causal effect estimate in the
analytic sample population as it exists at the end of study. However the
end-of-study sample population is an undesirable restriction of the
target population. We infer that our results in this setting risk
\textbf{weirdness}: (\textbf{w}rongly \textbf{e}stimated inferences due
to \textbf{i}nappropriate \textbf{r}estriction and \textbf{d}istortion)
because there is \textbf{i}nappropriate \textbf{r}estriction.

\subsubsection{Summary}\label{summary-1}

In this section, we examined how right-censoring, or attrition, can lead
to biased causal effect estimates. Even without confounding bias,
wherever the distribution of variables that modify treatment effects
differs between the sample population at the start and end of the study,
the average treatment effects may differ, leading to biased estimates
for the target population. To address such bias, investigators must
ensure that the distribution of potential outcomes at the end of the
study aligns with that of the target population. Again, techniques such
as inverse probability weighting and multiple imputation can help
mitigate this bias (refer to
\citeproc{ref-bulbulia2024PRACTICAL}{Bulbulia 2024a}).

Attrition is nearly inevitable. Before seeking ambitious samples that
extend a species understanding, we must ensure our design is not
\textbf{weird}: (\textbf{w}rongly \textbf{e}stimated inferences due to
\textbf{i}nappropriate \textbf{r}estriction and \textbf{d}istortion).

Next we investigate target population restriction bias at the start of
study (left-censoring), where structural motifs of measurement error
bias again reappear.

\newpage{}

\subsection{\texorpdfstring{Part 3: How Target Population Restriction
Bias At The Start of Study Makes Your Causal Inferences weird
(\textbf{w}rongly \textbf{e}stimated inferences due to
\textbf{i}nappropriate \textbf{r}estriction and
\textbf{d}istortion)}{Part 3: How Target Population Restriction Bias At The Start of Study Makes Your Causal Inferences weird (wrongly estimated inferences due to inappropriate restriction and distortion)}}\label{id-sec-3}

Consider target-restriction bias that occurs at the start of a study.
There are several failure modes. For example, the source population from
which participants are recruited might not align with the target
population. Moreover, even where there is such alignment, the
participants recruited into a study - the analytic sample -- might not
align with the source population. For simplicity, we imagine the sample
population at the start of study accurately aligns with the source
population. What constitutes `alignment'? We say the sample is
unrestrictive of the target population if there are no differences
between the sample and target population in the distribution both of
confounders (common causes of treatment and outcome) and of the
variables that modify treatment effects (effect modifiers). Proof of
alignment cannot be verifed with data (refer to
\hyperref[id-app-c]{Appendix}.

\subsubsection{Target Population Restriction Bias at Baseline Can Be
Collider-Restriction
Bias}\label{target-population-restriction-bias-at-baseline-can-be-collider-restriction-bias}

\begin{table}

\caption{\label{tbl-terminologyselectionrestrictionclassic}Collider-Stratification
bias at the start of a study (`M-bias')}

\centering{

\terminologyselectionrestrictionclassic

}

\end{table}%

Table~\ref{tbl-terminologyselectionrestrictionclassic} \(\mathcal{G}_1\)
illustrates an example of target population restriction bias at baseline
in which there is collider-restriction bias.

Suppose investigators want to estimate the causal effects of regular
physical activity, \(A\), and heart health, \(Y\), among adults visiting
a network of community health centres for routine check-ups.

Suppose there are two unmeasured variables that affect selection into
the study \(S=1\)

\begin{enumerate}
\def\labelenumi{\arabic{enumi}.}
\item
  Health Awareness, \(U1\), an unmeasured variable that influences both
  the probability of participating in the study, \(\boxed{S = 1}\), and
  the probability of being physically active, \(A\). Perhaps people with
  higher health awareness are more likely to (1) engage in physical
  activity and (2) participate in health-related studies.
\item
  Socioeconomic Status (SES), \(U2\), an unmeasured variable that
  influences both the probability of participating in the study,
  \(\boxed{S = 1}\), and heart health, \(Y\). We assume that individuals
  with higher SES have better access to healthcare and are more likely
  to participate in health surveys; they also tend to have better heart
  health from healthy lifestyles: joining expensive gyms, juicing, long
  vacations, and the like.
\end{enumerate}

As presented in Table~\ref{tbl-terminologyselectionrestrictionclassic}
\(\mathcal{G}_1\), there is collider-restriction bias from conditioning
on \(S=1\):

\begin{enumerate}
\def\labelenumi{\arabic{enumi}.}
\item
  \textbf{\(U1\)}: Because individuals with higher health awareness are
  more likely to be both physically active and participate in the study,
  the subsample over-represents physically active individuals. This
  overestimates the prevalence of physical activity, setting up a bias
  in overstating the potential benefits of physical activity on heart
  health in the general population.
\item
  \textbf{\(U2\)}: Because individuals with higher SES may have better
  heart health from SES-related factors, this opens a confounding path
  from physical activity and heart health through the selected sample,
  setting up the investigators for the potentially erroneous inference
  that physical activity has a greater positive impact on heart health
  than it actually does in the general population. The actual effect of
  physical activity on heart health in the general population might be
  less pronounced than observed.
\end{enumerate}

It might seem that researchers would need to sample from the target
population. However, by adjusting for health awareness or SES, or a
proxy for either, researchers may block the open path.
Table~\ref{tbl-terminologyselectionrestrictionclassic} \(\mathcal{G}_2\)
presents this solution. This strategy will only provide an unbiased
effect estimate for the population if either there is no causal effect
for all strata of the selected sample (the sharp null hypothesis) or
there are no interactions between the distribution of effect modifiers
in the sample population and the target population.

The next series of examples illustrate challenges to obtaining valid
causal effect estimates in the presence of interactions.

\subsubsection{Target Population Restriction Bias at Baseline Without
Collider-Restriction Bias at
Baseline}\label{target-population-restriction-bias-at-baseline-without-collider-restriction-bias-at-baseline}

\begin{table}

\caption{\label{tbl-terminologyselectionrestrictionbaseline}The
association in the population of selected individuals differs from the
causal association in the target population. Hernán calls this scenario
`selection bias off the null' (\citeproc{ref-hernuxe1n2017}{Hernán
2017}). Lu et al.~call this scenario `Type 2 selection bias'
(\citeproc{ref-lu2022}{Lu \emph{et al.} 2022}). We call this bias
`target population restriction bias at baseline'.}

\centering{

\terminologyselectionrestrictionbaseline

}

\end{table}%

\subsubsection{Problem 1: Target population is not WEIRD (Western,
Educated, Industrialised, Rich, and Democratic); sample population is
WEIRD}\label{problem-1-target-population-is-not-weird-western-educated-industrialised-rich-and-democratic-sample-population-is-weird}

Table~\ref{tbl-terminologyselectionrestrictionbaseline}
\(\mathcal{G}_{1.1}\) presents a scenario with target population
restriction bias at baseline. When the sample obtained at baseline
differs from the target population in the distributions of variables
that modify treatment effects, effect estimates may be biased, even in
the absence of confounding bias. Results may be \textbf{weird} without
arising from confounding bias.

Suppose we are interested in the effects of political campaigning but
only sample from our preferred political party. Results for the general
population will be distorted if the distribution of effect modifiers of
the treatment varies by party. One such effect modifier might be `party
affiliation'. This valid concern underscores the call for broader
sampling in the human sciences. WEIRD samples will not be informative
for science generally whenever the distribution of effect modifiers
among humans differs from those of the restricted population of humans
from which WEIRD analytic samples are drawn.

Note that we have encountered
Table~\ref{tbl-terminologyselectionrestrictionbaseline}
\(\mathcal{G}_{1.1}\) \emph{twice} before. It is the same causal
directed acyclic graph as we found in
Table~\ref{tbl-terminologycensoring} \(\mathcal{G}_5\). As we did
before, we may replacing the unmeasured effect modifiers
\(\circledotted{F}\) and \(U_{\Delta F}\) for \(\circledotted{U_Y}\) in
Table~\ref{tbl-terminologymeasurementerror} \(\mathcal{G}_2\) and
observe that we recover uncorrelated measurement error off the null
(i.e.~when there is a true treatment effect.)

The structural similarity suggests options might be easily overlooked.
Where the distributions of treatment-effect modifiers are known and
measured and where census (or other) weights are available for the
distributions of effect modifiers in the target population, it may be
possible to weight the sample to more closely approximate the target
population parameters of interest (refer to Stuart \emph{et al.}
(\citeproc{ref-stuart2015}{2015}).)

Let \(\widehat{ATE}_{target}\) denote the population average treatment
effect for the target population. Let
\(\widehat{ATE}_{\text{restricted}}\) denote the average treatment
effect at the end of treatment. Let \(W\) denote a set of variables upon
which the restricted and target populations structurally differ. We say
that results \emph{generalise} if we can ensure that:

\[
\widehat{ATE}_{target} = \widehat{ATE}_{restricted}
\]

or if there is a known function such that:

\[
ATE_{target} \approx f_W(ATE_{\text{restricted}}, W)
\]

In most cases, \(f_W\) will be unknown, as it must account for potential
heterogeneity of effects and unobserved sources of bias. For further
discussion on this topic, see Imai \emph{et al.}
(\citeproc{ref-imai2008misunderstandings}{2008}); Cole and Stuart
(\citeproc{ref-cole2010generalizing}{2010}); Stuart \emph{et al.}
(\citeproc{ref-stuart2018generalizability}{2018}).

Table~\ref{tbl-terminologyselectionrestrictionbaseline}
\(\mathcal{G}_{1.2}\) provides a graphical representation of the
solution.

Importantly, if there is considerable heterogeneity across humans,
\textbf{then we might not know how to interpret the average treatment
effect for the target population of all humans even if this causal
effect can be estimated.} In comparative research, we are often
precisely interested in treatment heterogeneity. If we seek explicitly
comparative models, however, we will need to ensure the validity of
estimates for every sample that we compare. If one stratum in the
comparative study is \textbf{weird}: (\textbf{w}rongly
\textbf{e}stimated inferences due to \textbf{i}nappropriate
\textbf{r}estriction and \textbf{d}istortion), errors will propogate to
the remainder of the comparative study. To understand such propogation
consider scenarios where the target population is a deliberate
restriction of the the source population from which the analytic sample
at baseline is drawn. We deliberately seek restriction wherever
`eligibility criteria' are desirable for a study. Although this point is
perhaps obvious, it is less clear whether many studies should be
conducted without eligibility criteria.

\subsubsection{Example 2: Target population is a sub-sample of WEIRD
(Western, Educated, Industrialised, Rich, and Democratic); sample
population is not WEIRD
enough.}\label{example-2-target-population-is-a-sub-sample-of-weird-western-educated-industrialised-rich-and-democratic-sample-population-is-not-weird-enough.}

Table~\ref{tbl-terminologyselectionrestrictionbaseline}
\(\mathcal{G}_{2.1}\) presents a scenario where the source population
does not meet eligibility criteria. Consider again the question of
whether vasectomy affects a sense of meaning and purpose in life.
Suppose further we want to evaluate effects in New Zealand among men
over the age of 40 who have no prior history of vasectomy, and who are
in relationships with heterosexual partners. The target population is a
stratum of WEIRD population (Western, Educated, Industrialised, Rich,
and Democratic). That is, the WIERD population would be too broad for
scientific interest. We should not sample from young children, the
elderly, or any who do not qualify. Not only is it clear that for many
scientific questions, a narrow population is desirable, it is hard to
imagine settings in comparative human science for which a fully
unrestricted human population would be desirable.

Note again Table~\ref{tbl-terminologyselectionrestrictionbaseline}
\(\mathcal{G}_{2.1}\) is identical to
Table~\ref{tbl-terminologycensoring} \(\mathcal{G}_5\) ---
right-censoring bias with effect modifiers in an otherwise unconfounded
study. The structure is also similar to
Table~\ref{tbl-terminologymeasurementerror} \(\mathcal{G}_2\) the
problem is structurally that of uncorrelated measurement error `off the
null.' Where it is the defusion of the effect-modifiers that causes we
may fix the measurement error by restricting the sample.

Table~\ref{tbl-terminologyselectionrestrictionbaseline}
\(\mathcal{G}_{2.2}\) presents a solution. Ensure eligibility criteria
are scientifically relevant and feasible. Sample from this eligible
population. With caution, apply survey or other weights where these
weights enable a closer approximation to the distributions of
effect-modifiers in the target population. Here, we avoid
\textbf{weird}: (\textbf{w}rongly \textbf{e}stimated inferences due to
\textbf{i}nappropriate \textbf{r}estriction and \textbf{d}istortion) by
imposing greater restriction on what had been an inappropriately
unrestricted target population.

\subsubsection{Example 3: Correlated Measurement Error of Covariates and
Outcome in the Absence of a Treatment
Effect}\label{example-3-correlated-measurement-error-of-covariates-and-outcome-in-the-absence-of-a-treatment-effect}

Table~\ref{tbl-terminologyselectionrestrictionbaseline}
\(\mathcal{G}_{3.1}\) considers the threats to target validity from
correlated measurement errors in the target population arising from
structured errors across heterogeneous stratums. For simplicity imagine
the groups with structured errors are cultures. Even if the treatment is
measured without error, multiple sources of error may lead to
association without causation.

Suppose investigators plan a cross-cultural investigation to clarify the
relationship between interventions on religious service attendance,
\(A\), and charitable giving, \(Y\). They plan to obtain measures of
covariates \(L\) sufficient to control for confounding. Suppose the
investigators observe religious attendance so that it is not measured
with error (as did \citeproc{ref-shaver2021comparison}{Shaver \emph{et
al.} 2021}), yet there is heterogeneity in the measurement of covariates
\(L\) and the outcome \(Y\). For example, if charitable giving measures
are included among the baseline covariates in \(L\), measurement errors
at baseline will be correlated with outcome measures. Perhaps in certain
cultures, charitable giving is under-reported (it is associated with the
vice gullibility), while in others, it is over-reported (only the
charitable are hired and promoted). Suppose further that true covariates
affect the treatment and outcome. As shown in
Table~\ref{tbl-terminologyselectionrestrictionbaseline}
\(\mathcal{G}_{3.1}\), in this setting, multiple paths of confounding
bias are open.

Moreover, because measurements are causally related to the phenomena
they record, investigators cannot apply statistical tests to verify
whether measures are recorded with error
(\citeproc{ref-vanderweele2022}{VanderWeele 2022};
\citeproc{ref-vansteelandt2022}{Vansteelandt and Dukes 2022b}). Whether
the phenomena that investigators hope to measure are functionally
equivalent across cultural settings remains unknown, and can generally
only be discovered slowly, through patient, careful work with local
experts. Although big cross-cultural projects are preferred at certain
science journals, including multiple cultures into a single analysis
imposes considerable burdens on investigators. All sources of error must
be evaluated -- and error from one culture can poison the wells of
others at analysis.

Table~\ref{tbl-terminologyselectionrestrictionbaseline}
\(\mathcal{G}_{3.2}\) provides a sensible solution: restrict one's study
to those cultures where causality can be identified. Democritus wrote,
`I would rather discover one cause than gain the kingdom of Persia'
(\citeproc{ref-freeman1948ancilla}{Freeman 1948}). Paraphrasing
Democritus, we might say, `I should rather discover one WEIRD cause than
the kingdom of \textbf{weird} comparative research.'

\subsubsection{Example 4: Correlated Measurement Error of
Effect-Modifiers for an Overly Ambitious Target
Population}\label{example-4-correlated-measurement-error-of-effect-modifiers-for-an-overly-ambitious-target-population}

Table~\ref{tbl-terminologyselectionrestrictionbaseline}
\(\mathcal{G}_{4.1}\) considers the threats to target validity from
correlated measurement errors in the target population arising from
structured errors linking measurements for the effect modifiers. Here,
we discover a familiar structural bias of correlated measurement error
bias Table~\ref{tbl-terminologymeasurementerror} \(\mathcal{G}_3\)

Even if the treatment is randomised so that there are no open backdoor
paths, and even if the treatment and outcome are measured without error,
investigators will not be able to obtain valid estimates for
treatment-effect heterogeneity from their data, nor will they be able to
apply target-sample weights (such as census weights) to obtain valid
estimates for the populations in which the measurement errors of effect
modifiers are manifest.

Table~\ref{tbl-terminologyselectionrestrictionbaseline}
\(\mathcal{G}_{4.2}\) suggests that where measures of effect
modification are uncertain, it is best to consider settings in which the
measurements are reliable --- whether or not the settings are WEIRD
(Western, Educated, Industrialised, Rich, and Democratic). Moreover, in
comparative settings where multiple cultures are measured, unless each
is proven innocent of structural measurement error bias, it is generally
best to report the results for each culture separately, without
attempting comparisons.

\subsection{Part 4 Measurement Error Bias Understood Through Single
World Intervention Graphs}\label{id-sec-4}

Not all biases in causal inference relate to confounding. In
\hyperref[id-sec-1]{Part 1}, we examined undirected/uncorrelated
measurement error bias and found that measurement bias can arise `off
the null' without any confounding
(Table~\ref{tbl-terminologymeasurementerror} \(\mathcal{G}_2\)). In
\hyperref[id-sec-2]{Part 2}, we examined population-restriction bias at
the end of a study, finding it to be a variety of undirected
uncorrelated measurement error bias
(Table~\ref{tbl-terminologycensoring} \(\mathcal{G}_5\)). In
\hyperref[id-sec-3]{Part 3}, we examined population-restriction bias at
baseline; of the five biases considered, only one could be classified as
confounding bias.

Throughout this article, we encountered challenges in using causal
directe acyclic graphs to represent biases that do not arise from
confounding. This should come as no surprise because causal DAGs are
designed to clarify confounding bias and not other biases
(\citeproc{ref-hernan2024WHATIF}{Hernan and Robins 2024};
\citeproc{ref-pearl2009a}{Pearl 2009}) \hyperref[id-app_E]{Appendix E}

To enrich our understanding of bias from restriction and measurement in
the absence of confounding bias, we turn to Single World Intervention
Graphs (SWIGs). SWIGs are causal diagrams that allow us to read
counterfactual dependencies directly off a graph
(\citeproc{ref-richardson2013}{Richardson and Robins 2013a}). SWIGs
unify the potential outcomes framework and Pearl's structural causal
model framework by allowing investigators to check the conditional
independencies of potential outcomes and the treatments that precede
them. Similar to causal DAGs, Single World Interventions Graphs function
to factorise conditional probability distributions from assumed causal
structures so that investigators can evaluate identifiability conditions
by inspecting features of a causal diagram. Single World Intervention
Graphs are not purpose build to evaluate measurement error and
restriction biases. However, because SWIGs encode assumptions about the
relationships of treatments to the counterfactual outcomes that arise
after an intervention is made, SWIGs may help us to better understand
the causal mechanisms at work when there are measurement error biases.
And because population restriction biases can be understood as
measurement error biases, certain observations generalise. We next
consider measurement error bias by representing counterfactual histories
directly on Single World Intervention Graphs.

SWIGs operate by `node-splitting' at each intervention
(\citeproc{ref-bulbulia2024swigstime}{Bulbulia 2024b};
\citeproc{ref-richardson2013}{Richardson and Robins 2013a}), dividing
the intervention into a random component and a fixed component. Nodes
that follow a fixed intervention are relabled with the value of the
intervention depicted in a SWIG. Importantly only one intervention is
represented in any given SWIG (we never observe the joint distribution
of more than one intervention at a time). Single World Intervention
Templates are functions that we may use to generate multiple Single
World Intervention Graphs
(\citeproc{ref-richardson2013swigsprimer}{Richardson and Robins 2013b}).
Whether there we imagine a single-point or sequential series of
interventions, one reads a SWIG just as one would read a causal DAG,
ensuring there are no backdoor paths linking the random part of the node
to the outcome. The deterministic part of a node is fixed, preventing
confounding in the counterfactual future from the fixed portion of the
node unless an open backdoor path arises before a subsequent
intervention that links the subsequent intervention to the outcome in
the absence of a causal effect. Again, although SWIGs, like causal DAGs,
are built to evaluate identification (no unmeasured confounding) by
factorising the conditional and marginal distributions associated with a
graph, the explicit representation of counterfactual states makes it
easier to understand how bias arises in the absence of confounding.

\begin{table}

\caption{\label{tbl-tblme}Uncorrelated/Undirected Measurement Error in
Single World Intervention Graph}

\centering{

\tblme

}

\end{table}%

Table~\ref{tbl-tblme}\(\mathcal{G}_{1.1}\) presents a Single World
Intervention Template from which we may generate two counterfactual
states of the world under two distinct interventions
\(A = \tilde{a} \in \{0,1\}\). We call the measurement of the
intervention a `reporter of A' and denote the state of the reporter
under \(A = \tilde{a}\) as \(B(\tilde{a})\). In our convention, if a
node in a Single World Intervention Graph (or Template) is unobserved,
we shade it in grey. \(E^A\) denotes an unmeasured variable or set of
variables that causes the reporter \(B(\tilde{a})\) to differ from
\(\tilde{a}\), the fixed state of the intervention when \(A\) is set to
\(\tilde{a}\). In template Table~\ref{tbl-tblme}\(\mathcal{G}_{1.1}\),
\(A = \tilde{a}\) remains unobserved. The only observed nodes are
\(B(\tilde{a})\) and \(Y(B(\tilde{a}))\), which is the potential outcome
for \(Y\) \emph{as reported} by \(B(\tilde{a})\). Note that here we
include reporters of the unobserved true state of the treatment directly
in our representation of the causal order as encoded in our Single World
Intervention Graph. Table~\ref{tbl-tblme}\(\mathcal{G}_{1.2}\)
corresponds to the assumed state of the world when the reporter of \(A\)
is set to \(B(0)\). Table~\ref{tbl-tblme}\(\mathcal{G}_{1.3}\)
corresponds to the assumed state of the world when the reporter of \(A\)
is set to \(B(1)\); in this world, investigators observe \(Y(B(1))\). We
assume that \(E^A\) is independent of both \(A\) and \(Y(\tilde{a})\).
However, we assume that \(E^A\) causes \(B(\tilde{a})\) to differ from
the true state \(A=\tilde{a}\). As a result of this misclassification,
we have no assurance whether
\(\mathbb{E}[Y(B(1)) - Y(B(0))] = \mathbb{E}[Y(1) - Y(0)]\). The SWIGs
make it apparent that although \(A\) is independent of \(E^A\),
\(A = \tilde{a}\) and \(E^A\) become statistically entangled in the
reporter \(B(\tilde{a})\), and it is this reporter, not the unobserved
true state of \(Y(\tilde{a})\), that investigators record.

Consider the following example. Coach Alice randomly assigns one of two
running programs to club runners: \(A = 1\) (train), \(A = 0\) (do not
train). Alice is not interested in estimating the effect of random
treatment assignment (the intent-to-treat effect). Rather, she wants to
understand the causal effect of training as compared with rest---a
per-protocol effect. Unknown to Alice, 20\% do not follow the program.
Table~\ref{tbl-tblme}\(\mathcal{G}{1.1}\) is a SWIG template that
presents bias from measurement error in the treatment. The template
serves as a `graph value function' that generates SWIGs
Table~\ref{tbl-tblme}\(\mathcal{G}{1.1}\), in which all runners receive
\(A = 0\) (do not train), and
Table~\ref{tbl-tblme}\(\mathcal{G}_{1.2}\), in which all runners receive
\(A = 1\) (train). Here, \(B(0)\) and \(B(1)\) denote the reporters of
the level of the intervention.

Again, we note that the treatment recorded is not the per-protocol
effect \(E[Y(1) - Y(0)]\) but rather the intention-to-treat effect
\(E[Y(B(1)) - Y(B(0))]\). Generally, the effect we obtain will
understate the per-protocol effect of training both on the difference
scale and the risk ratio scale. Those who were assigned to training but
rest will dilute the effect of training: \(E[Y(1)] > E[Y(B(1))]\). Those
who were assigned to rest but train will augment the rest effect:
\(E[Y(B(1))] > E[Y(1)]\). Hence:

\[
 E[Y(1) - Y(0)] > E[Y(B(1)) - Y(B(0))]
\]

Note that attenuation of a true treatment effect is not guaranteed
(\citeproc{ref-jurek2008brief}{Jurek \emph{et al.} 2008};
\citeproc{ref-lash2020}{Lash \emph{et al.} 2020}). The SWIGs in
Table~\ref{tbl-tblme}\(\mathcal{G}_{1.1-1.3}\) make the general
measurement bias problem clear: although the treatment that is estimated
remains \(d\)-separated from the potential outcomes, the causal contrast
that we obtain at the end of the study is not the treatment we seek and
will often (though not always) diminish a true treatment effect because
the reporter \emph{under treatment} is a common effect of the unmeasured
source of bias and the treatment that has been applied, and it is the
outcomes under mis-measured treatments that investigators contrast.

Table~\ref{tbl-tblme}\(\mathcal{G}_{2.1}\) presents a Single World
Intervention Template from which we may generate two counterfactual
states of the world under two distinct interventions
\(A = \tilde{a} \in \{0,1\}\). Here, the treatment is observed and
recorded without error. Hence we do not include a reporter of the
treatment. However, the true outcome is not observed, but only reported
with error. \(E^Y\) denotes the unmeasured source of error in the
reporting of \(Y(\tilde{a})\), which we assume to be independent of
\(A\) and of \(Y\). Because both \(E^Y\) and \(Y(\tilde{a})\) are not
observed, we shade these nodes in grey. The node \(V(Y(\tilde{a}))\)
denotes the observed state of \(Y\) when \(A = \tilde{a}\). In template
Table~\ref{tbl-tblme}\(\mathcal{G}_{2.1}\), \(A = \tilde{a}\) remains
unobserved. Table~\ref{tbl-tblme}\(\mathcal{G}_{2.2}\) corresponds to
the assumed state of the world when \(A=0\) and \(Y(0)\) is reported
with error as \(V(Y(0))\). Likewise,
Table~\ref{tbl-tblme}\(\mathcal{G}_{2.3}\) corresponds to the assumed
state of the world when \(A=1\) and \(Y(1)\) is reported with error as
\(V(Y(Y(1))\). We assume that \(E^Y\) is independent of \(A\) and of
\(Y\). Misclassification will tend to increase the variance of the
estimated treatment effect. If the outcome is continuous, the expected
difference in the mean of the outcome for the reported outcome may
differ from that for the true outcome. How bias affects the outcome will
vary depending on the scale we use to evaluate such bias. Suppose that
under training, the athlete runs a marathon in 3 hours, and under rest,
they run a marathon in 4 hours. To keep figures easy, let's use round
numbers. Suppose the bias in reporting is 1 hour. Thus, we have
\(E[Y(1)] = 3\), \(E[Y(0)] = 4\); \(E[V(Y(1))] = 2\) and
\(E[V(Y(0))] = 3\).

\[
\text{ATE Difference Scale: no measurement error} = E[Y(1)] - E[Y(0)] = 3 - 4 = -1
\]

\[
\text{ATE Difference Scale: measurement error} = E[V(Y(1))] - E[V(Y(0))] = 2 - 3 = -1
\]

The effect estimates do not differ:
\(\text{ATE no measurement error} = \text{ATE measurement error}\).

However, consider this bias on the risk ratio scale:

\[
\text{ATE Risk Ratio Scale: no measurement error} = E[Y(1)] / E[Y(0)] = 3 / 4 = 0.75
\]

\[
\text{ATE Risk Ratio Scale: measurement error} = E[V(Y(1))] / E[V(Y(0))] = 2 / 3 = 0.66
\]

These effect estimates differ:

\[\text{ATE RR no measurement error} \neq \text{ATE RR measurement error}\]

Imagine the bias was positive, such that runners added an hour to their
times---perhaps the runners do not want to stand out. The true risk
ratio for the treatment remains \(0.75\). However, the biased risk ratio
for the treatment would become:

\[
\text{ATE Risk Ratio Scale: measurement error} = E[V(Y(1))] / E[V(Y(0))] = 4 / 5 = 0.8
\]

Here we would understate the true treatment effect.

The Single World Intervention Graphs
Table~\ref{tbl-tblme}\(\mathcal{G}_{2.1-2.3}\) make clear the reason for
the scale sensitivity of the bias. Although the source of bias in the
outcome (\(E^Y\)) is independent of the treatment (\(A\)), \(E^Y\)
functions as an effect modifier for the reported outcome
\(V(Y(\tilde{a}))\).

It has long been understood that where treatment effects vary across
different strata of the population, an estimate of the causal effect on
the risk difference scale will differ from the estimate on the risk
ratio scale (\citeproc{ref-greenland2003quantifying}{Greenland 2003}).
Here we find that reporters of the outcome are subject to similar
relativity. For example, we might have constructed a multiplicative
error function for the outcome such that we subtract 1 hour if the
response is 3 and subtract 1.344 if the response is 4. Under this error
function, the risk ratio would remain stable at 0.75 irrespective of
whether the outcome was measured with error; however, the risk
difference would no longer be constant.

\subsubsection{Summary Part 4}\label{summary-part-4}

We have characterised target-population restriction bias, whether at the
start or the end of the study, as formally equivalent to undirected
uncorrelated measurement error. Here, Single World Intervention Graphs
allow us to apply lessons from the study of effect modification to the
analysis of measurement errir biases. Single World Intervention Graphs
Table~\ref{tbl-tblme}\(\mathcal{G}_{2.1-2.3}\) greatly clarify how
measurement error bias for the outcome arises in the absence of
confounding bias: the unmeasured causes of error function as effect
modifiers of the outcome reporters, such that causal contrasts will
differ on at least one scale of effect `off the null'.

\subsection{Conclusions}\label{conclusions}

In causal inference, we begin by specifying clear treatments and
outcomes, defining the contrasts to be made for treatments at specific
levels, and identifying a target population for whom results generalise.
Although causal inference is gaining popularity, much work remains to
address the threats posed by measurement error bias and target
population-restriction bias (often referred to as selection bias) to
causal inferences. These threats are particularly evident in the
comparative human sciences.

Here, we have considered how problems of target-population restriction
at the beginning and end of a study can be approached as variations of
measurement-error bias. Certain measurement error biases arise from the
effect modification of `reporters' by variables that cause reporters to
differ from the realities they measure. Using causal directed acyclic
graphs (DAGs) and Single World Intervention Graphs (SWIGs), we have
clarified the structural features of bias that recur in
measurement-error bias, target population restriction bias at the start
of a study, and target population restriction bias at the end of a
study.

We define any study that exhibits these biases as \textbf{weird}
(\textbf{w}rongly \textbf{e}stimated inferences due to
\textbf{i}nappropriate \textbf{r}estriction and \textbf{d}istortion).

It is laudable to seek species-level knowledge. Really. Science should
seek generalisations where it can because generalisation is knowledge.
However, before venturing into the vast wilderness of human existence,
locally-understood gardens must be cultivated. A long shadow of
measurement error casts its shade over every aspect of the human
condition that scientists hope to understand. However, the standard
workflow for causal inference offers guidance for research design,
whether or not the causal questions are comparative.

\subsubsection{Schematic Workflow For Avoiding weird Causal
Inferences}\label{schematic-workflow-for-avoiding-weird-causal-inferences}

We avoid \textbf{weird} (\textbf{w}rongly \textbf{e}stimated inferences
due to \textbf{i}nappropriate \textbf{r}estriction and
\textbf{d}istortion) inferences in comparative research in the same way
that we avoid \textbf{weird} inferences in any research by undertaking
the following steps:

\begin{enumerate}
\def\labelenumi{\arabic{enumi}.}
\tightlist
\item
  \textbf{State a well-defined intervention.}
\end{enumerate}

Clearly define the treatment or exposure to be evaluated. Of which
events do we hope to infer consequences? Which levels of intervention
will be compared?

\begin{enumerate}
\def\labelenumi{\arabic{enumi}.}
\setcounter{enumi}{1}
\tightlist
\item
  \textbf{State a well-defined outcome.}
\end{enumerate}

Clearly define the outcome to be evaluated. Which consequences are of
interest? Which comparisons will be made? At which time scale are we
interested? At which scale of causal contrast are we interested?

\begin{enumerate}
\def\labelenumi{\arabic{enumi}.}
\setcounter{enumi}{2}
\tightlist
\item
  \textbf{Clarify the target population.}
\end{enumerate}

Define the population to whom the results will generalise, understanding
that causal contrasts may differ for different populations, even in the
absence of confounding or measurement error biases.

\begin{enumerate}
\def\labelenumi{\arabic{enumi}.}
\setcounter{enumi}{3}
\tightlist
\item
  \textbf{Ensure treatments to be compared satisfy causal consistency.}
\end{enumerate}

Verify that the treatments correspond to interpretable interventions
(\citeproc{ref-hernan2024WHATIF}{Hernan and Robins 2024})

\begin{enumerate}
\def\labelenumi{\arabic{enumi}.}
\setcounter{enumi}{4}
\tightlist
\item
  \textbf{Evaluate whether treatment groups, conditional on measured
  covariates, are exchangeable.}
\end{enumerate}

Balancing confounding covariates across treatment levels ensures that
differences between groups are conditionally `ignorable', or
equivalently, are conditionally exchangeable, or equivalently, that all
backdoor paths have been closed.

\begin{enumerate}
\def\labelenumi{\arabic{enumi}.}
\setcounter{enumi}{5}
\tightlist
\item
  \textbf{Check if the positivity assumption is satisfied.}
\end{enumerate}

Confirm that all individuals have a non-zero probability of receiving
each treatment level, given their covariates.

\begin{enumerate}
\def\labelenumi{\arabic{enumi}.}
\setcounter{enumi}{6}
\tightlist
\item
  \textbf{Ensure that the measures relate to the scientific questions at
  hand.}
\end{enumerate}

Ensure that the data collected and the measures used directly relate to
the research question. As part of this, evaluate structural features of
measurement error bias. As we have considered, there are manifold
possibilities for measurement error bias to obscure the phenomena under
study, and these biases may interact with target-population-restriction
biases.

\begin{enumerate}
\def\labelenumi{\arabic{enumi}.}
\setcounter{enumi}{7}
\tightlist
\item
  \textbf{Consider strategies to ensure the study group measured at the
  end of the study represents the target population.}
\end{enumerate}

If the study population at both the beginning and end of treatment
differs in the distribution of variables that modify the effect of a
treatment on the outcome, the study will be biased in at least one
measure of effect.

\begin{enumerate}
\def\labelenumi{\arabic{enumi}.}
\setcounter{enumi}{8}
\tightlist
\item
  \textbf{Clearly communicate the reasoning, evidence, and
  decision-making that inform steps 1-8.}
\end{enumerate}

Provide transparent and thorough documentation of how steps 1-8 have
been made. This includes detailing the assumptions, methods, and
decisions. Perpare to conduct and report multiple analyses where there
is ambiguity in analystic decisions.

We have seen that demands of following this workflow in comparative
research are more stringent because measurement error bias must be
evaluated at every site to be compared. Heterogeneity in measurement
error can open biasing paths, and the target populations may not be
easily defined, sampled, or---where the scientific question requires---
appropriately restricted. Methodologists broadly agree on these points,
but they can easily forget them. We have shown how workflows for causal
inference act as essential preflight checklists for ambitious,
effective, and safe comparative cultural research. These workflows help
propel science forward without overreaching.

\newpage{}

\subsection{Funding}\label{funding}

This work is supported by a grant from the Templeton Religion Trust
(TRT0418) and RSNZ Marsden 3721245, 20-UOA-123; RSNZ 19-UOO-090. I also
received support from the Max Planck Institute for the Science of Human
History. The Funders had no role in preparing the manuscript or the
decision to publish it.

\newpage{}

\subsection{References}\label{references}

\phantomsection\label{refs}
\begin{CSLReferences}{1}{0}
\bibitem[\citeproctext]{ref-arnett2008neglected}
Arnett, JJ (2008) The neglected 95\%: Why american psychology needs to
become less american. \emph{American Psychologist}, \textbf{63}(7),
602--614.
doi:\href{https://doi.org/10.1037/0003-066X.63.7.602}{10.1037/0003-066X.63.7.602}.

\bibitem[\citeproctext]{ref-bareinboim2013general}
Bareinboim, E, and Pearl, J (2013) A general algorithm for deciding
transportability of experimental results. \emph{Journal of Causal
Inference}, \textbf{1}(1), 107--134.

\bibitem[\citeproctext]{ref-barrett2021}
Barrett, M (2021) \emph{Ggdag: Analyze and create elegant directed
acyclic graphs}. Retrieved from
\url{https://CRAN.R-project.org/package=ggdag}

\bibitem[\citeproctext]{ref-bulbulia2024PRACTICAL}
Bulbulia, J (2024a) A practical guide to causal inference in three-wave
panel studies. \emph{PsyArXiv Preprints}.
doi:\href{https://doi.org/10.31234/osf.io/uyg3d}{10.31234/osf.io/uyg3d}.

\bibitem[\citeproctext]{ref-bulbulia2024swigstime}
Bulbulia, J (2024b) Causal inference: Interaction, mediation, and
time-varying treatments. \emph{PsyArXiv}.
doi:\href{https://doi.org/10.31234/osf.io/vr268}{10.31234/osf.io/vr268}.

\bibitem[\citeproctext]{ref-bulbulia2022}
Bulbulia, JA (2022) A workflow for causal inference in cross-cultural
psychology. \emph{Religion, Brain \& Behavior}, \textbf{0}(0), 1--16.
doi:\href{https://doi.org/10.1080/2153599X.2022.2070245}{10.1080/2153599X.2022.2070245}.

\bibitem[\citeproctext]{ref-bulbulia2023}
Bulbulia, JA (2023) Causal diagrams (directed acyclic graphs): A
practical guide.

\bibitem[\citeproctext]{ref-margot2024}
Bulbulia, JA (2024c) \emph{Margot: MARGinal observational
treatment-effects}.
doi:\href{https://doi.org/10.5281/zenodo.10907724}{10.5281/zenodo.10907724}.

\bibitem[\citeproctext]{ref-bulbulia2023a}
Bulbulia, JA, Afzali, MU, Yogeeswaran, K, and Sibley, CG (2023)
Long-term causal effects of far-right terrorism in {N}ew {Z}ealand.
\emph{PNAS Nexus}, \textbf{2}(8), pgad242.

\bibitem[\citeproctext]{ref-carroll2006measurement}
Carroll, RJ, Ruppert, D, Stefanski, LA, and Crainiceanu, CM (2006)
\emph{Measurement error in nonlinear models: A modern perspective},
Chapman; Hall/CRC.

\bibitem[\citeproctext]{ref-chatton2020}
Chatton, A, Le Borgne, F, Leyrat, C, \ldots{} Foucher, Y (2020)
G-computation, propensity score-based methods, and targeted maximum
likelihood estimator for causal inference with different covariates
sets: a comparative simulation study. \emph{Scientific Reports},
\textbf{10}(1), 9219.
doi:\href{https://doi.org/10.1038/s41598-020-65917-x}{10.1038/s41598-020-65917-x}.

\bibitem[\citeproctext]{ref-cole2008}
Cole, SR, and Hernán, MA (2008) Constructing inverse probability weights
for marginal structural models. \emph{American Journal of Epidemiology},
\textbf{168}(6), 656--664.

\bibitem[\citeproctext]{ref-cole2010generalizing}
Cole, SR, and Stuart, EA (2010) Generalizing evidence from randomized
clinical trials to target populations: The ACTG 320 trial.
\emph{American Journal of Epidemiology}, \textbf{172}(1), 107--115.

\bibitem[\citeproctext]{ref-dahabreh2021study}
Dahabreh, IJ, Haneuse, SJA, Robins, JM, \ldots{} Hernán, MA (2021) Study
designs for extending causal inferences from a randomized trial to a
target population. \emph{American Journal of Epidemiology},
\textbf{190}(8), 1632--1642.

\bibitem[\citeproctext]{ref-dahabreh2019}
Dahabreh, IJ, and Hernán, MA (2019) Extending inferences from a
randomized trial to a target population. \emph{European Journal of
Epidemiology}, \textbf{34}(8), 719--722.
doi:\href{https://doi.org/10.1007/s10654-019-00533-2}{10.1007/s10654-019-00533-2}.

\bibitem[\citeproctext]{ref-dahabreh2019generalizing}
Dahabreh, IJ, Robins, JM, Haneuse, SJ, and Hernán, MA (2019)
Generalizing causal inferences from randomized trials: Counterfactual
and graphical identification. \emph{arXiv Preprint arXiv:1906.10792}.

\bibitem[\citeproctext]{ref-deffner2022}
Deffner, D, Rohrer, JM, and McElreath, R (2022) A Causal Framework for
Cross-Cultural Generalizability. \emph{Advances in Methods and Practices
in Psychological Science}, \textbf{5}(3), 25152459221106366.
doi:\href{https://doi.org/10.1177/25152459221106366}{10.1177/25152459221106366}.

\bibitem[\citeproctext]{ref-freeman1948ancilla}
Freeman, K (1948) \emph{Ancilla to the pre-socratic philosophers},
Reprint edition, Cambridge, MA: Harvard University Press.

\bibitem[\citeproctext]{ref-gachter2010}
Gaechter, S (2010) (Dis)advantages of student subjects: What is your
research question? \emph{Behavioral and Brain Sciences},
\textbf{33}(2-3), 92--93.
doi:\href{https://doi.org/10.1017/S0140525X10000099}{10.1017/S0140525X10000099}.

\bibitem[\citeproctext]{ref-greenland2003quantifying}
Greenland, S (2003) Quantifying biases in causal models: Classical
confounding vs collider-stratification bias. \emph{Epidemiology},
300--306.

\bibitem[\citeproctext]{ref-greenland2009commentary}
Greenland, S (2009) Commentary: Interactions in epidemiology: Relevance,
identification, and estimation. \emph{Epidemiology}, \textbf{20}(1),
14--17.

\bibitem[\citeproctext]{ref-henrich2010weirdest}
Henrich, J, Heine, SJ, and Norenzayan, A (2010) The weirdest people in
the world? \emph{Behavioral and Brain Sciences}, \textbf{33}(2-3),
61--83.

\bibitem[\citeproctext]{ref-hernan2024WHATIF}
Hernan, MA, and Robins, JM (2024) \emph{Causal inference: What if?},
Taylor \& Francis. Retrieved from
\url{https://www.hsph.harvard.edu/miguel-hernan/causal-inference-book/}

\bibitem[\citeproctext]{ref-hernuxe1n2004}
Hernán, MA (2004) A definition of causal effect for epidemiological
research. \emph{Journal of Epidemiology \& Community Health},
\textbf{58}(4), 265--271.
doi:\href{https://doi.org/10.1136/jech.2002.006361}{10.1136/jech.2002.006361}.

\bibitem[\citeproctext]{ref-hernuxe1n2017}
Hernán, MA (2017) Invited commentary: Selection bias without colliders
\textbar{} american journal of epidemiology \textbar{} oxford academic.
\emph{American Journal of Epidemiology}, \textbf{185}(11), 1048--1050.
Retrieved from \url{https://doi.org/10.1093/aje/kwx077}

\bibitem[\citeproctext]{ref-hernuxe1n2009}
Hernán, MA, and Cole, SR (2009) Invited commentary: Causal diagrams and
measurement bias. \emph{American Journal of Epidemiology},
\textbf{170}(8), 959--962.
doi:\href{https://doi.org/10.1093/aje/kwp293}{10.1093/aje/kwp293}.

\bibitem[\citeproctext]{ref-hernan2017per}
Hernán, MA, Robins, JM, et al. (2017) Per-protocol analyses of pragmatic
trials. \emph{N Engl J Med}, \textbf{377}(14), 1391--1398.

\bibitem[\citeproctext]{ref-holland1986}
Holland, PW (1986) Statistics and causal inference. \emph{Journal of the
American Statistical Association}, \textbf{81}(396), 945--960.

\bibitem[\citeproctext]{ref-hume1902}
Hume, D (1902) \emph{Enquiries Concerning the Human Understanding: And
Concerning the Principles of Morals}, Clarendon Press.

\bibitem[\citeproctext]{ref-imai2008misunderstandings}
Imai, K, King, G, and Stuart, EA (2008) Misunderstandings between
experimentalists and observationalists about causal inference.
\emph{Journal of the Royal Statistical Society Series A: Statistics in
Society}, \textbf{171}(2), 481--502.

\bibitem[\citeproctext]{ref-jurek2008brief}
Jurek, AM, Greenland, S, and Maldonado, G (2008) Brief report: How far
from non-differential does exposure or disease misclassification have to
be to bias measures of association away from the null?
\emph{International Journal of Epidemiology}, \textbf{37}(2), 382--385.

\bibitem[\citeproctext]{ref-jurek2005proper}
Jurek, AM, Greenland, S, Maldonado, G, and Church, TR (2005) Proper
interpretation of non-differential misclassification effects:
Expectations vs observations. \emph{International Journal of
Epidemiology}, \textbf{34}(3), 680--687.

\bibitem[\citeproctext]{ref-jurek2006exposure}
Jurek, AM, Maldonado, G, Greenland, S, and Church, TR (2006)
Exposure-measurement error is frequently ignored when interpreting
epidemiologic study results. \emph{European Journal of Epidemiology},
\textbf{21}(12), 871--876.
doi:\href{https://doi.org/10.1007/s10654-006-9083-0}{10.1007/s10654-006-9083-0}.

\bibitem[\citeproctext]{ref-lash2009applying}
Lash, TL, Fox, MP, and Fink, AK (2009) \emph{Applying quantitative bias
analysis to epidemiologic data}, Springer.

\bibitem[\citeproctext]{ref-lash2020}
Lash, TL, Rothman, KJ, VanderWeele, TJ, and Haneuse, S (2020)
\emph{Modern epidemiology}, Wolters Kluwer. Retrieved from
\url{https://books.google.co.nz/books?id=SiTSnQEACAAJ}

\bibitem[\citeproctext]{ref-lewis1973}
Lewis, D (1973) Causation. \emph{The Journal of Philosophy},
\textbf{70}(17), 556--567.
doi:\href{https://doi.org/10.2307/2025310}{10.2307/2025310}.

\bibitem[\citeproctext]{ref-leyrat2021}
Leyrat, C, Carpenter, JR, Bailly, S, and Williamson, EJ (2021) Common
methods for handling missing data in marginal structural models: What
works and why. \emph{American Journal of Epidemiology}, \textbf{190}(4),
663--672.

\bibitem[\citeproctext]{ref-li2023non}
Li, W, Miao, W, and Tchetgen Tchetgen, E (2023) Non-parametric inference
about mean functionals of non-ignorable non-response data without
identifying the joint distribution. \emph{Journal of the Royal
Statistical Society Series B: Statistical Methodology}, \textbf{85}(3),
913--935.

\bibitem[\citeproctext]{ref-lu2022}
Lu, H, Cole, SR, Howe, CJ, and Westreich, D (2022) Toward a Clearer
Definition of Selection Bias When Estimating Causal Effects.
\emph{Epidemiology (Cambridge, Mass.)}, \textbf{33}(5), 699--706.
doi:\href{https://doi.org/10.1097/EDE.0000000000001516}{10.1097/EDE.0000000000001516}.

\bibitem[\citeproctext]{ref-parameters2020}
Lüdecke, D, Ben-Shachar, MS, Patil, I, and Makowski, D (2020)
Extracting, computing and exploring the parameters of statistical models
using {R}. \emph{Journal of Open Source Software}, \textbf{5}(53), 2445.
doi:\href{https://doi.org/10.21105/joss.02445}{10.21105/joss.02445}.

\bibitem[\citeproctext]{ref-machery2010}
Machery, E (2010) Explaining why experimental behavior varies across
cultures: A missing step in "the weirdest people in the world?".
\emph{Behavioral and Brain Sciences}, \textbf{33}(2-3), 101--102.
doi:\href{https://doi.org/10.1017/S0140525X10000178}{10.1017/S0140525X10000178}.

\bibitem[\citeproctext]{ref-malinsky2022semiparametric}
Malinsky, D, Shpitser, I, and Tchetgen Tchetgen, EJ (2022)
Semiparametric inference for nonmonotone missing-not-at-random data: The
no self-censoring model. \emph{Journal of the American Statistical
Association}, \textbf{117}(539), 1415--1423.

\bibitem[\citeproctext]{ref-mcelreath2020}
McElreath, R (2020) \emph{Statistical rethinking: A {B}ayesian course
with examples in {R} and {S}tan}, CRC press.

\bibitem[\citeproctext]{ref-morgan2014}
Morgan, SL, and Winship, C (2014) \emph{Counterfactuals and causal
inference: Methods and principles for social research}, 2nd edn,
Cambridge: Cambridge University Press.
doi:\href{https://doi.org/10.1017/CBO9781107587991}{10.1017/CBO9781107587991}.

\bibitem[\citeproctext]{ref-neal2020introduction}
Neal, B (2020) Introduction to causal inference from a machine learning
perspective. \emph{Course Lecture Notes (Draft)}. Retrieved from
\url{https://www.bradyneal.com/Introduction_to_Causal_Inference-Dec17_2020-Neal.pdf}

\bibitem[\citeproctext]{ref-pearl1995}
Pearl, J (1995) Causal diagrams for empirical research.
\emph{Biometrika}, \textbf{82}(4), 669--688.

\bibitem[\citeproctext]{ref-pearl2009a}
Pearl, J (2009) \emph{Causality}, Cambridge University Press.

\bibitem[\citeproctext]{ref-richardson2013}
Richardson, TS, and Robins, JM (2013a) Single world intervention graphs:
A primer. In, Citeseer. Retrieved from
\url{https://core.ac.uk/display/102673558}

\bibitem[\citeproctext]{ref-richardson2013swigsprimer}
Richardson, TS, and Robins, JM (2013b) Single world intervention graphs:
A primer. In \emph{Second UAI workshop on causal structure learning,
{B}ellevue, {W}ashington}, Citeseer. Retrieved from
\url{https://citeseerx.ist.psu.edu/document?repid=rep1&type=pdf&doi=07bbcb458109d2663acc0d098e8913892389a2a7}

\bibitem[\citeproctext]{ref-richardson2014causal}
Richardson, TS, and Rotnitzky, A (2014) Causal etiology of the research
of {J}ames {M}. {R}obins. \emph{Statistical Science}, \textbf{29}(4),
459--484.

\bibitem[\citeproctext]{ref-robins2004effects}
Robins, JM, Hernán, MA, and SiEBERT, U (2004) Effects of multiple
interventions. \emph{Comparative Quantification of Health Risks: Global
and Regional Burden of Disease Attributable to Selected Major Risk
Factors}, \textbf{1}, 2191--2230.

\bibitem[\citeproctext]{ref-rubin1976}
Rubin, DB (1976) Inference and missing data. \emph{Biometrika},
\textbf{63}(3), 581--592.
doi:\href{https://doi.org/10.1093/biomet/63.3.581}{10.1093/biomet/63.3.581}.

\bibitem[\citeproctext]{ref-sears1986college}
Sears, DO (1986) College sophomores in the laboratory: Influences of a
narrow data base on social psychology's view of human nature.
\emph{Journal of Personality and Social Psychology}, \textbf{51}(3),
515.

\bibitem[\citeproctext]{ref-shaver2021comparison}
Shaver, JH, White, TA, Vakaoti, P, and Lang, M (2021) A comparison of
self-report, systematic observation and third-party judgments of church
attendance in a rural fijian village. \emph{Plos One}, \textbf{16}(10),
e0257160.

\bibitem[\citeproctext]{ref-shiba2021}
Shiba, K, and Kawahara, T (2021) Using propensity scores for causal
inference: Pitfalls and tips. \emph{Journal of Epidemiology},
\textbf{31}(8), 457--463.

\bibitem[\citeproctext]{ref-sjuxf6lander2016}
Sjölander, A (2016) Regression standardization with the R package
stdReg. \emph{European Journal of Epidemiology}, \textbf{31}(6),
563--574.
doi:\href{https://doi.org/10.1007/s10654-016-0157-3}{10.1007/s10654-016-0157-3}.

\bibitem[\citeproctext]{ref-stuart2018generalizability}
Stuart, EA, Ackerman, B, and Westreich, D (2018) Generalizability of
randomized trial results to target populations: Design and analysis
possibilities. \emph{Research on Social Work Practice}, \textbf{28}(5),
532--537.

\bibitem[\citeproctext]{ref-stuart2015}
Stuart, EA, Bradshaw, CP, and Leaf, PJ (2015) Assessing the
Generalizability of Randomized Trial Results to Target Populations.
\emph{Prevention Science}, \textbf{16}(3), 475--485.
doi:\href{https://doi.org/10.1007/s11121-014-0513-z}{10.1007/s11121-014-0513-z}.

\bibitem[\citeproctext]{ref-suzuki2013counterfactual}
Suzuki, E, Mitsuhashi, T, Tsuda, T, and Yamamoto, E (2013) A
counterfactual approach to bias and effect modification in terms of
response types. \emph{BMC Medical Research Methodology}, \textbf{13}(1),
1--17.

\bibitem[\citeproctext]{ref-suzuki2020}
Suzuki, E, Shinozaki, T, and Yamamoto, E (2020) Causal Diagrams:
Pitfalls and Tips. \emph{Journal of Epidemiology}, \textbf{30}(4),
153--162.
doi:\href{https://doi.org/10.2188/jea.JE20190192}{10.2188/jea.JE20190192}.

\bibitem[\citeproctext]{ref-tchetgen2017general}
Tchetgen Tchetgen, EJ, and Wirth, KE (2017) A general instrumental
variable framework for regression analysis with outcome missing not at
random. \emph{Biometrics}, \textbf{73}(4), 1123--1131.

\bibitem[\citeproctext]{ref-tripepi2007}
Tripepi, G, Jager, KJ, Dekker, FW, Wanner, C, and Zoccali, C (2007)
Measures of effect: Relative risks, odds ratios, risk difference, and
{`}number needed to treat{'}. \emph{Kidney International},
\textbf{72}(7), 789--791.
doi:\href{https://doi.org/10.1038/sj.ki.5002432}{10.1038/sj.ki.5002432}.

\bibitem[\citeproctext]{ref-vanderlaan2011}
Van Der Laan, MJ, and Rose, S (2011) \emph{Targeted Learning: Causal
Inference for Observational and Experimental Data}, New York, NY:
Springer. Retrieved from
\url{https://link.springer.com/10.1007/978-1-4419-9782-1}

\bibitem[\citeproctext]{ref-vanderweele2009}
VanderWeele, TJ (2009) Concerning the consistency assumption in causal
inference. \emph{Epidemiology}, \textbf{20}(6), 880.
doi:\href{https://doi.org/10.1097/EDE.0b013e3181bd5638}{10.1097/EDE.0b013e3181bd5638}.

\bibitem[\citeproctext]{ref-vanderweele2012}
VanderWeele, TJ (2012) Confounding and Effect Modification: Distribution
and Measure. \emph{Epidemiologic Methods}, \textbf{1}(1), 55--82.
doi:\href{https://doi.org/10.1515/2161-962X.1004}{10.1515/2161-962X.1004}.

\bibitem[\citeproctext]{ref-vanderweele2022}
VanderWeele, TJ (2022) Constructed measures and causal inference:
Towards a new model of measurement for psychosocial constructs.
\emph{Epidemiology}, \textbf{33}(1), 141.
doi:\href{https://doi.org/10.1097/EDE.0000000000001434}{10.1097/EDE.0000000000001434}.

\bibitem[\citeproctext]{ref-vanderweele2012a}
VanderWeele, TJ, and Hernán, MA (2012) Results on differential and
dependent measurement error of the exposure and the outcome using signed
directed acyclic graphs. \emph{American Journal of Epidemiology},
\textbf{175}(12), 1303--1310.
doi:\href{https://doi.org/10.1093/aje/kwr458}{10.1093/aje/kwr458}.

\bibitem[\citeproctext]{ref-vanderweele2007}
VanderWeele, TJ, and Robins, JM (2007) Four types of effect
modification: a classification based on directed acyclic graphs.
\emph{Epidemiology (Cambridge, Mass.)}, \textbf{18}(5), 561--568.
doi:\href{https://doi.org/10.1097/EDE.0b013e318127181b}{10.1097/EDE.0b013e318127181b}.

\bibitem[\citeproctext]{ref-vanderweele2022a}
VanderWeele, TJ, and Vansteelandt, S (2022) A statistical test to reject
the structural interpretation of a latent factor model. \emph{Journal of
the Royal Statistical Society Series B: Statistical Methodology},
\textbf{84}(5), 2032--2054.

\bibitem[\citeproctext]{ref-vansteelandt2022}
Vansteelandt, S, and Dukes, O (2022b) Assumption-lean inference for
generalised linear model parameters. \emph{Journal of the Royal
Statistical Society Series B: Statistical Methodology}, \textbf{84}(3),
657--685.

\bibitem[\citeproctext]{ref-vansteelandt2022a}
Vansteelandt, S, and Dukes, O (2022a) Assumption-lean inference for
generalised linear model parameters. \emph{Journal of the Royal
Statistical Society Series B: Statistical Methodology}, \textbf{84}(3),
657--685.

\bibitem[\citeproctext]{ref-westreich2010}
Westreich, D, and Cole, SR (2010) Invited commentary: positivity in
practice. \emph{American Journal of Epidemiology}, \textbf{171}(6).
doi:\href{https://doi.org/10.1093/aje/kwp436}{10.1093/aje/kwp436}.

\bibitem[\citeproctext]{ref-westreich2017}
Westreich, D, Edwards, JK, Lesko, CR, Stuart, E, and Cole, SR (2017a)
Transportability of trial results using inverse odds of sampling
weights. \emph{American Journal of Epidemiology}, \textbf{186}(8),
1010--1014.
doi:\href{https://doi.org/10.1093/aje/kwx164}{10.1093/aje/kwx164}.

\bibitem[\citeproctext]{ref-westreich2017transportability}
Westreich, D, Edwards, JK, Lesko, CR, Stuart, E, and Cole, SR (2017b)
Transportability of trial results using inverse odds of sampling
weights. \emph{American Journal of Epidemiology}, \textbf{186}(8),
1010--1014.

\end{CSLReferences}

\newpage{}

\subsection{Appendix A: Glossary}\label{id-app-a}

\begin{table}

\caption{\label{tbl-experiments}Glossary}

\centering{

\glossaryTerms

}

\end{table}%

\newpage{}

\subsection{Appendix B: Generalisability and
Transportability}\label{id-app-b}

\textbf{Generalisability:} When a study sample is drawn randomly from
the target population, we may generalise from the sample to the target
population as follows:

Suppose we sample randomly from the target population, where:

\begin{itemize}
\tightlist
\item
  \(n_S\) denotes the size of the study sample \(S\).
\item
  \(N_T\) denotes the total size of the target population \(T\).
\item
  \(\widehat{ATE}_{n_S}\) denotes the estimated average treatment effect
  in the study sample \(S\).
\item
  \(ATE_{T}\) denotes the true average treatment effect in the target
  population \(T\).
\end{itemize}

Assuming the rest of the causal inference workflow goes to plan
(randomisation succeeds, there is no measurement error, no model
misspecification, etc.), as the random sample size \(n_S\) increases,
the estimated treatment effect in the sample \(S\) converges in
probability to the true treatment effect in the target population \(T\):

\[
\lim_{n_S \to N_T} P(|\widehat{ATE}_{n_S} - ATE_{T}| < \epsilon) = 1
\]

for any small positive value of \(\epsilon\).

\textbf{Transportability:} We cannot directly generalise the findings
When the study sample is not drawn from the target population. However,
we can transport the estimated causal effect from the source population
to the target population under certain assumptions. This involves
adjusting for differences in the distributions of effect modifiers
between the two populations. The closer the source population is to the
target population, the more plausible the transportability assumptions
are, and the less we need to rely on complex adjustment methods.

Suppose we have a study sample \(n_S\) drawn from a source population
\(S\), and we want to estimate the average treatment effect in a target
population \(T\).

Define:

\begin{itemize}
\tightlist
\item
  \(\widehat{ATE}_{S}\) as the estimated average treatment effect in the
  source population \(S\).
\item
  \(\widehat{ATE}_{T}\) as the estimated average treatment effect in the
  target population \(T\).
\item
  \(f(n_S, R)\) as the mapping function that adjusts the estimated
  effect in the study sample using a set of measured covariates \(R\),
  allowing for valid projection to the target population.
\end{itemize}

The transportability assumption is that there exists a function \(f\)
such that:

\[
\widehat{ATE}_{T} = f(n_S, R)
\]

Finding a suitable function \(f\) is the central challenge in adjusting
for sampling bias and achieving transportability
(\citeproc{ref-bareinboim2013general}{Bareinboim and Pearl 2013};
\citeproc{ref-dahabreh2019generalizing}{Dahabreh \emph{et al.} 2019};
\citeproc{ref-deffner2022}{Deffner \emph{et al.} 2022};
\citeproc{ref-westreich2017transportability}{Westreich \emph{et al.}
2017b}).

\newpage{}

\subsection{Appendix C: Explanation for the Difference in Marginal
Effects between Censored and Uncensored Populations}\label{id-app-c}

\paragraph{Definitions:}\label{definitions}

\begin{itemize}
\tightlist
\item
  \textbf{\(A\)}: Exposure variable
\item
  \textbf{\(Y\)}: Outcome variable
\item
  \textbf{\(F\)}: Effect modifier
\item
  \textbf{\(C\)}: Indicator for the uncensored population (\(C = 0\)) or
  the censored population (\(C = 1\))
\end{itemize}

\paragraph{Average Treatment Effects:}\label{average-treatment-effects}

The average treatment effects for the uncensored and censored
populations are defined as:

\[
\Delta_{\text{uncensored}} = \mathbb{E}[Y(a^*) - Y(a) \mid C = 0]
\]

\[
\Delta_{\text{censored}} = \mathbb{E}[Y(a^*) - Y(a) \mid C = 1]
\]

\paragraph{Potential Outcomes:}\label{potential-outcomes}

By causal consistency, potential outcomes can be expressed in terms of
observed outcomes:

\[
\Delta_{\text{uncensored}} = \mathbb{E}[Y \mid A=a^*, C=0] - \mathbb{E}[Y \mid A=a, C=0]
\]

\[
\Delta_{\text{censored}} = \mathbb{E}[Y \mid A=a^*, C=1] - \mathbb{E}[Y \mid A=a, C=1]
\]

\paragraph{Law of Total Probability:}\label{law-of-total-probability}

Applying the Law of Total Probability, we can weight the average
treatment effects by the conditional probability of the effect modifier
\(F\):

\[
\Delta_{\text{uncensored}} = \sum_{f} \left\{\mathbb{E}[Y \mid A=a^*, F=f, C=0] - \mathbb{E}[Y \mid A=a, F=f, C=0]\right\} \times \Pr(F=f \mid C=0)
\]

\[
\Delta_{\text{censored}} = \sum_{f} \left\{\mathbb{E}[Y \mid A=a^*, F=f, C=1] - \mathbb{E}[Y \mid A=a, F=f, C=1]\right\} \times \Pr(F=f \mid C=1)
\]

\paragraph{Assumption of Informative
Censoring:}\label{assumption-of-informative-censoring}

We assume that the effect modifier \(F\) has a different distribution in
the censored and uncensored populations:

\[
\Pr(F=f \mid C=0) \neq \Pr(F=f \mid C=1)
\]

Under this assumption, the probability weights used to calculate the
marginal effects for the uncensored and censored populations differ.

\paragraph{Effect Estimates for Censored and Uncensored
Populations:}\label{effect-estimates-for-censored-and-uncensored-populations}

Given that \(\Pr(F=f \mid C=0) \neq \Pr(F=f \mid C=1)\), we cannot
guarantee that:

\[
\Delta_{\text{uncensored}} = \Delta_{\text{censored}}
\]

The equality of marginal effects between the two populations will only
hold if there is a universal null effect across all units, by chance, or
under specific conditions discussed by VanderWeele and Robins
(\citeproc{ref-vanderweele2007}{2007}) and further elucidated by Suzuki
\emph{et al.} (\citeproc{ref-suzuki2013counterfactual}{2013}).
Otherwise:

\[
\Delta_{\text{uncensored}} \ne \Delta_{\text{censored}}
\]

Furthermore, VanderWeele (\citeproc{ref-vanderweele2012}{2012}) proved
that if there is effect modification of \(A\) by \(F\), there will be a
difference in at least one scale of causal contrast, such that:

\[
\Delta^{\text{risk ratio}}_{\text{uncensored}} \ne \Delta^{\text{risk ratio}}_{\text{censored}}
\]

or

\[
\Delta^{\text{difference}}_{\text{uncensored}} \ne \Delta^{\text{difference}}_{\text{censored}}
\]

For comprehensive discussions on sampling and inference, refer to
Dahabreh and Hernán (\citeproc{ref-dahabreh2019}{2019}) and Dahabreh
\emph{et al.} (\citeproc{ref-dahabreh2021study}{2021}).

\newpage{}

\subsection{Appendix D: R Simulation to Clarify Why The Distribution of
Effect Modifiers Matters For Estimating Treatment Effects For A Target
Population}\label{id-app-d}

First, we load the \texttt{stdReg} library, which obtains marginal
effect estimates by simulating counterfactuals under different levels of
treatment (\citeproc{ref-sjuxf6lander2016}{Sjölander 2016}). If a
treatment is continuous, the levels can be specified.

We also load the \texttt{parameters} library, which creates nice tables
(\citeproc{ref-parameters2020}{Lüdecke \emph{et al.} 2020}).

\begin{Shaded}
\begin{Highlighting}[]
\CommentTok{\#|label: loadlibs}

\CommentTok{\# to obtain marginal effects}
\ControlFlowTok{if}\NormalTok{ (}\SpecialCharTok{!}\FunctionTok{requireNamespace}\NormalTok{(}\StringTok{\textquotesingle{}stdReg\textquotesingle{}}\NormalTok{, }\AttributeTok{quietly =} \ConstantTok{TRUE}\NormalTok{)) }\FunctionTok{install.packages}\NormalTok{(}\StringTok{\textquotesingle{}stdReg\textquotesingle{}}\NormalTok{)}
\FunctionTok{library}\NormalTok{(stdReg)}

\CommentTok{\#  to view data}
\ControlFlowTok{if}\NormalTok{ (}\SpecialCharTok{!}\FunctionTok{requireNamespace}\NormalTok{(}\StringTok{\textquotesingle{}skimr\textquotesingle{}}\NormalTok{, }\AttributeTok{quietly =} \ConstantTok{TRUE}\NormalTok{)) }\FunctionTok{install.packages}\NormalTok{(}\StringTok{\textquotesingle{}skimr\textquotesingle{}}\NormalTok{)}
\FunctionTok{library}\NormalTok{(skimr)}

\CommentTok{\# to create nice tables}
\ControlFlowTok{if}\NormalTok{ (}\SpecialCharTok{!}\FunctionTok{requireNamespace}\NormalTok{(}\StringTok{\textquotesingle{}parameters\textquotesingle{}}\NormalTok{, }\AttributeTok{quietly =} \ConstantTok{TRUE}\NormalTok{)) }\FunctionTok{install.packages}\NormalTok{(}\StringTok{\textquotesingle{}parameters\textquotesingle{}}\NormalTok{)}
\FunctionTok{library}\NormalTok{(parameters)}
\end{Highlighting}
\end{Shaded}

Next, we write a function to simulate data for the sample and target
populations.

We assume the treatment effect is the same in the sample and target
populations, that the coefficient for the effect modifier and the
coefficient for interaction are the same, that there is no unmeasured
confounding throughout the study, and that there is only selective
attrition of one effect modifier such that the baseline population
differs from the sample population at the end of the study.

That is: \textbf{the distribution of effect modifiers is the only
respect in which the sample will differ from the target population.}

This function will generate data under a range of scenarios.\footnote{See
  documentation in the \texttt{margot} package: Bulbulia
  (\citeproc{ref-margot2024}{2024c})}

\begin{Shaded}
\begin{Highlighting}[]
\CommentTok{\# function to generate data for the sample and population, }
\CommentTok{\# along with precise sample weights for the population, there are differences }
\CommentTok{\# in the distribution of the true effect modifier but no differences in the treatment effect }
\CommentTok{\# or the effect modification. all that differs between the sample and the population is }
\CommentTok{\# the distribution of effect{-}modifiers.}

\CommentTok{\# reproducibility}
\FunctionTok{set.seed}\NormalTok{(}\DecValTok{123}\NormalTok{)}

\CommentTok{\# simulate the data {-}{-} you can use different parameters}
\NormalTok{data }\OtherTok{\textless{}{-}}\NormalTok{ margot}\SpecialCharTok{::}\FunctionTok{simulate\_ate\_data\_with\_weights}\NormalTok{(}
  \AttributeTok{n\_sample =} \DecValTok{10000}\NormalTok{,}
  \AttributeTok{n\_population =} \DecValTok{100000}\NormalTok{,}
  \AttributeTok{p\_z\_sample =} \FloatTok{0.1}\NormalTok{,}
  \AttributeTok{p\_z\_population =} \FloatTok{0.5}\NormalTok{,}
  \AttributeTok{beta\_a =} \DecValTok{1}\NormalTok{,}
  \AttributeTok{beta\_z =} \FloatTok{2.5}\NormalTok{,}
  \AttributeTok{noise\_sd =} \FloatTok{0.5}
\NormalTok{)}

\CommentTok{\# inspect}
\CommentTok{\# skimr::skim(data)}
\end{Highlighting}
\end{Shaded}

We have generated both sample and population data.

Next, we verify that the distributions of effect modifiers differ in the
sample and in the target population:

\begin{Shaded}
\begin{Highlighting}[]
\CommentTok{\# obtain the generated data}
\NormalTok{sample\_data }\OtherTok{\textless{}{-}}\NormalTok{ data}\SpecialCharTok{$}\NormalTok{sample\_data}
\NormalTok{population\_data }\OtherTok{\textless{}{-}}\NormalTok{ data}\SpecialCharTok{$}\NormalTok{population\_data}

\CommentTok{\# check imbalance}
\FunctionTok{table}\NormalTok{(sample\_data}\SpecialCharTok{$}\NormalTok{z\_sample) }\CommentTok{\# type 1 is rare}
\end{Highlighting}
\end{Shaded}

\begin{verbatim}

   0    1 
9055  945 
\end{verbatim}

\begin{Shaded}
\begin{Highlighting}[]
\FunctionTok{table}\NormalTok{(population\_data}\SpecialCharTok{$}\NormalTok{z\_population) }\CommentTok{\# type 1 is common}
\end{Highlighting}
\end{Shaded}

\begin{verbatim}

    0     1 
49916 50084 
\end{verbatim}

The sample and population distributions differ.

Next, consider the question: `What are the differences in the
coefficients that we obtain from the study population at the end of the
study, compared with those we would obtain for the target population?'

First, we obtain the regression coefficients for the sample. They are as
follows:

\begin{Shaded}
\begin{Highlighting}[]
\CommentTok{\# model coefficients sample}
\NormalTok{model\_sample  }\OtherTok{\textless{}{-}} \FunctionTok{glm}\NormalTok{(y\_sample }\SpecialCharTok{\textasciitilde{}}\NormalTok{ a\_sample }\SpecialCharTok{*}\NormalTok{ z\_sample, }
  \AttributeTok{data =}\NormalTok{ sample\_data)}

\CommentTok{\# summary}
\NormalTok{parameters}\SpecialCharTok{::}\FunctionTok{model\_parameters}\NormalTok{(model\_sample, }\AttributeTok{ci\_method =} \StringTok{\textquotesingle{}wald\textquotesingle{}}\NormalTok{)}
\end{Highlighting}
\end{Shaded}

\begin{verbatim}
Parameter           | Coefficient |       SE |        95% CI | t(9996) |      p
-------------------------------------------------------------------------------
(Intercept)         |   -6.89e-03 | 7.38e-03 | [-0.02, 0.01] |   -0.93 | 0.350 
a sample            |        1.01 |     0.01 | [ 0.99, 1.03] |   95.84 | < .001
z sample            |        2.47 |     0.02 | [ 2.43, 2.52] |  104.09 | < .001
a sample × z sample |        0.51 |     0.03 | [ 0.44, 0.57] |   14.82 | < .001
\end{verbatim}

Next, we obtain the regression coefficients for the weighted regression
of the sample. Notice that the coefficients are virtually the same:

\begin{Shaded}
\begin{Highlighting}[]
\CommentTok{\# model the sample weighted to the population, again note that these coefficients are similar }
\NormalTok{model\_weighted\_sample }\OtherTok{\textless{}{-}} \FunctionTok{glm}\NormalTok{(y\_sample }\SpecialCharTok{\textasciitilde{}}\NormalTok{ a\_sample }\SpecialCharTok{*}\NormalTok{ z\_sample, }
  \AttributeTok{data =}\NormalTok{ sample\_data, }\AttributeTok{weights =}\NormalTok{ weights)}

\CommentTok{\# summary}
\FunctionTok{summary}\NormalTok{(parameters}\SpecialCharTok{::}\FunctionTok{model\_parameters}\NormalTok{(model\_weighted\_sample, }
  \AttributeTok{ci\_method =} \StringTok{\textquotesingle{}wald\textquotesingle{}}\NormalTok{))}
\end{Highlighting}
\end{Shaded}

\begin{verbatim}
Parameter           | Coefficient |        95% CI |      p
----------------------------------------------------------
(Intercept)         |   -6.89e-03 | [-0.03, 0.01] | 0.480 
a sample            |        1.01 | [ 0.98, 1.04] | < .001
z sample            |        2.47 | [ 2.45, 2.50] | < .001
a sample × z sample |        0.51 | [ 0.47, 0.55] | < .001

Model: y_sample ~ a_sample * z_sample (10000 Observations)
Residual standard deviation: 0.494 (df = 9996)
\end{verbatim}

We might be tempted to infer that weighting wasn't relevant to the
analysis. However, we'll see that such an interpretation would be a
mistake.

Next, we obtain model coefficients for the population. Note again there
is no difference -- only narrower errors owing to the large sample size.

\begin{Shaded}
\begin{Highlighting}[]
\CommentTok{\# model coefficients population {-}{-} note that these coefficients are very similar. }
\NormalTok{model\_population }\OtherTok{\textless{}{-}} \FunctionTok{glm}\NormalTok{(y\_population }\SpecialCharTok{\textasciitilde{}}\NormalTok{ a\_population }\SpecialCharTok{*}\NormalTok{ z\_population, }
  \AttributeTok{data =}\NormalTok{ population\_data)}

\NormalTok{parameters}\SpecialCharTok{::}\FunctionTok{model\_parameters}\NormalTok{(model\_population, }\AttributeTok{ci\_method =} \StringTok{\textquotesingle{}wald\textquotesingle{}}\NormalTok{)}
\end{Highlighting}
\end{Shaded}

\begin{verbatim}
Parameter                   | Coefficient |       SE |        95% CI | t(99996) |      p
----------------------------------------------------------------------------------------
(Intercept)                 |    2.49e-03 | 3.18e-03 | [ 0.00, 0.01] |     0.78 | 0.434 
a population                |        1.00 | 4.49e-03 | [ 0.99, 1.01] |   222.35 | < .001
z population                |        2.50 | 4.49e-03 | [ 2.49, 2.51] |   556.80 | < .001
a population × z population |        0.50 | 6.35e-03 | [ 0.49, 0.51] |    78.80 | < .001
\end{verbatim}

Again, there is no difference. That is, we find that all model
coefficients are practically equivalent. The different distribution of
effect modifiers does not result in different coefficient values for the
treatment effect, the effect-modifier `effect,' or the interaction of
the effect modifier and treatment.

Consider why this is the case: in a large sample where the causal
effects are invariant -- as we have simulated them to be -- we will have
good replication in the effect modifiers within the sample, so our
statistical model can recover the \emph{coefficients} for the population
without challenge.

However, \emph{in causal inference, we are interested in the marginal
effect of the treatment. That is, we seek an estimate for the
counterfactual }contrast* in which everyone in a pre-specified
population was subject to one level of treatment compared with a
counterfactual condition in which everyone in a population was subject
to another level of the same treatment.*

\textbf{The marginal effect estimates will typically differ When the
sample population differs in the distribution of effect modifiers from
the target population effect.}

To see this, we use the \texttt{stdReg} package to recover marginal
effect estimates, comparing (1) the sample ATE, (2) the true oracle ATE
for the population, and (3) the weighted sample ATE. We will use the
outputs of the same models above. The only difference is that we will
calculate marginal effects from these outputs. We will contrast a
difference from an intervention in which everyone receives treatment = 0
with one in which everyone receives treatment = 1; however, this choice
is arbitrary, and the general lessons apply irrespective of the
estimand.

First, consider this Average Treatment Effect for the sample population:

\begin{Shaded}
\begin{Highlighting}[]
\CommentTok{\# What inference do we draw?  }
\CommentTok{\# we cannot say the models are unbiased for the marginal effect estimates. }
\CommentTok{\# regression standardisation }
\FunctionTok{library}\NormalTok{(stdReg) }\CommentTok{\# to obtain marginal effects }

\CommentTok{\# obtain sample ate}
\NormalTok{fit\_std\_sample }\OtherTok{\textless{}{-}}\NormalTok{ stdReg}\SpecialCharTok{::}\FunctionTok{stdGlm}\NormalTok{(model\_sample, }
  \AttributeTok{data =}\NormalTok{ sample\_data, }\AttributeTok{X =} \StringTok{\textquotesingle{}a\_sample\textquotesingle{}}\NormalTok{)}

\CommentTok{\# summary}
\FunctionTok{summary}\NormalTok{(fit\_std\_sample, }\AttributeTok{contrast =} \StringTok{\textquotesingle{}difference\textquotesingle{}}\NormalTok{, }\AttributeTok{reference =} \DecValTok{0}\NormalTok{)}
\end{Highlighting}
\end{Shaded}

\begin{verbatim}

Formula: y_sample ~ a_sample * z_sample
Family: gaussian 
Link function: identity 
Exposure:  a_sample 
Reference level:  a_sample = 0 
Contrast:  difference 

  Estimate Std. Error lower 0.95 upper 0.95
0     0.00     0.0000       0.00       0.00
1     1.06     0.0101       1.04       1.08
\end{verbatim}

The treatment effect is given as a 1.06 unit change in the outcome
across the sample population, with a confidence interval from 1.04 to
1.08.

Next, we obtain the true (oracle) treatment effect for the population
under the same intervention:

\begin{Shaded}
\begin{Highlighting}[]
\DocumentationTok{\#\# note the population effect is different}

\CommentTok{\# obtain true ate}
\NormalTok{fit\_std\_population }\OtherTok{\textless{}{-}}\NormalTok{ stdReg}\SpecialCharTok{::}\FunctionTok{stdGlm}\NormalTok{(model\_population, }
  \AttributeTok{data =}\NormalTok{ population\_data, }\AttributeTok{X =} \StringTok{\textquotesingle{}a\_population\textquotesingle{}}\NormalTok{)}

\CommentTok{\# summary}
\FunctionTok{summary}\NormalTok{(fit\_std\_population, }\AttributeTok{contrast =} \StringTok{\textquotesingle{}difference\textquotesingle{}}\NormalTok{, }\AttributeTok{reference =} \DecValTok{0}\NormalTok{)}
\end{Highlighting}
\end{Shaded}

\begin{verbatim}

Formula: y_population ~ a_population * z_population
Family: gaussian 
Link function: identity 
Exposure:  a_population 
Reference level:  a_population = 0 
Contrast:  difference 

  Estimate Std. Error lower 0.95 upper 0.95
0     0.00    0.00000       0.00       0.00
1     1.25    0.00327       1.24       1.26
\end{verbatim}

Note, the true treatment effect is a 1.25 unit change in the population,
with a confidence bound between 1.24 and 1.26. This is well outside the
ATE that we obtain from the sample population!

Next, consider the ATE in the weighted regression, where the sample was
weighted to the target population's true distribution of effect
modifiers:

\begin{Shaded}
\begin{Highlighting}[]
\DocumentationTok{\#\# next try weights adjusted ate where we correctly assign population weights to the sample}
\NormalTok{fit\_std\_weighted\_sample\_weights }\OtherTok{\textless{}{-}}\NormalTok{ stdReg}\SpecialCharTok{::}\FunctionTok{stdGlm}\NormalTok{(model\_weighted\_sample, }
  \AttributeTok{data =}\NormalTok{ sample\_data, }\AttributeTok{X =} \StringTok{\textquotesingle{}a\_sample\textquotesingle{}}\NormalTok{)}

\CommentTok{\# this gives us the right answer}
\FunctionTok{summary}\NormalTok{(fit\_std\_weighted\_sample\_weights, }\AttributeTok{contrast =} \StringTok{\textquotesingle{}difference\textquotesingle{}}\NormalTok{, }\AttributeTok{reference =} \DecValTok{0}\NormalTok{)}
\end{Highlighting}
\end{Shaded}

\begin{verbatim}

Formula: y_sample ~ a_sample * z_sample
Family: gaussian 
Link function: identity 
Exposure:  a_sample 
Reference level:  a_sample = 0 
Contrast:  difference 

  Estimate Std. Error lower 0.95 upper 0.95
0     0.00     0.0000       0.00       0.00
1     1.25     0.0172       1.22       1.29
\end{verbatim}

We find that we obtain the population-level causal effect estimate with
accurate coverage by weighting the sample to the target population. So
with appropriate weights, our results generalise from the sample to the
target population.

\subsection{Lessons}\label{lessons}

\begin{itemize}
\tightlist
\item
  \textbf{Regression coefficients do not clarify the problem of
  sample/target population mismatch} --- or selection bias as discussed
  in this manuscript.
\item
  \textbf{Investigators should not rely on regression coefficients
  alone} when evaluating the biases that arise from sample attrition.
  This advice applies to both methods that authors use to investigate
  threats of bias. To implement this advice, authors must first take it
  themselves.
\item
  \textbf{Observed data are generally insufficient for assessing
  threats}. Observed data do not clarify structural sources of bias, nor
  do they clarify effect-modification in the full counterfactual data
  condition where all receive the treatment and all do not receive the
  treatment (at the same level).
\item
  \textbf{To properly assess bias, one needs access to the
  counterfactual outcome} --- what would have happened to the missing
  participants had they not been lost to follow-up or had they
  responded. The joint distributions over `full data' are inherently
  unobservable (\citeproc{ref-vanderlaan2011}{Van Der Laan and Rose
  2011}).
\item
  \textbf{In simple settings, like the one we just simulated, we can
  address the gap between the sample and target population using methods
  such as modelling the censoring (e.g., censoring weighting).} However,
  we never know what setting we are in or whether it is simple---such
  modelling must be handled carefully. There is a large and growing
  epidemiology literature on this topic (see, for example, Li \emph{et
  al.} (\citeproc{ref-li2023non}{2023})).
\end{itemize}

\newpage{}

\subsection{Appendix E: Relating Causal Directed Acyclic Graphs to
Probability Distributions Using Pearl's Non-Parametric Structural
Equation Models}\label{id-app-E}

In the potential outcomes framework, we represent interventions by
setting variables to specific levels, e.g., setting the treatment to a
specific value \(A = \tilde{a}\). Counterfactual outcomes are conceived
as the outcomes that would occur if, perhaps contrary to fact, the
treatment were set to a specific level. This is denoted as
\(Y(\tilde{a})\). To evaluate the conditional exchangeability assumption
of causal inference, we must ensure that the \emph{potential outcomes}
are independent of the treatments to be compared, conditional on
measured covariates \(L\):

\[
A \coprod Y(\tilde{a})|L
\]

Causal DAGs do not directly represent counterfactual outcomes. Instead,
they evaluate whether causality can be identified from hypothetical
interventions on the variables represented in a graph. Formally, causal
directed acyclic graphs rely on Judea Pearl's do-calculus
(\citeproc{ref-pearl2009a}{Pearl 2009}), which appeals to the concept of
an `interventional distribution'. The idea is that any node in a graph
can be intervened upon. Here, nodes and edges in a causal diagram
correspond to non-parametric structural equations. For instance, if
\(L\) denotes the common causes of treatment \(A\) and outcome \(Y\),
then: - The node \(L\) corresponds to the non-parametric structural
equation model \(f_L(U_L)\). - The treatment \(A = f_A(L, U_A)\). - The
outcome \(Y = f_Y(A, L, U_Y)\).

Next, we assume that \(U_L, U_A, U_Y\) are independent exogenous random
variables. Define \(O\) as a distribution of independent identically
distributed observations such that \(O = (L, A, Y)\). The true
distribution \(P_O\) is factorised as:

\[
P_O = P_O(Y|A, L) P_O(A|L) P_O(L)
\]

Where: - \(P_O(L)\) is the marginal distribution of the covariates
\(L\). - \(P_O(A|L)\) is the conditional distribution of the treatment
given the covariates. - \(P_O(Y|A, L)\) is the conditional expectation
of the outcome given the treatment and covariates.

Pearl's do-calculus allows us to evaluate the consequences of
intervening on variables represented in a causal DAG to interpret
probabilistic dependencies and independencies in the conditional and
marginal associations presented on a graph. By contrast, the potential
outcomes framework considers potential outcomes to be fixed and real
(even if assigned non-deterministically).

It is not inherently problematic that causal DAGs do not explicitly
represent potential outcomes. The theory of counterfactual inference on
which causal DAGs are based does not require this explicit
representation. Judea Pearl's rules of d-separation and the backdoor
adjustment criterion allow us to interpret conditional independencies
based on hypothetical interventions on the nodes of a causal DAG
(\citeproc{ref-neal2020introduction}{Neal 2020};
\citeproc{ref-pearl2009a}{Pearl 2009}). However, as we consider in
\hyperref[id-sec-4]{Part 4} and \hyperref[id-app-f]{Appendix F},
presenting counterfactual histories under specific interventions on
Single World Intervention Graphs can help clarify structural
relationships that are not well-represented on a causal DAG.

\newpage{}

\subsection{Appendix F: Bias Correction as Interventions on
Reporters}\label{id-app-F}

\begin{table}

\caption{\label{tbl-tblswigme}Single World Intervention Graph reveals
strategies for redressing measurement error.}

\centering{

\tblswigme

}

\end{table}%

Single World Intervention Graphs (SWIGs) help us understand why bias
correction works. We can think of bias correction without relying on
mathematically restrictive models by considering reporters of the true
but unobserved states of the world as elements of a causal reality that
we represent in SWIGs.

Table~\ref{tbl-tblswigme}\(\mathcal{G}_{1.1}\) shows how to represent
the true counterfactual outcome as a function
\(Y(\mathbf{h}(E^A, B(\tilde{a})))\). If this function were known, we
could intervene to correct the bias in reporter \(B\) when
\(A = \tilde{a}\) to obtain \(Y(\tilde{a})\). The dotted green arrows
indicate the counterfactual variables whose functional relationship to
the observed values \(B(\tilde{a})\) are relevant for correcting this
bias. Like an optometrist fitting spectacles to correct vision, knowing
how \(B(\tilde{a})\) relates to \(E^A\) would allow us to recover
\(A = \tilde{a}\) from \(B(\tilde{a})\) and thus obtain
\(E[Y(\tilde{a})]\) from \(E[Y(B(\tilde{a}))]\).

Similarly, Table~\ref{tbl-tblswigme}\(\mathcal{G}_{1.2}\) shows how to
represent the true counterfactual outcome as a function
\(V(\mathbf{h}(E^Y, V(\tilde{a})))\). If this function were known, we
could intervene to correct the bias of outcome reporter \(V(\tilde{a})\)
when \(A = \tilde{a}\) to recover the true state \(Y(\tilde{a})\) from
its distorted representation in \(V(\tilde{a})\). The dotted green
arrows indicate the counterfactual variables relevant for correcting
this bias. Knowing how \(V(\tilde{a})\) relates to \(E^Y\) would allow
us to recover \(Y(\tilde{a})\) from \(V(\tilde{a})\).

Table~\ref{tbl-tblswigmex}\(\mathcal{G}_{1.1-1.2}\) reveals that
obtaining corrections for biased reporters requires additional
information when there is directed measurement error. In this setting,
bias correction requires knowledge of a function in which the treatment
and unmeasured sources of error interact to distort reported potential
outcomes under treatment. The SWIG shows that directed measurement error
bias can occur if the treatment affects the outcome reporter, even
without a direct effect of the treatment on the error terms of the
outcome reporter.

Table~\ref{tbl-tblswigmex}\(\mathcal{G}_{2.1-2.2}\) clarifies that
correlated biases in the errors of the treatment and outcome reporters
create additional demands for measurement error correction. The
behaviour of the correlated error must be evaluated for both
\(B(\tilde{a})\) and \(V(\tilde{a})\). To obtain \(V(\tilde{a})\), we
must first obtain \(\tilde{a}\) from a function
\(f_{B}(B(\tilde{a}), E^{AY})\), which cannot be derived from the data
because \(E^{AY}\) is unobserved. Similarly, a function that recovers
\(Y(\tilde{a})\) from \(V(\tilde{a})\) cannot be obtained from the data
because of the unobserved \(E^{AY}\). Further complications arise when
considering bias in settings with both directed and correlated
measurement error.

Recall from \hyperref[id-sec-3]{Part 3} that we considered how the
distribution of effect modifiers across populations complicates
inference. These problems are compounded when we include treatment and
outcome reporters in our SWIGs. Even if treatment effects were constant
across populations, there might be effect modification in the
mis-measurement of treatments across populations. Statistical tests
alone cannot distinguish between effect modification from treatment
effect heterogeneity and effect modification from heterogeneous
reporting of treatments or outcomes.

\begin{table}

\caption{\label{tbl-tblswigmex}Single World Intervention Graph reveals
strategies for redressing measurement error.}

\centering{

\tblswigmex

}

\end{table}%

\subsubsection{Summary}\label{summary-2}

Our interest in SWIGs has been to understand the causal underpinnings of
certain population restriction biases and measurement error biases that
arise in the absence of confounding biases. Even assuming strong
sequential exchangeability, we can use SWIGs to clarify the mechanisms
by which non-confounding biases operate, methods for correcting such
biases, and the challenges of comparative research where the
distribution of effect modifiers of bias in reporters must be considered
to obtain valid causal contrasts for potential outcomes under treatment.

Considerations when using Single World Intervention Graphs for
clarifying structural sources of measurement error bias (and other
biases):

\begin{enumerate}
\def\labelenumi{\arabic{enumi}.}
\tightlist
\item
  There must be a directed edge from a latent variable to its reporter.
\item
  If the reporter of a treatment has an arrow entering it from another
  variable, and causal contrasts are obtained from outcomes under
  reported treatments, there will generally be measurement error bias on
  at least one causal contrast scale (ignoring accidental cancellations
  of errors), see \hyperref[id-sec-4]{Part 4}.
\item
  Likewise, if the reporter of an outcome has an arrow entering it from
  another variable, and causal contrasts are obtained from reported
  outcomes, there will generally be measurement error bias on at least
  one causal contrast scale (ignoring accidental cancellations of
  errors), see \hyperref[id-sec-4]{Part 4}.
\item
  We cannot often control for measurement error biases by
  \emph{conditioning} on variables in the model because these biases are
  not confounding biases.
\item
  However, if the functions that lead to differences between unobserved
  variables of interest and their reporters are known, investigators can
  correct for such differences by reweighting the data or applying
  direct corrections (\citeproc{ref-carroll2006measurement}{Carroll
  \emph{et al.} 2006}; \citeproc{ref-lash2009applying}{Lash \emph{et
  al.} 2009}).
\item
  Certain population restriction biases can be viewed as varieties of
  measurement error bias, as discussed in \hyperref[id-sec-2]{Part 2}
  and \hyperref[id-sec-3]{Part 3}. SWIGs clarify that certain
  measurement error biases arise from effect modification, where the
  error term interacts with the underlying variable of interest, as
  discussed in \hyperref[id-sec-4]{Part 4}.
\item
  Approaching measurement error as effect modification using SWIGs is
  useful because errors might not operate at all intervention levels.
  For example, if an intervention is continuous and measurement error in
  the reporter arises only when the intervention is more than one
  standard deviation above average, and no corrections are available,
  investigators might restrict their attention to causal comparisons
  within the unbiased range. Causal DAGs do not readily allow
  investigators to appreciate these prospects.
\item
  Despite the formal equivalence of certain forms of measurement error
  bias and certain forms of population restriction bias, we may use
  Single World Intervention Graphs to show that both biases may operate
  together and in conjunction with confounding biases. We would do so by
  adding effect modifier nodes to the SWIGs in Table~\ref{tbl-tblswigme}
  and Table~\ref{tbl-tblswigmex}.
\item
  Despite the utility of Single World Intervention Graphs (and causal
  DAGs) for clarifying structural features of bias, whether confounding
  or otherwise, investigators should not be distracted from the goal in
  using these tools: to understand whether and how valid causal effects
  may be obtained from observational data for the populations of
  interest. Every inclination to use causal diagrams should be resisted
  if their use complicates this objective.
\end{enumerate}



\end{document}
