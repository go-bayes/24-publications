% Options for packages loaded elsewhere
\PassOptionsToPackage{unicode}{hyperref}
\PassOptionsToPackage{hyphens}{url}
\PassOptionsToPackage{dvipsnames,svgnames,x11names}{xcolor}
%
\documentclass[
  single column]{article}

\usepackage{amsmath,amssymb}
\usepackage{iftex}
\ifPDFTeX
  \usepackage[T1]{fontenc}
  \usepackage[utf8]{inputenc}
  \usepackage{textcomp} % provide euro and other symbols
\else % if luatex or xetex
  \usepackage{unicode-math}
  \defaultfontfeatures{Scale=MatchLowercase}
  \defaultfontfeatures[\rmfamily]{Ligatures=TeX,Scale=1}
\fi
\usepackage[]{libertinus}
\ifPDFTeX\else  
    % xetex/luatex font selection
\fi
% Use upquote if available, for straight quotes in verbatim environments
\IfFileExists{upquote.sty}{\usepackage{upquote}}{}
\IfFileExists{microtype.sty}{% use microtype if available
  \usepackage[]{microtype}
  \UseMicrotypeSet[protrusion]{basicmath} % disable protrusion for tt fonts
}{}
\makeatletter
\@ifundefined{KOMAClassName}{% if non-KOMA class
  \IfFileExists{parskip.sty}{%
    \usepackage{parskip}
  }{% else
    \setlength{\parindent}{0pt}
    \setlength{\parskip}{6pt plus 2pt minus 1pt}}
}{% if KOMA class
  \KOMAoptions{parskip=half}}
\makeatother
\usepackage{xcolor}
\usepackage[top=30mm,left=20mm,heightrounded]{geometry}
\setlength{\emergencystretch}{3em} % prevent overfull lines
\setcounter{secnumdepth}{-\maxdimen} % remove section numbering
% Make \paragraph and \subparagraph free-standing
\ifx\paragraph\undefined\else
  \let\oldparagraph\paragraph
  \renewcommand{\paragraph}[1]{\oldparagraph{#1}\mbox{}}
\fi
\ifx\subparagraph\undefined\else
  \let\oldsubparagraph\subparagraph
  \renewcommand{\subparagraph}[1]{\oldsubparagraph{#1}\mbox{}}
\fi


\providecommand{\tightlist}{%
  \setlength{\itemsep}{0pt}\setlength{\parskip}{0pt}}\usepackage{longtable,booktabs,array}
\usepackage{calc} % for calculating minipage widths
% Correct order of tables after \paragraph or \subparagraph
\usepackage{etoolbox}
\makeatletter
\patchcmd\longtable{\par}{\if@noskipsec\mbox{}\fi\par}{}{}
\makeatother
% Allow footnotes in longtable head/foot
\IfFileExists{footnotehyper.sty}{\usepackage{footnotehyper}}{\usepackage{footnote}}
\makesavenoteenv{longtable}
\usepackage{graphicx}
\makeatletter
\def\maxwidth{\ifdim\Gin@nat@width>\linewidth\linewidth\else\Gin@nat@width\fi}
\def\maxheight{\ifdim\Gin@nat@height>\textheight\textheight\else\Gin@nat@height\fi}
\makeatother
% Scale images if necessary, so that they will not overflow the page
% margins by default, and it is still possible to overwrite the defaults
% using explicit options in \includegraphics[width, height, ...]{}
\setkeys{Gin}{width=\maxwidth,height=\maxheight,keepaspectratio}
% Set default figure placement to htbp
\makeatletter
\def\fps@figure{htbp}
\makeatother
% definitions for citeproc citations
\NewDocumentCommand\citeproctext{}{}
\NewDocumentCommand\citeproc{mm}{%
  \begingroup\def\citeproctext{#2}\cite{#1}\endgroup}
\makeatletter
 % allow citations to break across lines
 \let\@cite@ofmt\@firstofone
 % avoid brackets around text for \cite:
 \def\@biblabel#1{}
 \def\@cite#1#2{{#1\if@tempswa , #2\fi}}
\makeatother
\newlength{\cslhangindent}
\setlength{\cslhangindent}{1.5em}
\newlength{\csllabelwidth}
\setlength{\csllabelwidth}{3em}
\newenvironment{CSLReferences}[2] % #1 hanging-indent, #2 entry-spacing
 {\begin{list}{}{%
  \setlength{\itemindent}{0pt}
  \setlength{\leftmargin}{0pt}
  \setlength{\parsep}{0pt}
  % turn on hanging indent if param 1 is 1
  \ifodd #1
   \setlength{\leftmargin}{\cslhangindent}
   \setlength{\itemindent}{-1\cslhangindent}
  \fi
  % set entry spacing
  \setlength{\itemsep}{#2\baselineskip}}}
 {\end{list}}
\usepackage{calc}
\newcommand{\CSLBlock}[1]{\hfill\break\parbox[t]{\linewidth}{\strut\ignorespaces#1\strut}}
\newcommand{\CSLLeftMargin}[1]{\parbox[t]{\csllabelwidth}{\strut#1\strut}}
\newcommand{\CSLRightInline}[1]{\parbox[t]{\linewidth - \csllabelwidth}{\strut#1\strut}}
\newcommand{\CSLIndent}[1]{\hspace{\cslhangindent}#1}

\usepackage{booktabs}
\usepackage{longtable}
\usepackage{array}
\usepackage{multirow}
\usepackage{wrapfig}
\usepackage{float}
\usepackage{colortbl}
\usepackage{pdflscape}
\usepackage{tabu}
\usepackage{threeparttable}
\usepackage{threeparttablex}
\usepackage[normalem]{ulem}
\usepackage{makecell}
\usepackage{xcolor}
\input{/Users/joseph/GIT/latex/latex-for-quarto.tex}
\makeatletter
\@ifpackageloaded{caption}{}{\usepackage{caption}}
\AtBeginDocument{%
\ifdefined\contentsname
  \renewcommand*\contentsname{Table of contents}
\else
  \newcommand\contentsname{Table of contents}
\fi
\ifdefined\listfigurename
  \renewcommand*\listfigurename{List of Figures}
\else
  \newcommand\listfigurename{List of Figures}
\fi
\ifdefined\listtablename
  \renewcommand*\listtablename{List of Tables}
\else
  \newcommand\listtablename{List of Tables}
\fi
\ifdefined\figurename
  \renewcommand*\figurename{Figure}
\else
  \newcommand\figurename{Figure}
\fi
\ifdefined\tablename
  \renewcommand*\tablename{Table}
\else
  \newcommand\tablename{Table}
\fi
}
\@ifpackageloaded{float}{}{\usepackage{float}}
\floatstyle{ruled}
\@ifundefined{c@chapter}{\newfloat{codelisting}{h}{lop}}{\newfloat{codelisting}{h}{lop}[chapter]}
\floatname{codelisting}{Listing}
\newcommand*\listoflistings{\listof{codelisting}{List of Listings}}
\makeatother
\makeatletter
\makeatother
\makeatletter
\@ifpackageloaded{caption}{}{\usepackage{caption}}
\@ifpackageloaded{subcaption}{}{\usepackage{subcaption}}
\makeatother
\ifLuaTeX
  \usepackage{selnolig}  % disable illegal ligatures
\fi
\usepackage{bookmark}

\IfFileExists{xurl.sty}{\usepackage{xurl}}{} % add URL line breaks if available
\urlstyle{same} % disable monospaced font for URLs
\hypersetup{
  pdftitle={Causal Effects of Religious Service Attendance: Evidence Using Novel Measures From A National Longitudinal Panel},
  pdfauthor={Joseph A. Bulbulia; Don E Davis; Kenneth G. Rice; Chris G. Sibley; Geoffrey Troughton},
  pdfkeywords={Causal
Inference, Charity, Church, Cooperation, Religion, Shift
Intervention, Volunteering},
  colorlinks=true,
  linkcolor={blue},
  filecolor={Maroon},
  citecolor={Blue},
  urlcolor={Blue},
  pdfcreator={LaTeX via pandoc}}

\title{Causal Effects of Religious Service Attendance: Evidence Using
Novel Measures From A National Longitudinal Panel}
\author{Joseph A. Bulbulia \and Don E Davis \and Kenneth G.
Rice \and Chris G. Sibley \and Geoffrey Troughton}
\date{2024-04-18}

\begin{document}
\maketitle
\begin{abstract}
Causal investigations for the effects of religion on prosociality must
be precise. One must articulate a specific causal contrast for a feature
of religion, select appropriate prosociality measures, define the target
population, gather time-series data, and, only after meeting
identification assumptions, conduct statistical and sensitivity
analyses. Here, we examine three distinct interventions on religious
service attendance (increase, decrease, maintain) across a longitudinal
dataset of 33,198 New Zealanders from 2018 to 2021. Study 1 quantifies
the effect of religious service attendance on charitable contributions
and volunteerism. Studies 2 and 3 investigate the effects of religious
service on the relative risk of receiving aid and financial support from
others during the past week, employing measures designed to minimise
self-reporting bias. Across all studies, the observed effects are
substantially less pronounced than cross-sectional regressions might
imply. Nonetheless, regular attendance across the population would
enhance charitable donations by 4\% of the New Zealand Government's
annual spending. This research underscores the essential role of
formulating precise causal questions and establishes a methodological
framework for answering them in the scientific study of cultural
practices.
\end{abstract}

\subsection{Introduction}\label{introduction}

A central question in the scientific study of religion is whether
religion fosters cooperation (\citeproc{ref-bulbuliaj.2013}{Bulbulia, J.
\emph{et al.} 2013}; \citeproc{ref-johnson2005}{Johnson 2005};
\citeproc{ref-norenzayan2016}{Norenzayan \emph{et al.} 2016};
\citeproc{ref-sibley2012a}{Sibley and Bulbulia 2012};
\citeproc{ref-sosis2003cooperation}{Sosis and Bressler 2003};
\citeproc{ref-watts2015}{Watts \emph{et al.} 2015};
\citeproc{ref-watts2016}{Watts \emph{et al.} 2016}). However,
quantifying causal effects for religion, and many other social
behaviours, presents significant challenges. Much of religion eludes
randomised experiments. Additionally, valid causal conclusions from
observational data require both high-resolution time-series data and
robust methods for causal inference. Few studies combine these
characteristics (refer to Kelly \emph{et al.}
(\citeproc{ref-kelly2024religiosity}{2024}) for a recent research
summary and Chen \emph{et al.} (\citeproc{ref-chen2020religious}{2020})
for a well-designed observational study on religious service and mental
health).

Moreover, the question ``Does religion cause prosociality?'' lacks
specificity. To refine this question, we must articulate clear causal
contrasts and their scale, select specific measures of ``prosociality,''
define our target population, gather appropriate time-series data, and,
assuming that causal assumptions and identification criteria are met,
calculate statistical estimates. Having obtained these estimates, which
rely on assumptions, we must evaluate robustness using sensitivity
analyses (\citeproc{ref-bulbulia2022}{Bulbulia 2022};
\citeproc{ref-hernan2024WHATIF}{Hernan and Robins 2024};
\citeproc{ref-linden2020EVALUE}{Linden \emph{et al.} 2020};
\citeproc{ref-ogburn2021}{Ogburn and Shpitser 2021}).

Here, we use comprehensive panel data from 33,198 participants in the
New Zealand Attitudes and Values Study from 2018-2021 to quantify the
effects of clearly defined interventions in religious attendance across
the population of New Zealanders on two features of prosociality:
charitable financial donations and volunteering, measured both by
self-reported giving as well as help received.

We define causal effects as quantitative contrasts between potential
outcomes under intervention on religious service attendance, the
``treatment,'' within a specified population (see Rubin
(\citeproc{ref-rubin2005}{2005}); Splawa-Neyman \emph{et al.}
(\citeproc{ref-neyman1923}{1990}), Robins
(\citeproc{ref-robins1986}{1986}); our approach is consistent with the
counterfactual approaches Pearl (\citeproc{ref-pearl2009}{2009}), Van
Der Laan and Rose (\citeproc{ref-vanderlaan2018}{2018})). We express
these interventions as ``modified treatment policies'' and obtain causal
inferences by contrasting inferred population treatment averages
(\citeproc{ref-duxedaz2021}{Díaz \emph{et al.} 2021},
\citeproc{ref-diaz2023lmtp}{2023};
\citeproc{ref-haneuse2013estimation}{Haneuse and Rotnitzky 2013};
\citeproc{ref-hoffman2023}{Hoffman \emph{et al.} 2023}).

Our initial causal contrast investigates: ``What would be the average
difference across the New Zealand population if everyone attended
religious services regularly (at least four times per month) versus if
no one attended?'' This theoretical question simulates a hypothetical
experiment with random assignment to regular or non-attendance.

The second contrast investigates: ``What would be the average difference
across the New Zealand population if everyone attended religious
services regularly compared with maintaining the status quo?'' This
scenario does not require shifting regular attendees to non-attendance;
here, results inform practical policies targeting non-regular attendees
who might start.

The third contrast examines: ``What would be the average difference
across the New Zealand population if no one attended religious services
compared with the status quo?'' This scenario does not require shifting
those who never attend; here, results inform practical policies
targeting regular attendees who might stop.

Although the set of causal contrasts analysts might consider is
unbounded, those we have selected here specifically address scientific
and policy interests.

Consider that our approach does not focus on testing predefined
hypotheses; instead, we aim to compute causal effects with high accuracy
by combining appropriate time-series data and robust methods for causal
inference (\citeproc{ref-hernan2024stating}{Hernán and Greenland 2024}).

\subsection{Method}\label{method}

\subsubsection{Sample}\label{sample}

Data were collected by the New Zealand Attitudes and Values Study
(NZAVS), an annual longitudinal national probability panel study of
social attitudes, personality, ideology, and health outcomes in New
Zealand. Chris G. Sibley started the New Zealand Attitudes and Values
Study in 2009, which has grown to include a community of over fifty
researchers. Since its inception, The New Zealand Attitudes and Values
Study has accumulated questionnaire responses from 72,910 New Zealand
residents. The study operates independently of political or corporate
funding and is based in a university setting. Data summaries for our
study sample are found in \textbf{Appendices B-D}. For more details
about the New Zealand Attitudes and Values Study see:
\href{https://doi.org/10.17605/OSF.IO/75SNB}{OSF.IO/75SNB}.

\subsubsection{Treatment Indicator}\label{treatment-indicator}

Religious service attendance is assessed in the New Zealand Attitudes
and Values Study with the following questions:

\begin{itemize}
\tightlist
\item
  \emph{Do you identify with a religion and/or spiritual group? If
  yes\ldots How many times did you attend a church or place of worship
  during the last month?}
\end{itemize}

We rounded responses to the nearest whole number. Because there were few
responses greater than eight, we coded all above eight as eight (see
\emph{Appendix B}: we label this variable
\texttt{religion\_church\_round}). Note that contrasts with responses
greater than four were not intervened upon in the regular church service
condition; our decision was to ease the computational burden during
estimation.

\subsubsection{Measures of Prosociality}\label{measures-of-prosociality}

\textbf{Study 1: Self-reported charity} The New Zealand Attitudes and
Values Study includes two self-reported measures of pro-sociality:

\begin{itemize}
\item
  Volunteering: \emph{``Please estimate how many hours you spent doing
  each of the following things last week\ldots Volunteer/charitable
  work''}
\item
  Annual charitable financial donations: \emph{``How much money have you
  donated to charity in the last year?''}
\end{itemize}

\textbf{Study 2: Help received from others in the last week: \emph{time}
}

Participants were asked:

\emph{``Please estimate how much help you have received from the
following sources in the last week.''}

\begin{itemize}
\tightlist
\item
  \emph{Family\ldots TIME (hours)}
\item
  \emph{Friends\ldots TIME (hours)}
\item
  \emph{Community\ldots TIME (hours)}
\end{itemize}

Owing to the high variability of responses, we transformed responses
into binary indicators: \emph{0 = none/ 1 = any}.

\textbf{Study 3: Help received from others in the last week:
\emph{money} }

Similarly, participants were asked:

\emph{``Please estimate how much help you have received from the
following sources in the last week.''}

\begin{itemize}
\tightlist
\item
  \emph{Family\ldots MONEY (dollars)}
\item
  \emph{Friends\ldots MONEY (dollars)}
\item
  \emph{Community\ldots MONEY (dollars)}
\end{itemize}

These measures were also highly variable. Hence, we converted responses
to binary indicators: \emph{0 = none/ 1 = any}.

Studies 2 and 3 aim to minimise self-presentation bias by using external
measures of prosocial outcomes. We assume that if religious institutions
foster prosociality, the initiation of regular attendance --controlling
for past religious service, past measures of the prosocial outcomes, and
rich array of demographic, personality, and health measures recorded at
baseline -- will increase exposure to prosocial behaviours. These
self-reported measures of dependency on others are robust to
self-presentation biases that might associate religious service
attendance with indicators of prosociality in the absence of causation.

We provide comprehensive details of all measures in \textbf{Appendix A}.

\subsubsection{Causal Interventions}\label{causal-interventions}

We define three targeted causal contrasts (\emph{causal estimands}) as
interventions on prespecified modified treatment policies
(\citeproc{ref-diaz2023lmtp}{Díaz \emph{et al.} 2023};
\citeproc{ref-diaz2021nonparametric}{Dı́az \emph{et al.} 2021};
\citeproc{ref-haneuse2013estimation}{Haneuse and Rotnitzky 2013}). Let
\(A_t\) denote the treatment, the monthly frequency of religious
service. There are three time points, \(t\in{0,1,2}\), where \(t=0\)
denotes the baseline wave, \(t=1\), the treatment wave, and \(t=2\) the
end of study. \(\mathbf{d}(\cdot)\) denotes a modified treatment policy
\(f_\mathbf{d}\). When a treatment is fixed to a level defined by the
treatment, perhaps contrary to the observed treatment, we use the lower
case symbol \(a_1\). In this study the functions defined by
\(f_\mathbf{d}\) are inteventions that fix \(A_1\) to \(a_1\).

We define three targeted causal contrasts (\emph{causal estimands})
using modified treatment policies as outlined in
(\citeproc{ref-diaz2023lmtp}{Díaz \emph{et al.} 2023};
\citeproc{ref-diaz2021nonparametric}{Dı́az \emph{et al.} 2021};
\citeproc{ref-haneuse2013estimation}{Haneuse and Rotnitzky 2013}). Let
\(A_t\) represent the treatment variable, the monthly frequency of
religious service. This study has three distinct measurement intervals,
\(t \in \{0, 1, 2\}\), where \(t=0\) is the baseline wave, \(t=1\) is
the treatment wave, and \(t=2\) marks the end of the study. The notation
\(\mathbf{d}(\cdot)\) denotes to a modified treatment policy
\(f_\mathbf{d}\). When the policy prescribes a treatment level, perhaps
contrary to the observed treatment level, we denote this fixed level by
the lowercase symbol \(a_1\). In this study, the functions defined by
\(f_\mathbf{d}\) are interventions that set \(A_1\) to \(a_1\).

\begin{enumerate}
\def\labelenumi{\arabic{enumi}.}
\tightlist
\item
  \textbf{Regular Religious Service Treatment}: Administer regular
  religious service attendance to everyone in the adult population. If
  an individual's religious service attendance is below four times per
  month, shift to four; otherwise, maintain their current attendance:
\end{enumerate}

\[
\mathbf{d}^\lambda (a_1) = \begin{cases} 4 & \text{if } a_1 < 4 \\ 
a_1 & \text{otherwise} \end{cases}
\]

\begin{enumerate}
\def\labelenumi{\arabic{enumi}.}
\setcounter{enumi}{1}
\tightlist
\item
  \textbf{Zero Religious Service Treatment}: Ensure no religious service
  attendance for everyone in the adult population of New Zealand. If an
  individual's religious service attendance is greater than zero, shift
  to zero; otherwise, make no change:
\end{enumerate}

\[
\mathbf{d}^\phi (a_1) = \begin{cases} 0 & \text{if } a_1 > 0 \\ 
a_1 & \text{otherwise} \end{cases}
\]

\begin{enumerate}
\def\labelenumi{\arabic{enumi}.}
\setcounter{enumi}{2}
\tightlist
\item
  \textbf{Status Quo -- No Treatment}: Apply no treatment. Each expected
  mean outcome is calculated using the natural (observed) value of
  religious service attendance for each individual.
\end{enumerate}

\[
\mathbf{d}(a_1) = a_1
\]

\subsubsection{Causal Contrasts}\label{causal-contrasts}

From these policies, we computed the following causal contrasts.

\textbf{Target Contrast A: `Regular vs.~Zero'}: How do the prosocial
effects of a society with regular religious service attendance differ
from those of a society with zero religious service attendance?

\[ \text{Regular Religious Service vs. Zero Religious Service} = E[Y(\mathbf{d}^\lambda) - Y(\mathbf{d}^\phi)] \]

This contrast simulates a scientifically interesting hypothetical
experiment where we could randomise individuals to either regular
religious service or none, assessing the differences in prosociality
outcomes measured one year after the intervention.

\textbf{Target Contrast B: `Regular vs.~Status Quo'}: How does a society
with regular religious service attendance compare to its status quo?

\[ \text{Regular Religious Service vs. No Treatment} = E[Y(\mathbf{d}^\lambda) - Y(\mathbf{d})] \]

This contrast reflects a policy-relevant hypothetical experiment
examining the effect of transitioning people to regular religious
service, assessing whether this intervention reliably alters societal
prosociality compared to the current state.

\textbf{Target Contrast C: `Zero vs.~Status Quo'}: What are the social
consequences for society of zero religious service attendance compared
to its status quo?

\[ \text{Zero Religious Service vs. No Treatment} = E[Y(\mathbf{d}^\phi) - Y(\mathbf{d})] \]

This scenario investigates the policy implications of eliminating
religious services entirely, questioning whether such a shift would
meaningfully affect the prosocial outcomes we measure.

\subsubsection{Identification
Assumptions}\label{identification-assumptions}

To consistently estimate causal effects, we must satisfy three
assumptions:

\begin{enumerate}
\def\labelenumi{\arabic{enumi}.}
\item
  \textbf{Causal consistency:} potential outcomes must correspond with
  observed outcomes under the treatments in the data. Essentially, we
  assume potential outcomes do not depend on how the treatment was
  administered, conditional on measured covariates
  (\citeproc{ref-vanderweele2009}{VanderWeele 2009};
  \citeproc{ref-vanderweele2013}{VanderWeele and Hernan 2013}).
\item
  \textbf{Exchangeability} given observed covariates, we assume
  treatment assignment is independent of the potential outcomes to be
  contrasted. In other words, there is ``no unmeasured confounding''
  (\citeproc{ref-chatton2020}{Chatton \emph{et al.} 2020};
  \citeproc{ref-hernan2024WHATIF}{Hernan and Robins 2024}).
\item
  \textbf{Positivity:} for unbiased estimation, every subject must have
  a non-zero chance of receiving the treatment, regardless of their
  covariate values Westreich and Cole
  (\citeproc{ref-westreich2010}{2010}). We evaluate this assumption in
  each study by examining changes in religious service attendance from
  baseline (NZAVS time 10) to the treatment wave (NZAVS time 11). For
  further discussion of these assumptions in the context of NZAVS
  studies, see Bulbulia \emph{et al.}
  (\citeproc{ref-bulbulia2023a}{2023}).
\end{enumerate}

\subsubsection{Target Population}\label{target-population}

The target population for this study comprises New Zealand residents as
represented in the baseline wave of the New Zealand Attitudes and Values
Study (NZAVS) during the years 2018-2019. The NZAVS is a national
probability study designed to reflect the broader New Zealand population
accurately. Despite its comprehensive scope, the NZAVS does have some
limitations in its demographic representation. Notably, it tends to
under-sample males and individuals of Asian descent while over-sampling
females and Māori (the indigenous peoples of New Zealand). To address
these disparities and enhance the accuracy of our findings, we apply New
Zealand Census survey weights to the sample data. These weights adjust
for variations in age, gender, and ethnicity to better approximate the
national demographic composition. Detailed methodologies related to the
application of these weights are discussed in Sibley
(\citeproc{ref-sibley2021}{2021}).

\subsubsection{Eligibility Criteria}\label{eligibility-criteria}

To be included in the analysis of this study, participants needed to
meet the following eligibility criteria:

\begin{itemize}
\tightlist
\item
  Enrolled in the 2018 wave of the New Zealand Attitudes and Values
  Study (NZAVS time 10).
\item
  Complete data on religious service attendance at baseline (NZAVS time
  10), with no missing information in this area.
\item
  Missing covariate data at baseline were permitted and subjected to
  imputation methods to ensure comprehensive analysis.
\item
  Participants who were lost to follow-up or did not respond at the
  study's conclusion were still included in the initial sample set.
\end{itemize}

In total, 33,198 individuals fulfilled these criteria and were included
in the study.

\subsubsection{Causal Identification}\label{causal-identification}

\begin{table}

\caption{\label{tbl-02}This table presents three Single World
Intervention Graphs (SWIGs), one for each of the treatment conditions we
compare. Note that we obtain strong confounding control by including
baseline measures for both the treatments and outcomes, see VanderWeele
\emph{et al.} (\citeproc{ref-vanderweele2020}{2020}), Bulbulia
(\citeproc{ref-bulbulia2024PRACTICAL}{2024a}).}

\centering{

\lmtptablethree

}

\end{table}%

Table~\ref{tbl-02} presents three Single World Intervention Graphs
(SWIGs) that outline our identification strategy
(\citeproc{ref-richardson2023nested}{Richardson \emph{et al.} 2023};
\citeproc{ref-richardson2013swigsprimer}{Richardson and Robins 2013},
\citeproc{ref-richardson2023potential}{2023};
\citeproc{ref-robins2010alternative}{Robins and Richardson 2010};
\citeproc{ref-shpitser2022multivariate}{Shpitser \emph{et al.} 2022};
\citeproc{ref-shpitser2016causal}{Shpitser and Tchetgen 2016}). Our
approach consistently applies the same identification strategy across
all functions estimated in this study. Note that the natural value of
the treatment \(A\) is obtained from its observed instances and baseline
historical data. This method ensures that our analysis accurately
captures the causal effects of flexible treatment regimes
(\citeproc{ref-diaz2021nonparametric}{Dı́az \emph{et al.} 2021};
\citeproc{ref-diaz2012population}{Muñoz and Van Der Laan 2012};
\citeproc{ref-young2014identification}{Young \emph{et al.} 2014}).

\subsubsection{Confounding Control}\label{confounding-control}

To manage confounding in our analysis, we implemented VanderWeele
(\citeproc{ref-vanderweele2019}{2019})'s \emph{modified disjunctive
cause criterion} by following these steps:

\begin{enumerate}
\def\labelenumi{\arabic{enumi}.}
\tightlist
\item
  \textbf{Identified all common causes} of both the treatment and
  outcomes to ensure a comprehensive approach to confounding control.
\item
  \textbf{Excluded instrumental variables} that affect the exposure but
  not the outcome. Instrumental variables do not contribute to
  controlling confounding and can reduce the efficiency of the
  estimates.
\item
  \textbf{Included proxies for unmeasured confounders} affecting both
  exposure and outcome. According to the principles of d-separation,
  using proxies allows us to indirectly control for their associated
  unmeasured confounders.
\item
  \textbf{Controlled for baseline exposure} and \textbf{baseline
  outcome}. Both are used as proxies for unmeasured common causes,
  enhancing the robustness of our causal estimates.
\end{enumerate}

\hyperref[appendix-demographics]{Appendix B} details the covariates we
included for confounding control. These methods adhere to the guidelines
provided in (\citeproc{ref-bulbulia2024PRACTICAL}{Bulbulia 2024a}) and
were pre-specified in our study protocol \url{https://osf.io/ce4t9/}.

\subsubsection{Missing Data}\label{missing-data}

To mitigate the influence of missing data on our study results and
enhance the robustness of our analyses, we implemented the following
strategies:

\textbf{Baseline missingness}: we employed the \texttt{ppm} algorithm
from the \texttt{mice} package in R (\citeproc{ref-vanbuuren2018}{Van
Buuren 2018}) to impute missing baseline data. This method allowed us to
reconstruct incomplete datasets by estimating a plausible value for
missing observation. Because we could only pass one data set to the
\texttt{lmtp}, we employed a single imputation. About 2\% of covariate
values were missing at baseline. Eligibility for the study required
fully observed baseline treatment measures as well as treatment wave
treatment measures.

\textbf{Outcome missingness}: to address confounding and selection bias
arising from missing responses and panel attrition, we applied
non-parametrically estimated censoring weights using the \texttt{lmtp}
package in R (\citeproc{ref-williams2021}{Williams and Díaz 2021}).

\subsubsection{Statistical Estimator}\label{statistical-estimator}

We perform statistical estimation using non-parametric Targeted
Learning, specifically a Targeted Minimum Loss-based Estimation (TMLE)
estimator. TMLE is a robust method that combines machine learning
techniques with traditional statistical models to estimate causal
effects while providing valid statistical uncertainty measures for these
estimates (\citeproc{ref-van2012targeted}{Laan and Gruber 2012};
\citeproc{ref-van2014targeted}{Van der Laan 2014}).

TMLE operates through a two-step process that involves modelling both
the outcome and treatment (exposure). Initially, TMLE employs machine
learning algorithms to flexibly model the relationship between
treatments, covariates, and outcomes. This flexibility allows TMLE to
account for complex, high-dimensional covariate spaces
\emph{efficiently} without imposing restrictive model assumptions
(\citeproc{ref-van2014discussion}{Laan \emph{et al.} 2014};
\citeproc{ref-vanderlaan2011}{Van Der Laan and Rose 2011},
\citeproc{ref-vanderlaan2018}{2018}). The outcome of this step is a set
of initial estimates for these relationships.

The second step of TMLE involves ``targeting'' these initial estimates
by incorporating information about the observed data distribution to
improve the accuracy of the causal effect estimate. TMLE achieves this
precision through an iterative updating process, which adjusts the
initial estimates towards the true causal effect. This updating process
is guided by the efficient influence function, ensuring that the final
TMLE estimate is as close as possible, given the measures and data, to
the true causal effect while still being robust to
model-misspecification in either the outcome or the treatment model.

Again, a central feature of TMLE is its double-robustness property. If
either the treatment model or the outcome model is correctly specified,
the TMLE estimator will consistently estimate the causal effect.
Additionally, TMLE uses cross-validation to avoid over-fitting and
ensure that the estimator performs well in finite samples. Each step
contributes to a robust methodology for examining an intervention's
\emph{causal} effects on outcomes. The marriage of TMLE and machine
learning technologies reduces the dependence on restrictive modelling
assumptions and introduces an additional layer of robustness. For
further details of the specific targeted learning strategy we favour,
see (\citeproc{ref-duxedaz2021}{Díaz \emph{et al.} 2021};
\citeproc{ref-hoffman2022}{Hoffman \emph{et al.} 2022},
\citeproc{ref-hoffman2023}{2023}). We performed estimation using the
\texttt{lmtp} package (\citeproc{ref-williams2021}{Williams and Díaz
2021}). We used the \texttt{superlearner} library for non-parametric
estimation with the predefined libraries \texttt{SL.ranger},
\texttt{SL.glmnet}, and \texttt{SL.xgboost}
(\citeproc{ref-xgboost2023}{Chen \emph{et al.} 2023};
\citeproc{ref-polley2023}{Polley \emph{et al.} 2023};
\citeproc{ref-Ranger2017}{Wright and Ziegler 2017}). We created graphs,
tables and output reports using the \texttt{margot} package
(\citeproc{ref-margot2024}{Bulbulia 2024b}). All analysis methods follow
pre-stated protocols in Bulbulia
(\citeproc{ref-bulbulia2024PRACTICAL}{2024a}).

\subsubsection{Sensitivity Analysis Using the
E-value}\label{sensitivity-analysis-using-the-e-value}

To assess the sensitivity of results to unmeasured confounding, we
report VanderWeele and Ding's ``E-value'' in all analyses
(\citeproc{ref-vanderweele2017}{VanderWeele and Ding 2017}). The E-value
quantifies the minimum strength of association (on the risk ratio scale)
that an unmeasured confounder would need to have with both the exposure
and the outcome (after considering the measured covariates) to explain
away the observed exposure-outcome association
(\citeproc{ref-linden2020EVALUE}{Linden \emph{et al.} 2020};
\citeproc{ref-vanderweele2020}{VanderWeele \emph{et al.} 2020}). To
evaluate the strength of evidence, we use the bound of the E-value 95\%
confidence interval closest to 1.

\subsubsection{Scope of Interventions}\label{scope-of-interventions}

To illustrate the magnitude of the shift interventions we contrast, we
provide histograms in Figure~\ref{fig-hist}, which display the
distribution of treatment levels during the treatment wave.
Figure~\ref{fig-hist} \emph{A}: The intervention for regular religious
service, represented in these histograms, affects a larger portion of
the sample than the zero religious service intervention.
Figure~\ref{fig-hist} \emph{B}: presents the intervention for zero
religious service. It involves a smaller portion of the sample compared
to the regular religious service intervention. The comparative analysis
of the `Regular' versus `Zero' interventions addresses the
scientifically intriguing question: what is the effect difference in a
scenario where religious service is universal versus completely absent?
The intervention that increases attendance to regular service levels
from the status quo allows us to consider the potential costs and
benefits of widespread religious practice across society. The
intervention that eliminates religious service allows us to consider the
potential costs and benefits of the widespread loss of religious
practice across society.

\begin{figure}

\centering{

\includegraphics{24-bulbulia-lmtp-coop-church_files/figure-pdf/fig-hist-1.pdf}

}

\caption{\label{fig-hist}This figure shows a histogram of responses to
religious service frequency in the baseline + 1 wave. Responses above
eight were assigned to eight, and values were rounded to the nearest
whole number. The red dashed line shows the population average. (A)
Responses in the gold bars are shifted to four on the Regular Religious
Service intervention. All those responses in grey (four and above)
remain unchanged. (B) On the zero intervention, responses in the blue
bars are shifted.}

\end{figure}%

\newpage{}

\subsubsection{Evidence for Change in the Treatment
Variable}\label{evidence-for-change-in-the-treatment-variable}

Table~\ref{tbl-transition} clarifies the change in the treatment
variable from the baseline wave to the baseline + 1 wave across the
sample. Assessing change in a variable is essential for evaluating the
positivity assumption and recovering evidence for the incident exposure
effect of the treatment variable (\citeproc{ref-danaei2012}{Danaei
\emph{et al.} 2012}; \citeproc{ref-hernan2024WHATIF}{Hernan and Robins
2024}; \citeproc{ref-vanderweele2020}{VanderWeele \emph{et al.} 2020}).
We find that state 4 (weekly attendance) and state 0 present the highest
overall. However, we also find that movement between these states
reveals they are not deterministic. States 1, 2, 3, and 5 present
relatively frequent jumps in and out of these states, suggesting lower
stability or measurement error.

\begin{longtable}[]{@{}
  >{\centering\arraybackslash}p{(\columnwidth - 18\tabcolsep) * \real{0.0978}}
  >{\centering\arraybackslash}p{(\columnwidth - 18\tabcolsep) * \real{0.1196}}
  >{\centering\arraybackslash}p{(\columnwidth - 18\tabcolsep) * \real{0.0978}}
  >{\centering\arraybackslash}p{(\columnwidth - 18\tabcolsep) * \real{0.0978}}
  >{\centering\arraybackslash}p{(\columnwidth - 18\tabcolsep) * \real{0.0978}}
  >{\centering\arraybackslash}p{(\columnwidth - 18\tabcolsep) * \real{0.0978}}
  >{\centering\arraybackslash}p{(\columnwidth - 18\tabcolsep) * \real{0.0978}}
  >{\centering\arraybackslash}p{(\columnwidth - 18\tabcolsep) * \real{0.0978}}
  >{\centering\arraybackslash}p{(\columnwidth - 18\tabcolsep) * \real{0.0978}}
  >{\centering\arraybackslash}p{(\columnwidth - 18\tabcolsep) * \real{0.0978}}@{}}

\caption{\label{tbl-transition}This transition matrix captures stability
and change in religious service between the baseline and treatment wave.
Each cell in the matrix represents the count of individuals
transitioning from one state to another. The rows correspond to the
state at baseline (From), and the columns correspond to the state at the
treatment wave (To). \textbf{Diagonal entries} (in \textbf{bold})
signify the number of individuals who remained in their initial state
across both waves. \textbf{Off-diagonal entries} signify the transitions
of individuals from their baseline state to a different state in the
treatment wave. A higher number on the diagonal relative to the
off-diagonal entries in the same row indicates greater stability in a
state. Conversely, higher off-diagonal numbers suggest more frequent
shifts in the sample from the baseline state to other states.}

\tabularnewline

\toprule\noalign{}
\begin{minipage}[b]{\linewidth}\centering
From
\end{minipage} & \begin{minipage}[b]{\linewidth}\centering
State 0
\end{minipage} & \begin{minipage}[b]{\linewidth}\centering
State 1
\end{minipage} & \begin{minipage}[b]{\linewidth}\centering
State 2
\end{minipage} & \begin{minipage}[b]{\linewidth}\centering
State 3
\end{minipage} & \begin{minipage}[b]{\linewidth}\centering
State 4
\end{minipage} & \begin{minipage}[b]{\linewidth}\centering
State 5
\end{minipage} & \begin{minipage}[b]{\linewidth}\centering
State 6
\end{minipage} & \begin{minipage}[b]{\linewidth}\centering
State 7
\end{minipage} & \begin{minipage}[b]{\linewidth}\centering
State 8
\end{minipage} \\
\midrule\noalign{}
\endhead
\bottomrule\noalign{}
\endlastfoot
State 0 & \textbf{26762} & 405 & 174 & 71 & 126 & 26 & 13 & 8 & 68 \\
State 1 & 647 & \textbf{235} & 85 & 44 & 46 & 5 & 2 & 3 & 10 \\
State 2 & 236 & 105 & \textbf{188} & 104 & 96 & 12 & 13 & 2 & 21 \\
State 3 & 112 & 54 & 110 & \textbf{164} & 173 & 18 & 8 & 4 & 15 \\
State 4 & 150 & 71 & 127 & 205 & \textbf{881} & 124 & 64 & 16 & 91 \\
State 5 & 24 & 7 & 17 & 17 & 145 & \textbf{61} & 25 & 7 & 33 \\
State 6 & 14 & 5 & 13 & 17 & 84 & 22 & \textbf{29} & 5 & 37 \\
State 7 & 9 & 0 & 6 & 3 & 16 & 6 & 9 & \textbf{6} & 19 \\
State 8 & 74 & 14 & 17 & 14 & 105 & 34 & 42 & 17 & \textbf{351} \\

\end{longtable}

\newpage{}

\subsection{Results}\label{results}

\subsubsection{Study 1: Causal Effects of Regular Church Attendance on
Self-Reported Volunteering and Self-Reported Volunteering and
Donations}\label{study-1-causal-effects-of-regular-church-attendance-on-self-reported-volunteering-and-self-reported-volunteering-and-donations}

\paragraph{Regular Religious Service vs.~Zero Treatment Contrast for
Donations and
Volunteering}\label{regular-religious-service-vs.-zero-treatment-contrast-for-donations-and-volunteering}

Results for the treatment contrasts between Regular Religious Service
and Zero Religious Service, focusing on self-reported volunteering and
charitable donations, are displayed in Figure~\ref{fig-1_1} \emph{A} and
Table~\ref{tbl-1_1}. These results are measured on the difference scale.

\begin{longtable}[]{@{}lrrrrr@{}}

\caption{\label{tbl-1_1}This table reports the results of model
estimates for the causal effects of a universal gain of weekly religious
service vs.~a universal loss of weekly religious service on reported
charitable behaviours at the end of the study. Contrasts are expressed
in standard deviation units.}

\tabularnewline

\toprule\noalign{}
& E{[}Y(1){]}-E{[}Y(0){]} & 2.5 \% & 97.5 \% & E\_Value &
E\_Val\_bound \\
\midrule\noalign{}
\endhead
\bottomrule\noalign{}
\endlastfoot
donations & 0.132 & 0.102 & 0.161 & 1.507 & 1.426 \\
hours volunteer & 0.123 & 0.090 & 0.156 & 1.482 & 1.389 \\

\end{longtable}

For `donations', the effect estimate is 0.132 {[}0.102, 0.161{]}. The
E-value for this estimate is 1.507, with a lower bound of 1.426. At this
lower bound, unmeasured confounders would need a minimum association
strength with both the intervention sequence and outcome of 1.426 to
negate the observed effect. Weaker associations would not overturn it.
We infer \textbf{evidence for causality}. \textbf{On the data scale,
this intervention represents a difference of NZD 656.58 per adult per
year in charitable giving.}

The effect estimate for `hours volunteer' is 0.123 {[}0.09, 0.156{]}.
The E-value for this estimate is 1.482, with a lower bound of 1.389. We
infer \textbf{evidence for causality}. \textbf{On the data scale, this
intervention represents a difference of NZD 30.21 minutes per adult per
week in charitable giving.}

\paragraph{Regular Religious Service vs.~Status Quo Treatment Contrast
for Donations and
Volunteering}\label{regular-religious-service-vs.-status-quo-treatment-contrast-for-donations-and-volunteering}

Figure~\ref{fig-1_1} \emph{B} and Table~\ref{tbl-1_2} present results
for the treatment contrasts between Regular Religious Service and Status
Quo, focusing on self-reported volunteering and charitable donations.
These results are measured on the difference scale.

\begin{longtable}[]{@{}lrrrrr@{}}

\caption{\label{tbl-1_2}This table reports results of model estimates
for the causal effects of a universal gain of weekly religious service
vs.~the status quo on reported charitable behaviours at the end of the
study. Contrasts are expressed in standard deviation units.}

\tabularnewline

\toprule\noalign{}
& E{[}Y(1){]}-E{[}Y(0){]} & 2.5 \% & 97.5 \% & E\_Value &
E\_Val\_bound \\
\midrule\noalign{}
\endhead
\bottomrule\noalign{}
\endlastfoot
donations & 0.121 & 0.102 & 0.140 & 1.477 & 1.422 \\
hours volunteer & 0.095 & 0.066 & 0.123 & 1.404 & 1.317 \\

\end{longtable}

For `donations', the effect estimate is 0.121 {[}0.102, 0.14{]}. The
E-value for this estimate is 1.477, with a lower bound of 1.422. At this
lower bound, unmeasured confounders would need a minimum association
strength with both the intervention sequence and outcome of 1.422 to
negate the observed effect. Weaker confounding would not overturn it. We
infer \textbf{evidence for causality}. \textbf{On the data scale, this
intervention represents a difference of NZD 601.87 per adult per year in
charitable giving.}

For `hours volunteer', the effect estimate is 0.095 {[}0.066, 0.123{]}.
The E-value for this estimate is 1.404, with a lower bound of 1.317. At
this lower bound, unmeasured confounders would need a minimum
association strength with both the intervention sequence and outcome of
1.317 to negate the observed effect. Weaker confounding would not
overturn it. We infer \textbf{evidence for causality}. \textbf{On the
data scale, this intervention represents a difference of 23.33 minutes
per adult per week in charitable giving.}

\paragraph{Zero Religious Service vs.~The Status Quo Treatment Contrast
for Donations and
Volunteering}\label{zero-religious-service-vs.-the-status-quo-treatment-contrast-for-donations-and-volunteering}

Figure~\ref{fig-1_1} \emph{C} and Table~\ref{tbl-1_3} present results
for the treatment contrasts between Zero Religious Service and Status
Quo, focusing on self-reported volunteering and charitable donations.
These results are measured on the difference scale.

\begin{longtable}[]{@{}
  >{\raggedright\arraybackslash}p{(\columnwidth - 10\tabcolsep) * \real{0.2424}}
  >{\raggedleft\arraybackslash}p{(\columnwidth - 10\tabcolsep) * \real{0.2424}}
  >{\raggedleft\arraybackslash}p{(\columnwidth - 10\tabcolsep) * \real{0.1061}}
  >{\raggedleft\arraybackslash}p{(\columnwidth - 10\tabcolsep) * \real{0.1061}}
  >{\raggedleft\arraybackslash}p{(\columnwidth - 10\tabcolsep) * \real{0.1212}}
  >{\raggedleft\arraybackslash}p{(\columnwidth - 10\tabcolsep) * \real{0.1818}}@{}}

\caption{\label{tbl-1_3}This table reports the results of model
estimates for the causal effects of a universal loss of weekly religious
service vs.~the status quo on reported charitable behaviours at the end
of the study. Contrasts are expressed in standard deviation units.}

\tabularnewline

\toprule\noalign{}
\begin{minipage}[b]{\linewidth}\raggedright
\end{minipage} & \begin{minipage}[b]{\linewidth}\raggedleft
E{[}Y(1){]}-E{[}Y(0){]}
\end{minipage} & \begin{minipage}[b]{\linewidth}\raggedleft
2.5 \%
\end{minipage} & \begin{minipage}[b]{\linewidth}\raggedleft
97.5 \%
\end{minipage} & \begin{minipage}[b]{\linewidth}\raggedleft
E\_Value
\end{minipage} & \begin{minipage}[b]{\linewidth}\raggedleft
E\_Val\_bound
\end{minipage} \\
\midrule\noalign{}
\endhead
\bottomrule\noalign{}
\endlastfoot
donations & -0.011 & -0.029 & 0.008 & 1.111 & 1.000 \\
hours volunteer & -0.028 & -0.042 & -0.014 & 1.189 & 1.128 \\

\end{longtable}

For `donations', the effect estimate is -0.011 {[}-0.029, 0.008{]}. The
E-value for this estimate is 1.111, with a lower bound of 1. At this
lower bound, unmeasured confounders would need a minimum association
strength with both the intervention sequence and outcome of 1 to negate
the observed effect. Weaker confounding would not overturn it. We infer
that \textbf{the evidence for causality is not reliable}. \textbf{On the
data scale, this intervention represents a difference of NZD -54.72 per
adult per year in charitable giving, but again, the effect is not
reliable.}

The effect estimate for `hours volunteer' is -0.028 {[}-0.042,
-0.014{]}. The E-value for this estimate is 1.189, with a lower bound of
1.128. At this lower bound, unmeasured confounders would need a minimum
association strength with both the intervention sequence and outcome of
1.128 to negate the observed effect. Again, weaker confounding would not
overturn it. We infer \textbf{evidence for causality}. \textbf{On the
data scale, this intervention represents a difference of -6.88 in
volunteering minutes.}

\begin{figure}

\centering{

\includegraphics{24-bulbulia-lmtp-coop-church_files/figure-pdf/fig-1_1-1.pdf}

}

\caption{\label{fig-1_1}This figure graphs the results of model
estimates for the causal effects of the three causal contrasts of
interest on reported charitable behaviours at the study's end. The
causal contrasts are: (A) Regular vs.~Zero Religious Service (B) Regular
Religious Service vs.~Status Quo; (C) Zero Religious Service vs.~Status
Quo. Contrasts are expressed in standard deviation units.}

\end{figure}%

\newpage{}

\subsubsection{Study 2: Causal Effects of Regular Church Attendance on
Support Received From Others --
Time}\label{study-2-causal-effects-of-regular-church-attendance-on-support-received-from-others-time}

\paragraph{Regular vs.~Zero Causal Treatment Contrast for Time Received
From
Others}\label{regular-vs.-zero-causal-treatment-contrast-for-time-received-from-others}

Figure~\ref{fig-study2} \emph{A} and Table~\ref{tbl-study2A} present
results for the treatment contrasts between Regular Religious Service
and Zero, focusing on voluntary help received from others during the
past week (yes/no). These results are measured on the risk ratio scale.

\begin{longtable}[]{@{}
  >{\raggedright\arraybackslash}p{(\columnwidth - 10\tabcolsep) * \real{0.3000}}
  >{\raggedleft\arraybackslash}p{(\columnwidth - 10\tabcolsep) * \real{0.2286}}
  >{\raggedleft\arraybackslash}p{(\columnwidth - 10\tabcolsep) * \real{0.0857}}
  >{\raggedleft\arraybackslash}p{(\columnwidth - 10\tabcolsep) * \real{0.1000}}
  >{\raggedleft\arraybackslash}p{(\columnwidth - 10\tabcolsep) * \real{0.1143}}
  >{\raggedleft\arraybackslash}p{(\columnwidth - 10\tabcolsep) * \real{0.1714}}@{}}

\caption{\label{tbl-study2A}This table reports the results of model
estimates for the causal effects of a universal gain of weekly religious
service vs.~a universal loss of weekly religious service on voluntary
help received from others during the past week (yes/no) at the end of
the study. Contrasts are expressed on the risk ratio scale.}

\tabularnewline

\toprule\noalign{}
\begin{minipage}[b]{\linewidth}\raggedright
\end{minipage} & \begin{minipage}[b]{\linewidth}\raggedleft
E{[}Y(1){]}/E{[}Y(0){]}
\end{minipage} & \begin{minipage}[b]{\linewidth}\raggedleft
2.5 \%
\end{minipage} & \begin{minipage}[b]{\linewidth}\raggedleft
97.5 \%
\end{minipage} & \begin{minipage}[b]{\linewidth}\raggedleft
E\_Value
\end{minipage} & \begin{minipage}[b]{\linewidth}\raggedleft
E\_Val\_bound
\end{minipage} \\
\midrule\noalign{}
\endhead
\bottomrule\noalign{}
\endlastfoot
family gives time & 0.950 & 0.901 & 1.003 & 1.288 & 1.000 \\
friends give time & 1.187 & 1.108 & 1.271 & 1.658 & 1.454 \\
community gives time & 1.378 & 1.231 & 1.541 & 2.100 & 1.764 \\

\end{longtable}

For `community gives time', the effect estimate is 1.378 {[}1.231,
1.541{]}. The E-value for this estimate is 2.1, with a lower bound of
1.764. At this lower bound, unmeasured confounders would need a minimum
association strength with both the intervention sequence and outcome of
1.764 to negate the observed effect. Weaker confounding would not
overturn it. We infer \textbf{evidence for causality}.

For `friends give time', the effect estimate is 1.187 {[}1.108,
1.271{]}. The E-value for this estimate is 1.658, with a lower bound of
1.454. At this lower bound, unmeasured confounders would need a minimum
association strength with both the intervention sequence and outcome of
1.454 to negate the observed effect. Weaker confounding would not
overturn it. We infer \textbf{evidence for causality}.

For `family gives time', the effect estimate is 0.95 {[}0.901, 1.003{]}.
The E-value for this estimate is 1.288, with a lower bound of 1. At this
lower bound, unmeasured confounders would need a minimum association
strength with both the intervention sequence and outcome of 1 to negate
the observed effect. Weaker confounding would not overturn it. We infer
\textbf{that evidence for causality is not reliable}.

\paragraph{Regular Religious Service vs.~Status Quo Treatment Contrast
for Time Received From
Others}\label{regular-religious-service-vs.-status-quo-treatment-contrast-for-time-received-from-others}

Figure~\ref{fig-study2}\emph{B} and Table~\ref{tbl-2-2} present results
for the treatment contrasts between Regular Religious Service and Status
Quo, focusing on voluntary help received from others during the past
week (yes/no). These results are measured on the risk ratio scale.

\begin{longtable}[]{@{}
  >{\raggedright\arraybackslash}p{(\columnwidth - 10\tabcolsep) * \real{0.3000}}
  >{\raggedleft\arraybackslash}p{(\columnwidth - 10\tabcolsep) * \real{0.2286}}
  >{\raggedleft\arraybackslash}p{(\columnwidth - 10\tabcolsep) * \real{0.0857}}
  >{\raggedleft\arraybackslash}p{(\columnwidth - 10\tabcolsep) * \real{0.1000}}
  >{\raggedleft\arraybackslash}p{(\columnwidth - 10\tabcolsep) * \real{0.1143}}
  >{\raggedleft\arraybackslash}p{(\columnwidth - 10\tabcolsep) * \real{0.1714}}@{}}

\caption{\label{tbl-2-2}This table reports the results of model
estimates for the causal effects of a universal gain of weekly religious
service vs.~the status quo on voluntary help received from others during
the past week (yes/no) at the end of the study. Contrasts are expressed
on the risk ratio scale.}

\tabularnewline

\toprule\noalign{}
\begin{minipage}[b]{\linewidth}\raggedright
\end{minipage} & \begin{minipage}[b]{\linewidth}\raggedleft
E{[}Y(1){]}/E{[}Y(0){]}
\end{minipage} & \begin{minipage}[b]{\linewidth}\raggedleft
2.5 \%
\end{minipage} & \begin{minipage}[b]{\linewidth}\raggedleft
97.5 \%
\end{minipage} & \begin{minipage}[b]{\linewidth}\raggedleft
E\_Value
\end{minipage} & \begin{minipage}[b]{\linewidth}\raggedleft
E\_Val\_bound
\end{minipage} \\
\midrule\noalign{}
\endhead
\bottomrule\noalign{}
\endlastfoot
family gives time & 0.958 & 0.913 & 1.006 & 1.258 & 1.000 \\
friends give time & 1.128 & 1.061 & 1.199 & 1.508 & 1.315 \\
community gives time & 1.289 & 1.174 & 1.415 & 1.899 & 1.626 \\

\end{longtable}

For `community gives time', the effect estimate is 1.289 {[}1.174,
1.415{]}. The E-value for this estimate is 1.899, with a lower bound of
1.626. At this lower bound, unmeasured confounders would need a minimum
association strength with both the intervention sequence and outcome of
1.626 to negate the observed effect. Weaker confounding would not
overturn it. We infer \textbf{evidence for causality}.

For `friends give time', the effect estimate is 1.128 {[}1.061,
1.199{]}. The E-value for this estimate is 1.508, with a lower bound of
1.315. At this lower bound, unmeasured confounders would need a minimum
association strength with both the intervention sequence and outcome of
1.315 to negate the observed effect. Weaker confounding would not
overturn it. We infer \textbf{evidence for causality}.

For `family gives time', the effect estimate is 0.958 {[}0.913,
1.006{]}. The E-value for this estimate is 1.258, with a lower bound of
1. At this lower bound, unmeasured confounders would need a minimum
association strength with both the intervention sequence and outcome of
1 to negate the observed effect. Weaker confounding would not overturn
it. We infer \textbf{that evidence for causality is not reliable}.

\paragraph{Zero Religious Service vs.~Status Quo Treatment Contrast for
Time Received From
Others}\label{zero-religious-service-vs.-status-quo-treatment-contrast-for-time-received-from-others}

Figure~\ref{fig-study2} \emph{C} and Table~\ref{tbl-2_3} present results
for the treatment contrasts between Zero Religious Service and Status
Quo, focusing on voluntary help received from others during the past
week (yes/no). These results are measured on the risk ratio scale.

\begin{longtable}[]{@{}
  >{\raggedright\arraybackslash}p{(\columnwidth - 10\tabcolsep) * \real{0.3000}}
  >{\raggedleft\arraybackslash}p{(\columnwidth - 10\tabcolsep) * \real{0.2286}}
  >{\raggedleft\arraybackslash}p{(\columnwidth - 10\tabcolsep) * \real{0.0857}}
  >{\raggedleft\arraybackslash}p{(\columnwidth - 10\tabcolsep) * \real{0.1000}}
  >{\raggedleft\arraybackslash}p{(\columnwidth - 10\tabcolsep) * \real{0.1143}}
  >{\raggedleft\arraybackslash}p{(\columnwidth - 10\tabcolsep) * \real{0.1714}}@{}}

\caption{\label{tbl-2_3}This table reports results of model estimates
for the causal effects of a universal loss of weekly religious service
vs.~the status quo on voluntary help received from others during the
past week (yes/no) at the end of the study. Contrasts are expressed on
the risk ratio scale.}

\tabularnewline

\toprule\noalign{}
\begin{minipage}[b]{\linewidth}\raggedright
\end{minipage} & \begin{minipage}[b]{\linewidth}\raggedleft
E{[}Y(1){]}/E{[}Y(0){]}
\end{minipage} & \begin{minipage}[b]{\linewidth}\raggedleft
2.5 \%
\end{minipage} & \begin{minipage}[b]{\linewidth}\raggedleft
97.5 \%
\end{minipage} & \begin{minipage}[b]{\linewidth}\raggedleft
E\_Value
\end{minipage} & \begin{minipage}[b]{\linewidth}\raggedleft
E\_Val\_bound
\end{minipage} \\
\midrule\noalign{}
\endhead
\bottomrule\noalign{}
\endlastfoot
family gives time & 1.008 & 0.991 & 1.026 & 1.098 & 1.000 \\
friends give time & 0.950 & 0.928 & 0.973 & 1.288 & 1.197 \\
community gives time & 0.936 & 0.889 & 0.985 & 1.339 & 1.140 \\

\end{longtable}

For `family gives time', the effect estimate is 1.008 {[}0.991,
1.026{]}. The E-value for this estimate is 1.098, with a lower bound of
1. At this lower bound, unmeasured confounders would need a minimum
association strength with both the intervention sequence and outcome of
1 to negate the observed effect. Weaker confounding would not overturn
it. We infer \textbf{that evidence for causality is not reliable}.

For `friends give time', the effect estimate is 0.95 {[}0.928, 0.973{]}.
The E-value for this estimate is 1.288, with a lower bound of 1.197. At
this lower bound, unmeasured confounders would need a minimum
association strength with both the intervention sequence and outcome of
1.197 to negate the observed effect. Weaker confounding would not
overturn it. We infer \textbf{evidence for causality}.

For `community gives time', the effect estimate is 0.936 {[}0.889,
0.985{]}. The E-value for this estimate is 1.339, with a lower bound of
1.14. At this lower bound, unmeasured confounders would need a minimum
association strength with both the intervention sequence and outcome of
1.14 to negate the observed effect. Weaker confounding would not
overturn it. We infer \textbf{evidence for causality}.

\begin{figure}

\centering{

\includegraphics{24-bulbulia-lmtp-coop-church_files/figure-pdf/fig-study2-1.pdf}

}

\caption{\label{fig-study2}This figure reports the results of model
estimates for the three causal contrasts of interest on help received
from others during the past week (yes/no). The causal contrasts are: (A)
Regular vs.~Zero Religious Service (B) Regular Religious Service
vs.~Status Quo; (C) Zero Religious Service vs.~Status Quo. Contrasts are
expressed on the risk ratio scale.}

\end{figure}%

\newpage{}

\subsubsection{Study 3: Causal Effects of Regular Church Attendance on
Support Received From Others --
Money}\label{study-3-causal-effects-of-regular-church-attendance-on-support-received-from-others-money}

\paragraph{Regular vs.~Zero Causal Contrast on Money Received From
Others}\label{regular-vs.-zero-causal-contrast-on-money-received-from-others}

Figure~\ref{fig-study_3} \emph{A} and Table~\ref{tbl-3_1} present
results for the treatment contrasts between Regular Religious Service
and Zero, focusing on money received from others during the past week
(yes/no). These results are measured on the risk ratio scale.

\begin{longtable}[]{@{}
  >{\raggedright\arraybackslash}p{(\columnwidth - 10\tabcolsep) * \real{0.3099}}
  >{\raggedleft\arraybackslash}p{(\columnwidth - 10\tabcolsep) * \real{0.2254}}
  >{\raggedleft\arraybackslash}p{(\columnwidth - 10\tabcolsep) * \real{0.0845}}
  >{\raggedleft\arraybackslash}p{(\columnwidth - 10\tabcolsep) * \real{0.0986}}
  >{\raggedleft\arraybackslash}p{(\columnwidth - 10\tabcolsep) * \real{0.1127}}
  >{\raggedleft\arraybackslash}p{(\columnwidth - 10\tabcolsep) * \real{0.1690}}@{}}

\caption{\label{tbl-3_1}This table reports the results of model
estimates for the causal effects of a universal gain of weekly religious
service vs.~universal loss of weekly religious service on financial help
received from others during the past week (yes/no) at the end of the
study. Contrasts are expressed on the risk ratio scale.}

\tabularnewline

\toprule\noalign{}
\begin{minipage}[b]{\linewidth}\raggedright
\end{minipage} & \begin{minipage}[b]{\linewidth}\raggedleft
E{[}Y(1){]}/E{[}Y(0){]}
\end{minipage} & \begin{minipage}[b]{\linewidth}\raggedleft
2.5 \%
\end{minipage} & \begin{minipage}[b]{\linewidth}\raggedleft
97.5 \%
\end{minipage} & \begin{minipage}[b]{\linewidth}\raggedleft
E\_Value
\end{minipage} & \begin{minipage}[b]{\linewidth}\raggedleft
E\_Val\_bound
\end{minipage} \\
\midrule\noalign{}
\endhead
\bottomrule\noalign{}
\endlastfoot
family gives money & 1.137 & 1.028 & 1.258 & 1.532 & 1.198 \\
friends give money & 1.137 & 0.964 & 1.342 & 1.532 & 1.000 \\
community gives money & 1.376 & 1.112 & 1.703 & 2.095 & 1.465 \\

\end{longtable}

For `community gives money', the effect estimate is 1.376 {[}1.112,
1.703{]}. The E-value for this estimate is 2.095, with a lower bound of
1.465. At this lower bound, unmeasured confounders would need a minimum
association strength with both the intervention sequence and outcome of
1.465 to negate the observed effect. Weaker confounding would not
overturn it. We infer \textbf{evidence for causality}.

For `family gives money', the effect estimate is 1.137 {[}1.028,
1.258{]}. The E-value for this estimate is 1.532, with a lower bound of
1.198. At this lower bound, unmeasured confounders would need a minimum
association strength with both the intervention sequence and outcome of
1.198 to negate the observed effect. Weaker confounding would not
overturn it. We infer \textbf{evidence for causality}.

For `friends give money', the effect estimate is 1.137 {[}0.964,
1.342{]}. The E-value for this estimate is 1.532, with a lower bound of
1. At this lower bound, unmeasured confounders would need a minimum
association strength with both the intervention sequence and outcome of
1 to negate the observed effect. Weaker confounding would not overturn
it. We infer \textbf{that evidence for causality is not reliable}.

\paragraph{Regular vs.~Status Quo Causal Contrast on Money Received From
Others}\label{regular-vs.-status-quo-causal-contrast-on-money-received-from-others}

Figure~\ref{fig-study_3} \emph{B} and Table~\ref{tbl-3_2} present
results for the treatment contrasts between Regular Religious Service
and Status Quo, focusing on money received from others during the past
week (yes/no). These results are measured on the risk ratio scale.

\begin{longtable}[]{@{}
  >{\raggedright\arraybackslash}p{(\columnwidth - 10\tabcolsep) * \real{0.3099}}
  >{\raggedleft\arraybackslash}p{(\columnwidth - 10\tabcolsep) * \real{0.2254}}
  >{\raggedleft\arraybackslash}p{(\columnwidth - 10\tabcolsep) * \real{0.0845}}
  >{\raggedleft\arraybackslash}p{(\columnwidth - 10\tabcolsep) * \real{0.0986}}
  >{\raggedleft\arraybackslash}p{(\columnwidth - 10\tabcolsep) * \real{0.1127}}
  >{\raggedleft\arraybackslash}p{(\columnwidth - 10\tabcolsep) * \real{0.1690}}@{}}

\caption{\label{tbl-3_2}This table reports the results of model
estimates for the causal effects of a universal gain of weekly religious
service vs.~the status quo on financial help received from others during
the past week (yes/no) at the end of the study. Contrasts are expressed
on the risk ratio scale.}

\tabularnewline

\toprule\noalign{}
\begin{minipage}[b]{\linewidth}\raggedright
\end{minipage} & \begin{minipage}[b]{\linewidth}\raggedleft
E{[}Y(1){]}/E{[}Y(0){]}
\end{minipage} & \begin{minipage}[b]{\linewidth}\raggedleft
2.5 \%
\end{minipage} & \begin{minipage}[b]{\linewidth}\raggedleft
97.5 \%
\end{minipage} & \begin{minipage}[b]{\linewidth}\raggedleft
E\_Value
\end{minipage} & \begin{minipage}[b]{\linewidth}\raggedleft
E\_Val\_bound
\end{minipage} \\
\midrule\noalign{}
\endhead
\bottomrule\noalign{}
\endlastfoot
family gives money & 1.130 & 1.037 & 1.232 & 1.513 & 1.233 \\
friends give money & 1.041 & 0.951 & 1.139 & 1.248 & 1.000 \\
community gives money & 1.254 & 1.098 & 1.432 & 1.818 & 1.426 \\

\end{longtable}

For `community gives money', the effect estimate is 1.254 {[}1.098,
1.432{]}. The E-value for this estimate is 1.818, with a lower bound of
1.426. At this lower bound, unmeasured confounders would need a minimum
association strength with both the intervention sequence and outcome of
1.426 to negate the observed effect. Weaker confounding would not
overturn it. We infer \textbf{evidence for causality}.

For `family gives money', the effect estimate is 1.13 {[}1.037,
1.232{]}. The E-value for this estimate is 1.513, with a lower bound of
1.233. At this lower bound, unmeasured confounders would need a minimum
association strength with both the intervention sequence and outcome of
1.233 to negate the observed effect. Weaker confounding would not
overturn it. We infer \textbf{evidence for causality}.

For `friends give money', the effect estimate is 1.041 {[}0.951,
1.139{]}. The E-value for this estimate is 1.248, with a lower bound of
1. At this lower bound, unmeasured confounders would need a minimum
association strength with both the intervention sequence and outcome of
1 to negate the observed effect. Weaker confounding would not overturn
it. We infer \textbf{that evidence for causality is not reliable}.

\paragraph{Zero vs.~Status Quo Causal Contrast on Money Received From
Others}\label{zero-vs.-status-quo-causal-contrast-on-money-received-from-others}

Figure~\ref{fig-study_3} \emph{C} and Table~\ref{tbl-3_3} present
results for the treatment contrasts between Zero Religious Service and
Status Quo, focusing on money received from others during the past week
(yes/no). These results are measured on the risk ratio scale.

\begin{longtable}[]{@{}
  >{\raggedright\arraybackslash}p{(\columnwidth - 10\tabcolsep) * \real{0.3099}}
  >{\raggedleft\arraybackslash}p{(\columnwidth - 10\tabcolsep) * \real{0.2254}}
  >{\raggedleft\arraybackslash}p{(\columnwidth - 10\tabcolsep) * \real{0.0845}}
  >{\raggedleft\arraybackslash}p{(\columnwidth - 10\tabcolsep) * \real{0.0986}}
  >{\raggedleft\arraybackslash}p{(\columnwidth - 10\tabcolsep) * \real{0.1127}}
  >{\raggedleft\arraybackslash}p{(\columnwidth - 10\tabcolsep) * \real{0.1690}}@{}}

\caption{\label{tbl-3_3}Table reports results of model estimates for the
causal effects of a universal loss of weekly religious service vs.~the
status quo on financial help received from others during the past week
(yes/no) at the end of study. Contrasts are expressed on the risk ratio
scale.}

\tabularnewline

\toprule\noalign{}
\begin{minipage}[b]{\linewidth}\raggedright
\end{minipage} & \begin{minipage}[b]{\linewidth}\raggedleft
E{[}Y(1){]}/E{[}Y(0){]}
\end{minipage} & \begin{minipage}[b]{\linewidth}\raggedleft
2.5 \%
\end{minipage} & \begin{minipage}[b]{\linewidth}\raggedleft
97.5 \%
\end{minipage} & \begin{minipage}[b]{\linewidth}\raggedleft
E\_Value
\end{minipage} & \begin{minipage}[b]{\linewidth}\raggedleft
E\_Val\_bound
\end{minipage} \\
\midrule\noalign{}
\endhead
\bottomrule\noalign{}
\endlastfoot
family gives money & 0.993 & 0.953 & 1.035 & 1.091 & 1 \\
friends gives money & 0.915 & 0.809 & 1.036 & 1.412 & 1 \\
community gives money & 0.911 & 0.796 & 1.042 & 1.425 & 1 \\

\end{longtable}

For `family gives money', the effect estimate on the risk ratio scale is
0.993 {[}0.953, 1.035{]}. The E-value for this estimate is 1.091, with a
lower bound of 1. At this lower bound, unmeasured confounders would need
a minimum association strength with both the intervention sequence and
outcome of 1 to negate the observed effect. Weaker confounding would not
overturn it. We infer \textbf{that evidence for causality is not
reliable}.

For `friends give money', the effect estimate on the risk ratio scale is
0.915 {[}0.809, 1.036{]}. The E-value for this estimate is 1.412, with a
lower bound of 1. At this lower bound, unmeasured confounders would need
a minimum association strength with both the intervention sequence and
outcome of 1 to negate the observed effect. Weaker confounding would not
overturn it. We infer \textbf{that evidence for causality is not
reliable}.

For `community gives money', the effect estimate on the risk ratio scale
is 0.911 {[}0.796, 1.042{]}. The E-value for this estimate is 1.425,
with a lower bound of 1. At this lower bound, unmeasured confounders
would need a minimum association strength with both the intervention
sequence and outcome of 1 to negate the observed effect. Weaker
confounding would not overturn it. We infer \textbf{that evidence for
causality is not reliable}.

\begin{figure}

\centering{

\includegraphics{24-bulbulia-lmtp-coop-church_files/figure-pdf/fig-study_3-1.pdf}

}

\caption{\label{fig-study_3}This figure reports the results of model
estimates for the three causal contrasts of interest on help received
from others during the past week (yes/no). The causal contrasts are: (A)
Regular vs.~Zero Religious Service (B) Regular Religious Service
vs.~Status Quo; (C) Zero Religious Service vs.~Status Quo. Contrasts are
expressed on the risk ratio scale.}

\end{figure}%

\newpage{}

\subsubsection{Additional Study: Comparison of Causal Inference Results
with Cross-Sectional
Regressions}\label{additional-study-comparison-of-causal-inference-results-with-cross-sectional-regressions}

To better evaluate the contributions of our methodology to current
practice, we conducted a series of cross-sectional analyses using the
baseline wave data. We quantified the statistical associations between
religious service attendance and our focal prosocial outcomes. We
included all regression covariates from the causal models (including
sample weights) for each analysis, obviously omitting the outcome
measured at baseline, i.e.~the response variable, as a predictor.

\textbf{Cross-sectional volunteering result}: the change in expected
hours of volunteer work for a one-unit increase in religious service
attendance is b = 0.31; (95\% CI 0.28, 0.34). Multiplying this by 4.2
gives a monthly estimate of 77.95 minutes. This result is 2.58 per cent
greater than the effect estimated from the `regular vs.~zero' causal
contrast, indicating an overstatement in the regression model.

\textbf{Cross-sectional charitable donations result}: The coefficient
for religious service on annual charitable donations suggests a change
in expected donation amount per unit increase in attendance is: b = 451;
(95\% CI 408, 494). When adjusted to a monthly rate by multiplying by
4.2, this value equals NZ Dollars 1894.45. It is 2.89 per cent greater
than our causal contrast estimate, again indicating an overstatement by
the regression.

For Studies 2 and 3, which focus on community help received, we adjusted
our analysis for the non-collapsibility of odds ratios by assuming a
Poisson distribution for the outcome variables, obtaining a rate ratio
that approximates a risk ratio
(\citeproc{ref-huitfeldt2019collapsibility}{Huitfeldt \emph{et al.}
2019}; \citeproc{ref-vanderweele2020}{VanderWeele \emph{et al.} 2020}):

\textbf{Cross-sectional community assistance received result: Time}: the
exponentiated change in expectation for a one-unit change in religious
service attendance is b = 1.17; (95\% CI 1.14, 1.19) approximate risk
rate ratio. The monthly rate ratio derived by multiplying this
coefficient by 4.2 is 1.921. This estimate is 1.39 per cent greater than
the `regular vs.~zero' causal estimate, pointing to an overestimation in
the regression.

\textbf{Cross-sectional community assistance received result: Money}:
similarly, the exponentiated change for money received yields an
approximate risk ratio of b = 1.18; (95\% CI 1.08, 1.27). The monthly
risk ratio, after adjustment, is 1.996. This rate ratio is 1.45 per cent
greater than the causal estimate, again revealing the bias of
cross-sectional regressions.

\emph{These findings underscore that the results of cross-sectional
regressions, although suggestive, can considerably diverge from those
obtained from the causal analysis of panel data.}

\newpage{}

\subsubsection{What is the ``Cash Value'' of Religious Service
Attendance for Charitable Donations in New
Zealand?}\label{what-is-the-cash-value-of-religious-service-attendance-for-charitable-donations-in-new-zealand}

We leveraged our findings to estimate the economic value of religious
service attendance by comparing expected donation amounts under
different scenarios--- `Regular Religious Service', `Zero Religious
Service', and the `Status Quo'.

\begin{itemize}
\tightlist
\item
  \textbf{Regular Religious Service}: an increase in religious service
  attendance yields an individual average donation rate of 1638.98.
\item
  \textbf{Zero Religious Service}: Reducing religious service attendance
  to zero yields an average donation rate of 984.59.
\item
  \textbf{Status Quo}: the expected individual average donation rate is
  currently 1037.14.
\end{itemize}

With 3,989,000 adult residents in New Zealand in 2021,\footnote{\href{https://www.stats.govt.nz/information-releases/national-population-estimates-at-30-june-2021}{National
  Population Estimates at 30 June 2021}}.

\begin{itemize}
\tightlist
\item
  Multiplying the adult population by the average donation sum gives a
  status quo national estimate for charitable giving of NZD
  4,137,151,460.
\item
  The net gain to charity from country-wide regular attendance at
  religious services, compared to the status quo, is NZD 2,400,739,760.
\item
  Conversely, the net cost to charity from a complete cessation of
  regular religious service attendance is NZD -209,621,950. However,
  recall the confidence interval crosses zero, and this effect is not
  reliable.
\end{itemize}

To provide context, consider these economic consequences against the New
Zealand government's annual budget in year outcomes were measured
(2021-2022):

\begin{itemize}
\tightlist
\item
  The gain from a nationwide adoption of regular religious service
  represents 0.041 of New Zealand's annual government budget 2021.
\item
  Ignoring the unreliability of the loss intervention, a nationwide
  discontinuation of all religious services constitutes 0.004 of the
  annual government budget.
\end{itemize}

Clearly, some amount of charity is always better than none. We do not
devalue any form of charitable giving; \emph{every act matters.}
Furthermore, our population-wide estimates for the aggregated effects of
\emph{religious behaviour} do not clarify the effects of gaining or
losing \emph{religious institutions}; that is, our analysis reflects the
contributions made by those who participate in religious institutions.

Focusing on the individual level and aggregating effects across the
adult population, the counterfactual scenario in which all New Zealand
adults regularly attend religious services shows a substantial increase
in society-wide charitable support compared to the status quo, one year
after the intervention. However, suppose New Zealand were to experience
a complete cessation of religious service attendance. We do not find
evidence that the landscape of charitable giving would change from its
current state one year later.

\subsection{Discussion}\label{discussion}

\subsubsection{Considerations}\label{considerations}

First, consider \textbf{unmeasured confounding}: although we use robust
methods for causal inference, our results are contingent upon the
adequacy of our identification strategy. We employ non-parametric
machine learning ensembles for causal estimation, enhancing modelling
efficiency and reducing threats of model misspecification; moreover, our
sensitivity analyses describe the potential effects of unmeasured
confounding. However, the presence and influence of such confounders
remains uncertain because we have not, and clearly cannot, randomise
individuals into treatment conditions.

Second, \textbf{nonparametric estimation/machine learning is new}:
although the threat of model misspecification is greatly diminished in
non-parametric machine learning, whether our estimators obtain valid
standard errors is a function of sample size. Against this concern, our
relatively large sample size of 33,198 participants offers confidence.
However, investigators should be mindful that the application of
non-parametric machine learning ensembles to psychological data is new
and untested, and those who rely on small samples should use these
estimators with caution because standard errors are uncertain (so to
speak) in small samples (\citeproc{ref-van2014discussion}{Laan \emph{et
al.} 2014}).

Third, our study confronts the spectre of \textbf{measurement error}:
both direct and correlated measurement errors can introduce biases,
either by implying effects where none exist or by attenuating true
effects (\citeproc{ref-vanderweele2012MEASUREMENT}{VanderWeele and
Hernán 2012}). Although evaluating prosociality using multiple measures
helps to mitigate concerns about whether religious service affects
charitable donations and volunteering, unknown combinations of
measurement error might nevertheless bias our results. The outcomes and
estimates we report here are best-considered approximations.

Fourth, we do not examine \textbf{treatment effect heterogeneity}:
identifying which subgroups experience the strongest responses remains a
task for future research. Such investigations are crucial for making
informed policy decisions and tailoring advice relevant to those
subgroups of the population who might benefit most. Perhaps the most
obvious stratum of interest are those who already affiliate with a
religion at baseline.

Sixth, the \textbf{transportability of our findings remains unclear}:
New Zealand is our target population. Our findings generalise to this
population. The transportability of our findings to other settings --
that is, whether our results generalise beyond this target New Zealand
population -- remains an open question, a matter for future
investigations.

\subsubsection{Observations and
Recommendations}\label{observations-and-recommendations}

First, we emphasise that \textbf{different causal questions will lead to
different causal answers.} Our study obtains causal effect estimates by
(1) stating contrasts for specific interventions that increase and
decrease religious service attendance across the population; (2)
balancing these counterfactual interventions on confounders through
doubly robust estimators that flexibly employ machine learning emsembles
and cross-validation, and (3) evaluating contrasts of counterfactual
projections one year later. In the scientifically interesting contrast
`Regular vs Zero', we observe that the social benefits of regular
religious attendance on charitable donations and volunteering are
considerable. Moreover, we also find the expected effect of regular
religious service enacted in contemporary New Zealand would strongly
enhance charity and volunteering above the status quo, a point that
should shape discussions among critics of religion. Conversely, the
overall social effects of altogether ceasing religious services are
minimal compared with the status quo. Should New Zealand's society
entirely forego regular religious attendance, we would not anticipate
significant differences in the aggregate of charitable donations and
volunteering one year later. Although we did not investigate longer-term
effects, this finding should alleviate at least some concerns among
those wary of New Zealand's longstanding secular trend. \textbf{We
recommend that future research distinguish between models of cultural
gains and losses} (as exemplified by Van Tongeren \emph{et al.}
(\citeproc{ref-vantongeren2020}{2020})). More generally, \textbf{we
recommend that investigators clearly define, compute, and discuss their
prespecified causal contrasts in light of well-defined scientific
theories and policy interests.}

Second, we caution that traditional \textbf{measures of effect such as
Cohen's \emph{D} or \(R^2\) may present misleading pictures of practical
significance.} Our study computed standardised differences in the
continuous charity and volunteering outcomes. The contrast between
regular religious service and no service yeilds and effects size of
0.132 {[}0.102, 0.161{]}. This effect would be categorised as `small'
effect by contemporary experimental conventions. Nonetheless, we project
that the difference in annual charitable donations between the regular
service condition and the status quo amounts to NZD 1638.98 versus NZD
1037.14. Not only is this effect practically significant at the
individual level, the difference amounts to over 4\% of New Zealand's
Annual Government Budget (2021). \textbf{We recommend using causal
inference to prioritise, evaluate and communicate practical effect
sizes. Conventional statistical measures should not be mistaken for
metrics of practical significance.}

Third, our novel measures of prosociality---\textbf{assistance and
financial support received from one's community}--align with the
expectations of self-reported data on charitable donations and
volunteering. We detect marginal causal effects of religious service
attendance on \textbf{self-reported prosocial dependence and benefit}
compared to non-attendance. Note that just as our measures of prosocial
actions (charity/volunteering) do not capture the targets of prosocial
help, likewise our measures of community help received (time/money) do
not capture the specific sources of help beyond the categories of
``community,'' ``freinds,'' and ``family''. Nevertheless, the
evolutionary theories of religious prosociality that motivate this study
are grounded on within group-cooperative benefits
(\citeproc{ref-bulbulia2009charismatic}{Bulbulia 2009};
\citeproc{ref-johnson2015}{Johnson 2015};
\citeproc{ref-schloss2011evolutionary}{Schloss and Murray 2011};
\citeproc{ref-sosis2003cooperation}{Sosis and Bressler 2003};
\citeproc{ref-whitehouse2023}{Whitehouse \emph{et al.} 2023}). This
consistency between self-reported giving and help received aligns with
public data in New Zealand;\footnote{\url{https://www.charities.govt.nz/view-data/}}
where religious institutional charity accounts for 40\% of the
charitable sector (refer to McLeod (\citeproc{ref-McLeod2020}{2020}),
p.17, and also Brooks (\citeproc{ref-brooks2004faith}{2004}); Woodyard
and Grable (\citeproc{ref-woodyard2014doing}{2014}); Monsma
(\citeproc{ref-monsma2007religion}{2007})). Additionally, it has been
suggested that religious institutions are particularly efficient due to
low administrative costs and high volunteer engagement
(\citeproc{ref-bekkers2011literature}{Bekkers and Wiepking 2011};
\citeproc{ref-khanna1995charity}{Khanna \emph{et al.} 1995};
\citeproc{ref-McLeod2020}{McLeod 2020 p. 26}). We note that the
community signals of religious giving are consistent with the theory
that religion promotes altruism outside families, friendships, and even
among strangers (\citeproc{ref-mccullough2020kindness}{McCullough
2020}). \textbf{In contexts where investigators are concerned about
self-presentation biases when surveying charitable donations and
volunteering, we recommend piloting and deploying measures that capture
community-received assistance.}

Fourth, our analysis shows that standard cross-sectional regressions
overstate the causal effects estimates obtained from a conscientious
application of causal methods to time-series data. Importantly,
associational models might also \emph{understate} true causal
relationships, especially if adjustments involve mediators
(\citeproc{ref-mcelreath2020}{McElreath 2020};
\citeproc{ref-westreich2013}{Westreich and Greenland 2013}). Despite
their effectiveness and relevance, causal inference methods remain
uncommon in many human sciences, including the psychological sciences,
where traditional associational methods still dominate. Transitioning to
causal data science might seem daunting. Yet, causation inherently
unfolds in time, with causes preceding effects. Estimating average
causal effects involves multi-stepped workflows and robust time-series
data (\citeproc{ref-bulbulia2023}{Bulbulia 2023};
\citeproc{ref-neal2020introduction}{Neal 2020}). The need for
researchers to develop new skills and the requirement for robust
time-series data have profound implications for research design,
funding, and the accepted pace of scientific progress. However, causal
inference methods have become standard in epidemiology
(\citeproc{ref-lash2020}{Lash \emph{et al.} 2020}) and are rapidly
gaining traction in economics (\citeproc{ref-angrist2009mostly}{Angrist
and Pischke 2009}; \citeproc{ref-athey2019}{Athey \emph{et al.} 2019};
\citeproc{ref-athey2021}{Athey and Wager 2021}) and sociology
(\citeproc{ref-montgomery2018}{Montgomery \emph{et al.} 2018}).
Moreover, new global panel studies suited for researching religion
Johnson and VanderWeele (\citeproc{ref-johnson2022global}{2022}) offer
exciting opportunities for psychological scientists to apply causal
methods to those psychological questions at the heart of our discipline.
The intellectual benefits of retooling, then, are considerable (refer
also to Major-Smith (\citeproc{ref-major2023exploring}{2023})).

Suppose these intellectual benefits were not sufficient motivation. It
is vital to remember that investigators should be able to coherently
evaluate the practical interest of their findings. Absent causal
inferential workflows, associational methods cannot deliver. This
\emph{practical significance gap} remains even when associational
methods are applied to time series data and even when the associational
methods are sophisticated---as in the statistical structural equation
tradition and multilevel modelling traditions. A \emph{practical
significance gap} arises because we cannot interpret associations as
causation unless we have obtained balance on the treatments to be
compared, unless we know that causes precede effects in time, and unless
we ensure that all disabling assumptions have been addressed.

Because the search for both scientific knowledge and practical
understanding animates investigator interests, \textbf{we strongly
recommend the broader adoption of causal inferential methods in the
psychological sciences.} see VanderWeele
(\citeproc{ref-vanderweele2015}{2015})

Finally, it is essential to frame our contribution. This study is one of
the first to combine robust causal methods with national panel data to
quantitatively investigate the social consequences of religious
behaviour. It provides evidence that religious service attendance
causally affects charitable donations and volunteering. These findings
establish \emph{how much} religious service affects charity and
volunteering in New Zealand and possibly in similar cultural settings.
We hope this research will serve as a template for studies investigating
the social consequences of religious behaviour in other countries.

However, much remains to be discovered about the operations of these
effects, even within New Zealand. Here, we have raised a series of
\emph{how much?} questions about the influence of religious service on
charity and volunteering. We do not elucidate the pathways by which
these effects operate, nor do we describe their heterogeneity. A
substantial body of research has examined the psychological and cultural
features through which religious participation affects individuals
(\citeproc{ref-cristofori2016neural}{Cristofori \emph{et al.} 2016};
\citeproc{ref-konvalinka2011synchronized}{Konvalinka \emph{et al.}
2011}; \citeproc{ref-lang2015effects}{Lang \emph{et al.} 2015};
\citeproc{ref-mccullough2016christian}{McCullough \emph{et al.} 2016};
\citeproc{ref-norenzayan2016}{Norenzayan \emph{et al.} 2016};
\citeproc{ref-shaver2015evolution}{Shaver 2015};
\citeproc{ref-shaver2020church}{Shaver \emph{et al.} 2020};
\citeproc{ref-watts2016}{Watts \emph{et al.} 2016}). However, obtaining
a data-driven causal understanding about the pathways through which
these features operation presents both conception and data challenges
(\citeproc{ref-diaz2021nonparametric}{Dı́az \emph{et al.} 2021},
\citeproc{ref-Diaz2023}{2023}; \citeproc{ref-robins1992}{Robins and
Greenland 1992}; \citeproc{ref-vanderweele2015}{VanderWeele 2015}). Such
challenges are underappreciated but in theory surmountable
(\citeproc{ref-bulbulia2023}{Bulbulia 2023}). By carefully navigating
the uncertainties inherent in observational data and leveraging the
insights from comprehensive time-series data, we should eventually seek
inferences about these mechanisms of operation and their variations. In
the years ahead, \textbf{we recommend a broader application of causal
inferential methods over more extended repeated-measurment intervals to
address questions of causal mediation and interaction}.

\newpage{}

\subsubsection{Ethics}\label{ethics}

The University of Auckland Human Participants Ethics Committee reviews
the NZAVS every three years. Our most recent ethics approval statement
is as follows: The New Zealand Attitudes and Values Study was approved
by the University of Auckland Human Participants Ethics Committee on
26/05/2021 for six years until 26/05/2027, Reference Number UAHPEC22576.

\subsubsection{Data Availability}\label{data-availability}

The data described in the paper are part of the New Zealand Attitudes
and Values Study (NZAVS). Full copies of the NZAVS data files are held
by members of the NZAVS management team and research group. A
de-identified dataset containing only the variables analysed in this
manuscript is available upon request from the corresponding author, or
any member of the NZAVS advisory board for the purposes of replication
or checking of any published study using NZAVS data. The code for the
analysis can be found at:
\url{https://github.com/go-bayes/models/blob/main/scripts/24-bulbulia-church-prosocial.R}.

\subsubsection{Acknowledgements}\label{acknowledgements}

The New Zealand Attitudes and Values Study is supported by a grant from
the Templeton Religious Trust (TRT0196; TRT0418). JB received support
from the Max Planck Institute for the Science of Human History. The
funders had no role in preparing the manuscript or the decision to
publish.

\subsubsection{Author Statement}\label{author-statement}

JB conceived of the study and approach. CS led NZAVS data collection.
All authors contributed to the manuscript.

\newpage{}

\subsection{Appendix A: Measures}\label{appendix-measures}

\paragraph{Age (waves: 1-15)}\label{age-waves-1-15}

We asked participants' ages in an open-ended question (``What is your
age?'' or ``What is your date of birth?'').

\paragraph{Born in New Zealand}\label{born-in-new-zealand}

\paragraph{Charitable Donations (Study 1
outcome)}\label{charitable-donations-study-1-outcome}

Using one item from Hoverd and Sibley
(\citeproc{ref-hoverd_religious_2010}{2010}), we asked participants,
``How much money have you donated to charity in the last year?''.

\paragraph{Charitable Volunteering (Study 1
outcome)}\label{charitable-volunteering-study-1-outcome}

We measured hours of volunteering using one item from Sibley \emph{et
al.} (\citeproc{ref-sibley2011}{2011}): ``Hours spent \ldots{}
voluntary/charitable work.''

\paragraph{Children Number (waves: 1-3,
4-15)}\label{children-number-waves-1-3-4-15}

We measured the number of children using one item from Bulbulia \emph{et
al.} (\citeproc{ref-Bulbulia_2015}{2015}). We asked participants, ``How
many children have you given birth to, fathered, or adopted. How many
children have you given birth to, fathered, or adopted?'' or ``How many
children have you given birth to, fathered, or adopted. How many
children have you given birth to, fathered, and/or parented?'' (waves:
12-15).

\paragraph{Disability}\label{disability}

We assessed disability with a one-item indicator adapted from Verbrugge
(\citeproc{ref-verbrugge1997}{1997}). It asks, ``Do you have a health
condition or disability that limits you and that has lasted for 6+
months?'' (1 = Yes, 0 = No).

\paragraph{Education Attainment (waves: 1,
4-15)}\label{education-attainment-waves-1-4-15}

We asked participants, ``What is your highest level of qualification?''.
We coded participants' highest finished degree according to the New
Zealand Qualifications Authority. Ordinal-Rank 0-10 NZREG codes (with
overseas school quals coded as Level 3, and all other ancillary
categories coded as missing)
See:https://www.nzqa.govt.nz/assets/Studying-in-NZ/New-Zealand-Qualification-Framework/requirements-nzqf.pdf

\paragraph{Employment (waves: 1-3,
4-11)}\label{employment-waves-1-3-4-11}

We asked participants, ``Are you currently employed? (This includes
self-employed or casual work)''.

\paragraph{Ethnicity}\label{ethnicity}

Based on the New Zealand Census, we asked participants, ``Which ethnic
group(s) do you belong to?''. The responses were: (1) New Zealand
European; (2) Māori; (3) Samoan; (4) Cook Island Māori; (5) Tongan; (6)
Niuean; (7) Chinese; (8) Indian; (9) Other such as DUTCH, JAPANESE,
TOKELAUAN. Please state:. We coded their answers into four groups:
Maori, Pacific, Asian, and Euro (except for Time 3, which used an
open-ended measure).

\paragraph{Fatigue}\label{fatigue}

We assessed subjective fatigue by asking participants, ``During the last
30 days, how often did \ldots{} you feel exhausted?'' Responses were
collected on an ordinal scale (0 = None of The Time, 1 = A little of The
Time, 2 = Some of The Time, 3 = Most of The Time, 4 = All of The Time).

\paragraph{Honesty-Humility-Modesty Facet (waves:
10-14)}\label{honesty-humility-modesty-facet-waves-10-14}

Participants indicated the extent to which they agree with the following
four statements from Campbell \emph{et al.}
(\citeproc{ref-campbell2004}{2004}) , and Sibley \emph{et al.}
(\citeproc{ref-sibley2011}{2011}) (1 = Strongly Disagree to 7 = Strongly
Agree)

\begin{verbatim}
i.  I want people to know that I am an important person of high status, (Waves: 1, 10-14)
ii. I am an ordinary person who is no better than others.
iii. I wouldn't want people to treat me as though I were superior to them.
iv. I think that I am entitled to more respect than the average person is.
\end{verbatim}

\paragraph{Hours of Childcare}\label{hours-of-childcare}

We measured hours of exercising using one item from Sibley \emph{et al.}
(\citeproc{ref-sibley2011}{2011}): 'Hours spent \ldots{} looking after
children.''

To stabilise this indicator, we took the natural log of the response +
1.

\paragraph{Hours of Housework}\label{hours-of-housework}

We measured hours of exercising using one item from Sibley \emph{et al.}
(\citeproc{ref-sibley2011}{2011}): ``Hours spent \ldots{}
housework/cooking''

To stabilise this indicator, we took the natural log of the response +
1.

\paragraph{Hours of Exercise}\label{hours-of-exercise}

We measured hours of exercising using one item from Sibley \emph{et al.}
(\citeproc{ref-sibley2011}{2011}): ``Hours spent \ldots{}
exercising/physical activity''

To stabilise this indicator, we took the natural log of the response +
1.

\paragraph{Hours of Childcare}\label{hours-of-childcare-1}

We measured hours of exercising using one item from Sibley \emph{et al.}
(\citeproc{ref-sibley2011}{2011}): 'Hours spent \ldots{} looking after
children.''

To stabilise this indicator, we took the natural log of the response +
1.

\paragraph{Hours of Exercise}\label{hours-of-exercise-1}

We measured hours of exercising using one item from Sibley \emph{et al.}
(\citeproc{ref-sibley2011}{2011}): ``Hours spent \ldots{}
exercising/physical activity''

To stabilise this indicator, we took the natural log of the response +
1.

\paragraph{Hours of Housework}\label{hours-of-housework-1}

We measured hours of exercising using one item from Sibley \emph{et al.}
(\citeproc{ref-sibley2011}{2011}): ``Hours spent \ldots{}
housework/cooking''

To stabilise this indicator, we took the natural log of the response +
1.

\paragraph{Hours of Sleep}\label{hours-of-sleep}

Participants were asked, ``During the past month, on average, how many
hours of \emph{actual sleep} did you get per night?''.

\paragraph{Hours of Work}\label{hours-of-work}

We measured work hours using one item from Sibley \emph{et al.}
(\citeproc{ref-sibley2011}{2011}): ``Hours spent \ldots{} working in
paid employment.''

To stabilise this indicator, we took the natural log of the response +
1.

\paragraph{Income (waves: 1-3, 4-15)}\label{income-waves-1-3-4-15}

Participants were asked, ``Please estimate your total household income
(before tax) for the year XXXX''. To stabilise this indicator, we first
took the natural log of the response + 1, and then centred and
standardised the log-transformed indicator.

\paragraph{Kessler-6: Psychological Distress (waves:
2-3,4-15)}\label{kessler-6-psychological-distress-waves-2-34-15}

We measured psychological distress using the Kessler-6 scale
(kessler2002?), which exhibits strong diagnostic concordance for
moderate and severe psychological distress in large, crosscultural
samples (kessler2010?; prochaska2012?). Participants rated during the
past 30 days, how often did\ldots{} (1) ``\ldots{} you feel hopeless'';
(2) ``\ldots{} you feel so depressed that nothing could cheer you up'';
(3) ``\ldots{} you feel restless or fidgety''; (4)``\ldots{} you feel
that everything was an effort''; (5) ``\ldots{} you feel worthless'';
(6) '' you feel nervous?'' Ordinal response alternatives for the
Kessler-6 are: ``None of the time''; ``A little of the time''; ``Some of
the time''; ``Most of the time''; ``All of the time.''

\paragraph{Male Gender (waves: 1-15)}\label{male-gender-waves-1-15}

We asked participants' gender in an open-ended question: ``what is your
gender?'' or ``Are you male or female?'' (waves: 1-5). Female was coded
as 0, Male as 1, and gender diverse coded as 3
(\citeproc{ref-fraser_coding_2020}{Fraser \emph{et al.} 2020}). (or 0.5
= neither female nor male)

Here, we coded all those who responded as Male as 1, and those who did
not as 0.

\paragraph{Mini-IPIP 6 (waves:
1-3,4-15)}\label{mini-ipip-6-waves-1-34-15}

We measured participants' personalities with the Mini International
Personality Item Pool 6 (Mini-IPIP6) (\citeproc{ref-sibley2011}{Sibley
\emph{et al.} 2011}), which consists of six dimensions and each
dimension is measured with four items:

\begin{enumerate}
\def\labelenumi{\arabic{enumi}.}
\item
  agreeableness,

  \begin{enumerate}
  \def\labelenumii{\roman{enumii}.}
  \tightlist
  \item
    I sympathize with others' feelings.
  \item
    I am not interested in other people's problems. (r)
  \item
    I feel others' emotions.
  \item
    I am not really interested in others. (r)
  \end{enumerate}
\item
  conscientiousness,

  \begin{enumerate}
  \def\labelenumii{\roman{enumii}.}
  \tightlist
  \item
    I get chores done right away.
  \item
    I like order.
  \item
    I make a mess of things. (r)
  \item
    I often forget to put things back in their proper place. (r)
  \end{enumerate}
\item
  extraversion,

  \begin{enumerate}
  \def\labelenumii{\roman{enumii}.}
  \tightlist
  \item
    I am the life of the party.
  \item
    I don't talk a lot. (r)
  \item
    I keep in the background. (r)
  \item
    I talk to a lot of different people at parties.
  \end{enumerate}
\item
  honesty-humility,

  \begin{enumerate}
  \def\labelenumii{\roman{enumii}.}
  \tightlist
  \item
    I feel entitled to more of everything. (r)
  \item
    I deserve more things in life. (r)
  \item
    I would like to be seen driving around in a very expensive car. (r)
  \item
    I would get a lot of pleasure from owning expensive luxury goods.
    (r)
  \end{enumerate}
\item
  neuroticism, and

  \begin{enumerate}
  \def\labelenumii{\roman{enumii}.}
  \tightlist
  \item
    I have frequent mood swings.
  \item
    I am relaxed most of the time. (r)
  \item
    I get upset easily.
  \item
    I seldom feel blue. (r)
  \end{enumerate}
\item
  openness to experience

  \begin{enumerate}
  \def\labelenumii{\roman{enumii}.}
  \tightlist
  \item
    I have a vivid imagination.
  \item
    I have difficulty understanding abstract ideas. (r)
  \item
    I do not have a good imagination. (r)
  \item
    I am not interested in abstract ideas. (r)
  \end{enumerate}
\end{enumerate}

Each dimension was assessed with four items and participants rated the
accuracy of each item as it applies to them from 1 (Very Inaccurate) to
7 (Very Accurate). Items marked with (r) are reverse coded.

\paragraph{NZ-Born (waves: 1-2,4-15)}\label{nz-born-waves-1-24-15}

We asked participants, ``Which country were you born in?'' or ``Where
were you born? (please be specific, e.g., which town/city?)'' (waves:
6-15).

\paragraph{NZ Deprivation Index (waves:
1-15)}\label{nz-deprivation-index-waves-1-15}

We used the NZ Deprivation Index to assign each participant a score
based on where they live (\citeproc{ref-atkinson2019}{Atkinson \emph{et
al.} 2019}). This score combines data such as income, home ownership,
employment, qualifications, family structure, housing, and access to
transport and communication for an area into one deprivation score.

\paragraph{NZSEI Occupational Prestige and Status (waves:
8-15)}\label{nzsei-occupational-prestige-and-status-waves-8-15}

We assessed occupational prestige and status using the New Zealand
Socio-economic Index 13 (NZSEI-13) (\citeproc{ref-fahy2017a}{Fahy
\emph{et al.} 2017a}). This index uses the income, age, and education of
a reference group, in this case the 2013 New Zealand census, to
calculate a score for each occupational group. Scores range from 10
(Lowest) to 90 (Highest). This list of index scores for occupational
groups was used to assign each participant an NZSEI-13 score based on
their occupation.

We assessed occupational prestige and status using the New Zealand
Socio-economic Index 13 (NZSEI-13) (\citeproc{ref-fahy2017}{Fahy
\emph{et al.} 2017b}). This index uses the income, age, and education of
a reference group, in this case, the 2013 New Zealand census, to
calculate a score for each occupational group. Scores range from 10
(Lowest) to 90 (Highest). This list of index scores for occupational
groups was used to assign each participant an NZSEI-13 score based on
their occupation.

\paragraph{Opt-in}\label{opt-in}

The New Zealand Attitudes and Values Study allows opt-ins to the study.
Because the opt-in population may differ from those sampled randomly
from the New Zealand electoral roll; although the opt-in rate is low, we
include an indicator (yes/no) for this variable.

\paragraph{Partner (No/Yes)}\label{partner-noyes}

``What is your relationship status?'' (e.g., single, married, de-facto,
civil union, widowed, living together, etc.)

\paragraph{Politically Conservative}\label{politically-conservative}

We measured participants' political conservative orientation using a
single item adapted from Jost (\citeproc{ref-jost_end_2006-1}{2006}).

``Please rate how politically liberal versus conservative you see
yourself as being.''

(1 = Extremely Liberal to 7 = Extremely Conservative)

\subparagraph{Religious Service
Attendance}\label{religious-service-attendance}

If participants answered \emph{yes} to ``Do you identify with a religion
and/or spiritual group?'' we measured their frequency of church
attendence using one item from Sibley (\citeproc{ref-sibley2012}{2012}):
``how many times did you attend a church or place of worship in the last
month?''. Those participants who were not religious were imputed a score
of ``0''.

\paragraph{Rural/Urban Codes}\label{ruralurban-codes}

Participants residence locations were coded according to a five-level
ordinal categorisation ranging from ``Urban'' to Rural, see Sibley
(\citeproc{ref-sibley2021}{2021}).

\paragraph{Short-Form Health}\label{short-form-health}

Participants' subjective health was measured using one item (``Do you
have a health condition or disability that limits you, and that has
lasted for 6+ months?''; 1 = Yes, 0 = No) adapted from Verbrugge
(\citeproc{ref-verbrugge1997}{1997}).

\paragraph{Sample Origin}\label{sample-origin}

Wave enrolled in NZAVS, see Sibley (\citeproc{ref-sibley2021}{2021}).

\paragraph{Support received: money (waves 10-12) (Study 4
outcomes)}\label{support-received-money-waves-10-12-study-4-outcomes}

The NZAVS has a `revealed' measure of received help and support measured
in hours of support in the previous week. The items are:

\emph{Please estimate how much help you have received from the following
sources in the last week?}

\begin{itemize}
\tightlist
\item
  \emph{family\ldots MONEY (hours)}
\item
  \emph{friends\ldots MONEY (hours)}
\item
  \emph{members of my community\ldots MONEY (hours)}
\end{itemize}

Because this measure is highly variable, we convert responses to binary
indicators: \emph{0 = none/1 any}

\paragraph{Support received: time (waves 10-13) (Study 3
outcomes)}\label{support-received-time-waves-10-13-study-3-outcomes}

\emph{Please estimate how much help you have received from the following
sources in the last week.}

\begin{itemize}
\tightlist
\item
  \emph{family\ldots TIME (hours)}
\item
  \emph{friends\ldots TIME (hours)}
\item
  \emph{members of my community\ldots TIME (hours)}
\end{itemize}

Because this measure is highly variable, we convert responses to binary
indicators: \emph{0 = none/1 any}

\paragraph{Total Siblings}\label{total-siblings}

Participants were asked the following questions related to sibling
counts:

\begin{itemize}
\tightlist
\item
  Were you the 1st born, 2nd born, or 3rd born, etc, child of your
  mother?
\item
  Do you have siblings?
\item
  How many older sisters do you have?
\item
  How many younger sisters do you have?
\item
  How many older brothers do you have?
\item
  How many younger brothers do you have?
\end{itemize}

A single score was obtained from sibling counts by summing responses to
the ``How many\ldots{}'' items. From these scores, an ordered factor was
created ranging from 0 to 7, where participants with more than 7
siblings were grouped into the highest category.

\newpage{}

\subsection{Appendix B. Baseline Demographic
Statistics}\label{appendix-demographics}

\begin{longtable}[]{@{}ll@{}}

\caption{\label{tbl-table-demography}Baseline demographic statistics}

\tabularnewline

\toprule\noalign{}
\textbf{Exposure + Demographic Variables} & \textbf{N = 33,198} \\
\midrule\noalign{}
\endhead
\bottomrule\noalign{}
\endlastfoot
\textbf{Age} & NA \\
Mean (SD) & 51 (14) \\
Range & 18, 96 \\
IQR & 41, 61 \\
\textbf{Agreeableness} & NA \\
Mean (SD) & 5.37 (0.98) \\
Range & 1.00, 7.00 \\
IQR & 4.75, 6.00 \\
Unknown & 272 \\
\textbf{Born Nz} & 26,197 (79\%) \\
Unknown & 34 \\
\textbf{Children Num} & NA \\
Mean (SD) & 1.76 (1.44) \\
Range & 0.00, 14.00 \\
IQR & 0.00, 3.00 \\
\textbf{Conscientiousness} & NA \\
Mean (SD) & 5.14 (1.04) \\
Range & 1.00, 7.00 \\
IQR & 4.50, 6.00 \\
Unknown & 266 \\
\textbf{Education Level Coarsen} & NA \\
no\_qualification & 769 (2.3\%) \\
cert\_1\_to\_4 & 11,278 (34\%) \\
cert\_5\_to\_6 & 4,281 (13\%) \\
university & 8,947 (27\%) \\
post\_grad & 3,892 (12\%) \\
masters & 2,956 (9.0\%) \\
doctorate & 891 (2.7\%) \\
Unknown & 184 \\
\textbf{Employed} & 26,379 (80\%) \\
Unknown & 26 \\
\textbf{Eth Cat} & NA \\
euro & 27,404 (83\%) \\
maori & 3,424 (10\%) \\
pacific & 707 (2.1\%) \\
asian & 1,438 (4.4\%) \\
Unknown & 225 \\
\textbf{Extraversion} & NA \\
Mean (SD) & 3.88 (1.20) \\
Range & 1.00, 7.00 \\
IQR & 3.00, 4.75 \\
Unknown & 266 \\
\textbf{Hlth Disability} & 7,558 (23\%) \\
Unknown & 561 \\
\textbf{Hlth Fatigue} & NA \\
0 & 5,289 (16\%) \\
1 & 10,940 (33\%) \\
2 & 10,196 (31\%) \\
3 & 4,862 (15\%) \\
4 & 1,577 (4.8\%) \\
Unknown & 334 \\
\textbf{Hlth Sleep Hours} & NA \\
Mean (SD) & 6.95 (1.11) \\
Range & 2.50, 16.00 \\
IQR & 6.00, 8.00 \\
Unknown & 1,528 \\
\textbf{Honesty Humility} & NA \\
Mean (SD) & 5.49 (1.15) \\
Range & 1.00, 7.00 \\
IQR & 4.75, 6.50 \\
Unknown & 269 \\
\textbf{Hours Children log} & NA \\
Mean (SD) & 1.10 (1.58) \\
Range & 0.00, 5.13 \\
IQR & 0.00, 2.20 \\
Unknown & 875 \\
\textbf{Hours Exercise log} & NA \\
Mean (SD) & 1.57 (0.83) \\
Range & 0.00, 4.39 \\
IQR & 1.10, 2.08 \\
Unknown & 875 \\
\textbf{Hours Housework log} & NA \\
Mean (SD) & 2.15 (0.77) \\
Range & 0.00, 5.13 \\
IQR & 1.79, 2.71 \\
Unknown & 875 \\
\textbf{Hours Work log} & NA \\
Mean (SD) & 2.64 (1.59) \\
Range & 0.00, 4.62 \\
IQR & 1.10, 3.71 \\
Unknown & 875 \\
\textbf{Household Inc log} & NA \\
Mean (SD) & 11.41 (0.76) \\
Range & 0.69, 14.92 \\
IQR & 11.00, 11.92 \\
Unknown & 1,352 \\
\textbf{Kessler6 Sum} & NA \\
Mean (SD) & 5 (4) \\
Range & 0, 24 \\
IQR & 2, 7 \\
Unknown & 297 \\
\textbf{Male} & 11,975 (36\%) \\
\textbf{Modesty} & NA \\
Mean (SD) & 6.03 (0.90) \\
Range & 1.00, 7.00 \\
IQR & 5.50, 6.75 \\
Unknown & 11 \\
\textbf{Neuroticism} & NA \\
Mean (SD) & 3.45 (1.15) \\
Range & 1.00, 7.00 \\
IQR & 2.50, 4.25 \\
Unknown & 274 \\
\textbf{Nz Dep2018} & NA \\
Mean (SD) & 4.69 (2.70) \\
Range & 1.00, 10.00 \\
IQR & 2.00, 7.00 \\
Unknown & 233 \\
\textbf{Nzsei 13 l} & NA \\
Mean (SD) & 55 (16) \\
Range & 10, 90 \\
IQR & 42, 69 \\
Unknown & 172 \\
\textbf{Openness} & NA \\
Mean (SD) & 4.99 (1.12) \\
Range & 1.00, 7.00 \\
IQR & 4.25, 5.75 \\
Unknown & 267 \\
\textbf{Partner} & 24,869 (76\%) \\
Unknown & 422 \\
\textbf{Political Conservative} & NA \\
1 & 1,777 (5.6\%) \\
2 & 6,563 (21\%) \\
3 & 6,505 (20\%) \\
4 & 9,373 (29\%) \\
5 & 4,813 (15\%) \\
6 & 2,378 (7.5\%) \\
7 & 483 (1.5\%) \\
Unknown & 1,306 \\
\textbf{Religion Church Round} & NA \\
0 & 27,653 (83\%) \\
1 & 1,077 (3.2\%) \\
2 & 777 (2.3\%) \\
3 & 658 (2.0\%) \\
4 & 1,729 (5.2\%) \\
5 & 336 (1.0\%) \\
6 & 226 (0.7\%) \\
7 & 74 (0.2\%) \\
8 & 668 (2.0\%) \\
\textbf{Rural Gch 2018 l} & NA \\
1 & 20,361 (62\%) \\
2 & 6,390 (19\%) \\
3 & 4,020 (12\%) \\
4 & 1,816 (5.5\%) \\
5 & 380 (1.2\%) \\
Unknown & 231 \\
\textbf{Sample Frame Opt in} & 1,107 (3.3\%) \\
\textbf{Sample Origin} & NA \\
1-2 & 2,191 (6.6\%) \\
3-3.5 & 1,664 (5.0\%) \\
4 & 1,987 (6.0\%) \\
5-6-7 & 3,203 (9.6\%) \\
8-9 & 4,264 (13\%) \\
10 & 19,889 (60\%) \\
\textbf{Short Form Health} & NA \\
Mean (SD) & 5.06 (1.16) \\
Range & 1.00, 7.00 \\
IQR & 4.33, 6.00 \\
Unknown & 5 \\
\textbf{Total Siblings} & NA \\
Mean (SD) & 2.52 (1.80) \\
Range & 0.00, 23.00 \\
IQR & 1.00, 3.00 \\
Unknown & 689 \\

\end{longtable}

Table~\ref{tbl-table-demography} baseline demographic statistics for
couples who met inclusion criteria.

\newpage{}

\subsection{Appendix C: Treatment Statistics}\label{appendix-exposures}

\begin{longtable}[]{@{}
  >{\raggedright\arraybackslash}p{(\columnwidth - 4\tabcolsep) * \real{0.4247}}
  >{\raggedright\arraybackslash}p{(\columnwidth - 4\tabcolsep) * \real{0.2877}}
  >{\raggedright\arraybackslash}p{(\columnwidth - 4\tabcolsep) * \real{0.2877}}@{}}

\caption{\label{tbl-table-exposures-code}Exposures at baseline and
baseline + 1 (treatment) wave}

\tabularnewline

\toprule\noalign{}
\begin{minipage}[b]{\linewidth}\raggedright
\textbf{Exposure Variables by Wave}
\end{minipage} & \begin{minipage}[b]{\linewidth}\raggedright
\textbf{2018}, N = 33,198
\end{minipage} & \begin{minipage}[b]{\linewidth}\raggedright
\textbf{2019}, N = 33,198
\end{minipage} \\
\midrule\noalign{}
\endhead
\bottomrule\noalign{}
\endlastfoot
\textbf{Religion Church Round} & NA & NA \\
0 & 27,653 (83\%) & 28,028 (84\%) \\
1 & 1,077 (3.2\%) & 896 (2.7\%) \\
2 & 777 (2.3\%) & 737 (2.2\%) \\
3 & 658 (2.0\%) & 639 (1.9\%) \\
4 & 1,729 (5.2\%) & 1,672 (5.0\%) \\
5 & 336 (1.0\%) & 308 (0.9\%) \\
6 & 226 (0.7\%) & 205 (0.6\%) \\
7 & 74 (0.2\%) & 68 (0.2\%) \\
8 & 668 (2.0\%) & 645 (1.9\%) \\
Unknown & 0 & 0 \\
\textbf{Alert Level Combined} & NA & NA \\
no\_alert & 33,198 (100\%) & 23,751 (72\%) \\
early\_covid & 0 (0\%) & 3,643 (11\%) \\
alert\_level\_1 & 0 (0\%) & 2,821 (8.5\%) \\
alert\_level\_2 & 0 (0\%) & 836 (2.5\%) \\
alert\_level\_2\_5\_3 & 0 (0\%) & 552 (1.7\%) \\
alert\_level\_4 & 0 (0\%) & 1,595 (4.8\%) \\
Unknown & 0 & 0 \\

\end{longtable}

tbl-table-exposures-code presents baseline (NZAVS time 10) and exposure
wave (NZAVS time 11) statistics for the exposure variable: religious
service attendance (range 0-8). Responses coded as eight or above were
coded as ``8''. This decision to avoid spare treatments was based on
theoretical grounds, namely, that daily exposure would be similar in its
effects to more than daily exposure. We note that causal contrasts were
obtained for projects with either no attendance or four or more visits
per month. Hence this simplification of the measure is unlikely to
affect theoretical and practical inferences. All models adjusted for the
pandemic alert level because the treatment wave (NZAVS time 11) occurred
during New Zealand's COVID-19 pandemic. The pandemic is not a
``confounder'' because a confounder must be related to the treatment and
the outcome. At the end of the study, all participants had been exposed
to the pandemic. However, to satisfy the causal consistency assumption,
all treatments must be conditionally equivalent within levels of all
covariates (\citeproc{ref-vanderweele2013}{VanderWeele and Hernan
2013}). Because COVID affected the ability or willingness of individuals
to attend religious service, we included the lockdown condition as a
covariate (\citeproc{ref-sibley2021}{Sibley 2021}). To better enable
conditional independence within levels of the treatment variable, we
conditioned on the lead value of COVID-alert level at baseline. To
mitigate systematic biases arising from attrition and missingness, the
\texttt{lmtp} package uses inverse probability of censoring weights,
which were used when estimating the causal effects of the exposure on
the outcome.

\subsubsection{Binary Transition Table for The
Treatment}\label{binary-transition-table-for-the-treatment}

\begin{longtable}[]{@{}ccc@{}}

\caption{\label{tbl-transition-tablegain}Transition table for stability
and change in regular religious service (4x per month) between baseline
and treatment wave.}

\tabularnewline

\toprule\noalign{}
From & \textgreater=4 & \textless{} 4 \\
\midrule\noalign{}
\endhead
\bottomrule\noalign{}
\endlastfoot
\textgreater=4 & \textbf{29496} & 669 \\
\textless{} 4 & 804 & \textbf{2229} \\

\end{longtable}

Table~\ref{tbl-transition-tablegain} presents a transition matrix to
evaluate treatment shifts between baseline and treatment wave. Here, we
focus on the shift from/to monthly attendance at four or more visits per
month. Entries along the diagonal (in bold) indicate the number of
individuals who \textbf{stayed} in their initial state. By contrast, the
off-diagonal shows the transitions from the initial state (bold) to
another state in the following wave (off diagonal). Thus the cell
located at the intersection of row \(i\) and column \(j\), where
\(i \neq j\), gives us the counts of individuals moving from state \(i\)
to state \(j\).

\begin{longtable}[]{@{}ccc@{}}

\caption{\label{tbl-transition-tableloss}Transition table for stability
and change in zero religious service (0 x per month) between baseline
and treatment wave.}

\tabularnewline

\toprule\noalign{}
From & 0 & \textgreater{} 0 \\
\midrule\noalign{}
\endhead
\bottomrule\noalign{}
\endlastfoot
0 & \textbf{26762} & 891 \\
\textgreater{} 0 & 1266 & \textbf{4279} \\

\end{longtable}

Table~\ref{tbl-transition-tableloss} presents a transition matrix to
evaluate treatment shifts between baseline and treatment wave. Here, we
focus on the shift from/to zero religious service attendance. Again,
entries along the diagonal (in bold) indicate the number of individuals
who \textbf{stayed} in their initial state. By contrast, the
off-diagonal shows the transitions from the initial state (bold) to
another state in the following wave (off diagonal). Thus the cell
located at the intersection of row \(i\) and column \(j\), where
\(i \neq j\), gives us the counts of individuals moving from state \(i\)
to state \(j\).

\subsubsection{Imbalance of Confounding Covariates
Treatments}\label{imbalance-of-confounding-covariates-treatments}

Figure~\ref{fig-match_1} shows imbalance of covariates on the treatment
at the treatment wave. The variable on which there is strongest
imbalance is the baseline measure of religious service attendance. It is
important to adjust for this measure both for confounding control and to
better estimate an incident exposure effect for the religious service at
the treatment wave (in contrast to merely estimating a prevalence
effect). See VanderWeele \emph{et al.}
(\citeproc{ref-vanderweele2020}{2020}).

\begin{figure}

\centering{

\includegraphics{24-bulbulia-lmtp-coop-church_files/figure-pdf/fig-match_1-1.pdf}

}

\caption{\label{fig-match_1}This figure shows the imbalance in
covariates on the treatment}

\end{figure}%

\subsection{Appendix D: Baseline and End of Study Outcome
Statistics}\label{appendix-outcomes}

\begin{longtable}[]{@{}
  >{\raggedright\arraybackslash}p{(\columnwidth - 4\tabcolsep) * \real{0.4474}}
  >{\raggedright\arraybackslash}p{(\columnwidth - 4\tabcolsep) * \real{0.2763}}
  >{\raggedright\arraybackslash}p{(\columnwidth - 4\tabcolsep) * \real{0.2763}}@{}}

\caption{\label{tbl-table-outcomes}Outcomes at baseline and
end-of-study}

\tabularnewline

\toprule\noalign{}
\begin{minipage}[b]{\linewidth}\raggedright
\textbf{Outcome Variables by Wave}
\end{minipage} & \begin{minipage}[b]{\linewidth}\raggedright
\textbf{2018}, N = 33,198
\end{minipage} & \begin{minipage}[b]{\linewidth}\raggedright
\textbf{2020}, N = 33,198
\end{minipage} \\
\midrule\noalign{}
\endhead
\bottomrule\noalign{}
\endlastfoot
\textbf{Annual Charity} & 150 (40, 500) & 200 (20, 600) \\
Unknown & 1,076 & 6,730 \\
\textbf{Community Gives Money Binary} & 135 (0.4\%) & 118 (0.4\%) \\
Unknown & 669 & 6,959 \\
\textbf{Community Gives Time Binary} & 1,702 (5.2\%) & 1,669 (6.4\%) \\
Unknown & 669 & 6,959 \\
\textbf{Family Gives Money Binary} & 1,782 (5.5\%) & 1,236 (4.7\%) \\
Unknown & 669 & 6,959 \\
\textbf{Family Gives Time Binary} & 9,539 (29\%) & 7,600 (29\%) \\
Unknown & 669 & 6,959 \\
\textbf{Friends Give Money Binary} & 372 (1.1\%) & 270 (1.0\%) \\
Unknown & 669 & 6,959 \\
\textbf{Friends Give Time} & 5,765 (18\%) & 4,855 (19\%) \\
Unknown & 669 & 6,959 \\
\textbf{Sense Neighbourhood Community} & NA & NA \\
1 & 1,976 (6.0\%) & 1,124 (4.2\%) \\
2 & 4,037 (12\%) & 2,703 (10\%) \\
3 & 4,796 (15\%) & 3,561 (13\%) \\
4 & 6,840 (21\%) & 5,809 (22\%) \\
5 & 7,088 (21\%) & 6,477 (24\%) \\
6 & 5,753 (17\%) & 5,060 (19\%) \\
7 & 2,550 (7.7\%) & 2,104 (7.8\%) \\
Unknown & 158 & 6,360 \\
\textbf{Social Belonging} & 5.33 (4.33, 6.00) & 5.33 (4.33, 6.00) \\
Unknown & 268 & 6,418 \\
\textbf{Social Support} & 6.33 (5.33, 7.00) & 6.33 (5.33, 7.00) \\
Unknown & 19 & 6,286 \\
\textbf{Volunteering Hours} & 0.00 (0.00, 1.00) & 0.00 (0.00, 1.00) \\
Unknown & 875 & 6,863 \\
\textbf{Volunteers Binary} & 9,443 (29\%) & 6,881 (26\%) \\
Unknown & 875 & 6,863 \\

\end{longtable}

Table~\ref{tbl-table-outcomes} presents baseline and end-of-study
descriptive statistics for the outcome variables.

\newpage{}

\subsection*{References}\label{references}
\addcontentsline{toc}{subsection}{References}

\phantomsection\label{refs}
\begin{CSLReferences}{1}{0}
\bibitem[\citeproctext]{ref-angrist2009mostly}
Angrist, JD, and Pischke, J-S (2009) \emph{Mostly harmless econometrics:
An empiricist's companion}, Princeton university press.

\bibitem[\citeproctext]{ref-athey2019}
Athey, S, Tibshirani, J, and Wager, S (2019) Generalized random forests.
\emph{The Annals of Statistics}, \textbf{47}(2), 1148--1178.
doi:\href{https://doi.org/10.1214/18-AOS1709}{10.1214/18-AOS1709}.

\bibitem[\citeproctext]{ref-athey2021}
Athey, S, and Wager, S (2021) Policy Learning With Observational Data.
\emph{Econometrica}, \textbf{89}(1), 133--161.
doi:\href{https://doi.org/10.3982/ECTA15732}{10.3982/ECTA15732}.

\bibitem[\citeproctext]{ref-atkinson2019}
Atkinson, J, Salmond, C, and Crampton, P (2019) \emph{NZDep2018 index of
deprivation, user{'}s manual.}, Wellington.

\bibitem[\citeproctext]{ref-bekkers2011literature}
Bekkers, R, and Wiepking, P (2011) A literature review of empirical
studies of philanthropy: Eight mechanisms that drive charitable giving.
\emph{Nonprofit and Voluntary Sector Quarterly}, \textbf{40}(5),
924--973.

\bibitem[\citeproctext]{ref-brooks2004faith}
Brooks, AC (2004) Faith, secularism, and charity. \emph{Faith \&
Economics}, \textbf{43}(Spring), 1--8.

\bibitem[\citeproctext]{ref-bulbulia2009charismatic}
Bulbulia, J (2009) Charismatic signalling. \emph{Journal for the Study
of Religion, Nature \& Culture}, \textbf{3}(4).

\bibitem[\citeproctext]{ref-bulbulia2024PRACTICAL}
Bulbulia, J (2024a) A practical guide to causal inference in three-wave
panel studies. \emph{PsyArXiv Preprints}.
doi:\href{https://doi.org/10.31234/osf.io/uyg3d}{10.31234/osf.io/uyg3d}.

\bibitem[\citeproctext]{ref-bulbulia2022}
Bulbulia, JA (2022) A workflow for causal inference in cross-cultural
psychology. \emph{Religion, Brain \& Behavior}, \textbf{0}(0), 1--16.
doi:\href{https://doi.org/10.1080/2153599X.2022.2070245}{10.1080/2153599X.2022.2070245}.

\bibitem[\citeproctext]{ref-bulbulia2023}
Bulbulia, JA (2023) Causal diagrams (directed acyclic graphs): A
practical guide.

\bibitem[\citeproctext]{ref-margot2024}
Bulbulia, JA (2024b) \emph{Margot: MARGinal observational
treatment-effects}.
doi:\href{https://doi.org/10.5281/zenodo.10907724}{10.5281/zenodo.10907724}.

\bibitem[\citeproctext]{ref-bulbulia2023a}
Bulbulia, JA, Afzali, MU, Yogeeswaran, K, and Sibley, CG (2023)
Long-term causal effects of far-right terrorism in {N}ew {Z}ealand.
\emph{PNAS Nexus}, \textbf{2}(8), pgad242.

\bibitem[\citeproctext]{ref-Bulbulia_2015}
Bulbulia, JA, Shaver, JH, Greaves, L, Sosis, R, and Sibley, CG \& (2015)
Religion and parental cooperation: An empirical test of slone's sexual
signaling model. In S. J. D. amd Van Slyke J., ed., \emph{The attraction
of religion: A sexual selectionist account}, Bloomsbury Press, 29--62.

\bibitem[\citeproctext]{ref-bulbuliaj.2013}
Bulbulia, J., Geertz, AW, Atkinson, QD, \ldots{} Wilson, DS (2013) The
cultural evolution of religion. In P. J. Richerson and M. Christiansen,
eds., Cambridge, MA: MIT press, 381404.

\bibitem[\citeproctext]{ref-campbell2004}
Campbell, WK, Bonacci, AM, Shelton, J, Exline, JJ, and Bushman, BJ
(2004) Psychological entitlement: interpersonal consequences and
validation of a self-report measure. \emph{Journal of Personality
Assessment}, \textbf{83}(1), 29--45.
doi:\href{https://doi.org/10.1207/s15327752jpa8301_04}{10.1207/s15327752jpa8301\_04}.

\bibitem[\citeproctext]{ref-chatton2020}
Chatton, A, Le Borgne, F, Leyrat, C, \ldots{} Foucher, Y (2020)
G-computation, propensity score-based methods, and targeted maximum
likelihood estimator for causal inference with different covariates
sets: a comparative simulation study. \emph{Scientific Reports},
\textbf{10}(1), 9219.
doi:\href{https://doi.org/10.1038/s41598-020-65917-x}{10.1038/s41598-020-65917-x}.

\bibitem[\citeproctext]{ref-xgboost2023}
Chen, T, He, T, Benesty, M, \ldots{} Yuan, J (2023) \emph{Xgboost:
Extreme gradient boosting}. Retrieved from
\url{https://CRAN.R-project.org/package=xgboost}

\bibitem[\citeproctext]{ref-chen2020religious}
Chen, Y, Kim, ES, and VanderWeele, TJ (2020) Religious-service
attendance and subsequent health and well-being throughout adulthood:
Evidence from three prospective cohorts. \emph{International Journal of
Epidemiology}, \textbf{49}(6), 2030--2040.

\bibitem[\citeproctext]{ref-cristofori2016neural}
Cristofori, I, Bulbulia, J, Shaver, JH, Wilson, M, Krueger, F, and
Grafman, J (2016) Neural correlates of mystical experience.
\emph{Neuropsychologia}, \textbf{80}, 212--220.

\bibitem[\citeproctext]{ref-danaei2012}
Danaei, G, Tavakkoli, M, and Hernán, MA (2012) Bias in observational
studies of prevalent users: lessons for comparative effectiveness
research from a meta-analysis of statins. \emph{American Journal of
Epidemiology}, \textbf{175}(4), 250--262.
doi:\href{https://doi.org/10.1093/aje/kwr301}{10.1093/aje/kwr301}.

\bibitem[\citeproctext]{ref-duxedaz2021}
Díaz, I, Williams, N, Hoffman, KL, and Schenck, EJ (2021) Non-parametric
causal effects based on longitudinal modified treatment policies.
\emph{Journal of the American Statistical Association}.
doi:\href{https://doi.org/10.1080/01621459.2021.1955691}{10.1080/01621459.2021.1955691}.

\bibitem[\citeproctext]{ref-diaz2023lmtp}
Díaz, I, Williams, N, Hoffman, KL, and Schenck, EJ (2023) Nonparametric
causal effects based on longitudinal modified treatment policies.
\emph{Journal of the American Statistical Association},
\textbf{118}(542), 846--857.
doi:\href{https://doi.org/10.1080/01621459.2021.1955691}{10.1080/01621459.2021.1955691}.

\bibitem[\citeproctext]{ref-diaz2021nonparametric}
Dı́az, I, Hejazi, NS, Rudolph, KE, and Der Laan, MJ van (2021)
Nonparametric efficient causal mediation with intermediate confounders.
\emph{Biometrika}, \textbf{108}(3), 627--641.

\bibitem[\citeproctext]{ref-Diaz2023}
Dı́az, I, Williams, N, and Rudolph, KE (2023) \emph{Journal of Causal
Inference}, \textbf{11}(1), 20220077.
doi:\href{https://doi.org/doi:10.1515/jci-2022-0077}{doi:10.1515/jci-2022-0077}.

\bibitem[\citeproctext]{ref-fahy2017}
Fahy, KM, Lee, A, and Milne, BJ (2017b) \emph{New Zealand socio-economic
index 2013}, Wellington, New Zealand: Statistics New Zealand-Tatauranga
Aotearoa.

\bibitem[\citeproctext]{ref-fahy2017a}
Fahy, KM, Lee, A, and Milne, BJ (2017a) \emph{New Zealand socio-economic
index 2013}, Wellington, New Zealand: Statistics New Zealand-Tatauranga
Aotearoa.

\bibitem[\citeproctext]{ref-fraser_coding_2020}
Fraser, G, Bulbulia, J, Greaves, LM, Wilson, MS, and Sibley, CG (2020)
Coding responses to an open-ended gender measure in a {N}ew {Z}ealand
national sample. \emph{The Journal of Sex Research}, \textbf{57}(8),
979--986.
doi:\href{https://doi.org/10.1080/00224499.2019.1687640}{10.1080/00224499.2019.1687640}.

\bibitem[\citeproctext]{ref-haneuse2013estimation}
Haneuse, S, and Rotnitzky, A (2013) Estimation of the effect of
interventions that modify the received treatment. \emph{Statistics in
Medicine}, \textbf{32}(30), 5260--5277.

\bibitem[\citeproctext]{ref-hernan2024WHATIF}
Hernan, MA, and Robins, JM (2024) \emph{Causal inference: What if?},
Taylor \& Francis. Retrieved from
\url{https://www.hsph.harvard.edu/miguel-hernan/causal-inference-book/}

\bibitem[\citeproctext]{ref-hernan2024stating}
Hernán, MA, and Greenland, S (2024) Why stating hypotheses in grant
applications is unnecessary. \emph{JAMA}, \textbf{331}(4), 285--286.

\bibitem[\citeproctext]{ref-hoffman2023}
Hoffman, KL, Salazar-Barreto, D, Rudolph, KE, and Díaz, I (2023)
Introducing longitudinal modified treatment policies: A unified
framework for studying complex exposures.
doi:\href{https://doi.org/10.48550/arXiv.2304.09460}{10.48550/arXiv.2304.09460}.

\bibitem[\citeproctext]{ref-hoffman2022}
Hoffman, KL, Schenck, EJ, Satlin, MJ, \ldots{} Díaz, I (2022) Comparison
of a target trial emulation framework vs cox regression to estimate the
association of corticosteroids with COVID-19 mortality. \emph{JAMA
Network Open}, \textbf{5}(10), e2234425.
doi:\href{https://doi.org/10.1001/jamanetworkopen.2022.34425}{10.1001/jamanetworkopen.2022.34425}.

\bibitem[\citeproctext]{ref-hoverd_religious_2010}
Hoverd, WJ, and Sibley, CG (2010) Religious and denominational diversity
in new zealand 2009. \emph{New Zealand Sociology}, \textbf{25}(2),
59--87.

\bibitem[\citeproctext]{ref-huitfeldt2019collapsibility}
Huitfeldt, A, Stensrud, MJ, and Suzuki, E (2019) On the collapsibility
of measures of effect in the counterfactual causal framework.
\emph{Emerging Themes in Epidemiology}, \textbf{16}, 1--5.

\bibitem[\citeproctext]{ref-johnson2022global}
Johnson, BR, and VanderWeele, TJ (2022) The global flourishing study: A
new era for the study of well-being. \emph{International Bulletin of
Mission Research}, \textbf{46}(2), 272--275.

\bibitem[\citeproctext]{ref-johnson2005}
Johnson, DD (2005) God{'}s punishment and public goods: A test of the
supernatural punishment hypothesis in 186 world cultures. \emph{Human
Nature}, \textbf{16}, 410446.

\bibitem[\citeproctext]{ref-johnson2015}
Johnson, DD (2015) Big gods, small wonder: Supernatural punishment
strikes back. \emph{Religion, Brain \& Behavior}, \textbf{5}(4), 290298.

\bibitem[\citeproctext]{ref-jost_end_2006-1}
Jost, JT (2006) The end of the end of ideology. \emph{American
Psychologist}, \textbf{61}(7), 651--670.
doi:\href{https://doi.org/10.1037/0003-066X.61.7.651}{10.1037/0003-066X.61.7.651}.

\bibitem[\citeproctext]{ref-kelly2024religiosity}
Kelly, JM, Kramer, SR, and Shariff, AF (2024) Religiosity predicts
prosociality, especially when measured by self-report: A meta-analysis
of almost 60 years of research. \emph{Psychological Bulletin},
\textbf{150}(3), 284--318.

\bibitem[\citeproctext]{ref-khanna1995charity}
Khanna, J, Posnett, J, and Sandler, T (1995) Charity donations in the
UK: New evidence based on panel data. \emph{Journal of Public
Economics}, \textbf{56}(2), 257--272.

\bibitem[\citeproctext]{ref-konvalinka2011synchronized}
Konvalinka, I, Xygalatas, D, Bulbulia, J, \ldots{} Roepstorff, A (2011)
Synchronized arousal between performers and related spectators in a
fire-walking ritual. \emph{Proceedings of the National Academy of
Sciences}, \textbf{108}(20), 8514--8519.

\bibitem[\citeproctext]{ref-van2012targeted}
Laan, MJ van der, and Gruber, S (2012) Targeted minimum loss based
estimation of causal effects of multiple time point interventions.
\emph{The International Journal of Biostatistics}, \textbf{8}(1).

\bibitem[\citeproctext]{ref-van2014discussion}
Laan, MJ van der, Luedtke, AR, and Dı́az, I (2014) Discussion of
identification, estimation and approximation of risk under interventions
that depend on the natural value of treatment using observational data,
by {J}essica {Y}oung, {M}iguel {H}ern{á}n, and {J}ames {R}obins.
\emph{Epidemiologic Methods}, \textbf{3}(1), 21--31.

\bibitem[\citeproctext]{ref-lang2015effects}
Lang, M, Krátkỳ, J, Shaver, JH, Jerotijević, D, and Xygalatas, D (2015)
Effects of anxiety on spontaneous ritualized behavior. \emph{Current
Biology}, \textbf{25}(14), 1892--1897.

\bibitem[\citeproctext]{ref-lash2020}
Lash, TL, Rothman, KJ, VanderWeele, TJ, and Haneuse, S (2020)
\emph{Modern epidemiology}, Wolters Kluwer. Retrieved from
\url{https://books.google.co.nz/books?id=SiTSnQEACAAJ}

\bibitem[\citeproctext]{ref-linden2020EVALUE}
Linden, A, Mathur, MB, and VanderWeele, TJ (2020) Conducting sensitivity
analysis for unmeasured confounding in observational studies using
e-values: The evalue package. \emph{The Stata Journal}, \textbf{20}(1),
162--175.

\bibitem[\citeproctext]{ref-major2023exploring}
Major-Smith, D (2023) Exploring causality from observational data: An
example assessing whether religiosity promotes cooperation.
\emph{Evolutionary Human Sciences}, \textbf{5}, e22.

\bibitem[\citeproctext]{ref-mccullough2020kindness}
McCullough, ME (2020) \emph{The kindness of strangers: How a selfish ape
invented a new moral code}, Simon; Schuster.

\bibitem[\citeproctext]{ref-mccullough2016christian}
McCullough, ME, Swartwout, P, Shaver, JH, Carter, EC, and Sosis, R
(2016) Christian religious badges instill trust in christian and
non-christian perceivers. \emph{Psychology of Religion and
Spirituality}, \textbf{8}(2), 149.

\bibitem[\citeproctext]{ref-mcelreath2020}
McElreath, R (2020) \emph{Statistical rethinking: A {B}ayesian course
with examples in r and stan}, CRC press.

\bibitem[\citeproctext]{ref-McLeod2020}
McLeod, J (2020) The new zealand support report: The current state and
significance of giving in new zealand and the outlook for recipients.,
JBWere. Retrieved from
\url{https://www.jbwere.co.nz/media/1qudxw3q/jbwere-nz-support-report-digital.pdf}

\bibitem[\citeproctext]{ref-monsma2007religion}
Monsma, SV (2007) Religion and philanthropic giving and volunteering:
Building blocks for civic responsibility. \emph{Interdisciplinary
Journal of Research on Religion}, \textbf{3}.

\bibitem[\citeproctext]{ref-montgomery2018}
Montgomery, JM, Nyhan, B, and Torres, M (2018) How conditioning on
posttreatment variables can ruin your experiment and what to do about
It. \emph{American Journal of Political Science}, \textbf{62}(3),
760--775.
doi:\href{https://doi.org/10.1111/ajps.12357}{10.1111/ajps.12357}.

\bibitem[\citeproctext]{ref-diaz2012population}
Muñoz, ID, and Van Der Laan, M (2012) Population intervention causal
effects based on stochastic interventions. \emph{Biometrics},
\textbf{68}(2), 541--549.

\bibitem[\citeproctext]{ref-neal2020introduction}
Neal, B (2020) Introduction to causal inference from a machine learning
perspective. \emph{Course Lecture Notes (Draft)}. Retrieved from
\url{https://www.bradyneal.com/Introduction_to_Causal_Inference-Dec17_2020-Neal.pdf}

\bibitem[\citeproctext]{ref-norenzayan2016}
Norenzayan, A, Shariff, AF, Gervais, WM, \ldots{} Henrich, J (2016) The
cultural evolution of prosocial religions. \emph{Behavioral and Brain
Sciences}, \textbf{39}, e1.
doi:\href{https://doi.org/10.1017/S0140525X14001356}{10.1017/S0140525X14001356}.

\bibitem[\citeproctext]{ref-ogburn2021}
Ogburn, EL, and Shpitser, I (2021) Causal modelling: The two cultures.
\emph{Observational Studies}, \textbf{7}(1), 179--183.
doi:\href{https://doi.org/10.1353/obs.2021.0006}{10.1353/obs.2021.0006}.

\bibitem[\citeproctext]{ref-pearl2009}
Pearl, J (2009) \emph{\href{https://doi.org/10.1214/09-SS057}{Causal
inference in statistics: An overview}}.

\bibitem[\citeproctext]{ref-polley2023}
Polley, E, LeDell, E, Kennedy, C, and Laan, M van der (2023)
\emph{SuperLearner: Super learner prediction}. Retrieved from
\url{https://CRAN.R-project.org/package=SuperLearner}

\bibitem[\citeproctext]{ref-richardson2023nested}
Richardson, TS, Evans, RJ, Robins, JM, and Shpitser, I (2023) Nested
{M}arkov properties for acyclic directed mixed graphs. \emph{The Annals
of Statistics}, \textbf{51}(1), 334--361.

\bibitem[\citeproctext]{ref-richardson2013swigsprimer}
Richardson, TS, and Robins, JM (2013) Single world intervention graphs:
A primer. In \emph{Second UAI workshop on causal structure learning,
{B}ellevue, {W}ashington}, Citeseer. Retrieved from
\url{https://citeseerx.ist.psu.edu/document?repid=rep1&type=pdf&doi=07bbcb458109d2663acc0d098e8913892389a2a7}

\bibitem[\citeproctext]{ref-richardson2023potential}
Richardson, TS, and Robins, JM (2023) Potential outcome and decision
theoretic foundations for statistical causality. \emph{Journal of Causal
Inference}, \textbf{11}(1), 20220012.

\bibitem[\citeproctext]{ref-robins1986}
Robins, J (1986) A new approach to causal inference in mortality studies
with a sustained exposure period---application to control of the healthy
worker survivor effect. \emph{Mathematical Modelling}, \textbf{7}(9-12),
1393--1512.

\bibitem[\citeproctext]{ref-robins1992}
Robins, JM, and Greenland, S (1992) Identifiability and exchangeability
for direct and indirect effects. \emph{Epidemiology}, \textbf{3}(2),
143155.

\bibitem[\citeproctext]{ref-robins2010alternative}
Robins, JM, and Richardson, TS (2010) Alternative graphical causal
models and the identification of direct effects. \emph{Causality and
Psychopathology: Finding the Determinants of Disorders and Their Cures},
\textbf{84}, 103--158.

\bibitem[\citeproctext]{ref-rubin2005}
Rubin, DB (2005) Causal inference using potential outcomes: Design,
modeling, decisions. \emph{Journal of the American Statistical
Association}, \textbf{100}(469), 322--331. Retrieved from
\url{https://www.jstor.org/stable/27590541}

\bibitem[\citeproctext]{ref-schloss2011evolutionary}
Schloss, JP, and Murray, MJ (2011) Evolutionary accounts of belief in
supernatural punishment: A critical review. \emph{Religion, Brain \&
Behavior}, \textbf{1}(1), 46--99.

\bibitem[\citeproctext]{ref-shaver2015evolution}
Shaver, JH (2015) The evolution of stratification in fijian ritual
participation. \emph{Religion, Brain \& Behavior}, \textbf{5}(2),
101--117.

\bibitem[\citeproctext]{ref-shaver2020church}
Shaver, JH, Power, EA, Purzycki, BG, \ldots{} Bulbulia, JA (2020) Church
attendance and alloparenting: An analysis of fertility, social support
and child development among {E}nglish mothers. \emph{Philosophical
Transactions of the Royal Society B}, \textbf{375}(1805), 20190428.

\bibitem[\citeproctext]{ref-shpitser2022multivariate}
Shpitser, I, Richardson, TS, and Robins, JM (2022) Multivariate
counterfactual systems and causal graphical models. In
\emph{Probabilistic and causal inference: The works of {J}udea {P}earl},
813--852.

\bibitem[\citeproctext]{ref-shpitser2016causal}
Shpitser, I, and Tchetgen, ET (2016) Causal inference with a graphical
hierarchy of interventions. \emph{Annals of Statistics}, \textbf{44}(6),
2433.

\bibitem[\citeproctext]{ref-sibley2012}
Sibley, \&B, C. G. (2012) Healing those who need healing: How religious
practice affects social belonging. \emph{Journal for the Cognitive
Science of Religion}, \textbf{1}, 29--45.

\bibitem[\citeproctext]{ref-sibley2021}
Sibley, CG (2021)
\emph{\href{https://doi.org/10.31234/osf.io/wgqvy}{Sampling procedure
and sample details for the new zealand attitudes and values study}}.

\bibitem[\citeproctext]{ref-sibley2012a}
Sibley, CG, and Bulbulia, J (2012) Faith after an earthquake: A
longitudinal study of religion and perceived health before and after the
2011 {C}hristchurch {N}ew {Z}ealand earthquake. \emph{PloS One},
\textbf{7}(12), e49648.

\bibitem[\citeproctext]{ref-sibley2011}
Sibley, CG, Luyten, N, Purnomo, M, \ldots{} Robertson, A (2011) The
Mini-IPIP6: Validation and extension of a short measure of the Big-Six
factors of personality in New Zealand. \emph{New Zealand Journal of
Psychology}, \textbf{40}(3), 142--159.

\bibitem[\citeproctext]{ref-sosis2003cooperation}
Sosis, R, and Bressler, ER (2003) Cooperation and commune longevity: A
test of the costly signaling theory of religion. \emph{Cross-Cultural
Research}, \textbf{37}(2), 211--239.

\bibitem[\citeproctext]{ref-neyman1923}
Splawa-Neyman, J, Dabrowska, DM, and Speed, TP (1990) On the application
of probability theory to agricultural experiments. Essay on principles.
Section 9. \emph{Statistical Science}, 465--472.

\bibitem[\citeproctext]{ref-vanbuuren2018}
Van Buuren, S (2018) \emph{Flexible imputation of missing data}, CRC
press.

\bibitem[\citeproctext]{ref-van2014targeted}
Van der Laan, MJ (2014) Targeted estimation of nuisance parameters to
obtain valid statistical inference. \emph{The International Journal of
Biostatistics}, \textbf{10}(1), 29--57.

\bibitem[\citeproctext]{ref-vanderlaan2011}
Van Der Laan, MJ, and Rose, S (2011) \emph{Targeted Learning: Causal
Inference for Observational and Experimental Data}, New York, NY:
Springer. Retrieved from
\url{https://link.springer.com/10.1007/978-1-4419-9782-1}

\bibitem[\citeproctext]{ref-vanderlaan2018}
Van Der Laan, MJ, and Rose, S (2018) \emph{Targeted Learning in Data
Science: Causal Inference for Complex Longitudinal Studies}, Cham:
Springer International Publishing. Retrieved from
\url{http://link.springer.com/10.1007/978-3-319-65304-4}

\bibitem[\citeproctext]{ref-vantongeren2020}
Van Tongeren, DR, DeWall, CN, Chen, Z, Sibley, CG, and Bulbulia, J
(2020) Religious residue: Cross-cultural evidence that religious
psychology and behavior persist following deidentification.
\emph{Journal of Personality and Social Psychology}.

\bibitem[\citeproctext]{ref-vanderweele2009}
VanderWeele, TJ (2009) Concerning the consistency assumption in causal
inference. \emph{Epidemiology}, \textbf{20}(6), 880.
doi:\href{https://doi.org/10.1097/EDE.0b013e3181bd5638}{10.1097/EDE.0b013e3181bd5638}.

\bibitem[\citeproctext]{ref-vanderweele2015}
VanderWeele, TJ (2015) \emph{Explanation in causal inference: Methods
for mediation and interaction}, Oxford University Press.

\bibitem[\citeproctext]{ref-vanderweele2019}
VanderWeele, TJ (2019) Principles of confounder selection.
\emph{European Journal of Epidemiology}, \textbf{34}(3), 211219.

\bibitem[\citeproctext]{ref-vanderweele2017}
VanderWeele, TJ, and Ding, P (2017) Sensitivity analysis in
observational research: Introducing the e-value. \emph{Annals of
Internal Medicine}, \textbf{167}(4), 268--274.
doi:\href{https://doi.org/10.7326/M16-2607}{10.7326/M16-2607}.

\bibitem[\citeproctext]{ref-vanderweele2013}
VanderWeele, TJ, and Hernan, MA (2013) Causal inference under multiple
versions of treatment. \emph{Journal of Causal Inference},
\textbf{1}(1), 120.

\bibitem[\citeproctext]{ref-vanderweele2012MEASUREMENT}
VanderWeele, TJ, and Hernán, MA (2012) Results on differential and
dependent measurement error of the exposure and the outcome using signed
directed acyclic graphs. \emph{American Journal of Epidemiology},
\textbf{175}(12), 1303--1310.
doi:\href{https://doi.org/10.1093/aje/kwr458}{10.1093/aje/kwr458}.

\bibitem[\citeproctext]{ref-vanderweele2020}
VanderWeele, TJ, Mathur, MB, and Chen, Y (2020) Outcome-wide
longitudinal designs for causal inference: A new template for empirical
studies. \emph{Statistical Science}, \textbf{35}(3), 437--466.

\bibitem[\citeproctext]{ref-verbrugge1997}
Verbrugge, LM (1997) A global disability indicator. \emph{Journal of
Aging Studies}, \textbf{11}(4), 337--362.
doi:\href{https://doi.org/10.1016/S0890-4065(97)90026-8}{10.1016/S0890-4065(97)90026-8}.

\bibitem[\citeproctext]{ref-watts2016}
Watts, J, Bulbulia, J. A., Gray, RD, and Atkinson, QD (2016) Clarity and
causality needed in claims about big gods., \textbf{39}, 41--42.
doi:\href{https://doi.org/DOI:10.1017/S0140525X15000576}{DOI:10.1017/S0140525X15000576}.

\bibitem[\citeproctext]{ref-watts2015}
Watts, J, Greenhill, SJ, Atkinson, QD, Currie, TE, Bulbulia, J, and
Gray, RD (2015) \emph{Broad supernatural punishment but not moralizing
high gods precede the evolution of political complexity in
{A}ustronesia} \emph{Proceedings of the Royal Society B: Biological
Sciences}, Vol. 282, The Royal Society, 20142556.

\bibitem[\citeproctext]{ref-westreich2010}
Westreich, D, and Cole, SR (2010) Invited commentary: positivity in
practice. \emph{American Journal of Epidemiology}, \textbf{171}(6).
doi:\href{https://doi.org/10.1093/aje/kwp436}{10.1093/aje/kwp436}.

\bibitem[\citeproctext]{ref-westreich2013}
Westreich, D, and Greenland, S (2013) The table 2 fallacy: Presenting
and interpreting confounder and modifier coefficients. \emph{American
Journal of Epidemiology}, \textbf{177}(4), 292298.

\bibitem[\citeproctext]{ref-whitehouse2023}
Whitehouse, H, François, P, Savage, PE, \ldots{} Haar, B ter (2023)
Testing the big gods hypothesis with global historical data: A review
and {``}retake{''}. \emph{Religion, Brain \& Behavior}, \textbf{13}(2),
124166.

\bibitem[\citeproctext]{ref-williams2021}
Williams, NT, and Díaz, I (2021) \emph{{l}mtp: Non-parametric causal
effects of feasible interventions based on modified treatment policies}.
doi:\href{https://doi.org/10.5281/zenodo.3874931}{10.5281/zenodo.3874931}.

\bibitem[\citeproctext]{ref-woodyard2014doing}
Woodyard, A, and Grable, J (2014) Doing good and feeling well: Exploring
the relationship between charitable activity and perceived personal
wellness. \emph{VOLUNTAS: International Journal of Voluntary and
Nonprofit Organizations}, \textbf{25}, 905--928.

\bibitem[\citeproctext]{ref-Ranger2017}
Wright, MN, and Ziegler, A (2017) {ranger}: A fast implementation of
random forests for high dimensional data in {C++} and {R}. \emph{Journal
of Statistical Software}, \textbf{77}(1), 1--17.
doi:\href{https://doi.org/10.18637/jss.v077.i01}{10.18637/jss.v077.i01}.

\bibitem[\citeproctext]{ref-young2014identification}
Young, JG, Hernán, MA, and Robins, JM (2014) Identification, estimation
and approximation of risk under interventions that depend on the natural
value of treatment using observational data. \emph{Epidemiologic
Methods}, \textbf{3}(1), 1--19.

\end{CSLReferences}



\end{document}
