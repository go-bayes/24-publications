% Options for packages loaded elsewhere
\PassOptionsToPackage{unicode}{hyperref}
\PassOptionsToPackage{hyphens}{url}
\PassOptionsToPackage{dvipsnames,svgnames,x11names}{xcolor}
%
\documentclass[
  single column]{article}

\usepackage{amsmath,amssymb}
\usepackage{iftex}
\ifPDFTeX
  \usepackage[T1]{fontenc}
  \usepackage[utf8]{inputenc}
  \usepackage{textcomp} % provide euro and other symbols
\else % if luatex or xetex
  \usepackage{unicode-math}
  \defaultfontfeatures{Scale=MatchLowercase}
  \defaultfontfeatures[\rmfamily]{Ligatures=TeX,Scale=1}
\fi
\usepackage[]{libertinus}
\ifPDFTeX\else  
    % xetex/luatex font selection
\fi
% Use upquote if available, for straight quotes in verbatim environments
\IfFileExists{upquote.sty}{\usepackage{upquote}}{}
\IfFileExists{microtype.sty}{% use microtype if available
  \usepackage[]{microtype}
  \UseMicrotypeSet[protrusion]{basicmath} % disable protrusion for tt fonts
}{}
\makeatletter
\@ifundefined{KOMAClassName}{% if non-KOMA class
  \IfFileExists{parskip.sty}{%
    \usepackage{parskip}
  }{% else
    \setlength{\parindent}{0pt}
    \setlength{\parskip}{6pt plus 2pt minus 1pt}}
}{% if KOMA class
  \KOMAoptions{parskip=half}}
\makeatother
\usepackage{xcolor}
\usepackage[top=30mm,left=25mm,heightrounded,headsep=22pt,headheight=11pt,footskip=33pt,ignorehead,ignorefoot]{geometry}
\setlength{\emergencystretch}{3em} % prevent overfull lines
\setcounter{secnumdepth}{-\maxdimen} % remove section numbering
% Make \paragraph and \subparagraph free-standing
\makeatletter
\ifx\paragraph\undefined\else
  \let\oldparagraph\paragraph
  \renewcommand{\paragraph}{
    \@ifstar
      \xxxParagraphStar
      \xxxParagraphNoStar
  }
  \newcommand{\xxxParagraphStar}[1]{\oldparagraph*{#1}\mbox{}}
  \newcommand{\xxxParagraphNoStar}[1]{\oldparagraph{#1}\mbox{}}
\fi
\ifx\subparagraph\undefined\else
  \let\oldsubparagraph\subparagraph
  \renewcommand{\subparagraph}{
    \@ifstar
      \xxxSubParagraphStar
      \xxxSubParagraphNoStar
  }
  \newcommand{\xxxSubParagraphStar}[1]{\oldsubparagraph*{#1}\mbox{}}
  \newcommand{\xxxSubParagraphNoStar}[1]{\oldsubparagraph{#1}\mbox{}}
\fi
\makeatother


\providecommand{\tightlist}{%
  \setlength{\itemsep}{0pt}\setlength{\parskip}{0pt}}\usepackage{longtable,booktabs,array}
\usepackage{calc} % for calculating minipage widths
% Correct order of tables after \paragraph or \subparagraph
\usepackage{etoolbox}
\makeatletter
\patchcmd\longtable{\par}{\if@noskipsec\mbox{}\fi\par}{}{}
\makeatother
% Allow footnotes in longtable head/foot
\IfFileExists{footnotehyper.sty}{\usepackage{footnotehyper}}{\usepackage{footnote}}
\makesavenoteenv{longtable}
\usepackage{graphicx}
\makeatletter
\def\maxwidth{\ifdim\Gin@nat@width>\linewidth\linewidth\else\Gin@nat@width\fi}
\def\maxheight{\ifdim\Gin@nat@height>\textheight\textheight\else\Gin@nat@height\fi}
\makeatother
% Scale images if necessary, so that they will not overflow the page
% margins by default, and it is still possible to overwrite the defaults
% using explicit options in \includegraphics[width, height, ...]{}
\setkeys{Gin}{width=\maxwidth,height=\maxheight,keepaspectratio}
% Set default figure placement to htbp
\makeatletter
\def\fps@figure{htbp}
\makeatother
% definitions for citeproc citations
\NewDocumentCommand\citeproctext{}{}
\NewDocumentCommand\citeproc{mm}{%
  \begingroup\def\citeproctext{#2}\cite{#1}\endgroup}
\makeatletter
 % allow citations to break across lines
 \let\@cite@ofmt\@firstofone
 % avoid brackets around text for \cite:
 \def\@biblabel#1{}
 \def\@cite#1#2{{#1\if@tempswa , #2\fi}}
\makeatother
\newlength{\cslhangindent}
\setlength{\cslhangindent}{1.5em}
\newlength{\csllabelwidth}
\setlength{\csllabelwidth}{3em}
\newenvironment{CSLReferences}[2] % #1 hanging-indent, #2 entry-spacing
 {\begin{list}{}{%
  \setlength{\itemindent}{0pt}
  \setlength{\leftmargin}{0pt}
  \setlength{\parsep}{0pt}
  % turn on hanging indent if param 1 is 1
  \ifodd #1
   \setlength{\leftmargin}{\cslhangindent}
   \setlength{\itemindent}{-1\cslhangindent}
  \fi
  % set entry spacing
  \setlength{\itemsep}{#2\baselineskip}}}
 {\end{list}}
\usepackage{calc}
\newcommand{\CSLBlock}[1]{\hfill\break\parbox[t]{\linewidth}{\strut\ignorespaces#1\strut}}
\newcommand{\CSLLeftMargin}[1]{\parbox[t]{\csllabelwidth}{\strut#1\strut}}
\newcommand{\CSLRightInline}[1]{\parbox[t]{\linewidth - \csllabelwidth}{\strut#1\strut}}
\newcommand{\CSLIndent}[1]{\hspace{\cslhangindent}#1}

\input{/Users/joseph/GIT/latex/latex-for-quarto.tex}
\makeatletter
\@ifpackageloaded{caption}{}{\usepackage{caption}}
\AtBeginDocument{%
\ifdefined\contentsname
  \renewcommand*\contentsname{Table of contents}
\else
  \newcommand\contentsname{Table of contents}
\fi
\ifdefined\listfigurename
  \renewcommand*\listfigurename{List of Figures}
\else
  \newcommand\listfigurename{List of Figures}
\fi
\ifdefined\listtablename
  \renewcommand*\listtablename{List of Tables}
\else
  \newcommand\listtablename{List of Tables}
\fi
\ifdefined\figurename
  \renewcommand*\figurename{Figure}
\else
  \newcommand\figurename{Figure}
\fi
\ifdefined\tablename
  \renewcommand*\tablename{Table}
\else
  \newcommand\tablename{Table}
\fi
}
\@ifpackageloaded{float}{}{\usepackage{float}}
\floatstyle{ruled}
\@ifundefined{c@chapter}{\newfloat{codelisting}{h}{lop}}{\newfloat{codelisting}{h}{lop}[chapter]}
\floatname{codelisting}{Listing}
\newcommand*\listoflistings{\listof{codelisting}{List of Listings}}
\makeatother
\makeatletter
\makeatother
\makeatletter
\@ifpackageloaded{caption}{}{\usepackage{caption}}
\@ifpackageloaded{subcaption}{}{\usepackage{subcaption}}
\makeatother
\ifLuaTeX
  \usepackage{selnolig}  % disable illegal ligatures
\fi
\usepackage{bookmark}

\IfFileExists{xurl.sty}{\usepackage{xurl}}{} % add URL line breaks if available
\urlstyle{same} % disable monospaced font for URLs
\hypersetup{
  pdftitle={Supplementary files for ``Methods in Causal Inference Part 2: Interaction, Mediation, and Time-Varying Treatments''},
  pdfauthor={Joseph A. Bulbulia},
  colorlinks=true,
  linkcolor={blue},
  filecolor={Maroon},
  citecolor={Blue},
  urlcolor={Blue},
  pdfcreator={LaTeX via pandoc}}

\title{Supplementary files for ``Methods in Causal Inference Part 2:
Interaction, Mediation, and Time-Varying Treatments''}

\usepackage{academicons}
\usepackage{xcolor}

  \author{Joseph A. Bulbulia}
            \affil{%
             \small{     Victoria University of Wellington, New Zealand
          ORCID \textcolor[HTML]{A6CE39}{\aiOrcid} ~0000-0002-5861-2056 }
              }
      


\date{2024-06-19}
\begin{document}
\maketitle

\renewcommand*\contentsname{Table of contents}
{
\hypersetup{linkcolor=}
\setcounter{tocdepth}{2}
\tableofcontents
}
\listoftables
\newpage{}

\subsection{S1. Glossary}\label{id-app-a}

\begin{table}

\caption{\label{tbl-experiments}Glossary}

\centering{

\glossaryTerms

}

\end{table}%

\newpage{}

\subsection{S2. Single World Intervention Graphs Eludicate Complex
Identification Problems}\label{id-app-b}

\begin{table}

\caption{\label{tbl-pearltable}On the limitations of causal directed
acyclic graphs compared to Single World Intervention Graphs.}

\centering{

\pearltable

}

\end{table}%

According to Pearl (\citeproc{ref-pearl2009a}{2009}), example 11.3.3.
\(Y(x_0, x_1)\) is not independent of \(X_1\) given Z and \(X_0\).
Template \(\mathcal{G}_2\) and Single World Intervention Graphs
\(\mathcal{G}_2-6\) examine counterfactual independence. Counterfactual
nodes are obtained by node-splitting. \(\rightarrowlightgray\) denotes a
backdoor path that is closed when a treatment is fixed.
\(\rightarrowcyan\) highlights identifying paths for \(X_0 = x_0\) and
\(X_1 = x_1\). J. M. Robins \& Richardson
(\citeproc{ref-robins2010alternative}{2010}) uses a variation of
\(\mathcal{G}_2\) to show that there is sequential exchangeability of
\(X_t \forall t: Y(x_1, x_0)\coprod X_0\) (unconditionally) and
\(Y(x_1, x_0)\coprod X_1(x_0) | Z(x_0), X_0\). By causal consistency
\(Z(x_0) = Z|X_0 = x_0\). Single world interventions
\(\mathcal{G}_{3-6}\) make this sequential exchangeability clear (refer
to Richardson \& Robins (\citeproc{ref-richardson2013}{2013})). Such
clarity is, in my view, an excellent reason to use Single World
Intervention Graphs.

\newpage{}

\subsection{S3. Richardson and Robin's Extended Dynamic
G-formula}\label{id-app-c}

Richardson \& Robins (\citeproc{ref-richardson2013}{2013}) propose an
extension of J. Robins (\citeproc{ref-robins1986}{1986})'s dynamic
g-formula for identifying causality under dynamic treatment regimes.
Here, I reproduce the key details of their identification algorithm. I
refer readers to Richardson \& Robins
(\citeproc{ref-richardson2013}{2013}) for the full algorithm and its
proofs.

First, define the set of counterfactual variables in our dynamic Single
World Intervention Graph (or Template):

\(\mathbb{A}^+(\mathbf{g})\): denotes the set of modified treatment
variables under a dynamic regime \(\mathbb{V}(\mathbf{g})\): denotes the
set of counterfactual nodes following treatments.
\(\mathbb{W}(\mathbf{g})\): denotes the combined set of all
counterfactual variables under a dynamic regime corresponding to Table
12 \(\mathbf{G}_4\) in the main article.

Richardson \& Robins (\citeproc{ref-richardson2013}{2013}) define this
set as follows:

\[
\mathbb{W}(\mathbf{g}) \equiv \mathbb{A}^+(\mathbf{g}) \cup \mathbb{V}(\mathbf{g})
\]

Next, at each intervention node \(t\), our task is to find all ancestors
of \(Y(\mathbf{g})\) in \(\mathbb{W}(\mathbf{g})\) that are not in the
set of current or past treatment covariates. Richardson \& Robins
(\citeproc{ref-richardson2013}{2013}) define this set as follows:

\[
\mathbb{Z}_t(\mathbf{g}) \equiv \text{an}_{\mathcal{G}(\mathbf{g})}(Y(\mathbf{g})) \setminus (\mathbb{L}_t(\mathbf{g}) \cup \mathbb{A}_t(\mathbf{g}) \cup \mathbb{A}^+(\mathbf{g}))
\]

Third, our task is to map \(\mathbb{Z}\) to a new Single World
Intervention Graph \(\mathcal{G}(\mathbf{a}^*)\), where the intervention
\(\mathbf{a}^*\) is a specific value of \(A = a\) assigned under
\(f^g(\cdot)\).

This new \texttt{dSWIG} or
\texttt{dynamic\ Single\ World\ Intervention\ Graph}
\(\mathcal{G}(\mathbf{a}^*)\) is simply the original dSWIG
\(\mathcal{G}(\mathbf{g})\) with the dashed arrows removed.

Fourth, our task is to ensure conditional independence of the treatment
\(A_t = a^*\) with members of the set \(\mathbb{Z}_t\), for all
\(\mathbf{a}^*\) (fixed nodes) and all time points \(t \in 1...\tau\),
where \(\tau\) is the end of the study. Richardson \& Robins
(\citeproc{ref-richardson2013}{2013}) define this set as follows:

\[
\mathbb{Z}_t(\mathbf{a}^*) \coprod I(A_t(\mathbf{a}^*) = a^*_t) \mid \bar{\mathbb{L}}_t(\mathbf{a}^*), \bar{\mathbb{A}}_{t-1}(\mathbf{a}^*) = \bar{\mathbf{a}^*}_{t-1}
\]

where the authors use \(I\) to denote the indicator function:

\[
I(A_k(\mathbf{a}^*) = a^*_t) = 
\begin{cases} 
1 & \text{if } A_k(\mathbf{a}^*) = a^*_t, \\
0 & \text{otherwise}.
\end{cases}
\]

where:

\begin{itemize}
\tightlist
\item
  \textbf{\(\mathbb{Z}_t(\mathbf{a}^*)\)}: denotes the subset of
  vertices in \(\mathcal{G}(\mathbf{a}^*)\) corresponding to
  \(\mathbb{Z}_t(\mathbf{g})\).
\item
  \textbf{\(A_t(\mathbf{a}^*) = a^*_t\)}: denotes the specific value of
  the treatment variable at time \(t\) under the intervention
  \(\mathbf{a}^*\).
\item
  \textbf{\(\bar{\mathbb{L}}_t(\mathbf{a}^*)\)}: denotes the set of
  covariates up to time \(t\) under the intervention \(\mathbf{a}^*\).
\item
  \textbf{\(\bar{\mathbb{A}}_{t-1}(\mathbf{a}^*)\)}: denotes the set of
  past treatment variables up to time \(t-1\) under the intervention
  \(\mathbf{a}^*\).
\end{itemize}

In the example describe in the main article Table 12 \(\mathcal{G}_4\),
\(\mathbb{Z}_t(\mathbf{g})\), we apply Richardson \& Robins
(\citeproc{ref-richardson2013}{2013})'s dynamic extended g-formula, and
obtain:

\[
\begin{aligned}
\mathbb{Z}(\mathbf{g}) &= \{A_1, L_1(\mathbf{g}), A_1(\mathbf{g}), A_2(\mathbf{g}), Y(\mathbf{g})\} \\
\mathbb{Z}_1(\mathbf{g}) &= \{A_1(\mathbf{g}), L_1(\mathbf{g}), Y(\mathbf{g})\} \\
\mathbb{Z}_2(\mathbf{g}) &= \{Y(\mathbf{g})\}
\end{aligned}
\]

Then we check conditional independencies for each treatment in the main
manuscript Table 12 \(\mathcal{G}_4\) (which is Table 12
\(\mathcal{G}_5\) without the dashed green arrows). We inspect this
template and learn the dynamic treatment strategy under consideration is
not identified.

\newpage{}

\subsection{S4. Structural Causal Models and Shift
interventions}\label{s4.-structural-causal-models-and-shift-interventions}

Below is Díaz et al. (\citeproc{ref-duxedaz2021}{2021})`s formulation of
Pearl (\citeproc{ref-pearl2009a}{2009})'s mathematical representation of
a structural causal model with non-independent errors. I present Díaz et
al. (\citeproc{ref-duxedaz2021}{2021})'s formulation because it allows
us to present a structural causal model for dynamic treatment strategies
Richardson \& Robins (\citeproc{ref-richardson2013}{2013}), also known
as longitudinal modified treatment policies Hoffman et al.
(\citeproc{ref-hoffman2023}{2023}). Note further that Richardson \&
Robins (\citeproc{ref-richardson2013}{2013}) develop their account of
time-varying treatments using structural causal models. The difference
is that on the potential outcome framework, identification does not rely
on non-independent error terms or so-called 'cross-world' assumptions
(\citeproc{ref-richardson2013}{Richardson \& Robins, 2013, pp. 60--84}).

Following Díaz, we begin by defining the sequence of variables in our
model:

\[
S_i= (W, Y_0, L_1, A_1, L_2, A_2, ..., L_\tau, A_\tau, Y_{\tau}) \sim \mathbf{P}
\]

where \(S_i\) is a sample from the distribution \(\mathbf{P}\) and
includes baseline covariates \(W\), intermediate outcomes \(L_t\),
treatments \(A_t\), and final outcomes \(Y_{\tau}\) over time periods
\(t = 1, 2, \ldots, \tau\).

We define the final outcome:

\[
Y = A_{\tau + 1}
\]

We define the history of all variables up to treatment \(A_t\) as:

\[
H_t = (\bar{A}_{t-1}, \bar{L}_t)
\]

Here, \(\bar{A}_{t-1}\) represents the history of treatments up to time
\(t-1\), and \(\bar{L}_t\) represents the history of intermediate
outcomes up to time \(t\).

We define the vector of exogenous variables (error terms). Note on
Pearl's structural causal model account, but not the potential outcomes
framework, the error terms must always be independent:

\[
U = (U_{L,t}, U_{A,t}, U_{Y}: t \in \{1 \dots \tau\})
\]

Where \(U\) describes the set of exogenous variables affecting \(L_t\),
\(A_t\), and \(Y\).

We assume the following deterministic functions for the intermediate
outcomes, treatments, and final outcome:

\begin{enumerate}
\def\labelenumi{\arabic{enumi}.}
\tightlist
\item
  For intermediate outcomes:
\end{enumerate}

\[
L_t = f_{L_t}(A_{t-1}, H_{t-1}, U_{L,t})
\]

\begin{enumerate}
\def\labelenumi{\arabic{enumi}.}
\setcounter{enumi}{1}
\tightlist
\item
  For treatments:
\end{enumerate}

\[
A_t = f_{A_t}(H_t, U_{A,t})
\]

\begin{enumerate}
\def\labelenumi{\arabic{enumi}.}
\setcounter{enumi}{2}
\tightlist
\item
  For the final outcome:
\end{enumerate}

\[
Y = f_{Y}(A_{\tau}, H_{\tau}, U_{Y})
\]

Longitudinal modified treatment policies (LMTPs) are defined as
functions that assign treatments flexibly based on individual co-variate
histories. Note that where there are multiple treatments, these
histories will be partially counterfactual.

We replace the deterministic function for treatments:

\[
A_t = f_{A_t}(H_t, U_{A,t})
\]

With the intervention function:

\[
A(\mathbf{g}_t)
\]

On the structural causal model account, this intervention produces
counterfactual histories given:

\[
L_t(\bar{A}(\mathbf{g})_{t-1}) = f_{L_t}(A(\mathbf{g})_{t-1}, H_{t-1}(\bar{A}(\mathbf{g})_{t-2}), U_{L,t})
\]

For treatments, the counterfactual variable
\(A_t(\bar{A}^\mathbf{g}_{t-1})\) is defined as the natural value of the
treatment, i.e., the value of the treatment that would have been
observed at time \(t\) under the intervention history leading up to it
at \(t-1\), and then discontinued:

\[
A_t(\bar{A}^\mathbf{g}_{t-1}) = f_{A_t}(H_t(\bar{A}^\mathbf{g}_{t-1}), H_{t-1}(\bar{A}^\mathbf{g}_{t-2}), U_{L,t})
\]

When all variables are intervened on, the counterfactual final outcome
is:

\[
Y(\bar{A}^\mathbf{g}) = f_Y(A^\mathbf{g}_\tau, H_\tau(\bar{A}^\mathbf{g}_{\tau-1}), U_{Y})
\]

Williams \& Díaz (\citeproc{ref-williams2021}{2021}) have developed the
lmtp package in R for estimating time-varying treatments with
time-varying confounding. Among the many excellent features of their
software is that it uses semi-parametric estimators, which can be
specified from the Polley et al. (\citeproc{ref-SuperLearner2023}{2023})
library. Readers working with \texttt{lmtp} might find the following
collection of tools useful for evaluating assumptions, creating
graphical outputs and tables, and automating reporting: Bulbulia
(\citeproc{ref-margot2024}{2024}).

\newpage{}

\subsection{References}\label{references}

\phantomsection\label{refs}
\begin{CSLReferences}{1}{0}
\bibitem[\citeproctext]{ref-margot2024}
Bulbulia, J. A. (2024). \emph{Margot: MARGinal observational
treatment-effects}. \url{https://doi.org/10.5281/zenodo.10907724}

\bibitem[\citeproctext]{ref-duxedaz2021}
Díaz, I., Williams, N., Hoffman, K. L., \& Schenck, E. J. (2021).
Non-parametric causal effects based on longitudinal modified treatment
policies. \emph{Journal of the American Statistical Association}.
\url{https://doi.org/10.1080/01621459.2021.1955691}

\bibitem[\citeproctext]{ref-hoffman2023}
Hoffman, K. L., Salazar-Barreto, D., Rudolph, K. E., \& Díaz, I. (2023).
\emph{Introducing longitudinal modified treatment policies: A unified
framework for studying complex exposures}.
\url{https://doi.org/10.48550/arXiv.2304.09460}

\bibitem[\citeproctext]{ref-pearl2009a}
Pearl, J. (2009). \emph{Causality}. Cambridge University Press.

\bibitem[\citeproctext]{ref-SuperLearner2023}
Polley, E., LeDell, E., Kennedy, C., \& van der Laan, M. (2023).
\emph{SuperLearner: Super learner prediction}.
\url{https://github.com/ecpolley/SuperLearner}

\bibitem[\citeproctext]{ref-richardson2013}
Richardson, T. S., \& Robins, J. M. (2013). \emph{Single world
intervention graphs: A primer}.
\url{https://core.ac.uk/display/102673558}

\bibitem[\citeproctext]{ref-robins1986}
Robins, J. (1986). A new approach to causal inference in mortality
studies with a sustained exposure period---application to control of the
healthy worker survivor effect. \emph{Mathematical Modelling},
\emph{7}(9-12), 1393--1512.

\bibitem[\citeproctext]{ref-robins2010alternative}
Robins, J. M., \& Richardson, T. S. (2010). Alternative graphical causal
models and the identification of direct effects. \emph{Causality and
Psychopathology: Finding the Determinants of Disorders and Their Cures},
\emph{84}, 103--158.

\bibitem[\citeproctext]{ref-williams2021}
Williams, N. T., \& Díaz, I. (2021). \emph{{l}mtp: Non-parametric causal
effects of feasible interventions based on modified treatment policies}.
\url{https://doi.org/10.5281/zenodo.3874931}

\end{CSLReferences}



\end{document}
