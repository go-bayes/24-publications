% Options for packages loaded elsewhere
\PassOptionsToPackage{unicode}{hyperref}
\PassOptionsToPackage{hyphens}{url}
\PassOptionsToPackage{dvipsnames,svgnames,x11names}{xcolor}
%
\documentclass[
  single column]{article}

\usepackage{amsmath,amssymb}
\usepackage{iftex}
\ifPDFTeX
  \usepackage[T1]{fontenc}
  \usepackage[utf8]{inputenc}
  \usepackage{textcomp} % provide euro and other symbols
\else % if luatex or xetex
  \usepackage{unicode-math}
  \defaultfontfeatures{Scale=MatchLowercase}
  \defaultfontfeatures[\rmfamily]{Ligatures=TeX,Scale=1}
\fi
\usepackage[]{libertinus}
\ifPDFTeX\else  
    % xetex/luatex font selection
\fi
% Use upquote if available, for straight quotes in verbatim environments
\IfFileExists{upquote.sty}{\usepackage{upquote}}{}
\IfFileExists{microtype.sty}{% use microtype if available
  \usepackage[]{microtype}
  \UseMicrotypeSet[protrusion]{basicmath} % disable protrusion for tt fonts
}{}
\makeatletter
\@ifundefined{KOMAClassName}{% if non-KOMA class
  \IfFileExists{parskip.sty}{%
    \usepackage{parskip}
  }{% else
    \setlength{\parindent}{0pt}
    \setlength{\parskip}{6pt plus 2pt minus 1pt}}
}{% if KOMA class
  \KOMAoptions{parskip=half}}
\makeatother
\usepackage{xcolor}
\usepackage[top=30mm,left=20mm,heightrounded]{geometry}
\setlength{\emergencystretch}{3em} % prevent overfull lines
\setcounter{secnumdepth}{-\maxdimen} % remove section numbering
% Make \paragraph and \subparagraph free-standing
\ifx\paragraph\undefined\else
  \let\oldparagraph\paragraph
  \renewcommand{\paragraph}[1]{\oldparagraph{#1}\mbox{}}
\fi
\ifx\subparagraph\undefined\else
  \let\oldsubparagraph\subparagraph
  \renewcommand{\subparagraph}[1]{\oldsubparagraph{#1}\mbox{}}
\fi


\providecommand{\tightlist}{%
  \setlength{\itemsep}{0pt}\setlength{\parskip}{0pt}}\usepackage{longtable,booktabs,array}
\usepackage{calc} % for calculating minipage widths
% Correct order of tables after \paragraph or \subparagraph
\usepackage{etoolbox}
\makeatletter
\patchcmd\longtable{\par}{\if@noskipsec\mbox{}\fi\par}{}{}
\makeatother
% Allow footnotes in longtable head/foot
\IfFileExists{footnotehyper.sty}{\usepackage{footnotehyper}}{\usepackage{footnote}}
\makesavenoteenv{longtable}
\usepackage{graphicx}
\makeatletter
\def\maxwidth{\ifdim\Gin@nat@width>\linewidth\linewidth\else\Gin@nat@width\fi}
\def\maxheight{\ifdim\Gin@nat@height>\textheight\textheight\else\Gin@nat@height\fi}
\makeatother
% Scale images if necessary, so that they will not overflow the page
% margins by default, and it is still possible to overwrite the defaults
% using explicit options in \includegraphics[width, height, ...]{}
\setkeys{Gin}{width=\maxwidth,height=\maxheight,keepaspectratio}
% Set default figure placement to htbp
\makeatletter
\def\fps@figure{htbp}
\makeatother
% definitions for citeproc citations
\NewDocumentCommand\citeproctext{}{}
\NewDocumentCommand\citeproc{mm}{%
  \begingroup\def\citeproctext{#2}\cite{#1}\endgroup}
\makeatletter
 % allow citations to break across lines
 \let\@cite@ofmt\@firstofone
 % avoid brackets around text for \cite:
 \def\@biblabel#1{}
 \def\@cite#1#2{{#1\if@tempswa , #2\fi}}
\makeatother
\newlength{\cslhangindent}
\setlength{\cslhangindent}{1.5em}
\newlength{\csllabelwidth}
\setlength{\csllabelwidth}{3em}
\newenvironment{CSLReferences}[2] % #1 hanging-indent, #2 entry-spacing
 {\begin{list}{}{%
  \setlength{\itemindent}{0pt}
  \setlength{\leftmargin}{0pt}
  \setlength{\parsep}{0pt}
  % turn on hanging indent if param 1 is 1
  \ifodd #1
   \setlength{\leftmargin}{\cslhangindent}
   \setlength{\itemindent}{-1\cslhangindent}
  \fi
  % set entry spacing
  \setlength{\itemsep}{#2\baselineskip}}}
 {\end{list}}
\usepackage{calc}
\newcommand{\CSLBlock}[1]{\hfill\break\parbox[t]{\linewidth}{\strut\ignorespaces#1\strut}}
\newcommand{\CSLLeftMargin}[1]{\parbox[t]{\csllabelwidth}{\strut#1\strut}}
\newcommand{\CSLRightInline}[1]{\parbox[t]{\linewidth - \csllabelwidth}{\strut#1\strut}}
\newcommand{\CSLIndent}[1]{\hspace{\cslhangindent}#1}

\usepackage{booktabs}
\usepackage{longtable}
\usepackage{array}
\usepackage{multirow}
\usepackage{wrapfig}
\usepackage{float}
\usepackage{colortbl}
\usepackage{pdflscape}
\usepackage{tabu}
\usepackage{threeparttable}
\usepackage{threeparttablex}
\usepackage[normalem]{ulem}
\usepackage{makecell}
\usepackage{xcolor}
\input{/Users/joseph/GIT/latex/latex-for-quarto.tex}
\makeatletter
\@ifpackageloaded{caption}{}{\usepackage{caption}}
\AtBeginDocument{%
\ifdefined\contentsname
  \renewcommand*\contentsname{Table of contents}
\else
  \newcommand\contentsname{Table of contents}
\fi
\ifdefined\listfigurename
  \renewcommand*\listfigurename{List of Figures}
\else
  \newcommand\listfigurename{List of Figures}
\fi
\ifdefined\listtablename
  \renewcommand*\listtablename{List of Tables}
\else
  \newcommand\listtablename{List of Tables}
\fi
\ifdefined\figurename
  \renewcommand*\figurename{Figure}
\else
  \newcommand\figurename{Figure}
\fi
\ifdefined\tablename
  \renewcommand*\tablename{Table}
\else
  \newcommand\tablename{Table}
\fi
}
\@ifpackageloaded{float}{}{\usepackage{float}}
\floatstyle{ruled}
\@ifundefined{c@chapter}{\newfloat{codelisting}{h}{lop}}{\newfloat{codelisting}{h}{lop}[chapter]}
\floatname{codelisting}{Listing}
\newcommand*\listoflistings{\listof{codelisting}{List of Listings}}
\makeatother
\makeatletter
\makeatother
\makeatletter
\@ifpackageloaded{caption}{}{\usepackage{caption}}
\@ifpackageloaded{subcaption}{}{\usepackage{subcaption}}
\makeatother
\ifLuaTeX
  \usepackage{selnolig}  % disable illegal ligatures
\fi
\usepackage{bookmark}

\IfFileExists{xurl.sty}{\usepackage{xurl}}{} % add URL line breaks if available
\urlstyle{same} % disable monospaced font for URLs
\hypersetup{
  pdftitle={Causal effects of religious service attendance on prosociality evidence from a national longitudinal study},
  pdfauthor={Joseph A. Bulbulia; Don E Davis; Ken Rice; Chris G. Sibley; Geoffrey Troughton},
  pdfkeywords={Causal
Inference, Charity, Church, Cooperation, Religion, Shift
Intervention, Volunteering},
  colorlinks=true,
  linkcolor={blue},
  filecolor={Maroon},
  citecolor={Blue},
  urlcolor={Blue},
  pdfcreator={LaTeX via pandoc}}

\title{Causal effects of religious service attendance on prosociality
evidence from a national longitudinal study}
\author{Joseph A. Bulbulia \and Don E Davis \and Ken Rice \and Chris G.
Sibley \and Geoffrey Troughton}
\date{2024-04-06}

\begin{document}
\maketitle
\begin{abstract}
Whether religion causes prosociality is a poorly defined question. To
make it precise, we must state a causal contrast of interest, specify
our measures of ``prosociality,'' and clarify the target population for
whom results are meant to generalise. Here, we investigate three
hypothetical interventions on religious service attendance and examine
their effects on four prosociality measurement domains for the
population of New Zealand. Each treatment policy addresses a causal
question motivated by a distinct theoretical or practical interest.
Study 1 investigates the effects of the three interventions on
self-reported social behaviours; Study 2 investigates effects on
perceived social connection; Study 3 investigates effects on reported
hours of help received from others; Study 4 investigates the effects of
reported money received from others. Overall, we find evidence that
religion supports pro-social behaviours across all four measurement
domains of prosociality. However, the magnitude of benefit varies
depending on the causal contrast of interest -- and especially whether
religious service is gained or lost. Magnitudes of benefit are almost
three-times lower than is suggested by cross-sectional regression
models. Nevertheless, both the gain and loss of regular religious
service attendance have considerable economic consequences for
charitable donations, whose ``cash value'' we compute. Collectively,
these findings document the social consequences of religious behaviours
and illustrate a pathway for quantitative causal investigations of the
effects of cultural practices.
\end{abstract}

\subsection{Introduction}\label{introduction}

A central question in the scientific study of religion is whether
religion causes cooperation. However, to address this question is
challenging. Most features of religion are inaccessible to experiments.
Yet to obtain valid causal inferences from non-experimental data
requires a combination of high-resolution time-series data and robust
causal inferential methods. Few studies that combine these features.

Conceptually, the question of'' ``whether religion causes prosociality''
is poorly defined. To make this question precise, we must state the
causal contrasts of interest, the specific measures of ``prosociality''
of interest, and the target population for whom results are meant to
generalise.

Here, we leverage comprehensive panel data from 47948 who participated
in the New Zealand Attitudes and Values Study from 2018-2021 and
investigate the effects of well-defined population-level shifts in
religious attendance across four measurement domains of pro-sociality.
The target quantities for each of the four domains are longitudinal
modified shift policies. The first treatment policy addresses the
theoretical question: ``What would the population-wide difference be on
a well-defined measure of pro-sociality if everyone attended weekly
religious service compared with if no one attended?'' This question
holds theoretical interest. The second treatment policy addresses the
practical question: ``What would be the population-wide difference on a
well-defined measure domain of pro-sociality if everyone attended weekly
religious service compared to the status quo?'' Here, the contrast does
not require shifting those people who regularly attend religious
services. The question reflects policy interests in supporting the
\emph{gain} of religious attendance. The third treatment policy
addresses the practical question: ``What would be the population-wide
difference on a well-defined measure domain of pro-sociality if everyone
stopped attending weekly religious service compared to the status quo.''
Here, the contrast does not require shifting those who never attend
religious services; it reflects policy interests in supporting the
\emph{loss} of religious attendance (or in the New Zealand setting,
where loss exceeds gain, of failing to inhibit the loss of regular
religious service).

\subsection{Method}\label{method}

\subsubsection{Sample}\label{sample}

Data were collected as part of The New Zealand Attitudes and Values
Study (NZAVS), an annual longitudinal national probability panel study
of social attitudes, personality, ideology and health outcomes. The
NZAVS began in 2009. It includes questionnaire responses 72910 New
Zealand residents. Because the NZAVS follows the same people over time,
with solid retention, it can track subtle changes in attitudes and
values over time. The NZAVS is university-based, not-for-profit and
independent of political or corporate funding; see
https://doi.org/10.17605/OSF.IO/75SNB. Sample descriptive information is
given in Appendices B-D.

\subsubsection{Outcome-wide measures of
prosociality}\label{outcome-wide-measures-of-prosociality}

Rather than cherry-picking one or several measures of prosociality, we
adopted an unrestrictive ``outcome-wide'' approach that used a rich
array of measures to assess multiple dimensions of a potentially
multi-dimensional construct {[}{]}. The measures we use to investigate
this construct:

\textbf{Study 1. Self-reported prosocial behaviours} as measured by:

\begin{enumerate}
\def\labelenumi{(\alph{enumi})}
\tightlist
\item
  self-reported hours of weekly volunteering;
\item
  self-reported annual charitable financial donations (dollars);
\end{enumerate}

\textbf{Study 2. Perceived social connection} as measured by:

\begin{enumerate}
\def\labelenumi{(\alph{enumi})}
\tightlist
\item
  social belonging (ordinal 1-7),
\item
  social support (ordinal 1-7),
\item
  neighbourhood belonging (ordinal 1-7)
\end{enumerate}

\textbf{Study 3: Reported personal help received} in the previous week
from:

\begin{enumerate}
\def\labelenumi{(\alph{enumi})}
\tightlist
\item
  family (yes/no)
\item
  friends (yes/no)
\item
  community (yes/no)
\end{enumerate}

\textbf{Study 4: Reported financial help received} in the previous week
from:

\begin{enumerate}
\def\labelenumi{(\alph{enumi})}
\tightlist
\item
  family (yes/no)
\item
  friends (yes/no)
\item
  community (yes/no)
\end{enumerate}

The wording and sources of all measures is provided in Appendix A.

Studies 2-4 (and perhaps Study 2) avoid self-presentation bias by using
individuals as measures of prosocial effects from exposure to others who
attend religious services. The argument for including these measures is
as follows: if religious service attendance induced greater within-group
prosociality, then regular religious service attendance would increase
interaction with a prosocial community. Studies 3 and 4 are especially
robust to self-presentation bias because they rely on self-reported
dependency.

\subsubsection{Causal Contrasts}\label{causal-contrasts}

We again emphasise that the question ``Does religious service attendance
cause prosociality?'' is ill-defined. Answering a causal question
requires that we first clearly and precisely state the interventions
that are to be contrasted, the outcomes of interest, the scale at which
contrasts will be computed, and the population for whom effects are
meant to be generalised {[}{]}.

The contrasts of interest in this study take the form of longitudinal
modified treatment policies. To precisely state these policies, let
\(A\) denote the intervention of interest monthly frequency of religious
service. Our treatment policies evaluate interventions that take the
form of differing shift functions \(f(A)\):

\[
f(A) = 
\begin{cases} 
a, & \text{where } a \text{ is the observed value.}\\
a^* = 4, & \text{if } a < 4 \times \text{monthly religious service attendance; otherwise, } a^* = a.\\
a' = 0, & \text{if } a > 0 \text{ and monthly religious service attendance is positive; otherwise, } a' = a. 
\end{cases}
\]

where,

\begin{itemize}
\item
  \textbf{for \(f(A=a)\)}: the observed value of \(a\) remains
  unchanged. Estimation is performed for the observed value.
\item
  \textbf{for \(f(A =a^*)\)}: \(a^* = 4\) if the observed value of \(a\)
  is less than four times the monthly religious service attendance. If
  the observed value of \(a\) is four or greater, no intervention is
  performed, and \(a^*\) equals \(a\). This function enforces a minimum
  threshold for religious attendance at four but allows those who meet
  or exceed this threshold to remain unchanged.
\item
  \textbf{for \(a^\prime\)}: this condition modifies \(a\) to \(a' = 0\)
  when the observed value of \(a\) is greater than 0, thus setting all
  values of religious attendance to 0 if they are not already at 0.
  Again, the intervention only applies to those who attend religious
  services at any frequency.
\end{itemize}

We next formulate our causal contrasts precisely. Let \(Y(\bar{a})\)
denote a potential outcome at the end of the study under the treatment
rule \(f(A)\). In this three-wave panel study, we consider two
intervention intervals, one at baseline and one at the baseline + 1
wave. Outcomes are subsequently assessed in the baseline + 2 wave. \(L\)
denotes the set of measured covariates we assume sufficient for
confounding control. Measurements of \(L\) are obtained from the
baseline wave. Because causes precede effects, panel data allow us to
order measurements such that confounders \(L\) preceded causes \(A\) and
causes \(A\) precede effects \(Y\).

\begin{enumerate}
\def\labelenumi{\arabic{enumi}.}
\tightlist
\item
  Our first causal question asks: ``What would be the social
  consequences one year after an intervention in which everyone attended
  religious service for two years compared with an intervention in which
  no one attended religious service for two years?'' For continuous
  outcomes, we compute the causal contrast from this longitudinal
  modified treatment policy (LMTP) on the difference scale as:
\end{enumerate}

\[ \text{LMTP: ALL GAIN vs ALL LOSS} = E[Y(a*,a*)|\textcolor{red}{f(A)},L] - E[Y(a^\prime,a^\prime)|\textcolor{red}{f(A)},L] \]

Again this causal question reflects a classical theoretical interest in
which we compare the population estimates with everyone attending
regularly vs the population estimates with no one attending regularly on
each measurement domain of interest.

\begin{enumerate}
\def\labelenumi{\arabic{enumi}.}
\setcounter{enumi}{1}
\tightlist
\item
  Our second causal question asks: ``What would be the social
  consequences if everyone who was not attending weekly religious
  service in New Zealand were to do so?'' The contrast conditions for
  this estimand are given by (a) regular religious service attendance
  and (b) the status quo, which can be written on the difference scale:
\end{enumerate}

\[  \text{LMTP: ALL GAIN vs. STATUS QUO}  = E[Y(a*,a*)|\textcolor{red}{f(A)},L] - E[Y(a,a)|\textcolor{red}{f(A)},L] \]

Again this causal question reflects a practical interest in comparing
the population estimates with everyone attending regularly vs the
population at its status quo on each measurement domain of interest. The
question helps us understand the implications of promoting religious
service attendance.

\begin{enumerate}
\def\labelenumi{\arabic{enumi}.}
\setcounter{enumi}{1}
\tightlist
\item
  Our third causal question asks: ``What would be the social
  consequences if everyone attending weekly religious service in New
  Zealand were to stop?'' The contrast conditions for this estimand are
  given by (a) no religious service attendance and (b) the status quo,
  which can be written on the difference scale:
\end{enumerate}

\[ \text{LMTP: ALL LOSS vs. STATUS QUO}  = E[Y(a^\prime,a^\prime)|\textcolor{red}{f(A)},L] - E[Y(a,a)|\textcolor{red}{f(A)},L] \]

This contrast allows us to evaluate the social effects of a policy that
promoted the society-wide loss of religious service attendance in the
contemporary population. In the case of New Zealand, where religious
service is on the decline, we may use this quantity to clarify the
social consequences of failing to prevent loss.

\subsubsection{Identification
assumptions}\label{identification-assumptions}

To obtain valid causal effect estimates, we must satisfy three
identification assumptions.

\begin{enumerate}
\def\labelenumi{\arabic{enumi}.}
\item
  \textbf{Causal consistency}: each outcome observed in the data
  corresponds to at least one of the potential outcomes to be contrasted
  {[}{]}. Moreover, multiple treatment versions must be independent of
  the potential outcomes conditional on measured covariates {[}{]}.
\item
  \textbf{Conditional exchangeability}: given the observed covariates,
  treatment assignment is independent of the potential outcomes under
  consideration. Put differently, our statistical models obtain balance
  in confounding covariates across the treatments of interest.
\end{enumerate}

To improve the credibility of this assumption, we conduct analysis using
semi-parametric doubly robust estimators, as explained below.

\begin{enumerate}
\def\labelenumi{\arabic{enumi}.}
\setcounter{enumi}{2}
\tightlist
\item
  \textbf{Positivity}: every subject has a non-zero probability of
  receiving each treatment to be compared. Below, we evaluate the
  positivity assumption by reporting the number of instances in which
  individuals transitioned from one state of religious service in the
  baseline wave to another in the treatment wave.
\end{enumerate}

\subsubsection{Target Population}\label{target-population}

The target population is the population of New Zealand at the baseline
wave (years 2018-2019). The New Zealand Attitudes and Values Study is a
national probability study with a good representation of the New Zealand
population. However, its coverage is not perfect. For example, the study
under-samples males and Asians and over-samples women and Māori (New
Zealand's indigenous peoples). Therefore, we pass New Zealand Census
survey weights when computing marginal treatment effects; in this study
we weighted the sample to age, gender and ethnicity. Methods for these
weights are described in {[}{]}.

\subsubsection{Eligibility criteria}\label{eligibility-criteria}

\begin{itemize}
\tightlist
\item
  Enrolled in the New Zealand Attitudes and Values Study in 2018 (NZAVS
  time 10).
\item
  Missing data for all variables at baseline were allowed and imputed
  using the \texttt{ppm} algorithm from the \texttt{mice} package
  (\citeproc{ref-vanbuuren2018}{Van Buuren 2018}).
\item
  Attrition/loss to follow-up was permitted. Inverse probability of
  censoring weights were calculated as part of estimation in
  \texttt{lmtp} to adjust for missing outcomes at NZAVS Time 12 (years
  2020-2021, the outcome wave). See details of the \texttt{lmtp} package
  (\citeproc{ref-williams2021}{Williams and Díaz 2021}))
\end{itemize}

There were 47948 who met these criteria.

\subsubsection{Confounding control}\label{confounding-control}

We use VanderWeele \emph{et al.} (\citeproc{ref-vanderweele2020}{2020})
\emph{modified disjunctive cause criterion} in which we (1) identified
all common causes of the treatment and outcomes; (2) removed any
variables identified that might influence the exposure but not the
outcome -- instrumental variables -- because instrumental variables are
known to reduce efficiency; (3) included proxies for any unmeasured
variables affecting both exposure and outcome -- including instrumental
variables, because by the rules of d-separation, conditioning on a proxy
is akin to conditioning on its parent (in this case, a confounder) (4)
\textbf{control for baseline exposure} and (5) \textbf{control for
baseline outcome} Both serve as proxies for unmeasured common causes
(\citeproc{ref-vanderweele2020}{VanderWeele \emph{et al.} 2020}). Note
that our shift interventions obtain controlled effects for baseline
treatment and exposure wave treatment, with the naturally observed
treatment levels being explicitly modelled from the data, see {[}{]} for
details.

\paragraph{Sensitivity Analysis Using the
E-value}\label{sensitivity-analysis-using-the-e-value}

To assess the sensitivity of results to unmeasured confounding, we
report VanderWeele and Ding's ``E-value'' in all analyses
(\citeproc{ref-vanderweele2017}{VanderWeele and Ding 2017}). The E-value
quantifies the minimum strength of association (on the risk ratio scale)
that an unmeasured confounder would need to have with both the exposure
and the outcome (after considering the measured covariates) to explain
away the observed exposure-outcome association
(\citeproc{ref-linden2020EVALUE}{Linden \emph{et al.} 2020};
\citeproc{ref-vanderweele2020}{VanderWeele \emph{et al.} 2020}). To
evaluate the strength of evidence, we use the bound of the E-value 95\%
confidence interval closest to 1.

\subsubsection{Analytic Approach}\label{analytic-approach}

We employ a semi-parametric Targeted Minimum Loss-based Estimation
(TMLE) estimator. TMLE is a robust method that combines machine learning
techniques with traditional statistical models to estimate causal
effects while providing valid statistical uncertainty measures for these
estimates.

TMLE operates through a two-step process involving both the outcome and
treatment (exposure) models. Initially, it employs machine learning
algorithms to flexibly model the relationship between treatments,
covariates, and outcomes. This flexibility allows TMLE to account for
complex, high-dimensional covariate spaces without imposing restrictive
model assumptions. The outcome of this step is a set of initial
estimates for these relationships.

The second step of TMLE involves ``targeting'' these initial estimates
by incorporating information about the observed data distribution to
improve the accuracy of the causal effect estimate. This is achieved
through an iterative updating process, which adjusts the initial
estimates towards the true causal effect. This updating process is
guided by the efficient influence function, ensuring that the final TMLE
estimate is as close as possible, given the measures and data to the
true causal effect, while still being robust to model misspecification
in either the outcome model or the treatment model.

Again, a central feature of TMLE is its double-robustness property. This
means that if either the model for the treatment or the outcome is
correctly specified, the TMLE estimator will still consistently estimate
the causal effect. Additionally, TMLE uses cross-validation to avoid
over-fitting and ensure that the estimator performs well in finite
samples. Each step contributes to a robust methodology for examining the
\emph{causal} effects of interventions on outcomes. The marriage of TMLE
and machine learning technologies reduces the dependence on restrictive
modelling assumptions and introduces an additional layer of robustness.
For further details see (\citeproc{ref-duxedaz2021}{Díaz \emph{et al.}
2021}; \citeproc{ref-hoffman2022}{Hoffman \emph{et al.} 2022},
\citeproc{ref-hoffman2023}{2023}). We performed estimation using the
\texttt{lmtp} package (\citeproc{ref-williams2021}{Williams and Díaz
2021}). We used the \texttt{superlearner} library for non-parametric
estimation with the predefined libraries \texttt{SL.ranger},
\texttt{SL.glmnet}, and \texttt{SL.xboost}. Graphs, tables and output
reports were created using the \texttt{margot} package
(\citeproc{ref-margot2024}{Bulbulia 2024b}). All analysis methods follow
pre-stated protocols in Bulbulia
(\citeproc{ref-bulbulia2024PRACTICAL}{2024a}).

\newpage{}

Table~\ref{tbl-identification} presents our strategy for confounding
control.

\begin{table}

\caption{\label{tbl-identification}Identification strategy}

\centering{

\lmtp table

}

\end{table}%

\subsubsection{Scope of Interventions}\label{scope-of-interventions}

To better understand the scope of the interventions to be contrasted, we
recommend presenting histograms showing treatment instances. The
following presents responses from baseline + 1 wave.

The All-GAIN intervention \(f(A=a^*)\) intervention is shown in
Figure~\ref{fig-0up}

\begin{figure}

\centering{

\includegraphics{v5-man-lmtp-ow-coop-church_files/figure-pdf/fig-0up-1.pdf}

}

\caption{\label{fig-0up}Figure shows a histogram of responses to
religious service frequency in the baseline + 1 wave. Responses above
eight were assigned to eight and values were rounded to the nearest
whole number. The red dashed line shows the population average. All
responses in the gold bars are shifted to four on the ALL-GAIN
intervention. All those responses in grey (four and above) remain
unchanged}

\end{figure}%

Notably, the ALL-GAIN intervention requires shifting a considerable mass
of the sample population. However, it does not require shifting those
who are attending religious service more frequently that four times per
month.

The All-LOSS intervention \(f(A=a^\prime)\) intervention is shown in
Figure~\ref{fig-0down}

\begin{figure}

\centering{

\includegraphics{v5-man-lmtp-ow-coop-church_files/figure-pdf/fig-0down-1.pdf}

}

\caption{\label{fig-0down}Figure shows a histogram of responses to
religious service frequency in the baseline + 1 wave. Responses above
eight were assigned to eight, and values were rounded to the nearest
whole number. The red dashed line shows the population average. All
responses in the blue bars are shifted to zero on the ALL-LOSS
intervention. All those responses in grey (zero) remain unchanged}

\end{figure}%

Notably, the ALL-LOSS intervention requires shifting a smaller
proportion of the sample population than the ALL-GAIN intervention.
Because religious service is declining in New Zealand, this intervention
holds more policy interest: the implications of loss are relevant to a
society where loss is occurring.

\subsubsection{Evidence for Change in the Treatment
Variable}\label{evidence-for-change-in-the-treatment-variable}

Table~\ref{tbl-transition} clarifies the change in the treatment
variable from the baseline wave to the baseline + 1 wave across the
sample. Assessing change in a variable is essential for evaluating the
positivity assumption and for recovering evidence for the
incidence-effect of the treatment variable {[}{]}.

\begin{table}

\caption{\label{tbl-placeholder}}

\centering{

\captionsetup{labelsep=none}

}

\end{table}%

\begin{longtable}[]{@{}
  >{\centering\arraybackslash}p{(\columnwidth - 18\tabcolsep) * \real{0.0978}}
  >{\centering\arraybackslash}p{(\columnwidth - 18\tabcolsep) * \real{0.1196}}
  >{\centering\arraybackslash}p{(\columnwidth - 18\tabcolsep) * \real{0.0978}}
  >{\centering\arraybackslash}p{(\columnwidth - 18\tabcolsep) * \real{0.0978}}
  >{\centering\arraybackslash}p{(\columnwidth - 18\tabcolsep) * \real{0.0978}}
  >{\centering\arraybackslash}p{(\columnwidth - 18\tabcolsep) * \real{0.0978}}
  >{\centering\arraybackslash}p{(\columnwidth - 18\tabcolsep) * \real{0.0978}}
  >{\centering\arraybackslash}p{(\columnwidth - 18\tabcolsep) * \real{0.0978}}
  >{\centering\arraybackslash}p{(\columnwidth - 18\tabcolsep) * \real{0.0978}}
  >{\centering\arraybackslash}p{(\columnwidth - 18\tabcolsep) * \real{0.0978}}@{}}
\caption{This table presents a transition matrix to evaluate treatment
shifts between baseline and treatment wave. Entries along the diagonal
(in bold) indicate the number of individuals who \textbf{stayed} in
their initial state. By contrast, the off-diagonal shows the transitions
from the initial state (bold) to another state in the following wave
(off-diagonal). Thus cell located at the intersection of row \(i\) and
column \(j\), where \(i \neq j\), gives us the counts of individuals
moving from state \(i\) to state
\(j\)}\label{tbl-transition}\tabularnewline
\toprule\noalign{}
\begin{minipage}[b]{\linewidth}\centering
From
\end{minipage} & \begin{minipage}[b]{\linewidth}\centering
State 0
\end{minipage} & \begin{minipage}[b]{\linewidth}\centering
State 1
\end{minipage} & \begin{minipage}[b]{\linewidth}\centering
State 2
\end{minipage} & \begin{minipage}[b]{\linewidth}\centering
State 3
\end{minipage} & \begin{minipage}[b]{\linewidth}\centering
State 4
\end{minipage} & \begin{minipage}[b]{\linewidth}\centering
State 5
\end{minipage} & \begin{minipage}[b]{\linewidth}\centering
State 6
\end{minipage} & \begin{minipage}[b]{\linewidth}\centering
State 7
\end{minipage} & \begin{minipage}[b]{\linewidth}\centering
State 8
\end{minipage} \\
\midrule\noalign{}
\endfirsthead
\toprule\noalign{}
\begin{minipage}[b]{\linewidth}\centering
From
\end{minipage} & \begin{minipage}[b]{\linewidth}\centering
State 0
\end{minipage} & \begin{minipage}[b]{\linewidth}\centering
State 1
\end{minipage} & \begin{minipage}[b]{\linewidth}\centering
State 2
\end{minipage} & \begin{minipage}[b]{\linewidth}\centering
State 3
\end{minipage} & \begin{minipage}[b]{\linewidth}\centering
State 4
\end{minipage} & \begin{minipage}[b]{\linewidth}\centering
State 5
\end{minipage} & \begin{minipage}[b]{\linewidth}\centering
State 6
\end{minipage} & \begin{minipage}[b]{\linewidth}\centering
State 7
\end{minipage} & \begin{minipage}[b]{\linewidth}\centering
State 8
\end{minipage} \\
\midrule\noalign{}
\endhead
\bottomrule\noalign{}
\endlastfoot
State 0 & \textbf{26762} & 405 & 174 & 71 & 126 & 26 & 13 & 8 & 68 \\
State 1 & 647 & \textbf{235} & 85 & 44 & 46 & 5 & 2 & 3 & 10 \\
State 2 & 236 & 105 & \textbf{188} & 104 & 96 & 12 & 13 & 2 & 21 \\
State 3 & 112 & 54 & 110 & \textbf{164} & 173 & 18 & 8 & 4 & 15 \\
State 4 & 150 & 71 & 127 & 205 & \textbf{881} & 124 & 64 & 16 & 91 \\
State 5 & 24 & 7 & 17 & 17 & 145 & \textbf{61} & 25 & 7 & 33 \\
State 6 & 14 & 5 & 13 & 17 & 84 & 22 & \textbf{29} & 5 & 37 \\
State 7 & 9 & 0 & 6 & 3 & 16 & 6 & 9 & \textbf{6} & 19 \\
State 8 & 74 & 14 & 17 & 14 & 105 & 34 & 42 & 17 & \textbf{351} \\
\end{longtable}

\newpage{}

\subsection{Results}\label{results}

\subsection{Study 1: Causal Effects of Regular Church Attendance on
Self-Reported Volunteering and Self-Reported Volunteering and
Donations}\label{study-1-causal-effects-of-regular-church-attendance-on-self-reported-volunteering-and-self-reported-volunteering-and-donations}

Figure~\ref{fig-1_1} and Table~\ref{tbl-1_1} present results for the
ALL-GAIN vs.~ALL-LOSS treatment policy on self-reported volunteering and
charitable donations

\begin{figure}

\centering{

\includegraphics{v5-man-lmtp-ow-coop-church_files/figure-pdf/fig-1_1-1.pdf}

}

\caption{\label{fig-1_1}This figure reports the results of model
estimates for the causal effects of a universal gain of weekly religious
service vs universal loss of weekly religious service on reported
charitable behaviours at the end of the study. Outcomes are expressed in
standard deviation units.}

\end{figure}%

\begin{verbatim}
          outcome E[Y(1)]-E[Y(0)]  2.5 % 97.5 % E_Value E_Val_bound
1       donations           0.173  0.097  0.250   1.617       1.409
2 hours volunteer           0.091 -0.042  0.223   1.393       1.000
      Estimate                        estimate_lab
1     positive 0.173 (0.097-0.25) [EV 1.617/1.409]
2 not reliable   0.091 (-0.042-0.223) [EV 1.393/1]
\end{verbatim}

\begin{longtable}[]{@{}lrrrrr@{}}

\caption{\label{tbl-1_1}This table reports the results of model
estimates for the causal effects of a universal gain of weekly religious
service vs universal loss of weekly religious service on reported
charitable behaviours at the end of the study. Outcomes are expressed in
standard deviation units.}

\tabularnewline

\toprule\noalign{}
& E{[}Y(1){]}-E{[}Y(0){]} & 2.5 \% & 97.5 \% & E\_Value &
E\_Val\_bound \\
\midrule\noalign{}
\endhead
\bottomrule\noalign{}
\endlastfoot
donations & 0.17 & 0.10 & 0.25 & 1.62 & 1.41 \\
hours volunteer & 0.09 & -0.04 & 0.22 & 1.39 & 1.00 \\

\end{longtable}

The Longitudinal Modified Treatment Policy (LMTP) calculates the
expected outcome difference between treatment and contrast groups for a
specified population, focusing on longitudinal data.

For `donations', the effect estimate is 0.173 {[}0.097, 0.25{]}. The
E-value for this estimate is 1.617 with a lower bound of 1.409. Given
all modelled covariates, an unmeasured confounder would need a minimum
association strength with both the treatment and outcome of 1.409 to
nullify the observed effect. Weaker associations would not overturn it.
We infer \textbf{Evidence for causality}.

On the data scale, this intervention represents a difference of NZD
827.77 per adult per year in charitable giving.

For `hours volunteer', the effect estimate is 0.091 {[}-0.042, 0.223{]}.
The E-value for this estimate is 1.393 with a lower bound of 1. Given
all modelled covariates, an unmeasured confounder would need a minimum
association strength with both the treatment and outcome of 1 to nullify
the observed effect. Weaker associations would not overturn it. We infer
\textbf{Evidence for causality is not reliable}.

\newpage{}

Figure~\ref{fig-1_2} and Table~\ref{tbl-1_2} present results for the
ALL-GAIN vs.~STATUS QUO treatment policy on self-reported volunteering
and charitable donations

\begin{figure}

\centering{

\includegraphics{v5-man-lmtp-ow-coop-church_files/figure-pdf/fig-1_2-1.pdf}

}

\caption{\label{fig-1_2}This figure reports the results of model
estimates for the causal effects of a universal gain of weekly religious
service vs status quo on reported charitable behaviours at the end of
the study. Outcomes are expressed in standard deviation units.}

\end{figure}%

\begin{longtable}[]{@{}lrrrrr@{}}

\caption{\label{tbl-1_2}Table reports results of model estimates for the
causal effects of a universal gain of weekly religious service vs status
quo on reported charitable behaviours at the end of the study. Outcomes
are expressed in standard deviation units.}

\tabularnewline

\toprule\noalign{}
& E{[}Y(1){]}-E{[}Y(0){]} & 2.5 \% & 97.5 \% & E\_Value &
E\_Val\_bound \\
\midrule\noalign{}
\endhead
\bottomrule\noalign{}
\endlastfoot
donations & 0.13 & 0.06 & 0.2 & 1.50 & 1.29 \\
hours volunteer & 0.07 & -0.06 & 0.2 & 1.33 & 1.00 \\

\end{longtable}

The Longitudinal Modified Treatment Policy (LMTP) calculates the
expected outcome difference between treatment and contrast groups for a
specified population, focusing on longitudinal data.

For `donations', the effect estimate is 0.131 {[}0.058, 0.205{]}. The
E-value for this estimate is 1.504 with a lower bound of 1.295. Given
all modelled covariates, an unmeasured confounder would need a minimum
association strength with both the treatment and outcome of 1.295 to
nullify the observed effect. Weaker associations would not overturn it.
We infer \textbf{Evidence for causality}.

On the data scale, this intervention represents a difference of NZD
626.81 per adult per year in charitable giving.

For `hours volunteer', the effect estimate is 0.069 {[}-0.06, 0.199{]}.
The E-value for this estimate is 1.327 with a lower bound of 1. Given
all modelled covariates, an unmeasured confounder would need a minimum
association strength with both the treatment and outcome of 1 to nullify
the observed effect. Weaker associations would not overturn it. We infer
\textbf{Evidence for causality is not reliable}.

\newpage{}

\begin{figure}

\centering{

\includegraphics{v5-man-lmtp-ow-coop-church_files/figure-pdf/fig-1_3-1.pdf}

}

\caption{\label{fig-1_3}Figure reports results of model estimates for
the causal effects of a universal loss of weekly religious service vs
status quo on reported charitable behaviours at the end of the study.
Outcomes are expressed in standard deviation units.}

\end{figure}%

\begin{longtable}[]{@{}
  >{\raggedright\arraybackslash}p{(\columnwidth - 10\tabcolsep) * \real{0.2424}}
  >{\raggedleft\arraybackslash}p{(\columnwidth - 10\tabcolsep) * \real{0.2424}}
  >{\raggedleft\arraybackslash}p{(\columnwidth - 10\tabcolsep) * \real{0.1061}}
  >{\raggedleft\arraybackslash}p{(\columnwidth - 10\tabcolsep) * \real{0.1061}}
  >{\raggedleft\arraybackslash}p{(\columnwidth - 10\tabcolsep) * \real{0.1212}}
  >{\raggedleft\arraybackslash}p{(\columnwidth - 10\tabcolsep) * \real{0.1818}}@{}}

\caption{\label{tbl-1_3}Table reports results of model estimates for the
causal effects of a universal loss of weekly religious service vs status
quo on reported charitable behaviours at the end of the study. Outcomes
are expressed in standard deviation units.}

\tabularnewline

\toprule\noalign{}
\begin{minipage}[b]{\linewidth}\raggedright
\end{minipage} & \begin{minipage}[b]{\linewidth}\raggedleft
E{[}Y(1){]}-E{[}Y(0){]}
\end{minipage} & \begin{minipage}[b]{\linewidth}\raggedleft
2.5 \%
\end{minipage} & \begin{minipage}[b]{\linewidth}\raggedleft
97.5 \%
\end{minipage} & \begin{minipage}[b]{\linewidth}\raggedleft
E\_Value
\end{minipage} & \begin{minipage}[b]{\linewidth}\raggedleft
E\_Val\_bound
\end{minipage} \\
\midrule\noalign{}
\endhead
\bottomrule\noalign{}
\endlastfoot
donations & -0.042 & -0.056 & -0.028 & 1.240 & 1.190 \\
hours volunteer & -0.022 & -0.033 & -0.010 & 1.164 & 1.107 \\

\end{longtable}

The Longitudinal Modified Treatment Policy (LMTP) calculates the
expected outcome difference between treatment and contrast groups for a
specified population, focusing on longitudinal data.

For `hours volunteer', the effect estimate is -0.022 {[}-0.033,
-0.01{]}. The E-value for this estimate is 1.164 with a lower bound of
1.107. Given all modelled covariates, an unmeasured confounder would
need a minimum association strength with both the treatment and outcome
of 1.107 to nullify the observed effect. Weaker associations would not
overturn it. We infer \textbf{weak evidence for causality}.

On the data scale, this intervention represents a difference of -5.38 in
volunteering minutes.

For `donations', the effect estimate is -0.042 {[}-0.056, -0.028{]}. The
E-value for this estimate is 1.24 with a lower bound of 1.19. Given all
modelled covariates, an unmeasured confounder would need a minimum
association strength with both the treatment and outcome of 1.19 to
nullify the observed effect. Weaker associations would not overturn it.
We infer \textbf{evidence for causality}.

On the data scale, this intervention represents a difference of NZD
-200.96 per adult per year in charitable giving.

\newpage{}

\subsection{Study 2: Causal Effects of Regular Church Attendance on
Self-Reported Social
Connection}\label{study-2-causal-effects-of-regular-church-attendance-on-self-reported-social-connection}

\begin{figure}

\centering{

\includegraphics{v5-man-lmtp-ow-coop-church_files/figure-pdf/fig-2_1-1.pdf}

}

\caption{\label{fig-2_1}Figure reports results of model estimates for
the causal effects of a universal gain of weekly religious service vs
universal loss of weekly religious service on perceived social
connection at the end of study. Outcomes are expressed in standard
deviation units.}

\end{figure}%

\begin{longtable}[]{@{}
  >{\raggedright\arraybackslash}p{(\columnwidth - 10\tabcolsep) * \real{0.3288}}
  >{\raggedleft\arraybackslash}p{(\columnwidth - 10\tabcolsep) * \real{0.2192}}
  >{\raggedleft\arraybackslash}p{(\columnwidth - 10\tabcolsep) * \real{0.0822}}
  >{\raggedleft\arraybackslash}p{(\columnwidth - 10\tabcolsep) * \real{0.0959}}
  >{\raggedleft\arraybackslash}p{(\columnwidth - 10\tabcolsep) * \real{0.1096}}
  >{\raggedleft\arraybackslash}p{(\columnwidth - 10\tabcolsep) * \real{0.1644}}@{}}

\caption{\label{tbl-2_1}Table reports results of model estimates for the
causal effects of a universal gain of weekly religious service vs
universal loss of weekly religious service on domains of perceived
social connection at the end of study. Outcomes are expressed in
standard deviation units.}

\tabularnewline

\toprule\noalign{}
\begin{minipage}[b]{\linewidth}\raggedright
\end{minipage} & \begin{minipage}[b]{\linewidth}\raggedleft
E{[}Y(1){]}-E{[}Y(0){]}
\end{minipage} & \begin{minipage}[b]{\linewidth}\raggedleft
2.5 \%
\end{minipage} & \begin{minipage}[b]{\linewidth}\raggedleft
97.5 \%
\end{minipage} & \begin{minipage}[b]{\linewidth}\raggedleft
E\_Value
\end{minipage} & \begin{minipage}[b]{\linewidth}\raggedleft
E\_Val\_bound
\end{minipage} \\
\midrule\noalign{}
\endhead
\bottomrule\noalign{}
\endlastfoot
neighbourhood community & 0.03 & -0.12 & 0.18 & 1.19 & 1.0 \\
social belonging & 0.12 & 0.03 & 0.21 & 1.48 & 1.2 \\
social suport & 0.05 & -0.06 & 0.16 & 1.27 & 1.0 \\

\end{longtable}

The estimate for `social belonging' is 0.121 {[}0.031,0.211{]}. The
E-value for this effect estimate is 1.477 with a lower bound of 1.2. In
this scenario, assuming all modelled covariates, an unmeasured
confounder that exhibits an association with both the treatment and
outcome with a minimum strength (on the risk ratio scale) of 1.2 could
nullify the observed effect. However, the effect would survive weaker
unmeasured confounding. Here, we find \textbf{evidence for causality}.

For the outcome `social support', given the lower bound of the E-value
equals 1; we infer that \textbf{evidence for causality is not reliable.}

For the outcome `neighbourhood community', given the lower bound of the
E-value equals 1; we infer that \textbf{evidence for causality is not
reliable.}

\newpage{}

\begin{figure}

\centering{

\includegraphics{v5-man-lmtp-ow-coop-church_files/figure-pdf/fig-2_2-1.pdf}

}

\caption{\label{fig-2_2}This figure reports the results of model
estimates for the causal effects of a universal gain of weekly religious
service vs status quo on perceived social connection at the end of the
study. Outcomes are expressed in standard deviation units.}

\end{figure}%

\begin{longtable}[]{@{}
  >{\raggedright\arraybackslash}p{(\columnwidth - 10\tabcolsep) * \real{0.3243}}
  >{\raggedleft\arraybackslash}p{(\columnwidth - 10\tabcolsep) * \real{0.2162}}
  >{\raggedleft\arraybackslash}p{(\columnwidth - 10\tabcolsep) * \real{0.0946}}
  >{\raggedleft\arraybackslash}p{(\columnwidth - 10\tabcolsep) * \real{0.0946}}
  >{\raggedleft\arraybackslash}p{(\columnwidth - 10\tabcolsep) * \real{0.1081}}
  >{\raggedleft\arraybackslash}p{(\columnwidth - 10\tabcolsep) * \real{0.1622}}@{}}

\caption{\label{tbl-2_2}This table reports the results of model
estimates for the causal effects of a universal gain of weekly religious
service vs.~status quo on perceived social connection at the end of the
study. Outcomes are expressed in standard deviation units.}

\tabularnewline

\toprule\noalign{}
\begin{minipage}[b]{\linewidth}\raggedright
\end{minipage} & \begin{minipage}[b]{\linewidth}\raggedleft
E{[}Y(1){]}-E{[}Y(0){]}
\end{minipage} & \begin{minipage}[b]{\linewidth}\raggedleft
2.5 \%
\end{minipage} & \begin{minipage}[b]{\linewidth}\raggedleft
97.5 \%
\end{minipage} & \begin{minipage}[b]{\linewidth}\raggedleft
E\_Value
\end{minipage} & \begin{minipage}[b]{\linewidth}\raggedleft
E\_Val\_bound
\end{minipage} \\
\midrule\noalign{}
\endhead
\bottomrule\noalign{}
\endlastfoot
neighbourhood community & 0.013 & -0.136 & 0.163 & 1.122 & 1.000 \\
social belonging & 0.108 & 0.019 & 0.197 & 1.441 & 1.155 \\
social support & 0.039 & -0.075 & 0.153 & 1.230 & 1.000 \\

\end{longtable}

The estimate for `social belonging' is 0.108 {[}0.019,0.197{]}. The
E-value for this effect estimate is 1.441 with a lower bound of 1.155.
In this context, if an unmeasured confounder is associated with both the
treatment and the outcome and this association has a risk ratio of
1.155, such a confounder could negate the observed effect. Conversely,
any confounder with a weaker association (i.e., a risk ratio of less
than 1.155) would not be sufficient to fully account for the observed
effect. Here, we find evidence for causality.

For the outcome `social support', given the lower bound of the E-value
equals 1, we infer there is no reliable evidence for causality.

For the outcome `neighbourhood community', given the lower bound of the
E-value equals 1, we infer there is no reliable evidence for causality.
\textgreater{}

\newpage{}

\begin{figure}

\centering{

\includegraphics{v5-man-lmtp-ow-coop-church_files/figure-pdf/fig-2_3-1.pdf}

}

\caption{\label{fig-2_3}This figure reports the results of model
estimates for the causal effects of a universal loss of weekly religious
service vs.~the status quo on perceived social connection at the end of
the study. Outcomes are expressed in standard deviation units.}

\end{figure}%

\begin{longtable}[]{@{}
  >{\raggedright\arraybackslash}p{(\columnwidth - 10\tabcolsep) * \real{0.3243}}
  >{\raggedleft\arraybackslash}p{(\columnwidth - 10\tabcolsep) * \real{0.2162}}
  >{\raggedleft\arraybackslash}p{(\columnwidth - 10\tabcolsep) * \real{0.0946}}
  >{\raggedleft\arraybackslash}p{(\columnwidth - 10\tabcolsep) * \real{0.0946}}
  >{\raggedleft\arraybackslash}p{(\columnwidth - 10\tabcolsep) * \real{0.1081}}
  >{\raggedleft\arraybackslash}p{(\columnwidth - 10\tabcolsep) * \real{0.1622}}@{}}

\caption{\label{tbl-2_3}Table reports results of model estimates for the
causal effects of a universal loss of weekly religious service vs status
quo on perceived social connection at the end of the study. Outcomes are
expressed in standard deviation units.}

\tabularnewline

\toprule\noalign{}
\begin{minipage}[b]{\linewidth}\raggedright
\end{minipage} & \begin{minipage}[b]{\linewidth}\raggedleft
E{[}Y(1){]}-E{[}Y(0){]}
\end{minipage} & \begin{minipage}[b]{\linewidth}\raggedleft
2.5 \%
\end{minipage} & \begin{minipage}[b]{\linewidth}\raggedleft
97.5 \%
\end{minipage} & \begin{minipage}[b]{\linewidth}\raggedleft
E\_Value
\end{minipage} & \begin{minipage}[b]{\linewidth}\raggedleft
E\_Val\_bound
\end{minipage} \\
\midrule\noalign{}
\endhead
\bottomrule\noalign{}
\endlastfoot
neighbourhood community & -0.016 & -0.023 & -0.009 & 1.137 & 1.094 \\
social belonging & -0.013 & -0.019 & -0.006 & 1.122 & 1.087 \\
social support & -0.011 & -0.018 & -0.004 & 1.111 & 1.073 \\

\end{longtable}

The estimate for the outcome `social support' is -0.011
{[}-0.018,-0.004{]}. The E-value for this effect estimate is 1.111 with
a lower bound of 1.073. In this scenario, assuming all modelled
covariates, an unmeasured confounder that exhibits an association with
both the treatment and outcome with a minimum strength (on the risk
ratio scale) of 1.073 could nullify the observed effect. However, the
effect would survive weaker unmeasured confounding. Here, we find
\textbf{the evidence for causality is weak}.

The estimate for the outcome `social belonging' is -0.013
{[}-0.019,-0.006{]}. The E-value for this effect estimate is 1.122 with
a lower bound of 1.087. In this scenario, assuming all modelled
covariates, an unmeasured confounder that exhibits an association with
both the treatment and outcome with a minimum strength (on the risk
ratio scale) of 1.087 could nullify the observed effect. However, the
effect would survive weaker unmeasured confounding. Here, we find
\textbf{the evidence for causality is weak}.

The estimate for the outcome `neighbourhood community' is -0.016
{[}-0.023,-0.009{]}. The E-value for this effect estimate is 1.137 with
a lower bound of 1.094. In this scenario, assuming all modelled
covariates, an unmeasured confounder that exhibits an association with
both the treatment and outcome with a minimum strength (on the risk
ratio scale) of 1.094 could nullify the observed effect. However, the
effect would survive weaker unmeasured confounding. Here, we find
\textbf{the evidence for causality is weak}.

\newpage{}

\subsection{Study 3: Causal Effects of Regular Church Attendance on
Support Received From Others --
Time}\label{study-3-causal-effects-of-regular-church-attendance-on-support-received-from-others-time}

\begin{figure}

\centering{

\includegraphics{v5-man-lmtp-ow-coop-church_files/figure-pdf/fig-3_1-1.pdf}

}

\caption{\label{fig-3_1}This figure reports the results of model
estimates for the causal effects of a universal gain of weekly religious
service vs universal loss of weekly religious service on voluntary help
received from others during the past week (yes/no) at the end of the
study. Outcomes are expressed on the risk ratio scale.}

\end{figure}%

\begin{longtable}[]{@{}
  >{\raggedright\arraybackslash}p{(\columnwidth - 10\tabcolsep) * \real{0.3000}}
  >{\raggedleft\arraybackslash}p{(\columnwidth - 10\tabcolsep) * \real{0.2286}}
  >{\raggedleft\arraybackslash}p{(\columnwidth - 10\tabcolsep) * \real{0.0857}}
  >{\raggedleft\arraybackslash}p{(\columnwidth - 10\tabcolsep) * \real{0.1000}}
  >{\raggedleft\arraybackslash}p{(\columnwidth - 10\tabcolsep) * \real{0.1143}}
  >{\raggedleft\arraybackslash}p{(\columnwidth - 10\tabcolsep) * \real{0.1714}}@{}}

\caption{\label{tbl-3_1}Table reports results of model estimates for the
causal effects of a universal gain of weekly religious service vs
universal loss of weekly religious service on voluntary help received
from others during the past week (yes/no) at the end of study. Outcomes
are expressed on the risk ratio scale.}

\tabularnewline

\toprule\noalign{}
\begin{minipage}[b]{\linewidth}\raggedright
\end{minipage} & \begin{minipage}[b]{\linewidth}\raggedleft
E{[}Y(1){]}/E{[}Y(0){]}
\end{minipage} & \begin{minipage}[b]{\linewidth}\raggedleft
2.5 \%
\end{minipage} & \begin{minipage}[b]{\linewidth}\raggedleft
97.5 \%
\end{minipage} & \begin{minipage}[b]{\linewidth}\raggedleft
E\_Value
\end{minipage} & \begin{minipage}[b]{\linewidth}\raggedleft
E\_Val\_bound
\end{minipage} \\
\midrule\noalign{}
\endhead
\bottomrule\noalign{}
\endlastfoot
family gives time & 1.17 & 1.00 & 1.37 & 1.61 & 1.00 \\
friends give time & 1.24 & 0.98 & 1.55 & 1.77 & 1.00 \\
community gives time & 1.44 & 1.05 & 1.98 & 2.23 & 1.26 \\

\end{longtable}

For the outcome `community gives time', the estimate is 1.439
{[}1.046,1.979{]}. The E-value for this effect estimate is 2.234 with a
lower bound of 1.265. In this scenario, assuming all modelled
covariates, an unmeasured confounder that exhibits an association with
both the treatment and outcome with a minimum strength (on the risk
ratio scale) of 1.265 could nullify the observed effect. However, the
effect would survive weaker unmeasured confounding. Here, we find
\textbf{evidence for causality}.

For the outcome `friends gives time', given the lower bound of the
E-value equals 1; we infer that \textbf{evidence for causality is not
reliable.}

For the outcome `family gives time', given the lower bound of the
E-value equals 1; we infer that \textbf{evidence for causality is not
reliable.}

\newpage{}

\begin{figure}

\centering{

\includegraphics{v5-man-lmtp-ow-coop-church_files/figure-pdf/fig-3_2-1.pdf}

}

\caption{\label{fig-3_2}Figure reports results of model estimates for
the causal effects of a universal gain of weekly religious service vs
status quo on voluntary help received from others during the past week
(yes/no) at the end of study. Outcomes are expressed on the risk ratio
scale.}

\end{figure}%

\begin{longtable}[]{@{}
  >{\raggedright\arraybackslash}p{(\columnwidth - 10\tabcolsep) * \real{0.3000}}
  >{\raggedleft\arraybackslash}p{(\columnwidth - 10\tabcolsep) * \real{0.2286}}
  >{\raggedleft\arraybackslash}p{(\columnwidth - 10\tabcolsep) * \real{0.0857}}
  >{\raggedleft\arraybackslash}p{(\columnwidth - 10\tabcolsep) * \real{0.1000}}
  >{\raggedleft\arraybackslash}p{(\columnwidth - 10\tabcolsep) * \real{0.1143}}
  >{\raggedleft\arraybackslash}p{(\columnwidth - 10\tabcolsep) * \real{0.1714}}@{}}

\caption{\label{tbl-3_2}Table reports results of model estimates for the
causal effects of a universal gain of weekly religious service vs status
quo on voluntary help received from others during the past week (yes/no)
at the end of study. Outcomes are expressed on the risk ratio scale.}

\tabularnewline

\toprule\noalign{}
\begin{minipage}[b]{\linewidth}\raggedright
\end{minipage} & \begin{minipage}[b]{\linewidth}\raggedleft
E{[}Y(1){]}/E{[}Y(0){]}
\end{minipage} & \begin{minipage}[b]{\linewidth}\raggedleft
2.5 \%
\end{minipage} & \begin{minipage}[b]{\linewidth}\raggedleft
97.5 \%
\end{minipage} & \begin{minipage}[b]{\linewidth}\raggedleft
E\_Value
\end{minipage} & \begin{minipage}[b]{\linewidth}\raggedleft
E\_Val\_bound
\end{minipage} \\
\midrule\noalign{}
\endhead
\bottomrule\noalign{}
\endlastfoot
family gives time & 1.140 & 0.974 & 1.335 & 1.539 & 1 \\
friends gives time & 1.192 & 0.954 & 1.490 & 1.670 & 1 \\
community gives time & 1.291 & 0.946 & 1.761 & 1.904 & 1 \\

\end{longtable}

For the outcome `community gives time', given the lower bound of the
E-value equals 1; we infer that \textbf{evidence for causality is not
reliable.}

For the outcome `friends gives time', given the lower bound of the
E-value equals 1; we infer that \textbf{evidence for causality is not
reliable.}

For the outcome `family gives time', given the lower bound of the
E-value equals 1; we infer that \textbf{evidence for causality is not
reliable.} \textgreater{}

\newpage{}

\begin{figure}

\centering{

\includegraphics{v5-man-lmtp-ow-coop-church_files/figure-pdf/fig-3_3-1.pdf}

}

\caption{\label{fig-3_3}This figure reports the results of model
estimates for the causal effects of a universal loss of weekly religious
service vs.~status quo on voluntary help received from others during the
past week (yes/no) at the end of the study. Outcomes are expressed on
the risk ratio scale.}

\end{figure}%

\begin{longtable}[]{@{}
  >{\raggedright\arraybackslash}p{(\columnwidth - 10\tabcolsep) * \real{0.3000}}
  >{\raggedleft\arraybackslash}p{(\columnwidth - 10\tabcolsep) * \real{0.2286}}
  >{\raggedleft\arraybackslash}p{(\columnwidth - 10\tabcolsep) * \real{0.0857}}
  >{\raggedleft\arraybackslash}p{(\columnwidth - 10\tabcolsep) * \real{0.1000}}
  >{\raggedleft\arraybackslash}p{(\columnwidth - 10\tabcolsep) * \real{0.1143}}
  >{\raggedleft\arraybackslash}p{(\columnwidth - 10\tabcolsep) * \real{0.1714}}@{}}

\caption{\label{tbl-3_3}Table reports results of model estimates for the
causal effects of a universal loss of weekly religious service vs status
quo on voluntary help received from others during the past week (yes/no)
at the end of study. Outcomes are expressed on the risk ratio scale.}

\tabularnewline

\toprule\noalign{}
\begin{minipage}[b]{\linewidth}\raggedright
\end{minipage} & \begin{minipage}[b]{\linewidth}\raggedleft
E{[}Y(1){]}/E{[}Y(0){]}
\end{minipage} & \begin{minipage}[b]{\linewidth}\raggedleft
2.5 \%
\end{minipage} & \begin{minipage}[b]{\linewidth}\raggedleft
97.5 \%
\end{minipage} & \begin{minipage}[b]{\linewidth}\raggedleft
E\_Value
\end{minipage} & \begin{minipage}[b]{\linewidth}\raggedleft
E\_Val\_bound
\end{minipage} \\
\midrule\noalign{}
\endhead
\bottomrule\noalign{}
\endlastfoot
family gives time & 0.976 & 0.964 & 0.987 & 1.183 & 1.129 \\
friends gives time & 0.966 & 0.950 & 0.982 & 1.226 & 1.155 \\
community gives time & 0.897 & 0.866 & 0.930 & 1.473 & 1.360 \\

\end{longtable}

For the outcome `family gives time', the estimate is 0.976
{[}0.964,0.987{]}. The E-value for this effect estimate is 1.183, with a
lower bound of 1.129. In this scenario, assuming all modelled
covariates, an unmeasured confounder that exhibits an association with
both the treatment and outcome with a minimum strength (on the risk
ratio scale) of 1.129 could nullify the observed effect. However, the
effect would survive weaker unmeasured confounding. Here, we find
\textbf{evidence for causality}.

For the outcome `friends give time', the estimate is 0.966
{[}0.95,0.982{]}. The E-value for this effect estimate is 1.226 with a
lower bound of 1.155. In this scenario, assuming all modelled
covariates, an unmeasured confounder that exhibits an association with
both the treatment and outcome with a minimum strength (on the risk
ratio scale) of 1.155 could nullify the observed effect. However, the
effect would survive weaker unmeasured confounding. Here, we find
\textbf{evidence for causality}.

For the outcome `community gives time', the estimate is 0.897
{[}0.866,0.93{]}. The E-value for this effect estimate is 1.473 with a
lower bound of 1.36. In this scenario, assuming all modelled covariates,
an unmeasured confounder that exhibits an association with both the
treatment and outcome with a minimum strength (on the risk ratio scale)
of 1.36 could nullify the observed effect. However, the effect would
survive weaker unmeasured confounding. Here, we find \textbf{evidence
for causality}.

\newpage{}

\subsection{Study 4: Causal Effects of Regular Church Attendance on
Support Received From Others --
Money}\label{study-4-causal-effects-of-regular-church-attendance-on-support-received-from-others-money}

\begin{figure}

\centering{

\includegraphics{v5-man-lmtp-ow-coop-church_files/figure-pdf/fig-4_1-1.pdf}

}

\caption{\label{fig-4_1}This figure reports the results of model
estimates for the causal effects of a universal gain of weekly religious
service vs universal loss of weekly religious service on financial help
received from others during the past week (yes/no) at the end of the
study. Outcomes are expressed on the risk ratio scale.}

\end{figure}%

\begin{longtable}[]{@{}
  >{\raggedright\arraybackslash}p{(\columnwidth - 10\tabcolsep) * \real{0.3099}}
  >{\raggedleft\arraybackslash}p{(\columnwidth - 10\tabcolsep) * \real{0.2254}}
  >{\raggedleft\arraybackslash}p{(\columnwidth - 10\tabcolsep) * \real{0.0845}}
  >{\raggedleft\arraybackslash}p{(\columnwidth - 10\tabcolsep) * \real{0.0986}}
  >{\raggedleft\arraybackslash}p{(\columnwidth - 10\tabcolsep) * \real{0.1127}}
  >{\raggedleft\arraybackslash}p{(\columnwidth - 10\tabcolsep) * \real{0.1690}}@{}}

\caption{\label{tbl-4_1}This table reports the results of model
estimates for the causal effects of a universal gain of weekly religious
service vs universal loss of weekly religious service on financial help
received from others during the past week (yes/no) at the end of the
study. Outcomes are expressed on the risk ratio scale.}

\tabularnewline

\toprule\noalign{}
\begin{minipage}[b]{\linewidth}\raggedright
\end{minipage} & \begin{minipage}[b]{\linewidth}\raggedleft
E{[}Y(1){]}/E{[}Y(0){]}
\end{minipage} & \begin{minipage}[b]{\linewidth}\raggedleft
2.5 \%
\end{minipage} & \begin{minipage}[b]{\linewidth}\raggedleft
97.5 \%
\end{minipage} & \begin{minipage}[b]{\linewidth}\raggedleft
E\_Value
\end{minipage} & \begin{minipage}[b]{\linewidth}\raggedleft
E\_Val\_bound
\end{minipage} \\
\midrule\noalign{}
\endhead
\bottomrule\noalign{}
\endlastfoot
family gives money & 1.66 & 1.22 & 2.25 & 2.70 & 1.74 \\
friends give money & 2.05 & 1.32 & 3.18 & 3.52 & 1.98 \\
community gives money & 2.66 & 1.93 & 3.67 & 4.77 & 3.27 \\

\end{longtable}

For the outcome `community gives money', the estimate is 2.663
{[}1.932,3.671{]}. The E-value for this effect estimate is 4.767, with a
lower bound of 3.274. In this scenario, assuming all modelled
covariates, an unmeasured confounder that exhibits an association with
both the treatment and outcome with a minimum strength (on the risk
ratio scale) of 3.274 could nullify the observed effect. However, the
effect would survive weaker unmeasured confounding. Here, we find
\textbf{the evidence for causality is strong}.

For the outcome `friends give money', the estimate is 2.052
{[}1.323,3.184{]}. The E-value for this effect estimate is 3.521, with a
lower bound of 1.977. In this scenario, assuming all modelled
covariates, an unmeasured confounder that exhibits an association with
both the treatment and outcome with a minimum strength (on the risk
ratio scale) of 1.977 could nullify the observed effect. However, the
effect would survive weaker unmeasured confounding. Here, we find
\textbf{evidence for causality}.

For the outcome `family gives money', the estimate is 1.656
{[}1.219,2.249{]}. The E-value for this effect estimate is 2.698, with a
lower bound of 1.736. In this scenario, assuming all modelled
covariates, an unmeasured confounder that exhibits an association with
both the treatment and outcome with a minimum strength (on the risk
ratio scale) of 1.736 could nullify the observed effect. However, the
effect would survive weaker unmeasured confounding. Here, we find
\textbf{evidence for causality}.

\newpage{}

\begin{figure}

\centering{

\includegraphics{v5-man-lmtp-ow-coop-church_files/figure-pdf/fig-4_2-1.pdf}

}

\caption{\label{fig-4_2}Figure reports results of model estimates for
the causal effects of a universal gain of weekly religious service vs
status quo on financial help received from others during the past week
(yes/no) at the end of study. Outcomes are expressed on the risk ratio
scale.}

\end{figure}%

\begin{longtable}[]{@{}
  >{\raggedright\arraybackslash}p{(\columnwidth - 10\tabcolsep) * \real{0.3099}}
  >{\raggedleft\arraybackslash}p{(\columnwidth - 10\tabcolsep) * \real{0.2254}}
  >{\raggedleft\arraybackslash}p{(\columnwidth - 10\tabcolsep) * \real{0.0845}}
  >{\raggedleft\arraybackslash}p{(\columnwidth - 10\tabcolsep) * \real{0.0986}}
  >{\raggedleft\arraybackslash}p{(\columnwidth - 10\tabcolsep) * \real{0.1127}}
  >{\raggedleft\arraybackslash}p{(\columnwidth - 10\tabcolsep) * \real{0.1690}}@{}}

\caption{\label{tbl-4_2}This table reports the results of model
estimates for the causal effects of a universal gain of weekly religious
service vs status quo on financial help received from others during the
past week (yes/no) at the end of the study. Outcomes are expressed on
the risk ratio scale.}

\tabularnewline

\toprule\noalign{}
\begin{minipage}[b]{\linewidth}\raggedright
\end{minipage} & \begin{minipage}[b]{\linewidth}\raggedleft
E{[}Y(1){]}/E{[}Y(0){]}
\end{minipage} & \begin{minipage}[b]{\linewidth}\raggedleft
2.5 \%
\end{minipage} & \begin{minipage}[b]{\linewidth}\raggedleft
97.5 \%
\end{minipage} & \begin{minipage}[b]{\linewidth}\raggedleft
E\_Value
\end{minipage} & \begin{minipage}[b]{\linewidth}\raggedleft
E\_Val\_bound
\end{minipage} \\
\midrule\noalign{}
\endhead
\bottomrule\noalign{}
\endlastfoot
family gives money & 1.47 & 1.09 & 1.98 & 2.30 & 1.39 \\
friends give money & 1.72 & 1.13 & 2.62 & 2.83 & 1.51 \\
community gives money & 2.04 & 1.59 & 2.62 & 3.50 & 2.56 \\

\end{longtable}

For the outcome `community gives money', the estimate is 2.042
{[}1.59,2.623{]}. The E-value for this effect estimate is 3.501 with a
lower bound of 2.559. In this scenario, assuming all modelled
covariates, an unmeasured confounder that exhibits an association with
both the treatment and outcome with a minimum strength (on the risk
ratio scale) of 2.559 could nullify the observed effect. However, the
effect would survive weaker unmeasured confounding. Here, we find
\textbf{the evidence for causality is strong}.

For the outcome `friends give money', the estimate is 1.718
{[}1.129,2.616{]}. The E-value for this effect estimate is 2.829 with a
lower bound of 1.511. In this scenario, assuming all modelled
covariates, an unmeasured confounder that exhibits an association with
both the treatment and outcome with a minimum strength (on the risk
ratio scale) of 1.511 could nullify the observed effect. However, the
effect would survive weaker unmeasured confounding. Here, we find
\textbf{evidence for causality}.

For the outcome `family gives money', the estimate is 1.467
{[}1.086,1.981{]}. The E-value for this effect estimate is 2.295 with a
lower bound of 1.392. In this scenario, assuming all modelled
covariates, an unmeasured confounder that exhibits an association with
both the treatment and outcome with a minimum strength (on the risk
ratio scale) of 1.392.

\newpage{}

\begin{figure}

\centering{

\includegraphics{v5-man-lmtp-ow-coop-church_files/figure-pdf/fig-4_3-1.pdf}

}

\caption{\label{fig-4_3}Figure reports results of model estimates for
the causal effects of a universal loss of weekly religious service vs
status quo on financial help received from others during the past week
(yes/no) at the end of study. Outcomes are expressed on the risk ratio
scale.}

\end{figure}%

\begin{longtable}[]{@{}
  >{\raggedright\arraybackslash}p{(\columnwidth - 10\tabcolsep) * \real{0.3099}}
  >{\raggedleft\arraybackslash}p{(\columnwidth - 10\tabcolsep) * \real{0.2254}}
  >{\raggedleft\arraybackslash}p{(\columnwidth - 10\tabcolsep) * \real{0.0845}}
  >{\raggedleft\arraybackslash}p{(\columnwidth - 10\tabcolsep) * \real{0.0986}}
  >{\raggedleft\arraybackslash}p{(\columnwidth - 10\tabcolsep) * \real{0.1127}}
  >{\raggedleft\arraybackslash}p{(\columnwidth - 10\tabcolsep) * \real{0.1690}}@{}}

\caption{\label{tbl-4_3}Table reports results of model estimates for the
causal effects of a universal loss of weekly religious service vs status
quo on financial help received from others during the past week (yes/no)
at the end of study. Outcomes are expressed on the risk ratio scale.}

\tabularnewline

\toprule\noalign{}
\begin{minipage}[b]{\linewidth}\raggedright
\end{minipage} & \begin{minipage}[b]{\linewidth}\raggedleft
E{[}Y(1){]}/E{[}Y(0){]}
\end{minipage} & \begin{minipage}[b]{\linewidth}\raggedleft
2.5 \%
\end{minipage} & \begin{minipage}[b]{\linewidth}\raggedleft
97.5 \%
\end{minipage} & \begin{minipage}[b]{\linewidth}\raggedleft
E\_Value
\end{minipage} & \begin{minipage}[b]{\linewidth}\raggedleft
E\_Val\_bound
\end{minipage} \\
\midrule\noalign{}
\endhead
\bottomrule\noalign{}
\endlastfoot
family gives money & 0.89 & 0.86 & 0.91 & 1.51 & 1.43 \\
friends gives money & 0.84 & 0.78 & 0.89 & 1.68 & 1.48 \\
community gives money & 0.77 & 0.67 & 0.88 & 1.93 & 1.55 \\

\end{longtable}

For the outcome `family gives money', the estimate is 0.886
{[}0.863,0.91{]}. The E-value for this effect estimate is 1.51 with a
lower bound of 1.429. In this scenario, assuming all modelled
covariates, an unmeasured confounder that exhibits an association with
both the treatment and outcome with a minimum strength (on the risk
ratio scale) of 1.429 could nullify the observed effect. However, the
effect would survive weaker unmeasured confounding. Here, we find
\textbf{evidence for causality}.

For the outcome `friends gives money', the estimate is 0.837
{[}0.784,0.894{]}. The E-value for this effect estimate is 1.677, with a
lower bound of 1.483. In this scenario, assuming all modelled
covariates, an unmeasured confounder that exhibits an association with
both the treatment and outcome with a minimum strength (on the risk
ratio scale) of 1.483 could nullify the observed effect. However, the
effect would survive weaker unmeasured confounding. Here, we find
\textbf{evidence for causality}.

For the outcome `community gives money', the estimate is 0.767
{[}0.672,0.875{]}. The E-value for this effect estimate is 1.933 with a
lower bound of 1.547. In this scenario, assuming all modelled
covariates, an unmeasured confounder that exhibits an association with
both the treatment and outcome with a minimum strength (on the risk
ratio scale) of 1.547 could nullify the observed effect. However, the
effect would survive weaker unmeasured confounding. Here, we find
\textbf{evidence for causality}.

\subsection{Discussion}\label{discussion}

\subsubsection{Summary of findings}\label{summary-of-findings}

First, as expected, we observed that in theoretically relevant contrast,
ALL GAIN vs ALL LOSS had stronger effects than the policy-motivated
contrast ALL GAIN vs STATUS QUO. For example, whereas the measure of
social support revealed a causal effect for the theoretical
intervention, the policy-relevant intervention was unreliable. This
result suggests caution when inferring policy decisions on the results
from theoretically-motivated interventions.

Second, compared with the STATUS QUO, different policy motivations arise
for gaining and losing religious service attendance. In the first place,
the confidence intervals around loss expectations were narrower because
the underlying shifts involved required less counterfactual projection
across the population.

Third, perhaps most obviously, the measures themselves suggest different
interpretations of the importance of religion. For example, although
social connection measures did not generally reveal strong evidence for
causal influence from either the gain or loss of regular religious
service, except on social belonging in the ALL GAIN vs ALL LOSS
intervention, arguably, the concept of social connection does not
directly relate to prosociality. One may feel socially connected to
people without their help, and one can receive help from strangers
(\citeproc{ref-mccullough2020kindness}{McCullough 2020}). Nevertheless,
we include social connection measures to provide a clearer picture of
the multi-dimensional pro-social response phenotype. Notably, a long
tradition of reflection in the social scientific study of religion
implies that religion induces prosociality by collective effervescence,
or more recently, by postulated ego-fusion \& etc {[}{]}. Although
rituals may evoke these affective responses, whether and how charitable
behaviours and volunteering remain uncertain.

Fourth, we find corroborating evidence for the effects of religious
service attendance on charitable donations, measured by self-report data
and help-received responses. Presumably, much of this giving is to
religious institutions. Additional support for charitable giving comes
from public records, which in New Zealand must be reported because
religious institutions are charities that must report their earnings and
expenses {[}{]}. Whatever one may think of churches, the evidence from
New Zealand suggests they are efficient charities {[}{]}.

\section{Fifth, Daryl\ldots{}}\label{fifth-daryl}

\subsubsection{Economic value of regular religious
service}\label{economic-value-of-regular-religious-service}

We may use our results to compute the approximate economic value of
religious service attendance by comparing the sums of expected donations
across the New Zealand's adult population under the all-gain, all-loss,
and status quo.

We find that an increase in religious service attendance is expected to
result in an individual average donation rate of 1553.3262405,
Conversely, a decrease in religious service attendance is expected to an
average charitable donation rate of 725.145593. The status quo is
expected to yield an average donation rate of 925.6217757. The number of
adult residents living in New Zealand in 2021 was
\ensuremath{3.989\times 10^{6}},\footnote{\url{https://www.stats.govt.nz/information-releases/national-population-estimates-at-30-june-2021}}
Multiplying this number by the estimated rates suggests the net gain to
charity of country-wide religious service attendance compared with the
status quo is \ensuremath{2.5039131\times 10^{9}}. By contrast the net
cost of a country-wide loss of regular religious service attendance on
charitable donations is: \ensuremath{-7.9969949\times 10^{8}}.

Putting these figures in context, the government of New Zealand's annual
budget in 2021 was \ensuremath{5.7976\times 10^{10}}.\footnote{\url{https://www.treasury.govt.nz/publications/budgets/budget-2021}}.
The expected gain from the country-wide adoption of religious service is
0.0431888 of New Zealand's annual government budget in 2021 The expected
loss from the country-wide loss of all religious service amounts to
0.0137936 of New Zealand's annual government budget in 2021.

We infer that the declining religious service in New Zealand affects
charitable giving, however these costs sum to about 0.32 the opportunity
cost of hypothetical New Zealand in which the population attended
religious service regularly.

\subsubsection{Note of concern about the use of cross-sectional
regression models to investigate the social consequences of
religion}\label{note-of-concern-about-the-use-of-cross-sectional-regression-models-to-investigate-the-social-consequences-of-religion}

How do these results compare to cross-sectional regression?

We performed a cross-sectional analysis in which we estimated the
relationship between religious service attendance and charitable
donations. We included all regression covariates from the causal models
(including sample weights), except the baseline outcome. The coefficient
for religious service attendance in this model, which represents the
change in expectation from a one-unit change in the predictor, is b =
432; (95\% CI 399, 465). To obtain the monthly rate we multiplied this
estimate by 4.2 to obtain a prediction of 1813.3319878. This amount is
2.19 than we infer from the ALL-GAIN vs ALL LOSS causal model.

Recalled we did not find reliable evidence for self-reported
volunteering from gains in religious service attendance. However, the
coefficients we obtained for volunteering in a cross-sectional model
were estimated to be reliable at b = 0.25; (95\% CI 0.22, 0.28). To
obtain the monthly rate we multiplied this estimate by 4.2 and by 60 to
obtain a prediction of 63.0975753 minutes. This amount is 2.84 than we
infer from the ALL-GAIN vs ALL LOSS causal model.

The results underscore the importance of remembering that
cross-sectional regressions are unreliable for estimating causal
effects.

\subsubsection{Assumptions and
limitations}\label{assumptions-and-limitations}

\begin{enumerate}
\def\labelenumi{\arabic{enumi}.}
\item
  \textbf{Causality and confounding}: we employ rigorous causal
  inference techniques, but these are contingent on the assumption of no
  unmeasured confounding. Although our sensitivity analyses allow us to
  describe the amount of unmeasured confounding needed to explain away
  our results, it is unclear whether such confounders are present and,
  if so, how strongly they may be related to the treatments and
  outcomes. (to say more\ldots)
\item
  \textbf{Treatment effect heterogeneity}: even if we assume that the
  results are independent of the variations in treatments observed,
  because ``religious service attendance'' is not homogenous, we cannot
  give a precise interpretation of the results. Future work should
  investigate heterogeneity in treatment effects reported here.
\item
  \textbf{Measurement error}: Direct and/or correlated measurement
  errors may lead to bias by implying true effects in their absence.
  Additionally, uncorrelated measurement errors may lead to bias by
  attenuating true results in their presence. To some extent, our method
  for evaluating causation across multiple measures of ``prosociality''
  addresses these threats. However, variability in results may arise
  from presently unknown combinations of measurement error\ldots. (say
  more)
\item
  \textbf{Generalisability and transportability}: our findings should be
  interpreted within the context of the New Zealand population from
  which the data were sourced and weighted. Although these results may
  have broader relevance to similar populations, we advise against
  direct extrapolation to different populations or sociocultural
  settings.
\end{enumerate}

\newpage{}

\subsubsection{Ethics}\label{ethics}

The University of Auckland Human Participants Ethics Committee reviews
the NZAVS every three years. Our most recent ethics approval statement
is as follows: The New Zealand Attitudes and Values Study was approved
by the University of Auckland Human Participants Ethics Committee on
26/05/2021 for six years until 26/05/2027, Reference Number UAHPEC22576.

\subsubsection{Acknowledgements}\label{acknowledgements}

The New Zealand Attitudes and Values Study is supported by a grant from
the Templeton Religious Trust (TRT0196; TRT0418). JB received support
from the Max Planck Institute for the Science of Human History. The
funders had no role in preparing the manuscript or the decision to
publish.

\subsubsection{Author Statement}\label{author-statement}

\begin{itemize}
\tightlist
\item
  JB conceived of the study and analytic approach.\\
\item
  CS led NZAVS data collection.
\item
  All authors contributed to the manuscript.
\end{itemize}

\newpage{}

\subsection{Appendix A: Measures}\label{appendix-measures}

\paragraph{Age (waves: 1-15)}\label{age-waves-1-15}

We asked participants' ages in an open-ended question (``What is your
age?'' or ``What is your date of birth'').

\paragraph{Charitable Donations (Study 1
outcome)}\label{charitable-donations-study-1-outcome}

Using one item from Hoverd and Sibley
(\citeproc{ref-hoverd_religious_2010}{2010}), we asked participants
``How much money have you donated to charity in the last year?''.

\paragraph{Charitable Volunteering (Study 1
outcome)}\label{charitable-volunteering-study-1-outcome}

We measured hours of volunteering using one item from Sibley \emph{et
al.} (\citeproc{ref-sibley2011}{2011}): ``Hours spent \ldots{}
voluntary/charitable work.''

\paragraph{Community: Sense of Community (Study 2
outcome)}\label{community-sense-of-community-study-2-outcome}

We measured a sense of community with a single item from Sengupta
\emph{et al.} (\citeproc{ref-sengupta2013}{2013}): ``I feel a sense of
community with others in my local neighbourhood.'' Participants answered
on a scale of 1 (strongly disagree) to 7 (strongly agree).

\paragraph{Community: Felt Belongingness (Study 2
outcome)}\label{community-felt-belongingness-study-2-outcome}

We assessed felt belongingness with three items adapted from the Sense
of Belonging Instrument (\citeproc{ref-hagerty1995}{Hagerty and Patusky
1995}): (1) ``Know that people in my life accept and value me''; (2)
``Feel like an outsider''; (3) ``Know that people around me share my
attitudes and beliefs''. Participants responded on a scale from 1 (Very
Inaccurate) to 7 (Very Accurate). The second item was reversely coded.

\paragraph{Community: Social Support (Study 2
outcome)}\label{community-social-support-study-2-outcome}

Participants' perceived social support was measured using three items
from Cutrona and Russell (\citeproc{ref-cutrona1987}{1987}) and Williams
\emph{et al.} (\citeproc{ref-williams_cyberostracism_2000}{2000}): (1)
``There are people I can depend on to help me if I really need it''; (2)
``There is no one I can turn to for guidance in times of stress''; (3)
``I know there are people I can turn to when I need help.'' Participants
indicated the extent to which they agreed with those items (1 = Strongly
Disagree to 7 = Strongly Agree). The second item was negatively worded,
so we reversely recorded the responses to this item.

\paragraph{Disability}\label{disability}

We assessed disability with a one item indicator adapted from Verbrugge
(\citeproc{ref-verbrugge1997}{1997}), that asks ``Do you have a health
condition or disability that limits you, and that has lasted for 6+
months?'' (1 = Yes, 0 = No).

\paragraph{Education Attainment (waves: 1,
4-15)}\label{education-attainment-waves-1-4-15}

We asked participants, ``What is your highest level of qualification?''.
We coded participants' highest finished degree according to the New
Zealand Qualifications Authority. Ordinal-Rank 0-10 NZREG codes (with
overseas school quals coded as Level 3, and all other ancillary
categories coded as missing)
See:https://www.nzqa.govt.nz/assets/Studying-in-NZ/New-Zealand-Qualification-Framework/requirements-nzqf.pdf

\paragraph{Employment (waves: 1-3,
4-11)}\label{employment-waves-1-3-4-11}

We asked participants, ``Are you currently employed? (This includes
self-employed or casual work)''. * note: This question disappeared in
the updated NZAVS Technical documents (Data Dictionary).

\paragraph{Ethnicity}\label{ethnicity}

Based on the New Zealand Census, we asked participants, ``Which ethnic
group(s) do you belong to?''. The responses were: (1) New Zealand
European; (2) Māori; (3) Samoan; (4) Cook Island Māori; (5) Tongan; (6)
Niuean; (7) Chinese; (8) Indian; (9) Other such as DUTCH, JAPANESE,
TOKELAUAN. Please state:. We coded their answers into four groups:
Maori, Pacific, Asian, and Euro (except for Time 3, which used an
open-ended measure).

\paragraph{Fatigue}\label{fatigue}

We assessed subjective fatigue by asking participants, ``During the last
30 days, how often did \ldots{} you feel exhausted?'' Responses were
collected on an ordinal scale (0 = None of The Time, 1 = A little of The
Time, 2 = Some of The Time, 3 = Most of The Time, 4 = All of The Time).

\paragraph{Gender (waves: 1-15)}\label{gender-waves-1-15}

We asked participants' gender in an open-ended question: ``what is your
gender?'' or ``Are you male or female?'' (waves: 1-5). Female was coded
as 0, Male was coded as 1, and gender diverse coded as 3
(\citeproc{ref-fraser_coding_2020}{Fraser \emph{et al.} 2020}). (or 0.5
= neither female nor male)

Here, we coded all those who responded as Male as 1, and those who did
not as 0.

\paragraph{Honesty-Humility-Modesty Facet (waves:
10-14)}\label{honesty-humility-modesty-facet-waves-10-14}

Participants indicated the extent to which they agree with the following
four statements from Campbell \emph{et al.}
(\citeproc{ref-campbell2004}{2004}) , and Sibley \emph{et al.}
(\citeproc{ref-sibley2011}{2011}) (1 = Strongly Disagree to 7 = Strongly
Agree)

\begin{verbatim}
i.  I want people to know that I am an important person of high status, (Waves: 1, 10-14)
ii. I am an ordinary person who is no better than others.
iii. I wouldn't want people to treat me as though I were superior to them.
iv. I think that I am entitled to more respect than the average person is.
\end{verbatim}

\paragraph{Has Siblings}\label{has-siblings}

``Do you have siblings?'' (\citeproc{ref-stronge2019onlychild}{Stronge
\emph{et al.} 2019})

\paragraph{Hours of Childcare}\label{hours-of-childcare}

We measured hours of exercising using one item from Sibley \emph{et al.}
(\citeproc{ref-sibley2011}{2011}): 'Hours spent \ldots{} looking after
children.''

To stabilise this indicator, we took the natural log of the response +
1.

\paragraph{Hours of Housework}\label{hours-of-housework}

We measured hours of exercising using one item from Sibley \emph{et al.}
(\citeproc{ref-sibley2011}{2011}): ``Hours spent \ldots{}
housework/cooking''

To stabilise this indicator, we took the natural log of the response +
1.

\paragraph{Hours of Exercise}\label{hours-of-exercise}

We measured hours of exercising using one item from Sibley \emph{et al.}
(\citeproc{ref-sibley2011}{2011}): ``Hours spent \ldots{}
exercising/physical activity''

To stabilise this indicator, we took the natural log of the response +
1.

\paragraph{Hours of Childcare}\label{hours-of-childcare-1}

We measured hours of exercising using one item from Sibley \emph{et al.}
(\citeproc{ref-sibley2011}{2011}): 'Hours spent \ldots{} looking after
children.''

To stabilise this indicator, we took the natural log of the response +
1.

\paragraph{Hours of Exercise}\label{hours-of-exercise-1}

We measured hours of exercising using one item from Sibley \emph{et al.}
(\citeproc{ref-sibley2011}{2011}): ``Hours spent \ldots{}
exercising/physical activity''

To stabilise this indicator, we took the natural log of the response +
1.

\paragraph{Hours of Housework}\label{hours-of-housework-1}

We measured hours of exercising using one item from Sibley \emph{et al.}
(\citeproc{ref-sibley2011}{2011}): ``Hours spent \ldots{}
housework/cooking''

To stabilise this indicator, we took the natural log of the response +
1.

\paragraph{Hours of Sleep}\label{hours-of-sleep}

Participants were asked ``During the past month, on average, how many
hours of \emph{actual sleep} did you get per night''.

\paragraph{Socialising (time 10, time
11)}\label{socialising-time-10-time-11}

As part of the time usage measures, participants were asked to state how
many hour the spent in activities related to socialising with the
following groups:

\begin{itemize}
\tightlist
\item
  Hours spent \ldots{} socialising with family
\item
  Hours spent \ldots{} socialising with friends
\item
  Hours spent \ldots{} socialising with community groups
\end{itemize}

\paragraph{Hours of Work}\label{hours-of-work}

We measured hours of work using one item from Sibley \emph{et al.}
(\citeproc{ref-sibley2011}{2011}):``Hours spent \ldots{} working in paid
employment.''

To stabilise this indicator, we took the natural log of the response +
1.

\paragraph{Income (waves: 1-3, 4-15)}\label{income-waves-1-3-4-15}

Participants were asked ``Please estimate your total household income
(before tax) for the year XXXX''. To stabilise this indicator, we first
took the natural log of the response + 1, and then centred and
standardised the log-transformed indicator.

\paragraph{Living in an Urban Area (waves:
1-15)}\label{living-in-an-urban-area-waves-1-15}

We coded whether they are living in an urban or rural area (1 = Urban, 0
= Rural) based on the addresses provided.

We coded whether they were living in an urban or rural area (1 = Urban,
0 = Rural) based on the addresses provided.

\paragraph{Mini-IPIP 6 (waves:
1-3,4-15)}\label{mini-ipip-6-waves-1-34-15}

We measured participants' personalities with the Mini International
Personality Item Pool 6 (Mini-IPIP6) (\citeproc{ref-sibley2011}{Sibley
\emph{et al.} 2011}), which consists of six dimensions and each
dimension is measured with four items:

\begin{enumerate}
\def\labelenumi{\arabic{enumi}.}
\item
  agreeableness,

  \begin{enumerate}
  \def\labelenumii{\roman{enumii}.}
  \tightlist
  \item
    I sympathize with others' feelings.
  \item
    I am not interested in other people's problems. (r)
  \item
    I feel others' emotions.
  \item
    I am not really interested in others. (r)
  \end{enumerate}
\item
  conscientiousness,

  \begin{enumerate}
  \def\labelenumii{\roman{enumii}.}
  \tightlist
  \item
    I get chores done right away.
  \item
    I like order.
  \item
    I make a mess of things. (r)
  \item
    I often forget to put things back in their proper place. (r)
  \end{enumerate}
\item
  extraversion,

  \begin{enumerate}
  \def\labelenumii{\roman{enumii}.}
  \tightlist
  \item
    I am the life of the party.
  \item
    I don't talk a lot. (r)
  \item
    I keep in the background. (r)
  \item
    I talk to a lot of different people at parties.
  \end{enumerate}
\item
  honesty-humility,

  \begin{enumerate}
  \def\labelenumii{\roman{enumii}.}
  \tightlist
  \item
    I feel entitled to more of everything. (r)
  \item
    I deserve more things in life. (r)
  \item
    I would like to be seen driving around in a very expensive car. (r)
  \item
    I would get a lot of pleasure from owning expensive luxury goods.
    (r)
  \end{enumerate}
\item
  neuroticism, and

  \begin{enumerate}
  \def\labelenumii{\roman{enumii}.}
  \tightlist
  \item
    I have frequent mood swings.
  \item
    I am relaxed most of the time. (r)
  \item
    I get upset easily.
  \item
    I seldom feel blue. (r)
  \end{enumerate}
\item
  openness to experience

  \begin{enumerate}
  \def\labelenumii{\roman{enumii}.}
  \tightlist
  \item
    I have a vivid imagination.
  \item
    I have difficulty understanding abstract ideas. (r)
  \item
    I do not have a good imagination. (r)
  \item
    I am not interested in abstract ideas. (r)
  \end{enumerate}
\end{enumerate}

Each dimension was assessed with four items and participants rated the
accuracy of each item as it applies to them from 1 (Very Inaccurate) to
7 (Very Accurate). Items marked with (r) are reverse coded.

\paragraph{NZ-Born (waves: 1-2,4-15)}\label{nz-born-waves-1-24-15}

We asked participants, ``Which country were you born in?'' or ``Where
were you born? (please be specific, e.g., which town/city?)'' (waves:
6-15).

\paragraph{NZ Deprivation Index (waves:
1-15)}\label{nz-deprivation-index-waves-1-15}

We used the NZ Deprivation Index to assign each participant a score
based on where they live (\citeproc{ref-atkinson2019}{Atkinson \emph{et
al.} 2019}). This score combines data such as income, home ownership,
employment, qualifications, family structure, housing, and access to
transport and communication for an area into one deprivation score.

\paragraph{Opt-in}\label{opt-in}

The New Zealand Attitudes and Values Study allows opt-ins to the study.
Because the opt-in population may differ from those sampled randomly
from the New Zealand electoral roll; although the opt-in rate is low, we
include an indicator (yes/no) for this variable.

\paragraph{NZSEI Occupational Prestige and Status (waves:
8-15)}\label{nzsei-occupational-prestige-and-status-waves-8-15}

We assessed occupational prestige and status using the New Zealand
Socio-economic Index 13 (NZSEI-13) (\citeproc{ref-fahy2017a}{Fahy
\emph{et al.} 2017a}). This index uses the income, age, and education of
a reference group, in this case the 2013 New Zealand census, to
calculate a score for each occupational group. Scores range from 10
(Lowest) to 90 (Highest). This list of index scores for occupational
groups was used to assign each participant a NZSEI-13 score based on
their occupation.

We assessed occupational prestige and status using the New Zealand
Socio-economic Index 13 (NZSEI-13) (\citeproc{ref-fahy2017}{Fahy
\emph{et al.} 2017b}). This index uses the income, age, and education of
a reference group, in this case, the 2013 New Zealand census, to
calculate a score for each occupational group. Scores range from 10
(Lowest) to 90 (Highest). This list of index scores for occupational
groups was used to assign each participant a NZSEI-13 score based on
their occupation.

\paragraph{Number of Children (waves: 1-3,
4-15)}\label{number-of-children-waves-1-3-4-15}

We measured the number of children using one item from Bulbulia
(\citeproc{ref-Bulbulia_2015}{2015}). We asked participants, ``How many
children have you given birth to, fathered, or adopted. How many
children have you given birth to, fathered, or adopted?'' or ````How
many children have you given birth to, fathered, or adopted. How many
children have you given birth to, fathered, and/or parented?'' (waves:
12-15).

\paragraph{Partner: Has}\label{partner-has}

``What is your relationship status?'' (e.g., single, married, de-facto,
civil union, widowed, living together, etc.)

\paragraph{Politically Conservative}\label{politically-conservative}

We measured participants' political conservative orientation using a
single item adapted from Jost (\citeproc{ref-jost_end_2006-1}{2006}).

``Please rate how politically liberal versus conservative you see
yourself as being.''

(1 = Extremely Liberal to 7 = Extremely Conservative)

\paragraph{Politically Right Wing}\label{politically-right-wing}

We measured participants' political right-wing orientation using a
single item adapted from Jost (\citeproc{ref-jost_end_2006-1}{2006}).

``Please rate how politically left-wing versus right-wing you see
yourself as being..''

(1 = Extremely left-wing to 7 = Extremely right-wing)

\paragraph{Short-Form Health}\label{short-form-health}

Participants' subjective health was measured using one item (``Do you
have a health condition or disability that limits you, and that has
lasted for 6+ months?''; 1 = Yes, 0 = No) adapted from Verbrugge
(\citeproc{ref-verbrugge1997}{1997}).

\paragraph{Support received: money (waves 10-13) (Study 4
outcomes)}\label{support-received-money-waves-10-13-study-4-outcomes}

The NZAVS has a `revealed' measure of received help and support measured
in hours of support in the previous week. The items are:

Received help and support - money - family - friends - others in my
community

\paragraph{Support received: time (waves 10-13) (Study 3
outcomes)}\label{support-received-time-waves-10-13-study-3-outcomes}

The NZAVS has a `revealed' measure of received help and support measured
in hours of support in the previous week. The items are:

Received help and support - hours - family - friends - others in my
community

\paragraph{Total Siblings}\label{total-siblings}

Participants were asked the following questions related to sibling
counts:

\begin{itemize}
\tightlist
\item
  Were you the 1st born, 2nd born, or 3rd born, etc, child of your
  mother?
\item
  Do you have siblings?
\item
  How many older sisters do you have?
\item
  How many younger sisters do you have?
\item
  How many older brothers do you have?
\item
  How many younger brothers do you have?
\end{itemize}

A single score was obtained from sibling counts by summing responses to
the ``How many\ldots{}'' items. From these scores an ordered factor was
created ranging from 0 to 7, where participants with more than 7
siblings were grouped into the highest category.

\subsection{Appendix B. Baseline Demographic
Statistics}\label{appendix-demographics}

\begin{table}

\caption{\label{tbl-B}}

\centering{

\captionsetup{labelsep=none}

}

\end{table}%

\begin{longtable}[]{@{}ll@{}}
\caption{Baseline demographic
statistics}\label{tbl-table-demography}\tabularnewline
\toprule\noalign{}
\textbf{Exposure + Demographic Variables} & \textbf{N = 47,948} \\
\midrule\noalign{}
\endfirsthead
\toprule\noalign{}
\textbf{Exposure + Demographic Variables} & \textbf{N = 47,948} \\
\midrule\noalign{}
\endhead
\bottomrule\noalign{}
\endlastfoot
\textbf{Age} & NA \\
Mean (SD) & 49 (14) \\
Range & 18, 99 \\
IQR & 39, 60 \\
\textbf{Agreeableness} & NA \\
Mean (SD) & 5.35 (0.99) \\
Range & 1.00, 7.00 \\
IQR & 4.75, 6.00 \\
Unknown & 452 \\
\textbf{Alert Level Combined Lead} & NA \\
no\_alert & 24,789 (71\%) \\
early\_covid & 3,902 (11\%) \\
alert\_level\_1 & 3,010 (8.7\%) \\
alert\_level\_2 & 865 (2.5\%) \\
alert\_level\_2\_5\_3 & 569 (1.6\%) \\
alert\_level\_4 & 1,648 (4.7\%) \\
Unknown & 13,165 \\
\textbf{Born Nz} & 37,082 (78\%) \\
Unknown & 585 \\
\textbf{Children Num} & NA \\
Mean (SD) & 1.74 (1.46) \\
Range & 0.00, 14.00 \\
IQR & 0.00, 3.00 \\
Unknown & 393 \\
\textbf{Conscientiousness} & NA \\
Mean (SD) & 5.11 (1.06) \\
Range & 1.00, 7.00 \\
IQR & 4.50, 6.00 \\
Unknown & 443 \\
\textbf{Education Level Coarsen} & NA \\
no\_qualification & 1,239 (2.7\%) \\
cert\_1\_to\_4 & 16,700 (36\%) \\
cert\_5\_to\_6 & 5,969 (13\%) \\
university & 12,595 (27\%) \\
post\_grad & 5,118 (11\%) \\
masters & 3,924 (8.4\%) \\
doctorate & 1,125 (2.4\%) \\
Unknown & 1,278 \\
\textbf{Employed} & 38,024 (79\%) \\
Unknown & 107 \\
\textbf{Eth Cat} & NA \\
euro & 38,158 (81\%) \\
maori & 5,457 (12\%) \\
pacific & 1,137 (2.4\%) \\
asian & 2,543 (5.4\%) \\
Unknown & 653 \\
\textbf{Extraversion} & NA \\
Mean (SD) & 3.91 (1.20) \\
Range & 1.00, 7.00 \\
IQR & 3.00, 4.75 \\
Unknown & 443 \\
\textbf{Hlth Disability} & 10,540 (22\%) \\
Unknown & 926 \\
\textbf{Hlth Fatigue} & NA \\
0 & 7,326 (15\%) \\
1 & 15,242 (32\%) \\
2 & 14,803 (31\%) \\
3 & 7,448 (16\%) \\
4 & 2,572 (5.4\%) \\
Unknown & 557 \\
\textbf{Hlth Sleep Hours} & NA \\
Mean (SD) & 6.94 (1.13) \\
Range & 2.50, 16.00 \\
IQR & 6.00, 8.00 \\
Unknown & 2,482 \\
\textbf{Honesty Humility} & NA \\
Mean (SD) & 5.41 (1.19) \\
Range & 1.00, 7.00 \\
IQR & 4.75, 6.25 \\
Unknown & 448 \\
\textbf{Hours Children log} & NA \\
Mean (SD) & 1.17 (1.61) \\
Range & 0.00, 5.13 \\
IQR & 0.00, 2.40 \\
Unknown & 1,538 \\
\textbf{Hours Community log} & NA \\
Mean (SD) & 0.34 (0.65) \\
Range & 0.00, 4.80 \\
IQR & 0.00, 0.69 \\
Unknown & 1,538 \\
\textbf{Hours Exercise log} & NA \\
Mean (SD) & 1.54 (0.85) \\
Range & 0.00, 4.39 \\
IQR & 1.10, 2.08 \\
Unknown & 1,538 \\
\textbf{Hours Family log} & NA \\
Mean (SD) & 1.61 (1.04) \\
Range & 0.00, 5.13 \\
IQR & 1.10, 2.40 \\
Unknown & 1,538 \\
\textbf{Hours Friends log} & NA \\
Mean (SD) & 1.45 (0.87) \\
Range & 0.00, 5.02 \\
IQR & 1.10, 1.95 \\
Unknown & 1,538 \\
\textbf{Hours Housework log} & NA \\
Mean (SD) & 2.14 (0.78) \\
Range & 0.00, 5.13 \\
IQR & 1.61, 2.71 \\
Unknown & 1,538 \\
\textbf{Hours Religious Community log} & NA \\
Mean (SD) & 0.17 (0.49) \\
Range & 0.00, 5.04 \\
IQR & 0.00, 0.00 \\
Unknown & 1,538 \\
\textbf{Hours Work log} & NA \\
Mean (SD) & 2.66 (1.58) \\
Range & 0.00, 4.62 \\
IQR & 1.39, 3.71 \\
Unknown & 1,538 \\
\textbf{Household Inc log} & NA \\
Mean (SD) & 11.39 (0.77) \\
Range & 0.69, 14.92 \\
IQR & 11.00, 11.92 \\
Unknown & 3,726 \\
\textbf{Kessler6 Sum} & NA \\
Mean (SD) & 5 (4) \\
Range & 0, 24 \\
IQR & 2, 8 \\
Unknown & 489 \\
\textbf{Male} & 17,779 (37\%) \\
\textbf{Modesty} & NA \\
Mean (SD) & 5.98 (0.95) \\
Range & 1.00, 7.00 \\
IQR & 5.50, 6.75 \\
Unknown & 34 \\
\textbf{Neuroticism} & NA \\
Mean (SD) & 3.49 (1.15) \\
Range & 1.00, 7.00 \\
IQR & 2.75, 4.25 \\
Unknown & 454 \\
\textbf{Nz Dep2018} & NA \\
Mean (SD) & 4.77 (2.73) \\
Range & 1.00, 10.00 \\
IQR & 2.00, 7.00 \\
Unknown & 310 \\
\textbf{Nzsei 13 l} & NA \\
Mean (SD) & 54 (17) \\
Range & 10, 90 \\
IQR & 41, 69 \\
Unknown & 676 \\
\textbf{Openness} & NA \\
Mean (SD) & 4.96 (1.12) \\
Range & 1.00, 7.00 \\
IQR & 4.25, 5.75 \\
Unknown & 445 \\
\textbf{Partner} & 34,818 (75\%) \\
Unknown & 1,539 \\
\textbf{Political Conservative} & NA \\
1 & 2,515 (5.6\%) \\
2 & 8,676 (19\%) \\
3 & 8,820 (20\%) \\
4 & 13,913 (31\%) \\
5 & 6,694 (15\%) \\
6 & 3,299 (7.4\%) \\
7 & 741 (1.7\%) \\
Unknown & 3,290 \\
\textbf{Religion Church Round} & NA \\
0 & 38,479 (83\%) \\
1 & 1,517 (3.3\%) \\
2 & 1,125 (2.4\%) \\
3 & 907 (2.0\%) \\
4 & 2,478 (5.3\%) \\
5 & 475 (1.0\%) \\
6 & 326 (0.7\%) \\
7 & 106 (0.2\%) \\
8 & 964 (2.1\%) \\
Unknown & 1,571 \\
\textbf{Rural Gch 2018 l} & NA \\
1 & 29,479 (62\%) \\
2 & 9,014 (19\%) \\
3 & 5,865 (12\%) \\
4 & 2,694 (5.7\%) \\
5 & 588 (1.2\%) \\
Unknown & 308 \\
\textbf{Sample Frame Opt in} & 1,384 (2.9\%) \\
\textbf{Sample Origin} & NA \\
1-2 & 2,970 (6.2\%) \\
3-3.5 & 2,052 (4.3\%) \\
4 & 2,700 (5.6\%) \\
5-6-7 & 4,506 (9.4\%) \\
8-9 & 5,802 (12\%) \\
10 & 29,918 (62\%) \\
\textbf{Short Form Health} & NA \\
Mean (SD) & 5.04 (1.17) \\
Range & 1.00, 7.00 \\
IQR & 4.33, 6.00 \\
Unknown & 9 \\
\textbf{Total Siblings} & NA \\
Mean (SD) & 2.55 (1.87) \\
Range & 0.00, 23.00 \\
IQR & 1.00, 3.00 \\
Unknown & 1,205 \\
\textbf{Urban} & 38,493 (81\%) \\
Unknown & 308 \\
\end{longtable}

Table~\ref{tbl-table-demography} presents baseline demographic
statistics for couples who met inclusion criteria.

\subsection{Appendix C: Treatment Statistics}\label{appendix-exposures}

\begin{longtable}[]{@{}
  >{\raggedright\arraybackslash}p{(\columnwidth - 4\tabcolsep) * \real{0.4247}}
  >{\raggedright\arraybackslash}p{(\columnwidth - 4\tabcolsep) * \real{0.2877}}
  >{\raggedright\arraybackslash}p{(\columnwidth - 4\tabcolsep) * \real{0.2877}}@{}}
\caption{Baseline and treatment wave descriptive
statistics}\label{tbl-table-exposures}\tabularnewline
\toprule\noalign{}
\begin{minipage}[b]{\linewidth}\raggedright
\textbf{Exposure Variables by Wave}
\end{minipage} & \begin{minipage}[b]{\linewidth}\raggedright
\textbf{2018}, N = 47,948
\end{minipage} & \begin{minipage}[b]{\linewidth}\raggedright
\textbf{2019}, N = 47,948
\end{minipage} \\
\midrule\noalign{}
\endfirsthead
\toprule\noalign{}
\begin{minipage}[b]{\linewidth}\raggedright
\textbf{Exposure Variables by Wave}
\end{minipage} & \begin{minipage}[b]{\linewidth}\raggedright
\textbf{2018}, N = 47,948
\end{minipage} & \begin{minipage}[b]{\linewidth}\raggedright
\textbf{2019}, N = 47,948
\end{minipage} \\
\midrule\noalign{}
\endhead
\bottomrule\noalign{}
\endlastfoot
\textbf{Religion Church Round} & NA & NA \\
0 & 38,479 (83\%) & 28,671 (84\%) \\
1 & 1,517 (3.3\%) & 916 (2.7\%) \\
2 & 1,125 (2.4\%) & 751 (2.2\%) \\
3 & 907 (2.0\%) & 651 (1.9\%) \\
4 & 2,478 (5.3\%) & 1,700 (5.0\%) \\
5 & 475 (1.0\%) & 312 (0.9\%) \\
6 & 326 (0.7\%) & 209 (0.6\%) \\
7 & 106 (0.2\%) & 71 (0.2\%) \\
8 & 964 (2.1\%) & 655 (1.9\%) \\
Unknown & 1,571 & 14,012 \\
\textbf{Alert Level Combined} & NA & NA \\
no\_alert & 47,948 (100\%) & 24,789 (71\%) \\
early\_covid & 0 (0\%) & 3,902 (11\%) \\
alert\_level\_1 & 0 (0\%) & 3,010 (8.7\%) \\
alert\_level\_2 & 0 (0\%) & 865 (2.5\%) \\
alert\_level\_2\_5\_3 & 0 (0\%) & 569 (1.6\%) \\
alert\_level\_4 & 0 (0\%) & 1,648 (4.7\%) \\
Unknown & 0 & 13,165 \\
\end{longtable}

Table~\ref{tbl-table-exposures} presents baseline (NZAVS time 10) and
exposure wave (NZAVS time 11) statistics for the exposure variable:
religious service attendance (range 0-8). Responses coded as eight or
above were coded as ``8''. This decision to avoid spare treatments was
based on theoretical grounds, namely, that daily exposure would be
similar in its effects to more than daily exposure. We note that causal
contrasts were obtained for projects at either no attendance or
four-or-more visits per month. Hence this simplification of the measure
is unlikely to affect theoretical and practical inferences. Because the
treatment wave (NZAVS time 11) occurred in New Zealand's COVID-19
pandemic, all models adjusted for the pandemic alert-level. The pandemic
is not a ``confounder'' because a confounder must be related to the
treatment and the outcome. At the end of the study, all participants had
been exposed to the pandemic. However, to satisfy the causal consistency
assumption, all treatments must be conditionally equivalent within
levels of all covariates (\citeproc{ref-vanderweele2013}{VanderWeele and
Hernan 2013}). Because COVID affected the ability or willingness of
individuals to attend religious service, we included the lockdown
condition as a covariate (\citeproc{ref-sibley2012a}{Sibley and Bulbulia
2012}). To better enable conditional independence within levels of the
treatment variable, we conditioned on the lead value of COVID-alert
level at baseline. To mitigate systematic biases arising from attrition,
and missingness, the \texttt{lmtp} package uses inverse probability of
censoring weights, which were used when estimating the causal effects of
the exposure on the outcome.

\subsubsection{Binary Transition Table for The
Treatment}\label{binary-transition-table-for-the-treatment}

The table presents a transition matrix to evaluate treatment shifts
between baseline and treatment wave. Here we focus on the shift from/to
monthly attendance at four or more visits per month. Entries along the
diagonal (in bold) indicate the number of individuals who
\textbf{stayed} in their initial state. By contrast, the off-diagonal
shows the transitions from the initial state (bold) to another state in
the following wave (off diagonal). Thus the cell located at the
intersection of row \(i\) and column \(j\), where \(i \neq j\), gives us
the counts of individuals moving from state \(i\) to state \(j\).

\begin{longtable}[]{@{}ccc@{}}
\toprule\noalign{}
From & \textless{} weekly & \textgreater= weekly \\
\midrule\noalign{}
\endhead
\bottomrule\noalign{}
\endlastfoot
\textless{} weekly & \textbf{29496} & 669 \\
\textgreater= weekly & 804 & \textbf{2229} \\
\end{longtable}

\subsubsection{Imbalance of Confounding Covariates
Treatments}\label{imbalance-of-confounding-covariates-treatments}

Figure~\ref{fig-match_1} shows imbalance of covariates on the treatment
at the first wave

\begin{figure}

\centering{

\includegraphics{v5-man-lmtp-ow-coop-church_files/figure-pdf/fig-match_1-1.pdf}

}

\caption{\label{fig-match_1}Figure imbalance of covariates at the first
exposure (baseline) wave}

\end{figure}%

Figure~\ref{fig-match_2} shows imbalance of covariates on the treatment
at the second exposure. Note that church attendance at the first wave
shows the strongest relationship with church attendance at the second
wave. At both waves, the \texttt{lmtp} automatically adjusts for all
covariates, including the exposure at the previous wave. At both waves
the natural value of the exposure is used to estimate the policy-level
of the exposure.

\begin{figure}

\centering{

\includegraphics{v5-man-lmtp-ow-coop-church_files/figure-pdf/fig-match_2-1.pdf}

}

\caption{\label{fig-match_2}Figure imbalance of covariates at the second
exposure (baseline + 1) wave}

\end{figure}%

\subsection{Appendix D: Baseline and End of Study Outcome
Statistics}\label{appendix-outcomes}

\begin{longtable}[]{@{}
  >{\raggedright\arraybackslash}p{(\columnwidth - 4\tabcolsep) * \real{0.4474}}
  >{\raggedright\arraybackslash}p{(\columnwidth - 4\tabcolsep) * \real{0.2763}}
  >{\raggedright\arraybackslash}p{(\columnwidth - 4\tabcolsep) * \real{0.2763}}@{}}
\caption{Outcomes measured at baseline and
end-of-study}\label{tbl-table-outcomes}\tabularnewline
\toprule\noalign{}
\begin{minipage}[b]{\linewidth}\raggedright
\textbf{Outcome Variables by Wave}
\end{minipage} & \begin{minipage}[b]{\linewidth}\raggedright
\textbf{2018}, N = 47,948
\end{minipage} & \begin{minipage}[b]{\linewidth}\raggedright
\textbf{2020}, N = 47,948
\end{minipage} \\
\midrule\noalign{}
\endfirsthead
\toprule\noalign{}
\begin{minipage}[b]{\linewidth}\raggedright
\textbf{Outcome Variables by Wave}
\end{minipage} & \begin{minipage}[b]{\linewidth}\raggedright
\textbf{2018}, N = 47,948
\end{minipage} & \begin{minipage}[b]{\linewidth}\raggedright
\textbf{2020}, N = 47,948
\end{minipage} \\
\midrule\noalign{}
\endhead
\bottomrule\noalign{}
\endlastfoot
\textbf{Annual Charity} & 120 (25, 500) & 200 (20, 520) \\
Unknown & 3,181 & 16,514 \\
\textbf{Community Gives Money Binary} & 205 (0.4\%) & 144 (0.5\%) \\
Unknown & 2,074 & 16,836 \\
\textbf{Community Gives Time Binary} & 2,389 (5.2\%) & 1,934 (6.2\%) \\
Unknown & 2,074 & 16,836 \\
\textbf{Family Gives Money Binary} & 2,802 (6.1\%) & 1,477 (4.7\%) \\
Unknown & 2,074 & 16,836 \\
\textbf{Family Gives Time Binary} & 13,590 (30\%) & 8,867 (29\%) \\
Unknown & 2,074 & 16,836 \\
\textbf{Friends Give Money Binary} & 589 (1.3\%) & 321 (1.0\%) \\
Unknown & 2,074 & 16,836 \\
\textbf{Friends Give Time Binary} & 8,201 (18\%) & 5,664 (18\%) \\
Unknown & 2,074 & 16,836 \\
\textbf{Sense Neighbourhood Community} & NA & NA \\
1 & 3,025 (6.3\%) & 1,383 (4.3\%) \\
2 & 5,953 (12\%) & 3,252 (10\%) \\
3 & 7,033 (15\%) & 4,283 (13\%) \\
4 & 9,911 (21\%) & 6,961 (22\%) \\
5 & 9,924 (21\%) & 7,625 (24\%) \\
6 & 8,055 (17\%) & 5,921 (19\%) \\
7 & 3,781 (7.9\%) & 2,503 (7.8\%) \\
Unknown & 266 & 16,020 \\
\textbf{Social Belonging} & 5.33 (4.33, 6.00) & 5.00 (4.33, 6.00) \\
Unknown & 450 & 16,114 \\
\textbf{Social Support} & 6.33 (5.33, 7.00) & 6.33 (5.33, 7.00) \\
Unknown & 38 & 15,919 \\
\textbf{Volunteering Hours} & 0.00 (0.00, 1.00) & 0.00 (0.00, 1.00) \\
Unknown & 1,538 & 16,738 \\
\textbf{Volunteers Binary} & 12,905 (28\%) & 8,004 (26\%) \\
Unknown & 1,538 & 16,738 \\
\end{longtable}

Table~\ref{tbl-table-outcomes} presents baseline and end-of-study
descriptive statistics for the outcome variables.

\subsection*{References}\label{references}
\addcontentsline{toc}{subsection}{References}

\phantomsection\label{refs}
\begin{CSLReferences}{1}{0}
\bibitem[\citeproctext]{ref-atkinson2019}
Atkinson, J, Salmond, C, and Crampton, P (2019) \emph{NZDep2018 index of
deprivation, user{'}s manual.}, Wellington.

\bibitem[\citeproctext]{ref-bulbulia2024PRACTICAL}
Bulbulia, J (2024a) A practical guide to causal inference in three-wave
panel studies. \emph{PsyArXiv Preprints}.
doi:\href{https://doi.org/10.31234/osf.io/uyg3d}{10.31234/osf.io/uyg3d}.

\bibitem[\citeproctext]{ref-margot2024}
Bulbulia, JA (2024b) \emph{Margot: MARGinal observational
treatment-effects}.
doi:\href{https://doi.org/10.5281/zenodo.10907724}{10.5281/zenodo.10907724}.

\bibitem[\citeproctext]{ref-Bulbulia_2015}
Bulbulia, S, J. A. (2015) Religion and parental cooperation: An
empirical test of slone's sexual signaling model. In \&. V. S. J. Slone
D., ed., \emph{The attraction of religion: A sexual selectionist
account}, Bloomsbury Press, 29--62.

\bibitem[\citeproctext]{ref-campbell2004}
Campbell, WK, Bonacci, AM, Shelton, J, Exline, JJ, and Bushman, BJ
(2004) Psychological entitlement: interpersonal consequences and
validation of a self-report measure. \emph{Journal of Personality
Assessment}, \textbf{83}(1), 29--45.
doi:\href{https://doi.org/10.1207/s15327752jpa8301_04}{10.1207/s15327752jpa8301\_04}.

\bibitem[\citeproctext]{ref-cutrona1987}
Cutrona, CE, and Russell, DW (1987) The provisions of social
relationships and adaptation to stress. \emph{Advances in Personal
Relationships}, \textbf{1}, 37--67.

\bibitem[\citeproctext]{ref-duxedaz2021}
Díaz, I, Williams, N, Hoffman, KL, and Schenck, EJ (2021) Non-parametric
causal effects based on longitudinal modified treatment policies.
\emph{Journal of the American Statistical Association}.
doi:\href{https://doi.org/10.1080/01621459.2021.1955691}{10.1080/01621459.2021.1955691}.

\bibitem[\citeproctext]{ref-fahy2017}
Fahy, KM, Lee, A, and Milne, BJ (2017b) \emph{New Zealand socio-economic
index 2013}, Wellington, New Zealand: Statistics New Zealand-Tatauranga
Aotearoa.

\bibitem[\citeproctext]{ref-fahy2017a}
Fahy, KM, Lee, A, and Milne, BJ (2017a) \emph{New Zealand socio-economic
index 2013}, Wellington, New Zealand: Statistics New Zealand-Tatauranga
Aotearoa.

\bibitem[\citeproctext]{ref-fraser_coding_2020}
Fraser, G, Bulbulia, J, Greaves, LM, Wilson, MS, and Sibley, CG (2020)
Coding responses to an open-ended gender measure in a new zealand
national sample. \emph{The Journal of Sex Research}, \textbf{57}(8),
979--986.
doi:\href{https://doi.org/10.1080/00224499.2019.1687640}{10.1080/00224499.2019.1687640}.

\bibitem[\citeproctext]{ref-hagerty1995}
Hagerty, BMK, and Patusky, K (1995) Developing a Measure Of Sense of
Belonging: \emph{Nursing Research}, \textbf{44}(1), 9--13.
doi:\href{https://doi.org/10.1097/00006199-199501000-00003}{10.1097/00006199-199501000-00003}.

\bibitem[\citeproctext]{ref-hoffman2023}
Hoffman, KL, Salazar-Barreto, D, Rudolph, KE, and Díaz, I (2023)
Introducing longitudinal modified treatment policies: A unified
framework for studying complex exposures.
doi:\href{https://doi.org/10.48550/arXiv.2304.09460}{10.48550/arXiv.2304.09460}.

\bibitem[\citeproctext]{ref-hoffman2022}
Hoffman, KL, Schenck, EJ, Satlin, MJ, \ldots{} Díaz, I (2022) Comparison
of a target trial emulation framework vs cox regression to estimate the
association of corticosteroids with COVID-19 mortality. \emph{JAMA
Network Open}, \textbf{5}(10), e2234425.
doi:\href{https://doi.org/10.1001/jamanetworkopen.2022.34425}{10.1001/jamanetworkopen.2022.34425}.

\bibitem[\citeproctext]{ref-hoverd_religious_2010}
Hoverd, WJ, and Sibley, CG (2010) Religious and denominational diversity
in new zealand 2009. \emph{New Zealand Sociology}, \textbf{25}(2),
59--87.

\bibitem[\citeproctext]{ref-jost_end_2006-1}
Jost, JT (2006) The end of the end of ideology. \emph{American
Psychologist}, \textbf{61}(7), 651--670.
doi:\href{https://doi.org/10.1037/0003-066X.61.7.651}{10.1037/0003-066X.61.7.651}.

\bibitem[\citeproctext]{ref-linden2020EVALUE}
Linden, A, Mathur, MB, and VanderWeele, TJ (2020) Conducting sensitivity
analysis for unmeasured confounding in observational studies using
e-values: The evalue package. \emph{The Stata Journal}, \textbf{20}(1),
162--175.

\bibitem[\citeproctext]{ref-mccullough2020kindness}
McCullough, ME (2020) \emph{The kindness of strangers: How a selfish ape
invented a new moral code}, Simon; Schuster.

\bibitem[\citeproctext]{ref-sengupta2013}
Sengupta, NK, Luyten, N, Greaves, LM, \ldots{} Sibley, CG (2013) Sense
of Community in New Zealand Neighbourhoods: A Multi-Level Model
Predicting Social Capital. \emph{New Zealand Journal of Psychology},
\textbf{42}(1), 36--45.

\bibitem[\citeproctext]{ref-sibley2012a}
Sibley, CG, and Bulbulia, J (2012) Faith after an earthquake: A
longitudinal study of religion and perceived health before and after the
2011 christchurch new zealand earthquake. \emph{PloS One},
\textbf{7}(12), e49648.

\bibitem[\citeproctext]{ref-sibley2011}
Sibley, CG, Luyten, N, Purnomo, M, \ldots{} Robertson, A (2011) The
Mini-IPIP6: Validation and extension of a short measure of the Big-Six
factors of personality in New Zealand. \emph{New Zealand Journal of
Psychology}, \textbf{40}(3), 142--159.

\bibitem[\citeproctext]{ref-stronge2019onlychild}
Stronge, S, Shaver, JH, Bulbulia, J, and Sibley, CG (2019) Only children
in the 21st century: Personality differences between adults with and
without siblings are very, very small. \emph{Journal of Research in
Personality}, \textbf{83}, 103868.

\bibitem[\citeproctext]{ref-vanbuuren2018}
Van Buuren, S (2018) \emph{Flexible imputation of missing data}, CRC
press.

\bibitem[\citeproctext]{ref-vanderweele2017}
VanderWeele, TJ, and Ding, P (2017) Sensitivity analysis in
observational research: Introducing the e-value. \emph{Annals of
Internal Medicine}, \textbf{167}(4), 268--274.
doi:\href{https://doi.org/10.7326/M16-2607}{10.7326/M16-2607}.

\bibitem[\citeproctext]{ref-vanderweele2013}
VanderWeele, TJ, and Hernan, MA (2013) Causal inference under multiple
versions of treatment. \emph{Journal of Causal Inference},
\textbf{1}(1), 120.

\bibitem[\citeproctext]{ref-vanderweele2020}
VanderWeele, TJ, Mathur, MB, and Chen, Y (2020) Outcome-wide
longitudinal designs for causal inference: A new template for empirical
studies. \emph{Statistical Science}, \textbf{35}(3), 437466.

\bibitem[\citeproctext]{ref-verbrugge1997}
Verbrugge, LM (1997) A global disability indicator. \emph{Journal of
Aging Studies}, \textbf{11}(4), 337--362.
doi:\href{https://doi.org/10.1016/S0890-4065(97)90026-8}{10.1016/S0890-4065(97)90026-8}.

\bibitem[\citeproctext]{ref-williams_cyberostracism_2000}
Williams, KD, Cheung, CKT, and Choi, W (2000) Cyberostracism: Effects of
being ignored over the internet. \emph{Journal of Personality and Social
Psychology}, \textbf{79}(5), 748--762.
doi:\href{https://doi.org/10.1037/0022-3514.79.5.748}{10.1037/0022-3514.79.5.748}.

\bibitem[\citeproctext]{ref-williams2021}
Williams, NT, and Díaz, I (2021) \emph{Lmtp: Non-parametric causal
effects of feasible interventions based on modified treatment policies}.
doi:\href{https://doi.org/10.5281/zenodo.3874931}{10.5281/zenodo.3874931}.

\end{CSLReferences}



\end{document}
