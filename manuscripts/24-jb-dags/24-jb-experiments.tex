% Options for packages loaded elsewhere
\PassOptionsToPackage{unicode}{hyperref}
\PassOptionsToPackage{hyphens}{url}
\PassOptionsToPackage{dvipsnames,svgnames,x11names}{xcolor}
%
\documentclass[
  single column]{article}

\usepackage{amsmath,amssymb}
\usepackage{iftex}
\ifPDFTeX
  \usepackage[T1]{fontenc}
  \usepackage[utf8]{inputenc}
  \usepackage{textcomp} % provide euro and other symbols
\else % if luatex or xetex
  \usepackage{unicode-math}
  \defaultfontfeatures{Scale=MatchLowercase}
  \defaultfontfeatures[\rmfamily]{Ligatures=TeX,Scale=1}
\fi
\usepackage[]{libertinus}
\ifPDFTeX\else  
    % xetex/luatex font selection
\fi
% Use upquote if available, for straight quotes in verbatim environments
\IfFileExists{upquote.sty}{\usepackage{upquote}}{}
\IfFileExists{microtype.sty}{% use microtype if available
  \usepackage[]{microtype}
  \UseMicrotypeSet[protrusion]{basicmath} % disable protrusion for tt fonts
}{}
\makeatletter
\@ifundefined{KOMAClassName}{% if non-KOMA class
  \IfFileExists{parskip.sty}{%
    \usepackage{parskip}
  }{% else
    \setlength{\parindent}{0pt}
    \setlength{\parskip}{6pt plus 2pt minus 1pt}}
}{% if KOMA class
  \KOMAoptions{parskip=half}}
\makeatother
\usepackage{xcolor}
\usepackage[top=30mm,left=25mm,heightrounded,headsep=22pt,headheight=11pt,footskip=33pt,ignorehead,ignorefoot]{geometry}
\setlength{\emergencystretch}{3em} % prevent overfull lines
\setcounter{secnumdepth}{-\maxdimen} % remove section numbering
% Make \paragraph and \subparagraph free-standing
\makeatletter
\ifx\paragraph\undefined\else
  \let\oldparagraph\paragraph
  \renewcommand{\paragraph}{
    \@ifstar
      \xxxParagraphStar
      \xxxParagraphNoStar
  }
  \newcommand{\xxxParagraphStar}[1]{\oldparagraph*{#1}\mbox{}}
  \newcommand{\xxxParagraphNoStar}[1]{\oldparagraph{#1}\mbox{}}
\fi
\ifx\subparagraph\undefined\else
  \let\oldsubparagraph\subparagraph
  \renewcommand{\subparagraph}{
    \@ifstar
      \xxxSubParagraphStar
      \xxxSubParagraphNoStar
  }
  \newcommand{\xxxSubParagraphStar}[1]{\oldsubparagraph*{#1}\mbox{}}
  \newcommand{\xxxSubParagraphNoStar}[1]{\oldsubparagraph{#1}\mbox{}}
\fi
\makeatother


\providecommand{\tightlist}{%
  \setlength{\itemsep}{0pt}\setlength{\parskip}{0pt}}\usepackage{longtable,booktabs,array}
\usepackage{calc} % for calculating minipage widths
% Correct order of tables after \paragraph or \subparagraph
\usepackage{etoolbox}
\makeatletter
\patchcmd\longtable{\par}{\if@noskipsec\mbox{}\fi\par}{}{}
\makeatother
% Allow footnotes in longtable head/foot
\IfFileExists{footnotehyper.sty}{\usepackage{footnotehyper}}{\usepackage{footnote}}
\makesavenoteenv{longtable}
\usepackage{graphicx}
\makeatletter
\def\maxwidth{\ifdim\Gin@nat@width>\linewidth\linewidth\else\Gin@nat@width\fi}
\def\maxheight{\ifdim\Gin@nat@height>\textheight\textheight\else\Gin@nat@height\fi}
\makeatother
% Scale images if necessary, so that they will not overflow the page
% margins by default, and it is still possible to overwrite the defaults
% using explicit options in \includegraphics[width, height, ...]{}
\setkeys{Gin}{width=\maxwidth,height=\maxheight,keepaspectratio}
% Set default figure placement to htbp
\makeatletter
\def\fps@figure{htbp}
\makeatother
% definitions for citeproc citations
\NewDocumentCommand\citeproctext{}{}
\NewDocumentCommand\citeproc{mm}{%
  \begingroup\def\citeproctext{#2}\cite{#1}\endgroup}
\makeatletter
 % allow citations to break across lines
 \let\@cite@ofmt\@firstofone
 % avoid brackets around text for \cite:
 \def\@biblabel#1{}
 \def\@cite#1#2{{#1\if@tempswa , #2\fi}}
\makeatother
\newlength{\cslhangindent}
\setlength{\cslhangindent}{1.5em}
\newlength{\csllabelwidth}
\setlength{\csllabelwidth}{3em}
\newenvironment{CSLReferences}[2] % #1 hanging-indent, #2 entry-spacing
 {\begin{list}{}{%
  \setlength{\itemindent}{0pt}
  \setlength{\leftmargin}{0pt}
  \setlength{\parsep}{0pt}
  % turn on hanging indent if param 1 is 1
  \ifodd #1
   \setlength{\leftmargin}{\cslhangindent}
   \setlength{\itemindent}{-1\cslhangindent}
  \fi
  % set entry spacing
  \setlength{\itemsep}{#2\baselineskip}}}
 {\end{list}}
\usepackage{calc}
\newcommand{\CSLBlock}[1]{\hfill\break\parbox[t]{\linewidth}{\strut\ignorespaces#1\strut}}
\newcommand{\CSLLeftMargin}[1]{\parbox[t]{\csllabelwidth}{\strut#1\strut}}
\newcommand{\CSLRightInline}[1]{\parbox[t]{\linewidth - \csllabelwidth}{\strut#1\strut}}
\newcommand{\CSLIndent}[1]{\hspace{\cslhangindent}#1}

\usepackage{booktabs}
\usepackage{longtable}
\usepackage{array}
\usepackage{multirow}
\usepackage{wrapfig}
\usepackage{float}
\usepackage{colortbl}
\usepackage{pdflscape}
\usepackage{tabu}
\usepackage{threeparttable}
\usepackage{threeparttablex}
\usepackage[normalem]{ulem}
\usepackage{makecell}
\usepackage{xcolor}
\input{/Users/joseph/GIT/latex/latex-for-quarto.tex}
\makeatletter
\@ifpackageloaded{caption}{}{\usepackage{caption}}
\AtBeginDocument{%
\ifdefined\contentsname
  \renewcommand*\contentsname{Table of contents}
\else
  \newcommand\contentsname{Table of contents}
\fi
\ifdefined\listfigurename
  \renewcommand*\listfigurename{List of Figures}
\else
  \newcommand\listfigurename{List of Figures}
\fi
\ifdefined\listtablename
  \renewcommand*\listtablename{List of Tables}
\else
  \newcommand\listtablename{List of Tables}
\fi
\ifdefined\figurename
  \renewcommand*\figurename{Figure}
\else
  \newcommand\figurename{Figure}
\fi
\ifdefined\tablename
  \renewcommand*\tablename{Table}
\else
  \newcommand\tablename{Table}
\fi
}
\@ifpackageloaded{float}{}{\usepackage{float}}
\floatstyle{ruled}
\@ifundefined{c@chapter}{\newfloat{codelisting}{h}{lop}}{\newfloat{codelisting}{h}{lop}[chapter]}
\floatname{codelisting}{Listing}
\newcommand*\listoflistings{\listof{codelisting}{List of Listings}}
\makeatother
\makeatletter
\makeatother
\makeatletter
\@ifpackageloaded{caption}{}{\usepackage{caption}}
\@ifpackageloaded{subcaption}{}{\usepackage{subcaption}}
\makeatother
\ifLuaTeX
  \usepackage{selnolig}  % disable illegal ligatures
\fi
\usepackage{bookmark}

\IfFileExists{xurl.sty}{\usepackage{xurl}}{} % add URL line breaks if available
\urlstyle{same} % disable monospaced font for URLs
\hypersetup{
  pdftitle={On Confounding in Experiments},
  pdfauthor={Joseph A. Bulbulia},
  colorlinks=true,
  linkcolor={blue},
  filecolor={Maroon},
  citecolor={Blue},
  urlcolor={Blue},
  pdfcreator={LaTeX via pandoc}}

\title{On Confounding in Experiments}

\usepackage{academicons}
\usepackage{xcolor}

  \author{Joseph A. Bulbulia}
            \affil{%
             \small{     Victoria University of Wellington, New Zealand
          ORCID \textcolor[HTML]{A6CE39}{\aiOrcid} ~0000-0002-5861-2056 }
              }
      


\date{2024-05-17}
\begin{document}
\maketitle
\begin{abstract}
Confounding bias arises when a treatment and outcome share a common
cause. In randomised controlled trials (Experiments) treatment
assignment is random. It would appear there can be no confounding bias
in experiments. Here we use causal directed acyclic graphs to clarify
seven structural sources of bias in randomised controlled trials.

\textbf{KEYWORDS}: \emph{Causal Inference}; \emph{Experiments};
\emph{DAGs};* \emph{Evolution}; \emph{Per Protocol Effect};
\emph{Intention to Treat Effect}.
\end{abstract}

\subsection{Introduction}\label{introduction}

Here, we consider seven structural sources of bias in randomised
controlled trials, henceforth ``experiments.'' The examples are not
exhaustive. We do not consider biases from selection into study, which
may threaten generalisation. Nor do we consider censoring biases from
study attrition, in which population characteristics of the restricted
sample at the end-of-study differ from those of the sample at baseline
in respect of treatment effects. Both types of sample/target population
can invalidate experimental results. However such biases present as
failure modes in external validity. Our focus here will be on threats to
internal valid from confounding that relate the treatment to the outcome
in the absence of confouding. We will assume large samples such that
random differences in the distribution of variables that may affect
treatment do not come into play. We will also assume that the
experimental designs are double-blind, that treatment conditions are the
same across all arms, and that the investigators are careful and
scupulous. No chance event, other than randomisation, is left to chance.

``Doesn't randomisation, by its very nature, eliminate all systematic
causes of treatment assignment and so of treatment assigment and
outcome?''

We assume the answer is yes.

``Doesn't this mean that confounding is ruled out?''

The answer is no. Eight examples illustrate why. Understanding how
confounding arises in experiments is important for experimental design,
data-analysis, and inference. However, the examples clarify problems of
general interest about how causality operates over time, confounding
where the treatment affects attrition, non-compliance, and non-response,
and differences between the effects of randomisation (intention to treat
effects) and the causal effects of treatments themselves (per-protocol
effects).

We begin by definining our terms.

\subsubsection{Terminology}\label{terminology}

\textbf{Confounding}: treatment and outcome are associated independently
of causality or are disassociated in the presence of causality relative
to causal question at hand.

\textbf{Intention-to-Treat Effect}: The effect of treatment assignment,
analysed based on initial treatment assignment, reflecting real-world
effectiveness but possibly obscuring mechanisms.

\textbf{Per-Protocol Effect}: The causal effect of treatment under full
treatment adherence.

\subsubsection{Meaning of Symbols}\label{meaning-of-symbols}

We use the following conventions in our directed acyclic graphs:

\textbf{\(A\)}: Denotes the ``treatment'' or ``exposure'' - a random
variable.

This is the variable for which we seek to understand the effect of
intervening on it. It is the ``cause.''

\textbf{\(\bar{A}\)}: Denotes a sequence of treatments.

\textbf{\(Y\)}: Denotes the outcome or response, measured at the end of
study.

It is the ``effect.''

\textbf{\(L\)}: Denotes a measured confounder or set of confounders.

\textbf{\(U\)}: Denotes an unmeasured confounder or confounders.

\textbf{\(\mathcal{R}\)}: Denotes a randomisation or a chance event, as
when treatment assignment is random.

\textbf{\(\mathcal{G}\)}: Denotes a graph, here, a causal directed
acyclic graph.

\subsubsection{Elements of causal
graphs}\label{elements-of-causal-graphs}

\textbf{Node}: a node or vertex represents characteristics or features
of units within a population on a causal diagram -- that is a
``variable.'' In causal directed acyclic graphs, we draw nodes with
respect to the \emph{target population}, which is the population for
whom investigators seek causal inferences
(\citeproc{ref-suzuki2020}{Suzuki \emph{et al.} 2020}). Time-indexed
node: \(X_t\) denotes relative chronology

\textbf{Edge without an Arrow} (\(\association\)): path of association,
causality not asserted.

\textbf{Arrow} (\(\rightarrowNEW\)): denotes causal relationship from
the node at the base of the arrow (a parent) to the node at the tip of
the arrow (a child). We typically refrain from drawing an arrow from
treatment to outcome to avoid asserting a causal path from \(A\) to
\(Y\) because the function of a causal directed acyclic graph is to
evaluate whether causality can be identified for this path.

\textbf{Red Arrow} (\(\rightarrowred\)): path of non-causal association
between the treatment and outcome. Path is associational and may run
against arrows.

\textbf{Dashed Arrow} (\(\rightarrowdotted\)): denotes a true
association between the treatment and outcome that becomes partially
obscured when conditioning on a mediator, assuming \(A\) causes \(Y\).

\textbf{Dashed Red Arrow} (\(\rightarrowdottedred\)): highlights
over-conditioning bias from conditioning on a mediator.

\textbf{Boxed Variable} \(\big(\boxed{X}\big)\): conditioning or
adjustment for \(X\).

\textbf{Red-Boxed Variable} \(\big(\boxedred{X}\big)\): highlights the
source of confounding bias from adjustment.

\textbf{Dashed Circle} \(\big( \circledotted{X}\big)\): no adjustment is
made for a variable (implied for unmeasured confounders.)

\textbf{\(\mathbf{\mathcal{R}}\)}
\(\big(\mathcal{R} \rightarrow A\big)\) randomisation into the treatment
condition.

\subsubsection{Review of d-separation for Causal Identification on a
Graph}\label{review-of-d-separation-for-causal-identification-on-a-graph}

\begin{table}

\caption{\label{tbl-fiveelementary}The five elementary structures of
causality from which all causal directed acyclic graphs can be built.}

\centering{

\terminologydirectedgraph

}

\end{table}%

\subsection{Eight Example of Confounding Bias in
Experiments}\label{eight-example-of-confounding-bias-in-experiments}

\begin{table}

\caption{\label{tbl-terminologyelconfoundersexperiments}Eight
confounding biases in Randomised Controlled Trials.}

\centering{

\terminologyelconfoundersexperiments

}

\end{table}%

\subsubsection{Example 1: Post-treatment adjustment blocks treatment
effect}\label{example-1-post-treatment-adjustment-blocks-treatment-effect}

Table~\ref{tbl-terminologyelconfoundersexperiments} \(\mathcal{G} 1.1\)
illustrates the threat of confounding bias by conditioning on a
post-treatment mediator. Imagine investigators are interested in whether
the framing of an authority as religious or secular -- ``source
framing'' -- affects affects subjective ratings of confidence in the
authority -- ``source credibility.'' There are two conditions. A claim
is presented from an authority. The content of the claim does not vary
by condition. Participants are asked to rate the claim on a credibility
scale. Next, imagine that the investigators decide they should control
for religiosity. Furthermore imagine there is a true effect of
source-framing. Finally, assume that the source framing not only affects
source credibility but also affects relgiosity. Perhaps viewing a
religious authority makes religious people more religious. In this
scenario, measuring religiosity following the exeperimental invertion
will partially block the effect of the treatment on the outcome. It
might make it appear that the treatment does not work for religious
people, when in reality it works because it amplifies religiosity. (Note
that in this graph we assume that \(L_1\) occurs before \(Y_2\), however
investigators may have measured \(L_1\) after \(Y_2\). Our time-index
pertains to the occurance of the event, not of its measurement. This
statement applies to all examples that follow.)

Table~\ref{tbl-terminologyelconfoundersexperiments}
\(\mathcal{G} 1.2\)clarifies a response: do not control post-treatment
variables, here the intermediary effects of ``religiosity''. If
effect-modification by religiosity is scientifically interesting,
measure religiosity before randomisation. Randomisation did not prevent
confounding.

\subsubsection{Example 2: Post-treatment adjustment induces collider
stratification
bias}\label{example-2-post-treatment-adjustment-induces-collider-stratification-bias}

Table~\ref{tbl-terminologyelconfoundersexperiments} \(\mathcal{G} 2.1\)
illustrates the threat of confounding bias by conditioning on a
post-treatment collider. Imagine the same experiment as in Example 1 and
the same conditioning strategy in which religiosity is measured
following the treatment. We imagine the treatment affects religiosity.
However, in this example, imagine that religiosity has no causal effect
on the outcome, source credibility. Finally, imagine that an unmeasured
variable affects both the mediator, religiosity (\(L_1\)), and the
outcome, source-credibility (\(Y_2\)). This unmeasured confounder might
be religious education in childhood. In this scenario, conditioning on
the post-treatment variable religiosity will open a backdoor path
between the treatment and outcome, leading to association in the absence
of causation. Randomisation did not prevent confounding.

Table~\ref{tbl-terminologyelconfoundersexperiments} \(\mathcal{G} 2.2\)
clarifies a response: do not control post-treatment variables.

The point that investigators should not condition on post-treatment
variables is worth developing using a common flaw in experimental
designs: exclusion from `attention checks.'\,' Consider that if an
experimental condition affects attention and an unmesaured variable is a
common cause of attention and the outcome, then selection on attention
will induce confounding bias in a randomised experiment. For example,
imagine that people are more attentive in the scientific authority
design, because science is interesting -- whether or not one is
religious, yet religion is not interesting whether or not one is
religious. Suppose further that an unmeasured ``altruistic disposition''
affects both attention and ratings of source credibility. By selecting
on attention, investigators may unwittingly destroy randomisation. If
attention is a scientifically interesting effect-modifier it should be
measured before random assignment to treatment.

\subsubsection{Example 3: Demographic measures at end of study induce
collider stratification
bias}\label{example-3-demographic-measures-at-end-of-study-induce-collider-stratification-bias}

Table~\ref{tbl-terminologyelconfoundersexperiments} \(\mathcal{G} 3.1\)
illustrates the threat of confounding bias from adjusting for
post-treatment variables, here, one affected by the treatment and
outcome absent any unmeasured confounder. In our example, imagine both
the treatment, source framing, and the outcome, source credibility,
affect religiosity measured at the end of study. Investigators measure
religiosity at the end of study and include this measure as a covariate.
However, doing so induces collider bias such that if both the treatment
and outcome are positively associated with religiosity, the collider,
they will be negatively associated with each other. Conditioning on the
collider risks the illusion of a negative experimental effect in the
absence of casuality.

Table~\ref{tbl-terminologyelconfoundersexperiments} \(\mathcal{G} 3.2\)
clarifies a response: again, do not control post-treatment variables! --
here ``religiosity'' measured after the end of study. If the scientific
interest is in effect-modification or in obtaining statistical
precision, measure covariates before randomisation.

\subsubsection{Example 4: Demographic measures at end of study condition
on a collider that opens a back-door
path.}\label{example-4-demographic-measures-at-end-of-study-condition-on-a-collider-that-opens-a-back-door-path.}

Table~\ref{tbl-terminologyelconfoundersexperiments} \(\mathcal{G} 4.1\)
illustrates the threat of illustrates the threat of confounding bias by
adjusting for post-treatment variables, here that is affect only by the
treatment and an unmeasured cause of the outcome. Suppose source
crediblity affects religiosity (religious people are reminded of thier
faith), but there is no experimental affect of framing on credibility.
Imagine further that an unmeasured common cause of the covariate
religiosity and the outcome source credibility. Again, this unmeasured
confounder might be religious education in childhood. In this scenario,
conditioning on the post-treatment variable religiosity will open a
backdoor path between the treatment and outcome, leading to association
in the absence of causation. Again we find that that andomisation did
not prevent confounding.

Table~\ref{tbl-terminologyelconfoundersexperiments} \(\mathcal{G} 4.2\)
clarifies a response. Again, unless investigators are able to rule out
an effect of treatment, they should not condition on a post-treatment
covarite. The covariates of interest should be measured before
randomisation.

\subsubsection{Example 5: Treatment affects attrition biasing measure of
outcome}\label{example-5-treatment-affects-attrition-biasing-measure-of-outcome}

Table~\ref{tbl-terminologyelconfoundersexperiments} \(\mathcal{G} 5\)
Suppose that the experimentally condition affects measurement error in
self-reported source credibility \(U_{\Delta Y}\). For example, suppose
that source framing has no effect on credibility. However, those in the
scientific authority condition are more likely to express credibilty for
science from self-presentation bias. Likewise, perceiving the
investigators to be irreligious, participants in the religious source
framing condition might report less credibility for religious
authorities than they secretly harbour. Directed measurement error from
the treatment to the measurement error of the outcomes creates an
association in the absence of true causality, which we denote by
removing any arrow between the treatment \(A\) and the true outcome
\(Y\).

Table~\ref{tbl-terminologyelconfoundersexperiments} \(\mathcal{G} 5\)
reveals there is no easy solution to directed measurement error bias. If
the magnitude of bias were known, investigators could apply adjustments.
Additional experiments might be devised that are insensitive to directed
measurement error bias. Investigators might compute sensitivity analysis
to example how much measurement error bias would be required to explain
away a results (refer to Linden \emph{et al.}
(\citeproc{ref-linden2020EVALUE}{2020}) for relatively easy-to-implement
sensitivity analysis.) The point we make here is that randomisation does
not prevent confounding by directed measurement error bias.
Investigators must be vigilent.

\subsubsection{Example 6: Per protocol effect lost in sustained
treatments where treatment adherence is affected by a measured
confounder}\label{example-6-per-protocol-effect-lost-in-sustained-treatments-where-treatment-adherence-is-affected-by-a-measured-confounder}

Setting aside self-inflicted injuries of post-treatment conditioning and
directed measurement error, randomisation recovers unbiased causal
effect estimates for randomisation into treatment. Under perfect
adherence, these effect estimates correspond to the causal effects of
the treatments themselves. However, adherence is seldom perfect. The
following examples reveal challenges for recovering per-protocol effects
in settings where there is imperfect adherence.
Table~\ref{tbl-terminologyelconfoundersexperiments} \(\mathcal{G} 6-8\)
are adapted from Hernán \emph{et al.}
(\citeproc{ref-hernan2017per}{2017}).

Table~\ref{tbl-terminologyelconfoundersexperiments} \(\mathcal{G} 6\)
illustrates the threat for identifying the per-protocol effect in
sustained treatments where treatment adherence is affected by a measured
confounder. Consider a sequential experiment that investigates the
effects of sustained adherence to yoga on psychological distress,
measured at the end of study. Suppose that inflexible people are less
likely to adhere the protocols set out in the experiment, and so do not.
Suppose that flexibility is measured by indicator \(L\). If we do not
codition on \(L\) there is an open path from
\(A_1 \association L_0 \association U \association Y_2\). Although
investigators may recover the effect of randomisation into treatment,
the per-protocol effect is confounded.

Table~\ref{tbl-terminologyelconfoundersexperiments} \(\mathcal{G} 6\)
also clarifies a response. Conditioning on \(L_0\) and \(L_1\) will
block the backdoor path, leading to an unbiased per-protocol effect
estimate.

\subsubsection{Example 7: Per protocol effect lost in sustained
treatments where past treatments affect measured confounder of future
treatment
adherence}\label{example-7-per-protocol-effect-lost-in-sustained-treatments-where-past-treatments-affect-measured-confounder-of-future-treatment-adherence}

Table~\ref{tbl-terminologyelconfoundersexperiments} \(\mathcal{G} 7\)
illustrates the threat for identifying the per-protocol effect in
sustained treatments where past treatments affect measured confounder of
future treatment adherence. Suppose that yoga affects flexibility. We
should condition on pre-treatment measures of flexibility to identify
the per-protocal effect. However, conditioning on the post-treatment
measure of flexibilty, \(\boxed{L_1}\) induces collider stratification
bias. This path runs from
\(A_1 \association \L_1 \association U \association Y_3\). However, if
we do not condition on \(L_1\) there is an open backdoor path from
\(A_1 \association U \association Y_3\). We cannot estimate a
per-protocal effect by conditioning strategies.

Table~\ref{tbl-terminologyelconfoundersexperiments} \(\mathcal{G} 7\)
does not clarify the response. However, in a sequential treatment with
fixed strategies, in which there is sequential exchangeability -- or no
unmeasured confounding at each time point -- valid estimators for the
sequential treatments may be constructed (refer to Hernan and Robins
(\citeproc{ref-hernan2024WHATIF}{2024}); Dı́az \emph{et al.}
(\citeproc{ref-diaz2021nonparametric}{2021}); Hoffman \emph{et al.}
(\citeproc{ref-hoffman2023}{2023})). Although we may niavely obtain an
intention-to-treat effect estimate withouth special methods, infering an
effect of doing yoga on well-being -- the per-protocal effect, requires
special methods. These methods are not routinely used in the human
sciences.

\subsubsection{Example 8: Per protocol effect lost in sustained
treatments because both measured and unmeasured confounders affect
treatment
adherence}\label{example-8-per-protocol-effect-lost-in-sustained-treatments-because-both-measured-and-unmeasured-confounders-affect-treatment-adherence}

Table~\ref{tbl-terminologyelconfoundersexperiments} \(\mathcal{G} 8\)
llustrates the threat for identifying the per-protocol effect in
sustained treatments where there is both measured and unmeasured
confounders. Suppose flexibility affects adherence, that yoga affects
flexibilty, and that an unmeasured variable, say prejudice toward
eastern spiritual practices affects adherence. We have no measures for
this variable. There is unmeasured confounding.

If there were no effect of yoga on well-being except through
flexibility, and if flexibility were not affected by the unmeasured
antipathy toward eastern spiritual practices, and further if the effect
of flexibilty on yoga at each time point were condititionally
independent of all future counterfactual data, both for the treatments
and the outcomes, then it might be possibile to construct special
estimators that identify the per-protocol effect of yoga on well-being
in the presence of unmeasured confounding that affects adherence (refer
to Hernán \emph{et al.} (\citeproc{ref-hernan2017per}{2017})). Clearly
we have come a long way from the ANOVAs routinely deployed in
experimental studies. However, if we seek to understand the effect of
yoga on well-being and not the effect of random-assignment to yoga on
well-being, we require special estimators.

\subsection{Conclusions}\label{conclusions}

The examples we have considered here hardly exhaust threats to causal
inference in experiments. Wherever the sample at the end of study
differs in the distribution effect modifiers from the sample population
at the start of study, results will not generalise as we hope. Such bias
goes by different names, such as selection-bias or selection-restriction
bias or sample restriction bias. We have not considered these threats.
It does not take much imagination to imagine threats to valid inference
beyond those considered here. However, I hope the eight examples
considered here persuade investigators of the following:

First, confounding biases are possible in randomised experiments even
when randomisation succeeds.

Second, causal directed acyclic graphs are useful for clarifying these
biases.

Third, many such biases are self-inflicted. These self-inflicted biases
arise from conditioning on variables that may be affected by treatment
assignment. If an experiment consists of a single treatment, unless
investigators are certain that treatments to do not affect covariates,
covariate data should be collected before randomisation.

Fourth, ``attention checks'' should not be used to select participants
after treatments have been randomised. If attention is a relevant
covariate, measures should be taken before randomisation.

Fifth, investigators should not adopt naive practices of infering
per-protocal effects from the portion of the sample that has followed
experimental protocols. Not only is such selection nearly guaranteed to
results in differences between the study population at the start and end
of study, compromising external validity, we have considered that both
measured and unmeasured confounders may invalidate per-protocol results
for the retained sample.

Sixth, methods for identifying causal effects in observational settings
may be useful for the identification of causal effects in randomised
experiments because after randomisation, every experiment becomes and
observational study.

Seventh, the points that we consider here for experiments apply to
observational studies that have obtain psuedorandomisation through
baseline adjustments. Standardly employed methods in the observational
science such as structural equation models or multi-level models will
encounter the same problems that arise in experimental steetings with
sustained treatment strategies. In an aphorism, every observational
study becomes an observational study. That is, sustained treatment
strategies require sequential randomisation. Satisfying the assumptions
for valid causal inferences is typically much more challenging than many
investigators presently understand.

\newpage{}

\subsection{Funding}\label{funding}

This work is supported by a grant from the Templeton Religion Trust
(TRT0418) and RSNZ Marsden 3721245, 20-UOA-123; RSNZ 19-UOO-090. I also
received support from the Max Planck Institute for the Science of Human
History. The Funders had no role in preparing the manuscript or the
decision to publish it.

\subsection{Acknowledgements}\label{acknowledgements}

Errors are my own.

\newpage{}

\subsection{Appendix A: Glossary}\label{appendix-a-glossary}

\begin{table}

\caption{\label{tbl-experiments}Glossary}

\centering{

\glossaryTerms

}

\end{table}%

\phantomsection\label{refs}
\begin{CSLReferences}{1}{0}
\bibitem[\citeproctext]{ref-diaz2021nonparametric}
Dı́az, I, Hejazi, NS, Rudolph, KE, and Der Laan, MJ van (2021)
Nonparametric efficient causal mediation with intermediate confounders.
\emph{Biometrika}, \textbf{108}(3), 627--641.

\bibitem[\citeproctext]{ref-hernan2024WHATIF}
Hernan, MA, and Robins, JM (2024) \emph{Causal inference: What if?},
Taylor \& Francis. Retrieved from
\url{https://www.hsph.harvard.edu/miguel-hernan/causal-inference-book/}

\bibitem[\citeproctext]{ref-hernan2017per}
Hernán, MA, Robins, JM, et al. (2017) Per-protocol analyses of pragmatic
trials. \emph{N Engl J Med}, \textbf{377}(14), 1391--1398.

\bibitem[\citeproctext]{ref-hoffman2023}
Hoffman, KL, Salazar-Barreto, D, Rudolph, KE, and Díaz, I (2023)
Introducing longitudinal modified treatment policies: A unified
framework for studying complex exposures.
doi:\href{https://doi.org/10.48550/arXiv.2304.09460}{10.48550/arXiv.2304.09460}.

\bibitem[\citeproctext]{ref-linden2020EVALUE}
Linden, A, Mathur, MB, and VanderWeele, TJ (2020) Conducting sensitivity
analysis for unmeasured confounding in observational studies using
e-values: The evalue package. \emph{The Stata Journal}, \textbf{20}(1),
162--175.

\bibitem[\citeproctext]{ref-suzuki2020}
Suzuki, E, Shinozaki, T, and Yamamoto, E (2020) Causal Diagrams:
Pitfalls and Tips. \emph{Journal of Epidemiology}, \textbf{30}(4),
153--162.
doi:\href{https://doi.org/10.2188/jea.JE20190192}{10.2188/jea.JE20190192}.

\end{CSLReferences}



\end{document}
