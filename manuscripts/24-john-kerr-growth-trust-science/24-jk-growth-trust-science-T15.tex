% Options for packages loaded elsewhere
\PassOptionsToPackage{unicode}{hyperref}
\PassOptionsToPackage{hyphens}{url}
\PassOptionsToPackage{dvipsnames,svgnames,x11names}{xcolor}
%
\documentclass[
  single column]{article}

\usepackage{amsmath,amssymb}
\usepackage{iftex}
\ifPDFTeX
  \usepackage[T1]{fontenc}
  \usepackage[utf8]{inputenc}
  \usepackage{textcomp} % provide euro and other symbols
\else % if luatex or xetex
  \usepackage{unicode-math}
  \defaultfontfeatures{Scale=MatchLowercase}
  \defaultfontfeatures[\rmfamily]{Ligatures=TeX,Scale=1}
\fi
\usepackage[]{libertinus}
\ifPDFTeX\else  
    % xetex/luatex font selection
\fi
% Use upquote if available, for straight quotes in verbatim environments
\IfFileExists{upquote.sty}{\usepackage{upquote}}{}
\IfFileExists{microtype.sty}{% use microtype if available
  \usepackage[]{microtype}
  \UseMicrotypeSet[protrusion]{basicmath} % disable protrusion for tt fonts
}{}
\makeatletter
\@ifundefined{KOMAClassName}{% if non-KOMA class
  \IfFileExists{parskip.sty}{%
    \usepackage{parskip}
  }{% else
    \setlength{\parindent}{0pt}
    \setlength{\parskip}{6pt plus 2pt minus 1pt}}
}{% if KOMA class
  \KOMAoptions{parskip=half}}
\makeatother
\usepackage{xcolor}
\usepackage[top=30mm,left=25mm,heightrounded,headsep=22pt,headheight=11pt,footskip=33pt,ignorehead,ignorefoot]{geometry}
\setlength{\emergencystretch}{3em} % prevent overfull lines
\setcounter{secnumdepth}{-\maxdimen} % remove section numbering
% Make \paragraph and \subparagraph free-standing
\makeatletter
\ifx\paragraph\undefined\else
  \let\oldparagraph\paragraph
  \renewcommand{\paragraph}{
    \@ifstar
      \xxxParagraphStar
      \xxxParagraphNoStar
  }
  \newcommand{\xxxParagraphStar}[1]{\oldparagraph*{#1}\mbox{}}
  \newcommand{\xxxParagraphNoStar}[1]{\oldparagraph{#1}\mbox{}}
\fi
\ifx\subparagraph\undefined\else
  \let\oldsubparagraph\subparagraph
  \renewcommand{\subparagraph}{
    \@ifstar
      \xxxSubParagraphStar
      \xxxSubParagraphNoStar
  }
  \newcommand{\xxxSubParagraphStar}[1]{\oldsubparagraph*{#1}\mbox{}}
  \newcommand{\xxxSubParagraphNoStar}[1]{\oldsubparagraph{#1}\mbox{}}
\fi
\makeatother


\providecommand{\tightlist}{%
  \setlength{\itemsep}{0pt}\setlength{\parskip}{0pt}}\usepackage{longtable,booktabs,array}
\usepackage{calc} % for calculating minipage widths
% Correct order of tables after \paragraph or \subparagraph
\usepackage{etoolbox}
\makeatletter
\patchcmd\longtable{\par}{\if@noskipsec\mbox{}\fi\par}{}{}
\makeatother
% Allow footnotes in longtable head/foot
\IfFileExists{footnotehyper.sty}{\usepackage{footnotehyper}}{\usepackage{footnote}}
\makesavenoteenv{longtable}
\usepackage{graphicx}
\makeatletter
\newsavebox\pandoc@box
\newcommand*\pandocbounded[1]{% scales image to fit in text height/width
  \sbox\pandoc@box{#1}%
  \Gscale@div\@tempa{\textheight}{\dimexpr\ht\pandoc@box+\dp\pandoc@box\relax}%
  \Gscale@div\@tempb{\linewidth}{\wd\pandoc@box}%
  \ifdim\@tempb\p@<\@tempa\p@\let\@tempa\@tempb\fi% select the smaller of both
  \ifdim\@tempa\p@<\p@\scalebox{\@tempa}{\usebox\pandoc@box}%
  \else\usebox{\pandoc@box}%
  \fi%
}
% Set default figure placement to htbp
\def\fps@figure{htbp}
\makeatother
% definitions for citeproc citations
\NewDocumentCommand\citeproctext{}{}
\NewDocumentCommand\citeproc{mm}{%
  \begingroup\def\citeproctext{#2}\cite{#1}\endgroup}
\makeatletter
 % allow citations to break across lines
 \let\@cite@ofmt\@firstofone
 % avoid brackets around text for \cite:
 \def\@biblabel#1{}
 \def\@cite#1#2{{#1\if@tempswa , #2\fi}}
\makeatother
\newlength{\cslhangindent}
\setlength{\cslhangindent}{1.5em}
\newlength{\csllabelwidth}
\setlength{\csllabelwidth}{3em}
\newenvironment{CSLReferences}[2] % #1 hanging-indent, #2 entry-spacing
 {\begin{list}{}{%
  \setlength{\itemindent}{0pt}
  \setlength{\leftmargin}{0pt}
  \setlength{\parsep}{0pt}
  % turn on hanging indent if param 1 is 1
  \ifodd #1
   \setlength{\leftmargin}{\cslhangindent}
   \setlength{\itemindent}{-1\cslhangindent}
  \fi
  % set entry spacing
  \setlength{\itemsep}{#2\baselineskip}}}
 {\end{list}}
\usepackage{calc}
\newcommand{\CSLBlock}[1]{\hfill\break\parbox[t]{\linewidth}{\strut\ignorespaces#1\strut}}
\newcommand{\CSLLeftMargin}[1]{\parbox[t]{\csllabelwidth}{\strut#1\strut}}
\newcommand{\CSLRightInline}[1]{\parbox[t]{\linewidth - \csllabelwidth}{\strut#1\strut}}
\newcommand{\CSLIndent}[1]{\hspace{\cslhangindent}#1}

\usepackage{booktabs}
\usepackage{longtable}
\usepackage{array}
\usepackage{multirow}
\usepackage{wrapfig}
\usepackage{float}
\usepackage{colortbl}
\usepackage{pdflscape}
\usepackage{tabu}
\usepackage{threeparttable}
\usepackage{threeparttablex}
\usepackage[normalem]{ulem}
\usepackage{makecell}
\usepackage{xcolor}
\usepackage{tabularray}
\usepackage[normalem]{ulem}
\usepackage{graphicx}
\UseTblrLibrary{booktabs}
\UseTblrLibrary{rotating}
\UseTblrLibrary{siunitx}
\NewTableCommand{\tinytableDefineColor}[3]{\definecolor{#1}{#2}{#3}}
\newcommand{\tinytableTabularrayUnderline}[1]{\underline{#1}}
\newcommand{\tinytableTabularrayStrikeout}[1]{\sout{#1}}
\input{/Users/joseph/GIT/latex/latex-for-quarto.tex}
\let\oldtabular\tabular
\renewcommand{\tabular}{\small\oldtabular}
\setlength{\tabcolsep}{4pt}  % Adjust this value as needed
\makeatletter
\@ifpackageloaded{caption}{}{\usepackage{caption}}
\AtBeginDocument{%
\ifdefined\contentsname
  \renewcommand*\contentsname{Table of contents}
\else
  \newcommand\contentsname{Table of contents}
\fi
\ifdefined\listfigurename
  \renewcommand*\listfigurename{List of Figures}
\else
  \newcommand\listfigurename{List of Figures}
\fi
\ifdefined\listtablename
  \renewcommand*\listtablename{List of Tables}
\else
  \newcommand\listtablename{List of Tables}
\fi
\ifdefined\figurename
  \renewcommand*\figurename{Figure}
\else
  \newcommand\figurename{Figure}
\fi
\ifdefined\tablename
  \renewcommand*\tablename{Table}
\else
  \newcommand\tablename{Table}
\fi
}
\@ifpackageloaded{float}{}{\usepackage{float}}
\floatstyle{ruled}
\@ifundefined{c@chapter}{\newfloat{codelisting}{h}{lop}}{\newfloat{codelisting}{h}{lop}[chapter]}
\floatname{codelisting}{Listing}
\newcommand*\listoflistings{\listof{codelisting}{List of Listings}}
\makeatother
\makeatletter
\makeatother
\makeatletter
\@ifpackageloaded{caption}{}{\usepackage{caption}}
\@ifpackageloaded{subcaption}{}{\usepackage{subcaption}}
\makeatother

\usepackage{bookmark}

\IfFileExists{xurl.sty}{\usepackage{xurl}}{} % add URL line breaks if available
\urlstyle{same} % disable monospaced font for URLs
\hypersetup{
  pdftitle={Evidence for Declining Trust in Science From A Large National Panel Study in New Zealand (years 2019-2024)},
  pdfauthor={Authors},
  colorlinks=true,
  linkcolor={blue},
  filecolor={Maroon},
  citecolor={Blue},
  urlcolor={Blue},
  pdfcreator={LaTeX via pandoc}}


\title{Evidence for Declining Trust in Science From A Large National
Panel Study in New Zealand (years 2019-2024)}

\usepackage{academicons}
\usepackage{xcolor}

  \author{Authors}
            \affil{%
             \small{     New Zealand
          ORCID \textcolor[HTML]{A6CE39}{\aiOrcid} ~0000-0003-3169-6576 }
              }
      


\date{2024-12-08}
\begin{document}
\maketitle
\begin{abstract}
The public's perception of science has wide-ranging effects, from the
adoption public health behaviours to climate action. This study uses
nationally representative panel data from New Zealand to assess changes
in trust in science and trust in scientists between late 2019 and late
2024. We draw on a large cohort (N = 42,681, New Zealand Attitudes and
Values Study). Study 1 uses multiple imputation to address biases from
systematic attrition, ensuring more accurate population estimates of
trust. Study 2 models average trust responses over time, showing that
trust in science and scientists rose following New Zealand's Covid-19
response, but later declined. We also find evidence for variation above
and below the mean. Study 3 assesses proportional shifts across low,
medium, and high levels of trust, revealing rising mistrust at both the
lower and higher ends. Considering these shifts at a population level
suggests that nearly 60,000 New Zealanders who once held moderate trust
may now exhibit low trust in science. \textbf{KEYWORDS}:
\emph{Conservativism}; \emph{Institutional Trust}; \emph{Longitudinal};
\emph{Panel}; \emph{Political}; \emph{Science}.
\end{abstract}


\subsection{Introduction}\label{introduction}

Whether people are growing more sceptical of science is question of
considerable interest and concern.

Among the challenges for investigating mistrust in science is that
people who are mistrusting are less likely to participate in scientific
studies.

To address this question, we leverage for waves of comprehensive panel
data from 42,681 participants in the New Zealand Attitudes and Values
Study (NZAVS) -- a national probability sample of New Zealanders,
spanning the years 2019-2024. We estimate annual response trajectories
among the 2019 cohort population, using multiple imputation to adjust
for missing responses in the baseline wave and each subsequent wave.
Although we cannot estimate mistrust among those who were too
mistrusting of science to participate, we can reliably estimate the
trust trajectories among those whose skepticism caused them to drop out
of the study on the assumption that baseline mistrust, political
orientation, age, education, and a host of additional demographic and
ideological variables reliably predict missingness.

Note that each New Zealand and Attitudes Values Study wave typically
starts in October and finishing the following year. Thus we consider
NZAVS waves 2019-2023, also referred to as NZAVS Times 11-15. We start
with Time 11 because this was the first wave in which attitudes to
science and scientists were included in the study.

Study 1 reports expected averages each year in the 2019 cohort in both
the observed sample and in the observed sample that includes multiply
imputed missing values. The response scale for trust items is 1-7.

Study 2 reports categorical response models for the probability of
responding at the low (1-3), medium (4-5), and high (5-6) end of the
continuum. These models allow use to estimate patterns of response not
captured by average responses. Such models are useful for tracking
social divisions.

The aim of each study is descriptive and exploratory. Our primary
objective is to outline and characterise the patterns of change in
institutional trust, focusing on indicators of trust in science and
trust in scientists. However, by comparing observed patterns that do not
adjust for missing responses with those that do, our findings clarify
response bias in cross-sectional studies, which, to the extend that
science skeptics are unlikely to participate in surveys, will not
capture trends as investigators might hope, leading to erroneous
inferences about population-wide trajectories of institutional trust in
science.

Whether people are increasingly sceptical of science is a question of
substantial interest. Among the key challenges in assessing this issue
is that those who mistrust science are often less likely to participate
in surveys. To address this problem, we examine trust trajectories by
using four waves of comprehensive panel data from 42,681 participants in
the New Zealand Attitudes and Values Study (NZAVS), spanning 2019 to
2024. This national probability sample allows us to estimate annual
changes in trust while adjusting for missing responses through multiple
imputation. We assume that baseline mistrust, political orientation,
age, education, and other demographic and ideological factors predict
missingness.

NZAVS waves typically begin in October and finish the following year.
Thus, we use NZAVS waves from 2019 to 2023, also known as Times 11 to
15. We begin at Time 11 because this was the first wave that measured
trust in science and scientists.

Study 1 reports expected averages each year, comparing observed
responses and those adjusted by multiple imputation. The trust items use
a 1--7 response scale. Study 2 examines categorical response models.
These models estimate probabilities of responses in low (1--4), medium
(4--5), and high (5--6) categories. Such models help characterise
patterns not revealed by means and may capture important social
divisions. By comparing patterns without and with imputations, we
highlight how non-response can bias cross-sectional estimates of
population-wide trust. When sceptics are under-represented in surveys,
investigators risk forming misleading conclusions about true patterns of
public trust in science.

\subsection{Method}\label{method}

\subsubsection{Target Population}\label{target-population}

The target population for this study comprises the cohort New Zealand
residents in wave 2019 whose mistrust of science would not have prevent
them from participating in New Zealand Attitudes and Values Study that
year. Here our task is to infer population dynamics for this cohort uses
responses from individuals who participated in the New Zealand Attitudes
and Values Study (NZAVS) during 2019 the baseline wave for this study,
weighted by New Zealand Census weights for age, gender, and ethnicity
(refer to Sibley (\citeproc{ref-sibley2021}{2021})).

\subsubsection{Sample}\label{sample}

The New Zealand Attitudes and Values Study is a national probability
study designed to accurately reflect the broader New Zealand population.
It uses prize draws to incentivise participation. Although the New
Zealand Attitudes and Values Study has good demographic representation
of the country as a whole, it tends to under-sample males and
individuals of Asian descent and over-sample females and Māori (the
indigenous peoples of New Zealand). To address enhance the accuracy of
our findings for the target population, we apply 2018 New Zealand Census
survey weights to the sample data. These weights adjust for variations
in age, gender, and ethnicity (New Zealand European, Asian, Māori,
Pacific) to better approximate the national demographic composition
(\citeproc{ref-sibley2021}{Sibley 2021}). The New Zealand Attitudes and
Values Study uses randomised prize draws to encourage participation, and
thus does not entirely rely on participant motivations to support
science.

\subsubsection{Eligibility Criteria}\label{eligibility-criteria}

To be eligible for this study, participants needed to respond to the New
Zealand Attitudes and Values Study Time 11, years 2019-2024. The cohort
were then tracked for tje baseline waves and the following four waves.
Missing responses were permitted in the baseline wave and all follow up
waves through Zealand Attitudes and Values Study Time 14, years
2022-2024. A total of 42,681 individuals met these criteria and were
included in the study.

\hyperref[appendix-a]{Appendix A} \textbf{?@tbl-demography} presents
sample statistics for this cohort, and describes missing data.

\hyperref[appendix-b]{Appendix B} \textbf{?@fig-timeline} presents a
histogram of New Zealand Attitudes and Values Study daily data
collection for times 11-15 (years 2019-2024).

\subsection{Measures}\label{measures}

We estimated target population average responses for two indicators of
trust in science, which for simplicity we call ``trust in science'' and
``trust in scientists.'' Wording of these items is as follows.

\paragraph{Trust in Science}\label{trust-in-science}

\emph{Our society places too much emphasis on science (reversed).}

Ordinal response: (1 = Strongly Disagree, 7 = Strongly Agree)
(\citeproc{ref-hartman2017}{Hartman \emph{et al.} 2017}).

\paragraph{Trust in Scientists}\label{trust-in-scientists}

\emph{I have a high degree of confidence in the scientific community.}

Ordinal response: (1 = Strongly Disagree, 7 = Strongly Agree)
(\citeproc{ref-nisbet2015}{Nisbet \emph{et al.} 2015}).

Consider that we use the term ``trust in science'' as shorthand for the
value one places on society's emphasis on science. It is plausible that
at least some who disagree with society's emphasis on science are
nevertheless trusting of scientific institutions. That is, we use
``trust in science'' as a convenient short-hand.

\subsubsection{Study Design}\label{study-design}

Study 1 reports expected averages each year in the 2019 cohort in both
the observed sample and in the observed sample that includes multiply
imputed missing values. The response scale for trust in science items is
1-7.

Study 2 reports categorical response models for the probability of
responding at the low (1-3), medium (4-5), and high (5-6) end of the
continuum. These models allow use to estimate patterns of response not
captured by average responses.

\subsubsection{Missing Data}\label{missing-data}

We faced two primary challenges when estimating population-level trends.

First, although the NZAVS uses a national probability sampling strategy,
only about 10\% of those invited participate. It is plausible that the
proportion of those who mistrust science is higher among those who
declined participation, since the NZAVS is run by scientists aiming to
foster scientific understanding.

Second, although the NZAVS retains between 70--80\% of its sample each
year, attrition is inevitable. It is credible that participants who
become more sceptical of science may drop out, leading to an
overestimation of trust in the observed sample.

We cannot overcome the first challenge. We lack direct information about
mistrust among non-participants. However, for those who participated in
2019 but dropped out later, we can adjust estimates under the assumption
that the probability of missing responses is conditionally independent
given observed covariates. Multiple imputation methods allow for bias
adjustment by incorporating missing data uncertainty into estimates
(\citeproc{ref-blackwell_2017_unified}{Blackwell \emph{et al.} 2017};
\citeproc{ref-bulbulia2023a}{Bulbulia \emph{et al.} 2023}).

Here, we used the \texttt{Amelia} package in R
(\citeproc{ref-amelia_2011}{Honaker \emph{et al.} 2011}) to create ten
multiply imputed datasets for the NZAVS 2019 cohort. These cover NZAVS
time 11--15 (waves 2019--2023, years 2019-2024). We treated trust in
science and trust in scientists as ordinal responses bounded by 1 and 7
on the response scale. The Amelia algorithm is designed for within-unit
imputation in repeated-measures time-series data. All covariates in
\textbf{?@tbl-demography} were included. The time variable (year) was
modelled as a cubic spline to capture non-linear trends over four years.
The code for the analysis is deposited here: XXXX

\subsubsection{Statistical Estimator}\label{statistical-estimator}

In Study 1, we estimated mean responses for trust in science and trust
in scientists over time using generalised estimating equations (GEE). We
implemented these models with the geepack package
(\citeproc{ref-geepack_2006}{Halekoh \emph{et al.} 2006}). By specifying
participant id as the clustering variable, we adjusted for repeated
observations and obtained robust standard errors
(\citeproc{ref-mcneish_2017unnecessary}{McNeish \emph{et al.} 2017}). We
weighted the data using sample weights derived from the 2018 New Zealand
Census, ensuring that our estimates reflect population-level trends. We
fitted separate models for each of the ten imputed datasets and pooled
the results using Rubin's rule, thereby incorporating uncertainty from
the missing data process into all population estimates. We generated
predicted means and their associated confidence intervals using
\texttt{ggeffects} (\citeproc{ref-ggeffects_2018}{Ludecke 2018}).

In Study 2, we examined predicted probabilities of trust responses in
three discrete categories: low (1--3), medium (4--5), and high (6--7).
For this analysis, we employed neural network models using the
\texttt{nnet} package (\citeproc{ref-nnet_2002}{Venables and Ripley
2002}). These models provided category-specific probability estimates at
each time point. We applied sample weights constructed from the 2018
Census data to ensure population representativeness, and again used
Rubin's rule to combine inferences across imputed datasets. We
incorporated robust standard errors appropriate for clustered data and
produced predictions and confidence intervals with \texttt{ggeffects}
(\citeproc{ref-ggeffects_2018}{Ludecke 2018}). The final tables and
figures were created using \texttt{tinytable}
(\citeproc{ref-tinytable_2024}{Arel-Bundock 2024}), \texttt{ggplot2}
(\citeproc{ref-ggplot2_2016}{Wickham 2016}), and \texttt{margot}
(\citeproc{ref-margot2024}{Bulbulia 2024}).

\paragraph{Trust Scientists: Sample Means in Retained
Sample}\label{trust-scientists-sample-means-in-retained-sample}

Figure~\ref{fig-hist-outcomes} presents histograms of trust responses
for science and scientists over time in the observed sample. As evident
in the graphs, there is increasing average trust in science over time,
with a large boost in trust in science evident in wave 2020. This boost
coincides with the initially popular New Zealand COVID-19 pandemic
response (\citeproc{ref-sibley2020a}{Sibley \emph{et al.} 2020}).
Similarly, in the retained sample trust in science grew, coinciding with
New Zealand's COVID-19 pandemic response, and then stabalised.

\newpage{}

\begin{figure}

\centering{

\pandocbounded{\includegraphics[keepaspectratio]{24-jk-growth-trust-science-T15_files/figure-pdf/fig-hist-outcomes-1.pdf}}

}

\caption{\label{fig-hist-outcomes}Historgram of sample trust responses
over time in the retained sample.}

\end{figure}%

Below we present sample average responses and categorical distributions
of trust in science and trust in scientists across consecutive waves
from 2019 to 2023, along with sample sizes and missing data counts.
Although mean trust levels generally increased following New Zealand's
initial response to COVID-19, we also note rising missingness over time.
This suggests that straightforward interpretations of sample means may
be misleading, as individuals who become more sceptical of science may
be less likely to remain in the study.

\subsubsection{Sample Descriptive Means Over
Time}\label{sample-descriptive-means-over-time}

Table~\ref{tbl-sample-continuous} shows that mean trust in science rose
from 5.56 in 2019 to 5.86 in 2023, while mean trust in scientists
increased from 5.30 to 5.55 between 2019 and 2020, stabilised through
2022, and then declined slightly to 5.45 in 2023. These observed
increases, particularly for trust in science following the New Zealand
COVID-19 response, appear positive. However, missing responses grew
markedly, from 563 in 2019 to 20,600 in 2023 for trust in science, and
from 1,257 in 2019 to 21,177 in 2023 for trust in scientists. This
attrition may reflect systematic nonresponse patterns related to
mistrust.

\begin{table}

\caption{\label{tbl-sample-continuous}Retained sample average response
by wave.}

\centering{

\centering
\begin{tblr}[         %% tabularray outer open
]                     %% tabularray outer close
{                     %% tabularray inner open
colspec={Q[]Q[]Q[]Q[]Q[]},
cell{2}{1}={r=5,}{valign=h,cmd=\bfseries,},
cell{7}{1}={r=5,}{valign=h,cmd=\bfseries,},
column{1}={halign=l,},
column{2}={halign=l,},
column{3}={halign=l,},
column{4}={halign=l,},
}                     %% tabularray inner close
\toprule
n & Response & wave & mean & missing \\ \midrule %% TinyTableHeader
42,681 & Trust Science & 2019 & 5.56 &    563 \\
42,681 & Trust Science & 2020 & 5.8 &  9,632 \\
42,681 & Trust Science & 2021 & 5.84 & 15,111 \\
42,681 & Trust Science & 2022 & 5.84 & 18,222 \\
42,681 & Trust Science & 2023 & 5.86 & 20,600 \\
42,681 & Trust Scientists & 2019 & 5.3 &  1,257 \\
42,681 & Trust Scientists & 2020 & 5.55 & 10,172 \\
42,681 & Trust Scientists & 2021 & 5.56 & 15,519 \\
42,681 & Trust Scientists & 2022 & 5.55 & 18,949 \\
42,681 & Trust Scientists & 2023 & 5.45 & 21,177 \\
\bottomrule
\end{tblr}

}

\end{table}%

\subsubsection{Sample Categorical Distributions Over
Time}\label{sample-categorical-distributions-over-time}

\begin{table}

\caption{\label{tbl-sample-cat}Retained sample responses by wave by
response category classified as low (1-3), medium (4-5), or high (6-7).}

\centering{

\centering
\begin{tblr}[         %% tabularray outer open
]                     %% tabularray outer close
{                     %% tabularray inner open
colspec={Q[]Q[]Q[]Q[]Q[]Q[]Q[]},
cell{2}{1}={r=3,}{valign=h,cmd=\bfseries,},
cell{5}{1}={r=3,}{valign=h,cmd=\bfseries,},
column{1}={halign=l,},
column{2}={halign=l,},
column{3}={halign=l,},
column{4}={halign=l,},
column{5}={halign=l,},
column{6}={halign=l,},
}                     %% tabularray inner close
\toprule
Response & Level & 2019 & 2020 & 2021 & 2022 & 2023 \\ \midrule %% TinyTableHeader
Trust Science & low & 3434 (8.2%) & 2465 (7.5%) & 2012 (7.3%) & 1740 (7.1%) & 1475 (6.7%) \\
Trust Science & med & 13210 (31.4%) & 7855 (23.8%) & 6223 (22.6%) & 5720 (23.4%) & 5128 (23.2%) \\
Trust Science & high & 25474 (60.5%) & 22729 (68.8%) & 19335 (70.1%) & 16999 (69.5%) & 15478 (70.1%) \\
Trust Scientists & low & 4797 (11.6%) & 2818 (8.7%) & 2477 (9.1%) & 1910 (8.0%) & 2008 (9.3%) \\
Trust Scientists & med & 14161 (34.2%) & 9385 (28.9%) & 7448 (27.4%) & 7107 (29.9%) & 6958 (32.4%) \\
Trust Scientists & high & 22466 (54.2%) & 20306 (62.5%) & 17237 (63.5%) & 14715 (62.0%) & 12538 (58.3%) \\
\bottomrule
\end{tblr}

}

\end{table}%

Table~\ref{tbl-sample-cat} categorises responses into low (1--3), medium
(4--5), and high (6--7) ranges. For trust in science, the proportion
reporting low trust declined from 8.2\% in 2019 to 6.7\% in 2023, while
the proportion reporting high trust rose from 60.5\% to 70.1\% over the
same period. \hyperref[appendix-c]{Appendix C}
\textbf{?@fig-alluv-science-sample} presents these transitions,
revealing increasing tendencies in the retained sample to value the
institutions science.

Similar patterns emerged for trust in scientists, with initial declines
in the low category and increases in the high category, followed by some
reversal in later waves. \hyperref[appendix-c]{Appendix C}
\textbf{?@fig-alluv-scientists-sample} graphically displays these
transitions, suggesting gradually erosion of mistrust in scientists
evident the low end of the response scale, in the retained sample.

These shifts might convey a narrative of strengthening trust during
early pandemic periods. Yet the steady increase in missing responses
across waves underscores the importance of careful interpretation. The
observed improvements in trust could be overstated if sceptical
individuals disproportionately discontinued their participation. In
other words, the rising average trust estimates may in part be an
artefact of differential attrition rather than a genuine population-wide
trend.

\newpage{}

\phantomsection\label{refs}
\begin{CSLReferences}{1}{0}
\bibitem[\citeproctext]{ref-tinytable_2024}
Arel-Bundock, V (2024) \emph{Tinytable: Simple and configurable tables
in 'HTML', 'LaTeX', 'markdown', 'word', 'PNG', 'PDF', and 'typst'
formats}. Retrieved from
\url{https://CRAN.R-project.org/package=tinytable}

\bibitem[\citeproctext]{ref-blackwell_2017_unified}
Blackwell, M, Honaker, J, and King, G (2017) A unified approach to
measurement error and missing data: Overview and applications.
\emph{Sociological Methods \& Research}, \textbf{46}(3), 303--341.

\bibitem[\citeproctext]{ref-margot2024}
Bulbulia, JA (2024) \emph{Margot: MARGinal observational
treatment-effects}.
doi:\href{https://doi.org/10.5281/zenodo.10907724}{10.5281/zenodo.10907724}.

\bibitem[\citeproctext]{ref-bulbulia2023a}
Bulbulia, JA, Afzali, MU, Yogeeswaran, K, and Sibley, CG (2023)
Long-term causal effects of far-right terrorism in {N}ew {Z}ealand.
\emph{PNAS Nexus}, \textbf{2}(8), pgad242.

\bibitem[\citeproctext]{ref-geepack_2006}
Halekoh, U, Højsgaard, S, and Yan, J (2006) The {R} package geepack for
generalized estimating equations. \emph{Journal of Statistical
Software}, \textbf{15/2}, 1--11. Retrieved from
\url{https://www.jstatsoft.org/v15/i02/}

\bibitem[\citeproctext]{ref-hartman2017}
Hartman, RO, Dieckmann, NF, Sprenger, AM, Stastny, BJ, and DeMarree, KG
(2017) Modeling attitudes toward science: Development and validation of
the credibility of science scale. \emph{Basic and Applied Social
Psychology}, \textbf{39}, 358--371.
doi:\href{https://doi.org/10.1080/01973533.2017.1372284}{10.1080/01973533.2017.1372284}.

\bibitem[\citeproctext]{ref-amelia_2011}
Honaker, J, King, G, and Blackwell, M (2011) {Amelia II}: A program for
missing data. \emph{Journal of Statistical Software}, \textbf{45}(7),
1--47.

\bibitem[\citeproctext]{ref-ggeffects_2018}
Ludecke, D (2018) Ggeffects: Tidy data frames of marginal effects from
regression models. \emph{Journal of Open Source Software},
\textbf{3}(26), 772.
doi:\href{https://doi.org/10.21105/joss.00772}{10.21105/joss.00772}.

\bibitem[\citeproctext]{ref-mcneish_2017unnecessary}
McNeish, D, Stapleton, LM, and Silverman, RD (2017) On the unnecessary
ubiquity of hierarchical linear modeling. \emph{Psychological Methods},
\textbf{22}(1), 114.

\bibitem[\citeproctext]{ref-nisbet2015}
Nisbet, EC, Cooper, KE, and Garrett, RK (2015) The partisan brain: How
dissonant science messages lead conservatives and liberals to (dis)trust
science. \emph{The ANNALS of the American Academy of Political and
Social Science}, \textbf{658}(1), 36--66.
doi:\href{https://doi.org/10.1177/0002716214555474}{10.1177/0002716214555474}.

\bibitem[\citeproctext]{ref-sibley2021}
Sibley, CG (2021)
\emph{\href{https://doi.org/10.31234/osf.io/wgqvy}{Sampling procedure
and sample details for the {N}ew {Z}ealand {A}ttitudes and {V}alues
{S}tudy}}.

\bibitem[\citeproctext]{ref-sibley2020a}
Sibley, CG, Greaves, L, Satherley, N, \ldots{} al, et (2020) What
happened to people in {N}ew {Z}ealand during covid-19 home lockdown?
Institutional trust, attitudes to government, mental health and
subjective wellbeing. Retrieved from
\href{https://osf.io/e765a}{osf.io/e765a}

\bibitem[\citeproctext]{ref-nnet_2002}
Venables, WN, and Ripley, BD (2002) \emph{Modern applied statistics with
s}, Fourth, New York: Springer. Retrieved from
\url{https://www.stats.ox.ac.uk/pub/MASS4/}

\bibitem[\citeproctext]{ref-ggplot2_2016}
Wickham, H (2016) \emph{ggplot2: Elegant graphics for data analysis},
Springer-Verlag New York. Retrieved from
\url{https://ggplot2.tidyverse.org}

\end{CSLReferences}




\end{document}
