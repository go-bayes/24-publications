% Options for packages loaded elsewhere
\PassOptionsToPackage{unicode}{hyperref}
\PassOptionsToPackage{hyphens}{url}
\PassOptionsToPackage{dvipsnames,svgnames,x11names}{xcolor}
%
\documentclass[
  single column]{article}

\usepackage{amsmath,amssymb}
\usepackage{iftex}
\ifPDFTeX
  \usepackage[T1]{fontenc}
  \usepackage[utf8]{inputenc}
  \usepackage{textcomp} % provide euro and other symbols
\else % if luatex or xetex
  \usepackage{unicode-math}
  \defaultfontfeatures{Scale=MatchLowercase}
  \defaultfontfeatures[\rmfamily]{Ligatures=TeX,Scale=1}
\fi
\usepackage[]{libertinus}
\ifPDFTeX\else  
    % xetex/luatex font selection
\fi
% Use upquote if available, for straight quotes in verbatim environments
\IfFileExists{upquote.sty}{\usepackage{upquote}}{}
\IfFileExists{microtype.sty}{% use microtype if available
  \usepackage[]{microtype}
  \UseMicrotypeSet[protrusion]{basicmath} % disable protrusion for tt fonts
}{}
\makeatletter
\@ifundefined{KOMAClassName}{% if non-KOMA class
  \IfFileExists{parskip.sty}{%
    \usepackage{parskip}
  }{% else
    \setlength{\parindent}{0pt}
    \setlength{\parskip}{6pt plus 2pt minus 1pt}}
}{% if KOMA class
  \KOMAoptions{parskip=half}}
\makeatother
\usepackage{xcolor}
\usepackage[top=30mm,left=25mm,heightrounded,headsep=22pt,headheight=11pt,footskip=33pt,ignorehead,ignorefoot]{geometry}
\setlength{\emergencystretch}{3em} % prevent overfull lines
\setcounter{secnumdepth}{-\maxdimen} % remove section numbering
% Make \paragraph and \subparagraph free-standing
\makeatletter
\ifx\paragraph\undefined\else
  \let\oldparagraph\paragraph
  \renewcommand{\paragraph}{
    \@ifstar
      \xxxParagraphStar
      \xxxParagraphNoStar
  }
  \newcommand{\xxxParagraphStar}[1]{\oldparagraph*{#1}\mbox{}}
  \newcommand{\xxxParagraphNoStar}[1]{\oldparagraph{#1}\mbox{}}
\fi
\ifx\subparagraph\undefined\else
  \let\oldsubparagraph\subparagraph
  \renewcommand{\subparagraph}{
    \@ifstar
      \xxxSubParagraphStar
      \xxxSubParagraphNoStar
  }
  \newcommand{\xxxSubParagraphStar}[1]{\oldsubparagraph*{#1}\mbox{}}
  \newcommand{\xxxSubParagraphNoStar}[1]{\oldsubparagraph{#1}\mbox{}}
\fi
\makeatother


\providecommand{\tightlist}{%
  \setlength{\itemsep}{0pt}\setlength{\parskip}{0pt}}\usepackage{longtable,booktabs,array}
\usepackage{calc} % for calculating minipage widths
% Correct order of tables after \paragraph or \subparagraph
\usepackage{etoolbox}
\makeatletter
\patchcmd\longtable{\par}{\if@noskipsec\mbox{}\fi\par}{}{}
\makeatother
% Allow footnotes in longtable head/foot
\IfFileExists{footnotehyper.sty}{\usepackage{footnotehyper}}{\usepackage{footnote}}
\makesavenoteenv{longtable}
\usepackage{graphicx}
\makeatletter
\newsavebox\pandoc@box
\newcommand*\pandocbounded[1]{% scales image to fit in text height/width
  \sbox\pandoc@box{#1}%
  \Gscale@div\@tempa{\textheight}{\dimexpr\ht\pandoc@box+\dp\pandoc@box\relax}%
  \Gscale@div\@tempb{\linewidth}{\wd\pandoc@box}%
  \ifdim\@tempb\p@<\@tempa\p@\let\@tempa\@tempb\fi% select the smaller of both
  \ifdim\@tempa\p@<\p@\scalebox{\@tempa}{\usebox\pandoc@box}%
  \else\usebox{\pandoc@box}%
  \fi%
}
% Set default figure placement to htbp
\def\fps@figure{htbp}
\makeatother
% definitions for citeproc citations
\NewDocumentCommand\citeproctext{}{}
\NewDocumentCommand\citeproc{mm}{%
  \begingroup\def\citeproctext{#2}\cite{#1}\endgroup}
\makeatletter
 % allow citations to break across lines
 \let\@cite@ofmt\@firstofone
 % avoid brackets around text for \cite:
 \def\@biblabel#1{}
 \def\@cite#1#2{{#1\if@tempswa , #2\fi}}
\makeatother
\newlength{\cslhangindent}
\setlength{\cslhangindent}{1.5em}
\newlength{\csllabelwidth}
\setlength{\csllabelwidth}{3em}
\newenvironment{CSLReferences}[2] % #1 hanging-indent, #2 entry-spacing
 {\begin{list}{}{%
  \setlength{\itemindent}{0pt}
  \setlength{\leftmargin}{0pt}
  \setlength{\parsep}{0pt}
  % turn on hanging indent if param 1 is 1
  \ifodd #1
   \setlength{\leftmargin}{\cslhangindent}
   \setlength{\itemindent}{-1\cslhangindent}
  \fi
  % set entry spacing
  \setlength{\itemsep}{#2\baselineskip}}}
 {\end{list}}
\usepackage{calc}
\newcommand{\CSLBlock}[1]{\hfill\break\parbox[t]{\linewidth}{\strut\ignorespaces#1\strut}}
\newcommand{\CSLLeftMargin}[1]{\parbox[t]{\csllabelwidth}{\strut#1\strut}}
\newcommand{\CSLRightInline}[1]{\parbox[t]{\linewidth - \csllabelwidth}{\strut#1\strut}}
\newcommand{\CSLIndent}[1]{\hspace{\cslhangindent}#1}

\usepackage{booktabs}
\usepackage{longtable}
\usepackage{array}
\usepackage{multirow}
\usepackage{wrapfig}
\usepackage{float}
\usepackage{colortbl}
\usepackage{pdflscape}
\usepackage{tabu}
\usepackage{threeparttable}
\usepackage{threeparttablex}
\usepackage[normalem]{ulem}
\usepackage{makecell}
\usepackage{xcolor}
\input{/Users/joseph/GIT/latex/latex-for-quarto.tex}
\makeatletter
\@ifpackageloaded{caption}{}{\usepackage{caption}}
\AtBeginDocument{%
\ifdefined\contentsname
  \renewcommand*\contentsname{Table of contents}
\else
  \newcommand\contentsname{Table of contents}
\fi
\ifdefined\listfigurename
  \renewcommand*\listfigurename{List of Figures}
\else
  \newcommand\listfigurename{List of Figures}
\fi
\ifdefined\listtablename
  \renewcommand*\listtablename{List of Tables}
\else
  \newcommand\listtablename{List of Tables}
\fi
\ifdefined\figurename
  \renewcommand*\figurename{Figure}
\else
  \newcommand\figurename{Figure}
\fi
\ifdefined\tablename
  \renewcommand*\tablename{Table}
\else
  \newcommand\tablename{Table}
\fi
}
\@ifpackageloaded{float}{}{\usepackage{float}}
\floatstyle{ruled}
\@ifundefined{c@chapter}{\newfloat{codelisting}{h}{lop}}{\newfloat{codelisting}{h}{lop}[chapter]}
\floatname{codelisting}{Listing}
\newcommand*\listoflistings{\listof{codelisting}{List of Listings}}
\makeatother
\makeatletter
\makeatother
\makeatletter
\@ifpackageloaded{caption}{}{\usepackage{caption}}
\@ifpackageloaded{subcaption}{}{\usepackage{subcaption}}
\makeatother

\usepackage{bookmark}

\IfFileExists{xurl.sty}{\usepackage{xurl}}{} % add URL line breaks if available
\urlstyle{same} % disable monospaced font for URLs
\hypersetup{
  pdftitle={Supplemental Materials For: The Causal Effects of Religious Service Attendance on Prosocial Behaviours in New Zealand: A National Longitudinal Study},
  pdfauthor={Joseph A. Bulbulia; Don E Davis; Kenneth G. Rice; Chris G. Sibley; Geoffrey Troughton},
  pdfkeywords={Use, use},
  colorlinks=true,
  linkcolor={blue},
  filecolor={Maroon},
  citecolor={Blue},
  urlcolor={Blue},
  pdfcreator={LaTeX via pandoc}}


\title{Supplemental Materials For: The Causal Effects of Religious
Service Attendance on Prosocial Behaviours in New Zealand: A National
Longitudinal Study}

\usepackage{academicons}
\usepackage{xcolor}

  \author{Joseph A. Bulbulia}
            \affil{%
             \small{     Victoria University of Wellington, New Zealand
          ORCID \textcolor[HTML]{A6CE39}{\aiOrcid} ~0000-0002-5861-2056 }
              }
      \usepackage{academicons}
\usepackage{xcolor}

  \author{Don E Davis}
            \affil{%
             \small{     Georgia State University, Matheny Center for
the Study of Stress, Trauma, and Resilience
          ORCID \textcolor[HTML]{A6CE39}{\aiOrcid} ~0000-0003-3169-6576 }
              }
      \usepackage{academicons}
\usepackage{xcolor}

  \author{Kenneth G. Rice}
            \affil{%
             \small{     Georgia State University, Matheny Center for
the Study of Stress, Trauma, and Resilience
          ORCID \textcolor[HTML]{A6CE39}{\aiOrcid} ~0000-0002-0558-2818 }
              }
      \usepackage{academicons}
\usepackage{xcolor}

  \author{Chris G. Sibley}
            \affil{%
             \small{     School of Psychology, University of Auckland
          ORCID \textcolor[HTML]{A6CE39}{\aiOrcid} ~0000-0002-4064-8800 }
              }
      \usepackage{academicons}
\usepackage{xcolor}

  \author{Geoffrey Troughton}
            \affil{%
             \small{     School of Social and Cultural Studies, Victoria
University of Wellington
          ORCID \textcolor[HTML]{A6CE39}{\aiOrcid} ~0000-0001-7423-0640 }
              }
      


\date{2024-11-22}
\begin{document}
\maketitle


\subsection{Appendix A: Measures}\label{appendix-measures}

\paragraph{Age (waves: 1-15)}\label{age-waves-1-15}

We asked participants' ages in an open-ended question (``What is your
age?'' or ``What is your date of birth?'').

\paragraph{Born in New Zealand}\label{born-in-new-zealand}

\paragraph{Charitable Donations (Study 1
outcome)}\label{charitable-donations-study-1-outcome}

Using one item from Hoverd and Sibley
(\citeproc{ref-hoverd_religious_2010}{2010}), we asked participants,
``How much money have you donated to charity in the last year?''.

\paragraph{Charitable Volunteering (Study 1
outcome)}\label{charitable-volunteering-study-1-outcome}

We measured hours of volunteering using one item from Sibley \emph{et
al.} (\citeproc{ref-sibley2011}{2011}): ``Hours spent \ldots{}
voluntary/charitable work.''

\paragraph{Children Number (waves: 1-3,
4-15)}\label{children-number-waves-1-3-4-15}

We measured the number of children using one item from Bulbulia \emph{et
al.} (\citeproc{ref-Bulbulia_2015}{2015}). We asked participants, ``How
many children have you given birth to, fathered, or adopted. How many
children have you given birth to, fathered, or adopted?'' or ``How many
children have you given birth to, fathered, or adopted. How many
children have you given birth to, fathered, and/or parented?'' (waves:
12-15).

\paragraph{Disability}\label{disability}

We assessed disability with a one-item indicator adapted from Verbrugge
(\citeproc{ref-verbrugge1997}{1997}). It asks, ``Do you have a health
condition or disability that limits you and that has lasted for 6+
months?'' (1 = Yes, 0 = No).

\paragraph{Education Attainment (waves: 1,
4-15)}\label{education-attainment-waves-1-4-15}

We asked participants, ``What is your highest level of qualification?''.
We coded participants' highest finished degree according to the New
Zealand Qualifications Authority. Ordinal-Rank 0-10 NZREG codes (with
overseas school quals coded as Level 3, and all other ancillary
categories coded as missing)
See:https://www.nzqa.govt.nz/assets/Studying-in-NZ/New-Zealand-Qualification-Framework/requirements-nzqf.pdf

\paragraph{Employment (waves: 1-3,
4-11)}\label{employment-waves-1-3-4-11}

We asked participants, ``Are you currently employed? (This includes
self-employed or casual work)''.

\paragraph{Ethnicity}\label{ethnicity}

Based on the New Zealand Census, we asked participants, ``Which ethnic
group(s) do you belong to?''. The responses were: (1) New Zealand
European; (2) Māori; (3) Samoan; (4) Cook Island Māori; (5) Tongan; (6)
Niuean; (7) Chinese; (8) Indian; (9) Other such as DUTCH, JAPANESE,
TOKELAUAN. Please state:. We coded their answers into four groups:
Maori, Pacific, Asian, and Euro (except for Time 3, which used an
open-ended measure).

\paragraph{Fatigue}\label{fatigue}

We assessed subjective fatigue by asking participants, ``During the last
30 days, how often did \ldots{} you feel exhausted?'' Responses were
collected on an ordinal scale (0 = None of The Time, 1 = A little of The
Time, 2 = Some of The Time, 3 = Most of The Time, 4 = All of The Time).

\paragraph{Honesty-Humility-Modesty Facet (waves:
10-14)}\label{honesty-humility-modesty-facet-waves-10-14}

Participants indicated the extent to which they agree with the following
four statements from Campbell \emph{et al.}
(\citeproc{ref-campbell2004}{2004}) , and Sibley \emph{et al.}
(\citeproc{ref-sibley2011}{2011}) (1 = Strongly Disagree to 7 = Strongly
Agree)

\begin{enumerate}
\def\labelenumi{\roman{enumi}.}
\tightlist
\item
  I want people to know that I am an important person of high status,
  (Waves: 1, 10-14)
\item
  I am an ordinary person who is no better than others.
\item
  I wouldn't want people to treat me as though I were superior to them.
\item
  I think that I am entitled to more respect than the average person is.
\end{enumerate}

\paragraph{Hours of Childcare}\label{hours-of-childcare}

We measured hours of exercising using one item from Sibley \emph{et al.}
(\citeproc{ref-sibley2011}{2011}): 'Hours spent \ldots{} looking after
children.''

To stabilise this indicator, we took the natural log of the response +
1.

\paragraph{Hours of Housework}\label{hours-of-housework}

We measured hours of exercising using one item from Sibley \emph{et al.}
(\citeproc{ref-sibley2011}{2011}): ``Hours spent \ldots{}
housework/cooking''

To stabilise this indicator, we took the natural log of the response +
1.

\paragraph{Hours of Exercise}\label{hours-of-exercise}

We measured hours of exercising using one item from Sibley \emph{et al.}
(\citeproc{ref-sibley2011}{2011}): ``Hours spent \ldots{}
exercising/physical activity''

To stabilise this indicator, we took the natural log of the response +
1.

\paragraph{Hours of Childcare}\label{hours-of-childcare-1}

We measured hours of exercising using one item from Sibley \emph{et al.}
(\citeproc{ref-sibley2011}{2011}): 'Hours spent \ldots{} looking after
children.''

To stabilise this indicator, we took the natural log of the response +
1.

\paragraph{Hours of Exercise}\label{hours-of-exercise-1}

We measured hours of exercising using one item from Sibley \emph{et al.}
(\citeproc{ref-sibley2011}{2011}): ``Hours spent \ldots{}
exercising/physical activity''

To stabilise this indicator, we took the natural log of the response +
1.

\paragraph{Hours of Housework}\label{hours-of-housework-1}

We measured hours of exercising using one item from Sibley \emph{et al.}
(\citeproc{ref-sibley2011}{2011}): ``Hours spent \ldots{}
housework/cooking''

To stabilise this indicator, we took the natural log of the response +
1.

\paragraph{Hours of Sleep}\label{hours-of-sleep}

Participants were asked, ``During the past month, on average, how many
hours of \emph{actual sleep} did you get per night?''.

\paragraph{Hours of Work}\label{hours-of-work}

We measured work hours using one item from Sibley \emph{et al.}
(\citeproc{ref-sibley2011}{2011}): ``Hours spent \ldots{} working in
paid employment.''

To stabilise this indicator, we took the natural log of the response +
1.

\paragraph{Income (waves: 1-3, 4-15)}\label{income-waves-1-3-4-15}

Participants were asked, ``Please estimate your total household income
(before tax) for the year XXXX''. To stabilise this indicator, we first
took the natural log of the response + 1, and then centred and
standardised the log-transformed indicator.

\paragraph{Kessler-6: Psychological Distress (waves:
2-3,4-15)}\label{kessler-6-psychological-distress-waves-2-34-15}

We measured psychological distress using the Kessler-6 scale
(kessler2002?), which exhibits strong diagnostic concordance for
moderate and severe psychological distress in large, crosscultural
samples (kessler2010?; prochaska2012?). Participants rated during the
past 30 days, how often did\ldots{} (1) ``\ldots{} you feel hopeless'';
(2) ``\ldots{} you feel so depressed that nothing could cheer you up'';
(3) ``\ldots{} you feel restless or fidgety''; (4)``\ldots{} you feel
that everything was an effort''; (5) ``\ldots{} you feel worthless'';
(6) '' you feel nervous?'' Ordinal response alternatives for the
Kessler-6 are: ``None of the time''; ``A little of the time''; ``Some of
the time''; ``Most of the time''; ``All of the time.''

\paragraph{Male Gender (waves: 1-15)}\label{male-gender-waves-1-15}

We asked participants' gender in an open-ended question: ``what is your
gender?'' or ``Are you male or female?'' (waves: 1-5). Female was coded
as 0, Male as 1, and gender diverse coded as 3
(\citeproc{ref-fraser_coding_2020}{Fraser \emph{et al.} 2020}). (or 0.5
= neither female nor male)

Here, we coded all those who responded as Male as 1, and those who did
not as 0.

\paragraph{Mini-IPIP 6 (waves:
1-3,4-15)}\label{mini-ipip-6-waves-1-34-15}

We measured participants' personalities with the Mini International
Personality Item Pool 6 (Mini-IPIP6) (\citeproc{ref-sibley2011}{Sibley
\emph{et al.} 2011}), which consists of six dimensions and each
dimension is measured with four items:

\begin{enumerate}
\def\labelenumi{\arabic{enumi}.}
\item
  agreeableness,

  \begin{enumerate}
  \def\labelenumii{\roman{enumii}.}
  \tightlist
  \item
    I sympathize with others' feelings.
  \item
    I am not interested in other people's problems. (r)
  \item
    I feel others' emotions.
  \item
    I am not really interested in others. (r)
  \end{enumerate}
\item
  conscientiousness,

  \begin{enumerate}
  \def\labelenumii{\roman{enumii}.}
  \tightlist
  \item
    I get chores done right away.
  \item
    I like order.
  \item
    I make a mess of things. (r)
  \item
    I often forget to put things back in their proper place. (r)
  \end{enumerate}
\item
  extraversion,

  \begin{enumerate}
  \def\labelenumii{\roman{enumii}.}
  \tightlist
  \item
    I am the life of the party.
  \item
    I don't talk a lot. (r)
  \item
    I keep in the background. (r)
  \item
    I talk to a lot of different people at parties.
  \end{enumerate}
\item
  honesty-humility,

  \begin{enumerate}
  \def\labelenumii{\roman{enumii}.}
  \tightlist
  \item
    I feel entitled to more of everything. (r)
  \item
    I deserve more things in life. (r)
  \item
    I would like to be seen driving around in a very expensive car. (r)
  \item
    I would get a lot of pleasure from owning expensive luxury goods.
    (r)
  \end{enumerate}
\item
  neuroticism, and

  \begin{enumerate}
  \def\labelenumii{\roman{enumii}.}
  \tightlist
  \item
    I have frequent mood swings.
  \item
    I am relaxed most of the time. (r)
  \item
    I get upset easily.
  \item
    I seldom feel blue. (r)
  \end{enumerate}
\item
  openness to experience

  \begin{enumerate}
  \def\labelenumii{\roman{enumii}.}
  \tightlist
  \item
    I have a vivid imagination.
  \item
    I have difficulty understanding abstract ideas. (r)
  \item
    I do not have a good imagination. (r)
  \item
    I am not interested in abstract ideas. (r)
  \end{enumerate}
\end{enumerate}

Each dimension was assessed with four items and participants rated the
accuracy of each item as it applies to them from 1 (Very Inaccurate) to
7 (Very Accurate). Items marked with (r) are reverse coded.

\paragraph{NZ-Born (waves: 1-2,4-15)}\label{nz-born-waves-1-24-15}

We asked participants, ``Which country were you born in?'' or ``Where
were you born? (please be specific, e.g., which town/city?)'' (waves:
6-15).

\paragraph{NZ Deprivation Index (waves:
1-15)}\label{nz-deprivation-index-waves-1-15}

We used the NZ Deprivation Index to assign each participant a score
based on where they live (\citeproc{ref-atkinson2019}{Atkinson \emph{et
al.} 2019}). This score combines data such as income, home ownership,
employment, qualifications, family structure, housing, and access to
transport and communication for an area into one deprivation score.

\paragraph{NZSEI Occupational Prestige and Status (waves:
8-15)}\label{nzsei-occupational-prestige-and-status-waves-8-15}

We assessed occupational prestige and status using the New Zealand
Socio-economic Index 13 (NZSEI-13) (\citeproc{ref-fahy2017a}{Fahy
\emph{et al.} 2017a}). This index uses the income, age, and education of
a reference group, in this case the 2013 New Zealand census, to
calculate a score for each occupational group. Scores range from 10
(Lowest) to 90 (Highest). This list of index scores for occupational
groups was used to assign each participant an NZSEI-13 score based on
their occupation.

We assessed occupational prestige and status using the New Zealand
Socio-economic Index 13 (NZSEI-13) (\citeproc{ref-fahy2017}{Fahy
\emph{et al.} 2017b}). This index uses the income, age, and education of
a reference group, in this case, the 2013 New Zealand census, to
calculate a score for each occupational group. Scores range from 10
(Lowest) to 90 (Highest). This list of index scores for occupational
groups was used to assign each participant an NZSEI-13 score based on
their occupation.

\paragraph{Opt-in}\label{opt-in}

The New Zealand Attitudes and Values Study allows opt-ins to the study.
Because the opt-in population may differ from those sampled randomly
from the New Zealand electoral roll; although the opt-in rate is low, we
include an indicator (yes/no) for this variable.

\paragraph{Partner (No/Yes)}\label{partner-noyes}

``What is your relationship status?'' (e.g., single, married, de-facto,
civil union, widowed, living together, etc.)

\paragraph{Politically Conservative}\label{politically-conservative}

We measured participants' political conservative orientation using a
single item adapted from Jost (\citeproc{ref-jost_end_2006-1}{2006}).

``Please rate how politically liberal versus conservative you see
yourself as being.''

(1 = Extremely Liberal to 7 = Extremely Conservative)

\subparagraph{Religious Service
Attendance}\label{religious-service-attendance}

If participants answered \emph{yes} to ``Do you identify with a religion
and/or spiritual group?'' we measured their frequency of church
attendence using one item from Sibley and Bulbulia
(\citeproc{ref-sibley2012}{2012}): ``how many times did you attend a
church or place of worship in the last month?''. Those participants who
were not religious were imputed a score of ``0''.

\paragraph{Rural/Urban Codes}\label{ruralurban-codes}

Participants residence locations were coded according to a five-level
ordinal categorisation ranging from ``Urban'' to Rural, see Sibley
(\citeproc{ref-sibley2021}{2021}).

\paragraph{Short-Form Health}\label{short-form-health}

Participants' subjective health was measured using one item (``Do you
have a health condition or disability that limits you, and that has
lasted for 6+ months?''; 1 = Yes, 0 = No) adapted from Verbrugge
(\citeproc{ref-verbrugge1997}{1997}).

\paragraph{Sample Origin}\label{sample-origin}

Wave enrolled in NZAVS, see Sibley (\citeproc{ref-sibley2021}{2021}).

\paragraph{Support received: money (waves 10-12) (Study 4
outcomes)}\label{support-received-money-waves-10-12-study-4-outcomes}

The NZAVS has a `revealed' measure of received help and support measured
in hours of support in the previous week. The items are:

\emph{Please estimate how much help you have received from the following
sources in the last week?}

\begin{itemize}
\tightlist
\item
  \emph{family\ldots MONEY (hours)}
\item
  \emph{friends\ldots MONEY (hours)}
\item
  \emph{members of my community\ldots MONEY (hours)}
\end{itemize}

Because this measure is highly variable, we convert responses to binary
indicators: \emph{0 = none/1 any}

\paragraph{Support received: time (waves 10-13) (Study 3
outcomes)}\label{support-received-time-waves-10-13-study-3-outcomes}

\emph{Please estimate how much help you have received from the following
sources in the last week.}

\begin{itemize}
\tightlist
\item
  \emph{family\ldots TIME (hours)}
\item
  \emph{friends\ldots TIME (hours)}
\item
  \emph{members of my community\ldots TIME (hours)}
\end{itemize}

Because this measure is highly variable, we convert responses to binary
indicators: \emph{0 = none/1 any}

\paragraph{Total Siblings}\label{total-siblings}

Participants were asked the following questions related to sibling
counts:

\begin{itemize}
\tightlist
\item
  Were you the 1st born, 2nd born, or 3rd born, etc, child of your
  mother?
\item
  Do you have siblings?
\item
  How many older sisters do you have?
\item
  How many younger sisters do you have?
\item
  How many older brothers do you have?
\item
  How many younger brothers do you have?
\end{itemize}

A single score was obtained from sibling counts by summing responses to
the ``How many\ldots{}'' items. From these scores, an ordered factor was
created ranging from 0 to 7, where participants with more than 7
siblings were grouped into the highest category.

\newpage{}

\subsection{Appendix B. Baseline Demographic
Statistics}\label{appendix-demographics}

\begin{longtable}[]{@{}ll@{}}

\caption{\label{tbl-table-demography}Baseline demographic statistics}

\tabularnewline

\toprule\noalign{}
\textbf{Exposure + Demographic Variables} & \textbf{N = 33,198} \\
\midrule\noalign{}
\endhead
\bottomrule\noalign{}
\endlastfoot
\textbf{Age} & NA \\
Mean (SD) & 51 (14) \\
Min, Max & 18, 96 \\
Q1, Q3 & 41, 61 \\
\textbf{Agreeableness} & NA \\
Mean (SD) & 5.37 (0.98) \\
Min, Max & 1.00, 7.00 \\
Q1, Q3 & 4.75, 6.00 \\
Unknown & 272 \\
\textbf{Born Nz} & 26,197 (79\%) \\
Unknown & 34 \\
\textbf{Children Num} & NA \\
Mean (SD) & 1.76 (1.44) \\
Min, Max & 0.00, 14.00 \\
Q1, Q3 & 0.00, 3.00 \\
\textbf{Conscientiousness} & NA \\
Mean (SD) & 5.14 (1.04) \\
Min, Max & 1.00, 7.00 \\
Q1, Q3 & 4.50, 6.00 \\
Unknown & 266 \\
\textbf{Education Level Coarsen} & NA \\
no\_qualification & 769 (2.3\%) \\
cert\_1\_to\_4 & 11,278 (34\%) \\
cert\_5\_to\_6 & 4,281 (13\%) \\
university & 8,947 (27\%) \\
post\_grad & 3,892 (12\%) \\
masters & 2,956 (9.0\%) \\
doctorate & 891 (2.7\%) \\
Unknown & 184 \\
\textbf{Employed} & 26,379 (80\%) \\
Unknown & 26 \\
\textbf{Eth Cat} & NA \\
euro & 27,404 (83\%) \\
maori & 3,424 (10\%) \\
pacific & 707 (2.1\%) \\
asian & 1,438 (4.4\%) \\
Unknown & 225 \\
\textbf{Extraversion} & NA \\
Mean (SD) & 3.88 (1.20) \\
Min, Max & 1.00, 7.00 \\
Q1, Q3 & 3.00, 4.75 \\
Unknown & 266 \\
\textbf{Hlth Disability} & 7,558 (23\%) \\
Unknown & 561 \\
\textbf{Hlth Fatigue} & NA \\
0 & 5,289 (16\%) \\
1 & 10,940 (33\%) \\
2 & 10,196 (31\%) \\
3 & 4,862 (15\%) \\
4 & 1,577 (4.8\%) \\
Unknown & 334 \\
\textbf{Hlth Sleep Hours} & NA \\
Mean (SD) & 6.95 (1.11) \\
Min, Max & 2.50, 16.00 \\
Q1, Q3 & 6.00, 8.00 \\
Unknown & 1,528 \\
\textbf{Honesty Humility} & NA \\
Mean (SD) & 5.49 (1.15) \\
Min, Max & 1.00, 7.00 \\
Q1, Q3 & 4.75, 6.50 \\
Unknown & 269 \\
\textbf{Hours Children log} & NA \\
Mean (SD) & 1.10 (1.58) \\
Min, Max & 0.00, 5.13 \\
Q1, Q3 & 0.00, 2.20 \\
Unknown & 875 \\
\textbf{Hours Exercise log} & NA \\
Mean (SD) & 1.57 (0.83) \\
Min, Max & 0.00, 4.39 \\
Q1, Q3 & 1.10, 2.08 \\
Unknown & 875 \\
\textbf{Hours Housework log} & NA \\
Mean (SD) & 2.15 (0.77) \\
Min, Max & 0.00, 5.13 \\
Q1, Q3 & 1.79, 2.71 \\
Unknown & 875 \\
\textbf{Hours Work log} & NA \\
Mean (SD) & 2.64 (1.59) \\
Min, Max & 0.00, 4.62 \\
Q1, Q3 & 1.10, 3.71 \\
Unknown & 875 \\
\textbf{Household Inc log} & NA \\
Mean (SD) & 11.41 (0.76) \\
Min, Max & 0.69, 14.92 \\
Q1, Q3 & 11.00, 11.92 \\
Unknown & 1,352 \\
\textbf{Kessler6 Sum} & NA \\
Mean (SD) & 5 (4) \\
Min, Max & 0, 24 \\
Q1, Q3 & 2, 7 \\
Unknown & 297 \\
\textbf{Male} & 11,975 (36\%) \\
\textbf{Modesty} & NA \\
Mean (SD) & 6.03 (0.90) \\
Min, Max & 1.00, 7.00 \\
Q1, Q3 & 5.50, 6.75 \\
Unknown & 11 \\
\textbf{Neuroticism} & NA \\
Mean (SD) & 3.45 (1.15) \\
Min, Max & 1.00, 7.00 \\
Q1, Q3 & 2.50, 4.25 \\
Unknown & 274 \\
\textbf{Nz Dep2018} & NA \\
Mean (SD) & 4.69 (2.70) \\
Min, Max & 1.00, 10.00 \\
Q1, Q3 & 2.00, 7.00 \\
Unknown & 233 \\
\textbf{Nzsei 13 l} & NA \\
Mean (SD) & 55 (16) \\
Min, Max & 10, 90 \\
Q1, Q3 & 42, 69 \\
Unknown & 172 \\
\textbf{Openness} & NA \\
Mean (SD) & 4.99 (1.12) \\
Min, Max & 1.00, 7.00 \\
Q1, Q3 & 4.25, 5.75 \\
Unknown & 267 \\
\textbf{Partner} & 24,869 (76\%) \\
Unknown & 422 \\
\textbf{Political Conservative} & NA \\
1 & 1,777 (5.6\%) \\
2 & 6,563 (21\%) \\
3 & 6,505 (20\%) \\
4 & 9,373 (29\%) \\
5 & 4,813 (15\%) \\
6 & 2,378 (7.5\%) \\
7 & 483 (1.5\%) \\
Unknown & 1,306 \\
\textbf{Religion Church Round} & NA \\
0 & 27,653 (83\%) \\
1 & 1,077 (3.2\%) \\
2 & 777 (2.3\%) \\
3 & 658 (2.0\%) \\
4 & 1,729 (5.2\%) \\
5 & 336 (1.0\%) \\
6 & 226 (0.7\%) \\
7 & 74 (0.2\%) \\
8 & 668 (2.0\%) \\
\textbf{Rural Gch 2018 l} & NA \\
1 & 20,361 (62\%) \\
2 & 6,390 (19\%) \\
3 & 4,020 (12\%) \\
4 & 1,816 (5.5\%) \\
5 & 380 (1.2\%) \\
Unknown & 231 \\
\textbf{Sample Frame Opt in} & 1,107 (3.3\%) \\
\textbf{Sample Origin} & NA \\
1-2 & 2,191 (6.6\%) \\
3-3.5 & 1,664 (5.0\%) \\
4 & 1,987 (6.0\%) \\
5-6-7 & 3,203 (9.6\%) \\
8-9 & 4,264 (13\%) \\
10 & 19,889 (60\%) \\
\textbf{Short Form Health} & NA \\
Mean (SD) & 5.06 (1.16) \\
Min, Max & 1.00, 7.00 \\
Q1, Q3 & 4.33, 6.00 \\
Unknown & 5 \\
\textbf{Total Siblings} & NA \\
Mean (SD) & 2.52 (1.80) \\
Min, Max & 0.00, 23.00 \\
Q1, Q3 & 1.00, 3.00 \\
Unknown & 689 \\

\end{longtable}

Table~\ref{tbl-table-demography} baseline demographic statistics for
couples who met inclusion criteria.

\newpage{}

\subsection{Appendix C: Treatment Statistics}\label{appendix-exposures}

\begin{table}

\caption{\label{tbl-table-exposures-code}Exposures at baseline and
baseline + 1 (treatment) wave}

\centering{

\begin{tabular}[t]{lll}
\toprule
**Exposure Variables by Wave** & **2018**  
N = 33,198 & **2019**  
N = 33,198\\
\midrule
\_\_Religion Church Round\_\_ & NA & NA\\
0 & 27,653 (83\%) & 28,028 (84\%)\\
1 & 1,077 (3.2\%) & 896 (2.7\%)\\
2 & 777 (2.3\%) & 737 (2.2\%)\\
3 & 658 (2.0\%) & 639 (1.9\%)\\
\addlinespace
4 & 1,729 (5.2\%) & 1,672 (5.0\%)\\
5 & 336 (1.0\%) & 308 (0.9\%)\\
6 & 226 (0.7\%) & 205 (0.6\%)\\
7 & 74 (0.2\%) & 68 (0.2\%)\\
8 & 668 (2.0\%) & 645 (1.9\%)\\
\addlinespace
Unknown & 0 & 0\\
\_\_Alert Level Combined\_\_ & NA & NA\\
no\_alert & 33,198 (100\%) & 23,751 (72\%)\\
early\_covid & 0 (0\%) & 3,643 (11\%)\\
alert\_level\_1 & 0 (0\%) & 2,821 (8.5\%)\\
\addlinespace
alert\_level\_2 & 0 (0\%) & 836 (2.5\%)\\
alert\_level\_2\_5\_3 & 0 (0\%) & 552 (1.7\%)\\
alert\_level\_4 & 0 (0\%) & 1,595 (4.8\%)\\
Unknown & 0 & 0\\
\bottomrule
\end{tabular}

}

\end{table}%

tbl-table-exposures-code presents baseline (NZAVS time 10) and exposure
wave (NZAVS time 11) statistics for the exposure variable: religious
service attendance (range 0-8). Responses coded as eight or above were
coded as ``8''. This decision to avoid spare treatments was based on
theoretical grounds, namely, that daily exposure would be similar in its
effects to more than daily exposure. We note that causal contrasts were
obtained for projects with either no attendance or four or more visits
per month. Hence this simplification of the measure is unlikely to
affect theoretical and practical inferences. All models adjusted for the
pandemic alert level because the treatment wave (NZAVS time 11) occurred
during New Zealand's COVID-19 pandemic. The pandemic is not a
``confounder'' because a confounder must be related to the treatment and
the outcome. At the end of the study, all participants had been exposed
to the pandemic. However, to satisfy the causal consistency assumption,
all treatments must be conditionally equivalent within levels of all
covariates (\citeproc{ref-vanderweele2013}{VanderWeele and Hernan
2013}). Because COVID affected the ability or willingness of individuals
to attend religious service, we included the lockdown condition as a
covariate (\citeproc{ref-sibley2021}{Sibley 2021}). To better enable
conditional independence within levels of the treatment variable, we
conditioned on the lead value of COVID-alert level at baseline. To
mitigate systematic biases arising from attrition and missingness, the
\texttt{lmtp} package uses inverse probability of censoring weights,
which were used when estimating the causal effects of the exposure on
the outcome.

\subsubsection{Binary Transition Table for The
Treatment}\label{binary-transition-table-for-the-treatment}

\begin{longtable}[]{@{}ccc@{}}

\caption{\label{tbl-transition-tablegain}Transition table for stability
and change in regular religious service (4x per month) between baseline
and treatment wave.}

\tabularnewline

\toprule\noalign{}
From & \textgreater=4 & \textless{} 4 \\
\midrule\noalign{}
\endhead
\bottomrule\noalign{}
\endlastfoot
\textgreater=4 & \textbf{29496} & 669 \\
\textless{} 4 & 804 & \textbf{2229} \\

\end{longtable}

Table~\ref{tbl-transition-tablegain} presents a transition matrix to
evaluate treatment shifts between baseline and treatment wave. Here, we
focus on the shift from/to monthly attendance at four or more visits per
month. Entries along the diagonal (in bold) indicate the number of
individuals who \textbf{stayed} in their initial state. By contrast, the
off-diagonal shows the transitions from the initial state (bold) to
another state in the following wave (off diagonal). Thus the cell
located at the intersection of row \(i\) and column \(j\), where
\(i \neq j\), gives us the counts of individuals moving from state \(i\)
to state \(j\).

\begin{longtable}[]{@{}ccc@{}}

\caption{\label{tbl-transition-tableloss}Transition table for stability
and change in zero religious service (0 x per month) between baseline
and treatment wave.}

\tabularnewline

\toprule\noalign{}
From & 0 & \textgreater{} 0 \\
\midrule\noalign{}
\endhead
\bottomrule\noalign{}
\endlastfoot
0 & \textbf{26762} & 891 \\
\textgreater{} 0 & 1266 & \textbf{4279} \\

\end{longtable}

Table~\ref{tbl-transition-tableloss} presents a transition matrix to
evaluate treatment shifts between baseline and treatment wave. Here, we
focus on the shift from/to zero religious service attendance. Again,
entries along the diagonal (in bold) indicate the number of individuals
who \textbf{stayed} in their initial state. By contrast, the
off-diagonal shows the transitions from the initial state (bold) to
another state in the following wave (off diagonal). Thus the cell
located at the intersection of row \(i\) and column \(j\), where
\(i \neq j\), gives us the counts of individuals moving from state \(i\)
to state \(j\).

\subsubsection{Imbalance of Confounding Covariates
Treatments}\label{imbalance-of-confounding-covariates-treatments}

Figure~\ref{fig-match_1} shows imbalance of covariates on the treatment
at the treatment wave. The variable on which there is strongest
imbalance is the baseline measure of religious service attendance. It is
important to adjust for this measure both for confounding control and to
better estimate an incident exposure effect for the religious service at
the treatment wave (in contrast to merely estimating a prevalence
effect). See VanderWeele \emph{et al.}
(\citeproc{ref-vanderweele2020}{2020}).

\begin{figure}

\centering{

\pandocbounded{\includegraphics[keepaspectratio]{Supplement-The-causal-effects-of-religious-service-attendance-on-prosocial-behaviours-in-New-Zealand--A-national-longitudinal-study_files/figure-pdf/fig-match_1-1.pdf}}

}

\caption{\label{fig-match_1}This figure shows the imbalance in
covariates on the treatment}

\end{figure}%

\subsection{Appendix D: Baseline and End of Study Outcome
Statistics}\label{appendix-outcomes}

\begin{table}

\caption{\label{tbl-table-outcomes}Outcomes at baseline and
end-of-study}

\centering{

\begin{tabular}[t]{lll}
\toprule
**Outcome Variables by Wave** & **2018**  
N = 33,198 & **2020**  
N = 33,198\\
\midrule
\_\_Annual Charity\_\_ & 150 (40, 500) & 200 (20, 600)\\
Unknown & 1,076 & 6,730\\
\_\_Community Gives Money Binary\_\_ & 135 (0.4\%) & 118 (0.4\%)\\
Unknown & 669 & 6,959\\
\_\_Community Gives Time Binary\_\_ & 1,702 (5.2\%) & 1,669 (6.4\%)\\
\addlinespace
Unknown & 669 & 6,959\\
\_\_Family Gives Money Binary\_\_ & 1,782 (5.5\%) & 1,236 (4.7\%)\\
Unknown & 669 & 6,959\\
\_\_Family Gives Time Binary\_\_ & 9,539 (29\%) & 7,600 (29\%)\\
Unknown & 669 & 6,959\\
\addlinespace
\_\_Friends Give Money Binary\_\_ & 372 (1.1\%) & 270 (1.0\%)\\
Unknown & 669 & 6,959\\
\_\_Friends Give Time\_\_ & 5,765 (18\%) & 4,855 (19\%)\\
Unknown & 669 & 6,959\\
\_\_Sense Neighbourhood Community\_\_ & NA & NA\\
\addlinespace
1 & 1,976 (6.0\%) & 1,124 (4.2\%)\\
2 & 4,037 (12\%) & 2,703 (10\%)\\
3 & 4,796 (15\%) & 3,561 (13\%)\\
4 & 6,840 (21\%) & 5,809 (22\%)\\
5 & 7,088 (21\%) & 6,477 (24\%)\\
\addlinespace
6 & 5,753 (17\%) & 5,060 (19\%)\\
7 & 2,550 (7.7\%) & 2,104 (7.8\%)\\
Unknown & 158 & 6,360\\
\_\_Social Belonging\_\_ & 5.33 (4.33, 6.00) & 5.33 (4.33, 6.00)\\
Unknown & 268 & 6,418\\
\addlinespace
\_\_Social Support\_\_ & 6.33 (5.33, 7.00) & 6.33 (5.33, 7.00)\\
Unknown & 19 & 6,286\\
\_\_Volunteering Hours\_\_ & 0.00 (0.00, 1.00) & 0.00 (0.00, 1.00)\\
Unknown & 875 & 6,863\\
\_\_Volunteers Binary\_\_ & 9,443 (29\%) & 6,881 (26\%)\\
\addlinespace
Unknown & 875 & 6,863\\
\bottomrule
\end{tabular}

}

\end{table}%

Table~\ref{tbl-table-outcomes} presents baseline and end-of-study
descriptive statistics for the outcome variables.

\newpage{}

\subsection*{References}\label{references}
\addcontentsline{toc}{subsection}{References}

\phantomsection\label{refs}
\begin{CSLReferences}{1}{0}
\bibitem[\citeproctext]{ref-atkinson2019}
Atkinson, J, Salmond, C, and Crampton, P (2019) \emph{NZDep2018 index of
deprivation, user{'}s manual.}, Wellington.

\bibitem[\citeproctext]{ref-Bulbulia_2015}
Bulbulia, JA, Shaver, JH, Greaves, L, Sosis, R, and Sibley, CG (2015)
Religion and parental cooperation: An empirical test of slone's sexual
signaling model. In S. J. D. amd Van Slyke J., ed., \emph{The attraction
of religion: A sexual selectionist account}, Bloomsbury Press, 29--62.

\bibitem[\citeproctext]{ref-campbell2004}
Campbell, WK, Bonacci, AM, Shelton, J, Exline, JJ, and Bushman, BJ
(2004) Psychological entitlement: Interpersonal consequences and
validation of a self-report measure. \emph{Journal of Personality
Assessment}, \textbf{83}(1), 29--45.

\bibitem[\citeproctext]{ref-fahy2017}
Fahy, KM, Lee, A, and Milne, BJ (2017b) \emph{{N}ew {Z}ealand
socio-economic index 2013}, Wellington, New Zealand: Statistics New
Zealand-Tatauranga Aotearoa.

\bibitem[\citeproctext]{ref-fahy2017a}
Fahy, KM, Lee, A, and Milne, BJ (2017a) \emph{{N}ew {Z}ealand
socio-economic index 2013}, Wellington, New Zealand: Statistics New
Zealand-Tatauranga Aotearoa.

\bibitem[\citeproctext]{ref-fraser_coding_2020}
Fraser, G, Bulbulia, J, Greaves, LM, Wilson, MS, and Sibley, CG (2020)
Coding responses to an open-ended gender measure in a {N}ew {Z}ealand
national sample. \emph{The Journal of Sex Research}, \textbf{57}(8),
979--986.
doi:\href{https://doi.org/10.1080/00224499.2019.1687640}{10.1080/00224499.2019.1687640}.

\bibitem[\citeproctext]{ref-hoverd_religious_2010}
Hoverd, WJ, and Sibley, CG (2010) Religious and denominational diversity
in {N}ew {Z}ealand 2009. \emph{New Zealand Sociology}, \textbf{25}(2),
59--87.

\bibitem[\citeproctext]{ref-jost_end_2006-1}
Jost, JT (2006) The end of the end of ideology. \emph{American
Psychologist}, \textbf{61}(7), 651--670.
doi:\href{https://doi.org/10.1037/0003-066X.61.7.651}{10.1037/0003-066X.61.7.651}.

\bibitem[\citeproctext]{ref-sibley2012}
Sibley, C. G., and Bulbulia, JA (2012) Healing those who need healing:
How religious practice affects social belonging. \emph{Journal for the
Cognitive Science of Religion}, \textbf{1}, 29--45.

\bibitem[\citeproctext]{ref-sibley2021}
Sibley, CG (2021)
\emph{\href{https://doi.org/10.31234/osf.io/wgqvy}{Sampling procedure
and sample details for the {N}ew {Z}ealand {A}ttitudes and {V}alues
{S}tudy}}.

\bibitem[\citeproctext]{ref-sibley2011}
Sibley, CG, Luyten, N, Purnomo, M, \ldots{} Robertson, A (2011) The
Mini-IPIP6: Validation and extension of a short measure of the Big-Six
factors of personality in {N}ew {Z}ealand. \emph{New Zealand Journal of
Psychology}, \textbf{40}(3), 142--159.

\bibitem[\citeproctext]{ref-vanderweele2013}
VanderWeele, TJ, and Hernan, MA (2013) Causal inference under multiple
versions of treatment. \emph{Journal of Causal Inference},
\textbf{1}(1), 1--20.

\bibitem[\citeproctext]{ref-vanderweele2020}
VanderWeele, TJ, Mathur, MB, and Chen, Y (2020) Outcome-wide
longitudinal designs for causal inference: A new template for empirical
studies. \emph{Statistical Science}, \textbf{35}(3), 437--466.

\bibitem[\citeproctext]{ref-verbrugge1997}
Verbrugge, LM (1997) A global disability indicator. \emph{Journal of
Aging Studies}, \textbf{11}(4), 337--362.
doi:\href{https://doi.org/10.1016/S0890-4065(97)90026-8}{10.1016/S0890-4065(97)90026-8}.

\end{CSLReferences}




\end{document}
