\documentclass[
  man,
  longtable,
  nolmodern,
  notxfonts,
  notimes,
  colorlinks=true,linkcolor=blue,citecolor=blue,urlcolor=blue]{apa7}

\usepackage{amsmath}
\usepackage{amssymb}



\usepackage[bidi=default]{babel}
\babelprovide[main,import]{english}


% get rid of language-specific shorthands (see #6817):
\let\LanguageShortHands\languageshorthands
\def\languageshorthands#1{}

\RequirePackage{longtable}
% \setlength\LTleft{0pt}
\RequirePackage{threeparttablex}

% % % \RequirePackage{rotating}
% \RequirePackage{threeparttablex}
% \DeclareDelayedFloatFlavor{sidewaysfigure}{figure}
% \DeclareDelayedFloatFlavor{sidewaystable}{table}
% \DeclareDelayedFloatFlavor{longtable}{table}
\DeclareDelayedFloatFlavor{ThreePartTable}{table}


% % 


\makeatletter
\renewcommand{\paragraph}{\@startsection{paragraph}{4}{\parindent}%
	{0\baselineskip \@plus 0.2ex \@minus 0.2ex}%
	{-.5em}%
	{\normalfont\normalsize\bfseries\typesectitle}}

\renewcommand{\subparagraph}[1]{\@startsection{subparagraph}{5}{0.5em}%
	{0\baselineskip \@plus 0.2ex \@minus 0.2ex}%
	{-\z@\relax}%
	{\normalfont\normalsize\bfseries\itshape\hspace{\parindent}{#1}\textit{\addperi}}{\relax}}
\makeatother




\usepackage{longtable, booktabs, multirow, multicol, colortbl, hhline, caption, array, float, xpatch}
\setcounter{topnumber}{2}
\setcounter{bottomnumber}{2}
\setcounter{totalnumber}{4}
\renewcommand{\topfraction}{0.85}
\renewcommand{\bottomfraction}{0.85}
\renewcommand{\textfraction}{0.15}
\renewcommand{\floatpagefraction}{0.7}

\usepackage{tcolorbox}
\tcbuselibrary{listings,theorems, breakable, skins}
\usepackage{fontawesome5}

\definecolor{quarto-callout-color}{HTML}{909090}
\definecolor{quarto-callout-note-color}{HTML}{0758E5}
\definecolor{quarto-callout-important-color}{HTML}{CC1914}
\definecolor{quarto-callout-warning-color}{HTML}{EB9113}
\definecolor{quarto-callout-tip-color}{HTML}{00A047}
\definecolor{quarto-callout-caution-color}{HTML}{FC5300}
\definecolor{quarto-callout-color-frame}{HTML}{ACACAC}
\definecolor{quarto-callout-note-color-frame}{HTML}{4582EC}
\definecolor{quarto-callout-important-color-frame}{HTML}{D9534F}
\definecolor{quarto-callout-warning-color-frame}{HTML}{F0AD4E}
\definecolor{quarto-callout-tip-color-frame}{HTML}{02B875}
\definecolor{quarto-callout-caution-color-frame}{HTML}{FD7E14}

\newlength\Oldarrayrulewidth
\newlength\Oldtabcolsep


\usepackage{hyperref}




\providecommand{\tightlist}{%
  \setlength{\itemsep}{0pt}\setlength{\parskip}{0pt}}
\usepackage{longtable,booktabs,array}
\usepackage{calc} % for calculating minipage widths
% Correct order of tables after \paragraph or \subparagraph
\usepackage{etoolbox}
\makeatletter
\patchcmd\longtable{\par}{\if@noskipsec\mbox{}\fi\par}{}{}
\makeatother
% Allow footnotes in longtable head/foot
\IfFileExists{footnotehyper.sty}{\usepackage{footnotehyper}}{\usepackage{footnote}}
\makesavenoteenv{longtable}

\usepackage{graphicx}
\makeatletter
\def\maxwidth{\ifdim\Gin@nat@width>\linewidth\linewidth\else\Gin@nat@width\fi}
\def\maxheight{\ifdim\Gin@nat@height>\textheight\textheight\else\Gin@nat@height\fi}
\makeatother
% Scale images if necessary, so that they will not overflow the page
% margins by default, and it is still possible to overwrite the defaults
% using explicit options in \includegraphics[width, height, ...]{}
\setkeys{Gin}{width=\maxwidth,height=\maxheight,keepaspectratio}
% Set default figure placement to htbp
\makeatletter
\def\fps@figure{htbp}
\makeatother


% definitions for citeproc citations
\NewDocumentCommand\citeproctext{}{}
\NewDocumentCommand\citeproc{mm}{%
  \begingroup\def\citeproctext{#2}\cite{#1}\endgroup}
\makeatletter
 % allow citations to break across lines
 \let\@cite@ofmt\@firstofone
 % avoid brackets around text for \cite:
 \def\@biblabel#1{}
 \def\@cite#1#2{{#1\if@tempswa , #2\fi}}
\makeatother
\newlength{\cslhangindent}
\setlength{\cslhangindent}{1.5em}
\newlength{\csllabelwidth}
\setlength{\csllabelwidth}{3em}
\newenvironment{CSLReferences}[2] % #1 hanging-indent, #2 entry-spacing
 {\begin{list}{}{%
  \setlength{\itemindent}{0pt}
  \setlength{\leftmargin}{0pt}
  \setlength{\parsep}{0pt}
  % turn on hanging indent if param 1 is 1
  \ifodd #1
   \setlength{\leftmargin}{\cslhangindent}
   \setlength{\itemindent}{-1\cslhangindent}
  \fi
  % set entry spacing
  \setlength{\itemsep}{#2\baselineskip}}}
 {\end{list}}
\usepackage{calc}
\newcommand{\CSLBlock}[1]{\hfill\break\parbox[t]{\linewidth}{\strut\ignorespaces#1\strut}}
\newcommand{\CSLLeftMargin}[1]{\parbox[t]{\csllabelwidth}{\strut#1\strut}}
\newcommand{\CSLRightInline}[1]{\parbox[t]{\linewidth - \csllabelwidth}{\strut#1\strut}}
\newcommand{\CSLIndent}[1]{\hspace{\cslhangindent}#1}





\usepackage{newtx}

\defaultfontfeatures{Scale=MatchLowercase}
\defaultfontfeatures[\rmfamily]{Ligatures=TeX,Scale=1}

\usepackage[]{libertinus}




\title{Causal Effects of Religious Service Attendance on Charity and
Volunteering: Evidence Using Novel Measures From A National Longitudinal
Panel}
\shorttitle{Causal Effects of Religious Service}


\usepackage{etoolbox}








\authorsnames[{1},{2},{2},{3},{4}]{Joseph A. Bulbulia,Don E
Davis,Kenneth G. Rice,Chris G. Sibley,Geoffrey Troughton}







\authorsaffiliations{
{Victoria University of Wellington, New Zealand},{Georgia State
University},{School of Psychology, University of Auckland},{Victoria
University of Wellington}}






\leftheader{Bulbulia, Davis, Rice, Sibley and Troughton}



\abstract{Causal investigations for the effects of religion on
prosociality must be precise. One should articulate a specific causal
contrast for a feature of religion, select appropriate prosociality
measures, define the target population, gather time-series data, and,
only after identification assumptions are met, conduct statistical and
sensitivity analyses. Here, we examine three distinct interventions on
religious service attendance (increase, decrease, maintain) in a
longitudinal sample of 33,198 New Zealanders (years 2018 to 2021). Study
1 investigates effects of religious service on charitable contributions
and volunteerism. Studies 2 and 3 investigate effects of religious
service on the relative risks of \emph{receiving} aid or financial
support from others during the past week -- measures designed to
minimise self-reporting bias. Across all studies, inferred causal
effects are substantially less pronounced than observed cross-sectional
associations. Nonetheless, regular attendance across the population
would enhance charitable donations by 4\% of the New Zealand
Government's annual spending. This research underscores the essential
role of formulating precise causal questions and recommends a workflow
for answering them in the scientific study of cultural practices.}
% 
\keywords{Causal Inference, keyword2, keyword3}

\authornote{\par{\addORCIDlink{Joseph A.
Bulbulia}{0000-0002-5861-2056}}\par{\addORCIDlink{Don E
Davis}{0000-0003-3169-6576}}\par{\addORCIDlink{Kenneth G.
Rice}{0000-0002-0558-2818}}\par{\addORCIDlink{Chris G.
Sibley}{0000-0002-4064-8800}}\par{\addORCIDlink{Geoffrey
Troughton}{0000-0001-7423-0640}}
\par{ }
\par{       }
\par{Correspondence concerning this article should be addressed
to Joseph A. Bulbulia, Email: joseph.bulbulia@vuw.ac.nz}
}


\makeatletter
\let\endoldlt\endlongtable
\def\endlongtable{
\hline
\endoldlt
}
\makeatother
\RequirePackage{longtable}
\DeclareDelayedFloatFlavor{longtable}{table}
% \RequirePackage{threeparttablex}
% \DeclareDelayedFloatFlavor{ThreePartTable}{table}

\urlstyle{same}



% From https://tex.stackexchange.com/a/645996/211326
%%% apa7 doesn't want to add appendix section titles in the toc
%%% let's make it do it
\makeatletter
\xpatchcmd{\appendix}
  {\par}
  {\addcontentsline{toc}{section}{\@currentlabelname}\par}
  {}{}
\makeatother

\begin{document}

\maketitle


\setcounter{secnumdepth}{-\maxdimen} % remove section numbering

\setlength\LTleft{0pt}




This is my introductory paragraph. The title will be placed above it
automatically. \emph{Do not start with an introductory heading} (e.g.,
``Introduction''). The title acts as your Level 1 heading for the
introduction.

Details about writing headings with markdown in APA style are
\href{https://wjschne.github.io/apaquarto/writing.html\#headings-in-apa-style}{here}.

\subsection{Displaying Figures}\label{displaying-figures}

A reference label for a figure must have the prefix \texttt{fig-}, and
in a code chunk, the caption must be set with \texttt{fig-cap}. Captions
are in
\href{https://apastyle.apa.org/style-grammar-guidelines/capitalization/title-case}{title
case}.

\begin{figure}[!htbp]

{\caption{{The Figure Caption}{\label{fig-myplot}}}}

\includegraphics{example_files/figure-pdf/fig-myplot-1.pdf}

{\noindent \emph{Note.} This is the note below the figure.}

\end{figure}

To refer to any figure or table, use the \texttt{@} symbol followed by
the reference label (e.g., Figure~\ref{fig-myplot}).

\subsection{Imported Graphics}\label{imported-graphics}

One way to import an existing graphic as a figure is to use
\texttt{knitr::include\_graphics} in a code chunk. For example,
Figure~\ref{fig-import1} is an imported image. Note that in
apaquarto-pdf documents, we can specify that that a figure or table
should span both columns when in journal mode by setting the
\texttt{apa-twocolumn} chunk option to \texttt{true}. For other formats,
this distinction does not matter.

\begin{figure}[!htbp]

{\caption{{An Imported Graphic}{\label{fig-import1}}}}

\includegraphics[width=0.48\textwidth,height=\textheight]{sampleimage.png}

{\noindent \emph{Note.} A note below the figure}

\end{figure}

Figure graphics can be imported directly with Markdown:

\begin{figure}[!htbp]

{\caption{{Another Way to Import Graphics}{\label{fig-import2}}}}

\includegraphics[width=0.49\textwidth,height=\textheight]{sampleimage.png}

{\noindent \emph{Note.} A note below the figure}

\end{figure}

Which style of creating figures you choose depends on preference and
need.

\subsection{Displaying Tables}\label{displaying-tables}

We can make a table the same way as a figure. Generating a table that
conforms to APA format in all document formats can be tricky. When the
table is simple, the \texttt{kable} function from knitr works well. Feel
free to experiment with different methods, but I have found that David
Gohel's \href{https://davidgohel.github.io/flextable/}{flextable} to be
the best option when I need something more complex.

\begin{table}

{\caption{{The Table Caption}{\label{tbl-mytable}}}
\vspace{-20pt}}

\global\setlength{\Oldarrayrulewidth}{\arrayrulewidth}

\global\setlength{\Oldtabcolsep}{\tabcolsep}

\setlength{\tabcolsep}{2pt}

\renewcommand*{\arraystretch}{1.5}



\providecommand{\ascline}[3]{\noalign{\global\arrayrulewidth #1}\arrayrulecolor[HTML]{#2}\cline{#3}}

\begin{longtable*}[l]{|p{0.75in}|p{0.75in}}



\ascline{0.75pt}{000000}{1-2}

\multicolumn{1}{>{\centering}m{\dimexpr 0.75in+0\tabcolsep}}{\textcolor[HTML]{000000}{\fontsize{11}{11}\selectfont{Numbers}}} & \multicolumn{1}{>{\centering}m{\dimexpr 0.75in+0\tabcolsep}}{\textcolor[HTML]{000000}{\fontsize{11}{11}\selectfont{Letters}}} \\

\ascline{0.75pt}{000000}{1-2}\endfirsthead 

\ascline{0.75pt}{000000}{1-2}

\multicolumn{1}{>{\centering}m{\dimexpr 0.75in+0\tabcolsep}}{\textcolor[HTML]{000000}{\fontsize{11}{11}\selectfont{Numbers}}} & \multicolumn{1}{>{\centering}m{\dimexpr 0.75in+0\tabcolsep}}{\textcolor[HTML]{000000}{\fontsize{11}{11}\selectfont{Letters}}} \\

\ascline{0.75pt}{000000}{1-2}\endhead



\multicolumn{1}{>{\centering}m{\dimexpr 0.75in+0\tabcolsep}}{\textcolor[HTML]{000000}{\fontsize{11}{11}\selectfont{1}}} & \multicolumn{1}{>{\centering}m{\dimexpr 0.75in+0\tabcolsep}}{\textcolor[HTML]{000000}{\fontsize{11}{11}\selectfont{A}}} \\





\multicolumn{1}{>{\centering}m{\dimexpr 0.75in+0\tabcolsep}}{\textcolor[HTML]{000000}{\fontsize{11}{11}\selectfont{2}}} & \multicolumn{1}{>{\centering}m{\dimexpr 0.75in+0\tabcolsep}}{\textcolor[HTML]{000000}{\fontsize{11}{11}\selectfont{B}}} \\





\multicolumn{1}{>{\centering}m{\dimexpr 0.75in+0\tabcolsep}}{\textcolor[HTML]{000000}{\fontsize{11}{11}\selectfont{3}}} & \multicolumn{1}{>{\centering}m{\dimexpr 0.75in+0\tabcolsep}}{\textcolor[HTML]{000000}{\fontsize{11}{11}\selectfont{C}}} \\





\multicolumn{1}{>{\centering}m{\dimexpr 0.75in+0\tabcolsep}}{\textcolor[HTML]{000000}{\fontsize{11}{11}\selectfont{4}}} & \multicolumn{1}{>{\centering}m{\dimexpr 0.75in+0\tabcolsep}}{\textcolor[HTML]{000000}{\fontsize{11}{11}\selectfont{D}}} \\

\ascline{0.75pt}{000000}{1-2}



\end{longtable*}



\arrayrulecolor[HTML]{000000}

\global\setlength{\arrayrulewidth}{\Oldarrayrulewidth}

\global\setlength{\tabcolsep}{\Oldtabcolsep}

\renewcommand*{\arraystretch}{1}

{\vspace{-20pt}
\noindent \emph{Note.} The note below the table.}

\end{table}

To refer to this table in text, use the \texttt{@} symbol followed by
the reference label like so: As seen in Table~\ref{tbl-mytable}, the
first few numbers and letters of the alphabet are displayed.

In Table~\ref{tbl-mymarkdowntable}, there is an example of a plain
markdown table with a note below it.

\begin{table}

{\caption{{Table Caption of a Markdown
Table}{\label{tbl-mymarkdowntable}}}
\vspace{-20pt}}

\begin{longtable}[]{@{}llrc@{}}
\toprule\noalign{}
Default & Left & Right & Center \\
\midrule\noalign{}
\endhead
\bottomrule\noalign{}
\endlastfoot
12 & 12 & 12 & 12 \\
123 & 123 & 123 & 123 \\
1 & 1 & 1 & 1 \\
\end{longtable}

{\vspace{-20pt}
\noindent \emph{Note.} This is a note below the markdown table.}

\end{table}

What if you want the tables and figures to be at the end of the
document? In the .pdf format, you can set the \texttt{floatsintext}
option to false. For .html and .docx documents, there is not yet an
automatic way to put tables and figures at the end. You can, of course,
just put them all at the end, in order. The reference labels will work
no matter where they are in the text.

\subsection{Citations}\label{citations}

See
\href{https://quarto.org/docs/authoring/footnotes-and-citations.html}{here}
for instructions on setting up citations and references.

A parenthetical citation requires square brackets
(\textbf{CameronTrivedi2013?}). This reference was in my bibliography
file. An in-text citation is done like so:

(\textbf{CameronTrivedi2013?}) make some important points \ldots{}

See
\href{https://wjschne.github.io/apaquarto/writing.html\#references}{here}
for explanations, examples, and citation features exclusive to
apaquarto. For example, apaquarto can automatically handle possessive
citations:

(\textbf{schneider2012cattell?}) position was \ldots{}

\subsection{Masking Author Identity for Peer
Review}\label{masking-author-identity-for-peer-review}

Setting \texttt{mask} to \texttt{true} will remove author names,
affiliations, and correspondence from the title page. Any references
listed in the \texttt{masked-citations} field will be masked as well.
See
\href{https://wjschne.github.io/apaquarto/writing.html\#masked-citations-for-anonymous-peer-review}{here}
for more information.

\subsection{Hypotheses, Aims, and
Objectives}\label{hypotheses-aims-and-objectives}

The last paragraph of the introduction usually states the specific
hypotheses of the study, often in a way that links them to the research
design.

\section{Math and Equations}\label{math-and-equations}

Inline math uses \LaTeX syntax with single dollar signs. For example,
the reliability coefficient of my measure is \(r_{XX}=.95\).

If you want to display and refer to a specific formula, enclose the
formula in two dollar signs. After the second pair of dollar signs,
place the label in curly braces. The label should have an \texttt{\#eq-}
prefix. To refer to the formula, use the same label but with the
\texttt{@} symbol. For example, Equation~\ref{eq-euler} is Euler's
Identity, which is much admired for its elegance.

\begin{equation}\phantomsection\label{eq-euler}{
e^{i\pi}+1=0
}\end{equation}

\section{Method}\label{method}

General remarks on method. This paragraph is optional.

Not all papers require each of these sections. Edit them as needed.
Consult the \href{https://apastyle.apa.org/jars}{Journal Article
Reporting Standards} for what is needed for your type of article.

\subsection{Participants}\label{participants}

Who are they? How were they recruited? Report criteria for participant
inclusion and exclusion. Perhaps some basic demographic stats are in
order. A table is a great way to avoid repetition in statistical
reporting.

\subsection{Measures}\label{measures}

This section can also be titled \textbf{Materials} or
\textbf{Apparatus}. Whatever tools, equipment, or measurement devices
used in the study should be described.

\subsubsection{Measure A}\label{measure-a}

Describe Measure A.

\subsubsection{Measure B}\label{measure-b}

Describe Measure B.

\subsection{Procedure}\label{procedure}

What did participants do?

How are the data going to be analyzed?

\section{Results}\label{results}

\subsection{Descriptive Statistics}\label{descriptive-statistics}

Here we describe the basic characteristics of our primary variables.

\section{Discussion}\label{discussion}

Describe results in non-statistical terms.

\subsection{Limitations and Future
Directions}\label{limitations-and-future-directions}

Every study has limitations. Based on this study, some additional steps
might include\ldots{}

\subsection{Conclusion}\label{conclusion}

Let's sum this up.

\section{References}\label{references}

\phantomsection\label{refs}
\begin{CSLReferences}{0}{1}
\end{CSLReferences}

\appendix

\section{The Title for Appendix}\label{the-title-for-appendix}

If there are multiple appendices, label them with level 1 headings as
Appendix A, Appendix B, and so forth.

See Figure~\ref{fig-appendfig}.

\begin{figure}

{\caption{{Appendix Figure}{\label{fig-appendfig}}}}

\includegraphics{sampleimage.png}

{\noindent \emph{Note.} A note below the figure}

\end{figure}






\end{document}
