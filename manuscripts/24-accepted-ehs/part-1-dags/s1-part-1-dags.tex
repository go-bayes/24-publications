% Options for packages loaded elsewhere
\PassOptionsToPackage{unicode}{hyperref}
\PassOptionsToPackage{hyphens}{url}
\PassOptionsToPackage{dvipsnames,svgnames,x11names}{xcolor}
%
\documentclass[
  single column]{article}

\usepackage{amsmath,amssymb}
\usepackage{iftex}
\ifPDFTeX
  \usepackage[T1]{fontenc}
  \usepackage[utf8]{inputenc}
  \usepackage{textcomp} % provide euro and other symbols
\else % if luatex or xetex
  \usepackage{unicode-math}
  \defaultfontfeatures{Scale=MatchLowercase}
  \defaultfontfeatures[\rmfamily]{Ligatures=TeX,Scale=1}
\fi
\usepackage[]{libertinus}
\ifPDFTeX\else  
    % xetex/luatex font selection
\fi
% Use upquote if available, for straight quotes in verbatim environments
\IfFileExists{upquote.sty}{\usepackage{upquote}}{}
\IfFileExists{microtype.sty}{% use microtype if available
  \usepackage[]{microtype}
  \UseMicrotypeSet[protrusion]{basicmath} % disable protrusion for tt fonts
}{}
\makeatletter
\@ifundefined{KOMAClassName}{% if non-KOMA class
  \IfFileExists{parskip.sty}{%
    \usepackage{parskip}
  }{% else
    \setlength{\parindent}{0pt}
    \setlength{\parskip}{6pt plus 2pt minus 1pt}}
}{% if KOMA class
  \KOMAoptions{parskip=half}}
\makeatother
\usepackage{xcolor}
\usepackage[top=30mm,left=25mm,heightrounded,headsep=22pt,headheight=11pt,footskip=33pt,ignorehead,ignorefoot]{geometry}
\setlength{\emergencystretch}{3em} % prevent overfull lines
\setcounter{secnumdepth}{-\maxdimen} % remove section numbering
% Make \paragraph and \subparagraph free-standing
\makeatletter
\ifx\paragraph\undefined\else
  \let\oldparagraph\paragraph
  \renewcommand{\paragraph}{
    \@ifstar
      \xxxParagraphStar
      \xxxParagraphNoStar
  }
  \newcommand{\xxxParagraphStar}[1]{\oldparagraph*{#1}\mbox{}}
  \newcommand{\xxxParagraphNoStar}[1]{\oldparagraph{#1}\mbox{}}
\fi
\ifx\subparagraph\undefined\else
  \let\oldsubparagraph\subparagraph
  \renewcommand{\subparagraph}{
    \@ifstar
      \xxxSubParagraphStar
      \xxxSubParagraphNoStar
  }
  \newcommand{\xxxSubParagraphStar}[1]{\oldsubparagraph*{#1}\mbox{}}
  \newcommand{\xxxSubParagraphNoStar}[1]{\oldsubparagraph{#1}\mbox{}}
\fi
\makeatother


\providecommand{\tightlist}{%
  \setlength{\itemsep}{0pt}\setlength{\parskip}{0pt}}\usepackage{longtable,booktabs,array}
\usepackage{calc} % for calculating minipage widths
% Correct order of tables after \paragraph or \subparagraph
\usepackage{etoolbox}
\makeatletter
\patchcmd\longtable{\par}{\if@noskipsec\mbox{}\fi\par}{}{}
\makeatother
% Allow footnotes in longtable head/foot
\IfFileExists{footnotehyper.sty}{\usepackage{footnotehyper}}{\usepackage{footnote}}
\makesavenoteenv{longtable}
\usepackage{graphicx}
\makeatletter
\def\maxwidth{\ifdim\Gin@nat@width>\linewidth\linewidth\else\Gin@nat@width\fi}
\def\maxheight{\ifdim\Gin@nat@height>\textheight\textheight\else\Gin@nat@height\fi}
\makeatother
% Scale images if necessary, so that they will not overflow the page
% margins by default, and it is still possible to overwrite the defaults
% using explicit options in \includegraphics[width, height, ...]{}
\setkeys{Gin}{width=\maxwidth,height=\maxheight,keepaspectratio}
% Set default figure placement to htbp
\makeatletter
\def\fps@figure{htbp}
\makeatother
% definitions for citeproc citations
\NewDocumentCommand\citeproctext{}{}
\NewDocumentCommand\citeproc{mm}{%
  \begingroup\def\citeproctext{#2}\cite{#1}\endgroup}
\makeatletter
 % allow citations to break across lines
 \let\@cite@ofmt\@firstofone
 % avoid brackets around text for \cite:
 \def\@biblabel#1{}
 \def\@cite#1#2{{#1\if@tempswa , #2\fi}}
\makeatother
\newlength{\cslhangindent}
\setlength{\cslhangindent}{1.5em}
\newlength{\csllabelwidth}
\setlength{\csllabelwidth}{3em}
\newenvironment{CSLReferences}[2] % #1 hanging-indent, #2 entry-spacing
 {\begin{list}{}{%
  \setlength{\itemindent}{0pt}
  \setlength{\leftmargin}{0pt}
  \setlength{\parsep}{0pt}
  % turn on hanging indent if param 1 is 1
  \ifodd #1
   \setlength{\leftmargin}{\cslhangindent}
   \setlength{\itemindent}{-1\cslhangindent}
  \fi
  % set entry spacing
  \setlength{\itemsep}{#2\baselineskip}}}
 {\end{list}}
\usepackage{calc}
\newcommand{\CSLBlock}[1]{\hfill\break\parbox[t]{\linewidth}{\strut\ignorespaces#1\strut}}
\newcommand{\CSLLeftMargin}[1]{\parbox[t]{\csllabelwidth}{\strut#1\strut}}
\newcommand{\CSLRightInline}[1]{\parbox[t]{\linewidth - \csllabelwidth}{\strut#1\strut}}
\newcommand{\CSLIndent}[1]{\hspace{\cslhangindent}#1}

\input{/Users/joseph/GIT/latex/latex-for-quarto.tex}
\makeatletter
\@ifpackageloaded{caption}{}{\usepackage{caption}}
\AtBeginDocument{%
\ifdefined\contentsname
  \renewcommand*\contentsname{Table of contents}
\else
  \newcommand\contentsname{Table of contents}
\fi
\ifdefined\listfigurename
  \renewcommand*\listfigurename{List of Figures}
\else
  \newcommand\listfigurename{List of Figures}
\fi
\ifdefined\listtablename
  \renewcommand*\listtablename{List of Tables}
\else
  \newcommand\listtablename{List of Tables}
\fi
\ifdefined\figurename
  \renewcommand*\figurename{Figure}
\else
  \newcommand\figurename{Figure}
\fi
\ifdefined\tablename
  \renewcommand*\tablename{Table}
\else
  \newcommand\tablename{Table}
\fi
}
\@ifpackageloaded{float}{}{\usepackage{float}}
\floatstyle{ruled}
\@ifundefined{c@chapter}{\newfloat{codelisting}{h}{lop}}{\newfloat{codelisting}{h}{lop}[chapter]}
\floatname{codelisting}{Listing}
\newcommand*\listoflistings{\listof{codelisting}{List of Listings}}
\makeatother
\makeatletter
\makeatother
\makeatletter
\@ifpackageloaded{caption}{}{\usepackage{caption}}
\@ifpackageloaded{subcaption}{}{\usepackage{subcaption}}
\makeatother
\ifLuaTeX
  \usepackage{selnolig}  % disable illegal ligatures
\fi
\usepackage{bookmark}

\IfFileExists{xurl.sty}{\usepackage{xurl}}{} % add URL line breaks if available
\urlstyle{same} % disable monospaced font for URLs
\hypersetup{
  pdftitle={Supplementary files for ``Methods in Causal Inference Part 1: Causal Diagrams and Confounding''},
  pdfauthor={Joseph A. Bulbulia},
  colorlinks=true,
  linkcolor={blue},
  filecolor={Maroon},
  citecolor={Blue},
  urlcolor={Blue},
  pdfcreator={LaTeX via pandoc}}

\title{Supplementary files for ``Methods in Causal Inference Part 1:
Causal Diagrams and Confounding''}

\usepackage{academicons}
\usepackage{xcolor}

  \author{Joseph A. Bulbulia}
            \affil{%
             \small{     Victoria University of Wellington, New Zealand
          ORCID \textcolor[HTML]{A6CE39}{\aiOrcid} ~0000-0002-5861-2056 }
              }
      


\date{2024-06-19}
\begin{document}
\maketitle

\renewcommand*\contentsname{Table of contents}
{
\hypersetup{linkcolor=}
\setcounter{tocdepth}{2}
\tableofcontents
}
\listoftables
\newpage{}

\subsection{S1. Glossary}\label{id-app-a}

\begin{table}

\caption{\label{tbl-experiments}Glossary}

\centering{

\glossaryTerms

}

\end{table}%

\newpage{}

\subsection{S2. Causal Inference in History: The Difficulty in
Satisfying the Three Fundamental Assumptions for Causal
Inference}\label{id-app-b}

Consider the Protestant Reformation of the 16th century, which initiated
religious change throughout much of Europe. Historians have argued that
Protestantism caused social, cultural, and economic changes in those
societies where it took hold (see: Weber
(\citeproc{ref-weber1905}{1905}); Weber
(\citeproc{ref-weber1993}{1993}); Swanson
(\citeproc{ref-swanson1967}{1967}); Swanson
(\citeproc{ref-swanson1971}{1971}); Basten \& Betz
(\citeproc{ref-basten2013}{2013}), and for an overview, see: Becker et
al. (\citeproc{ref-becker2016}{2016})).

Suppose we want to estimate the Protestant Reformation's `Average
Treatment Effect'. Let \(A = a^*\) denote the adoption of Protestantism.
We compare this effect with that of remaining Catholic, represented as
\(A = a\). We assume that both the concepts of `adopting Protestantism'
and `economic development' are well-defined (e.g., GDP +1 century after
a country has a Protestant majority contrasted with remaining Catholic).
The causal effect for any individual country is \(Y_i(a^*) - Y_i(a)\).
Although we cannot identify this effect, if the basic assumptions of
causal inference are met, we can estimate the average or marginal effect
by conditioning on the confounding effects of \(L\):

\[
ATE_{\textnormal{economic~development}} = \mathbb{E}[Y(\textnormal{Became~Protestant}|L) - Y(\textnormal{Remained~Catholic}|L)]
\]

When asking causal questions about the economic effect of adopting
Protestantism versus remaining Catholic, several challenges arise
regarding the three fundamental assumptions required for causal
inference.

\textbf{Causal Consistency}: This requires that the outcome under each
level of treatment to be compared is well-defined. In this context,
defining what `adopting Protestantism' and `remaining Catholic' mean may
present challenges. The practices and beliefs of each religion might
vary significantly across countries and time periods, making it
difficult to create a consistent, well-defined treatment. Furthermore,
the outcome---economic development---may also be challenging to measure
consistently across different countries and time periods.

There is undoubtedly considerable heterogeneity in the `Protestant
treatment.' In England, Protestantism was closely tied to the monarchy
(\citeproc{ref-collinson2003}{Collinson, 2003}). In Germany, Martin
Luther's teachings emphasised individual faith in scripture, which, it
has been claimed, supported economic development by promoting literacy
(\citeproc{ref-gawthrop1984}{Gawthrop \& Strauss, 1984}). In England,
King Henry VIII abolished Catholicism
(\citeproc{ref-collinson2003}{Collinson, 2003}). The Reformation, then,
occurred differently in different places. The treatment needs to be
better defined.

There is also ample scope for interference: 16th-century societies were
interconnected through trade, diplomacy, and warfare. Thus, the
religious decisions of one society were unlikely to have been
independent from those of other societies.

\textbf{Exchangeability}: This requires that given the confounders, the
potential outcomes are independent of the treatment assignment. It might
be difficult to account for all possible confounders in this context.
For example, historical, political, social, and geographical factors
could influence both a country's religious affiliations and its economic
development.

\textbf{Positivity}: This requires that there is a non-zero probability
of every level of treatment for every stratum of confounders. If we
consider various confounding factors such as geographical location,
historical events, or political circumstances, some countries might only
ever have the possibility of either remaining Catholic or becoming
Protestant, but not both. For example, it is unclear under which
conditions 16th-century Spain could have been randomly assigned to
Protestantism (\citeproc{ref-nalle1987}{Nalle, 1987};
\citeproc{ref-westreich2010}{Westreich \& Cole, 2010}).

Perhaps a more credible measure of effect in the region of our interests
is the Average Treatment Effect in the Treated (ATT) expressed:

\[
ATT_{\textnormal{economic~development}} = \mathbb{E}[(Y(a^*) - Y(a))|A = a^*,L]
\]

Where \(Y(a^*)\) represents the potential outcome if treated, and
\(Y(a)\) represents the potential outcome if not treated. The
expectation is taken over the distribution of the treated units (i.e.,
those for whom \(A = a^*\)). \(L\) is a set of covariates on which we
condition to ensure that the potential outcomes \(Y(a^*)\) and \(Y(a)\)
are independent of the treatment assignment \(A\), given \(L\). This
accounts for any confounding factors that might bias the estimate of the
treatment effect.

Here, the ATT defines the expected difference in economic success for
cultures that became Protestant compared with the expected economic
success if those cultures had not become Protestant, conditional on
measured confounders \(L\), among the exposed (\(A = a^*\)). To estimate
this contrast, our models would need to match Protestant cultures with
comparable Catholic cultures effectively. By estimating the ATT, we
avoid the assumption of non-deterministic positivity for the untreated.
However, whether matching is conceptually plausible remains debatable.
Ostensibly, it would seem that assigning a religion to a culture is not
as easy as administering a pill (\citeproc{ref-watts2018}{Watts et al.,
2018}).

\newpage{}

\subsection{S3. Causal Consistency Under Multiple Versions of
Treatment}\label{id-app-c}

To better understand how the causal consistency assumption might fail,
consider a question discussed in the evolutionary human science
literature about whether a society's beliefs in big Gods affect its
development of social complexity (\citeproc{ref-beheim2021}{Beheim et
al., 2021}; \citeproc{ref-johnson2015}{Johnson, 2015};
\citeproc{ref-norenzayan2016}{Norenzayan et al., 2016};
\citeproc{ref-sheehan2022}{Sheehan et al., 2022};
\citeproc{ref-slingerland2020coding}{Slingerland et al., 2020};
\citeproc{ref-watts2015}{Watts et al., 2015};
\citeproc{ref-whitehouse2023}{Whitehouse et al., 2023}). Historians and
anthropologists report that such beliefs vary over time and across
cultures in intensity, interpretations, institutional management, and
rituals (\citeproc{ref-bulbuliaj.2013}{Bulbulia, J. et al., 2013};
\citeproc{ref-decoulanges1903}{De Coulanges, 1903};
\citeproc{ref-geertz2013}{Geertz et al., 2013};
\citeproc{ref-wheatley1971}{Wheatley, 1971}). This variation in content
and settings could influence social complexity. Moreover, the treatments
realised in one society might affect those realised in other societies,
resulting in \emph{spill-over} effects in the exposures (`treatments')
to be compared (\citeproc{ref-murray2021a}{Murray et al., 2021};
\citeproc{ref-shiba2023uncovering}{Shiba et al., 2023}).

The theory of causal inference under multiple versions of treatment,
developed by VanderWeele and Hernán, formally addresses this challenge
of treatment-effect heterogeneity
(\citeproc{ref-vanderweele2009}{VanderWeele, 2009},
\citeproc{ref-vanderweele2018}{2018};
\citeproc{ref-vanderweele2013}{VanderWeele \& Hernan, 2013}). The
authors proved that if the treatment variations, \(K\), are
conditionally independent of the potential outcomes, \(Y(k)\), given
covariates \(L\), then conditioning on \(L\) allows us to consistently
estimate causal effects over the heterogeneous treatments
(\citeproc{ref-vanderweele2009}{VanderWeele, 2009}).

Where \(\coprod\) denotes independence, we may assume causal consistency
where the interventions to be compared are independent of their
potential outcomes, conditional on covariates, \(L\):

\[
K \coprod Y(k) | L
\]

According to the theory of causal inference under multiple versions of
treatment, we may think of \(K\) as a `coarsened indicator' for \(A\).
Although the theory of causal inference under multiple versions of
treatment provides a formal solution to the problems of treatment-effect
heterogeneity and treatment-effect dependencies (also known as
SUTVA---the `stable unit treatment value assumption'; refer to Rubin
(\citeproc{ref-rubin1980randomization}{1980})), computing and
interpreting causal effect estimates under this theory can be
challenging.

Consider the question of whether a reduction in Body Mass Index (BMI)
affects health (\citeproc{ref-hernuxe1n2008}{Hernán \& Taubman, 2008}).
Weight loss can occur through various methods, each with different
health implications. Specific methods, such as regular exercise or a
calorie-reduced diet, benefit health. However, weight loss might result
from adverse conditions such as infectious diseases, cancers,
depression, famine, or accidental amputations, which are generally not
beneficial to health. Hence, even if causal effects of `weight loss'
could be consistently estimated when adjusting for covariates \(L\), we
might be uncertain about how to interpret the effect we are consistently
estimating. This uncertainty highlights the need for precise and
well-defined causal questions. For example, rather than stating the
intervention vaguely as `weight loss', we could state the intervention
clearly and specifically as `weight loss achieved through aerobic
exercise over at least five years, compared with no weight loss.' This
specificity in the definition of the treatment, along with comparable
specificity in the statement of the outcomes, helps ensure that the
causal estimates we obtain are not merely unbiased but also
interpretable; for discussion, see Hernán et al.
(\citeproc{ref-hernuxe1n2022}{2022}); Murray et al.
(\citeproc{ref-murray2021a}{2021}); Hernán \& Taubman
(\citeproc{ref-hernuxe1n2008}{2008}).

Beyond uncertainties for the interpretation of heterogeneous treatment
effect estimates, there is the additional consideration that we cannot
fully verify from data whether the measured covariates \(L\) suffice to
render the multiple versions of treatment independent of the
counterfactual outcomes. This problem is acute when there is
\emph{interference}, which occurs when treatment effects are relative to
the density and distribution of treatment effects in a population. Scope
for interference will often make it difficult to warrant the assumption
that the potential outcomes are independent of the many versions of
treatment that have been realised, dependently, on the administration of
previous versions of treatments across the population
(\citeproc{ref-bulbulia2023a}{Bulbulia et al., 2023};
\citeproc{ref-ogburn2022}{Ogburn et al., 2022};
\citeproc{ref-vanderweele2013}{VanderWeele \& Hernan, 2013}).

In short, although the theory of causal inference under multiple
versions of treatment provides a formal solution for consistent causal
effect estimation in observational settings, \emph{treatment
heterogeneity} remains a practical threat. Generally, we should assume
that causal consistency is unrealistic unless proven innocent.

For now, we note that the causal consistency assumption provides a
theoretical starting point for recovering the missing counterfactuals
required for computing causal contrasts. It identifies half of these
missing counterfactuals directly from observed data. The concept of
conditional exchangeability, which we examine next, offers a means for
recovering the remaining half.

\newpage{}

\subsection{References}\label{references}

\phantomsection\label{refs}
\begin{CSLReferences}{1}{0}
\bibitem[\citeproctext]{ref-basten2013}
Basten, C., \& Betz, F. (2013). Beyond work ethic: Religion, individual,
and political preferences. \emph{American Economic Journal: Economic
Policy}, \emph{5}(3), 67--91. \url{https://doi.org/10.1257/pol.5.3.67}

\bibitem[\citeproctext]{ref-becker2016}
Becker, S. O., Pfaff, S., \& Rubin, J. (2016). Causes and consequences
of the protestant reformation. \emph{Explorations in Economic History},
\emph{62}, 1--25.

\bibitem[\citeproctext]{ref-beheim2021}
Beheim, B., Atkinson, Q. D., Bulbulia, J., Gervais, W., Gray, R. D.,
Henrich, J., Lang, M., Monroe, M. W., Muthukrishna, M., Norenzayan, A.,
Purzycki, B. G., Shariff, A., Slingerland, E., Spicer, R., \& Willard,
A. K. (2021). Treatment of missing data determined conclusions regarding
moralizing gods. \emph{Nature}, \emph{595}(7866), E29--E34.
\url{https://doi.org/10.1038/s41586-021-03655-4}

\bibitem[\citeproctext]{ref-bulbulia2023a}
Bulbulia, J. A., Afzali, M. U., Yogeeswaran, K., \& Sibley, C. G.
(2023). Long-term causal effects of far-right terrorism in {N}ew
{Z}ealand. \emph{PNAS Nexus}, \emph{2}(8), pgad242.

\bibitem[\citeproctext]{ref-bulbuliaj.2013}
Bulbulia, J., Geertz, A. W., Atkinson, Q. D., Cohen, E., Evans, N.,
Francois, P., Gintis, H., Gray, R. D., Henrich, J., Jordon, F. M.,
Norenzayan, A., Richerson, P. J., Slingerland, E., Turchin, P.,
Whitehouse, H., Widlok, T., \& Wilson, D. S. (2013). \emph{The cultural
evolution of religion} (P. J. Richerson \& M. Christiansen, Eds.; pp.
381--404). MIT press.

\bibitem[\citeproctext]{ref-collinson2003}
Collinson, P. (2003). \emph{The reformation: A history}. Weidenfeld;
Nicholson; London, England.

\bibitem[\citeproctext]{ref-decoulanges1903}
De Coulanges, F. (1903). \emph{La cité antique: Étude sur le culte, le
droit, les institutions de la grèce et de rome}. Hachette.

\bibitem[\citeproctext]{ref-gawthrop1984}
Gawthrop, R., \& Strauss, G. (1984). Protestantism and literacy in early
modern germany. \emph{Past \& Present}, \emph{104}, 31--55.

\bibitem[\citeproctext]{ref-geertz2013}
Geertz, A. W., Atkinson, Q. D., Cohen, E., Evans, N., Francois, P.,
Gintis, H., Gray, R. D., Henrich, J., Jordon, F. M., Norenzayan, A.,
Richerson, P. J., Slingerland, E., Turchin, P., Whitehouse, H., Widlok,
T., \& Wilson, D. S. (2013). \emph{The cultural evolution of religion}
(P. J. Richerson \& M. Christiansen, Eds.; pp. 381--404). MIT press.

\bibitem[\citeproctext]{ref-hernuxe1n2008}
Hernán, M. A., \& Taubman, S. L. (2008). Does obesity shorten life? The
importance of well-defined interventions to answer causal questions.
\emph{International Journal of Obesity (2005)}, \emph{32 Suppl 3},
S8--14. \url{https://doi.org/10.1038/ijo.2008.82}

\bibitem[\citeproctext]{ref-hernuxe1n2022}
Hernán, M. A., Wang, W., \& Leaf, D. E. (2022). Target trial emulation:
A framework for causal inference from observational data. \emph{JAMA},
\emph{328}(24), 2446--2447.
\url{https://doi.org/10.1001/jama.2022.21383}

\bibitem[\citeproctext]{ref-johnson2015}
Johnson, D. D. (2015). Big gods, small wonder: Supernatural punishment
strikes back. \emph{Religion, Brain \& Behavior}, \emph{5}(4), 290--298.

\bibitem[\citeproctext]{ref-murray2021a}
Murray, E. J., Marshall, B. D. L., \& Buchanan, A. L. (2021). Emulating
target trials to improve causal inference from agent-based models.
\emph{American Journal of Epidemiology}, \emph{190}(8), 1652--1658.
\url{https://doi.org/10.1093/aje/kwab040}

\bibitem[\citeproctext]{ref-nalle1987}
Nalle, S. T. (1987). Inquisitors, priests, and the people during the
catholic reformation in spain. \emph{The Sixteenth Century Journal},
557--587.

\bibitem[\citeproctext]{ref-norenzayan2016}
Norenzayan, A., Shariff, A. F., Gervais, W. M., Willard, A. K.,
McNamara, R. A., Slingerland, E., \& Henrich, J. (2016). The cultural
evolution of prosocial religions. \emph{Behavioral and Brain Sciences},
\emph{39}, e1. \url{https://doi.org/10.1017/S0140525X14001356}

\bibitem[\citeproctext]{ref-ogburn2022}
Ogburn, E. L., Sofrygin, O., Díaz, I., \& Laan, M. J. van der. (2022).
Causal inference for social network data. \emph{Journal of the American
Statistical Association}, \emph{0}(0), 1--15.
\url{https://doi.org/10.1080/01621459.2022.2131557}

\bibitem[\citeproctext]{ref-rubin1980randomization}
Rubin, D. B. (1980). Randomization analysis of experimental data: The
fisher randomization test comment. \emph{Journal of the American
Statistical Association}, \emph{75}(371), 591--593.
\url{https://doi.org/10.2307/2287653}

\bibitem[\citeproctext]{ref-sheehan2022}
Sheehan, O., Watts, J., Gray, R. D., Bulbulia, J., Claessens, S.,
Ringen, E. J., \& Atkinson, Q. D. (2022). Coevolution of religious and
political authority in austronesian societies. \emph{Nature Human
Behaviour}. \url{https://doi.org/10.1038/s41562-022-01471-y}

\bibitem[\citeproctext]{ref-shiba2023uncovering}
Shiba, K., Daoud, A., Hikichi, H., Yazawa, A., Aida, J., Kondo, K., \&
Kawachi, I. (2023). Uncovering heterogeneous associations between
disaster-related trauma and subsequent functional limitations: A
machine-learning approach. \emph{American Journal of Epidemiology},
\emph{192}(2), 217--229.

\bibitem[\citeproctext]{ref-slingerland2020coding}
Slingerland, E., Atkinson, Q. D., Ember, C. R., Sheehan, O.,
Muthukrishna, M., Bulbulia, J., \& Gray, R. D. (2020). Coding culture:
Challenges and recommendations for comparative cultural databases.
\emph{Evolutionary Human Sciences}, \emph{2}, e29.

\bibitem[\citeproctext]{ref-swanson1967}
Swanson, G. E. (1967). \emph{Religion and regime: A sociological account
of the {R}eformation}.

\bibitem[\citeproctext]{ref-swanson1971}
Swanson, G. E. (1971). Interpreting the reformation. \emph{The Journal
of Interdisciplinary History}, \emph{1}(3), 419--446.
\url{http://www.jstor.org/stable/202620}

\bibitem[\citeproctext]{ref-vanderweele2009}
VanderWeele, T. J. (2009). Concerning the consistency assumption in
causal inference. \emph{Epidemiology}, \emph{20}(6), 880.
\url{https://doi.org/10.1097/EDE.0b013e3181bd5638}

\bibitem[\citeproctext]{ref-vanderweele2018}
VanderWeele, T. J. (2018). On well-defined hypothetical interventions in
the potential outcomes framework. \emph{Epidemiology}, \emph{29}(4),
e24. \url{https://doi.org/10.1097/EDE.0000000000000823}

\bibitem[\citeproctext]{ref-vanderweele2013}
VanderWeele, T. J., \& Hernan, M. A. (2013). Causal inference under
multiple versions of treatment. \emph{Journal of Causal Inference},
\emph{1}(1), 1--20.

\bibitem[\citeproctext]{ref-watts2015}
Watts, J., Greenhill, S. J., Atkinson, Q. D., Currie, T. E., Bulbulia,
J., \& Gray, R. D. (2015). Broad supernatural punishment but not
moralizing high gods precede the evolution of political complexity in
{A}ustronesia. In \emph{Proceedings of the Royal Society B: Biological
Sciences} (Vol. 282, p. 20142556). The Royal Society.

\bibitem[\citeproctext]{ref-watts2018}
Watts, J., Sheehan, O., Bulbulia, Joseph A, Gray, R. D., \& Atkinson, Q.
D. (2018). Christianity spread faster in small, politically structured
societies. \emph{Nature Human Behaviour}, \emph{2}(8), 559--564.
\url{https://doi.org/gdvnjn}

\bibitem[\citeproctext]{ref-weber1905}
Weber, M. (1905). \emph{The protestant ethic and the spirit of
capitalism: And other writings}. Penguin.

\bibitem[\citeproctext]{ref-weber1993}
Weber, M. (1993). \emph{The sociology of religion}. Beacon Press.

\bibitem[\citeproctext]{ref-westreich2010}
Westreich, D., \& Cole, S. R. (2010). Invited commentary: positivity in
practice. \emph{American Journal of Epidemiology}, \emph{171}(6).
\url{https://doi.org/10.1093/aje/kwp436}

\bibitem[\citeproctext]{ref-wheatley1971}
Wheatley, P. (1971). \emph{The pivot of the four quarters : A
preliminary enquiry into the origins and character of the ancient
chinese city}. Edinburgh University Press.
\url{https://cir.nii.ac.jp/crid/1130000795717727104}

\bibitem[\citeproctext]{ref-whitehouse2023}
Whitehouse, H., Francois, P., Savage, P. E., Hoyer, D., Feeney, K. C.,
Cioni, E., Purcell, R., Larson, J., Baines, J., Haar, B. ter, Covey, A.,
\& Turchin, P. (2023). Testing the big gods hypothesis with global
historical data: A review and retake. \emph{Religion, Brain \&
Behavior}, \emph{13}(2), 124--166.

\end{CSLReferences}



\end{document}
