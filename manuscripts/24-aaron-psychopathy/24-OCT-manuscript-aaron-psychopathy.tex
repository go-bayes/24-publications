% Options for packages loaded elsewhere
\PassOptionsToPackage{unicode}{hyperref}
\PassOptionsToPackage{hyphens}{url}
\PassOptionsToPackage{dvipsnames,svgnames,x11names}{xcolor}
%
\documentclass[
  singlecolumn]{article}

\usepackage{amsmath,amssymb}
\usepackage{iftex}
\ifPDFTeX
  \usepackage[T1]{fontenc}
  \usepackage[utf8]{inputenc}
  \usepackage{textcomp} % provide euro and other symbols
\else % if luatex or xetex
  \usepackage{unicode-math}
  \defaultfontfeatures{Scale=MatchLowercase}
  \defaultfontfeatures[\rmfamily]{Ligatures=TeX,Scale=1}
\fi
\usepackage[]{libertinus}
\ifPDFTeX\else  
    % xetex/luatex font selection
\fi
% Use upquote if available, for straight quotes in verbatim environments
\IfFileExists{upquote.sty}{\usepackage{upquote}}{}
\IfFileExists{microtype.sty}{% use microtype if available
  \usepackage[]{microtype}
  \UseMicrotypeSet[protrusion]{basicmath} % disable protrusion for tt fonts
}{}
\makeatletter
\@ifundefined{KOMAClassName}{% if non-KOMA class
  \IfFileExists{parskip.sty}{%
    \usepackage{parskip}
  }{% else
    \setlength{\parindent}{0pt}
    \setlength{\parskip}{6pt plus 2pt minus 1pt}}
}{% if KOMA class
  \KOMAoptions{parskip=half}}
\makeatother
\usepackage{xcolor}
\usepackage[top=30mm,left=20mm,heightrounded]{geometry}
\setlength{\emergencystretch}{3em} % prevent overfull lines
\setcounter{secnumdepth}{-\maxdimen} % remove section numbering
% Make \paragraph and \subparagraph free-standing
\makeatletter
\ifx\paragraph\undefined\else
  \let\oldparagraph\paragraph
  \renewcommand{\paragraph}{
    \@ifstar
      \xxxParagraphStar
      \xxxParagraphNoStar
  }
  \newcommand{\xxxParagraphStar}[1]{\oldparagraph*{#1}\mbox{}}
  \newcommand{\xxxParagraphNoStar}[1]{\oldparagraph{#1}\mbox{}}
\fi
\ifx\subparagraph\undefined\else
  \let\oldsubparagraph\subparagraph
  \renewcommand{\subparagraph}{
    \@ifstar
      \xxxSubParagraphStar
      \xxxSubParagraphNoStar
  }
  \newcommand{\xxxSubParagraphStar}[1]{\oldsubparagraph*{#1}\mbox{}}
  \newcommand{\xxxSubParagraphNoStar}[1]{\oldsubparagraph{#1}\mbox{}}
\fi
\makeatother


\providecommand{\tightlist}{%
  \setlength{\itemsep}{0pt}\setlength{\parskip}{0pt}}\usepackage{longtable,booktabs,array}
\usepackage{calc} % for calculating minipage widths
% Correct order of tables after \paragraph or \subparagraph
\usepackage{etoolbox}
\makeatletter
\patchcmd\longtable{\par}{\if@noskipsec\mbox{}\fi\par}{}{}
\makeatother
% Allow footnotes in longtable head/foot
\IfFileExists{footnotehyper.sty}{\usepackage{footnotehyper}}{\usepackage{footnote}}
\makesavenoteenv{longtable}
\usepackage{graphicx}
\makeatletter
\def\maxwidth{\ifdim\Gin@nat@width>\linewidth\linewidth\else\Gin@nat@width\fi}
\def\maxheight{\ifdim\Gin@nat@height>\textheight\textheight\else\Gin@nat@height\fi}
\makeatother
% Scale images if necessary, so that they will not overflow the page
% margins by default, and it is still possible to overwrite the defaults
% using explicit options in \includegraphics[width, height, ...]{}
\setkeys{Gin}{width=\maxwidth,height=\maxheight,keepaspectratio}
% Set default figure placement to htbp
\makeatletter
\def\fps@figure{htbp}
\makeatother
% definitions for citeproc citations
\NewDocumentCommand\citeproctext{}{}
\NewDocumentCommand\citeproc{mm}{%
  \begingroup\def\citeproctext{#2}\cite{#1}\endgroup}
\makeatletter
 % allow citations to break across lines
 \let\@cite@ofmt\@firstofone
 % avoid brackets around text for \cite:
 \def\@biblabel#1{}
 \def\@cite#1#2{{#1\if@tempswa , #2\fi}}
\makeatother
\newlength{\cslhangindent}
\setlength{\cslhangindent}{1.5em}
\newlength{\csllabelwidth}
\setlength{\csllabelwidth}{3em}
\newenvironment{CSLReferences}[2] % #1 hanging-indent, #2 entry-spacing
 {\begin{list}{}{%
  \setlength{\itemindent}{0pt}
  \setlength{\leftmargin}{0pt}
  \setlength{\parsep}{0pt}
  % turn on hanging indent if param 1 is 1
  \ifodd #1
   \setlength{\leftmargin}{\cslhangindent}
   \setlength{\itemindent}{-1\cslhangindent}
  \fi
  % set entry spacing
  \setlength{\itemsep}{#2\baselineskip}}}
 {\end{list}}
\usepackage{calc}
\newcommand{\CSLBlock}[1]{\hfill\break\parbox[t]{\linewidth}{\strut\ignorespaces#1\strut}}
\newcommand{\CSLLeftMargin}[1]{\parbox[t]{\csllabelwidth}{\strut#1\strut}}
\newcommand{\CSLRightInline}[1]{\parbox[t]{\linewidth - \csllabelwidth}{\strut#1\strut}}
\newcommand{\CSLIndent}[1]{\hspace{\cslhangindent}#1}

\usepackage{booktabs}
\usepackage{longtable}
\usepackage{array}
\usepackage{multirow}
\usepackage{wrapfig}
\usepackage{float}
\usepackage{colortbl}
\usepackage{pdflscape}
\usepackage{tabu}
\usepackage{threeparttable}
\usepackage{threeparttablex}
\usepackage[normalem]{ulem}
\usepackage{makecell}
\usepackage{xcolor}
\input{/Users/joseph/GIT/latex/latex-for-quarto.tex}
\makeatletter
\@ifpackageloaded{caption}{}{\usepackage{caption}}
\AtBeginDocument{%
\ifdefined\contentsname
  \renewcommand*\contentsname{Table of contents}
\else
  \newcommand\contentsname{Table of contents}
\fi
\ifdefined\listfigurename
  \renewcommand*\listfigurename{List of Figures}
\else
  \newcommand\listfigurename{List of Figures}
\fi
\ifdefined\listtablename
  \renewcommand*\listtablename{List of Tables}
\else
  \newcommand\listtablename{List of Tables}
\fi
\ifdefined\figurename
  \renewcommand*\figurename{Figure}
\else
  \newcommand\figurename{Figure}
\fi
\ifdefined\tablename
  \renewcommand*\tablename{Table}
\else
  \newcommand\tablename{Table}
\fi
}
\@ifpackageloaded{float}{}{\usepackage{float}}
\floatstyle{ruled}
\@ifundefined{c@chapter}{\newfloat{codelisting}{h}{lop}}{\newfloat{codelisting}{h}{lop}[chapter]}
\floatname{codelisting}{Listing}
\newcommand*\listoflistings{\listof{codelisting}{List of Listings}}
\makeatother
\makeatletter
\makeatother
\makeatletter
\@ifpackageloaded{caption}{}{\usepackage{caption}}
\@ifpackageloaded{subcaption}{}{\usepackage{subcaption}}
\makeatother

\ifLuaTeX
  \usepackage{selnolig}  % disable illegal ligatures
\fi
\usepackage{bookmark}

\IfFileExists{xurl.sty}{\usepackage{xurl}}{} % add URL line breaks if available
\urlstyle{same} % disable monospaced font for URLs
\hypersetup{
  pdftitle={Causal effect of Psychopathy on Partner Well-Being: A National Longitudinal Study},
  pdfauthor={Aaron Hissey; Hedwig Eisenbarth; Chris G. Sibley; Matthew Hammond; Joseph A. Bulbulia},
  pdfkeywords={Causal
Inference, Relationships, Panel, Psychopathy, Personality, Well-being, Outcome-wide},
  colorlinks=true,
  linkcolor={blue},
  filecolor={Maroon},
  citecolor={Blue},
  urlcolor={Blue},
  pdfcreator={LaTeX via pandoc}}


\title{Causal effect of Psychopathy on Partner Well-Being: A National
Longitudinal Study}
\author{Aaron Hissey \and Hedwig Eisenbarth \and Chris G.
Sibley \and Matthew Hammond \and Joseph A. Bulbulia}
\date{2024-10-25}

\begin{document}
\maketitle


\subsection{Introduction}\label{introduction}

A central question in psychopathy research is how it affects
relationships.

To address this question, we develop an outcome-wide longitudinal study
using a national longitudinal probability sample in New Zealand.

\subsection{Method}\label{method}

\subsubsection{Sample}\label{sample}

Data were collected as part of The New Zealand Attitudes and Values
Study (NZAVS) is an annual longitudinal national probability panel study
of social attitudes, personality, ideology and health outcomes. The
NZAVS began in 2009. It includes questionnaire responses from over
72,000 New Zealand residents. The study includes researchers from many
New Zealand universities, including the University of Auckland, Victoria
University of Wellington, the University of Canterbury, the University
of Otago, and Waikato University. Because the survey asks the same
people to respond each year, it can track subtle changes in attitudes
and values over time and is an important resource for researchers in New
Zealand and worldwide. The NZAVS is university-based, not-for-profit and
independent of political or corporate
funding.https://doi.org/10.17605/OSF.IO/75SNB

\hyperref[appendix-demographics]{Appendix A} provides information about
demographic covariates used for confounding control (NZAVS time 10,
years 2018-2019).

\subsubsection{Eligibility criteria}\label{eligibility-criteria}

The sample consisted of respondents to NZAVS times 10 (baseline, years
2018-2019), time 11 (treatment wave, years 2019-2020), and time 12
(outcome wave) (years 2020-2021).

\begin{itemize}
\tightlist
\item
  Participants who were identified as being in couples.\\
\item
  Participants who provided full information to the psychopathy measures
  in the 2018 and 2019. Participants may have been lost to follow up in
  the outcome wave 2020.
\item
  Missing data for all variables at baseline were allowed. Missing data
  at baseline were imputed through the \texttt{mice} package
  (\citeproc{ref-vanbuuren2018}{Van Buuren 2018}), and an extra column
  indicator was included to denote whether a variable was missing,
  following protocols described by (\citeproc{ref-williams2021}{Williams
  and Díaz 2021}).
\item
  Inverse probability of censoring weights were calculated as part of
  estimation in \texttt{lmtp} to adjust for missing outcomes at NZAVS
  Time 12 (years 2020-2021, the outcome wave
  (\citeproc{ref-williams2021}{Williams and Díaz 2021}))
\end{itemize}

There were 1,012 NZAVS participants who met these criteria in 506
couples.

\subsubsection{Causal Contrast}\label{causal-contrast}

Here, we leverage panel data to estimate potential outcomes in a
hypothetical world in which partners in couples were shifted up or down
on a one-point scale for psychopathic personality.

Our causal estimand takes a ``shift function'' or ``modified treatment
policy'':

\[ \text{Average Treatment Effect} = E\{(E[Y(\text{Shift up})|A,L) - (E[Y(\text{Shift down})|A,L)\} \]

\textbf{Intervention Functionals:}

\begin{itemize}
\tightlist
\item
  \textbf{Shift up by +1 point:}
\end{itemize}

\[
\text{Shift up} = f(A = a) = 
\begin{cases}
  a^* & \text{if the exposure is shifted up by +1 point} \\
  \tilde{a} & \text{if the observed value of the exposure is used}
\end{cases}
\]

\begin{itemize}
\tightlist
\item
  \textbf{Shift down by -1 point:}
\end{itemize}

\[
\text{Shift down} = f(A = a) = 
\begin{cases}
  a^* & \text{if the exposure is shifted down by -1 point} \\
  \tilde{a} & \text{if the observed value of the exposure is used}
\end{cases}
\]

Our primary interest is in the contrast of these intervention
functionals. \{Appendix XX{]}(\#appendix-XX) reports contrssts for each
intervention against the NULL of no intervention.

\subsubsection{Identification
Assumptions}\label{identification-assumptions}

To consistently estimate causal effects, we rely on three key
assumptions:

\begin{enumerate}
\def\labelenumi{\arabic{enumi}.}
\item
  \textbf{Causal Consistency:} potential outcomes must align with
  observed outcomes under the treatments in our data. Essentially, we
  assume potential outcomes do not depend on the specific way treatment
  was administered, as long as we consider measured covariates.
\item
  \textbf{Conditional Exchangeability:} given the observed covariates,
  we assume treatment assignment is independent of potential outcomes.
  In simpler terms, this means ``no unmeasured confounding'' -- any
  factors influencing both treatment assignment and outcomes must be
  included in our measured covariates.
\item
  \textbf{Positivity:} for unbiased estimation, every subject must have
  a non-zero chance of receiving the treatment, regardless of their
  covariate values. We evaluate this assumption in each study by
  examining changes in psychopathy ``treatments'' from baseline (NZAVS
  time 10) to the treatment wave (NZAVS time 11). It is the initiation
  of a psychopathic trait and their combinations, whose causal effect we
  seek to quantify
\end{enumerate}

\subsubsection{Analysis}\label{analysis}

Criteria for analysis were pre-specified and pre-registered in advance
of the analysis at:\url{https://osf.io/ce4t9/}, and followed standard
NZAVS protocols for three-wave causal inference designs
(\citeproc{ref-bulbulia2024PRACTICAL}{Bulbulia 2024}).

\subsubsection{Confounding Control
Strategy}\label{confounding-control-strategy}

We followed VanderWeele \emph{et al.}
(\citeproc{ref-vanderweele2020}{2020})'s \emph{minimally modified
disjunctive criteria} for confounding control.

\begin{enumerate}
\def\labelenumi{\arabic{enumi}.}
\item
  \textbf{Initial identify confounders}: using causal diagrams, we began
  by enumerating all covariates that may influence either the treatment
  (exposure) or outcomes, spanning five domains. This includes variables
  directly affecting the exposure or outcome, as well as potential
  consequences of these variables (i.e.~proxies.)
\item
  \textbf{Remove instrumental variables}: we removed variables that are
  identified as instrumental variables, i.e., those influencing the
  exposure but not the outcome. Their inclusion can reduce the
  efficiency of the analysis.
\item
  \textbf{Inclusion of proxy variables}: for unmeasured variables that
  affect both the exposure and outcome, include proxy variables wherever
  possible. These proxies act as indicators for the unmeasured common
  causes. \hyperref[appendix-demographics]{Appendix B} lists covariates
  we used for confounding control. These protocols follow advice in
  (\citeproc{ref-bulbulia2024PRACTICAL}{Bulbulia 2024}) as prespecified
  in \url{https://osf.io/ce4t9/}.
\item
  \textbf{Baseline exposure control}: we included baseline measures of
  the outcome to adjust for unmeasured confounding
  (\citeproc{ref-vanderweele2020}{VanderWeele \emph{et al.} 2020}).
  Accounting for prior exposure also allowed us to recover an incident
  exposure effect as opposed to a prevalent exposure effect: that is, we
  were able to focus on the one-year effect on partner well-being from
  initiation to psychopathy or a psychopathic trait
  (\citeproc{ref-danaei2012}{Danaei \emph{et al.} 2012};
  \citeproc{ref-hernan2023}{Hernan and Robins 2023}).
\item
  \textbf{Baseline outcome control}: we also included baseline measures
  of the outcome to adjust for unmeasured confounding
  (\citeproc{ref-vanderweele2020}{VanderWeele \emph{et al.} 2020}). This
  strategy also minimises the possibility of reverse causation.
\end{enumerate}

Additionally, we adopted the following confounding-control strategies:

\begin{enumerate}
\def\labelenumi{\arabic{enumi}.}
\setcounter{enumi}{5}
\tightlist
\item
  \textbf{Partner's covariates}: because the effect of changes in one
  individual's variables on their partner's outcomes is assessed, we
  include baseline measures for the partner's covariates, as we as of
  the focal partner. Additionally, we clustered individuals with their
  dyadic partnerships, which helps adjust for unmeasured features that
  might affect results.
\end{enumerate}

\paragraph{Estimator}\label{estimator}

We employ a semi-parametric estimator known as Targeted Minimum
Loss-based Estimation (TMLE), which is able to estimate the causal
effect of modified treatment policies on outcomes over time
(\citeproc{ref-vanderlaan2011}{Van Der Laan and Rose 2011},
\citeproc{ref-vanderlaan2018}{2018}). Estimation was performed using
\texttt{lmtp} package (\citeproc{ref-duxedaz2021}{Díaz \emph{et al.}
2021}; \citeproc{ref-hoffman2023}{Hoffman \emph{et al.} 2023};
\citeproc{ref-williams2021}{Williams and Díaz 2021}). TMLE is a robust
method that combines machine learning techniques with traditional
statistical models to estimate causal effects while providing valid
statistical uncertainty measures for these estimates.

TMLE operates through a two-step process involving outcome and treatment
(exposure) models. Initially, it employes machine learning algorithms to
flexibly model the relationship between treatments, covariates, and
outcomes. This flexibility allows TMLE to account for complex,
high-dimensional covariate spaces without imposing restrictive model
assumptions. The outcome of this step is a set of initial estimates for
these relationships.

The second step of TMLE involves ``targeting'' these initial estimates
by incorporating information about the observed data distribution to
improve the accuracy of the causal effect estimate. This is achieved
through an iterative updating process, which adjusts the initial
estimates towards the true causal effect. This updating process is
guided by the efficient influence function, ensuring that the final TMLE
estimate is as close as possible to the true causal effect while
remaining robust to model misspecification in either the outcome or
treatment model.

A central feature of TMLE is its double-robustness property, meaning
that if either the model for the treatment or the outcome is correctly
specified, the TMLE estimator will still consistently estimate the
causal effect. Additionally, TMLE uses cross-validation to avoid
overfitting and ensure that the estimator performs well in finite
samples. Each of these steps contributes to a robust methodology for
examining the \emph{causal} effects on of interventions on outcomes. The
marriage of TMLE and machine learning technologies reduces the
dependence on restrictive modelling assumptions and introduces an
additional layer of robustness. For further details see
(\citeproc{ref-duxedaz2021}{Díaz \emph{et al.} 2021};
\citeproc{ref-hoffman2022}{Hoffman \emph{et al.} 2022},
\citeproc{ref-hoffman2023}{2023})

\paragraph{Estimation}\label{estimation}

We used the \texttt{lmtp} R package to estimate treatments' causal
effects on well-being outcomes for each of the four trait dimensions and
the composite measure. Measurements of the treatments and outcomes were
included with baseline covariates. Our models included an indicator for
censorship in the next wave (C) which adjusted for loss to
follow-up/attrition conditional on measured covariates. As mentioned,
models estimated the effects of shifting the treatment variable by one
unit (both increase and decrease) using custom shift functions (f and
f\_1), which adjust the treatment variable within the observed range.
Two custom shift functions were defined to model the hypothetical
intervention of increasing or decreasing the treatment variable by one
unit, and these were contrasted against the expectation from a `null'
model.

\begin{itemize}
\tightlist
\item
  \textbf{increase +1 function}: increases the treatment variable by one
  unit, capped at the maximum observed score.
\item
  \textbf{decrease -1 function}: decreases the treatment variable by one
  unit, floored at the minimum observed score.
\item
  \textbf{`null' model:} the null model was specified identically to the
  treatment effect models in terms of the included covariates, outcome
  variable, and model structure but without applying any shift to the
  treatment variable. This ensured that any differences observed between
  the treatment models and the null model can be attributed to the
  treatment effect rather than differences in model specification. By
  estimating a model under the assumption of no change in the treatment
  status, we obtained clear baselines against which the effects of
  treatment shifts could be assessed.
\end{itemize}

\texttt{lmtp} draws on the the \texttt{SuperLearner} library, which
comprises various machine learning algorithms
(\citeproc{ref-SuperLearner2023}{Polley \emph{et al.} 2023}). Given the
relatively sample (N=1070) we used the \texttt{Ranger} and
\texttt{randomForest} estimators - both causal forest estimators -- a
subset of non-parametric estimators distinguished by their resilience
against overfitting {[}Ranger2017; randomForest2002{]}. Causal forests
excel in identifying complex, non-linear relationships between variables
without presupposing a specific model form, and thus making fewer
assumptions about the underlying data distribution.

\paragraph{Cross-Validation}\label{cross-validation}

We implemented a 10-fold cross-validation. This method partitions the
data into ten subsets of approximately equal size. During the
cross-validation process, nine subsets are used to train the model, and
the remaining subset is used for testing. This process is repeated ten
times, with each of the ten subsets used exactly once as the test set.
Using different subsets for training and validation, minimises the risk
of the model being overly complex and fitting the noise in the training
dataset, which can lead to poor performance on new data. Moreover, each
observation is used for training and validation, which is particularly
beneficial in scenarios where the amount of data is limited. The
cross-validation process results in ten estimates of model accuracy,
which can be averaged to provide a more comprehensive measure of model
performance.

\paragraph{Sensitiviy Analysis Using the
E-value}\label{sensitiviy-analysis-using-the-e-value}

To measure senstivity to unmeasured confounding, we report VanderWeele
and Ding's ``E-value'' for all analyses
(\citeproc{ref-vanderweele2017}{VanderWeele and Ding 2017}). The E-value
quantifies the minimum strength of association (on the risk ratio scale)
that an unmeasured confounder would need to have with both the exposure
and the outcome (after considering the measured covariates) to
explain-away the observed exposure-outcome association
(\citeproc{ref-linden2020EVALUE}{Linden \emph{et al.} 2020};
\citeproc{ref-vanderweele2020}{VanderWeele \emph{et al.} 2020}). For
example, if the E-value were 1.3, it would mean an unmeasured confounder
associated with both the treatment and outcome by a risk ratio of 1.3
each (or 30\% increase in risk) could explain away the observed effect.
Weaker confounding would not suffice. We also report the bound of the
E-value closest to 1. If the lower bound of the CI were 1.1, to explain
away the result, the strength of an unmeasured confounder in its
association with both the treatment and the outcome would need to be at
least 1.1 on the risk ratio scale (or a 10\% increase in risk) to
explain away the result (\citeproc{ref-vanderweele2017}{VanderWeele and
Ding 2017}).

\subsubsection{Deviations from pre-registered
research}\label{deviations-from-pre-registered-research}

Our current research deviates from our preregistered plan in three
areas, all aimed at enhancing the robustness and validity of our
findings.

First, we adjusted our primary analysis to compare the average treatment
effects of increasing psychopathic personality traits by one point for
each individual (without exceeding the scale's maximum) versus
decreasing these traits by one point. This adjustment allows for greater
separation between exposure conditions, closely emulating an
experimental design and providing clearer insights into the effects of
personality trait variations. Although it is possible to estimate
numerous causal contrasts, comparing gains or losses with the null
hypothesis of no change would yield contrasts that are not directly
comparable. For completeness, all contrasts, including those from our
original plan, are reported in Appendix XXX.

Second, we incorporated an additional baseline confounder: reported
health disability. The decision is based on the assumption that health
disability may be a common cause influencing both personality traits and
well-being outcomes.

Third, we introduced indicators to denote when baseline variables had
been imputed. This methodological refinement aligns our approach with
the recommendations of Hoffman \emph{et al.}
(\citeproc{ref-hoffman2023}{2023}), ensuring transparency and robustness
in handling missing data. By accounting for imputed data, we mitigate
potential biases and strengthen the reliability of our results.

The adjustments made in the second and third deviations resulted in a
slight attenuation of the causal effect estimates, which was
anticipated. Importantly, these changes did not alter our substantive
conclusions, yet support credibility for our findings. Comprehensive
reporting for the original preregistered study is provided in Appendix
XXX.

\subsection{Results Study 1:
Antagonism}\label{results-study-1-antagonism}

\subsubsection{Results Study 1A: Antagonism:
Partner}\label{results-study-1a-antagonism-partner}

\begin{figure}

\centering{

\includegraphics{24-OCT-manuscript-aaron-psychopathy_files/figure-pdf/fig-results-antagonism-partner-1.pdf}

}

\caption{\label{fig-results-antagonism-partner}Results for antagonism
effect on partner multi-dimensional well-being: z-transformed}

\end{figure}%

\newpage{}

\subsubsection{Results Study 1B: Antagonism: Self vs
Partner}\label{results-study-1b-antagonism-self-vs-partner}

\begin{figure}

\centering{

\includegraphics{24-OCT-manuscript-aaron-psychopathy_files/figure-pdf/fig-results-antagonism-self-1.pdf}

}

\caption{\label{fig-results-antagonism-self}Results for antagonism
effect for self vs partner on multi-dimensional well-being:
z-transformed}

\end{figure}%

\newpage{}

\newpage{}

\subsection{Results Study 2
Disinhibition}\label{results-study-2-disinhibition}

\subsubsection{Results Study 2A Disinhibition:
Partner}\label{results-study-2a-disinhibition-partner}

\begin{figure}

\centering{

\includegraphics{24-OCT-manuscript-aaron-psychopathy_files/figure-pdf/fig-results-disinhibition-partner-1.pdf}

}

\caption{\label{fig-results-disinhibition-partner}Results for
disinhibition effect for partner on multi-dimensional well-being.}

\end{figure}%

\newpage{}

\begin{verbatim}
For 'Personal Well-Being Index: Partner', the effect estimate (RD) is 0.023 (-0.026, 0.072). On the original data scale, the estimated effect is 0.033 (-0.037, 0.102). The E-value for this estimate is 1.168, with a lower bound of 1. Here, **the evidence for causality is not reliable**.

For 'Satisfaction with Relationship: Partner', the effect estimate (RD) is 0.023 (-0.021, 0.067). On the original data scale, the estimated effect is 0.024 (-0.022, 0.071). The E-value for this estimate is 1.168, with a lower bound of 1. Here, **the evidence for causality is not reliable**.

For 'Self Esteem: Partner', the effect estimate (RD) is 0.018 (-0.03, 0.066). On the original data scale, the estimated effect is 0.022 (-0.036, 0.08). The E-value for this estimate is 1.146, with a lower bound of 1. Here, **the evidence for causality is not reliable**.

For 'Kessler 6 Depression: Partner', the effect estimate (RD) is 0.007 (-0.051, 0.065). On the original data scale, the estimated effect is 0.005 (-0.033, 0.042). The E-value for this estimate is 1.087, with a lower bound of 1. Here, **the evidence for causality is not reliable**.

For 'Life Satisfaction: Partner', the effect estimate (RD) is -0.018 (-0.07, 0.033). On the original data scale, the estimated effect is -0.02 (-0.076, 0.037). The E-value for this estimate is 1.146, with a lower bound of 1. Here, **the evidence for causality is not reliable**.

For 'Kessler 6 Distress: Partner', the effect estimate (RD) is -0.022 (-0.073, 0.029). On the original data scale, the estimated effect is -0.08 (-0.265, 0.105). The E-value for this estimate is 1.164, with a lower bound of 1. Here, **the evidence for causality is not reliable**.

For 'Kessler 6 Anxiety: Partner', the effect estimate (RD) is -0.03 (-0.086, 0.026). On the original data scale, the estimated effect is -0.021 (-0.061, 0.018). The E-value for this estimate is 1.196, with a lower bound of 1. Here, **the evidence for causality is not reliable**.

For 'Conflict in Relationship: Partner', the effect estimate (RD) is -0.062 (-0.133, 0.01). On the original data scale, the estimated effect is -0.082 (-0.176, 0.013). The E-value for this estimate is 1.306, with a lower bound of 1. Here, **the evidence for causality is not reliable**.
\end{verbatim}

\begin{longtable}[]{@{}
  >{\raggedright\arraybackslash}p{(\columnwidth - 10\tabcolsep) * \real{0.4494}}
  >{\raggedleft\arraybackslash}p{(\columnwidth - 10\tabcolsep) * \real{0.1798}}
  >{\raggedleft\arraybackslash}p{(\columnwidth - 10\tabcolsep) * \real{0.0674}}
  >{\raggedleft\arraybackslash}p{(\columnwidth - 10\tabcolsep) * \real{0.0787}}
  >{\raggedleft\arraybackslash}p{(\columnwidth - 10\tabcolsep) * \real{0.0899}}
  >{\raggedleft\arraybackslash}p{(\columnwidth - 10\tabcolsep) * \real{0.1348}}@{}}

\caption{\label{tbl-results-disinhibition-partner}Table for
disinhibition effect for partner on multi-dimensional well-being.}

\tabularnewline

\toprule\noalign{}
\begin{minipage}[b]{\linewidth}\raggedright
\end{minipage} & \begin{minipage}[b]{\linewidth}\raggedleft
E{[}Y(1){]}-E{[}Y(0){]}
\end{minipage} & \begin{minipage}[b]{\linewidth}\raggedleft
2.5 \%
\end{minipage} & \begin{minipage}[b]{\linewidth}\raggedleft
97.5 \%
\end{minipage} & \begin{minipage}[b]{\linewidth}\raggedleft
E\_Value
\end{minipage} & \begin{minipage}[b]{\linewidth}\raggedleft
E\_Val\_bound
\end{minipage} \\
\midrule\noalign{}
\endhead
\bottomrule\noalign{}
\endlastfoot
Conflict in Relationship: Partner & -0.06 & -0.13 & 0.01 & 1.31 & 1 \\
Kessler 6 Anxiety: Partner & -0.03 & -0.09 & 0.03 & 1.20 & 1 \\
Kessler 6 Depression: Partner & 0.01 & -0.05 & 0.06 & 1.09 & 1 \\
Kessler 6 Distress: Partner & -0.02 & -0.07 & 0.03 & 1.16 & 1 \\
Life Satisfaction: Partner & -0.02 & -0.07 & 0.03 & 1.15 & 1 \\
Personal Well-Being Index: Partner & 0.02 & -0.03 & 0.07 & 1.17 & 1 \\
Satisfaction with Relationship: Partner & 0.02 & -0.02 & 0.07 & 1.17 &
1 \\
Self Esteem: Partner & 0.02 & -0.03 & 0.07 & 1.15 & 1 \\

\end{longtable}

\newpage{}

\subsubsection{Results Study 2B Disinhibition: Self vs
Partner}\label{results-study-2b-disinhibition-self-vs-partner}

\begin{figure}

\centering{

\includegraphics{24-OCT-manuscript-aaron-psychopathy_files/figure-pdf/fig-results-disinhibition-self-1.pdf}

}

\caption{\label{fig-results-disinhibition-self}Results for disinhibition
effect for self vs partner on multi-dimensional well-being.}

\end{figure}%

\newpage{}

\begin{verbatim}
For 'Personal Well-Being Index: Self', the effect estimate (RD) is 0.053 (-0.003, 0.11). On the original data scale, the estimated effect is 0.075 (-0.005, 0.155). The E-value for this estimate is 1.277, with a lower bound of 1. Here, **the evidence for causality is not reliable**.

For 'Self Esteem: Self', the effect estimate (RD) is 0.051 (0.003, 0.099). On the original data scale, the estimated effect is 0.062 (0.004, 0.12). The E-value for this estimate is 1.271, with a lower bound of 1.065. At this lower bound, unmeasured confounders would need a minimum association strength with both the intervention sequence and outcome of 1.065 to negate the observed effect. Weaker confounding would not overturn it. Here, **the evidence for causality is weak**.

For 'Life Satisfaction: Self', the effect estimate (RD) is 0.015 (-0.035, 0.065). On the original data scale, the estimated effect is 0.017 (-0.039, 0.072). The E-value for this estimate is 1.132, with a lower bound of 1. Here, **the evidence for causality is not reliable**.

For 'Kessler 6 Depression: Self', the effect estimate (RD) is -0.001 (-0.05, 0.049). On the original data scale, the estimated effect is -0.001 (-0.033, 0.031). The E-value for this estimate is 1.031, with a lower bound of 1. Here, **the evidence for causality is not reliable**.

For 'Satisfaction with Relationship: Self', the effect estimate (RD) is -0.021 (-0.073, 0.032). On the original data scale, the estimated effect is -0.022 (-0.078, 0.033). The E-value for this estimate is 1.16, with a lower bound of 1. Here, **the evidence for causality is not reliable**.

For 'Kessler 6 Distress: Self', the effect estimate (RD) is -0.053 (-0.1, -0.006). On the original data scale, the estimated effect is -0.192 (-0.363, -0.022). The E-value for this estimate is 1.277, with a lower bound of 1.08. At this lower bound, unmeasured confounders would need a minimum association strength with both the intervention sequence and outcome of 1.08 to negate the observed effect. Weaker confounding would not overturn it. Here, **the evidence for causality is weak**.

For 'Conflict in Relationship: Self', the effect estimate (RD) is -0.056 (-0.118, 0.006). On the original data scale, the estimated effect is -0.074 (-0.155, 0.008). The E-value for this estimate is 1.287, with a lower bound of 1. Here, **the evidence for causality is not reliable**.

For 'Kessler 6 Anxiety: Self', the effect estimate (RD) is -0.1 (-0.148, -0.053). On the original data scale, the estimated effect is -0.071 (-0.104, -0.037). The E-value for this estimate is 1.418, with a lower bound of 1.277. At this lower bound, unmeasured confounders would need a minimum association strength with both the intervention sequence and outcome of 1.277 to negate the observed effect. Weaker confounding would not overturn it. Here, **there is evidence for causality**.
\end{verbatim}

\begin{longtable}[]{@{}
  >{\raggedright\arraybackslash}p{(\columnwidth - 10\tabcolsep) * \real{0.4302}}
  >{\raggedleft\arraybackslash}p{(\columnwidth - 10\tabcolsep) * \real{0.1860}}
  >{\raggedleft\arraybackslash}p{(\columnwidth - 10\tabcolsep) * \real{0.0698}}
  >{\raggedleft\arraybackslash}p{(\columnwidth - 10\tabcolsep) * \real{0.0814}}
  >{\raggedleft\arraybackslash}p{(\columnwidth - 10\tabcolsep) * \real{0.0930}}
  >{\raggedleft\arraybackslash}p{(\columnwidth - 10\tabcolsep) * \real{0.1395}}@{}}

\caption{\label{tbl-results-disinhibition-self}Table for disinhibition
effect for self on multi-dimensional well-being.}

\tabularnewline

\toprule\noalign{}
\begin{minipage}[b]{\linewidth}\raggedright
\end{minipage} & \begin{minipage}[b]{\linewidth}\raggedleft
E{[}Y(1){]}-E{[}Y(0){]}
\end{minipage} & \begin{minipage}[b]{\linewidth}\raggedleft
2.5 \%
\end{minipage} & \begin{minipage}[b]{\linewidth}\raggedleft
97.5 \%
\end{minipage} & \begin{minipage}[b]{\linewidth}\raggedleft
E\_Value
\end{minipage} & \begin{minipage}[b]{\linewidth}\raggedleft
E\_Val\_bound
\end{minipage} \\
\midrule\noalign{}
\endhead
\bottomrule\noalign{}
\endlastfoot
Conflict in Relationship: Self & -0.06 & -0.12 & 0.01 & 1.29 & 1.00 \\
Kessler 6 Anxiety: Self & -0.10 & -0.15 & -0.05 & 1.42 & 1.28 \\
Kessler 6 Depression: Self & 0.00 & -0.05 & 0.05 & 1.03 & 1.00 \\
Kessler 6 Distress: Self & -0.05 & -0.10 & -0.01 & 1.28 & 1.08 \\
Life Satisfaction: Self & 0.01 & -0.04 & 0.06 & 1.13 & 1.00 \\
Personal Well-Being Index: Self & 0.05 & 0.00 & 0.11 & 1.28 & 1.00 \\
Satisfaction with Relationship: Self & -0.02 & -0.07 & 0.03 & 1.16 &
1.00 \\
Self Esteem: Self & 0.05 & 0.00 & 0.10 & 1.27 & 1.06 \\

\end{longtable}

\newpage{}

\subsection{Results Study 3 Emotional
Stability}\label{results-study-3-emotional-stability}

\subsubsection{Results Study 3A Emotional Stability:
Partner}\label{results-study-3a-emotional-stability-partner}

\begin{figure}

\centering{

\includegraphics{24-OCT-manuscript-aaron-psychopathy_files/figure-pdf/fig-results-emotional-partner-1.pdf}

}

\caption{\label{fig-results-emotional-partner}Results for emotional
stability effect for partner on multi-dimensional well-being.}

\end{figure}%

\newpage{}

\begin{verbatim}
For 'Satisfaction with Relationship: Partner', the effect estimate (RD) is 0.025 (-0.02, 0.071). On the original data scale, the estimated effect is 0.026 (-0.022, 0.074). The E-value for this estimate is 1.176, with a lower bound of 1. Here, **the evidence for causality is not reliable**.

For 'Personal Well-Being Index: Partner', the effect estimate (RD) is 0.022 (-0.026, 0.071). On the original data scale, the estimated effect is 0.031 (-0.038, 0.1). The E-value for this estimate is 1.164, with a lower bound of 1. Here, **the evidence for causality is not reliable**.

For 'Self Esteem: Partner', the effect estimate (RD) is 0.018 (-0.033, 0.069). On the original data scale, the estimated effect is 0.022 (-0.04, 0.084). The E-value for this estimate is 1.146, with a lower bound of 1. Here, **the evidence for causality is not reliable**.

For 'Kessler 6 Depression: Partner', the effect estimate (RD) is 0.01 (-0.045, 0.064). On the original data scale, the estimated effect is 0.006 (-0.029, 0.042). The E-value for this estimate is 1.105, with a lower bound of 1. Here, **the evidence for causality is not reliable**.

For 'Kessler 6 Distress: Partner', the effect estimate (RD) is -0.021 (-0.073, 0.031). On the original data scale, the estimated effect is -0.076 (-0.265, 0.112). The E-value for this estimate is 1.16, with a lower bound of 1. Here, **the evidence for causality is not reliable**.

For 'Life Satisfaction: Partner', the effect estimate (RD) is -0.021 (-0.07, 0.028). On the original data scale, the estimated effect is -0.023 (-0.077, 0.031). The E-value for this estimate is 1.16, with a lower bound of 1. Here, **the evidence for causality is not reliable**.

For 'Kessler 6 Anxiety: Partner', the effect estimate (RD) is -0.029 (-0.084, 0.027). On the original data scale, the estimated effect is -0.021 (-0.06, 0.019). The E-value for this estimate is 1.192, with a lower bound of 1. Here, **the evidence for causality is not reliable**.

For 'Conflict in Relationship: Partner', the effect estimate (RD) is -0.065 (-0.137, 0.006). On the original data scale, the estimated effect is -0.086 (-0.18, 0.009). The E-value for this estimate is 1.315, with a lower bound of 1. Here, **the evidence for causality is not reliable**.
\end{verbatim}

\begin{longtable}[]{@{}
  >{\raggedright\arraybackslash}p{(\columnwidth - 10\tabcolsep) * \real{0.4494}}
  >{\raggedleft\arraybackslash}p{(\columnwidth - 10\tabcolsep) * \real{0.1798}}
  >{\raggedleft\arraybackslash}p{(\columnwidth - 10\tabcolsep) * \real{0.0674}}
  >{\raggedleft\arraybackslash}p{(\columnwidth - 10\tabcolsep) * \real{0.0787}}
  >{\raggedleft\arraybackslash}p{(\columnwidth - 10\tabcolsep) * \real{0.0899}}
  >{\raggedleft\arraybackslash}p{(\columnwidth - 10\tabcolsep) * \real{0.1348}}@{}}

\caption{\label{tbl-results-emotional-partner}Table for emotional
stability effect for partner on multi-dimensional well-being.}

\tabularnewline

\toprule\noalign{}
\begin{minipage}[b]{\linewidth}\raggedright
\end{minipage} & \begin{minipage}[b]{\linewidth}\raggedleft
E{[}Y(1){]}-E{[}Y(0){]}
\end{minipage} & \begin{minipage}[b]{\linewidth}\raggedleft
2.5 \%
\end{minipage} & \begin{minipage}[b]{\linewidth}\raggedleft
97.5 \%
\end{minipage} & \begin{minipage}[b]{\linewidth}\raggedleft
E\_Value
\end{minipage} & \begin{minipage}[b]{\linewidth}\raggedleft
E\_Val\_bound
\end{minipage} \\
\midrule\noalign{}
\endhead
\bottomrule\noalign{}
\endlastfoot
Conflict in Relationship: Partner & -0.06 & -0.14 & 0.01 & 1.31 & 1 \\
Kessler 6 Anxiety: Partner & -0.03 & -0.08 & 0.03 & 1.19 & 1 \\
Kessler 6 Depression: Partner & 0.01 & -0.04 & 0.06 & 1.10 & 1 \\
Kessler 6 Distress: Partner & -0.02 & -0.07 & 0.03 & 1.16 & 1 \\
Life Satisfaction: Partner & -0.02 & -0.07 & 0.03 & 1.16 & 1 \\
Personal Well-Being Index: Partner & 0.02 & -0.03 & 0.07 & 1.16 & 1 \\
Satisfaction with Relationship: Partner & 0.03 & -0.02 & 0.07 & 1.18 &
1 \\
Self Esteem: Partner & 0.02 & -0.03 & 0.07 & 1.15 & 1 \\

\end{longtable}

\newpage{}

\subsubsection{Results Study 3B Emotional Stability: Self vs
Partner}\label{results-study-3b-emotional-stability-self-vs-partner}

\begin{figure}

\centering{

\includegraphics{24-OCT-manuscript-aaron-psychopathy_files/figure-pdf/fig-results-emotional-self-1.pdf}

}

\caption{\label{fig-results-emotional-self}Results for emotional
stability effect for self vs partner on multi-dimensional well-being.}

\end{figure}%

\newpage{}

\begin{verbatim}
For 'Personal Well-Being Index: Self', the effect estimate (RD) is 0.054 (-0.001, 0.109). On the original data scale, the estimated effect is 0.077 (-0.001, 0.155). The E-value for this estimate is 1.28, with a lower bound of 1. Here, **the evidence for causality is not reliable**.

For 'Self Esteem: Self', the effect estimate (RD) is 0.054 (0.007, 0.101). On the original data scale, the estimated effect is 0.065 (0.008, 0.122). The E-value for this estimate is 1.28, with a lower bound of 1.087. At this lower bound, unmeasured confounders would need a minimum association strength with both the intervention sequence and outcome of 1.087 to negate the observed effect. Weaker confounding would not overturn it. Here, **the evidence for causality is weak**.

For 'Life Satisfaction: Self', the effect estimate (RD) is 0.018 (-0.036, 0.071). On the original data scale, the estimated effect is 0.02 (-0.039, 0.079). The E-value for this estimate is 1.146, with a lower bound of 1. Here, **the evidence for causality is not reliable**.

For 'Kessler 6 Depression: Self', the effect estimate (RD) is -0.002 (-0.049, 0.045). On the original data scale, the estimated effect is -0.001 (-0.032, 0.029). The E-value for this estimate is 1.045, with a lower bound of 1. Here, **the evidence for causality is not reliable**.

For 'Satisfaction with Relationship: Self', the effect estimate (RD) is -0.031 (-0.088, 0.025). On the original data scale, the estimated effect is -0.033 (-0.092, 0.027). The E-value for this estimate is 1.2, with a lower bound of 1. Here, **the evidence for causality is not reliable**.

For 'Kessler 6 Distress: Self', the effect estimate (RD) is -0.059 (-0.104, -0.014). On the original data scale, the estimated effect is -0.214 (-0.377, -0.051). The E-value for this estimate is 1.296, with a lower bound of 1.127. At this lower bound, unmeasured confounders would need a minimum association strength with both the intervention sequence and outcome of 1.127 to negate the observed effect. Weaker confounding would not overturn it. Here, **there is evidence for causality**.

For 'Conflict in Relationship: Self', the effect estimate (RD) is -0.063 (-0.124, -0.001). On the original data scale, the estimated effect is -0.083 (-0.164, -0.002). The E-value for this estimate is 1.309, with a lower bound of 1.049. At this lower bound, unmeasured confounders would need a minimum association strength with both the intervention sequence and outcome of 1.049 to negate the observed effect. Weaker confounding would not overturn it. Here, **the evidence for causality is weak**.

For 'Kessler 6 Anxiety: Self', the effect estimate (RD) is -0.1 (-0.148, -0.051). On the original data scale, the estimated effect is -0.071 (-0.105, -0.036). The E-value for this estimate is 1.418, with a lower bound of 1.271. At this lower bound, unmeasured confounders would need a minimum association strength with both the intervention sequence and outcome of 1.271 to negate the observed effect. Weaker confounding would not overturn it. Here, **there is evidence for causality**.
\end{verbatim}

\begin{longtable}[]{@{}
  >{\raggedright\arraybackslash}p{(\columnwidth - 10\tabcolsep) * \real{0.4302}}
  >{\raggedleft\arraybackslash}p{(\columnwidth - 10\tabcolsep) * \real{0.1860}}
  >{\raggedleft\arraybackslash}p{(\columnwidth - 10\tabcolsep) * \real{0.0698}}
  >{\raggedleft\arraybackslash}p{(\columnwidth - 10\tabcolsep) * \real{0.0814}}
  >{\raggedleft\arraybackslash}p{(\columnwidth - 10\tabcolsep) * \real{0.0930}}
  >{\raggedleft\arraybackslash}p{(\columnwidth - 10\tabcolsep) * \real{0.1395}}@{}}

\caption{\label{tbl-results-emotional-self}Table for emotional stability
effect for self on multi-dimensional well-being.}

\tabularnewline

\toprule\noalign{}
\begin{minipage}[b]{\linewidth}\raggedright
\end{minipage} & \begin{minipage}[b]{\linewidth}\raggedleft
E{[}Y(1){]}-E{[}Y(0){]}
\end{minipage} & \begin{minipage}[b]{\linewidth}\raggedleft
2.5 \%
\end{minipage} & \begin{minipage}[b]{\linewidth}\raggedleft
97.5 \%
\end{minipage} & \begin{minipage}[b]{\linewidth}\raggedleft
E\_Value
\end{minipage} & \begin{minipage}[b]{\linewidth}\raggedleft
E\_Val\_bound
\end{minipage} \\
\midrule\noalign{}
\endhead
\bottomrule\noalign{}
\endlastfoot
Conflict in Relationship: Self & -0.06 & -0.12 & 0.00 & 1.31 & 1.05 \\
Kessler 6 Anxiety: Self & -0.10 & -0.15 & -0.05 & 1.42 & 1.27 \\
Kessler 6 Depression: Self & 0.00 & -0.05 & 0.04 & 1.04 & 1.00 \\
Kessler 6 Distress: Self & -0.06 & -0.10 & -0.01 & 1.30 & 1.13 \\
Life Satisfaction: Self & 0.02 & -0.04 & 0.07 & 1.15 & 1.00 \\
Personal Well-Being Index: Self & 0.05 & 0.00 & 0.11 & 1.28 & 1.00 \\
Satisfaction with Relationship: Self & -0.03 & -0.09 & 0.03 & 1.20 &
1.00 \\
Self Esteem: Self & 0.05 & 0.01 & 0.10 & 1.28 & 1.09 \\

\end{longtable}

\newpage{}

\subsection{Results Study 4
Narcissism}\label{results-study-4-narcissism}

\subsubsection{Results Study 4A Narcissism:
Partner}\label{results-study-4a-narcissism-partner}

\begin{figure}

\centering{

\includegraphics{24-OCT-manuscript-aaron-psychopathy_files/figure-pdf/fig-results-narcissism-partner-1.pdf}

}

\caption{\label{fig-results-narcissism-partner}Results for Narcissism
stability gain for partner on multi-dimensional well-being.}

\end{figure}%

\newpage{}

\begin{verbatim}
For 'Satisfaction with Relationship: Partner', the effect estimate (RD) is 0.076 (0.015, 0.137). On the original data scale, the estimated effect is 0.08 (0.016, 0.144). The E-value for this estimate is 1.349, with a lower bound of 1.134. At this lower bound, unmeasured confounders would need a minimum association strength with both the intervention sequence and outcome of 1.134 to negate the observed effect. Weaker confounding would not overturn it. Here, **there is evidence for causality**.

For 'Personal Well-Being Index: Partner', the effect estimate (RD) is 0.068 (0.018, 0.118). On the original data scale, the estimated effect is 0.097 (0.026, 0.167). The E-value for this estimate is 1.324, with a lower bound of 1.142. At this lower bound, unmeasured confounders would need a minimum association strength with both the intervention sequence and outcome of 1.142 to negate the observed effect. Weaker confounding would not overturn it. Here, **there is evidence for causality**.

For 'Self Esteem: Partner', the effect estimate (RD) is -0.006 (-0.058, 0.047). On the original data scale, the estimated effect is -0.007 (-0.071, 0.056). The E-value for this estimate is 1.08, with a lower bound of 1. Here, **the evidence for causality is not reliable**.

For 'Life Satisfaction: Partner', the effect estimate (RD) is -0.008 (-0.059, 0.043). On the original data scale, the estimated effect is -0.009 (-0.065, 0.047). The E-value for this estimate is 1.093, with a lower bound of 1. Here, **the evidence for causality is not reliable**.

For 'Kessler 6 Depression: Partner', the effect estimate (RD) is -0.026 (-0.082, 0.03). On the original data scale, the estimated effect is -0.017 (-0.053, 0.019). The E-value for this estimate is 1.181, with a lower bound of 1. Here, **the evidence for causality is not reliable**.

For 'Kessler 6 Anxiety: Partner', the effect estimate (RD) is -0.042 (-0.095, 0.01). On the original data scale, the estimated effect is -0.03 (-0.067, 0.007). The E-value for this estimate is 1.24, with a lower bound of 1. Here, **the evidence for causality is not reliable**.

For 'Kessler 6 Distress: Partner', the effect estimate (RD) is -0.043 (-0.098, 0.011). On the original data scale, the estimated effect is -0.156 (-0.354, 0.042). The E-value for this estimate is 1.244, with a lower bound of 1. Here, **the evidence for causality is not reliable**.

For 'Conflict in Relationship: Partner', the effect estimate (RD) is -0.051 (-0.121, 0.019). On the original data scale, the estimated effect is -0.067 (-0.159, 0.025). The E-value for this estimate is 1.271, with a lower bound of 1. Here, **the evidence for causality is not reliable**.
\end{verbatim}

\begin{longtable}[]{@{}
  >{\raggedright\arraybackslash}p{(\columnwidth - 10\tabcolsep) * \real{0.4494}}
  >{\raggedleft\arraybackslash}p{(\columnwidth - 10\tabcolsep) * \real{0.1798}}
  >{\raggedleft\arraybackslash}p{(\columnwidth - 10\tabcolsep) * \real{0.0674}}
  >{\raggedleft\arraybackslash}p{(\columnwidth - 10\tabcolsep) * \real{0.0787}}
  >{\raggedleft\arraybackslash}p{(\columnwidth - 10\tabcolsep) * \real{0.0899}}
  >{\raggedleft\arraybackslash}p{(\columnwidth - 10\tabcolsep) * \real{0.1348}}@{}}

\caption{\label{tbl-results-narcissism-partner}Table for Narcissism gain
for partner on multi-dimensional well-being.}

\tabularnewline

\toprule\noalign{}
\begin{minipage}[b]{\linewidth}\raggedright
\end{minipage} & \begin{minipage}[b]{\linewidth}\raggedleft
E{[}Y(1){]}-E{[}Y(0){]}
\end{minipage} & \begin{minipage}[b]{\linewidth}\raggedleft
2.5 \%
\end{minipage} & \begin{minipage}[b]{\linewidth}\raggedleft
97.5 \%
\end{minipage} & \begin{minipage}[b]{\linewidth}\raggedleft
E\_Value
\end{minipage} & \begin{minipage}[b]{\linewidth}\raggedleft
E\_Val\_bound
\end{minipage} \\
\midrule\noalign{}
\endhead
\bottomrule\noalign{}
\endlastfoot
Conflict in Relationship: Partner & -0.05 & -0.12 & 0.02 & 1.27 &
1.00 \\
Kessler 6 Anxiety: Partner & -0.04 & -0.10 & 0.01 & 1.24 & 1.00 \\
Kessler 6 Depression: Partner & -0.03 & -0.08 & 0.03 & 1.18 & 1.00 \\
Kessler 6 Distress: Partner & -0.04 & -0.10 & 0.01 & 1.24 & 1.00 \\
Life Satisfaction: Partner & -0.01 & -0.06 & 0.04 & 1.09 & 1.00 \\
Personal Well-Being Index: Partner & 0.07 & 0.02 & 0.12 & 1.32 & 1.14 \\
Satisfaction with Relationship: Partner & 0.08 & 0.01 & 0.14 & 1.35 &
1.13 \\
Self Esteem: Partner & -0.01 & -0.06 & 0.05 & 1.08 & 1.00 \\

\end{longtable}

\newpage{}

\subsubsection{Results Study 4B Narcissism: Self vs
Partner}\label{results-study-4b-narcissism-self-vs-partner}

\begin{figure}

\centering{

\includegraphics{24-OCT-manuscript-aaron-psychopathy_files/figure-pdf/fig-results-narcissism-self-1.pdf}

}

\caption{\label{fig-results-narcissism-self}Results for Narcissism
stability effect for self vs partner on multi-dimensional well-being:
z-transformed}

\end{figure}%

\newpage{}

\begin{verbatim}
For 'Life Satisfaction: Self', the effect estimate (RD) is 0.121 (0.063, 0.179). On the original data scale, the estimated effect is 0.133 (0.069, 0.197). The E-value for this estimate is 1.477, with a lower bound of 1.307. At this lower bound, unmeasured confounders would need a minimum association strength with both the intervention sequence and outcome of 1.307 to negate the observed effect. Weaker confounding would not overturn it. Here, **there is evidence for causality**.

For 'Self Esteem: Self', the effect estimate (RD) is 0.114 (0.061, 0.167). On the original data scale, the estimated effect is 0.138 (0.074, 0.202). The E-value for this estimate is 1.458, with a lower bound of 1.303. At this lower bound, unmeasured confounders would need a minimum association strength with both the intervention sequence and outcome of 1.303 to negate the observed effect. Weaker confounding would not overturn it. Here, **there is evidence for causality**.

For 'Satisfaction with Relationship: Self', the effect estimate (RD) is 0.063 (-0.004, 0.131). On the original data scale, the estimated effect is 0.066 (-0.005, 0.138). The E-value for this estimate is 1.309, with a lower bound of 1. Here, **the evidence for causality is not reliable**.

For 'Personal Well-Being Index: Self', the effect estimate (RD) is 0.053 (-0.01, 0.115). On the original data scale, the estimated effect is 0.075 (-0.013, 0.164). The E-value for this estimate is 1.277, with a lower bound of 1. Here, **the evidence for causality is not reliable**.

For 'Conflict in Relationship: Self', the effect estimate (RD) is -0.03 (-0.098, 0.038). On the original data scale, the estimated effect is -0.04 (-0.129, 0.05). The E-value for this estimate is 1.196, with a lower bound of 1. Here, **the evidence for causality is not reliable**.

For 'Kessler 6 Anxiety: Self', the effect estimate (RD) is -0.081 (-0.137, -0.025). On the original data scale, the estimated effect is -0.057 (-0.097, -0.018). The E-value for this estimate is 1.363, with a lower bound of 1.173. At this lower bound, unmeasured confounders would need a minimum association strength with both the intervention sequence and outcome of 1.173 to negate the observed effect. Weaker confounding would not overturn it. Here, **there is evidence for causality**.

For 'Kessler 6 Distress: Self', the effect estimate (RD) is -0.096 (-0.151, -0.041). On the original data scale, the estimated effect is -0.348 (-0.547, -0.149). The E-value for this estimate is 1.407, with a lower bound of 1.237. At this lower bound, unmeasured confounders would need a minimum association strength with both the intervention sequence and outcome of 1.237 to negate the observed effect. Weaker confounding would not overturn it. Here, **there is evidence for causality**.

For 'Kessler 6 Depression: Self', the effect estimate (RD) is -0.101 (-0.158, -0.045). On the original data scale, the estimated effect is -0.065 (-0.102, -0.029). The E-value for this estimate is 1.421, with a lower bound of 1.248. At this lower bound, unmeasured confounders would need a minimum association strength with both the intervention sequence and outcome of 1.248 to negate the observed effect. Weaker confounding would not overturn it. Here, **there is evidence for causality**.
\end{verbatim}

\begin{longtable}[]{@{}
  >{\raggedright\arraybackslash}p{(\columnwidth - 10\tabcolsep) * \real{0.4302}}
  >{\raggedleft\arraybackslash}p{(\columnwidth - 10\tabcolsep) * \real{0.1860}}
  >{\raggedleft\arraybackslash}p{(\columnwidth - 10\tabcolsep) * \real{0.0698}}
  >{\raggedleft\arraybackslash}p{(\columnwidth - 10\tabcolsep) * \real{0.0814}}
  >{\raggedleft\arraybackslash}p{(\columnwidth - 10\tabcolsep) * \real{0.0930}}
  >{\raggedleft\arraybackslash}p{(\columnwidth - 10\tabcolsep) * \real{0.1395}}@{}}

\caption{\label{tbl-results-narcissism-self}Table for Narcissism effect
for self on multi-dimensional well-being}

\tabularnewline

\toprule\noalign{}
\begin{minipage}[b]{\linewidth}\raggedright
\end{minipage} & \begin{minipage}[b]{\linewidth}\raggedleft
E{[}Y(1){]}-E{[}Y(0){]}
\end{minipage} & \begin{minipage}[b]{\linewidth}\raggedleft
2.5 \%
\end{minipage} & \begin{minipage}[b]{\linewidth}\raggedleft
97.5 \%
\end{minipage} & \begin{minipage}[b]{\linewidth}\raggedleft
E\_Value
\end{minipage} & \begin{minipage}[b]{\linewidth}\raggedleft
E\_Val\_bound
\end{minipage} \\
\midrule\noalign{}
\endhead
\bottomrule\noalign{}
\endlastfoot
Conflict in Relationship: Self & -0.03 & -0.10 & 0.04 & 1.20 & 1.00 \\
Kessler 6 Anxiety: Self & -0.08 & -0.14 & -0.03 & 1.36 & 1.17 \\
Kessler 6 Depression: Self & -0.10 & -0.16 & -0.04 & 1.42 & 1.25 \\
Kessler 6 Distress: Self & -0.10 & -0.15 & -0.04 & 1.41 & 1.24 \\
Life Satisfaction: Self & 0.12 & 0.06 & 0.18 & 1.48 & 1.31 \\
Personal Well-Being Index: Self & 0.05 & -0.01 & 0.12 & 1.28 & 1.00 \\
Satisfaction with Relationship: Self & 0.06 & 0.00 & 0.13 & 1.31 &
1.00 \\
Self Esteem: Self & 0.11 & 0.06 & 0.17 & 1.46 & 1.30 \\

\end{longtable}

\newpage{}

\subsection{Results Study 5: Psychopathy Combined
Score}\label{results-study-5-psychopathy-combined-score}

\subsubsection{Results Study 5A: Psychopathy Combined Score:
Partner}\label{results-study-5a-psychopathy-combined-score-partner}

\begin{figure}

\centering{

\includegraphics{24-OCT-manuscript-aaron-psychopathy_files/figure-pdf/fig-results-psychopathy-partner-1.pdf}

}

\caption{\label{fig-results-psychopathy-partner}Results forPsychopathy
Combined Score effect for partner on multi-dimensional well-being.}

\end{figure}%

\newpage{}

\begin{verbatim}
For 'Kessler 6 Anxiety: Partner', the effect estimate (RD) is 0.068 (0.027, 0.108). On the original data scale, the estimated effect is 0.048 (0.019, 0.077). The E-value for this estimate is 1.324, with a lower bound of 1.184. At this lower bound, unmeasured confounders would need a minimum association strength with both the intervention sequence and outcome of 1.184 to negate the observed effect. Weaker confounding would not overturn it. Here, **there is evidence for causality**.

For 'Kessler 6 Distress: Partner', the effect estimate (RD) is 0.053 (0.011, 0.094). On the original data scale, the estimated effect is 0.192 (0.042, 0.343). The E-value for this estimate is 1.277, with a lower bound of 1.116. At this lower bound, unmeasured confounders would need a minimum association strength with both the intervention sequence and outcome of 1.116 to negate the observed effect. Weaker confounding would not overturn it. Here, **there is evidence for causality**.

For 'Conflict in Relationship: Partner', the effect estimate (RD) is 0.034 (-0.012, 0.081). On the original data scale, the estimated effect is 0.045 (-0.016, 0.106). The E-value for this estimate is 1.211, with a lower bound of 1. Here, **the evidence for causality is not reliable**.

For 'Kessler 6 Depression: Partner', the effect estimate (RD) is 0.03 (-0.009, 0.07). On the original data scale, the estimated effect is 0.019 (-0.006, 0.045). The E-value for this estimate is 1.196, with a lower bound of 1. Here, **the evidence for causality is not reliable**.

For 'Life Satisfaction: Partner', the effect estimate (RD) is 0.027 (-0.015, 0.069). On the original data scale, the estimated effect is 0.03 (-0.017, 0.076). The E-value for this estimate is 1.185, with a lower bound of 1. Here, **the evidence for causality is not reliable**.

For 'Satisfaction with Relationship: Partner', the effect estimate (RD) is -0.064 (-0.1, -0.029). On the original data scale, the estimated effect is -0.067 (-0.105, -0.03). The E-value for this estimate is 1.312, with a lower bound of 1.192. At this lower bound, unmeasured confounders would need a minimum association strength with both the intervention sequence and outcome of 1.192 to negate the observed effect. Weaker confounding would not overturn it. Here, **there is evidence for causality**.

For 'Self Esteem: Partner', the effect estimate (RD) is -0.092 (-0.129, -0.056). On the original data scale, the estimated effect is -0.111 (-0.156, -0.067). The E-value for this estimate is 1.395, with a lower bound of 1.283. At this lower bound, unmeasured confounders would need a minimum association strength with both the intervention sequence and outcome of 1.283 to negate the observed effect. Weaker confounding would not overturn it. Here, **there is evidence for causality**.

For 'Personal Well-Being Index: Partner', the effect estimate (RD) is -0.106 (-0.141, -0.07). On the original data scale, the estimated effect is -0.15 (-0.201, -0.1). The E-value for this estimate is 1.435, with a lower bound of 1.333. At this lower bound, unmeasured confounders would need a minimum association strength with both the intervention sequence and outcome of 1.333 to negate the observed effect. Weaker confounding would not overturn it. Here, **there is evidence for causality**.
\end{verbatim}

\begin{longtable}[]{@{}
  >{\raggedright\arraybackslash}p{(\columnwidth - 10\tabcolsep) * \real{0.4494}}
  >{\raggedleft\arraybackslash}p{(\columnwidth - 10\tabcolsep) * \real{0.1798}}
  >{\raggedleft\arraybackslash}p{(\columnwidth - 10\tabcolsep) * \real{0.0674}}
  >{\raggedleft\arraybackslash}p{(\columnwidth - 10\tabcolsep) * \real{0.0787}}
  >{\raggedleft\arraybackslash}p{(\columnwidth - 10\tabcolsep) * \real{0.0899}}
  >{\raggedleft\arraybackslash}p{(\columnwidth - 10\tabcolsep) * \real{0.1348}}@{}}

\caption{\label{tbl-results-psychopathy-partner}Table for Psychopathy
Combined Score effect for partner on multi-dimensional well-being.}

\tabularnewline

\toprule\noalign{}
\begin{minipage}[b]{\linewidth}\raggedright
\end{minipage} & \begin{minipage}[b]{\linewidth}\raggedleft
E{[}Y(1){]}-E{[}Y(0){]}
\end{minipage} & \begin{minipage}[b]{\linewidth}\raggedleft
2.5 \%
\end{minipage} & \begin{minipage}[b]{\linewidth}\raggedleft
97.5 \%
\end{minipage} & \begin{minipage}[b]{\linewidth}\raggedleft
E\_Value
\end{minipage} & \begin{minipage}[b]{\linewidth}\raggedleft
E\_Val\_bound
\end{minipage} \\
\midrule\noalign{}
\endhead
\bottomrule\noalign{}
\endlastfoot
Conflict in Relationship: Partner & 0.03 & -0.01 & 0.08 & 1.21 & 1.00 \\
Kessler 6 Anxiety: Partner & 0.07 & 0.03 & 0.11 & 1.32 & 1.18 \\
Kessler 6 Depression: Partner & 0.03 & -0.01 & 0.07 & 1.20 & 1.00 \\
Kessler 6 Distress: Partner & 0.05 & 0.01 & 0.09 & 1.28 & 1.12 \\
Life Satisfaction: Partner & 0.03 & -0.01 & 0.07 & 1.19 & 1.00 \\
Personal Well-Being Index: Partner & -0.11 & -0.14 & -0.07 & 1.44 &
1.33 \\
Satisfaction with Relationship: Partner & -0.06 & -0.10 & -0.03 & 1.31 &
1.19 \\
Self Esteem: Partner & -0.09 & -0.13 & -0.06 & 1.40 & 1.28 \\

\end{longtable}

\subsubsection{Results Study 5B Psychopathy Combined Score: Self vs
Partner}\label{results-study-5b-psychopathy-combined-score-self-vs-partner}

\begin{figure}

\centering{

\includegraphics{24-OCT-manuscript-aaron-psychopathy_files/figure-pdf/fig-results-psychopathy-self-1.pdf}

}

\caption{\label{fig-results-psychopathy-self}Results for Psychopathy
Combined Score effect for self vs partner on multi-dimensional
well-being.}

\end{figure}%

\newpage{}

\begin{verbatim}
For 'Kessler 6 Anxiety: Self', the effect estimate (RD) is 0.087 (0.047, 0.126). On the original data scale, the estimated effect is 0.062 (0.034, 0.09). The E-value for this estimate is 1.381, with a lower bound of 1.26. At this lower bound, unmeasured confounders would need a minimum association strength with both the intervention sequence and outcome of 1.26 to negate the observed effect. Weaker confounding would not overturn it. Here, **there is evidence for causality**.

For 'Conflict in Relationship: Self', the effect estimate (RD) is 0.039 (-0.006, 0.084). On the original data scale, the estimated effect is 0.051 (-0.008, 0.111). The E-value for this estimate is 1.23, with a lower bound of 1. Here, **the evidence for causality is not reliable**.

For 'Kessler 6 Distress: Self', the effect estimate (RD) is 0.024 (-0.013, 0.061). On the original data scale, the estimated effect is 0.087 (-0.047, 0.221). The E-value for this estimate is 1.172, with a lower bound of 1. Here, **the evidence for causality is not reliable**.

For 'Self Esteem: Self', the effect estimate (RD) is -0.087 (-0.122, -0.052). On the original data scale, the estimated effect is -0.105 (-0.148, -0.063). The E-value for this estimate is 1.381, with a lower bound of 1.273. At this lower bound, unmeasured confounders would need a minimum association strength with both the intervention sequence and outcome of 1.273 to negate the observed effect. Weaker confounding would not overturn it. Here, **there is evidence for causality**.

For 'Kessler 6 Depression: Self', the effect estimate (RD) is -0.094 (-0.13, -0.057). On the original data scale, the estimated effect is -0.061 (-0.084, -0.037). The E-value for this estimate is 1.401, with a lower bound of 1.29. At this lower bound, unmeasured confounders would need a minimum association strength with both the intervention sequence and outcome of 1.29 to negate the observed effect. Weaker confounding would not overturn it. Here, **there is evidence for causality**.

For 'Life Satisfaction: Self', the effect estimate (RD) is -0.101 (-0.142, -0.06). On the original data scale, the estimated effect is -0.111 (-0.156, -0.066). The E-value for this estimate is 1.421, with a lower bound of 1.299. At this lower bound, unmeasured confounders would need a minimum association strength with both the intervention sequence and outcome of 1.299 to negate the observed effect. Weaker confounding would not overturn it. Here, **there is evidence for causality**.

For 'Satisfaction with Relationship: Self', the effect estimate (RD) is -0.117 (-0.167, -0.068). On the original data scale, the estimated effect is -0.123 (-0.176, -0.071). The E-value for this estimate is 1.466, with a lower bound of 1.325. At this lower bound, unmeasured confounders would need a minimum association strength with both the intervention sequence and outcome of 1.325 to negate the observed effect. Weaker confounding would not overturn it. Here, **there is evidence for causality**.

For 'Personal Well-Being Index: Self', the effect estimate (RD) is -0.122 (-0.165, -0.08). On the original data scale, the estimated effect is -0.173 (-0.233, -0.113). The E-value for this estimate is 1.48, with a lower bound of 1.357. At this lower bound, unmeasured confounders would need a minimum association strength with both the intervention sequence and outcome of 1.357 to negate the observed effect. Weaker confounding would not overturn it. Here, **there is evidence for causality**.
\end{verbatim}

\begin{longtable}[]{@{}
  >{\raggedright\arraybackslash}p{(\columnwidth - 10\tabcolsep) * \real{0.4302}}
  >{\raggedleft\arraybackslash}p{(\columnwidth - 10\tabcolsep) * \real{0.1860}}
  >{\raggedleft\arraybackslash}p{(\columnwidth - 10\tabcolsep) * \real{0.0698}}
  >{\raggedleft\arraybackslash}p{(\columnwidth - 10\tabcolsep) * \real{0.0814}}
  >{\raggedleft\arraybackslash}p{(\columnwidth - 10\tabcolsep) * \real{0.0930}}
  >{\raggedleft\arraybackslash}p{(\columnwidth - 10\tabcolsep) * \real{0.1395}}@{}}

\caption{\label{tbl-results-psychopathy-self}Table for Psychopathy
Combined Score effect on self multi-dimensional well-being}

\tabularnewline

\toprule\noalign{}
\begin{minipage}[b]{\linewidth}\raggedright
\end{minipage} & \begin{minipage}[b]{\linewidth}\raggedleft
E{[}Y(1){]}-E{[}Y(0){]}
\end{minipage} & \begin{minipage}[b]{\linewidth}\raggedleft
2.5 \%
\end{minipage} & \begin{minipage}[b]{\linewidth}\raggedleft
97.5 \%
\end{minipage} & \begin{minipage}[b]{\linewidth}\raggedleft
E\_Value
\end{minipage} & \begin{minipage}[b]{\linewidth}\raggedleft
E\_Val\_bound
\end{minipage} \\
\midrule\noalign{}
\endhead
\bottomrule\noalign{}
\endlastfoot
Conflict in Relationship: Self & -0.03 & -0.10 & 0.04 & 1.20 & 1.00 \\
Kessler 6 Anxiety: Self & -0.08 & -0.14 & -0.03 & 1.36 & 1.17 \\
Kessler 6 Depression: Self & -0.10 & -0.16 & -0.04 & 1.42 & 1.25 \\
Kessler 6 Distress: Self & -0.10 & -0.15 & -0.04 & 1.41 & 1.24 \\
Life Satisfaction: Self & 0.12 & 0.06 & 0.18 & 1.48 & 1.31 \\
Personal Well-Being Index: Self & 0.05 & -0.01 & 0.12 & 1.28 & 1.00 \\
Satisfaction with Relationship: Self & 0.06 & 0.00 & 0.13 & 1.31 &
1.00 \\
Self Esteem: Self & 0.11 & 0.06 & 0.17 & 1.46 & 1.30 \\

\end{longtable}

\subsection{Discussion}\label{discussion}

\subsubsection{Antagonism}\label{antagonism}

Antagonism affects partners and the self in different ways. For the
\textbf{partner}, the only reliable result is a reduction in life
satisfaction (\(-0.10\), 95\% CI: \([-0.17, -0.04]\)).

For the \textbf{self}, antagonism reliably increases life satisfaction
(\(0.10\), 95\% CI: \([0.04, 0.15]\)), personal well-being (\(0.13\),
95\% CI: \([0.07, 0.19]\)), satisfaction with the relationship
(\(0.11\), 95\% CI: \([0.05, 0.17]\)), and self-esteem (\(0.08\), 95\%
CI: \([0.02, 0.13]\)).

\subsubsection{Disinhibition}\label{disinhibition}

For the \textbf{partner}, none of the results are reliable as all
confidence intervals cross zero.

For the \textbf{self}, disinhibition reliably reduces anxiety
(\(-0.10\), 95\% CI: \([-0.15, -0.05]\)) and distress (\(-0.05\), 95\%
CI: \([-0.10, -0.01]\)). There is also a marginally reliable improvement
in personal well-being (\(0.05\), 95\% CI: \([0.00, 0.11]\)) and
self-esteem (\(0.05\), 95\% CI: \([0.00, 0.10]\)).

\subsubsection{Emotional Stability}\label{emotional-stability}

For the \textbf{partner}, none of the results are reliable as all
confidence intervals cross zero.

For the \textbf{self}, emotional stability reliably reduces anxiety
(\(-0.10\), 95\% CI: \([-0.15, -0.05]\)) and distress (\(-0.06\), 95\%
CI: \([-0.10, -0.01]\)). There is also a marginally reliable improvement
in personal well-being (\(0.05\), 95\% CI: \([0.00, 0.11]\)) and
self-esteem (\(0.05\), 95\% CI: \([0.01, 0.10]\)).

\subsubsection{Narcicissm:}\label{narcicissm}

For \textbf{partners}, narcissism reliably increases personal well-being
(\(0.07\), 95\% CI: \([0.02, 0.12]\)) and satisfaction with the
relationship (\(0.08\), 95\% CI: \([0.01, 0.14]\)). No other reliable
effects are observed.

For the \textbf{self}, narcissism reliably reduces anxiety (\(-0.08\),
95\% CI: \([-0.14, -0.03]\)), depression (\(-0.10\), 95\% CI:
\([-0.16, -0.04]\)), and distress (\(-0.10\), 95\% CI:
\([-0.15, -0.04]\)). It also reliably improves life satisfaction
(\(0.12\), 95\% CI: \([0.06, 0.18]\)) and self-esteem (\(0.11\), 95\%
CI: \([0.06, 0.17]\)).

\subsubsection{Psychopathy Combined}\label{psychopathy-combined}

For the \textbf{partner}, psychopathy reliably increases anxiety
(\(0.07\), 95\% CI: \([0.03, 0.11]\)) and distress (\(0.05\), 95\% CI:
\([0.01, 0.09]\)). It also reliably decreases personal well-being
(\(-0.11\), 95\% CI: \([-0.14, -0.07]\)), satisfaction with the
relationship (\(-0.06\), 95\% CI: \([-0.10, -0.03]\)), and self-esteem
(\(-0.09\), 95\% CI: \([-0.13, -0.06]\)).

For the \textbf{self}, psychopathy reliably reduces anxiety (\(-0.08\),
95\% CI: \([-0.14, -0.03]\)), depression (\(-0.10\), 95\% CI:
\([-0.16, -0.04]\)), and distress (\(-0.10\), 95\% CI:
\([-0.15, -0.04]\)). It also reliably improves life satisfaction
(\(0.12\), 95\% CI: \([0.06, 0.18]\)) and self-esteem (\(0.11\), 95\%
CI: \([0.06, 0.17]\)).

\subsubsection{Importance of Research}\label{importance-of-research}

These findings reveal that focussing on specific traits of psychopathic
personality affects people differently than does psychopathy as a whole.
When we target individual traits such as antagonism or narcissism,
people often feel better about themselves -- they report higher life
satisfaction, well-being, and self-esteem, and experience reduced
depression and anxiety. However, these personal improvements can come at
a cost to their partners, who may suffer from decreased life
satisfaction and increased distress. This indicates a complex interplay
between self-improvement and relational harm.

On the other hand, when psychopathy is treated as a single, combined
dimension, partners tend to suffer more. They experience more anxiety
and distress, along with lower personal well-being and satisfaction in
the relationship. Although, individuals for whom psychopathy increases
might feel less depressed, they do not experience an overall improvement
in well-being. As for partners, self-esteem, relationship satisfaction,
and personal well being declines. Moreover, life satisifaction declines
and anxiety increases. There is a darker effects of psychopathy in
couples are not restricted to the partners, but also extend to those in
whom psychopathy personality is manifest.

\newpage{}

\subsubsection{Ethics}\label{ethics}

The NZAVS is reviewed every three years by the University of Auckland
Human Participants Ethics Committee. Our most recent ethics approval
statement is as follows: The New Zealand Attitudes and Values Study was
approved by the University of Auckland Human Participants Ethics
Committee on 26/05/2021 for six years until 26/05/2027, Reference Number
UAHPEC22576.

\subsubsection{Acknowledgements}\label{acknowledgements}

The New Zealand Attitudes and Values Study is supported by a grant from
the TempletoReligion Trust (TRT0196; TRT0418). JB received support from
the Max Planck Institute for the Science of Human History. The funders
had no role in preparing the manuscript or the decision to publish.

\subsubsection{Author Statement}\label{author-statement}

TBA\ldots{} e.g.~

\begin{itemize}
\tightlist
\item
  AH conceived of the study, led the validation study, and developed the
  theory
\item
  HE \ldots{} MH \ldots contributed to the theory.
\item
  CS led data collection.
\item
  JB developed the inferential approach and did the analysis.\\
\item
  All authors contributed to writing the final version of the
  manuscript, which was substantially AH's work.
\end{itemize}

\newpage{}

\subsection{References}\label{references}

\phantomsection\label{refs}
\begin{CSLReferences}{1}{0}
\bibitem[\citeproctext]{ref-atkinson2019}
Atkinson, J, Salmond, C, and Crampton, P (2019) \emph{NZDep2018 index of
deprivation, user{'}s manual.}, Wellington.

\bibitem[\citeproctext]{ref-bulbulia2024PRACTICAL}
Bulbulia, JA (2024) A practical guide to causal inference in three-wave
panel studies. \emph{PsyArXiv Preprints}.
doi:\href{https://doi.org/10.31234/osf.io/uyg3d}{10.31234/osf.io/uyg3d}.

\bibitem[\citeproctext]{ref-danaei2012}
Danaei, G, Tavakkoli, M, and Hernán, MA (2012) Bias in observational
studies of prevalent users: lessons for comparative effectiveness
research from a meta-analysis of statins. \emph{American Journal of
Epidemiology}, \textbf{175}(4), 250--262.
doi:\href{https://doi.org/10.1093/aje/kwr301}{10.1093/aje/kwr301}.

\bibitem[\citeproctext]{ref-duxedaz2021}
Díaz, I, Williams, N, Hoffman, KL, and Schenck, EJ (2021) Non-parametric
causal effects based on longitudinal modified treatment policies.
\emph{Journal of the American Statistical Association}.
doi:\href{https://doi.org/10.1080/01621459.2021.1955691}{10.1080/01621459.2021.1955691}.

\bibitem[\citeproctext]{ref-fahy2017}
Fahy, KM, Lee, A, and Milne, BJ (2017) \emph{{N}ew {Z}ealand
socio-economic index 2013}, Wellington, New Zealand: Statistics New
Zealand-Tatauranga Aotearoa.

\bibitem[\citeproctext]{ref-fraser_coding_2020}
Fraser, G, Bulbulia, J, Greaves, LM, Wilson, MS, and Sibley, CG (2020)
Coding responses to an open-ended gender measure in a {N}ew {Z}ealand
national sample. \emph{The Journal of Sex Research}, \textbf{57}(8),
979--986.
doi:\href{https://doi.org/10.1080/00224499.2019.1687640}{10.1080/00224499.2019.1687640}.

\bibitem[\citeproctext]{ref-hernan2023}
Hernan, MA, and Robins, JM (2023) \emph{Causal inference}, Taylor \&
Francis. Retrieved from
\url{https://books.google.co.nz/books?id=/_KnHIAAACAAJ}

\bibitem[\citeproctext]{ref-hoffman2023}
Hoffman, KL, Salazar-Barreto, D, Rudolph, KE, and Díaz, I (2023)
Introducing longitudinal modified treatment policies: A unified
framework for studying complex exposures.
doi:\href{https://doi.org/10.48550/arXiv.2304.09460}{10.48550/arXiv.2304.09460}.

\bibitem[\citeproctext]{ref-hoffman2022}
Hoffman, KL, Schenck, EJ, Satlin, MJ, \ldots{} Díaz, I (2022) Comparison
of a target trial emulation framework vs cox regression to estimate the
association of corticosteroids with COVID-19 mortality. \emph{JAMA
Network Open}, \textbf{5}(10), e2234425.
doi:\href{https://doi.org/10.1001/jamanetworkopen.2022.34425}{10.1001/jamanetworkopen.2022.34425}.

\bibitem[\citeproctext]{ref-jost_end_2006-1}
Jost, JT (2006) The end of the end of ideology. \emph{American
Psychologist}, \textbf{61}(7), 651--670.
doi:\href{https://doi.org/10.1037/0003-066X.61.7.651}{10.1037/0003-066X.61.7.651}.

\bibitem[\citeproctext]{ref-linden2020EVALUE}
Linden, A, Mathur, MB, and VanderWeele, TJ (2020) Conducting sensitivity
analysis for unmeasured confounding in observational studies using
e-values: The evalue package. \emph{The Stata Journal}, \textbf{20}(1),
162--175.

\bibitem[\citeproctext]{ref-SuperLearner2023}
Polley, E, LeDell, E, Kennedy, C, and van der Laan, M (2023)
\emph{SuperLearner: Super learner prediction}. Retrieved from
\url{https://github.com/ecpolley/SuperLearner}

\bibitem[\citeproctext]{ref-vanbuuren2018}
Van Buuren, S (2018) \emph{Flexible imputation of missing data}, CRC
press.

\bibitem[\citeproctext]{ref-vanderlaan2011}
Van Der Laan, MJ, and Rose, S (2011) \emph{Targeted Learning: Causal
Inference for Observational and Experimental Data}, New York, NY:
Springer. Retrieved from
\url{https://link.springer.com/10.1007/978-1-4419-9782-1}

\bibitem[\citeproctext]{ref-vanderlaan2018}
Van Der Laan, MJ, and Rose, S (2018) \emph{Targeted Learning in Data
Science: Causal Inference for Complex Longitudinal Studies}, Cham:
Springer International Publishing. Retrieved from
\url{http://link.springer.com/10.1007/978-3-319-65304-4}

\bibitem[\citeproctext]{ref-vanderweele2017}
VanderWeele, TJ, and Ding, P (2017) Sensitivity analysis in
observational research: Introducing the {E}-value. \emph{Annals of
Internal Medicine}, \textbf{167}(4), 268--274.
doi:\href{https://doi.org/10.7326/M16-2607}{10.7326/M16-2607}.

\bibitem[\citeproctext]{ref-vanderweele2020}
VanderWeele, TJ, Mathur, MB, and Chen, Y (2020) Outcome-wide
longitudinal designs for causal inference: A new template for empirical
studies. \emph{Statistical Science}, \textbf{35}(3), 437--466.

\bibitem[\citeproctext]{ref-verbrugge1997}
Verbrugge, LM (1997) A global disability indicator. \emph{Journal of
Aging Studies}, \textbf{11}(4), 337--362.
doi:\href{https://doi.org/10.1016/S0890-4065(97)90026-8}{10.1016/S0890-4065(97)90026-8}.

\bibitem[\citeproctext]{ref-williams2021}
Williams, NT, and Díaz, I (2021) \emph{{l}mtp: Non-parametric causal
effects of feasible interventions based on modified treatment policies}.
doi:\href{https://doi.org/10.5281/zenodo.3874931}{10.5281/zenodo.3874931}.

\end{CSLReferences}

\newpage{}

\subsection{Appendix A: Measurues}\label{appendix-measures}

\paragraph{Age (waves: 1-15)}\label{age-waves-1-15}

We asked participants' age in an open-ended question (``What is your
age?'' or ``What is your date of birth'').

\paragraph{Disability}\label{disability}

We assessed disability with a one-item indicator adapted from Verbrugge
(\citeproc{ref-verbrugge1997}{1997}). It asks, ``Do you have a health
condition or disability that limits you and that has lasted for 6+
months?'' (1 = Yes, 0 = No).

\paragraph{Education Attainment (waves: 1,
4-15)}\label{education-attainment-waves-1-4-15}

Participants were asked ``What is your highest level of
qualification?''. We coded participans highest finished degree according
to the New Zealand Qualifications Authority. Ordinal-Rank 0-10 NZREG
codes (with overseas school quals coded as Level 3, and all other
ancillary categories coded as missing)
See:https://www.nzqa.govt.nz/assets/Studying-in-NZ/New-Zealand-Qualification-Framework/requirements-nzqf.pdf

\paragraph{Ethnicity (waves: 3)}\label{ethnicity-waves-3}

Based on the New Zealand Census, we asked participants ``Which ethnic
group(s) do you belong to?''. The responses were: (1) New Zealand
European; (2) Māori; (3) Samoan; (4) Cook Island Māori; (5) Tongan; (6)
Niuean; (7) Chinese; (8) Indian; (9) Other such as DUTCH, JAPANESE,
TOKELAUAN. Please state:. We coded their answers into four groups:
Maori, Pacific, Asian, and Euro (except for Time 3, which used an
open-ended measure).

\paragraph{Gender (waves: 1-15)}\label{gender-waves-1-15}

We asked participants' gender in an open-ended question: ``what is your
gender?'' or ``Are you male or female?'' (waves: 1-5). Female was coded
as 0, Male was coded as 1, and gender diverse coded as 3
(\citeproc{ref-fraser_coding_2020}{Fraser \emph{et al.} 2020}). (or 0.5
= neither female nor male)

Here, we coded all those who responded as Male as 1, and those who did
not as 0.

\paragraph{Income (waves: 1-3, 4-15)}\label{income-waves-1-3-4-15}

Participants were asked ``Please estimate your total household income
(before tax) for the year XXXX''. To stablise this indicator, we first
took the natural log of the response + 1, and then centred and
standardised the log-transformed indicator.

\paragraph{Parent (waves: 5-15)
--\textgreater{}}\label{parent-waves-5-15}

Participants were asked ``If you are a parent, what is the birth date of
your eldest child?'' or ``If you are a parent, in which year was your
eldest child born?'' (waves: 10-15). Parents were coded as 1, while the
others were coded as 0. --\textgreater{}

\paragraph{Political Orientation}\label{political-orientation}

We measured participants' political orientation using a single item
adapted from Jost (\citeproc{ref-jost_end_2006-1}{2006}).

``Please rate how politically liberal versus conservative you see
yourself as being.''

(1 = Extremely Liberal to 7 = Extremely Conservative)

\paragraph{NZSEI-13 (waves: 8-15)}\label{nzsei-13-waves-8-15}

We assessed occupational prestige and status using the New Zealand
Socio-economic Index 13 (NZSEI-13) (\citeproc{ref-fahy2017}{Fahy
\emph{et al.} 2017}). This index uses the income, age, and education of
a reference group, in this case, the 2013 New Zealand census, to
calculate a score for each occupational group. Scores range from 10
(Lowest) to 90 (Highest). This list of index scores for occupational
groups was used to assign each participant a NZSEI-13 score based on
their occupation.

Participants were asked ``If you are a parent, what is the birth date of
your eldest child?''.

\paragraph{Living with Partner}\label{living-with-partner}

Participants were asekd ``Do you live with your partner?'' (1 = Yes, 0 =
No).

\paragraph{Living in an Urban Area (waves:
1-15)}\label{living-in-an-urban-area-waves-1-15}

We coded whether they are living in an urban or rural area (1 = Urban, 0
= Rural) based on the addresses provided.

We coded whether they are living in an urban or rural area (1 = Urban, 0
= Rural) based on the addresses provided.

\paragraph{NZ Deprivation Index (waves:
1-15)}\label{nz-deprivation-index-waves-1-15}

We used the NZ Deprivation Index to assign each participant a score
based on where they live (\citeproc{ref-atkinson2019}{Atkinson \emph{et
al.} 2019}). This score combines data such as income, home ownership,
employment, qualifications, family structure, housing, and access to
transport and communication for an area into one deprivation score.

\subsection{Appendix B. Baseline Demographic
Statistics}\label{appendix-demographics}

\subsection{Appendix C: Baseline and Treatment Wave Exposure
Statistics}\label{appendix-exposures}

\subsection{Appendix E: Baseline and End of Study Outcome
Statistics}\label{appendix-outcomes}

\subsection{Appendix XX: Baseline and End of Study Outcome
Statistics}\label{appendix-xx}

\subsubsection{Results for OSF planned
study}\label{results-for-osf-planned-study}

\subsection{Results Study 1 Antagonism OSF
(Partner)}\label{results-study-1-antagonism-osf-partner}

\begin{figure}

\centering{

\includegraphics{24-OCT-manuscript-aaron-psychopathy_files/figure-pdf/fig-results-antagonism-partner_osf-1.pdf}

}

\caption{\label{fig-results-antagonism-partner_osf}Results for
antagonism effect on osf partner multi-dimensional well-being:
z-transformed}

\end{figure}%

\newpage{}

\subsubsection{Results Study 1: Compare Antagonism OSF vs Reported
(Partner)}\label{results-study-1-compare-antagonism-osf-vs-reported-partner}

\begin{figure}

\centering{

\includegraphics{24-OCT-manuscript-aaron-psychopathy_files/figure-pdf/fig-results-antagonism-osf-compare-1.pdf}

}

\caption{\label{fig-results-antagonism-osf-compare}Results for
antagonism effect for osf vs reported partner on multi-dimensional
well-being: z-transformed}

\end{figure}%

\newpage{}

\subsection{Results Study 2 Disinhibition OSF
(Partner)}\label{results-study-2-disinhibition-osf-partner}

\begin{figure}

\centering{

\includegraphics{24-OCT-manuscript-aaron-psychopathy_files/figure-pdf/fig-results-disinhibition-partner-osf-1.pdf}

}

\caption{\label{fig-results-disinhibition-partner-osf}Results for
disinhibition effect for osf partner on multi-dimensional well-being.}

\end{figure}%

\newpage{}

\begin{verbatim}
For 'Personal Well-Being Index: Partner', the effect estimate (RD) is 0.032 (-0.025, 0.089). On the original data scale, the estimated effect is 0.045 (-0.035, 0.126). The E-value for this estimate is 1.204, with a lower bound of 1. Here, **the evidence for causality is not reliable**.

For 'Satisfaction with Relationship: Partner', the effect estimate (RD) is 0.027 (-0.027, 0.08). On the original data scale, the estimated effect is 0.028 (-0.028, 0.085). The E-value for this estimate is 1.185, with a lower bound of 1. Here, **the evidence for causality is not reliable**.

For 'Kessler 6 Depression: Partner', the effect estimate (RD) is 0.006 (-0.058, 0.071). On the original data scale, the estimated effect is 0.004 (-0.038, 0.046). The E-value for this estimate is 1.08, with a lower bound of 1. Here, **the evidence for causality is not reliable**.

For 'Self Esteem: Partner', the effect estimate (RD) is 0.004 (-0.053, 0.061). On the original data scale, the estimated effect is 0.005 (-0.064, 0.074). The E-value for this estimate is 1.064, with a lower bound of 1. Here, **the evidence for causality is not reliable**.

For 'Life Satisfaction: Partner', the effect estimate (RD) is -0.026 (-0.08, 0.028). On the original data scale, the estimated effect is -0.029 (-0.088, 0.031). The E-value for this estimate is 1.181, with a lower bound of 1. Here, **the evidence for causality is not reliable**.

For 'Kessler 6 Distress: Partner', the effect estimate (RD) is -0.03 (-0.089, 0.028). On the original data scale, the estimated effect is -0.109 (-0.321, 0.103). The E-value for this estimate is 1.196, with a lower bound of 1. Here, **the evidence for causality is not reliable**.

For 'Kessler 6 Anxiety: Partner', the effect estimate (RD) is -0.036 (-0.101, 0.029). On the original data scale, the estimated effect is -0.025 (-0.072, 0.021). The E-value for this estimate is 1.219, with a lower bound of 1. Here, **the evidence for causality is not reliable**.

For 'Conflict in Relationship: Partner', the effect estimate (RD) is -0.087 (-0.169, -0.005). On the original data scale, the estimated effect is -0.115 (-0.223, -0.007). The E-value for this estimate is 1.381, with a lower bound of 1.071. At this lower bound, unmeasured confounders would need a minimum association strength with both the intervention sequence and outcome of 1.071 to negate the observed effect. Weaker confounding would not overturn it. Here, **the evidence for causality is weak**.
\end{verbatim}

\begin{longtable}[]{@{}
  >{\raggedright\arraybackslash}p{(\columnwidth - 10\tabcolsep) * \real{0.4494}}
  >{\raggedleft\arraybackslash}p{(\columnwidth - 10\tabcolsep) * \real{0.1798}}
  >{\raggedleft\arraybackslash}p{(\columnwidth - 10\tabcolsep) * \real{0.0674}}
  >{\raggedleft\arraybackslash}p{(\columnwidth - 10\tabcolsep) * \real{0.0787}}
  >{\raggedleft\arraybackslash}p{(\columnwidth - 10\tabcolsep) * \real{0.0899}}
  >{\raggedleft\arraybackslash}p{(\columnwidth - 10\tabcolsep) * \real{0.1348}}@{}}

\caption{\label{tbl-results-disinhibition-partner-osf}Table for
disinhibition effect for osf partner on multi-dimensional well-being.}

\tabularnewline

\toprule\noalign{}
\begin{minipage}[b]{\linewidth}\raggedright
\end{minipage} & \begin{minipage}[b]{\linewidth}\raggedleft
E{[}Y(1){]}-E{[}Y(0){]}
\end{minipage} & \begin{minipage}[b]{\linewidth}\raggedleft
2.5 \%
\end{minipage} & \begin{minipage}[b]{\linewidth}\raggedleft
97.5 \%
\end{minipage} & \begin{minipage}[b]{\linewidth}\raggedleft
E\_Value
\end{minipage} & \begin{minipage}[b]{\linewidth}\raggedleft
E\_Val\_bound
\end{minipage} \\
\midrule\noalign{}
\endhead
\bottomrule\noalign{}
\endlastfoot
Conflict in Relationship: Partner & -0.09 & -0.17 & 0.00 & 1.38 &
1.07 \\
Kessler 6 Anxiety: Partner & -0.04 & -0.10 & 0.03 & 1.22 & 1.00 \\
Kessler 6 Depression: Partner & 0.01 & -0.06 & 0.07 & 1.08 & 1.00 \\
Kessler 6 Distress: Partner & -0.03 & -0.09 & 0.03 & 1.20 & 1.00 \\
Life Satisfaction: Partner & -0.03 & -0.08 & 0.03 & 1.18 & 1.00 \\
Personal Well-Being Index: Partner & 0.03 & -0.03 & 0.09 & 1.20 &
1.00 \\
Satisfaction with Relationship: Partner & 0.03 & -0.03 & 0.08 & 1.19 &
1.00 \\
Self Esteem: Partner & 0.00 & -0.05 & 0.06 & 1.06 & 1.00 \\

\end{longtable}

\newpage{}

\subsubsection{Results Study Compare Disinhibition OSF vs Reported
(Partner)}\label{results-study-compare-disinhibition-osf-vs-reported-partner}

\begin{figure}

\centering{

\includegraphics{24-OCT-manuscript-aaron-psychopathy_files/figure-pdf/fig-results-disinhibition-osf-compare-1.pdf}

}

\caption{\label{fig-results-disinhibition-osf-compare}Results for
disinhibition effect for osf vs reported partner on multi-dimensional
well-being.}

\end{figure}%

\newpage{}

\subsection{Results Study 3 Emotional Stability: OSF
(Partner)}\label{results-study-3-emotional-stability-osf-partner}

\begin{figure}

\centering{

\includegraphics{24-OCT-manuscript-aaron-psychopathy_files/figure-pdf/fig-results-emotional-partner-osf-1.pdf}

}

\caption{\label{fig-results-emotional-partner-osf}Results for emotional
stability effect for osf partner on multi-dimensional well-being.}

\end{figure}%

\newpage{}

\begin{verbatim}
For 'Personal Well-Being Index: Partner', the effect estimate (RD) is 0.032 (-0.024, 0.088). On the original data scale, the estimated effect is 0.045 (-0.034, 0.125). The E-value for this estimate is 1.204, with a lower bound of 1. Here, **the evidence for causality is not reliable**.

For 'Satisfaction with Relationship: Partner', the effect estimate (RD) is 0.026 (-0.026, 0.078). On the original data scale, the estimated effect is 0.027 (-0.027, 0.082). The E-value for this estimate is 1.181, with a lower bound of 1. Here, **the evidence for causality is not reliable**.

For 'Self Esteem: Partner', the effect estimate (RD) is 0.01 (-0.046, 0.067). On the original data scale, the estimated effect is 0.012 (-0.056, 0.08). The E-value for this estimate is 1.105, with a lower bound of 1. Here, **the evidence for causality is not reliable**.

For 'Kessler 6 Depression: Partner', the effect estimate (RD) is 0.009 (-0.057, 0.074). On the original data scale, the estimated effect is 0.006 (-0.037, 0.048). The E-value for this estimate is 1.099, with a lower bound of 1. Here, **the evidence for causality is not reliable**.

For 'Kessler 6 Distress: Partner', the effect estimate (RD) is -0.024 (-0.086, 0.038). On the original data scale, the estimated effect is -0.087 (-0.312, 0.138). The E-value for this estimate is 1.172, with a lower bound of 1. Here, **the evidence for causality is not reliable**.

For 'Life Satisfaction: Partner', the effect estimate (RD) is -0.026 (-0.085, 0.033). On the original data scale, the estimated effect is -0.029 (-0.094, 0.036). The E-value for this estimate is 1.181, with a lower bound of 1. Here, **the evidence for causality is not reliable**.

For 'Kessler 6 Anxiety: Partner', the effect estimate (RD) is -0.032 (-0.098, 0.034). On the original data scale, the estimated effect is -0.023 (-0.069, 0.024). The E-value for this estimate is 1.204, with a lower bound of 1. Here, **the evidence for causality is not reliable**.

For 'Conflict in Relationship: Partner', the effect estimate (RD) is -0.079 (-0.162, 0.003). On the original data scale, the estimated effect is -0.104 (-0.213, 0.005). The E-value for this estimate is 1.358, with a lower bound of 1. Here, **the evidence for causality is not reliable**.
\end{verbatim}

\begin{longtable}[]{@{}
  >{\raggedright\arraybackslash}p{(\columnwidth - 10\tabcolsep) * \real{0.4494}}
  >{\raggedleft\arraybackslash}p{(\columnwidth - 10\tabcolsep) * \real{0.1798}}
  >{\raggedleft\arraybackslash}p{(\columnwidth - 10\tabcolsep) * \real{0.0674}}
  >{\raggedleft\arraybackslash}p{(\columnwidth - 10\tabcolsep) * \real{0.0787}}
  >{\raggedleft\arraybackslash}p{(\columnwidth - 10\tabcolsep) * \real{0.0899}}
  >{\raggedleft\arraybackslash}p{(\columnwidth - 10\tabcolsep) * \real{0.1348}}@{}}

\caption{\label{tbl-results-emotional-partner-osf}Table for emotional
stability effect for osf partner on multi-dimensional well-being.}

\tabularnewline

\toprule\noalign{}
\begin{minipage}[b]{\linewidth}\raggedright
\end{minipage} & \begin{minipage}[b]{\linewidth}\raggedleft
E{[}Y(1){]}-E{[}Y(0){]}
\end{minipage} & \begin{minipage}[b]{\linewidth}\raggedleft
2.5 \%
\end{minipage} & \begin{minipage}[b]{\linewidth}\raggedleft
97.5 \%
\end{minipage} & \begin{minipage}[b]{\linewidth}\raggedleft
E\_Value
\end{minipage} & \begin{minipage}[b]{\linewidth}\raggedleft
E\_Val\_bound
\end{minipage} \\
\midrule\noalign{}
\endhead
\bottomrule\noalign{}
\endlastfoot
Conflict in Relationship: Partner & -0.08 & -0.16 & 0.00 & 1.36 & 1 \\
Kessler 6 Anxiety: Partner & -0.03 & -0.10 & 0.03 & 1.20 & 1 \\
Kessler 6 Depression: Partner & 0.01 & -0.06 & 0.07 & 1.10 & 1 \\
Kessler 6 Distress: Partner & -0.02 & -0.09 & 0.04 & 1.17 & 1 \\
Life Satisfaction: Partner & -0.03 & -0.09 & 0.03 & 1.18 & 1 \\
Personal Well-Being Index: Partner & 0.03 & -0.02 & 0.09 & 1.20 & 1 \\
Satisfaction with Relationship: Partner & 0.03 & -0.03 & 0.08 & 1.18 &
1 \\
Self Esteem: Partner & 0.01 & -0.05 & 0.07 & 1.10 & 1 \\

\end{longtable}

\newpage{}

\subsubsection{Results Study 3 Emotional Stability OSF vs Reported
(Partner)}\label{results-study-3-emotional-stability-osf-vs-reported-partner}

\begin{figure}

\centering{

\includegraphics{24-OCT-manuscript-aaron-psychopathy_files/figure-pdf/fig-results-emotional-osf-compare-1.pdf}

}

\caption{\label{fig-results-emotional-osf-compare}Results for emotional
stability effect for osf vs reported partner on multi-dimensional
well-being.}

\end{figure}%

\newpage{}

\subsection{Results Study 4
Narcissism}\label{results-study-4-narcissism-1}

\subsubsection{Results Study 4A Narcissism:
Partner}\label{results-study-4a-narcissism-partner-1}

\begin{figure}

\centering{

\includegraphics{24-OCT-manuscript-aaron-psychopathy_files/figure-pdf/fig-results-narcissism-partner-osf-1.pdf}

}

\caption{\label{fig-results-narcissism-partner-osf}Results for
Narcissism stability gain for partner on multi-dimensional well-being.}

\end{figure}%

\newpage{}

\begin{verbatim}
For 'Satisfaction with Relationship: Partner', the effect estimate (RD) is 0.073 (0.004, 0.142). On the original data scale, the estimated effect is 0.077 (0.004, 0.15). The E-value for this estimate is 1.34, with a lower bound of 1.069. At this lower bound, unmeasured confounders would need a minimum association strength with both the intervention sequence and outcome of 1.069 to negate the observed effect. Weaker confounding would not overturn it. Here, **the evidence for causality is weak**.

For 'Personal Well-Being Index: Partner', the effect estimate (RD) is 0.06 (0, 0.12). On the original data scale, the estimated effect is 0.085 (0, 0.17). The E-value for this estimate is 1.3, with a lower bound of 1.036. At this lower bound, unmeasured confounders would need a minimum association strength with both the intervention sequence and outcome of 1.036 to negate the observed effect. Weaker confounding would not overturn it. Here, **the evidence for causality is not reliable**.

For 'Life Satisfaction: Partner', the effect estimate (RD) is -0.007 (-0.069, 0.056). On the original data scale, the estimated effect is -0.008 (-0.076, 0.061). The E-value for this estimate is 1.087, with a lower bound of 1. Here, **the evidence for causality is not reliable**.

For 'Self Esteem: Partner', the effect estimate (RD) is -0.01 (-0.073, 0.053). On the original data scale, the estimated effect is -0.012 (-0.088, 0.064). The E-value for this estimate is 1.105, with a lower bound of 1. Here, **the evidence for causality is not reliable**.

For 'Kessler 6 Depression: Partner', the effect estimate (RD) is -0.032 (-0.103, 0.039). On the original data scale, the estimated effect is -0.021 (-0.067, 0.025). The E-value for this estimate is 1.204, with a lower bound of 1. Here, **the evidence for causality is not reliable**.

For 'Kessler 6 Distress: Partner', the effect estimate (RD) is -0.036 (-0.1, 0.028). On the original data scale, the estimated effect is -0.131 (-0.363, 0.102). The E-value for this estimate is 1.219, with a lower bound of 1. Here, **the evidence for causality is not reliable**.

For 'Kessler 6 Anxiety: Partner', the effect estimate (RD) is -0.038 (-0.101, 0.026). On the original data scale, the estimated effect is -0.027 (-0.072, 0.018). The E-value for this estimate is 1.226, with a lower bound of 1. Here, **the evidence for causality is not reliable**.

For 'Conflict in Relationship: Partner', the effect estimate (RD) is -0.052 (-0.133, 0.028). On the original data scale, the estimated effect is -0.069 (-0.175, 0.038). The E-value for this estimate is 1.274, with a lower bound of 1. Here, **the evidence for causality is not reliable**.
\end{verbatim}

\begin{longtable}[]{@{}
  >{\raggedright\arraybackslash}p{(\columnwidth - 10\tabcolsep) * \real{0.4494}}
  >{\raggedleft\arraybackslash}p{(\columnwidth - 10\tabcolsep) * \real{0.1798}}
  >{\raggedleft\arraybackslash}p{(\columnwidth - 10\tabcolsep) * \real{0.0674}}
  >{\raggedleft\arraybackslash}p{(\columnwidth - 10\tabcolsep) * \real{0.0787}}
  >{\raggedleft\arraybackslash}p{(\columnwidth - 10\tabcolsep) * \real{0.0899}}
  >{\raggedleft\arraybackslash}p{(\columnwidth - 10\tabcolsep) * \real{0.1348}}@{}}

\caption{\label{tbl-results-narcissism-partner-osf}Table for Narcissism
gain for partner on multi-dimensional well-being.}

\tabularnewline

\toprule\noalign{}
\begin{minipage}[b]{\linewidth}\raggedright
\end{minipage} & \begin{minipage}[b]{\linewidth}\raggedleft
E{[}Y(1){]}-E{[}Y(0){]}
\end{minipage} & \begin{minipage}[b]{\linewidth}\raggedleft
2.5 \%
\end{minipage} & \begin{minipage}[b]{\linewidth}\raggedleft
97.5 \%
\end{minipage} & \begin{minipage}[b]{\linewidth}\raggedleft
E\_Value
\end{minipage} & \begin{minipage}[b]{\linewidth}\raggedleft
E\_Val\_bound
\end{minipage} \\
\midrule\noalign{}
\endhead
\bottomrule\noalign{}
\endlastfoot
Conflict in Relationship: Partner & -0.05 & -0.13 & 0.03 & 1.27 &
1.00 \\
Kessler 6 Anxiety: Partner & -0.04 & -0.10 & 0.03 & 1.23 & 1.00 \\
Kessler 6 Depression: Partner & -0.03 & -0.10 & 0.04 & 1.20 & 1.00 \\
Kessler 6 Distress: Partner & -0.04 & -0.10 & 0.03 & 1.22 & 1.00 \\
Life Satisfaction: Partner & -0.01 & -0.07 & 0.06 & 1.09 & 1.00 \\
Personal Well-Being Index: Partner & 0.06 & 0.00 & 0.12 & 1.30 & 1.04 \\
Satisfaction with Relationship: Partner & 0.07 & 0.00 & 0.14 & 1.34 &
1.07 \\
Self Esteem: Partner & -0.01 & -0.07 & 0.05 & 1.10 & 1.00 \\

\end{longtable}

\newpage{}

\subsubsection{Results Study B Narcissism OSF vs Reported
(Partner)}\label{results-study-b-narcissism-osf-vs-reported-partner}

\begin{figure}

\centering{

\includegraphics{24-OCT-manuscript-aaron-psychopathy_files/figure-pdf/fig-results-narcissism-self-osf-1.pdf}

}

\caption{\label{fig-results-narcissism-self-osf}Results for Narcissism
stability effect for self vs partner on multi-dimensional well-being:
z-transformed}

\end{figure}%

\subsection{Results Study 5: Psychopathy Combined Score: OSF
(Partner)}\label{results-study-5-psychopathy-combined-score-osf-partner}

\begin{figure}

\centering{

\includegraphics{24-OCT-manuscript-aaron-psychopathy_files/figure-pdf/fig-results-psychopathy-partner-osf-1.pdf}

}

\caption{\label{fig-results-psychopathy-partner-osf}Results
forPsychopathy Combined Score effect for partner on multi-dimensional
well-being.}

\end{figure}%

\newpage{}

\begin{verbatim}
For 'Kessler 6 Distress: Partner', the effect estimate (RD) is 0.071 (0.027, 0.114). On the original data scale, the estimated effect is 0.257 (0.1, 0.415). The E-value for this estimate is 1.334, with a lower bound of 1.188. At this lower bound, unmeasured confounders would need a minimum association strength with both the intervention sequence and outcome of 1.188 to negate the observed effect. Weaker confounding would not overturn it. Here, **there is evidence for causality**.

For 'Kessler 6 Anxiety: Partner', the effect estimate (RD) is 0.062 (0.017, 0.107). On the original data scale, the estimated effect is 0.044 (0.012, 0.076). The E-value for this estimate is 1.306, with a lower bound of 1.141. At this lower bound, unmeasured confounders would need a minimum association strength with both the intervention sequence and outcome of 1.141 to negate the observed effect. Weaker confounding would not overturn it. Here, **there is evidence for causality**.

For 'Conflict in Relationship: Partner', the effect estimate (RD) is 0.048 (-0.004, 0.101). On the original data scale, the estimated effect is 0.063 (-0.006, 0.132). The E-value for this estimate is 1.261, with a lower bound of 1. Here, **the evidence for causality is not reliable**.

For 'Kessler 6 Depression: Partner', the effect estimate (RD) is 0.039 (-0.004, 0.082). On the original data scale, the estimated effect is 0.025 (-0.003, 0.053). The E-value for this estimate is 1.23, with a lower bound of 1. Here, **the evidence for causality is not reliable**.

For 'Life Satisfaction: Partner', the effect estimate (RD) is 0.037 (-0.009, 0.083). On the original data scale, the estimated effect is 0.041 (-0.01, 0.091). The E-value for this estimate is 1.222, with a lower bound of 1. Here, **the evidence for causality is not reliable**.

For 'Satisfaction with Relationship: Partner', the effect estimate (RD) is -0.067 (-0.105, -0.029). On the original data scale, the estimated effect is -0.071 (-0.111, -0.031). The E-value for this estimate is 1.321, with a lower bound of 1.196. At this lower bound, unmeasured confounders would need a minimum association strength with both the intervention sequence and outcome of 1.196 to negate the observed effect. Weaker confounding would not overturn it. Here, **there is evidence for causality**.

For 'Self Esteem: Partner', the effect estimate (RD) is -0.105 (-0.147, -0.062). On the original data scale, the estimated effect is -0.127 (-0.179, -0.076). The E-value for this estimate is 1.432, with a lower bound of 1.306. At this lower bound, unmeasured confounders would need a minimum association strength with both the intervention sequence and outcome of 1.306 to negate the observed effect. Weaker confounding would not overturn it. Here, **there is evidence for causality**.

For 'Personal Well-Being Index: Partner', the effect estimate (RD) is -0.113 (-0.156, -0.069). On the original data scale, the estimated effect is -0.16 (-0.222, -0.099). The E-value for this estimate is 1.455, with a lower bound of 1.33. At this lower bound, unmeasured confounders would need a minimum association strength with both the intervention sequence and outcome of 1.33 to negate the observed effect. Weaker confounding would not overturn it. Here, **there is evidence for causality**.
\end{verbatim}

\begin{longtable}[]{@{}
  >{\raggedright\arraybackslash}p{(\columnwidth - 10\tabcolsep) * \real{0.4494}}
  >{\raggedleft\arraybackslash}p{(\columnwidth - 10\tabcolsep) * \real{0.1798}}
  >{\raggedleft\arraybackslash}p{(\columnwidth - 10\tabcolsep) * \real{0.0674}}
  >{\raggedleft\arraybackslash}p{(\columnwidth - 10\tabcolsep) * \real{0.0787}}
  >{\raggedleft\arraybackslash}p{(\columnwidth - 10\tabcolsep) * \real{0.0899}}
  >{\raggedleft\arraybackslash}p{(\columnwidth - 10\tabcolsep) * \real{0.1348}}@{}}

\caption{\label{tbl-results-psychopathy-partner-osf}Table for
Psychopathy Combined Score effect for partner on multi-dimensional
well-being.}

\tabularnewline

\toprule\noalign{}
\begin{minipage}[b]{\linewidth}\raggedright
\end{minipage} & \begin{minipage}[b]{\linewidth}\raggedleft
E{[}Y(1){]}-E{[}Y(0){]}
\end{minipage} & \begin{minipage}[b]{\linewidth}\raggedleft
2.5 \%
\end{minipage} & \begin{minipage}[b]{\linewidth}\raggedleft
97.5 \%
\end{minipage} & \begin{minipage}[b]{\linewidth}\raggedleft
E\_Value
\end{minipage} & \begin{minipage}[b]{\linewidth}\raggedleft
E\_Val\_bound
\end{minipage} \\
\midrule\noalign{}
\endhead
\bottomrule\noalign{}
\endlastfoot
Conflict in Relationship: Partner & 0.05 & 0.00 & 0.10 & 1.26 & 1.00 \\
Kessler 6 Anxiety: Partner & 0.06 & 0.02 & 0.11 & 1.31 & 1.14 \\
Kessler 6 Depression: Partner & 0.04 & 0.00 & 0.08 & 1.23 & 1.00 \\
Kessler 6 Distress: Partner & 0.07 & 0.03 & 0.11 & 1.33 & 1.19 \\
Life Satisfaction: Partner & 0.04 & -0.01 & 0.08 & 1.22 & 1.00 \\
Personal Well-Being Index: Partner & -0.11 & -0.16 & -0.07 & 1.46 &
1.33 \\
Satisfaction with Relationship: Partner & -0.07 & -0.10 & -0.03 & 1.32 &
1.20 \\
Self Esteem: Partner & -0.10 & -0.15 & -0.06 & 1.43 & 1.31 \\

\end{longtable}

\subsubsection{Psychopathy Combined Score OSF vs Reported
(Partner)}\label{psychopathy-combined-score-osf-vs-reported-partner}

\begin{figure}

\centering{

\includegraphics{24-OCT-manuscript-aaron-psychopathy_files/figure-pdf/fig-results-psychopathy-osf-compare-1.pdf}

}

\caption{\label{fig-results-psychopathy-osf-compare}Results for
Psychopathy Combined Score effect for self vs partner on
multi-dimensional well-being.}

\end{figure}%

\subsection{Appendix XXX Null Contrasts for the reported
model}\label{appendix-xxx-null-contrasts-for-the-reported-model}

\subsubsection{Antagonism}\label{antagonism-1}

\paragraph{Antagonism (Partner): Gain vs
Null}\label{antagonism-partner-gain-vs-null}

\begin{longtable}[]{@{}
  >{\raggedright\arraybackslash}p{(\columnwidth - 10\tabcolsep) * \real{0.4432}}
  >{\raggedleft\arraybackslash}p{(\columnwidth - 10\tabcolsep) * \real{0.1818}}
  >{\raggedleft\arraybackslash}p{(\columnwidth - 10\tabcolsep) * \real{0.0682}}
  >{\raggedleft\arraybackslash}p{(\columnwidth - 10\tabcolsep) * \real{0.0795}}
  >{\raggedleft\arraybackslash}p{(\columnwidth - 10\tabcolsep) * \real{0.0909}}
  >{\raggedleft\arraybackslash}p{(\columnwidth - 10\tabcolsep) * \real{0.1364}}@{}}

\caption{\label{tbl-results-antagonism-partner-null-gain}Table for
Antagonism on partner multi-dimensional well-being: gain vs null.}

\tabularnewline

\toprule\noalign{}
\begin{minipage}[b]{\linewidth}\raggedright
\end{minipage} & \begin{minipage}[b]{\linewidth}\raggedleft
E{[}Y(1){]}-E{[}Y(0){]}
\end{minipage} & \begin{minipage}[b]{\linewidth}\raggedleft
2.5 \%
\end{minipage} & \begin{minipage}[b]{\linewidth}\raggedleft
97.5 \%
\end{minipage} & \begin{minipage}[b]{\linewidth}\raggedleft
E\_Value
\end{minipage} & \begin{minipage}[b]{\linewidth}\raggedleft
E\_Val\_bound
\end{minipage} \\
\midrule\noalign{}
\endhead
\bottomrule\noalign{}
\endlastfoot
t2\_partner\_conflict\_in\_relationship\_z & 0.03 & -0.02 & 0.08 & 1.19
& 1.00 \\
t2\_partner\_kessler\_latent\_anxiety\_z & 0.00 & -0.04 & 0.05 & 1.04 &
1.00 \\
t2\_partner\_kessler\_latent\_depression\_z & 0.03 & -0.02 & 0.07 & 1.18
& 1.00 \\
t2\_partner\_kessler6\_sum\_z & 0.01 & -0.03 & 0.05 & 1.10 & 1.00 \\
t2\_partner\_lifesat\_z & -0.10 & -0.15 & -0.04 & 1.41 & 1.24 \\
t2\_partner\_pwi\_z & 0.00 & -0.04 & 0.05 & 1.04 & 1.00 \\
t2\_partner\_sat\_relationship\_z & -0.05 & -0.11 & 0.01 & 1.27 &
1.00 \\
t2\_partner\_self\_esteem\_z & -0.02 & -0.06 & 0.03 & 1.15 & 1.00 \\

\end{longtable}

\paragraph{Antagonism (Partner): Loss vs
Null}\label{antagonism-partner-loss-vs-null}

\begin{longtable}[]{@{}
  >{\raggedright\arraybackslash}p{(\columnwidth - 10\tabcolsep) * \real{0.4432}}
  >{\raggedleft\arraybackslash}p{(\columnwidth - 10\tabcolsep) * \real{0.1818}}
  >{\raggedleft\arraybackslash}p{(\columnwidth - 10\tabcolsep) * \real{0.0682}}
  >{\raggedleft\arraybackslash}p{(\columnwidth - 10\tabcolsep) * \real{0.0795}}
  >{\raggedleft\arraybackslash}p{(\columnwidth - 10\tabcolsep) * \real{0.0909}}
  >{\raggedleft\arraybackslash}p{(\columnwidth - 10\tabcolsep) * \real{0.1364}}@{}}

\caption{\label{tbl-results-antagonism-null-loss}Table for Antagonism on
partner multi-dimensional well-being: loss vs null.}

\tabularnewline

\toprule\noalign{}
\begin{minipage}[b]{\linewidth}\raggedright
\end{minipage} & \begin{minipage}[b]{\linewidth}\raggedleft
E{[}Y(1){]}-E{[}Y(0){]}
\end{minipage} & \begin{minipage}[b]{\linewidth}\raggedleft
2.5 \%
\end{minipage} & \begin{minipage}[b]{\linewidth}\raggedleft
97.5 \%
\end{minipage} & \begin{minipage}[b]{\linewidth}\raggedleft
E\_Value
\end{minipage} & \begin{minipage}[b]{\linewidth}\raggedleft
E\_Val\_bound
\end{minipage} \\
\midrule\noalign{}
\endhead
\bottomrule\noalign{}
\endlastfoot
t2\_partner\_conflict\_in\_relationship\_z & -0.04 & -0.10 & 0.02 & 1.23
& 1 \\
t2\_partner\_kessler\_latent\_anxiety\_z & 0.00 & -0.04 & 0.05 & 1.06 &
1 \\
t2\_partner\_kessler\_latent\_depression\_z & -0.01 & -0.06 & 0.03 &
1.13 & 1 \\
t2\_partner\_kessler6\_sum\_z & 0.00 & -0.05 & 0.04 & 1.07 & 1 \\
t2\_partner\_lifesat\_z & 0.01 & -0.04 & 0.06 & 1.10 & 1 \\
t2\_partner\_pwi\_z & -0.03 & -0.07 & 0.02 & 1.18 & 1 \\
t2\_partner\_sat\_relationship\_z & 0.01 & -0.04 & 0.07 & 1.12 & 1 \\
t2\_partner\_self\_esteem\_z & 0.02 & -0.03 & 0.06 & 1.14 & 1 \\

\end{longtable}

\subsubsection{Disinhibition}\label{disinhibition-1}

\paragraph{Disinhibition (Partner): Gain vs
Null}\label{disinhibition-partner-gain-vs-null}

\begin{longtable}[]{@{}
  >{\raggedright\arraybackslash}p{(\columnwidth - 10\tabcolsep) * \real{0.4432}}
  >{\raggedleft\arraybackslash}p{(\columnwidth - 10\tabcolsep) * \real{0.1818}}
  >{\raggedleft\arraybackslash}p{(\columnwidth - 10\tabcolsep) * \real{0.0682}}
  >{\raggedleft\arraybackslash}p{(\columnwidth - 10\tabcolsep) * \real{0.0795}}
  >{\raggedleft\arraybackslash}p{(\columnwidth - 10\tabcolsep) * \real{0.0909}}
  >{\raggedleft\arraybackslash}p{(\columnwidth - 10\tabcolsep) * \real{0.1364}}@{}}

\caption{\label{tbl-results-disinhibition-partner-null-gain}Table for
Disinhibition effect for partner on multi-dimensional well-being: gain
vs null.}

\tabularnewline

\toprule\noalign{}
\begin{minipage}[b]{\linewidth}\raggedright
\end{minipage} & \begin{minipage}[b]{\linewidth}\raggedleft
E{[}Y(1){]}-E{[}Y(0){]}
\end{minipage} & \begin{minipage}[b]{\linewidth}\raggedleft
2.5 \%
\end{minipage} & \begin{minipage}[b]{\linewidth}\raggedleft
97.5 \%
\end{minipage} & \begin{minipage}[b]{\linewidth}\raggedleft
E\_Value
\end{minipage} & \begin{minipage}[b]{\linewidth}\raggedleft
E\_Val\_bound
\end{minipage} \\
\midrule\noalign{}
\endhead
\bottomrule\noalign{}
\endlastfoot
t2\_partner\_conflict\_in\_relationship\_z & 0.00 & -0.05 & 0.06 & 1.07
& 1 \\
t2\_partner\_kessler\_latent\_anxiety\_z & -0.01 & -0.05 & 0.02 & 1.13 &
1 \\
t2\_partner\_kessler\_latent\_depression\_z & 0.03 & -0.02 & 0.07 & 1.18
& 1 \\
t2\_partner\_kessler6\_sum\_z & 0.00 & -0.04 & 0.04 & 1.04 & 1 \\
t2\_partner\_lifesat\_z & -0.01 & -0.05 & 0.03 & 1.09 & 1 \\
t2\_partner\_pwi\_z & 0.00 & -0.03 & 0.04 & 1.05 & 1 \\
t2\_partner\_sat\_relationship\_z & 0.03 & -0.01 & 0.07 & 1.18 & 1 \\
t2\_partner\_self\_esteem\_z & 0.01 & -0.02 & 0.05 & 1.12 & 1 \\

\end{longtable}

\paragraph{Disinhibition (Partner): Loss vs
Null}\label{disinhibition-partner-loss-vs-null}

\begin{longtable}[]{@{}
  >{\raggedright\arraybackslash}p{(\columnwidth - 10\tabcolsep) * \real{0.4432}}
  >{\raggedleft\arraybackslash}p{(\columnwidth - 10\tabcolsep) * \real{0.1818}}
  >{\raggedleft\arraybackslash}p{(\columnwidth - 10\tabcolsep) * \real{0.0682}}
  >{\raggedleft\arraybackslash}p{(\columnwidth - 10\tabcolsep) * \real{0.0795}}
  >{\raggedleft\arraybackslash}p{(\columnwidth - 10\tabcolsep) * \real{0.0909}}
  >{\raggedleft\arraybackslash}p{(\columnwidth - 10\tabcolsep) * \real{0.1364}}@{}}

\caption{\label{tbl-results-disinhibition-null-loss}Table for
Disinhibition on partner multi-dimensional well-being: loss vs null.}

\tabularnewline

\toprule\noalign{}
\begin{minipage}[b]{\linewidth}\raggedright
\end{minipage} & \begin{minipage}[b]{\linewidth}\raggedleft
E{[}Y(1){]}-E{[}Y(0){]}
\end{minipage} & \begin{minipage}[b]{\linewidth}\raggedleft
2.5 \%
\end{minipage} & \begin{minipage}[b]{\linewidth}\raggedleft
97.5 \%
\end{minipage} & \begin{minipage}[b]{\linewidth}\raggedleft
E\_Value
\end{minipage} & \begin{minipage}[b]{\linewidth}\raggedleft
E\_Val\_bound
\end{minipage} \\
\midrule\noalign{}
\endhead
\bottomrule\noalign{}
\endlastfoot
t2\_partner\_conflict\_in\_relationship\_z & 0.07 & 0.01 & 0.12 & 1.32 &
1.14 \\
t2\_partner\_kessler\_latent\_anxiety\_z & 0.02 & -0.02 & 0.06 & 1.14 &
1.00 \\
t2\_partner\_kessler\_latent\_depression\_z & 0.02 & -0.02 & 0.06 & 1.15
& 1.00 \\
t2\_partner\_kessler6\_sum\_z & 0.02 & -0.01 & 0.06 & 1.17 & 1.00 \\
t2\_partner\_lifesat\_z & 0.01 & -0.03 & 0.04 & 1.10 & 1.00 \\
t2\_partner\_pwi\_z & -0.02 & -0.06 & 0.02 & 1.16 & 1.00 \\
t2\_partner\_sat\_relationship\_z & 0.00 & -0.04 & 0.04 & 1.05 & 1.00 \\
t2\_partner\_self\_esteem\_z & -0.01 & -0.04 & 0.03 & 1.08 & 1.00 \\

\end{longtable}

\subsubsection{Emotional Stability}\label{emotional-stability-1}

\paragraph{Emotional Stability (Partner): Gain vs
Null}\label{emotional-stability-partner-gain-vs-null}

\begin{longtable}[]{@{}
  >{\raggedright\arraybackslash}p{(\columnwidth - 10\tabcolsep) * \real{0.4432}}
  >{\raggedleft\arraybackslash}p{(\columnwidth - 10\tabcolsep) * \real{0.1818}}
  >{\raggedleft\arraybackslash}p{(\columnwidth - 10\tabcolsep) * \real{0.0682}}
  >{\raggedleft\arraybackslash}p{(\columnwidth - 10\tabcolsep) * \real{0.0795}}
  >{\raggedleft\arraybackslash}p{(\columnwidth - 10\tabcolsep) * \real{0.0909}}
  >{\raggedleft\arraybackslash}p{(\columnwidth - 10\tabcolsep) * \real{0.1364}}@{}}

\caption{\label{tbl-results-emotional-stability-partner-null-gain}Table
for Disinhibition effect for partner on multi-dimensional well-being:
gain vs null.}

\tabularnewline

\toprule\noalign{}
\begin{minipage}[b]{\linewidth}\raggedright
\end{minipage} & \begin{minipage}[b]{\linewidth}\raggedleft
E{[}Y(1){]}-E{[}Y(0){]}
\end{minipage} & \begin{minipage}[b]{\linewidth}\raggedleft
2.5 \%
\end{minipage} & \begin{minipage}[b]{\linewidth}\raggedleft
97.5 \%
\end{minipage} & \begin{minipage}[b]{\linewidth}\raggedleft
E\_Value
\end{minipage} & \begin{minipage}[b]{\linewidth}\raggedleft
E\_Val\_bound
\end{minipage} \\
\midrule\noalign{}
\endhead
\bottomrule\noalign{}
\endlastfoot
t2\_partner\_conflict\_in\_relationship\_z & 0.00 & -0.05 & 0.05 & 1.04
& 1 \\
t2\_partner\_kessler\_latent\_anxiety\_z & -0.02 & -0.06 & 0.02 & 1.14 &
1 \\
t2\_partner\_kessler\_latent\_depression\_z & 0.03 & -0.01 & 0.07 & 1.18
& 1 \\
t2\_partner\_kessler6\_sum\_z & 0.00 & -0.03 & 0.04 & 1.07 & 1 \\
t2\_partner\_lifesat\_z & -0.01 & -0.04 & 0.03 & 1.09 & 1 \\
t2\_partner\_pwi\_z & 0.00 & -0.03 & 0.04 & 1.07 & 1 \\
t2\_partner\_sat\_relationship\_z & 0.02 & -0.02 & 0.06 & 1.17 & 1 \\
t2\_partner\_self\_esteem\_z & 0.00 & -0.04 & 0.04 & 1.06 & 1 \\

\end{longtable}

\paragraph{Emotional Stability (Partner): Loss vs
Null}\label{emotional-stability-partner-loss-vs-null}

\begin{longtable}[]{@{}
  >{\raggedright\arraybackslash}p{(\columnwidth - 10\tabcolsep) * \real{0.4432}}
  >{\raggedleft\arraybackslash}p{(\columnwidth - 10\tabcolsep) * \real{0.1818}}
  >{\raggedleft\arraybackslash}p{(\columnwidth - 10\tabcolsep) * \real{0.0682}}
  >{\raggedleft\arraybackslash}p{(\columnwidth - 10\tabcolsep) * \real{0.0795}}
  >{\raggedleft\arraybackslash}p{(\columnwidth - 10\tabcolsep) * \real{0.0909}}
  >{\raggedleft\arraybackslash}p{(\columnwidth - 10\tabcolsep) * \real{0.1364}}@{}}

\caption{\label{tbl-results-emotional-stability-null-loss}Table for
Disinhibition on partner multi-dimensional well-being: loss vs null.}

\tabularnewline

\toprule\noalign{}
\begin{minipage}[b]{\linewidth}\raggedright
\end{minipage} & \begin{minipage}[b]{\linewidth}\raggedleft
E{[}Y(1){]}-E{[}Y(0){]}
\end{minipage} & \begin{minipage}[b]{\linewidth}\raggedleft
2.5 \%
\end{minipage} & \begin{minipage}[b]{\linewidth}\raggedleft
97.5 \%
\end{minipage} & \begin{minipage}[b]{\linewidth}\raggedleft
E\_Value
\end{minipage} & \begin{minipage}[b]{\linewidth}\raggedleft
E\_Val\_bound
\end{minipage} \\
\midrule\noalign{}
\endhead
\bottomrule\noalign{}
\endlastfoot
t2\_partner\_conflict\_in\_relationship\_z & 0.07 & 0.02 & 0.12 & 1.32 &
1.14 \\
t2\_partner\_kessler\_latent\_anxiety\_z & 0.01 & -0.03 & 0.05 & 1.12 &
1.00 \\
t2\_partner\_kessler\_latent\_depression\_z & 0.02 & -0.02 & 0.05 & 1.14
& 1.00 \\
t2\_partner\_kessler6\_sum\_z & 0.03 & -0.01 & 0.06 & 1.19 & 1.00 \\
t2\_partner\_lifesat\_z & 0.01 & -0.02 & 0.05 & 1.13 & 1.00 \\
t2\_partner\_pwi\_z & -0.02 & -0.06 & 0.02 & 1.15 & 1.00 \\
t2\_partner\_sat\_relationship\_z & 0.00 & -0.04 & 0.04 & 1.04 & 1.00 \\
t2\_partner\_self\_esteem\_z & -0.01 & -0.05 & 0.02 & 1.13 & 1.00 \\

\end{longtable}

\subsubsection{Narcissism}\label{narcissism}

\#\#\#\#Narcissism (Partner): Gain vs Null

\begin{longtable}[]{@{}
  >{\raggedright\arraybackslash}p{(\columnwidth - 10\tabcolsep) * \real{0.4432}}
  >{\raggedleft\arraybackslash}p{(\columnwidth - 10\tabcolsep) * \real{0.1818}}
  >{\raggedleft\arraybackslash}p{(\columnwidth - 10\tabcolsep) * \real{0.0682}}
  >{\raggedleft\arraybackslash}p{(\columnwidth - 10\tabcolsep) * \real{0.0795}}
  >{\raggedleft\arraybackslash}p{(\columnwidth - 10\tabcolsep) * \real{0.0909}}
  >{\raggedleft\arraybackslash}p{(\columnwidth - 10\tabcolsep) * \real{0.1364}}@{}}

\caption{\label{tbl-results-narcissism-partner-null-gain}Table for
Narcissism effect for partner on multi-dimensional well-being: gain vs
null.}

\tabularnewline

\toprule\noalign{}
\begin{minipage}[b]{\linewidth}\raggedright
\end{minipage} & \begin{minipage}[b]{\linewidth}\raggedleft
E{[}Y(1){]}-E{[}Y(0){]}
\end{minipage} & \begin{minipage}[b]{\linewidth}\raggedleft
2.5 \%
\end{minipage} & \begin{minipage}[b]{\linewidth}\raggedleft
97.5 \%
\end{minipage} & \begin{minipage}[b]{\linewidth}\raggedleft
E\_Value
\end{minipage} & \begin{minipage}[b]{\linewidth}\raggedleft
E\_Val\_bound
\end{minipage} \\
\midrule\noalign{}
\endhead
\bottomrule\noalign{}
\endlastfoot
t2\_partner\_conflict\_in\_relationship\_z & -0.02 & -0.07 & 0.03 & 1.15
& 1 \\
t2\_partner\_kessler\_latent\_anxiety\_z & -0.02 & -0.06 & 0.01 & 1.16 &
1 \\
t2\_partner\_kessler\_latent\_depression\_z & 0.00 & -0.04 & 0.04 & 1.03
& 1 \\
t2\_partner\_kessler6\_sum\_z & -0.02 & -0.05 & 0.02 & 1.14 & 1 \\
t2\_partner\_lifesat\_z & 0.00 & -0.04 & 0.03 & 1.05 & 1 \\
t2\_partner\_pwi\_z & 0.03 & 0.00 & 0.07 & 1.21 & 1 \\
t2\_partner\_sat\_relationship\_z & 0.03 & -0.01 & 0.07 & 1.21 & 1 \\
t2\_partner\_self\_esteem\_z & -0.01 & -0.04 & 0.03 & 1.09 & 1 \\

\end{longtable}

\paragraph{Narcissism (Partner): Loss vs
Null}\label{narcissism-partner-loss-vs-null}

\begin{longtable}[]{@{}
  >{\raggedright\arraybackslash}p{(\columnwidth - 10\tabcolsep) * \real{0.4432}}
  >{\raggedleft\arraybackslash}p{(\columnwidth - 10\tabcolsep) * \real{0.1818}}
  >{\raggedleft\arraybackslash}p{(\columnwidth - 10\tabcolsep) * \real{0.0682}}
  >{\raggedleft\arraybackslash}p{(\columnwidth - 10\tabcolsep) * \real{0.0795}}
  >{\raggedleft\arraybackslash}p{(\columnwidth - 10\tabcolsep) * \real{0.0909}}
  >{\raggedleft\arraybackslash}p{(\columnwidth - 10\tabcolsep) * \real{0.1364}}@{}}

\caption{\label{tbl-results-psychopathy_combined_null-loss}Table for
Narcissism on partner multi-dimensional well-being: loss vs null.}

\tabularnewline

\toprule\noalign{}
\begin{minipage}[b]{\linewidth}\raggedright
\end{minipage} & \begin{minipage}[b]{\linewidth}\raggedleft
E{[}Y(1){]}-E{[}Y(0){]}
\end{minipage} & \begin{minipage}[b]{\linewidth}\raggedleft
2.5 \%
\end{minipage} & \begin{minipage}[b]{\linewidth}\raggedleft
97.5 \%
\end{minipage} & \begin{minipage}[b]{\linewidth}\raggedleft
E\_Value
\end{minipage} & \begin{minipage}[b]{\linewidth}\raggedleft
E\_Val\_bound
\end{minipage} \\
\midrule\noalign{}
\endhead
\bottomrule\noalign{}
\endlastfoot
t2\_partner\_conflict\_in\_relationship\_z & 0.03 & -0.02 & 0.09 & 1.21
& 1.00 \\
t2\_partner\_kessler\_latent\_anxiety\_z & 0.04 & 0.00 & 0.09 & 1.24 &
1.00 \\
t2\_partner\_kessler\_latent\_depression\_z & 0.08 & 0.03 & 0.13 & 1.36
& 1.21 \\
t2\_partner\_kessler6\_sum\_z & 0.06 & 0.01 & 0.10 & 1.30 & 1.14 \\
t2\_partner\_lifesat\_z & -0.01 & -0.06 & 0.03 & 1.12 & 1.00 \\
t2\_partner\_pwi\_z & -0.07 & -0.12 & -0.02 & 1.33 & 1.17 \\
t2\_partner\_sat\_relationship\_z & -0.08 & -0.13 & -0.02 & 1.35 &
1.18 \\
t2\_partner\_self\_esteem\_z & -0.06 & -0.11 & -0.02 & 1.31 & 1.16 \\

\end{longtable}

\subsubsection{Psychopathy Combined
Score}\label{psychopathy-combined-score}

\paragraph{Psychopathy Combined Score (Partner): Gain vs
Null}\label{psychopathy-combined-score-partner-gain-vs-null}

\begin{longtable}[]{@{}
  >{\raggedright\arraybackslash}p{(\columnwidth - 10\tabcolsep) * \real{0.4432}}
  >{\raggedleft\arraybackslash}p{(\columnwidth - 10\tabcolsep) * \real{0.1818}}
  >{\raggedleft\arraybackslash}p{(\columnwidth - 10\tabcolsep) * \real{0.0682}}
  >{\raggedleft\arraybackslash}p{(\columnwidth - 10\tabcolsep) * \real{0.0795}}
  >{\raggedleft\arraybackslash}p{(\columnwidth - 10\tabcolsep) * \real{0.0909}}
  >{\raggedleft\arraybackslash}p{(\columnwidth - 10\tabcolsep) * \real{0.1364}}@{}}

\caption{\label{tbl-results-psychopathy_combined_partner-null-gain}Table
for Narcissism effect for partner on multi-dimensional well-being: gain
vs null.}

\tabularnewline

\toprule\noalign{}
\begin{minipage}[b]{\linewidth}\raggedright
\end{minipage} & \begin{minipage}[b]{\linewidth}\raggedleft
E{[}Y(1){]}-E{[}Y(0){]}
\end{minipage} & \begin{minipage}[b]{\linewidth}\raggedleft
2.5 \%
\end{minipage} & \begin{minipage}[b]{\linewidth}\raggedleft
97.5 \%
\end{minipage} & \begin{minipage}[b]{\linewidth}\raggedleft
E\_Value
\end{minipage} & \begin{minipage}[b]{\linewidth}\raggedleft
E\_Val\_bound
\end{minipage} \\
\midrule\noalign{}
\endhead
\bottomrule\noalign{}
\endlastfoot
t2\_partner\_conflict\_in\_relationship\_z & 0.00 & -0.06 & 0.06 & 1.03
& 1.00 \\
t2\_partner\_kessler\_latent\_anxiety\_z & -0.02 & -0.07 & 0.02 & 1.17 &
1.00 \\
t2\_partner\_kessler\_latent\_depression\_z & 0.05 & 0.00 & 0.10 & 1.27
& 1.01 \\
t2\_partner\_kessler6\_sum\_z & 0.01 & -0.04 & 0.05 & 1.09 & 1.00 \\
t2\_partner\_lifesat\_z & -0.04 & -0.09 & 0.01 & 1.23 & 1.00 \\
t2\_partner\_pwi\_z & 0.04 & -0.01 & 0.08 & 1.22 & 1.00 \\
t2\_partner\_sat\_relationship\_z & -0.01 & -0.06 & 0.04 & 1.11 &
1.00 \\
t2\_partner\_self\_esteem\_z & 0.03 & -0.01 & 0.07 & 1.19 & 1.00 \\

\end{longtable}

\paragraph{Psychopathy Combined Score (Partner): Loss vs
Null}\label{psychopathy-combined-score-partner-loss-vs-null}

\begin{longtable}[]{@{}
  >{\raggedright\arraybackslash}p{(\columnwidth - 10\tabcolsep) * \real{0.4432}}
  >{\raggedleft\arraybackslash}p{(\columnwidth - 10\tabcolsep) * \real{0.1818}}
  >{\raggedleft\arraybackslash}p{(\columnwidth - 10\tabcolsep) * \real{0.0682}}
  >{\raggedleft\arraybackslash}p{(\columnwidth - 10\tabcolsep) * \real{0.0795}}
  >{\raggedleft\arraybackslash}p{(\columnwidth - 10\tabcolsep) * \real{0.0909}}
  >{\raggedleft\arraybackslash}p{(\columnwidth - 10\tabcolsep) * \real{0.1364}}@{}}

\caption{\label{tbl-results-narcissism-null-loss}Table for Narcissism on
partner multi-dimensional well-being: loss vs null.}

\tabularnewline

\toprule\noalign{}
\begin{minipage}[b]{\linewidth}\raggedright
\end{minipage} & \begin{minipage}[b]{\linewidth}\raggedleft
E{[}Y(1){]}-E{[}Y(0){]}
\end{minipage} & \begin{minipage}[b]{\linewidth}\raggedleft
2.5 \%
\end{minipage} & \begin{minipage}[b]{\linewidth}\raggedleft
97.5 \%
\end{minipage} & \begin{minipage}[b]{\linewidth}\raggedleft
E\_Value
\end{minipage} & \begin{minipage}[b]{\linewidth}\raggedleft
E\_Val\_bound
\end{minipage} \\
\midrule\noalign{}
\endhead
\bottomrule\noalign{}
\endlastfoot
t2\_partner\_conflict\_in\_relationship\_z & 0.03 & -0.02 & 0.09 & 1.21
& 1.00 \\
t2\_partner\_kessler\_latent\_anxiety\_z & 0.04 & 0.00 & 0.09 & 1.24 &
1.00 \\
t2\_partner\_kessler\_latent\_depression\_z & 0.08 & 0.03 & 0.13 & 1.36
& 1.21 \\
t2\_partner\_kessler6\_sum\_z & 0.06 & 0.01 & 0.10 & 1.30 & 1.14 \\
t2\_partner\_lifesat\_z & -0.01 & -0.06 & 0.03 & 1.12 & 1.00 \\
t2\_partner\_pwi\_z & -0.07 & -0.12 & -0.02 & 1.33 & 1.17 \\
t2\_partner\_sat\_relationship\_z & -0.08 & -0.13 & -0.02 & 1.35 &
1.18 \\
t2\_partner\_self\_esteem\_z & -0.06 & -0.11 & -0.02 & 1.31 & 1.16 \\

\end{longtable}

\subsection{Appendix XXX Null Contrasts for the OSF
model}\label{appendix-xxx-null-contrasts-for-the-osf-model}

\subsubsection{Antagonism OSF}\label{antagonism-osf}

\paragraph{Antagonism OSF (Partner): Gain vs
Null}\label{antagonism-osf-partner-gain-vs-null}

\begin{longtable}[]{@{}
  >{\raggedright\arraybackslash}p{(\columnwidth - 10\tabcolsep) * \real{0.4432}}
  >{\raggedleft\arraybackslash}p{(\columnwidth - 10\tabcolsep) * \real{0.1818}}
  >{\raggedleft\arraybackslash}p{(\columnwidth - 10\tabcolsep) * \real{0.0682}}
  >{\raggedleft\arraybackslash}p{(\columnwidth - 10\tabcolsep) * \real{0.0795}}
  >{\raggedleft\arraybackslash}p{(\columnwidth - 10\tabcolsep) * \real{0.0909}}
  >{\raggedleft\arraybackslash}p{(\columnwidth - 10\tabcolsep) * \real{0.1364}}@{}}

\caption{\label{tbl-results-antagonism-partner-null-gain-osf}Table for
Antagonism on partner multi-dimensional well-being: gain vs null.}

\tabularnewline

\toprule\noalign{}
\begin{minipage}[b]{\linewidth}\raggedright
\end{minipage} & \begin{minipage}[b]{\linewidth}\raggedleft
E{[}Y(1){]}-E{[}Y(0){]}
\end{minipage} & \begin{minipage}[b]{\linewidth}\raggedleft
2.5 \%
\end{minipage} & \begin{minipage}[b]{\linewidth}\raggedleft
97.5 \%
\end{minipage} & \begin{minipage}[b]{\linewidth}\raggedleft
E\_Value
\end{minipage} & \begin{minipage}[b]{\linewidth}\raggedleft
E\_Val\_bound
\end{minipage} \\
\midrule\noalign{}
\endhead
\bottomrule\noalign{}
\endlastfoot
t2\_partner\_conflict\_in\_relationship\_z & 0.03 & -0.02 & 0.08 & 1.19
& 1.00 \\
t2\_partner\_kessler\_latent\_anxiety\_z & 0.00 & -0.04 & 0.05 & 1.04 &
1.00 \\
t2\_partner\_kessler\_latent\_depression\_z & 0.03 & -0.02 & 0.07 & 1.18
& 1.00 \\
t2\_partner\_kessler6\_sum\_z & 0.01 & -0.03 & 0.05 & 1.10 & 1.00 \\
t2\_partner\_lifesat\_z & -0.10 & -0.15 & -0.04 & 1.41 & 1.24 \\
t2\_partner\_pwi\_z & 0.00 & -0.04 & 0.05 & 1.04 & 1.00 \\
t2\_partner\_sat\_relationship\_z & -0.05 & -0.11 & 0.01 & 1.27 &
1.00 \\
t2\_partner\_self\_esteem\_z & -0.02 & -0.06 & 0.03 & 1.15 & 1.00 \\

\end{longtable}

\paragraph{Antagonism OSF (Partner): Loss vs
Null}\label{antagonism-osf-partner-loss-vs-null}

\begin{longtable}[]{@{}
  >{\raggedright\arraybackslash}p{(\columnwidth - 10\tabcolsep) * \real{0.4432}}
  >{\raggedleft\arraybackslash}p{(\columnwidth - 10\tabcolsep) * \real{0.1818}}
  >{\raggedleft\arraybackslash}p{(\columnwidth - 10\tabcolsep) * \real{0.0682}}
  >{\raggedleft\arraybackslash}p{(\columnwidth - 10\tabcolsep) * \real{0.0795}}
  >{\raggedleft\arraybackslash}p{(\columnwidth - 10\tabcolsep) * \real{0.0909}}
  >{\raggedleft\arraybackslash}p{(\columnwidth - 10\tabcolsep) * \real{0.1364}}@{}}

\caption{\label{tbl-results-antagonism-null-loss-osf}Table for
Antagonism on partner multi-dimensional well-being: loss vs null.}

\tabularnewline

\toprule\noalign{}
\begin{minipage}[b]{\linewidth}\raggedright
\end{minipage} & \begin{minipage}[b]{\linewidth}\raggedleft
E{[}Y(1){]}-E{[}Y(0){]}
\end{minipage} & \begin{minipage}[b]{\linewidth}\raggedleft
2.5 \%
\end{minipage} & \begin{minipage}[b]{\linewidth}\raggedleft
97.5 \%
\end{minipage} & \begin{minipage}[b]{\linewidth}\raggedleft
E\_Value
\end{minipage} & \begin{minipage}[b]{\linewidth}\raggedleft
E\_Val\_bound
\end{minipage} \\
\midrule\noalign{}
\endhead
\bottomrule\noalign{}
\endlastfoot
t2\_partner\_conflict\_in\_relationship\_z & -0.04 & -0.10 & 0.02 & 1.23
& 1 \\
t2\_partner\_kessler\_latent\_anxiety\_z & 0.00 & -0.04 & 0.05 & 1.06 &
1 \\
t2\_partner\_kessler\_latent\_depression\_z & -0.01 & -0.06 & 0.03 &
1.13 & 1 \\
t2\_partner\_kessler6\_sum\_z & 0.00 & -0.05 & 0.04 & 1.07 & 1 \\
t2\_partner\_lifesat\_z & 0.01 & -0.04 & 0.06 & 1.10 & 1 \\
t2\_partner\_pwi\_z & -0.03 & -0.07 & 0.02 & 1.18 & 1 \\
t2\_partner\_sat\_relationship\_z & 0.01 & -0.04 & 0.07 & 1.12 & 1 \\
t2\_partner\_self\_esteem\_z & 0.02 & -0.03 & 0.06 & 1.14 & 1 \\

\end{longtable}

\subsubsection{Disinhibition OSF}\label{disinhibition-osf}

\paragraph{Disinhibition OSF (Partner): Gain vs
Null}\label{disinhibition-osf-partner-gain-vs-null}

\begin{longtable}[]{@{}
  >{\raggedright\arraybackslash}p{(\columnwidth - 10\tabcolsep) * \real{0.4432}}
  >{\raggedleft\arraybackslash}p{(\columnwidth - 10\tabcolsep) * \real{0.1818}}
  >{\raggedleft\arraybackslash}p{(\columnwidth - 10\tabcolsep) * \real{0.0682}}
  >{\raggedleft\arraybackslash}p{(\columnwidth - 10\tabcolsep) * \real{0.0795}}
  >{\raggedleft\arraybackslash}p{(\columnwidth - 10\tabcolsep) * \real{0.0909}}
  >{\raggedleft\arraybackslash}p{(\columnwidth - 10\tabcolsep) * \real{0.1364}}@{}}

\caption{\label{tbl-results-disinhibition-partner-null-gain-osf}Table
for Disinhibition effect for partner on multi-dimensional well-being:
gain vs null.}

\tabularnewline

\toprule\noalign{}
\begin{minipage}[b]{\linewidth}\raggedright
\end{minipage} & \begin{minipage}[b]{\linewidth}\raggedleft
E{[}Y(1){]}-E{[}Y(0){]}
\end{minipage} & \begin{minipage}[b]{\linewidth}\raggedleft
2.5 \%
\end{minipage} & \begin{minipage}[b]{\linewidth}\raggedleft
97.5 \%
\end{minipage} & \begin{minipage}[b]{\linewidth}\raggedleft
E\_Value
\end{minipage} & \begin{minipage}[b]{\linewidth}\raggedleft
E\_Val\_bound
\end{minipage} \\
\midrule\noalign{}
\endhead
\bottomrule\noalign{}
\endlastfoot
t2\_partner\_conflict\_in\_relationship\_z & 0.00 & -0.06 & 0.06 & 1.00
& 1 \\
t2\_partner\_kessler\_latent\_anxiety\_z & -0.02 & -0.06 & 0.03 & 1.15 &
1 \\
t2\_partner\_kessler\_latent\_depression\_z & 0.04 & -0.01 & 0.08 & 1.22
& 1 \\
t2\_partner\_kessler6\_sum\_z & 0.00 & -0.04 & 0.04 & 1.00 & 1 \\
t2\_partner\_lifesat\_z & -0.01 & -0.06 & 0.03 & 1.12 & 1 \\
t2\_partner\_pwi\_z & 0.01 & -0.04 & 0.05 & 1.09 & 1 \\
t2\_partner\_sat\_relationship\_z & 0.02 & -0.03 & 0.07 & 1.16 & 1 \\
t2\_partner\_self\_esteem\_z & 0.00 & -0.04 & 0.04 & 1.04 & 1 \\

\end{longtable}

\paragraph{Disinhibition OSF (Partner): Loss vs
Null}\label{disinhibition-osf-partner-loss-vs-null}

\begin{longtable}[]{@{}
  >{\raggedright\arraybackslash}p{(\columnwidth - 10\tabcolsep) * \real{0.4432}}
  >{\raggedleft\arraybackslash}p{(\columnwidth - 10\tabcolsep) * \real{0.1818}}
  >{\raggedleft\arraybackslash}p{(\columnwidth - 10\tabcolsep) * \real{0.0682}}
  >{\raggedleft\arraybackslash}p{(\columnwidth - 10\tabcolsep) * \real{0.0795}}
  >{\raggedleft\arraybackslash}p{(\columnwidth - 10\tabcolsep) * \real{0.0909}}
  >{\raggedleft\arraybackslash}p{(\columnwidth - 10\tabcolsep) * \real{0.1364}}@{}}

\caption{\label{tbl-results-disinhibition-null-loss-osf}Table for
Disinhibition on partner multi-dimensional well-being: loss vs null.}

\tabularnewline

\toprule\noalign{}
\begin{minipage}[b]{\linewidth}\raggedright
\end{minipage} & \begin{minipage}[b]{\linewidth}\raggedleft
E{[}Y(1){]}-E{[}Y(0){]}
\end{minipage} & \begin{minipage}[b]{\linewidth}\raggedleft
2.5 \%
\end{minipage} & \begin{minipage}[b]{\linewidth}\raggedleft
97.5 \%
\end{minipage} & \begin{minipage}[b]{\linewidth}\raggedleft
E\_Value
\end{minipage} & \begin{minipage}[b]{\linewidth}\raggedleft
E\_Val\_bound
\end{minipage} \\
\midrule\noalign{}
\endhead
\bottomrule\noalign{}
\endlastfoot
t2\_partner\_conflict\_in\_relationship\_z & 0.09 & 0.03 & 0.14 & 1.38 &
1.19 \\
t2\_partner\_kessler\_latent\_anxiety\_z & 0.02 & -0.02 & 0.06 & 1.15 &
1.00 \\
t2\_partner\_kessler\_latent\_depression\_z & 0.03 & -0.01 & 0.07 & 1.19
& 1.00 \\
t2\_partner\_kessler6\_sum\_z & 0.03 & -0.01 & 0.07 & 1.20 & 1.00 \\
t2\_partner\_lifesat\_z & 0.01 & -0.03 & 0.05 & 1.12 & 1.00 \\
t2\_partner\_pwi\_z & -0.02 & -0.07 & 0.02 & 1.17 & 1.00 \\
t2\_partner\_sat\_relationship\_z & 0.00 & -0.05 & 0.04 & 1.07 & 1.00 \\
t2\_partner\_self\_esteem\_z & 0.00 & -0.04 & 0.04 & 1.04 & 1.00 \\

\end{longtable}

\subsubsection{Emotional Stability OSF}\label{emotional-stability-osf}

\paragraph{Emotional Stability OSF (Partner): Gain vs
Null}\label{emotional-stability-osf-partner-gain-vs-null}

\begin{longtable}[]{@{}
  >{\raggedright\arraybackslash}p{(\columnwidth - 10\tabcolsep) * \real{0.4432}}
  >{\raggedleft\arraybackslash}p{(\columnwidth - 10\tabcolsep) * \real{0.1818}}
  >{\raggedleft\arraybackslash}p{(\columnwidth - 10\tabcolsep) * \real{0.0682}}
  >{\raggedleft\arraybackslash}p{(\columnwidth - 10\tabcolsep) * \real{0.0795}}
  >{\raggedleft\arraybackslash}p{(\columnwidth - 10\tabcolsep) * \real{0.0909}}
  >{\raggedleft\arraybackslash}p{(\columnwidth - 10\tabcolsep) * \real{0.1364}}@{}}

\caption{\label{tbl-results-emotional-stability-partner-null-gain-osf}Table
for Disinhibition effect for partner on multi-dimensional well-being:
gain vs null.}

\tabularnewline

\toprule\noalign{}
\begin{minipage}[b]{\linewidth}\raggedright
\end{minipage} & \begin{minipage}[b]{\linewidth}\raggedleft
E{[}Y(1){]}-E{[}Y(0){]}
\end{minipage} & \begin{minipage}[b]{\linewidth}\raggedleft
2.5 \%
\end{minipage} & \begin{minipage}[b]{\linewidth}\raggedleft
97.5 \%
\end{minipage} & \begin{minipage}[b]{\linewidth}\raggedleft
E\_Value
\end{minipage} & \begin{minipage}[b]{\linewidth}\raggedleft
E\_Val\_bound
\end{minipage} \\
\midrule\noalign{}
\endhead
\bottomrule\noalign{}
\endlastfoot
t2\_partner\_conflict\_in\_relationship\_z & 0.00 & -0.06 & 0.06 & 1.06
& 1 \\
t2\_partner\_kessler\_latent\_anxiety\_z & -0.02 & -0.07 & 0.03 & 1.14 &
1 \\
t2\_partner\_kessler\_latent\_depression\_z & 0.03 & -0.02 & 0.07 & 1.19
& 1 \\
t2\_partner\_kessler6\_sum\_z & 0.00 & -0.04 & 0.05 & 1.05 & 1 \\
t2\_partner\_lifesat\_z & -0.01 & -0.06 & 0.03 & 1.12 & 1 \\
t2\_partner\_pwi\_z & 0.01 & -0.04 & 0.05 & 1.09 & 1 \\
t2\_partner\_sat\_relationship\_z & 0.02 & -0.03 & 0.07 & 1.16 & 1 \\
t2\_partner\_self\_esteem\_z & 0.00 & -0.04 & 0.04 & 1.00 & 1 \\

\end{longtable}

\paragraph{Emotional Stability OSF (Partner): Loss vs
Null}\label{emotional-stability-osf-partner-loss-vs-null}

\begin{longtable}[]{@{}
  >{\raggedright\arraybackslash}p{(\columnwidth - 10\tabcolsep) * \real{0.4432}}
  >{\raggedleft\arraybackslash}p{(\columnwidth - 10\tabcolsep) * \real{0.1818}}
  >{\raggedleft\arraybackslash}p{(\columnwidth - 10\tabcolsep) * \real{0.0682}}
  >{\raggedleft\arraybackslash}p{(\columnwidth - 10\tabcolsep) * \real{0.0795}}
  >{\raggedleft\arraybackslash}p{(\columnwidth - 10\tabcolsep) * \real{0.0909}}
  >{\raggedleft\arraybackslash}p{(\columnwidth - 10\tabcolsep) * \real{0.1364}}@{}}

\caption{\label{tbl-results-emotional-stability-null-loss-osf}Table for
Disinhibition on partner multi-dimensional well-being: loss vs null.}

\tabularnewline

\toprule\noalign{}
\begin{minipage}[b]{\linewidth}\raggedright
\end{minipage} & \begin{minipage}[b]{\linewidth}\raggedleft
E{[}Y(1){]}-E{[}Y(0){]}
\end{minipage} & \begin{minipage}[b]{\linewidth}\raggedleft
2.5 \%
\end{minipage} & \begin{minipage}[b]{\linewidth}\raggedleft
97.5 \%
\end{minipage} & \begin{minipage}[b]{\linewidth}\raggedleft
E\_Value
\end{minipage} & \begin{minipage}[b]{\linewidth}\raggedleft
E\_Val\_bound
\end{minipage} \\
\midrule\noalign{}
\endhead
\bottomrule\noalign{}
\endlastfoot
t2\_partner\_conflict\_in\_relationship\_z & 0.08 & 0.03 & 0.14 & 1.37 &
1.18 \\
t2\_partner\_kessler\_latent\_anxiety\_z & 0.01 & -0.03 & 0.06 & 1.13 &
1.00 \\
t2\_partner\_kessler\_latent\_depression\_z & 0.02 & -0.02 & 0.06 & 1.16
& 1.00 \\
t2\_partner\_kessler6\_sum\_z & 0.03 & -0.01 & 0.07 & 1.19 & 1.00 \\
t2\_partner\_lifesat\_z & 0.01 & -0.03 & 0.05 & 1.12 & 1.00 \\
t2\_partner\_pwi\_z & -0.03 & -0.07 & 0.02 & 1.18 & 1.00 \\
t2\_partner\_sat\_relationship\_z & 0.00 & -0.05 & 0.04 & 1.06 & 1.00 \\
t2\_partner\_self\_esteem\_z & -0.01 & -0.05 & 0.03 & 1.10 & 1.00 \\

\end{longtable}

\subsubsection{Narcissism OSF}\label{narcissism-osf}

\#\#\#\#Narcissism OSF (Partner): Gain vs Null

\begin{longtable}[]{@{}
  >{\raggedright\arraybackslash}p{(\columnwidth - 10\tabcolsep) * \real{0.4432}}
  >{\raggedleft\arraybackslash}p{(\columnwidth - 10\tabcolsep) * \real{0.1818}}
  >{\raggedleft\arraybackslash}p{(\columnwidth - 10\tabcolsep) * \real{0.0682}}
  >{\raggedleft\arraybackslash}p{(\columnwidth - 10\tabcolsep) * \real{0.0795}}
  >{\raggedleft\arraybackslash}p{(\columnwidth - 10\tabcolsep) * \real{0.0909}}
  >{\raggedleft\arraybackslash}p{(\columnwidth - 10\tabcolsep) * \real{0.1364}}@{}}

\caption{\label{tbl-results-narcissism-partner-null-gain-osf}Table for
Narcissism effect for partner on multi-dimensional well-being: gain vs
null.}

\tabularnewline

\toprule\noalign{}
\begin{minipage}[b]{\linewidth}\raggedright
\end{minipage} & \begin{minipage}[b]{\linewidth}\raggedleft
E{[}Y(1){]}-E{[}Y(0){]}
\end{minipage} & \begin{minipage}[b]{\linewidth}\raggedleft
2.5 \%
\end{minipage} & \begin{minipage}[b]{\linewidth}\raggedleft
97.5 \%
\end{minipage} & \begin{minipage}[b]{\linewidth}\raggedleft
E\_Value
\end{minipage} & \begin{minipage}[b]{\linewidth}\raggedleft
E\_Val\_bound
\end{minipage} \\
\midrule\noalign{}
\endhead
\bottomrule\noalign{}
\endlastfoot
t2\_partner\_conflict\_in\_relationship\_z & -0.03 & -0.08 & 0.03 & 1.18
& 1 \\
t2\_partner\_kessler\_latent\_anxiety\_z & -0.03 & -0.07 & 0.02 & 1.18 &
1 \\
t2\_partner\_kessler\_latent\_depression\_z & 0.00 & -0.04 & 0.04 & 1.00
& 1 \\
t2\_partner\_kessler6\_sum\_z & -0.01 & -0.06 & 0.03 & 1.13 & 1 \\
t2\_partner\_lifesat\_z & -0.01 & -0.05 & 0.04 & 1.09 & 1 \\
t2\_partner\_pwi\_z & 0.03 & -0.01 & 0.08 & 1.20 & 1 \\
t2\_partner\_sat\_relationship\_z & 0.03 & -0.02 & 0.07 & 1.19 & 1 \\
t2\_partner\_self\_esteem\_z & -0.01 & -0.05 & 0.03 & 1.12 & 1 \\

\end{longtable}

\paragraph{Narcissism OSF (Partner): Loss vs
Null}\label{narcissism-osf-partner-loss-vs-null}

\begin{longtable}[]{@{}
  >{\raggedright\arraybackslash}p{(\columnwidth - 10\tabcolsep) * \real{0.4432}}
  >{\raggedleft\arraybackslash}p{(\columnwidth - 10\tabcolsep) * \real{0.1818}}
  >{\raggedleft\arraybackslash}p{(\columnwidth - 10\tabcolsep) * \real{0.0682}}
  >{\raggedleft\arraybackslash}p{(\columnwidth - 10\tabcolsep) * \real{0.0795}}
  >{\raggedleft\arraybackslash}p{(\columnwidth - 10\tabcolsep) * \real{0.0909}}
  >{\raggedleft\arraybackslash}p{(\columnwidth - 10\tabcolsep) * \real{0.1364}}@{}}

\caption{\label{tbl-results-psychopathy_combined_null-loss-osf}Table for
Narcissism on partner multi-dimensional well-being: loss vs null.}

\tabularnewline

\toprule\noalign{}
\begin{minipage}[b]{\linewidth}\raggedright
\end{minipage} & \begin{minipage}[b]{\linewidth}\raggedleft
E{[}Y(1){]}-E{[}Y(0){]}
\end{minipage} & \begin{minipage}[b]{\linewidth}\raggedleft
2.5 \%
\end{minipage} & \begin{minipage}[b]{\linewidth}\raggedleft
97.5 \%
\end{minipage} & \begin{minipage}[b]{\linewidth}\raggedleft
E\_Value
\end{minipage} & \begin{minipage}[b]{\linewidth}\raggedleft
E\_Val\_bound
\end{minipage} \\
\midrule\noalign{}
\endhead
\bottomrule\noalign{}
\endlastfoot
t2\_partner\_conflict\_in\_relationship\_z & 0.05 & -0.01 & 0.11 & 1.27
& 1.00 \\
t2\_partner\_kessler\_latent\_anxiety\_z & 0.04 & -0.01 & 0.09 & 1.23 &
1.00 \\
t2\_partner\_kessler\_latent\_depression\_z & 0.08 & 0.03 & 0.13 & 1.36
& 1.19 \\
t2\_partner\_kessler6\_sum\_z & 0.07 & 0.03 & 0.12 & 1.35 & 1.19 \\
t2\_partner\_lifesat\_z & -0.01 & -0.06 & 0.04 & 1.08 & 1.00 \\
t2\_partner\_pwi\_z & -0.07 & -0.12 & -0.02 & 1.34 & 1.16 \\
t2\_partner\_sat\_relationship\_z & -0.07 & -0.13 & -0.01 & 1.33 &
1.13 \\
t2\_partner\_self\_esteem\_z & -0.07 & -0.11 & -0.02 & 1.32 & 1.15 \\

\end{longtable}

\subsubsection{Psychopathy Combined Score
OSF}\label{psychopathy-combined-score-osf}

\paragraph{Psychopathy Combined Score OSF (Partner): Gain vs
Null}\label{psychopathy-combined-score-osf-partner-gain-vs-null}

\begin{longtable}[]{@{}
  >{\raggedright\arraybackslash}p{(\columnwidth - 10\tabcolsep) * \real{0.4432}}
  >{\raggedleft\arraybackslash}p{(\columnwidth - 10\tabcolsep) * \real{0.1818}}
  >{\raggedleft\arraybackslash}p{(\columnwidth - 10\tabcolsep) * \real{0.0682}}
  >{\raggedleft\arraybackslash}p{(\columnwidth - 10\tabcolsep) * \real{0.0795}}
  >{\raggedleft\arraybackslash}p{(\columnwidth - 10\tabcolsep) * \real{0.0909}}
  >{\raggedleft\arraybackslash}p{(\columnwidth - 10\tabcolsep) * \real{0.1364}}@{}}

\caption{\label{tbl-results-psychopathy_combined_partner-null-gain-osf}Table
for Narcissism effect for partner on multi-dimensional well-being: gain
vs null.}

\tabularnewline

\toprule\noalign{}
\begin{minipage}[b]{\linewidth}\raggedright
\end{minipage} & \begin{minipage}[b]{\linewidth}\raggedleft
E{[}Y(1){]}-E{[}Y(0){]}
\end{minipage} & \begin{minipage}[b]{\linewidth}\raggedleft
2.5 \%
\end{minipage} & \begin{minipage}[b]{\linewidth}\raggedleft
97.5 \%
\end{minipage} & \begin{minipage}[b]{\linewidth}\raggedleft
E\_Value
\end{minipage} & \begin{minipage}[b]{\linewidth}\raggedleft
E\_Val\_bound
\end{minipage} \\
\midrule\noalign{}
\endhead
\bottomrule\noalign{}
\endlastfoot
t2\_partner\_conflict\_in\_relationship\_z & 0.00 & -0.06 & 0.06 & 1.05
& 1 \\
t2\_partner\_kessler\_latent\_anxiety\_z & -0.02 & -0.07 & 0.03 & 1.17 &
1 \\
t2\_partner\_kessler\_latent\_depression\_z & 0.04 & -0.01 & 0.10 & 1.24
& 1 \\
t2\_partner\_kessler6\_sum\_z & 0.00 & -0.04 & 0.05 & 1.07 & 1 \\
t2\_partner\_lifesat\_z & -0.04 & -0.09 & 0.01 & 1.24 & 1 \\
t2\_partner\_pwi\_z & 0.04 & -0.01 & 0.09 & 1.23 & 1 \\
t2\_partner\_sat\_relationship\_z & 0.00 & -0.06 & 0.05 & 1.07 & 1 \\
t2\_partner\_self\_esteem\_z & 0.04 & -0.01 & 0.09 & 1.23 & 1 \\

\end{longtable}

\paragraph{Psychopathy Combined Score OSF (Partner): Loss vs
Null}\label{psychopathy-combined-score-osf-partner-loss-vs-null}

\begin{longtable}[]{@{}
  >{\raggedright\arraybackslash}p{(\columnwidth - 10\tabcolsep) * \real{0.4432}}
  >{\raggedleft\arraybackslash}p{(\columnwidth - 10\tabcolsep) * \real{0.1818}}
  >{\raggedleft\arraybackslash}p{(\columnwidth - 10\tabcolsep) * \real{0.0682}}
  >{\raggedleft\arraybackslash}p{(\columnwidth - 10\tabcolsep) * \real{0.0795}}
  >{\raggedleft\arraybackslash}p{(\columnwidth - 10\tabcolsep) * \real{0.0909}}
  >{\raggedleft\arraybackslash}p{(\columnwidth - 10\tabcolsep) * \real{0.1364}}@{}}

\caption{\label{tbl-results-narcissism-null-loss-osf}Table for
Narcissism on partner multi-dimensional well-being: loss vs null.}

\tabularnewline

\toprule\noalign{}
\begin{minipage}[b]{\linewidth}\raggedright
\end{minipage} & \begin{minipage}[b]{\linewidth}\raggedleft
E{[}Y(1){]}-E{[}Y(0){]}
\end{minipage} & \begin{minipage}[b]{\linewidth}\raggedleft
2.5 \%
\end{minipage} & \begin{minipage}[b]{\linewidth}\raggedleft
97.5 \%
\end{minipage} & \begin{minipage}[b]{\linewidth}\raggedleft
E\_Value
\end{minipage} & \begin{minipage}[b]{\linewidth}\raggedleft
E\_Val\_bound
\end{minipage} \\
\midrule\noalign{}
\endhead
\bottomrule\noalign{}
\endlastfoot
t2\_partner\_conflict\_in\_relationship\_z & 0.05 & -0.01 & 0.11 & 1.27
& 1.00 \\
t2\_partner\_kessler\_latent\_anxiety\_z & 0.04 & -0.01 & 0.09 & 1.23 &
1.00 \\
t2\_partner\_kessler\_latent\_depression\_z & 0.08 & 0.03 & 0.13 & 1.36
& 1.19 \\
t2\_partner\_kessler6\_sum\_z & 0.07 & 0.03 & 0.12 & 1.35 & 1.19 \\
t2\_partner\_lifesat\_z & -0.01 & -0.06 & 0.04 & 1.08 & 1.00 \\
t2\_partner\_pwi\_z & -0.07 & -0.12 & -0.02 & 1.34 & 1.16 \\
t2\_partner\_sat\_relationship\_z & -0.07 & -0.13 & -0.01 & 1.33 &
1.13 \\
t2\_partner\_self\_esteem\_z & -0.07 & -0.11 & -0.02 & 1.32 & 1.15 \\

\end{longtable}




\end{document}
