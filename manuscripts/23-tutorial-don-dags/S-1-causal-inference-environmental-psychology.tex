% Options for packages loaded elsewhere
\PassOptionsToPackage{unicode}{hyperref}
\PassOptionsToPackage{hyphens}{url}
\PassOptionsToPackage{dvipsnames,svgnames,x11names}{xcolor}
%
\documentclass[
  singlecolumn]{article}

\usepackage{amsmath,amssymb}
\usepackage{iftex}
\ifPDFTeX
  \usepackage[T1]{fontenc}
  \usepackage[utf8]{inputenc}
  \usepackage{textcomp} % provide euro and other symbols
\else % if luatex or xetex
  \usepackage{unicode-math}
  \defaultfontfeatures{Scale=MatchLowercase}
  \defaultfontfeatures[\rmfamily]{Ligatures=TeX,Scale=1}
\fi
\usepackage[]{libertinus}
\ifPDFTeX\else  
    % xetex/luatex font selection
\fi
% Use upquote if available, for straight quotes in verbatim environments
\IfFileExists{upquote.sty}{\usepackage{upquote}}{}
\IfFileExists{microtype.sty}{% use microtype if available
  \usepackage[]{microtype}
  \UseMicrotypeSet[protrusion]{basicmath} % disable protrusion for tt fonts
}{}
\makeatletter
\@ifundefined{KOMAClassName}{% if non-KOMA class
  \IfFileExists{parskip.sty}{%
    \usepackage{parskip}
  }{% else
    \setlength{\parindent}{0pt}
    \setlength{\parskip}{6pt plus 2pt minus 1pt}}
}{% if KOMA class
  \KOMAoptions{parskip=half}}
\makeatother
\usepackage{xcolor}
\usepackage[top=30mm,left=20mm,heightrounded]{geometry}
\setlength{\emergencystretch}{3em} % prevent overfull lines
\setcounter{secnumdepth}{-\maxdimen} % remove section numbering
% Make \paragraph and \subparagraph free-standing
\makeatletter
\ifx\paragraph\undefined\else
  \let\oldparagraph\paragraph
  \renewcommand{\paragraph}{
    \@ifstar
      \xxxParagraphStar
      \xxxParagraphNoStar
  }
  \newcommand{\xxxParagraphStar}[1]{\oldparagraph*{#1}\mbox{}}
  \newcommand{\xxxParagraphNoStar}[1]{\oldparagraph{#1}\mbox{}}
\fi
\ifx\subparagraph\undefined\else
  \let\oldsubparagraph\subparagraph
  \renewcommand{\subparagraph}{
    \@ifstar
      \xxxSubParagraphStar
      \xxxSubParagraphNoStar
  }
  \newcommand{\xxxSubParagraphStar}[1]{\oldsubparagraph*{#1}\mbox{}}
  \newcommand{\xxxSubParagraphNoStar}[1]{\oldsubparagraph{#1}\mbox{}}
\fi
\makeatother

\usepackage{color}
\usepackage{fancyvrb}
\newcommand{\VerbBar}{|}
\newcommand{\VERB}{\Verb[commandchars=\\\{\}]}
\DefineVerbatimEnvironment{Highlighting}{Verbatim}{commandchars=\\\{\}}
% Add ',fontsize=\small' for more characters per line
\newenvironment{Shaded}{}{}
\newcommand{\AlertTok}[1]{\textcolor[rgb]{1.00,0.33,0.33}{\textbf{#1}}}
\newcommand{\AnnotationTok}[1]{\textcolor[rgb]{0.42,0.45,0.49}{#1}}
\newcommand{\AttributeTok}[1]{\textcolor[rgb]{0.84,0.23,0.29}{#1}}
\newcommand{\BaseNTok}[1]{\textcolor[rgb]{0.00,0.36,0.77}{#1}}
\newcommand{\BuiltInTok}[1]{\textcolor[rgb]{0.84,0.23,0.29}{#1}}
\newcommand{\CharTok}[1]{\textcolor[rgb]{0.01,0.18,0.38}{#1}}
\newcommand{\CommentTok}[1]{\textcolor[rgb]{0.42,0.45,0.49}{#1}}
\newcommand{\CommentVarTok}[1]{\textcolor[rgb]{0.42,0.45,0.49}{#1}}
\newcommand{\ConstantTok}[1]{\textcolor[rgb]{0.00,0.36,0.77}{#1}}
\newcommand{\ControlFlowTok}[1]{\textcolor[rgb]{0.84,0.23,0.29}{#1}}
\newcommand{\DataTypeTok}[1]{\textcolor[rgb]{0.84,0.23,0.29}{#1}}
\newcommand{\DecValTok}[1]{\textcolor[rgb]{0.00,0.36,0.77}{#1}}
\newcommand{\DocumentationTok}[1]{\textcolor[rgb]{0.42,0.45,0.49}{#1}}
\newcommand{\ErrorTok}[1]{\textcolor[rgb]{1.00,0.33,0.33}{\underline{#1}}}
\newcommand{\ExtensionTok}[1]{\textcolor[rgb]{0.84,0.23,0.29}{\textbf{#1}}}
\newcommand{\FloatTok}[1]{\textcolor[rgb]{0.00,0.36,0.77}{#1}}
\newcommand{\FunctionTok}[1]{\textcolor[rgb]{0.44,0.26,0.76}{#1}}
\newcommand{\ImportTok}[1]{\textcolor[rgb]{0.01,0.18,0.38}{#1}}
\newcommand{\InformationTok}[1]{\textcolor[rgb]{0.42,0.45,0.49}{#1}}
\newcommand{\KeywordTok}[1]{\textcolor[rgb]{0.84,0.23,0.29}{#1}}
\newcommand{\NormalTok}[1]{\textcolor[rgb]{0.14,0.16,0.18}{#1}}
\newcommand{\OperatorTok}[1]{\textcolor[rgb]{0.14,0.16,0.18}{#1}}
\newcommand{\OtherTok}[1]{\textcolor[rgb]{0.44,0.26,0.76}{#1}}
\newcommand{\PreprocessorTok}[1]{\textcolor[rgb]{0.84,0.23,0.29}{#1}}
\newcommand{\RegionMarkerTok}[1]{\textcolor[rgb]{0.42,0.45,0.49}{#1}}
\newcommand{\SpecialCharTok}[1]{\textcolor[rgb]{0.00,0.36,0.77}{#1}}
\newcommand{\SpecialStringTok}[1]{\textcolor[rgb]{0.01,0.18,0.38}{#1}}
\newcommand{\StringTok}[1]{\textcolor[rgb]{0.01,0.18,0.38}{#1}}
\newcommand{\VariableTok}[1]{\textcolor[rgb]{0.89,0.38,0.04}{#1}}
\newcommand{\VerbatimStringTok}[1]{\textcolor[rgb]{0.01,0.18,0.38}{#1}}
\newcommand{\WarningTok}[1]{\textcolor[rgb]{1.00,0.33,0.33}{#1}}

\providecommand{\tightlist}{%
  \setlength{\itemsep}{0pt}\setlength{\parskip}{0pt}}\usepackage{longtable,booktabs,array}
\usepackage{calc} % for calculating minipage widths
% Correct order of tables after \paragraph or \subparagraph
\usepackage{etoolbox}
\makeatletter
\patchcmd\longtable{\par}{\if@noskipsec\mbox{}\fi\par}{}{}
\makeatother
% Allow footnotes in longtable head/foot
\IfFileExists{footnotehyper.sty}{\usepackage{footnotehyper}}{\usepackage{footnote}}
\makesavenoteenv{longtable}
\usepackage{graphicx}
\makeatletter
\def\maxwidth{\ifdim\Gin@nat@width>\linewidth\linewidth\else\Gin@nat@width\fi}
\def\maxheight{\ifdim\Gin@nat@height>\textheight\textheight\else\Gin@nat@height\fi}
\makeatother
% Scale images if necessary, so that they will not overflow the page
% margins by default, and it is still possible to overwrite the defaults
% using explicit options in \includegraphics[width, height, ...]{}
\setkeys{Gin}{width=\maxwidth,height=\maxheight,keepaspectratio}
% Set default figure placement to htbp
\makeatletter
\def\fps@figure{htbp}
\makeatother
% definitions for citeproc citations
\NewDocumentCommand\citeproctext{}{}
\NewDocumentCommand\citeproc{mm}{%
  \begingroup\def\citeproctext{#2}\cite{#1}\endgroup}
\makeatletter
 % allow citations to break across lines
 \let\@cite@ofmt\@firstofone
 % avoid brackets around text for \cite:
 \def\@biblabel#1{}
 \def\@cite#1#2{{#1\if@tempswa , #2\fi}}
\makeatother
\newlength{\cslhangindent}
\setlength{\cslhangindent}{1.5em}
\newlength{\csllabelwidth}
\setlength{\csllabelwidth}{3em}
\newenvironment{CSLReferences}[2] % #1 hanging-indent, #2 entry-spacing
 {\begin{list}{}{%
  \setlength{\itemindent}{0pt}
  \setlength{\leftmargin}{0pt}
  \setlength{\parsep}{0pt}
  % turn on hanging indent if param 1 is 1
  \ifodd #1
   \setlength{\leftmargin}{\cslhangindent}
   \setlength{\itemindent}{-1\cslhangindent}
  \fi
  % set entry spacing
  \setlength{\itemsep}{#2\baselineskip}}}
 {\end{list}}
\usepackage{calc}
\newcommand{\CSLBlock}[1]{\hfill\break\parbox[t]{\linewidth}{\strut\ignorespaces#1\strut}}
\newcommand{\CSLLeftMargin}[1]{\parbox[t]{\csllabelwidth}{\strut#1\strut}}
\newcommand{\CSLRightInline}[1]{\parbox[t]{\linewidth - \csllabelwidth}{\strut#1\strut}}
\newcommand{\CSLIndent}[1]{\hspace{\cslhangindent}#1}

\usepackage{booktabs}
\usepackage{longtable}
\usepackage{array}
\usepackage{multirow}
\usepackage{wrapfig}
\usepackage{float}
\usepackage{colortbl}
\usepackage{pdflscape}
\usepackage{tabu}
\usepackage{threeparttable}
\usepackage{threeparttablex}
\usepackage[normalem]{ulem}
\usepackage{makecell}
\usepackage{xcolor}
\input{/Users/joseph/GIT/latex/latex-for-quarto.tex}
\makeatletter
\@ifpackageloaded{caption}{}{\usepackage{caption}}
\AtBeginDocument{%
\ifdefined\contentsname
  \renewcommand*\contentsname{Table of contents}
\else
  \newcommand\contentsname{Table of contents}
\fi
\ifdefined\listfigurename
  \renewcommand*\listfigurename{List of Figures}
\else
  \newcommand\listfigurename{List of Figures}
\fi
\ifdefined\listtablename
  \renewcommand*\listtablename{List of Tables}
\else
  \newcommand\listtablename{List of Tables}
\fi
\ifdefined\figurename
  \renewcommand*\figurename{Figure}
\else
  \newcommand\figurename{Figure}
\fi
\ifdefined\tablename
  \renewcommand*\tablename{Table}
\else
  \newcommand\tablename{Table}
\fi
}
\@ifpackageloaded{float}{}{\usepackage{float}}
\floatstyle{ruled}
\@ifundefined{c@chapter}{\newfloat{codelisting}{h}{lop}}{\newfloat{codelisting}{h}{lop}[chapter]}
\floatname{codelisting}{Listing}
\newcommand*\listoflistings{\listof{codelisting}{List of Listings}}
\makeatother
\makeatletter
\makeatother
\makeatletter
\@ifpackageloaded{caption}{}{\usepackage{caption}}
\@ifpackageloaded{subcaption}{}{\usepackage{subcaption}}
\makeatother
\ifLuaTeX
  \usepackage{selnolig}  % disable illegal ligatures
\fi
\usepackage{bookmark}

\IfFileExists{xurl.sty}{\usepackage{xurl}}{} % add URL line breaks if available
\urlstyle{same} % disable monospaced font for URLs
\hypersetup{
  pdftitle={Supplement: Causal Inference in Environmental Psychology},
  pdfkeywords={DAGS, Causal
Inference, Confounding, Environmental, Psychology, Panel},
  colorlinks=true,
  linkcolor={blue},
  filecolor={Maroon},
  citecolor={Blue},
  urlcolor={Blue},
  pdfcreator={LaTeX via pandoc}}

\title{Supplement: Causal Inference in Environmental Psychology}

\usepackage{academicons}
\usepackage{xcolor}

  \author{Joseph A Bulbulia}
            \affil{%
             \small{     Victoria University of Wellington, New Zealand,
School of Psychology, Centre for Applied Cross-Cultural Research
          ORCID \textcolor[HTML]{A6CE39}{\aiOrcid} ~0000-0002-5861-2056 }
              }
      \usepackage{academicons}
\usepackage{xcolor}

  \author{Donald W Hine}
            \affil{%
             \small{     University of Canterbury, School of Psychology,
Speech and Hearing
          ORCID \textcolor[HTML]{A6CE39}{\aiOrcid} ~0000-0002-3905-7026 }
              }
      


\date{2024-10-04}
\begin{document}
\maketitle

\subsection{Appendix A: Glossary}\label{appendix-a}

This appendix provides a glossary of common terminology in causal
inference.

\textbf{Acyclic}: a causal diagram cannot contain feedback loops. More
precisely, no variable can be an ancestor or descendant of itself. If
variables are repeatedly measured here, it is vital to index nodes by
the relative timing of the nodes.

\textbf{Adjustment set}: a collection of variables we must either
condition upon or deliberately avoid conditioning upon to obtain a
consistent causal estimate for the effect of interest
(\citeproc{ref-pearl2009}{Pearl 2009}).

\textbf{Ancestor (parent)}: a node with a direct or indirect influence
on others, positioned upstream in the causal chain.

\textbf{Arrow}: denotes a causal relationship linking nodes.

\textbf{Backdoor path}: a ``backdoor path'' between a treatment
variable, \(A\), and an outcome variable, \(Y\), is a sequence of links
in a causal diagram that starts with an arrow into \(A\) and reaches
\(Y\) through common causes, introducing potential confounding bias such
that statistical association does not reflect causality. To estimate the
causal effect of \(A\) on \(Y\) without bias, these paths must be
blocked by adjusting for confounders. The backdoor criterion guides the
selection of variables for adjustment to ensure unbiased causal
inference.

\textbf{Conditioning}: explicitly accounting for a variable in our
statistical analysis to address the identification problem. In causal
diagrams, we usually represent conditioning by drawing a box around a
node of the conditioned variable, for example,
\(\boxed{L_{0}}\to A_{1} \to L_{2}\). We do not box exposures and
outcomes because we assume they are included in a model by default.
Depending on the setting, we may condition by regression stratification,
inverse probability of treatment weighting, g-methods, doubly robust
machine learning algorithms, or other methods. We do not cover such
methods in this tutorial; however, see Hernan and Robins
(\citeproc{ref-hernan2023}{2023}).

\textbf{Counterfactual}: a hypothetical outcome that would have occurred
for the same individuals under a different treatment condition than the
one they experienced.

\textbf{Direct effect}: the portion of the total effect of a treatment
on an outcome that is not mediated by other variables within the causal
pathway.

\textbf{Collider}: a variable in a causal diagram at which two incoming
paths meet head-to-head. For example, if
\(A \rightarrowred \boxed{L} \leftarrowred Y\), then \(L\) is a
collider. If we do not condition on a collider (or its descendants), the
path between \(A\) and \(Y\) remains closed. Conditioning on a collider
(or its descendants) will induce an association between \(A\) and \(Y\).

\textbf{Confounder}: a member of an adjustment set. Notice a variable is
a ``confounder'' in relation to a specific adjustment set.
``Confounder'' is a relative concept (\citeproc{ref-lash2020}{Lash
\emph{et al.} 2020}).

\textbf{d-separation}: in a causal diagram, a path is ``blocked'' or
``d-separated'' if a node along it interrupts causation. Two variables
are d-separated if all paths connecting them are blocked, making them
conditionally independent. Conversely, unblocked paths result in
``d-connected'' variables, implying potential dependence
(\citeproc{ref-pearl1995}{Pearl 1995}).

\textbf{Descendant (child)}: a node directly or indirectly influenced by
upstream nodes (parents).

\textbf{Effect-modifier}: a variable is an effect-modifier, or
``effect-measure modifie'' if its presence changes the magnitude or
direction of the effect of an exposure or treatment on an outcome across
the levels or values of this variable. In other words, the effect of the
exposure is different at different levels of the effect modifier.

\textbf{External validity}: the extent to which causal inferences can be
generalised to other populations, settings, or times, also called
``Target Validity.''

\textbf{Identification problem}: the challenge of estimating the causal
effect of a variable by adjusting for measured variables on units in a
study. Causal diagrams were developed to address the identification
problem by application of the rules of d-separation to a causal diagram.

\textbf{Indirect effect (mediated effect)}: The portion of the total
effect transmitted through a mediator variable.

\textbf{Internal validity}: the degree to which the design and conduct
have prevented bias, ensuring that the causal relationship observed can
be confidently attributed to the treatment and not to other factors.

\textbf{Instrumental variable}: an ancestor of the exposure but not of
the outcome. An instrumental variable affects the outcome only through
its effect on the exposure and not otherwise. Whereas conditioning on a
variable causally associated with the outcome rather than with the
exposure will generally increase modelling precision, we should refrain
from conditioning on instrumental variables
(\citeproc{ref-cinelli2022}{Cinelli \emph{et al.} 2022}). Second, when
an instrumental variable is the descendant of an unmeasured confounder,
we should generally condition the instrumental variable to provide a
partial adjustment for a confounder.

\textbf{Mediator}: a variable that transmits the effect of the treatment
variable on the outcome variable, part of the causal pathway between
treatment and outcome.

\textbf{Modified Disjunctive Cause Criterion}: VanderWeele
(\citeproc{ref-vanderweele2019}{2019}) recommends obtaining a maximally
efficient adjustment, which he calls a ``confounder set.'' A member of
this set is any set of variables that can reduce or remove structural
sources of bias. The strategy is as follows:

\begin{enumerate}
\def\labelenumi{\alph{enumi}.}
\tightlist
\item
  Control for any variable that causes the exposure, the outcome, or
  both.
\item
  Control for any proxy for an unmeasured variable that is a shared
  cause of the exposure and outcome.
\item
  Define an instrumental variable as a variable associated with the
  exposure but does not influence the outcome independently, except
  through the exposure. Exclude any instrumental variable that is not a
  proxy for an unmeasured confounder from the confounder set
  (\citeproc{ref-vanderweele2019}{VanderWeele 2019}).
\end{enumerate}

Note that the concept of a ``confounder set'' is broader than that of an
``adjustment set'' Every adjustment set is a member of a confounder set.
Hence, the Modified Disjunctive Cause Criterion will eliminate bias when
the data permit. However, a confounder set includes variables that
reduce bias in cases where confounding cannot be eliminated.

\textbf{Node}: characteristic or features of units in a population (a
variable) represented on a causal diagram. In a causal diagram, nodes
are drawn with reference to variable distributions for the target
population.

\textbf{Randomisation}: the process of randomly assigning subjects to
different treatments or control groups to eliminate selection bias in
experimental studies.

\textbf{Reverse causation}: \(\atoyassert\), but in reality \(\ytoa\)

\textbf{Statistical model:} a mathematical representation of the
relationships between variables in which we quantify covariances and
their corresponding uncertainties in the data. Statistical models
typically correspond to multiple causal structures
(\citeproc{ref-hernan2023}{Hernan and Robins 2023};
\citeproc{ref-pearl2018}{Pearl and Mackenzie 2018};
\citeproc{ref-vanderweele2022b}{\textbf{vanderweele2022b?}}). That is,
the causes of such covariances cannot be identified without assumptions.

\textbf{Structural model:} defines assumptions about causal
relationships. Causal diagrams graphically encode these assumptions
(\citeproc{ref-hernan2023}{Hernan and Robins 2023}), leaving out the
assumption about whether the exposure and outcome are causally
associated. We can only compute causal effects outside of randomised
experiments with structural models. A structural model is needed to
interpret the statistical findings in causal terms. Structural
assumptions should be developed in consultation with experts. The role
of structural assumptions when interpreting statistical results needs to
be better understood across many human sciences and forms the motivation
for my work here.

\textbf{Time-varying confounding:} occurs when a confounder that changes
over time acts as a mediator or collider in the causal pathway between
exposure and outcome. Controlling for such a confounder can introduce
bias. Not controlling for it can retain bias.

\newpage{}

\subsection{Appendix B: Causal Consistency in observational
settings}\label{appendix-b}

In observational research, there are typically multiple versions of the
treatment. The theory of causal inference under multiple versions of
treatment proves we can consistently estimate causal effects where the
different versions of treatment are conditionally independent of the
outcomes VanderWeele (\citeproc{ref-vanderweele2009}{2009})

Let \(\coprod\) denote independence. Where there are \(K\) different
versions of treatment \(A\) and no confounding for \(K\)'s effect on
\(Y\) given measured confounders \(L\) such that

\[
Y(k) \coprod K | L
\]

Then it can be proved that causal consistency follows. According to the
theory of causal inference under multiple versions of treatment, the
measured variable \(A\) functions as a ``coarsened indicator'' for
estimating the causal effect of the multiple versions of treatment \(K\)
on \(Y(k)\) (\citeproc{ref-vanderweele2009}{VanderWeele 2009},
\citeproc{ref-vanderweele2018}{2018};
\citeproc{ref-vanderweele2013}{VanderWeele and Hernan 2013}).

In the context of green spaces, let \(A\) represent the general action
of moving closer to any green space and \(K\) represent the different
versions of this treatment. For instance, \(K\) could denote moving
closer to different green spaces such as parks, forests, community
gardens, or green spaces with varying amenities and features.

Here, the conditional independence implies that, given measured
confounders \(L\) (e.g.~socioeconomic status, age, personal values), the
type of green space one moves closer to (\(K\)) is independent of the
outcomes \(Y(k)\) (e.g.~mental well-being under the \(K\) conditions).
In other words, the version of green space one chooses to live near does
not affect the \(K\) potential outcomes, provided the confounders \(L\)
are appropriately controlled for in our statistical models.

Put simply, strategies for confounding control and consistently
estimating causal effects when multiple treatment versions converge.
However, the quantities we estimate under multiple treatment versions
might need clearer interpretations. For example, we cannot readily
determine which of the many treatment versions is most causally
efficacious and which lack any causal effect or are harmful.

\newpage{}

\subsubsection{Appendix C Simulation of Cross-Sectional Data to Compute
the Average Treatment Effect When Conditioning on a
Mediator}\label{appendix-c}

This appendix outlines a simulation designed to demonstrate the
potential pitfalls of conditioning on a mediator in cross-sectional
analyses. The simulation examines the scenario where the effect of
access to green space (\(A\)) on happiness (\(Y\)) is fully mediated by
exercise (\(L\)). This setup aims to illustrate how incorrect
assumptions about the role of a variable (mediator vs.~confounder) can
lead to misleading estimates of the Average Treatment Effect (ATE).

\paragraph{Methodology}\label{methodology}

\textbf{Data Generation}: we simulate a dataset for 1,000 individuals,
where access to green space (\(A\)) influences exercise (\(L\)), which
in turn affects happiness (\(Y\)\$). The simulation is based on
predefined parameters that establish \(L\) as a mediator between \(A\)
and \(Y\).

\textbf{Parameter Definitions}:

\begin{itemize}
\tightlist
\item
  The probability of access to green space (\(A\)) is set at 0.5.
\item
  The effect of \(A\) on \(L\) (exercise) is given by \(\beta = 2\).
\item
  The effect of \(L\) on \(Y\) (happiness) is given by \(\delta = 1.5\).
\item
  Standard deviations for \(L\) and \(Y\) are set at 1 and 1.5,
  respectively.
\end{itemize}

\textbf{Model 1} (Correct Assumption): fits a linear regression model
assuming \(L\) as a mediator, including both \(A\) and \(L\) as
regressors on \(Y\). This model aligns with the data-generating process,
and, by the rules of d-separation, induces mediator bias for the
\(A\to Y\) path.

\textbf{Model 2} (Incorrect Assumption): fits a linear regression model
including only \(A\) as a regressor on \(Y\), omitting the mediator
\(L\). This model assesses the direct effect of A on Y without
accounting for mediation.

\textbf{Analysis}: We compares the estimated effects of \(A\) on \(Y\)
under each model specification.

\begin{Shaded}
\begin{Highlighting}[]
\CommentTok{\# load libraries}
\SpecialCharTok{!}\FunctionTok{require}\NormalTok{(kableExtra)}\ErrorTok{)}\NormalTok{\{}\FunctionTok{install.packages}\NormalTok{(}\StringTok{"kableExtra"}\NormalTok{)\} }\CommentTok{\# tables}
\ControlFlowTok{if}\NormalTok{(}\SpecialCharTok{!}\FunctionTok{require}\NormalTok{(gtsummary))\{}\FunctionTok{install.packages}\NormalTok{(}\StringTok{"gtsummary"}\NormalTok{)\} }\CommentTok{\# tables}

\CommentTok{\# simulation seed}
\FunctionTok{set.seed}\NormalTok{(}\DecValTok{123}\NormalTok{) }\CommentTok{\#  reproducibility}

\CommentTok{\# define the parameters }
\NormalTok{n }\OtherTok{=} \DecValTok{1000} \CommentTok{\# Number of observations}
\NormalTok{p }\OtherTok{=} \FloatTok{0.5}  \CommentTok{\# Probability of A = 1 (access to greenspace)}
\NormalTok{alpha }\OtherTok{=} \DecValTok{0} \CommentTok{\# Intercept for L (exercise)}
\NormalTok{beta }\OtherTok{=} \DecValTok{2}  \CommentTok{\# Effect of A on L }
\NormalTok{gamma }\OtherTok{=} \DecValTok{1} \CommentTok{\# Intercept for Y }
\NormalTok{delta }\OtherTok{=} \FloatTok{1.5} \CommentTok{\# Effect of L on Y}
\NormalTok{sigma\_L }\OtherTok{=} \DecValTok{1} \CommentTok{\# Standard deviation of L}
\NormalTok{sigma\_Y }\OtherTok{=} \FloatTok{1.5} \CommentTok{\# Standard deviation of Y}

\CommentTok{\# simulate the data: fully mediated effect by L}
\NormalTok{A }\OtherTok{=} \FunctionTok{rbinom}\NormalTok{(n, }\DecValTok{1}\NormalTok{, p) }\CommentTok{\# binary exposure variable}
\NormalTok{L }\OtherTok{=}\NormalTok{ alpha }\SpecialCharTok{+}\NormalTok{ beta}\SpecialCharTok{*}\NormalTok{A }\SpecialCharTok{+} \FunctionTok{rnorm}\NormalTok{(n, }\DecValTok{0}\NormalTok{, sigma\_L) }\CommentTok{\# mediator L affect by A}
\NormalTok{Y }\OtherTok{=}\NormalTok{ gamma }\SpecialCharTok{+}\NormalTok{ delta}\SpecialCharTok{*}\NormalTok{L }\SpecialCharTok{+} \FunctionTok{rnorm}\NormalTok{(n, }\DecValTok{0}\NormalTok{, sigma\_Y) }\CommentTok{\# Y affected only by L,}

\CommentTok{\# make the data frame}
\NormalTok{data }\OtherTok{=} \FunctionTok{data.frame}\NormalTok{(}\AttributeTok{A =}\NormalTok{ A, }\AttributeTok{L =}\NormalTok{ L, }\AttributeTok{Y =}\NormalTok{ Y)}

\CommentTok{\# fit regression in which we control for L, a mediator}
\CommentTok{\# (cross{-}sectional data is consistent with this model)}
\NormalTok{fit\_1 }\OtherTok{\textless{}{-}} \FunctionTok{lm}\NormalTok{( Y }\SpecialCharTok{\textasciitilde{}}\NormalTok{ A }\SpecialCharTok{+}\NormalTok{ L, }\AttributeTok{data =}\NormalTok{ data)}

\CommentTok{\# fit regression in which L is assumed to be a mediator, not a confounder.}
\CommentTok{\# (cross{-}sectional data is also consistent with this model)}
\NormalTok{fit\_2 }\OtherTok{\textless{}{-}} \FunctionTok{lm}\NormalTok{( Y }\SpecialCharTok{\textasciitilde{}}\NormalTok{ A, }\AttributeTok{data =}\NormalTok{ data)}

\CommentTok{\# create gtsummary tables for each regression model}
\NormalTok{table1 }\OtherTok{\textless{}{-}}\NormalTok{ gtsummary}\SpecialCharTok{::}\FunctionTok{tbl\_regression}\NormalTok{(fit\_1)}
\NormalTok{table2 }\OtherTok{\textless{}{-}}\NormalTok{ gtsummary}\SpecialCharTok{::}\FunctionTok{tbl\_regression}\NormalTok{(fit\_2)}

\CommentTok{\# merge the tables for comparison}
\NormalTok{table\_comparison }\OtherTok{\textless{}{-}}\NormalTok{ gtsummary}\SpecialCharTok{::}\FunctionTok{tbl\_merge}\NormalTok{(}
  \FunctionTok{list}\NormalTok{(table1, table2),}
  \AttributeTok{tab\_spanner =} \FunctionTok{c}\NormalTok{(}\StringTok{"Model: Exercise assumed confounder"}\NormalTok{, }
                  \StringTok{"Model: Exercise assumed to be a mediator"}\NormalTok{)}
\NormalTok{)}
\CommentTok{\# make latex table (for publication)}
\NormalTok{markdown\_table\_0 }\OtherTok{\textless{}{-}} \FunctionTok{as\_kable\_extra}\NormalTok{(table\_comparison, }
                                   \AttributeTok{format =} \StringTok{"latex"}\NormalTok{, }
                                   \AttributeTok{booktabs =} \ConstantTok{TRUE}\NormalTok{)}
\CommentTok{\# print latex table (note, you might prefer "markdown" or another format)                                }
\NormalTok{markdown\_table\_0}
\end{Highlighting}
\end{Shaded}

The following code is designed to estimate the Average Treatment Effect
(ATE) using the \texttt{clarify} package in R, which is referenced here
as (\citeproc{ref-greifer2023}{Greifer \emph{et al.} 2023}). The
procedure involves two steps: simulating coefficient distributions for
regression models and then calculating the ATE based on these
simulations. This process is applied to two distinct models to
demonstrate the effects of including versus excluding a mediator
variable in the analysis.

\subsubsection{Steps to Estimate the
ATE}\label{steps-to-estimate-the-ate}

\begin{enumerate}
\def\labelenumi{\arabic{enumi}.}
\item
  \textbf{Load the \texttt{clarify} Package}: this package provides
  functions to simulate regression coefficients and compute average
  marginal effects (AME), robustly facilitating the estimation of ATE.
\item
  \textbf{Set seed}: \texttt{set.seed(123)} ensures that the results of
  the simulations are reproducible, allowing for consistent outcomes
  across different code runs.
\item
  \textbf{Simulate the data distribution}:

  \texttt{sim\_coefs\_fit\_1} and \texttt{sim\_coefs\_fit\_2} are
  generated using the \texttt{sim} function from the \texttt{clarify}
  package, applied to two fitted models (\texttt{fit\_1} and
  \texttt{fit\_2}). These functions simulate the distribution of
  coefficients based on the specified models, capturing the uncertainty
  around the estimated parameters.
\item
  \textbf{Calculate ATE}:
\end{enumerate}

For both models, the \texttt{sim\_ame} function calculates the ATE as
the marginal risk difference (RD) when the treatment variable
(\texttt{A}) is present (\texttt{A\ ==\ 1}). This function uses the
simulated coefficients to estimate the treatment effect across the
simulated distributions, providing a comprehensive view of the ATE under
each model.

To streamline the output, the function is set to verbose mode off
(\texttt{verbose\ =\ FALSE}).

\begin{enumerate}
\def\labelenumi{\arabic{enumi}.}
\setcounter{enumi}{4}
\tightlist
\item
  \textbf{Results}:
\end{enumerate}

Summaries of these estimates (\texttt{summary\_sim\_est\_fit\_1} and
\texttt{summary\_sim\_est\_fit\_2}) are obtained, providing detailed
statistics including the estimated ATE and its 95\% confidence intervals
(CI).

\begin{enumerate}
\def\labelenumi{\arabic{enumi}.}
\setcounter{enumi}{5}
\item
  \textbf{Presentation: report ATE and CIs}:

  Using the \texttt{glue} package, the ATE and its 95\% CIs for both
  models are formatted into a string for easy reporting. This step
  transforms the statistical output into a more interpretable form,
  highlighting the estimated treatment effect and its precision.
\end{enumerate}

\begin{Shaded}
\begin{Highlighting}[]
\CommentTok{\# use \textasciigrave{}clarify\textasciigrave{} package to obtain ATE}
\ControlFlowTok{if}\NormalTok{(}\SpecialCharTok{!}\FunctionTok{require}\NormalTok{(clarify))\{}\FunctionTok{install.packages}\NormalTok{(}\StringTok{"clarify"}\NormalTok{)\} }\CommentTok{\# clarify package}
\CommentTok{\# simulate fit 1 ATE}
\FunctionTok{set.seed}\NormalTok{(}\DecValTok{123}\NormalTok{)}
\NormalTok{sim\_coefs\_fit\_1 }\OtherTok{\textless{}{-}} \FunctionTok{sim}\NormalTok{(fit\_1)}
\NormalTok{sim\_coefs\_fit\_2 }\OtherTok{\textless{}{-}} \FunctionTok{sim}\NormalTok{(fit\_2)}

\CommentTok{\# marginal risk difference ATE, simulation{-}based: model 1 (L is a confounder)}
\NormalTok{sim\_est\_fit\_1 }\OtherTok{\textless{}{-}}
  \FunctionTok{sim\_ame}\NormalTok{(}
\NormalTok{    sim\_coefs\_fit\_1,}
    \AttributeTok{var =} \StringTok{"A"}\NormalTok{,}
    \AttributeTok{subset =}\NormalTok{ A }\SpecialCharTok{==} \DecValTok{1}\NormalTok{,}
    \AttributeTok{contrast =} \StringTok{"RD"}\NormalTok{,}
    \AttributeTok{verbose =} \ConstantTok{FALSE}
\NormalTok{  )}
\CommentTok{\# marginal risk difference ATE, simulation{-}based: model 2 (L is a mediator)}
\NormalTok{sim\_est\_fit\_2 }\OtherTok{\textless{}{-}}
  \FunctionTok{sim\_ame}\NormalTok{(}
\NormalTok{    sim\_coefs\_fit\_2,}
    \AttributeTok{var =} \StringTok{"A"}\NormalTok{,}
    \AttributeTok{subset =}\NormalTok{ A }\SpecialCharTok{==} \DecValTok{1}\NormalTok{,}
    \AttributeTok{contrast =} \StringTok{"RD"}\NormalTok{,}
    \AttributeTok{verbose =} \ConstantTok{FALSE}
\NormalTok{  )}
\CommentTok{\# obtain summaries}
\NormalTok{summary\_sim\_est\_fit\_1 }\OtherTok{\textless{}{-}} \FunctionTok{summary}\NormalTok{(sim\_est\_fit\_1, }\AttributeTok{null =} \FunctionTok{c}\NormalTok{(}\StringTok{\textasciigrave{}}\AttributeTok{RD}\StringTok{\textasciigrave{}} \OtherTok{=} \DecValTok{0}\NormalTok{))}
\NormalTok{summary\_sim\_est\_fit\_2 }\OtherTok{\textless{}{-}} \FunctionTok{summary}\NormalTok{(sim\_est\_fit\_2, }\AttributeTok{null =} \FunctionTok{c}\NormalTok{(}\StringTok{\textasciigrave{}}\AttributeTok{RD}\StringTok{\textasciigrave{}} \OtherTok{=} \DecValTok{0}\NormalTok{))}

\CommentTok{\# reporting }
\CommentTok{\# ate for fit 1, with 95\% CI}
\NormalTok{ATE\_fit\_1 }\OtherTok{\textless{}{-}}\NormalTok{ glue}\SpecialCharTok{::}\FunctionTok{glue}\NormalTok{(}
  \StringTok{"ATE =}
\StringTok{                        \{round(summary\_sim\_est\_fit\_1[3, 1], 2)\},}
\StringTok{                        CI = [\{round(summary\_sim\_est\_fit\_1[3, 2], 2)\},}
\StringTok{                        \{round(summary\_sim\_est\_fit\_1[3, 3], 2)\}]"}
\NormalTok{)}
\CommentTok{\# ate for fit 2, with 95\% CI}
\NormalTok{ATE\_fit\_2 }\OtherTok{\textless{}{-}}
\NormalTok{  glue}\SpecialCharTok{::}\FunctionTok{glue}\NormalTok{(}
    \StringTok{"ATE = \{round(summary\_sim\_est\_fit\_2[3, 1], 2)\},}
\StringTok{                        CI = [\{round(summary\_sim\_est\_fit\_2[3, 2], 2)\},}
\StringTok{                        \{round(summary\_sim\_est\_fit\_2[3, 3], 2)\}]"}
\NormalTok{  )}
\end{Highlighting}
\end{Shaded}

\subsubsection{Upshot of the Simulation and
Analysis}\label{upshot-of-the-simulation-and-analysis}

\begin{itemize}
\item
  \textbf{Model 1 (L as a Confounder)}: this analysis assumes that
  \texttt{L} is a confounder in the relationship between the treatment
  (\texttt{A}) and the outcome (\texttt{Y}), and thus, it includes
  \texttt{L} in the model. The ATE estimated here reflects the effect of
  \texttt{A} while controlling for \texttt{L}.
\item
  \textbf{Model 2 (L as a Mediator)}: in contrast, this analysis
  considers \texttt{L} to be a mediator, and the model either includes
  \texttt{L} explicitly in its estimation process or excludes it to
  examine the direct effect of \texttt{A} on \texttt{Y}. The approach to
  mediation analysis here is crucial as it influences the interpretation
  of the ATE.
\end{itemize}

By comparing the ATEs from both models, researchers can understand the
effect of mediation (or the lack thereof) on the estimated treatment
effect. This comparison sheds light on how assumptions about variable
roles (confounder vs.~mediator) can significantly alter causal
inferences drawn from cross-sectional data.

\textbf{Wherever it is uncertain whether a variable is a confounder or a
mediator, we suggest creating two causal diagrams and reporting both
analyses.}

\newpage{}

\subsection{Appendix D: Simulation of Different Confounding Control
Strategies}\label{appendix-d}

This appendix outlines the methodology and results of a data simulation
designed to compare different strategies for controlling confounding in
the context of environmental psychology research. Specifically, the
simulation examines the effect of access to open green spaces
(treatment, \(A_1\)) on happiness (outcome, \(Y_2\)) while addressing
the challenge of unmeasured confounding. The simulation incorporates
baseline measures of exposure and outcome (\(A_0\), \(Y_0\)), baseline
confounders (\(L_0\)), and an unmeasured confounder (\(U\)) to evaluate
the effectiveness of different analytical approaches.

\subsubsection{Methodology}\label{methodology-1}

1.\textbf{Load Libraries \texttt{kableExtra}, \texttt{gtsummary}, and
\texttt{grf}.}

\begin{enumerate}
\def\labelenumi{\arabic{enumi}.}
\item
  \textbf{Target}: we simulate data for 10,000 individuals, including
  baseline exposure to green spaces (\(A_0\)), baseline happiness
  (\(Y_0\)), baseline confounders (\(L_0\)), and an unmeasured
  confounder (\(U\)). The simulation uses a logistic model for treatment
  assignment and a linear model for the continuous outcome,
  incorporating interactions to assess how baseline characteristics
  modify the treatment effect.
\item
  \textbf{Set seed and simulate the data distribution}:
\end{enumerate}

Treatment assignment coefficients: \(\beta_{A0} = 0.25\),
\(\beta_{Y0} = 0.3\), \(\beta_{L0} = 0.2\), and \(\beta_{U} = 0.1\).
Outcome model coefficients: \(\delta_{A1} = 0.3\),
\(\delta_{Y0} = 0.9\), \(\delta_{A0} = 0.1\), \(\delta_{L0} = 0.3\),
with an interaction effect (\(\theta_{A0Y0L0} = 0.5\)) indicating the
combined influence of baseline exposure, outcome, and confounders on the
follow-up outcome.

\begin{enumerate}
\def\labelenumi{\arabic{enumi}.}
\setcounter{enumi}{2}
\item
  \textbf{Model comparison}:

  \begin{itemize}
  \tightlist
  \item
    \textbf{No control model}: estimates the effect of \(A_1\) on
    \(Y_2\) without controlling for any confounders.
  \item
    \textbf{Standard covariate control model}: controls for baseline
    confounders (\(L_0\)) alongside treatment (\(A_1\)).
  \item
    \textbf{Baseline exposure and outcome model}: extends the standard
    model by including baseline treatment and outcome (\(A_0\), \(Y_0\))
    and their interaction with \(L_0\).
  \end{itemize}
\item
  \textbf{Results}: each model's effectiveness in estimating the true
  treatment effect is assessed by comparing regression outputs. The
  simulation evaluates how well each model addresses the bias introduced
  by unmeasured confounding and the role of baseline characteristics in
  modifying treatment effects.
\item
  \textbf{Presentation}: the results are synthesised in a comparative
  table, formatted using the \texttt{kableExtra} \{Zhu
  (\citeproc{ref-zhu2021KableExtra}{2021}){]} and \texttt{gtsummary}
  packages (\citeproc{ref-gtsummary2021}{Sjoberg \emph{et al.} 2021}),
  highlighting the estimated treatment effects and their statistical
  significance across models.
\end{enumerate}

Overall, we use the simulation to illustrate the importance of
incorporating baseline characteristics and their interactions to
mitigate the influence of unmeasured confounding.

Here is the simulation/model code:

\begin{Shaded}
\begin{Highlighting}[]
\FunctionTok{library}\NormalTok{(kableExtra)}
\ControlFlowTok{if}\NormalTok{(}\SpecialCharTok{!}\FunctionTok{require}\NormalTok{(kableExtra))\{}\FunctionTok{install.packages}\NormalTok{(}\StringTok{"kableExtra"}\NormalTok{)\} }\CommentTok{\# causal forest}
\ControlFlowTok{if}\NormalTok{(}\SpecialCharTok{!}\FunctionTok{require}\NormalTok{(gtsummary))\{}\FunctionTok{install.packages}\NormalTok{(}\StringTok{"gtsummary"}\NormalTok{)\} }\CommentTok{\# causal forest}
\ControlFlowTok{if}\NormalTok{(}\SpecialCharTok{!}\FunctionTok{require}\NormalTok{(grf))\{}\FunctionTok{install.packages}\NormalTok{(}\StringTok{"grf"}\NormalTok{)\} }\CommentTok{\# causal forest}

\CommentTok{\# r\_texmf()eproducibility}
\FunctionTok{set.seed}\NormalTok{(}\DecValTok{123}\NormalTok{) }

\CommentTok{\# set number of observations}
\NormalTok{n }\OtherTok{\textless{}{-}} \DecValTok{10000} 

\CommentTok{\# baseline covariates}
\NormalTok{U }\OtherTok{\textless{}{-}} \FunctionTok{rnorm}\NormalTok{(n) }\CommentTok{\# Unmeasured confounder}
\NormalTok{A\_0 }\OtherTok{\textless{}{-}} \FunctionTok{rbinom}\NormalTok{(n, }\DecValTok{1}\NormalTok{, }\AttributeTok{prob =} \FunctionTok{plogis}\NormalTok{(U)) }\CommentTok{\# Baseline exposure}
\NormalTok{Y\_0 }\OtherTok{\textless{}{-}} \FunctionTok{rnorm}\NormalTok{(n, }\AttributeTok{mean =}\NormalTok{ U, }\AttributeTok{sd =} \DecValTok{1}\NormalTok{) }\CommentTok{\# Baseline outcome}
\NormalTok{L\_0 }\OtherTok{\textless{}{-}} \FunctionTok{rnorm}\NormalTok{(n, }\AttributeTok{mean =}\NormalTok{ U, }\AttributeTok{sd =} \DecValTok{1}\NormalTok{) }\CommentTok{\# Baseline confounders}

\CommentTok{\# coefficients for treatment assignment}
\NormalTok{beta\_A0 }\OtherTok{=} \FloatTok{0.25}
\NormalTok{beta\_Y0 }\OtherTok{=} \FloatTok{0.3}
\NormalTok{beta\_L0 }\OtherTok{=} \FloatTok{0.2}
\NormalTok{beta\_U }\OtherTok{=} \FloatTok{0.1}

\CommentTok{\# simulate treatment assignment}
\NormalTok{A\_1 }\OtherTok{\textless{}{-}} \FunctionTok{rbinom}\NormalTok{(n, }\DecValTok{1}\NormalTok{, }\AttributeTok{prob =} \FunctionTok{plogis}\NormalTok{(}\SpecialCharTok{{-}}\FloatTok{0.5} \SpecialCharTok{+} 
\NormalTok{                                    beta\_A0 }\SpecialCharTok{*}\NormalTok{ A\_0 }\SpecialCharTok{+}
\NormalTok{                                    beta\_Y0 }\SpecialCharTok{*}\NormalTok{ Y\_0 }\SpecialCharTok{+} 
\NormalTok{                                    beta\_L0 }\SpecialCharTok{*}\NormalTok{ L\_0 }\SpecialCharTok{+} 
\NormalTok{                                    beta\_U }\SpecialCharTok{*}\NormalTok{ U))}
\CommentTok{\# coefficients for continuous outcome}
\NormalTok{delta\_A1 }\OtherTok{=} \FloatTok{0.3}
\NormalTok{delta\_Y0 }\OtherTok{=} \FloatTok{0.9}
\NormalTok{delta\_A0 }\OtherTok{=} \FloatTok{0.1}
\NormalTok{delta\_L0 }\OtherTok{=} \FloatTok{0.3}
\NormalTok{theta\_A0Y0L0 }\OtherTok{=} \FloatTok{0.5} \CommentTok{\# Interaction effect between A\_1 and L\_0}
\NormalTok{delta\_U }\OtherTok{=} \FloatTok{0.05}
\CommentTok{\# simulate continuous outcome including interaction}
\NormalTok{Y\_2 }\OtherTok{\textless{}{-}} \FunctionTok{rnorm}\NormalTok{(n,}
             \AttributeTok{mean =} \DecValTok{0} \SpecialCharTok{+}
\NormalTok{               delta\_A1 }\SpecialCharTok{*}\NormalTok{ A\_1 }\SpecialCharTok{+} 
\NormalTok{               delta\_Y0 }\SpecialCharTok{*}\NormalTok{ Y\_0 }\SpecialCharTok{+} 
\NormalTok{               delta\_A0 }\SpecialCharTok{*}\NormalTok{ A\_0 }\SpecialCharTok{+} 
\NormalTok{               delta\_L0 }\SpecialCharTok{*}\NormalTok{ L\_0 }\SpecialCharTok{+} 
\NormalTok{               theta\_A0Y0L0 }\SpecialCharTok{*}\NormalTok{ Y\_0 }\SpecialCharTok{*} 
\NormalTok{               A\_0 }\SpecialCharTok{*}\NormalTok{ L\_0 }\SpecialCharTok{+} 
\NormalTok{               delta\_U }\SpecialCharTok{*}\NormalTok{ U,}
             \AttributeTok{sd =}\NormalTok{ .}\DecValTok{5}\NormalTok{)}
\CommentTok{\# assemble data frame}
\NormalTok{data }\OtherTok{\textless{}{-}} \FunctionTok{data.frame}\NormalTok{(Y\_2, A\_0, A\_1, L\_0, Y\_0, U)}

\CommentTok{\# model: no control}
\NormalTok{fit\_no\_control }\OtherTok{\textless{}{-}} \FunctionTok{lm}\NormalTok{(Y\_2 }\SpecialCharTok{\textasciitilde{}}\NormalTok{ A\_1, }\AttributeTok{data =}\NormalTok{ data)}

\CommentTok{\# model: standard covariate control}
\NormalTok{fit\_standard }\OtherTok{\textless{}{-}} \FunctionTok{lm}\NormalTok{(Y\_2 }\SpecialCharTok{\textasciitilde{}}\NormalTok{ A\_1 }\SpecialCharTok{+}\NormalTok{ L\_0, }\AttributeTok{data =}\NormalTok{ data)}

\CommentTok{\# model: interaction with baseline confounders, and baseline outcome and exposure}
\NormalTok{fit\_interaction  }\OtherTok{\textless{}{-}} \FunctionTok{lm}\NormalTok{(Y\_2 }\SpecialCharTok{\textasciitilde{}}\NormalTok{ A\_1 }\SpecialCharTok{*}\NormalTok{ (L\_0 }\SpecialCharTok{+}\NormalTok{ A\_0 }\SpecialCharTok{+}\NormalTok{ Y\_0), }\AttributeTok{data =}\NormalTok{ data)}

\CommentTok{\# create gtsummary tables for each regression model}
\NormalTok{tbl\_fit\_no\_control}\OtherTok{\textless{}{-}} \FunctionTok{tbl\_regression}\NormalTok{(fit\_no\_control)  }
\NormalTok{tbl\_fit\_standard }\OtherTok{\textless{}{-}} \FunctionTok{tbl\_regression}\NormalTok{(fit\_standard)}
\NormalTok{tbl\_fit\_interaction }\OtherTok{\textless{}{-}} \FunctionTok{tbl\_regression}\NormalTok{(fit\_interaction)}

\CommentTok{\# get only the treatment variable}
\NormalTok{tbl\_list\_modified }\OtherTok{\textless{}{-}} \FunctionTok{lapply}\NormalTok{(}\FunctionTok{list}\NormalTok{(}
\NormalTok{  tbl\_fit\_no\_control,}
\NormalTok{  tbl\_fit\_standard,}
\NormalTok{  tbl\_fit\_interaction),}
\ControlFlowTok{function}\NormalTok{(tbl) \{}
\NormalTok{  tbl }\SpecialCharTok{\%\textgreater{}\%}
    \FunctionTok{modify\_table\_body}\NormalTok{(}\SpecialCharTok{\textasciitilde{}}\NormalTok{ .x }\SpecialCharTok{\%\textgreater{}\%}\NormalTok{ dplyr}\SpecialCharTok{::}\FunctionTok{filter}\NormalTok{(variable }\SpecialCharTok{==} \StringTok{"A\_1"}\NormalTok{))}
\NormalTok{\})}
\CommentTok{\# merge tables}
\NormalTok{table\_comparison }\OtherTok{\textless{}{-}} \FunctionTok{tbl\_merge}\NormalTok{(}
  \AttributeTok{tbls =}\NormalTok{ tbl\_list\_modified,}
  \AttributeTok{tab\_spanner =} \FunctionTok{c}\NormalTok{(}
    \StringTok{"No Control"}\NormalTok{,}
    \StringTok{"Standard"}\NormalTok{,}
    \StringTok{"Interaction"}\NormalTok{)}
\NormalTok{) }\SpecialCharTok{|\textgreater{}}
  \FunctionTok{modify\_table\_styling}\NormalTok{(}
    \AttributeTok{column =} \FunctionTok{c}\NormalTok{(p.value\_1, p.value\_2, p.value\_3),}
    \AttributeTok{hide =} \ConstantTok{TRUE}
\NormalTok{  )}
\CommentTok{\# latex table for publication}
\NormalTok{markdown\_table }\OtherTok{\textless{}{-}}
  \FunctionTok{as\_kable\_extra}\NormalTok{(table\_comparison, }\AttributeTok{format =} \StringTok{"latex"}\NormalTok{, }\AttributeTok{booktabs =} \ConstantTok{TRUE}\NormalTok{) }\SpecialCharTok{|\textgreater{}}
  \FunctionTok{kable\_styling}\NormalTok{(}\AttributeTok{latex\_options =} \StringTok{"scale\_down"}\NormalTok{)}
\FunctionTok{print}\NormalTok{(markdown\_table)}
\end{Highlighting}
\end{Shaded}

\begin{verbatim}
\begin{table}
\centering
\resizebox{\ifdim\width>\linewidth\linewidth\else\width\fi}{!}{
\begin{tabular}{lcccccc}
\toprule
\multicolumn{1}{c}{ } & \multicolumn{2}{c}{No Control} & \multicolumn{2}{c}{Standard} & \multicolumn{2}{c}{Interaction} \\
\cmidrule(l{3pt}r{3pt}){2-3} \cmidrule(l{3pt}r{3pt}){4-5} \cmidrule(l{3pt}r{3pt}){6-7}
\textbf{Characteristic} & \textbf{Beta} & \textbf{95\% CI} & \textbf{Beta} & \textbf{95\% CI} & \textbf{Beta} & \textbf{95\% CI}\\
\midrule
A\_1 & 1.5 & 1.5, 1.6 & 0.86 & 0.80, 0.93 & 0.25 & 0.20, 0.31\\
\bottomrule
\multicolumn{7}{l}{\rule{0pt}{1em}\textsuperscript{1} CI = Confidence Interval}\\
\end{tabular}}
\end{table}
\end{verbatim}

Next, in the following code, we calculate the Average Treatment Effect
(ATE) using simulation-based approaches for two distinct models: one
with standard covariate control and another incorporating interaction.
This approach leverages the \texttt{clarify} package in R, which
facilitates the simulation and interpretation of estimated coefficients
from linear models to derive ATEs under different modelling assumptions
(\citeproc{ref-greifer2023}{Greifer \emph{et al.} 2023}).

First, we use the \texttt{sim} function from the \texttt{clarify}
package to generate simulated coefficient distributions for the standard
model (\texttt{fit\_standard}) and the interaction model
(\texttt{fit\_interaction}). This step is crucial for capturing the
uncertainty in our estimates arising from sampling variability.

Next, we employ each model's \texttt{sim\_ame} function to compute the
average marginal effects (AME), focusing on the treatment variable
(\texttt{A\_1}). The calculation is done under the assumption that all
individuals are treated (i.e., \texttt{A\_1\ ==\ 1}), and we specify the
contrast type as ``RD'' (Risk Difference) to directly obtain the ATE
(Average Treatment Effect). The \texttt{sim\_ame} function simulates the
treatment effect across the distribution of simulated coefficients,
providing a robust estimate of the ATE and its variability.

The summaries of these simulations (\texttt{summary\_sim\_est\_fit\_std}
and \texttt{summary\_sim\_est\_fit\_int}) are then extracted to provide
concise estimates of the ATE along with 95\% confidence intervals (CIs)
for both the standard and interaction models. This step is essential for
understanding the magnitude and precision of the treatment effects
estimated by the models.

Finally, we use the \texttt{glue} package to format these estimates into
a human-readable form, presenting the ATE and its corresponding 95\% CIs
for each model. This presentation facilitates clear communication of the
estimated treatment effects, allowing for direct comparison between the
models and highlighting the effect of including baseline characteristics
and their interactions on estimating the ATE
(\citeproc{ref-hester2022GLUE}{Hester and Bryan 2022}).

This simulation-based approach to estimating the ATE underscores the
importance of considering model complexity and the roles of confounders
and mediators in causal inference analyses. By comparing the ATE
estimates from different models, we can assess the sensitivity of our
causal conclusions to various assumptions and modelling strategies.

\begin{Shaded}
\begin{Highlighting}[]
\CommentTok{\# use \textasciigrave{}clarify\textasciigrave{} package to obtain ATE}
\ControlFlowTok{if}\NormalTok{(}\SpecialCharTok{!}\FunctionTok{require}\NormalTok{(clarify))\{}\FunctionTok{install.packages}\NormalTok{(}\StringTok{"clarify"}\NormalTok{)\} }\CommentTok{\# clarify package}

\CommentTok{\# simulate fit 1 ATE}
\FunctionTok{set.seed}\NormalTok{(}\DecValTok{123}\NormalTok{)}
\NormalTok{sim\_coefs\_fit\_no\_control}\OtherTok{\textless{}{-}} \FunctionTok{sim}\NormalTok{(fit\_no\_control)  }
\NormalTok{sim\_coefs\_fit\_std }\OtherTok{\textless{}{-}} \FunctionTok{sim}\NormalTok{(fit\_standard)}
\NormalTok{sim\_coefs\_fit\_int }\OtherTok{\textless{}{-}} \FunctionTok{sim}\NormalTok{(fit\_interaction)}

\CommentTok{\# marginal risk difference ATE, no controls}
\NormalTok{sim\_est\_fit\_no\_control }\OtherTok{\textless{}{-}}
  \FunctionTok{sim\_ame}\NormalTok{(}
\NormalTok{    sim\_coefs\_fit\_no\_control,}
    \AttributeTok{var =} \StringTok{"A\_1"}\NormalTok{,}
    \AttributeTok{subset =}\NormalTok{ A\_1 }\SpecialCharTok{==} \DecValTok{1}\NormalTok{,}
    \AttributeTok{contrast =} \StringTok{"RD"}\NormalTok{,}
    \AttributeTok{verbose =} \ConstantTok{FALSE}
\NormalTok{  )}
\CommentTok{\# marginal risk difference ATE, simulation{-}based: model 1 (L is a confounder)}
\NormalTok{sim\_est\_fit\_std }\OtherTok{\textless{}{-}}
  \FunctionTok{sim\_ame}\NormalTok{(}
\NormalTok{    sim\_coefs\_fit\_std,}
    \AttributeTok{var =} \StringTok{"A\_1"}\NormalTok{,}
    \AttributeTok{subset =}\NormalTok{ A\_1 }\SpecialCharTok{==} \DecValTok{1}\NormalTok{,}
    \AttributeTok{contrast =} \StringTok{"RD"}\NormalTok{,}
    \AttributeTok{verbose =} \ConstantTok{FALSE}
\NormalTok{  )}
\CommentTok{\# marginal risk difference ATE, simulation{-}based: model 2 (L is a mediator)}
\NormalTok{sim\_est\_fit\_int }\OtherTok{\textless{}{-}}
  \FunctionTok{sim\_ame}\NormalTok{(}
\NormalTok{    sim\_coefs\_fit\_int,}
    \AttributeTok{var =} \StringTok{"A\_1"}\NormalTok{,}
    \AttributeTok{subset =}\NormalTok{ A\_1 }\SpecialCharTok{==} \DecValTok{1}\NormalTok{,}
    \AttributeTok{contrast =} \StringTok{"RD"}\NormalTok{,}
    \AttributeTok{verbose =} \ConstantTok{FALSE}
\NormalTok{  )}
\CommentTok{\# obtain summaries}
\NormalTok{summary\_sim\_coefs\_fit\_no\_control }\OtherTok{\textless{}{-}}
  \FunctionTok{summary}\NormalTok{(sim\_est\_fit\_no\_control, }\AttributeTok{null =} \FunctionTok{c}\NormalTok{(}\StringTok{\textasciigrave{}}\AttributeTok{RD}\StringTok{\textasciigrave{}} \OtherTok{=} \DecValTok{0}\NormalTok{))}
\NormalTok{summary\_sim\_est\_fit\_std }\OtherTok{\textless{}{-}}
  \FunctionTok{summary}\NormalTok{(sim\_est\_fit\_std, }\AttributeTok{null =} \FunctionTok{c}\NormalTok{(}\StringTok{\textasciigrave{}}\AttributeTok{RD}\StringTok{\textasciigrave{}} \OtherTok{=} \DecValTok{0}\NormalTok{))}
\NormalTok{summary\_sim\_est\_fit\_int }\OtherTok{\textless{}{-}}
  \FunctionTok{summary}\NormalTok{(sim\_est\_fit\_int, }\AttributeTok{null =} \FunctionTok{c}\NormalTok{(}\StringTok{\textasciigrave{}}\AttributeTok{RD}\StringTok{\textasciigrave{}} \OtherTok{=} \DecValTok{0}\NormalTok{))}

\CommentTok{\# get coefficients for reporting}
\CommentTok{\# ate for fit 1, with 95\% CI}
\NormalTok{ATE\_fit\_no\_control  }\OtherTok{\textless{}{-}}\NormalTok{ glue}\SpecialCharTok{::}\FunctionTok{glue}\NormalTok{(}
  \StringTok{"ATE = \{round(summary\_sim\_coefs\_fit\_no\_control[3, 1], 2)\}, }
\StringTok{  CI = [\{round(summary\_sim\_coefs\_fit\_no\_control[3, 2], 2)\},}
\StringTok{  \{round(summary\_sim\_coefs\_fit\_no\_control[3, 3], 2)\}]"}
\NormalTok{)}
\CommentTok{\# ate for fit 2, with 95\% CI}
\NormalTok{ATE\_fit\_std }\OtherTok{\textless{}{-}}\NormalTok{ glue}\SpecialCharTok{::}\FunctionTok{glue}\NormalTok{(}
  \StringTok{"ATE = \{round(summary\_sim\_est\_fit\_std[3, 1], 2)\}, }
\StringTok{  CI = [\{round(summary\_sim\_est\_fit\_std[3, 2], 2)\},}
\StringTok{  \{round(summary\_sim\_est\_fit\_std[3, 3], 2)\}]"}
\NormalTok{)}
\CommentTok{\# ate for fit 3, with 95\% CI}
\NormalTok{ATE\_fit\_int }\OtherTok{\textless{}{-}}
\NormalTok{  glue}\SpecialCharTok{::}\FunctionTok{glue}\NormalTok{(}
    \StringTok{"ATE = \{round(summary\_sim\_est\_fit\_int[3, 1], 2)\},}
\StringTok{    CI = [\{round(summary\_sim\_est\_fit\_int[3, 2], 2)\},}
\StringTok{    \{round(summary\_sim\_est\_fit\_int[3, 3], 2)\}]"}
\NormalTok{  )}
\CommentTok{\# coefs they used in the manuscript}
\end{Highlighting}
\end{Shaded}

Using the \texttt{clarify} package, we infer the ATE for the standard
model is ATE = 0.86, CI = {[}0.8, 0.92{]}.

Using the \texttt{clarify} package, we infer the ATE for the model that
conditions on the baseline exposure and baseline outcome to be: ATE =
0.43, CI = {[}0.39, 0.47{]}, which is close to the values supplied to
the data-generating mechanism.

\textbf{Take-home message:}

The baseline exposure and baseline outcome are often the most important
variables to include for confounding control. The baseline exposure also
allows us to estimate an incident-exposure effect. For this reason, we
should endeavour to obtain at least three waves of data such that these
variables and other baseline confounders are included at time 0, the
exposure is included at time 1, and the outcome is included at time 2.

\newpage{}

\subsection{Appendix E: Non-parametric Estimation of Average Treatment
Effects Using Causal Forests}\label{appendix-causal-forests}

This appendix provides a practical example of estimating average
treatment effects (ATE) using a non-parametric approach, specifically
applying causal forests. Unlike traditional regression models, causal
forests allow for estimating treatment effects without imposing strict
assumptions about the form of the relationship between treatment,
covariates, and outcomes. This flexibility makes them particularly
useful for analysing complex datasets where the treatment effect may
vary across observations.

\paragraph{Causal Forest Model
Implementation}\label{causal-forest-model-implementation}

\begin{enumerate}
\def\labelenumi{\arabic{enumi}.}
\item
  \textbf{Libraries}: the implementation begins with loading the
  necessary R libraries: \texttt{grf} for estimating conditional and
  average treatment effects using causal forests and \texttt{glue} for
  formatting the results for reporting.
\item
  \begin{enumerate}
  \def\labelenumii{\arabic{enumii}.}
  \tightlist
  \item
    \textbf{Data generation}: the code assumes the presence of a data
    frame \texttt{data} generated from the previous code snippet
    containing the variables:
  \end{enumerate}

  \begin{itemize}
  \tightlist
  \item
    \texttt{A\_1}: Treatment indicator.
  \item
    \texttt{L\_0}: A covariate.
  \item
    \texttt{Y\_2}: Outcome of interest.
  \item
    \texttt{A\_0} and \texttt{Y\_0}: Baseline exposure and outcome,
    respectively.
  \end{itemize}

  Treatment (\texttt{W}) and outcome (\texttt{Y}) vectors are extracted
  from \texttt{data} alongside a matrix \texttt{X} that includes
  covariates and baseline characteristics.
\item
  \textbf{Causal Forest model}: a causal forest model is fitted using
  the \texttt{causal\_forest} function from the \texttt{grf} package
  (\citeproc{ref-grf2024}{Tibshirani \emph{et al.} 2024}). This function
  takes the covariate matrix \texttt{X}, the outcome vector \texttt{Y},
  and the treatment vector \texttt{W} as inputs, and it returns a model
  object that can be used for further analysis.
\item
  \textbf{Average Treatment Effect estimation}: the
  \texttt{average\_treatment\_effect} function computes the ATE from the
  fitted causal forest model. This step is crucial as it quantifies the
  overall effect of the treatment across the population, adjusting for
  covariates included in the model.
\item
  \textbf{Reporting}: The estimated ATE and its standard error (se) are
  extracted and formatted for reporting using the \texttt{glue} package
  (\citeproc{ref-hester2022GLUE}{Hester and Bryan 2022}). This
  facilitates clear communication of the results, showing the estimated
  effect size and its uncertainty.
\end{enumerate}

\paragraph{Key Takeaways}\label{key-takeaways}

First, causal forests offer a robust way to estimate treatment effects
without making parametric solid assumptions. This approach is
particularly advantageous in settings where the treatment effect may
vary with covariates or across different subpopulations.

Second, the model estimates the ATE as the difference in expected
outcomes between treated and untreated units, averaged across the
population. This estimate reflects the overall effect of the treatment,
accounting for the distribution of covariates in the sample.

Third, we find that the estimated ATE by the causal forest model
converges to the actual value used in the data-generating process
(assumed to be 0.3). This demonstrates the effectiveness of causal
forests in uncovering the true treatment effect from complex data.

This example underscores the utility of semi-parametric and
non-parametric methods, such as causal forests, in causal inference
analyses.

\begin{Shaded}
\begin{Highlighting}[]
\CommentTok{\# load causal forest library }
\FunctionTok{library}\NormalTok{(grf) }\CommentTok{\# estimate conditional and average treatment effects}
\FunctionTok{library}\NormalTok{(glue) }\CommentTok{\# reporting }

\CommentTok{\#  \textquotesingle{}data\textquotesingle{} is our data frame with columns \textquotesingle{}A\_1\textquotesingle{} for treatment, \textquotesingle{}L\_0\textquotesingle{} for a covariate, and \textquotesingle{}Y\_2\textquotesingle{} for the outcome}
\CommentTok{\#  we also have the baseline exposure \textquotesingle{}A\_0\textquotesingle{} and \textquotesingle{}Y\_0\textquotesingle{}}
\CommentTok{\#  ensure W (treatment) and Y (outcome) are vectors}
\NormalTok{W }\OtherTok{\textless{}{-}} \FunctionTok{as.matrix}\NormalTok{(data}\SpecialCharTok{$}\NormalTok{A\_1)  }\CommentTok{\# Treatment}
\NormalTok{Y }\OtherTok{\textless{}{-}} \FunctionTok{as.matrix}\NormalTok{(data}\SpecialCharTok{$}\NormalTok{Y\_2)  }\CommentTok{\# Outcome}
\NormalTok{X }\OtherTok{\textless{}{-}} \FunctionTok{as.matrix}\NormalTok{(data[, }\FunctionTok{c}\NormalTok{(}\StringTok{"L\_0"}\NormalTok{, }\StringTok{"A\_0"}\NormalTok{, }\StringTok{"Y\_0"}\NormalTok{)])}

\CommentTok{\# fit causal forest model }
\NormalTok{fit\_causal\_forest }\OtherTok{\textless{}{-}} \FunctionTok{causal\_forest}\NormalTok{(X, Y, W)}

\CommentTok{\# estimate the average treatment effect (ATE)}
\NormalTok{ate }\OtherTok{\textless{}{-}} \FunctionTok{average\_treatment\_effect}\NormalTok{(fit\_causal\_forest)}

\CommentTok{\# make data frame for reporting using "glue\textquotesingle{} }
\NormalTok{ate}\OtherTok{\textless{}{-}} \FunctionTok{data.frame}\NormalTok{(ate)}

\CommentTok{\# obtain ate for report}
\NormalTok{ATE\_fit\_causal\_forest }\OtherTok{\textless{}{-}}
\NormalTok{  glue}\SpecialCharTok{::}\FunctionTok{glue}\NormalTok{(}
    \StringTok{"ATE = \{round(ate[1, 1], 2)\}, se = \{round(ate[2, 1], 2)\}"}
\NormalTok{  )}
\end{Highlighting}
\end{Shaded}

Causal forest estimates the average treatment effect as ATE = 0.3, se =
0.01. This approach converges to the true value supplied to the
generating mechanism of 0.3

\phantomsection\label{refs}
\begin{CSLReferences}{1}{0}
\bibitem[\citeproctext]{ref-cinelli2022}
Cinelli, C, Forney, A, and Pearl, J (2022) A Crash Course in Good and
Bad Controls. \emph{Sociological Methods \&Research}, 00491241221099552.
doi:\href{https://doi.org/10.1177/00491241221099552}{10.1177/00491241221099552}.

\bibitem[\citeproctext]{ref-greifer2023}
Greifer, N, Worthington, S, Iacus, S, and King, G (2023) \emph{Clarify:
Simulation-based inference for regression models}. Retrieved from
\url{https://iqss.github.io/clarify/}

\bibitem[\citeproctext]{ref-hernan2023}
Hernan, MA, and Robins, JM (2023) \emph{Causal inference}, Taylor \&
Francis. Retrieved from
\url{https://books.google.co.nz/books?id=/_KnHIAAACAAJ}

\bibitem[\citeproctext]{ref-hester2022GLUE}
Hester, J, and Bryan, J (2022) \emph{Glue: Interpreted string literals}.
Retrieved from \url{https://CRAN.R-project.org/package=glue}

\bibitem[\citeproctext]{ref-lash2020}
Lash, TL, Rothman, KJ, VanderWeele, TJ, and Haneuse, S (2020)
\emph{Modern epidemiology}, Wolters Kluwer. Retrieved from
\url{https://books.google.co.nz/books?id=SiTSnQEACAAJ}

\bibitem[\citeproctext]{ref-pearl1995}
Pearl, J (1995) Causal diagrams for empirical research.
\emph{Biometrika}, \textbf{82}(4), 669--688.

\bibitem[\citeproctext]{ref-pearl2009}
Pearl, J (2009) \emph{\href{https://doi.org/10.1214/09-SS057}{Causal
inference in statistics: An overview}}.

\bibitem[\citeproctext]{ref-pearl2018}
Pearl, J, and Mackenzie, D (2018) \emph{The book of why: The new science
of cause and effect}, Basic books.

\bibitem[\citeproctext]{ref-gtsummary2021}
Sjoberg, DD, Whiting, K, Curry, M, Lavery, JA, and Larmarange, J (2021)
Reproducible summary tables with the gtsummary package. \emph{{The R
Journal}}, \textbf{13}, 570--580.
doi:\href{https://doi.org/10.32614/RJ-2021-053}{10.32614/RJ-2021-053}.

\bibitem[\citeproctext]{ref-grf2024}
Tibshirani, J, Athey, S, Sverdrup, E, and Wager, S (2024) \emph{Grf:
Generalized random forests}. Retrieved from
\url{https://github.com/grf-labs/grf}

\bibitem[\citeproctext]{ref-vanderweele2009}
VanderWeele, TJ (2009) Concerning the consistency assumption in causal
inference. \emph{Epidemiology}, \textbf{20}(6), 880.
doi:\href{https://doi.org/10.1097/EDE.0b013e3181bd5638}{10.1097/EDE.0b013e3181bd5638}.

\bibitem[\citeproctext]{ref-vanderweele2018}
VanderWeele, TJ (2018) On well-defined hypothetical interventions in the
potential outcomes framework. \emph{Epidemiology}, \textbf{29}(4), e24.
doi:\href{https://doi.org/10.1097/EDE.0000000000000823}{10.1097/EDE.0000000000000823}.

\bibitem[\citeproctext]{ref-vanderweele2019}
VanderWeele, TJ (2019) Principles of confounder selection.
\emph{European Journal of Epidemiology}, \textbf{34}(3), 211--219.

\bibitem[\citeproctext]{ref-vanderweele2013}
VanderWeele, TJ, and Hernan, MA (2013) Causal inference under multiple
versions of treatment. \emph{Journal of Causal Inference},
\textbf{1}(1), 1--20.

\bibitem[\citeproctext]{ref-zhu2021KableExtra}
Zhu, H (2021) \emph{kableExtra: Construct complex table with 'kable' and
pipe syntax}. Retrieved from
\url{https://CRAN.R-project.org/package=kableExtra}

\end{CSLReferences}



\end{document}
