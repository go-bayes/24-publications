% Options for packages loaded elsewhere
\PassOptionsToPackage{unicode}{hyperref}
\PassOptionsToPackage{hyphens}{url}
\PassOptionsToPackage{dvipsnames,svgnames,x11names}{xcolor}
%
\documentclass[
  single column]{article}

\usepackage{amsmath,amssymb}
\usepackage{iftex}
\ifPDFTeX
  \usepackage[T1]{fontenc}
  \usepackage[utf8]{inputenc}
  \usepackage{textcomp} % provide euro and other symbols
\else % if luatex or xetex
  \usepackage{unicode-math}
  \defaultfontfeatures{Scale=MatchLowercase}
  \defaultfontfeatures[\rmfamily]{Ligatures=TeX,Scale=1}
\fi
\usepackage[]{libertinus}
\ifPDFTeX\else  
    % xetex/luatex font selection
\fi
% Use upquote if available, for straight quotes in verbatim environments
\IfFileExists{upquote.sty}{\usepackage{upquote}}{}
\IfFileExists{microtype.sty}{% use microtype if available
  \usepackage[]{microtype}
  \UseMicrotypeSet[protrusion]{basicmath} % disable protrusion for tt fonts
}{}
\makeatletter
\@ifundefined{KOMAClassName}{% if non-KOMA class
  \IfFileExists{parskip.sty}{%
    \usepackage{parskip}
  }{% else
    \setlength{\parindent}{0pt}
    \setlength{\parskip}{6pt plus 2pt minus 1pt}}
}{% if KOMA class
  \KOMAoptions{parskip=half}}
\makeatother
\usepackage{xcolor}
\usepackage[top=30mm,left=25mm,heightrounded,headsep=22pt,headheight=11pt,footskip=33pt,ignorehead,ignorefoot]{geometry}
\setlength{\emergencystretch}{3em} % prevent overfull lines
\setcounter{secnumdepth}{-\maxdimen} % remove section numbering
% Make \paragraph and \subparagraph free-standing
\makeatletter
\ifx\paragraph\undefined\else
  \let\oldparagraph\paragraph
  \renewcommand{\paragraph}{
    \@ifstar
      \xxxParagraphStar
      \xxxParagraphNoStar
  }
  \newcommand{\xxxParagraphStar}[1]{\oldparagraph*{#1}\mbox{}}
  \newcommand{\xxxParagraphNoStar}[1]{\oldparagraph{#1}\mbox{}}
\fi
\ifx\subparagraph\undefined\else
  \let\oldsubparagraph\subparagraph
  \renewcommand{\subparagraph}{
    \@ifstar
      \xxxSubParagraphStar
      \xxxSubParagraphNoStar
  }
  \newcommand{\xxxSubParagraphStar}[1]{\oldsubparagraph*{#1}\mbox{}}
  \newcommand{\xxxSubParagraphNoStar}[1]{\oldsubparagraph{#1}\mbox{}}
\fi
\makeatother


\providecommand{\tightlist}{%
  \setlength{\itemsep}{0pt}\setlength{\parskip}{0pt}}\usepackage{longtable,booktabs,array}
\usepackage{calc} % for calculating minipage widths
% Correct order of tables after \paragraph or \subparagraph
\usepackage{etoolbox}
\makeatletter
\patchcmd\longtable{\par}{\if@noskipsec\mbox{}\fi\par}{}{}
\makeatother
% Allow footnotes in longtable head/foot
\IfFileExists{footnotehyper.sty}{\usepackage{footnotehyper}}{\usepackage{footnote}}
\makesavenoteenv{longtable}
\usepackage{graphicx}
\makeatletter
\newsavebox\pandoc@box
\newcommand*\pandocbounded[1]{% scales image to fit in text height/width
  \sbox\pandoc@box{#1}%
  \Gscale@div\@tempa{\textheight}{\dimexpr\ht\pandoc@box+\dp\pandoc@box\relax}%
  \Gscale@div\@tempb{\linewidth}{\wd\pandoc@box}%
  \ifdim\@tempb\p@<\@tempa\p@\let\@tempa\@tempb\fi% select the smaller of both
  \ifdim\@tempa\p@<\p@\scalebox{\@tempa}{\usebox\pandoc@box}%
  \else\usebox{\pandoc@box}%
  \fi%
}
% Set default figure placement to htbp
\def\fps@figure{htbp}
\makeatother
% definitions for citeproc citations
\NewDocumentCommand\citeproctext{}{}
\NewDocumentCommand\citeproc{mm}{%
  \begingroup\def\citeproctext{#2}\cite{#1}\endgroup}
\makeatletter
 % allow citations to break across lines
 \let\@cite@ofmt\@firstofone
 % avoid brackets around text for \cite:
 \def\@biblabel#1{}
 \def\@cite#1#2{{#1\if@tempswa , #2\fi}}
\makeatother
\newlength{\cslhangindent}
\setlength{\cslhangindent}{1.5em}
\newlength{\csllabelwidth}
\setlength{\csllabelwidth}{3em}
\newenvironment{CSLReferences}[2] % #1 hanging-indent, #2 entry-spacing
 {\begin{list}{}{%
  \setlength{\itemindent}{0pt}
  \setlength{\leftmargin}{0pt}
  \setlength{\parsep}{0pt}
  % turn on hanging indent if param 1 is 1
  \ifodd #1
   \setlength{\leftmargin}{\cslhangindent}
   \setlength{\itemindent}{-1\cslhangindent}
  \fi
  % set entry spacing
  \setlength{\itemsep}{#2\baselineskip}}}
 {\end{list}}
\usepackage{calc}
\newcommand{\CSLBlock}[1]{\hfill\break\parbox[t]{\linewidth}{\strut\ignorespaces#1\strut}}
\newcommand{\CSLLeftMargin}[1]{\parbox[t]{\csllabelwidth}{\strut#1\strut}}
\newcommand{\CSLRightInline}[1]{\parbox[t]{\linewidth - \csllabelwidth}{\strut#1\strut}}
\newcommand{\CSLIndent}[1]{\hspace{\cslhangindent}#1}

\usepackage{booktabs}
\usepackage{longtable}
\usepackage{array}
\usepackage{multirow}
\usepackage{wrapfig}
\usepackage{float}
\usepackage{colortbl}
\usepackage{pdflscape}
\usepackage{tabu}
\usepackage{threeparttable}
\usepackage{threeparttablex}
\usepackage[normalem]{ulem}
\usepackage{makecell}
\usepackage{xcolor}
\usepackage{tabularray}
\usepackage[normalem]{ulem}
\usepackage{graphicx}
\UseTblrLibrary{booktabs}
\UseTblrLibrary{rotating}
\UseTblrLibrary{siunitx}
\NewTableCommand{\tinytableDefineColor}[3]{\definecolor{#1}{#2}{#3}}
\newcommand{\tinytableTabularrayUnderline}[1]{\underline{#1}}
\newcommand{\tinytableTabularrayStrikeout}[1]{\sout{#1}}
\input{/Users/joseph/GIT/latex/latex-for-quarto.tex}
\makeatletter
\@ifpackageloaded{caption}{}{\usepackage{caption}}
\AtBeginDocument{%
\ifdefined\contentsname
  \renewcommand*\contentsname{Table of contents}
\else
  \newcommand\contentsname{Table of contents}
\fi
\ifdefined\listfigurename
  \renewcommand*\listfigurename{List of Figures}
\else
  \newcommand\listfigurename{List of Figures}
\fi
\ifdefined\listtablename
  \renewcommand*\listtablename{List of Tables}
\else
  \newcommand\listtablename{List of Tables}
\fi
\ifdefined\figurename
  \renewcommand*\figurename{Figure}
\else
  \newcommand\figurename{Figure}
\fi
\ifdefined\tablename
  \renewcommand*\tablename{Table}
\else
  \newcommand\tablename{Table}
\fi
}
\@ifpackageloaded{float}{}{\usepackage{float}}
\floatstyle{ruled}
\@ifundefined{c@chapter}{\newfloat{codelisting}{h}{lop}}{\newfloat{codelisting}{h}{lop}[chapter]}
\floatname{codelisting}{Listing}
\newcommand*\listoflistings{\listof{codelisting}{List of Listings}}
\makeatother
\makeatletter
\makeatother
\makeatletter
\@ifpackageloaded{caption}{}{\usepackage{caption}}
\@ifpackageloaded{subcaption}{}{\usepackage{subcaption}}
\makeatother

\usepackage{bookmark}

\IfFileExists{xurl.sty}{\usepackage{xurl}}{} % add URL line breaks if available
\urlstyle{same} % disable monospaced font for URLs
\hypersetup{
  pdftitle={Evidence for Declining Trust in Science From A Large National Panel Study in New Zealand (years 2019-2023)},
  pdfauthor={Authors (and author order TBA)},
  colorlinks=true,
  linkcolor={blue},
  filecolor={Maroon},
  citecolor={Blue},
  urlcolor={Blue},
  pdfcreator={LaTeX via pandoc}}


\title{Evidence for Declining Trust in Science From A Large National
Panel Study in New Zealand (years 2019-2023)}

\usepackage{academicons}
\usepackage{xcolor}

  \author{Authors (and author order TBA)}
            \affil{%
             \small{     New Zealand
          ORCID \textcolor[HTML]{A6CE39}{\aiOrcid} ~0000-0003-3169-6576 }
              }
      


\date{2024-11-10}
\begin{document}
\maketitle
\begin{abstract}
The public perceptions of science has wide-ranging effects, from the
adoption public health behaviours to climate action. Here, drawing on a
nationally diverse panel study in New Zealand, we report attitudes to
trust in the institution of science and trust in scientists from October
2019- October 2023 in a large nationally diverse cohort of (N = 42,681,
New Zealand Attitudes and Values Study), using multiple imputation to
address systematic bias from attrition. Study 1 focuses on average
responses, revealing stability in rolling average responses over time.
Study 2 focuses on the proportional change in predicted probabilities
across the low, medium, and high ends of the response scale Since 2019,
the proportion of individuals in the low trust category has been slowly
increasing. After the COVID pandemic, trust in the medium category
declined, while trust in the high category rose. The European majority
remains the most trusting of science, but the post-pandemic increase in
trust is waning. Māori have lower trust in science, with a smaller
post-pandemic boost that is now diminishing. Māori and Pacific peoples
show the lowest levels of trust, with this mistrust growing since the
pandemic. Political conservatives are also experiencing an increase in
mistrust of science. Initially, men were more trusting of science, but
declining trust among men, alongside rising high trust in other genders,
has equalised trust levels between the sexes. Trust among political
conservatives has shown a steady increase, though causality is unclear,
as greater trust in science may correlate with a shift towards
conservatism. Overall, these patterns indicate that, despite stable
averages, there is preliminary evidence of rising mistrust in scientific
institutions and scientists. \textbf{KEYWORDS}: \emph{Conservativism};
\emph{Institutional Trust}; \emph{Longitudinal}; \emph{Panel};
\emph{Political}; \emph{Science}.
\end{abstract}


\subsection{Introduction}\label{introduction}

Whether people are growing more sceptical of science is question of
considerable interest and concern.

To address this question, we leverage for waves of comprehensive panel
data from 42,681 participants in the New Zealand Attitudes and Values
Study, spanning the years 2019-2023.

Study 1 reports averages response among the 2019 cohort during this
period, and also stratifies responses by ethnicity, gender (binary) and
political orientation.

Study 2 investigates dynamics across the response scale, considering
dynamics at the low, medium, and high end the trust in science and trust
in scientist scales.

The focus of this study is descriptive and exploratory.

\subsection{Method}\label{method}

\subsection{Method}\label{method-1}

\subsubsection{Sample}\label{sample}

\subsubsection{Target Population}\label{target-population}

The target population for this study comprises New Zealand residents as
represented by the New Zealand Attitudes and Values Study (NZAVS) during
2019 (the baseline wave for this study) weighted by New Zealand Census
weights for age, gender, and ethnicity (refer to Sibley
(\citeproc{ref-sibley2021}{2021})). The NZAVS is a national probability
study designed to accurately reflect the broader New Zealand population.
Despite its comprehensive scope, the NZAVS has some limitations in its
demographic representation. Notably, it tends to under-sample males and
individuals of Asian descent while over-sampling females and Māori (the
indigenous peoples of New Zealand). To address these disparities and
enhance the accuracy of our findings, we apply 2018 New Zealand Census
survey weights to the sample data. These weights adjust for variations
in age, gender, and ethnicity to better approximate the national
demographic composition (\citeproc{ref-sibley2021}{Sibley 2021}).

\subsubsection{Cohort}\label{cohort}

To be included in the analysis of this study, participants needed to
participate in New Zealand Attitudes and Values Study Time 11, years
2019-2020. Participants may have been lost to follow-up at the end of
the study if they met eligibility criteria. Missing covariate data this
baseline wave and all follow up waves through Zealand Attitudes and
Values Study Time 14, years 2022-2023. The proportion of missing data
for each variable by wave is describe in \textbf{?@tbl-summary-vars}.

A total of 42,681 individuals met these criteria and were included in
the study, and their responses were tracked over time.
\hyperref[appendix-a]{Appendix A} \textbf{?@tbl-baseline} presents
sample demographic data.

\subsection{Measures}\label{measures}

We estimated target population average responses for two indicators of
trust in science, which for simplicity we call Trust in Science and
Trust in Scientists.

\paragraph{Trust Science}\label{trust-science}

\emph{Our society places too much emphasis on science (reversed).}

Ordinal response: (1 = Strongly Disagree, 7 = Strongly Agree)
(\citeproc{ref-hartman2017}{Hartman \emph{et al.} 2017}).

\paragraph{Trust Scientists}\label{trust-scientists}

\emph{I have a high degree of confidence in the scientific community.}

Ordinal response: (1 = Strongly Disagree, 7 = Strongly Agree)
(\citeproc{ref-nisbet2015}{Nisbet \emph{et al.} 2015}).

We note note that the term ``Trust in Science'' is an over-simpification
of the measure, we focuses in the emphasis placed on science within
society. It is plausible that at least some who disagree with societies
emphasis on science are nevertheless trusting of its institutions, even
as they think society would be better off emphasising the religion, the
arts, family, or other domains. Again we use ``Trust in Science'' as a
shorthand.

\subsubsection{Study Design}\label{study-design}

In Study 1, we report on population-level changes in average trust in
science and scientists from late 2019 to late 2022. Our results clarify
predicted means for the New Zealand population from late 2019 and over
the following years through late 2023.

In Study 2, we focus on changes across different points of the response
scale, examining shifts at the low, medium, and high ends. Responses
rated 3 or below were categorised as ``low,'' those rated 4 or 5 were
categorised as ``medium,'' and responses of 6 or 7 were categorised as
``high.'' Figure~\ref{fig-hist-outcomes} displays a histogram of the
response distribution in the observed sample, prior to adjustments for
attrition.

\newpage{}

\begin{figure}

\centering{

\pandocbounded{\includegraphics[keepaspectratio]{test2_files/figure-pdf/fig-hist-outcomes-1.pdf}}

}

\caption{\label{fig-hist-outcomes}}

\end{figure}%

\newpage{}

In both Study 1 and Study 2, we also compare average responses based on
(1) ethnicity, (2) gender (a binary indicator), and (3) political
conservatism.

Responses were recorded to the following questions:

\paragraph{Ethnicity (Categorical)}\label{ethnicity-categorical}

\emph{Which ethnic group(s) do you belong to?}

Responses were coded as follows, using New Zealand Census standards: (1
= New Zealand European, 2 = Māori, 3 = Pacific, 4 = Asian).

\paragraph{Gender (Binary Indicator)}\label{gender-binary-indicator}

What is your gender?

Gender was assessed through an open-ended question: ``What is your
gender?.'' Female was coded as 0, Male as 1, and gender diverse as 3 (or
0.5 for responses indicating neither female nor male) following the
coding system described by Fraser et al.~(2020). For the purpose of this
analysis, we coded all participants who identified as Male as 1 and all
others as 0 (\citeproc{ref-fraser_coding_2020}{Fraser \emph{et al.}
2020}).

\paragraph{Political Conservatism}\label{political-conservatism}

Please rate how politically liberal versus conservative you see yourself
as being.

Responses were recorded on an ordinal scale from 1 (Extremely Liberal)
to 7 (Extremely Conservative) (\citeproc{ref-jost_end_2006-1}{Jost
2006}).

\paragraph{Trust Scientists: Observed Sample
Means}\label{trust-scientists-observed-sample-means}

\begin{table}

\caption{\label{tbl-sample-means}Sample average response}

\centering{

\centering
\begin{tblr}[         %% tabularray outer open
]                     %% tabularray outer close
{                     %% tabularray inner open
colspec={Q[]Q[]Q[]Q[]},
cell{2}{1}={r=4,}{valign=h,cmd=\bfseries,},
cell{6}{1}={r=4,}{valign=h,cmd=\bfseries,},
column{1}={halign=l,},
column{2}={halign=l,},
column{3}={halign=l,},
column{4}={halign=l,},
}                     %% tabularray inner close
\toprule
Response & Year & mean & missing \\ \midrule %% TinyTableHeader
Trust Science & 2019 & 5.56 &    563 \\
Trust Science & 2020 & 5.8 &  9,632 \\
Trust Science & 2021 & 5.84 & 15,111 \\
Trust Science & 2022 & 5.84 & 18,221 \\
Trust Scientists & 2019 & 5.3 &  1,257 \\
Trust Scientists & 2020 & 5.55 & 10,172 \\
Trust Scientists & 2021 & 5.56 & 15,519 \\
Trust Scientists & 2022 & 5.55 & 18,948 \\
\bottomrule
\end{tblr}

}

\end{table}%

Figure~\ref{fig-alluv-science-sample} and Table~\ref{tbl-sample-means}
presents sample average responses for trust in science over time.
Between 2019 and 2022, the average trust levels in both science and
scientists suggests a gradient of increase. For Trust in Science, the
average score rose from 5.56 in 2019 to 5.84 by 2021, maintaining the
same level through 2022. This indicates a steady improvement in trust
over the period in the retained sample.

For Trust in Scientists, the average response increased from 5.30 in
2019 to 5.55 in 2020 and stayed stable through 2021 and 2022.

However, number of respondents with unknown trust status increased
substantially over the years for both trust measures. For Trust in
Science, the count rose from 563 in 2019 to 18,221 in 2022. Similarly,
for Trust in Scientists, the unknown category grew from 1,257 in 2019 to
18,948 in 2022.

Hence, the sample data reveals both a positive trend in average trust
levels alongside a notable rise in the number of unknown responses over
the period. Of course, attrition is to be expected in a panel study.
However, we must be cautious when naively interpreting these data
because mistrust in science may be -- and credibly is -- systematically
related to panel attrition and non-response.

\paragraph{Trust Scientists: Observed Categorical
Responses}\label{trust-scientists-observed-categorical-responses}

\begin{figure}

\centering{

\pandocbounded{\includegraphics[keepaspectratio]{test2_files/figure-pdf/fig-alluv-science-sample-1.pdf}}

}

\caption{\label{fig-alluv-science-sample}Historgram of sample trust
responses over time.}

\end{figure}%

\begin{figure}

\centering{

\pandocbounded{\includegraphics[keepaspectratio]{test2_files/figure-pdf/fig-alluv-st-sample-1.pdf}}

}

\caption{\label{fig-alluv-st-sample}Historgram of sample trust responses
over time.}

\end{figure}%

\begin{table}

\caption{\label{tbl-sample-cat}Sample proportions classified as low,
medium, or high}

\centering{

\centering
\begin{tblr}[         %% tabularray outer open
]                     %% tabularray outer close
{                     %% tabularray inner open
colspec={Q[]Q[]Q[]Q[]Q[]Q[]},
cell{2}{1}={r=3,}{valign=h,cmd=\bfseries,},
cell{5}{1}={r=3,}{valign=h,cmd=\bfseries,},
column{1}={halign=l,},
column{2}={halign=l,},
column{3}={halign=l,},
column{4}={halign=l,},
column{5}={halign=l,},
column{6}={halign=l,},
}                     %% tabularray inner close
\toprule
Response & Level & 2019 & 2020 & 2021 & 2022 \\ \midrule %% TinyTableHeader
Trust Science & low & 3434 (8.0%) & 2465 (5.8%) & 2012 (4.7%) & 1740 (4.1%) \\
Trust Science & med & 13210 (31.0%) & 7855 (18.4%) & 6223 (14.6%) & 5720 (13.4%) \\
Trust Science & high & 26037 (61.0%) & 32361 (75.8%) & 34446 (80.7%) & 35221 (82.5%) \\
Trust Scientists & low & 4797 (11.2%) & 2818 (6.6%) & 2477 (5.8%) & 1910 (4.5%) \\
Trust Scientists & med & 14161 (33.2%) & 9385 (22.0%) & 7448 (17.5%) & 7107 (16.7%) \\
Trust Scientists & high & 23723 (55.6%) & 30478 (71.4%) & 32756 (76.7%) & 33664 (78.9%) \\
\bottomrule
\end{tblr}

}

\end{table}%

Table~\ref{tbl-sample-cat} summaries responses by low (1-3), medium
(4,5) and high (6,7) levels of trust in science and in scientists. Over
the four-year period from 2019 to 2022, there were notable changes in
the levels of trust in science and scientists among respondents.

For trust in science, the proportion of respondents reporting low trust
declined steadily from 8.0\% (3434 respondents) in 2019 to 4.1\% (1740
respondents) in 2022. A similar decreasing trend is evident for the
medium trust category, which dropped from 31.0\% (13210 respondents) in
2019 to 13.4\% (5720 respondents) in 2022. By contrast, the proportion
of respondents reporting high trust in science increased from 61.0\%
(26037 respondents) in 2019 to 82.5\% (35221 respondents) in 2022. The
largest increase in high trust occurred between 2019 and 2020, after
which a consistent upward trend continued.
Figure~\ref{fig-alluv-science-sample} is an alluvial graph that shows
the predominately upward shift in trust in science.

For trust in scientists, there was a comparable pattern. Low trust
decreased from 11.2\% (4797 respondents) in 2019 to 4.5\% (1910
respondents) in 2022. Medium trust also declined, from 33.2\% (14161
respondents) in 2019 to 16.7\% (7107 respondents) in 2022, with the
steepest reduction occurring between 2019 and 2020. High trust in
scientists rose significantly from 55.6\% (23723 respondents) in 2019 to
78.9\% (33664 respondents) in 2022, showing substantial growth,
particularly between 2019 and 2020, followed by smaller but continuous
increases. Figure~\ref{fig-alluv-st-sample} is an alluvial graph that
shows the predominately upward shift for trust in scientists.

Overall, these findings suggest a marked shift towards higher levels of
trust in both science and scientists over time, characterised by
declines in low and medium trust and corresponding increases in high
trust, with the most pronounced changes were observed between 2019 and
2020. However, observed responses are unlikely to be the same in the
cohort that did not respond, particularly if -- as seems credible -- low
trust in science or scientists leads to attrition. That is, although the
data reveal a positive trend in average trust levels alongside, the
number of unknown responses for a cohort also increases over this same
period. If we suppose, credibly, that those who become mistrusting of
science are more likely to drop out of a scientific study the picture of
growth in trust may be overly positive.

\subsubsection{Missing Data}\label{missing-data}

Our objective is to estimate the population levels of trust in science
and scientists in the general New Zealand population from late 2019 to
late 2023 using data from the New Zealand Attitudes and Values Study
(NZAVS). We face two primary data challenges when estimating
population-level trends.

First, although the NZAVS is a national probability study that recruits
participants using randomised mailouts from the New Zealand Electoral
Roll, only approximately 10\% of those invited choose to participate. It
is plausible that the proportion of science sceptics is higher among
those who declined participation, as the NZAVS is conducted by
scientists with the aim of fostering scientific understanding.

Second, although the NZAVS maintains an annual retention rate between
70--80\% of its sample, attrition over time is inevitable in any
longitudinal panel. It is credible that participants who develop
scepticism towards science or scientists may be more likely to drop out,
leading to an overestimation of trust levels in the remaining sample.

Although we cannot directly address the first challenge -- we cannot
accurately estimate the density of those who mistrust science among
those who never participated in our scientific study -- the second
challenge, inferring mistrust among those participated in the New
Zealand Attitudes and Values Study in 2019 and subsequently abandoned
the study, can potentially be mitigated. If the probability of missing
responses, both within a wave and over time, can be assumed to be
conditionally independent given observed covariates, we can apply
multiple imputation methods to adjust for bias in attrition. This
approach allows us to systematically incorporate the uncertainty arising
from missing data into our estimates
(\citeproc{ref-blackwell_2017_unified}{Blackwell \emph{et al.} 2017};
\citeproc{ref-bulbulia2023a}{Bulbulia \emph{et al.} 2023}).

Here, we use the Amelia package in R (\citeproc{ref-amelia_2011}{Honaker
\emph{et al.} 2011}) to create ten multiply imputed datasets for the
2019 New Zealand Attitudes and Values Study cohort for waves 11-14
(years 2019-2022). The \texttt{Amelia} package is purpose-built for
within-unit imputation in repeated-measures time-series data. All
covariates in \textbf{?@tbl-summary-vars} were included in the
imputation model, and the time variable (``year'') was modelled as a
cubic spline to account for non-linear trends over the four-year period.

\subsubsection{Statistical Estimator}\label{statistical-estimator}

In Study 1, we examined mean responses for trust in science and trust in
scientists over time using generalised estimating equations (GEE), using
the \texttt{geepack} package in R (\citeproc{ref-geepack_2006}{Halekoh
\emph{et al.} 2006}). This method provides robust standard errors,
ensuring valid statistical inference even when the data exhibit
intra-individual clustering, with fewer assumptions than are required
for multi-level models (citation). We specified GEE models using
participant ID as the clustering variable to adjust for repeated
observations across survey waves. To obtain inferences for the New
Zealand population, we incorporated sample weights derived from the 2018
New Zealand Census data. We modelled responses separately for the ten
imputed datasets and then pooled uncertainty over these estimates using
Rubin's rule (employed using \texttt{ggeffect:pool\_predictions()}
(\citeproc{ref-ggeffects_2018}{Ludecke 2018})).

In Study 2, we focused the proportional change in predicted
probabilities across low, medium, and high categories of trust
responses. For this purpose, we employed neural networks using the
\texttt{nnet} package in R (\citeproc{ref-nnet_2002}{Venables and Ripley
2002}). The neural network models provided estimates of the probability
of responses falling into each category across different survey waves.
The outputs included predicted probabilities for each response category.
As with Study 1, we used sample weights constructed from the 2018 New
Zealand Census data to ensure that our estimates were representative of
the broader population, and cluster robust prediction using cluster
robust standard errors im \texttt{ggeffects}
(\citeproc{ref-ggeffects_2018}{Ludecke 2018}). Again we used sample
weights to adjust for any sampling biases and ensure that the analyses
produced estimates that generalise to the New Zealand adult population.
These weights were based on demographic distributions from the 2018 New
Zealand Census and were applied during model estimation to account for
potential non-representativeness in the sample data. We plot the
predicted means at confidence intervals using
\texttt{ggeffects}(\citeproc{ref-ggeffects_2018}{Ludecke 2018}). Tables
were created using \texttt{tinytable}
(\citeproc{ref-tinytable_2024}{Arel-Bundock 2024}); \texttt{ggplot2}
(\citeproc{ref-ggplot2_2016}{Wickham 2016}); and the \texttt{margot}
package (\citeproc{ref-margot2024}{Bulbulia 2024}).

\subsection{Results}\label{results}

\subsubsection{Multiple Imputation}\label{multiple-imputation}

\begin{table}

\caption{\label{tbl-sample-means-imp}Sample average response}

\centering{

\centering
\begin{tblr}[         %% tabularray outer open
]                     %% tabularray outer close
{                     %% tabularray inner open
colspec={Q[]Q[]Q[]Q[]},
cell{2}{1}={r=4,}{valign=h,cmd=\bfseries,},
cell{6}{1}={r=4,}{valign=h,cmd=\bfseries,},
column{1}={halign=l,},
column{2}={halign=l,},
column{3}={halign=l,},
column{4}={halign=l,},
}                     %% tabularray inner close
\toprule
Response & Year & mean & missing \\ \midrule %% TinyTableHeader
Trust Science & 2019 & 5.56 & 0 \\
Trust Science & 2020 & 5.68 & 0 \\
Trust Science & 2021 & 5.64 & 0 \\
Trust Science & 2022 & 5.61 & 0 \\
Trust Scientists & 2019 & 5.28 & 0 \\
Trust Scientists & 2020 & 5.42 & 0 \\
Trust Scientists & 2021 & 5.37 & 0 \\
Trust Scientists & 2022 & 5.32 & 0 \\
\bottomrule
\end{tblr}

}

\end{table}%

Table~\ref{tbl-sample-means-imp} presents averages in the first imputed
dataset. For Trust in Science, the imputed average slightly increases
from 5.56 in 2019 to 5.68 in 2020 and then declined gradually to 5.64 in
2021 and 5.61 in 2022.

For Trust in Scientists, the average starts at 5.28 in 2019 rising to to
5.42 in 2020, but steadily decreases to 5.37 in 2021 and further to 5.32
by 2022. These patterns suggest that initial improvements in average
trust levels following the New Zealand Pandemic response in 2020 were
followed by a slight decrease over the four-year period. Notably the
pattern evident in the imputed dataset differs markedly from the pattern
of increasing growth evident for the observed cohort, which suggests
steady growth in trust, both for science and scientists.

\textbf{?@tbl-sample-cat-imp} displays the distribution of responses for
Trust in Science and Trust in Scientists within the first imputed
dataset. For trust in scientists, the proportion of the cohort reporting
low trust increased slightly from 11.8\% (5056) in 2019 to 12.2\     (5224)
in 2022. Medium trust declined from 34.3\% (14660) in 2019 to 31.1\%
(13279) in 2020, and ended at 34.0\% (14498) in 2022. High trust rose
from 53.8\% (22965) in 2019 to 58.0\% (24770) in 2020, and then declined
to 53.8\% (22959) by 2022.

For trust in science, low trust increased gradually from 8.3\% (3523) in
2019 to 9.8\% (4163) in 2022. Medium trust fell from 31.5\% (13446) in
2019 to 26.5\% (11329) in 2020, then rose to 28.5\% (12184) by 2022.
High trust increased from 60.2\% (25712) in 2019 to 64.6\% (27570
respondents) in 2020, followed by a decline to 61.7\% (26334) in 2022.
Overall, while high trust peaked in 2020 for both measures, subsequent
years showed modest decreases or stabilisation.

These results, drawn from the first imputed dataset, indicate how
responses may vary year by year and underscore the importance of
imputation to account for potential non-response bias and ensure more
reliable estimations of trust trends in a longitudinal cohort that may
become compromised by mistrust in science. In particular, a substantial
number of inferred low trust responses suggests that the decline in low
trust observed in the sample may not accurately reflect the true
sentiments of the broader cohort.

To quantitatively evaluate patterns of stability and change for the
population, we next report statistical models that employ all ten
multiply imputed datasets. We emphasise that purposes are descriptive
and exploratory, we do not test specific hypothesis, or address causal
theories.

\phantomsection\label{refs}
\begin{CSLReferences}{1}{0}
\bibitem[\citeproctext]{ref-tinytable_2024}
Arel-Bundock, V (2024) \emph{Tinytable: Simple and configurable tables
in 'HTML', 'LaTeX', 'markdown', 'word', 'PNG', 'PDF', and 'typst'
formats}. Retrieved from
\url{https://CRAN.R-project.org/package=tinytable}

\bibitem[\citeproctext]{ref-blackwell_2017_unified}
Blackwell, M, Honaker, J, and King, G (2017) A unified approach to
measurement error and missing data: Overview and applications.
\emph{Sociological Methods \& Research}, \textbf{46}(3), 303--341.

\bibitem[\citeproctext]{ref-margot2024}
Bulbulia, JA (2024) \emph{Margot: MARGinal observational
treatment-effects}.
doi:\href{https://doi.org/10.5281/zenodo.10907724}{10.5281/zenodo.10907724}.

\bibitem[\citeproctext]{ref-bulbulia2023a}
Bulbulia, JA, Afzali, MU, Yogeeswaran, K, and Sibley, CG (2023)
Long-term causal effects of far-right terrorism in {N}ew {Z}ealand.
\emph{PNAS Nexus}, \textbf{2}(8), pgad242.

\bibitem[\citeproctext]{ref-fraser_coding_2020}
Fraser, G, Bulbulia, J, Greaves, LM, Wilson, MS, and Sibley, CG (2020)
Coding responses to an open-ended gender measure in a {N}ew {Z}ealand
national sample. \emph{The Journal of Sex Research}, \textbf{57}(8),
979--986.
doi:\href{https://doi.org/10.1080/00224499.2019.1687640}{10.1080/00224499.2019.1687640}.

\bibitem[\citeproctext]{ref-geepack_2006}
Halekoh, U, Højsgaard, S, and Yan, J (2006) The {R} package geepack for
generalized estimating equations. \emph{Journal of Statistical
Software}, \textbf{15/2}, 1--11. Retrieved from
\url{https://www.jstatsoft.org/v15/i02/}

\bibitem[\citeproctext]{ref-hartman2017}
Hartman, RO, Dieckmann, NF, Sprenger, AM, Stastny, BJ, and DeMarree, KG
(2017) Modeling attitudes toward science: Development and validation of
the credibility of science scale. \emph{Basic and Applied Social
Psychology}, \textbf{39}, 358--371.
doi:\href{https://doi.org/10.1080/01973533.2017.1372284}{10.1080/01973533.2017.1372284}.

\bibitem[\citeproctext]{ref-amelia_2011}
Honaker, J, King, G, and Blackwell, M (2011) {Amelia II}: A program for
missing data. \emph{Journal of Statistical Software}, \textbf{45}(7),
1--47.

\bibitem[\citeproctext]{ref-jost_end_2006-1}
Jost, JT (2006) The end of the end of ideology. \emph{American
Psychologist}, \textbf{61}(7), 651--670.
doi:\href{https://doi.org/10.1037/0003-066X.61.7.651}{10.1037/0003-066X.61.7.651}.

\bibitem[\citeproctext]{ref-ggeffects_2018}
Ludecke, D (2018) Ggeffects: Tidy data frames of marginal effects from
regression models. \emph{Journal of Open Source Software},
\textbf{3}(26), 772.
doi:\href{https://doi.org/10.21105/joss.00772}{10.21105/joss.00772}.

\bibitem[\citeproctext]{ref-nisbet2015}
Nisbet, EC, Cooper, KE, and Garrett, RK (2015) The partisan brain: How
dissonant science messages lead conservatives and liberals to (dis)trust
science. \emph{The ANNALS of the American Academy of Political and
Social Science}, \textbf{658}(1), 36--66.
doi:\href{https://doi.org/10.1177/0002716214555474}{10.1177/0002716214555474}.

\bibitem[\citeproctext]{ref-sibley2021}
Sibley, CG (2021)
\emph{\href{https://doi.org/10.31234/osf.io/wgqvy}{Sampling procedure
and sample details for the {N}ew {Z}ealand {A}ttitudes and {V}alues
{S}tudy}}.

\bibitem[\citeproctext]{ref-nnet_2002}
Venables, WN, and Ripley, BD (2002) \emph{Modern applied statistics with
s}, Fourth, New York: Springer. Retrieved from
\url{https://www.stats.ox.ac.uk/pub/MASS4/}

\bibitem[\citeproctext]{ref-ggplot2_2016}
Wickham, H (2016) \emph{ggplot2: Elegant graphics for data analysis},
Springer-Verlag New York. Retrieved from
\url{https://ggplot2.tidyverse.org}

\end{CSLReferences}




\end{document}
